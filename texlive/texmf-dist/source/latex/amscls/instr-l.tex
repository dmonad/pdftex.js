%%% ====================================================================
%%% @LaTeX-file{
%%%   filename  = "instr-l.tex",
%%%   version   = "2.20",
%%%   date      = "2004/08/06",
%%%   time      = "16:17:21 EDT",
%%%   checksum  = "60855 1233 6587 50183",
%%%   author    = "American Mathematical Society",
%%%   copyright = "Copyright 1994, 2004 American Mathematical Society,
%%%                all rights reserved.  Copying of this file is
%%%                authorized only if either:
%%%                (1) you make absolutely no changes to your copy,
%%%                including name; OR
%%%                (2) if you do make changes, you first rename it
%%%                to some other name.",
%%%   address   = "American Mathematical Society,
%%%                Technical Support,
%%%                Publications Technical Group,
%%%                201 Charles Street,
%%%                Providence, RI 02904,
%%%                USA",
%%%   telephone = "401-455-4080 or (in the USA and Canada)
%%%                800-321-4AMS (321-4267)",
%%%   FAX       = "401-331-3842",
%%%   email     = "tech-support@ams.org (Internet)",
%%%   codetable = "ISO/ASCII",
%%%   keywords  = "latex, AMS, amslatex, ams-latex, author instructions",
%%%   supported = "yes",
%%%   abstract  = "These are instructions for preparing documents for
%%%                submission to the AMS, using amsmath and AMS LaTeX
%%%                document classes.",
%%%   docstring = "The checksum field above contains a CRC-16 checksum
%%%                as the first value, followed by the equivalent of
%%%                the standard UNIX wc (word count) utility output of
%%%                lines, words, and characters.  This is produced by
%%%                Robert Solovay's checksum utility.",
%%% }
%%% ====================================================================
\NeedsTeXFormat{LaTeX2e}% LaTeX 2.09 can't be used (nor non-LaTeX)
[1995/06/01]% LaTeX date must be June 1995 or later
\documentclass{amsart}

\hyphenation{man-u-script man-u-scripts over-view pre-par-ing which-ever}

%%  Define macros for text substitution and for presentation of examples
\newcommand{\AMS}{American Mathematical Society}
\def\latex/{{\protect\LaTeX}}
\def\latexe/{{\protect\LaTeXe}}
\def\amslatex/{{\protect\AmS-\protect\LaTeX}}
\def\tex/{{\protect\TeX}}
\def\amstex/{{\protect\AmS-\protect\TeX}}
\def\bibtex/{{Bib\protect\TeX}}
\def\makeindx/{\textit{MakeIndex}}

\newcommand{\filnam}[1]{\mbox{\texttt{\ignorespaces#1\unskip}}}
\let\fn=\filnam
\let\cls=\filnam
\let\env=\filnam
\let\pkg=\filnam
\def\opt#1{\filnam{[#1]}}

%%  For this manual, add a section number to the references header, and
%%  include that section header in the contents list.
\makeatletter
\def\thebibliography#1{\section\refname
  \normalfont\small\labelsep .5em\relax
  \list{\@arabic\c@enumiv.}{\settowidth\labelwidth{#1.}%
  \leftmargin\labelwidth \advance\leftmargin\labelsep
  \usecounter{enumiv}}%
  \sloppy \clubpenalty4000\relax \widowpenalty\clubpenalty
  \sfcode`\.\@m}
%
%%  Turn off series logo and copyright note
\let\serieslogo@\relax
\let\@setcopyright\relax

%%%%%%%%%%%%%%%%%%%%%%%%%%%%%%%%%%%%%%%%%%%%%%%%%%%%%%%%%%%%%%%%%%%%%%%%
%%  Table of contents customizations

\let\tocori=\tableofcontents
\def\tableofcontents{\tocori
  \skip@\lastskip \vskip-\lastskip
  \addtocontents{toc}{\protect\tocpatch}%
  \vskip\skip@ \relax}

\def\tocsection#1#2#3{%
  \indentlabel{\@ifnotempty{#2}{\tocnbox{\hss#1 #2.\quad}}}#3}
\let\tocnbox=\hbox
\def\tocpatch{%
  \@xp\let\csname r@tocindent2\@xp\endcsname\csname r@tocindent1\endcsname
  \def\tocnbox{\hbox to\@tempdima}}
%%  In this document, subsections are unnumbered, but we want an indent
%%  to be used.
\def\tocsubsection#1#2#3{\hbox to\leftskip{}#3}
\def\l@subsection{\@tocline{2}{0pt}{.5pc}{}{}}
\makeatother

\newdimen\exindent
\exindent=3\parindent
%% Add a high penalty to discourage line breaks within an example
%% without absolutely prohibiting them.
{\obeylines
 \gdef^^M{\par\penalty9999 }%
 \gdef\beginexample#1{\medskip\bgroup %
   \def\(##1){\makebox[0pt]{\normalfont\mdseries##1}}%
   \def~{\char`\~}\def\\{\char`\\}%
   \normalfont\ttfamily\frenchspacing %
   \parindent=0pt#1\leftskip=\exindent\obeylines}
}%  end \obeylines
\def\endexample{\endgraf\egroup\medskip}
\newdimen\exboxwidth
\exboxwidth=3in
\def\exbox#1#2{\noindent \hangindent=\exboxwidth
  \makebox[0pt][r]{\null\normalfont\mdseries#1\unskip\enspace}%
  \makebox[\exboxwidth][l]{\normalfont\ttfamily\ignorespaces#2}%
  \normalfont\mdseries\ignorespaces}

\newcommand\cn[1]{\mbox{\def\\{\char`\\}%
    \normalfont\ttfamily\\\ignorespaces#1\unskip}}
\let\ttcs=\cn
\newcommand\ttcsb[2]{%
  \mbox{\def\\{\char`\\}%
    \normalfont\ttfamily
    \\begin\{\ignorespaces#1\unskip\}\ignorespaces#2\unskip}}
\newcommand\ttcse[1]{\mbox{\def\\{\char`\\}%
    \normalfont\ttfamily\\end\{\ignorespaces#1\unskip\}}}
\def\{{\char`\{\relax}
\def\}{\char`\}\relax}

%%  Provide a meta-command facility; provide an alternate escape
%%  character so it can be used within the verbatim environment.

\catcode`\|=0
\begingroup \catcode`\>=13 % in LaTeX2e verbatim env makes > active
\gdef\?#1>{{\normalfont$\langle$\textup{#1}$\rangle$}}
\gdef\0{\relax}
\endgroup
\def\<#1>{{\normalfont$\langle$\textup{#1}$\rangle$}}
\newcommand{\Dimen}{\<dimen>}
\newcommand{\tab}{\textsc{tab}}

%%%%%%%%%%%%%%%%%%%%%%%%%%%%%%%%%%%%%%%%%%%%%%%%%%%%%%%%%%%%%%%%%%%%%%%%

%%  Provide a dummy titlepage, saving maketitle for the article title.
%%  Table of contents goes onto the inside front cover.

\def\instrversion{2.20}
\def\instrdate{August 2004}

\begin{document}

\thispagestyle{empty}
\vspace*{1.5in}
\begin{center}
  {\huge\bfseries Instructions\\[2pt] for Preparation\\[2pt]
    of Papers\\[2pt] and Monographs
    \par\vspace{\baselineskip}
    \amslatex/
    \par}
  \vspace{3\baselineskip}
  \Large
  Version \instrversion\\[4pt] \instrdate
\end{center}
\newpage

\thispagestyle{empty}
\tableofcontents
\newpage

\setcounter{page}{1}

\title [ELECTRONIC MANUSCRIPTS: \amslatex/]
   {Instructions for Preparation\\
   of Papers and Monographs: \amslatex/}
\dedicatory{%
  \mbox{\normalfont\mdseries\small Version \instrversion: \instrdate}}

\maketitle

\section{Introduction}

This manual is directed mainly to authors preparing material for
publication by the AMS, using \amslatex/ document classes.  As such,
it deals with the AMS publishing style.  Since these document classes
are also used by authors who are not submitting items to the AMS,
the manual also covers topics of more general relevance.  However, it
assumes familiarity with standard \latex/ techniques and conventions,
and contains only material specific to AMS packages.

Links to information useful in preparing manuscripts and graphics
can be found on the AMS Author Resource Center web page \cite{ARC}\@.
That information supplements this manual, and we recommend that you
review it.

The AMS produces three major types of publications: journals (both
print and electronic), proceedings volumes, and monographs.  There
is a core AMS document class for each: \cls{amsart}, \cls{amsproc}
and \cls{amsbook}, respectively.  For most journals or book series,
there is a specific author package, based on one of the core classes.
If the specific publication is known, the appropriate author package
should be obtained from the AMS web server \cite{AP} and used for
preparation of manuscripts.  In addition, there is a separate package
for editors of proceedings volumes \cite{EP}, with facilities for
preparing material that does not fit into the normal ``article''
category.

If the specific publication is not known, an appropriate generic AMS
document class can be used:

\begin{itemize}
\item \pkg{gen-j-l} for journal articles (based on \pkg{amsart})
\item \pkg{gen-p-l} for articles in book proceedings (based on \pkg{amsproc})
\item \pkg{gen-m-l} for monographs (based on \pkg{amsbook})
\end{itemize}

The tagging of elements in a manuscript---title, author(s), section
headings, theorems, etc.--- is consistent through all author packages.
Thus a manuscript prepared using the appropriate generic document
class can be modified trivially to use the specific package, simply by
updating the \cn{documentclass} statement.  For example, to specify
the \textit{Contemporary Mathematics} proceedings series:
\begin{verbatim}
\documentclass{conm-p-l}
\end{verbatim}

Each author package contains an electronic version of this manual,
class files, sample documents, and a \texttt{README} file that
contains information about each of the files in the package.

When an AMS author package is used, both \latex/ and a recent version
of \amslatex/ must be installed on the computer being used for
manuscript preparation.  The AMS Author Resource Center web page
\cite{ARC} has a link to download the most recent version of
\amslatex/.  The version number appears at the top of the file
\pkg{amsmath.sty} (and other files in the collection), and is also
reported in the log file of a \latex/ run; make sure you have
version 2.0 or later.

\section{Article preparation}

We first address articles intended for journals and proceedings.
Monographs are addressed in a separate section (see
page~\pageref{s:mono}).

\subsection*{Starting a new article}\label{ss:newamsart}
%%
Individual articles generally have the following structure:
\begin{itemize}
\item \verb+\documentclass{...}+
\item preamble (for document-specific customizations)
\item \verb+\begin{document}+
\item top matter information
\item \verb+\maketitle+ (to set the top matter)
\item article body
\item reference list
\item \verb+\end{document}+
\end{itemize}
A detailed template file (included in each package) provides
a complete set of commands for the article top matter, and the
instructions for using them.  The most efficient way to start a
new article is to make a copy of the template and fill it in.


\section{The preamble}

The area between the \verb+\documentclass+ statement and the line
\verb+\begin{document}+ is referred to as the ``preamble''.  This is
the place to load external packages and define document-specific
commands.

Available \cn{documentclass} options are described in the file
\filnam{amsclass.dtx}.  Process this file with \latex/ to obtain a
version with human-readable documentation.  One option does need to
be mentioned here: \opt{psamsfonts}; include this option as
\begin{verbatim}
\documentclass[psamsfonts]{...}
\end{verbatim}
to ensure correct sizing of fonts in environments other than normal text.

\subsection*{Packages}
%%
The AMS document classes incorporate the code for the AMS theorem
(\pkg{amsthm}) package and automatically load the \pkg{amsmath}
package.  It is not necessary to request either one explicitly.
The user guides for these packages \cite{ATP,AMP} contain the details,
which will not be repeated here.  The \pkg{amsfonts} package is loaded
as well, unless the \opt{noamsfonts} option is specified; see the
AMSFonts User's Guide \cite{AFG} for the features provided.

Not all publicly available packages are compatible with the AMS
document classes.  For this reason, we mention some that are:
\begin{itemize}
\item \pkg{amssymb} provides names for additional symbols from the
  AMS symbol fonts; see \cite{AFG} for details.
\item \pkg{amsmidx} supports multiple indexes for a book; see
  page~\pageref{ss:indexing}.
\item \pkg{graphicx} is part of the \latex/ suite, preferred for
  inclusion of graphics.
\item \pkg{longtable} facilitates tables longer than one page.
\item \pkg{upref} forces cross-references with \cn{ref} to roman;
  see ``Roman type'', page~\pageref{ss:roman}.
\item \pkg{xypic} is recommended for creating commutative diagrams.
\end{itemize}

There are several packages that cause particular problems for AMS
production.  Their use should be avoided if possible:
\begin{itemize}
\item \pkg{psfrag}, for applying labels externally to included graphics,
  may result in a PostScript that can't be converted properly to PDF.
\item \pkg{epsfig} has been superseded by \pkg{graphicx}.
\end{itemize}

Don't load any packages that aren't actually needed.


\subsection*{New commands}
%%
Place all definitions for new commands in the preamble.  This includes
instructions to access fonts that are not already defined in the AMS
document classes or \pkg{amsmath}, such as a new math alphabet (see
\cite{AMP}).

Always use \cn{newcommand}; this will let you know if the name you
have chosen has already been used.  Do not redefine any command from
\latex/, \pkg{amsmath}, or any AMS document class, as this could cause
problems in AMS production.  New definitions may be used to provide
shorthand forms for mathematical expressions that are used frequently,
but do not use new definitions for text; for articles posted on-line,
this will prevent indexing software from identifying relevant
references.

Use commands and environments provided by the AMS document classes
and standard packages whenever applicable---for example, you should
use the AMS \env{proof} environment rather than define your own
alternative.

When the file is complete, remove any commands that haven't been used.


\section{Top matter}\label{s:topmatter}

The top matter associated with an article includes information that
would appear in a bibliographic reference to the paper, plus additional
information about the author(s), subject classifications, key words,
acknowledgments of support, and the abstract.


\subsection*{Summary of tags and elements}
%%
Table 1 lists the top matter commands provided by \amslatex/
version 2.0 and later.  Not every tag is necessary for each paper.
The table shows which tags are necessary and which are optional.
Requirements for monographs are somewhat different and are described
later.  The template file included in every author package contains
all the necessary tags and instructions for using them.

\begin{table}[ht]
\caption{Top matter tags}
\begin{minipage}{\textwidth}
\begin{verbatim}
\documentclass{amsart}
|?preamble commands, such as |ttcs|bgroup|0newcommand|egroup, etc.>

\begin{document}
\end{verbatim}
\begin{center}
\begin{tabular}{lcc}
\                 & \multicolumn{2}{c}{Required by}\\
\                               & Journals  & Books \\
\verb+\title[...]{...}+         & yes\rlap{${}^1$} & yes\rlap{${}^1$} \\

\verb+\author[...]{...}+        & yes\rlap{${}^1$} & yes\rlap{${}^1$} \\
\verb+\contrib[...]{...}+       & no\rlap{${}^2$} & --- \\
\verb+\address{...}+            & yes & yes \\
\verb+\curraddr{...}+           & no  & no \\
\verb+\email{...}+              & no  & no \\
\verb+\urladdr{...}+		& no  & no \\
\verb+\dedicatory{...}+         & no  & no \\
\verb+\date{...}+               & ---\rlap{${}^3$} & --- \\
\verb+\thanks{...}+             & no & no \\
\verb+\translator{...}+         & ---\rlap{${}^3$} & ---\rlap{${}^3$} \\
\verb+\keywords{...}+           & no & no \\
\verb+\subjclass[2000]{...}+    & yes & yes \\
\verb+\begin{abstract}...\end{abstract}+ & yes\rlap{${}^4$} & no \\
\verb+\maketitle+               & yes & yes \\
\end{tabular}
\end{center}
\vspace*{1pc}

{\Small
${}^1$ If no short form is needed, omit the brackets.

${}^2$ For contributors, see details on page~\pageref{ss:contrib}.

${}^3$ If this is necessary, it  will be filled in by the AMS staff.

${}^4$ For the \textit{Journal of the  American Mathematical Society},
abstracts are optional.
\par
}
\end{minipage}
\vspace{.25\baselineskip}
\hrule width \textwidth
\vspace{.5\baselineskip}
\end{table}

\subsection*{Title}
%%
For AMS journal articles, capitalize only the first word and proper
nouns in the title.  For articles in an AMS proceedings volume,
capitalize the first and last words of the title and all nouns,
pronouns, adjectives, adverbs, and verbs.  Articles, conjunctions, and
prepositions should be lowercased unless they are the first or last
word of the title.  These rules should be followed even if the
publication in which the article appears has another style.  The style
for the particular publication will be generated automatically when
the article is processed at the AMS.

A multiline title may be left for \latex/ to break, or a desired break
may be indicated by \cn{\\}.

Unless the title is very short, provide a form of the title suitable
for use in running heads.  This should be entered in brackets between the
tag \cn{title} and the full title:
\verb+\title[+short title\verb+]{+full title\verb+}+.
Do not use author-defined macros in the title.

\subsection*{Author information}\label{ss:authorinfo}
%%
Enter the name(s) of the author(s) with the tag \cn{author}, using
standard capitalization for proper names.  Use a separate \cn{author}
command for each author.  Names will be combined by \latex/ according
to the dictates of the document class.  See also ``contributors'',
below.

If the author name(s) cannot fit in the space available for the running
head, enter a shortened form for each name in [brackets]
between the tag \cn{author} and the full name.  Acceptable shortened
forms use initials for all but the surname(s).  For example,
\begin{verbatim}
\author[J. Smith]{Joseph Smith}
\end{verbatim}
If the list of shortened author names is still too long for the running
head (including a reasonable space for the page number), it can be
shortened as in the following example, as a last resort:
\begin{trivlist}
\item\relax
\verb+\markleft{J. SMITH ET AL.}+
\end{trivlist}

\noindent
Note that \cn{markleft} is available only for \amslatex/ versions 2.20
and later; it is not a basic \latex/ command.
Data provided with \cn{markleft\{\}} \emph{must} be in all capital letters
(except for included math). Place \cn{markleft\{\}} before the first
\cn{author} in your file.

For each author you should provide one or more addresses.  Tag them
as follows, grouping them in this order by author:
\begin{verbatim}
\address{...}   |rm address where the research was carried out |tt
\curraddr{...}  |rm current address, if different from the research address |tt
\email{...}     |rm address for electronic mail |tt
\urladdr{...}   |rm URL address (optional)
\end{verbatim}
Do \emph{not} include any addresses within the scope of an
\cn{author\{\}} command (this is different from basic \latex/).
Do not use abbreviations in addresses.

The main address should be divided by \cn{\\} into segments that
correspond to address lines for use on an envelope.  When typeset, the
lines will be strung together separated by commas, but the division by
\cn{\\} will be useful to a person preparing an envelope to return
proofs.

Addresses are considered part of the top matter, but in AMS articles
they are ordinarily printed at the end of the article following the
bibliography (this is different from basic \latex/).  Suitable labels
will indicate the current, e-mail, and URL addresses, typically
\textit{Current address}:, \textit{E-mail address}:, \textit{URL}:,
respectively.

If \texttt{\textasciitilde} is needed in a URL address, enter it
directly as \texttt{\textasciitilde} if you are using \amslatex/
version 2.20 or later; otherwise, see the Author FAQ \cite{AF}
for instructions.

\subsection*{Contributors
  {\mdseries (new with version 2.20)}}\label{ss:contrib}
%%
Contributors are much like authors, but are responsible for only
a portion of an article, e.g., an appendix or auxiliary tables.
The bracketed portion is not optional; it gives the exact text that
will be printed to indicate the nature of the contribution:
\begin{verbatim}
\contrib[with an appendix by]{William Rogers}
\end{verbatim}
Multiple contributors and contributions are input like this:
\begin{verbatim}
\contrib[with Appendix A by]{William Rogers}
\contrib[]{Henry Taylor}
\contrib[and Appendix B by]{John Henderson}
\end{verbatim}
Note the empty brackets on the second line.  Multiple contributors
indicated in this way will be combined and ``andified'' appropriately.
Contributors will not be included in running heads.  Addresses provided
for contributors will be treated the same as for regular authors.

\subsection*{Dedication}
%%
Use the tag \cn{dedicatory} for such things as ``Dedicated to
Professor X on the occasion of his eightieth birthday.''  If the
dedication is longer than one line, you may indicate a break with
\cn{\\}.

\subsection*{Acknowledgments of support and other first-page footnotes}
%%
Use the command \cn{thanks} to acknowledge grants and other kinds of
support or for other general information not covered by one of the
more specific commands such as \cn{keywords} or \cn{subjclass}.  Enter
\cn{thanks} immediately following the address(es) for the relevant
author, \emph{not} within the scope of the \cn{author\{\}} command
(this is different from basic \latex/).  Like \cn{address}, \cn{thanks}
can appear more than once in the top matter.  Each occurrence will be
printed as an unnumbered footnote at the bottom of the first page of
the article.

\subsection*{Subject information}
%%
Subject classifications and key words, like acknowledg-\linebreak ments, are
part of the top matter and appear as unnumbered footnotes at the bottom of
the first page.

Subject classifications may be primary (the major topic(s) of the
paper) or secondary (subject areas covered by ancillary results,
motivation or origin of problems discussed, intended or potential
field of application, or other significant aspects worthy of notice).
At least one primary subject classification is \textbf{required}.
Additional primaries and secondaries are optional.

\begin{samepage}
These classifications are entered as
\begin{verbatim}
\subjclass[2000]{Primary |?primary class>; Secondary |?secondary classes>}
\end{verbatim}
(Omitting \texttt{[2000]} will cause the 1991 Subject Classification
to be cited.)
\end{samepage}

To determine the classifications, use the 2000 Mathematics Subject
Classification scheme that appears in annual indexes of \textit{Mathematical
Reviews} beginning in 1999.  It can also be accessed on the Web at URL:
\texttt{http://www.ams.org/msc/} or obtained in printed form from
Customer Services.
Please use the full number; the two-digit code from the Contents of
\textit{Mathematical Reviews} is \textbf{not} sufficient.

Key words are not required but may be provided by an author if desired.
They should be tagged as \cn{keywords\{...\}}.

\subsection*{Abstract}
%%
With an AMS document class, place the abstract before \cn{maketitle},
contrary to the practice with the basic \latex/ document classes.
This is necessary to ensure that the abstract can always be printed in
the proper location and style.  If the abstract is given after
\cn{maketitle}, it will be printed in place and generate a warning
message.

Input the abstract text between \verb+\begin{abstract}...\end{abstract}+.
It may comprise multiple paragraphs and include displayed material if
appropriate.  The length of the abstract depends primarily on the
length of the paper itself and on the difficulty of summarizing the
material.  An upper limit of about 150 words for short papers and 300
words for long papers is suggested.  Do not use author-defined macros
or \cn{ref} in the abstract.


\section{Document body (all document classes)}

\subsection*{Linking for the Web}
%%
All AMS primary journals are posted on-line.  Electronic
manuscripts prepared in \latex/ for any AMS journal should be tagged
for maximum linking on the Web. For use of \cn{label}, \cn{ref},
and \cn{cite}, see \cite{LC}; for \cn{eqref}, see \cite{AMP}.

\subsection*{Headings}
%%
Four levels of headings are provided to permit logical sectioning of
a manuscript.  These headings are applicable to individual articles
and to chapters of a monograph.  (Headings specific to monographs are
listed under \textit{Monograph formatting}.)

\begin{verbatim}
\specialsection{...}
\section[...]{...}
\subsection[...]{...}
\subsubsection[...]{...}
\end{verbatim}

\cn{specialsection} is for long articles that need extra divisions
(e.g., parts) at a level above the \cn{section} level.

Explicit line breaks are obtained by \verb+\\+ in first-level section
headings.

Any heading may be given a label to allow references to be made to it,
by including a \cn{label\{...\}} command with a unique identifier
directly \emph{after} the heading.  References are made using the
command \cn{ref\{...\}} and the same identifier.  For example,
\begin{verbatim}
\section{Monograph formatting}
\label{s:mono}
\end{verbatim}
will establish a label for this section that can be referred to with
\cn{ref\{s:mono\}}. Cross references of
this sort will require \latex/ to be run at least twice for proper
resolution.  A warning at the end of the \latex/ run, ``Cross
references may have changed\dots\unskip'', should be heeded in this regard.


\subsection*{Mathematical text}
%%
For instructions on preparing mathematical text, refer to
\cite{AMP} and \cite{ML}.  See also ``Roman type'', below.

\subsection*{Lists}
%%
Follow usual \latex/ conventions for producing list environments.

\subsection*{Theorems, lemmas, and other proclamations}
%%
See the manual \textit{Using the \pkg{amsthm} package} \cite{ATP}
for details. Theorem environments following AMS style with respect to
punctuation, spacing and paragraphing are defined in the AMS document
classes.  This is incompatible with the \latex/ \pkg{theorem} package,
which cannot be used with the AMS document classes.

Three different theorem styles are provided by AMS document classes:
\cls{plain}, \cls{definition}, and \cls{remark}. By referring to these
styles and using the \cn{newtheorem} command, you can build a
complement of theorem environments appropriate for any paper or
monograph.  All \cn{newtheorem} specifications should be included in
the preamble; a starter set is included in the template for every
AMS author package.

Theorem support is also available separately in the \pkg{amsthm} package
for users of document classes other than those from the AMS.

\subsection*{Equations}
%%
Check displayed equations carefully, making sure they are broken and
aligned following the guidelines in \cite[pp. 44--48]{MIT}.

\subsection*{Roman type}\label{ss:roman}
%%
Numbers, punctuation, (parentheses), [brackets],
$\lbrace$braces$\rbrace$, and symbols used as labels should always be
set in roman type.  This is true even within the statement of a theorem,
which is ordinarily set in italic type.

Be careful to distinguish between roman elements that are mathematical
in nature (e.g., ``a group of class 2''), and those that are part of
the text (e.g., a label or a year).  Mathematical expressions, no matter
how short or insignificant, should be coded as math, by enclosing them
within dollar signs \verb+$...$+ or \verb+\(...\)+.  Roman text elements
should be coded as \cn{textup\{...\}} in potentially nonroman
environments such as theorems.

Abbreviated forms of mathematical terms are also usually set in roman
type to distinguish them from mathematical variables or constants.
Use the control sequences for common mathematical functions and
operators like \verb+log+ and \verb+lim+; use \cn{DeclareMathOperator}
to add additional operator names (see \cite[\S5]{AMP}).

The style of reference citations, though publication-dependent, is
usually roman.  To ensure consistency and support linking for the Web,
\textbf{always} use the standard \latex/ \cn{cite} command when citing
a reference.  Internal references with the \cn{ref} command are not
automatically roman in non-roman environments, but can be forced by
using the \pkg{upref} package (part of the \amslatex/ collection).

\subsection*{Exercises}
%%
Exercises are produced using the \env{xca} and \env{xcb} environments.
\env{xca} may be used in any publication, but must be defined in
the document preamble with \cn{theoremstyle\{definition\}} and the
\cn{newtheorem} command; it is used for exercises that occur within a
section.  \env{xcb} is defined only for monographs and is used for
exercises that occur at the end of a chapter.

\subsection*{Indexing facilities}
\label{ss:indexing}
%%
The tags needed for \makeindx/ are already included in AMS document
classes.  Do not include the \pkg{makeidx} package, but do put the
command \cn{makeindex} in the preamble to launch a single index.
See \cite[chapter 11]{LC} and \cite[chapter 11]{GL} for use of
\makeindx/.

A package for multiple indexes, \pkg{amsmidx}, is included in the
\amslatex/ collection beginning with version 2.20.  Unlike other
multiple-index packages, \pkg{amsmidx} is compatible with the AMS
document classes.  To use this package, put these commands into the
preamble:
\begin{verbatim}
\usepackage{amsmidx}
\makeindex{|?name of first index file>}
\makeindex{|?name of second index file>}
\end{verbatim}
To identify index entries in the text:
\begin{verbatim}
\index{|?name of index file>}{|?index term>}
\end{verbatim}
In the backmatter, where the indexes are to appear:
\begin{verbatim}
\Printindex{|?name of first index file>}{|?title of first index>}
\Printindex{|?name of second index file>}{|?title of second index>}
\end{verbatim}
Note the capitalization of the \cn{Printindex} command.

\section{Floating objects: Figures and tables}

Figures and tables are usually handled as floating inserts.  Such
items are often so large that fitting them into the document at the
point of reference may cause problems with pagination.  Placing such
items into a floating insert allows them to be repositioned
automatically by \latex/ as required for good pagination.

A floating insert generally contains one of these possibilities:
\begin{itemize}
\item \latex/ code that produces an object such as a table or
  commutative diagram;
\item a command to incorporate an item produced by another application
  (most often an Encapsulated PostScript (EPS) file produced by a
  graphics utility; see \textit{Electronic graphics},
  page~\pageref{ss:electronicgraphics}).
\end{itemize}

If a figure is described by \latex/ code (e.g., using \pkg{xypic} or
the \env{picture} environment), it should be included directly in the
file.

\subsection*{Placement}
%%
Floats should
\begin{enumerate}
\item be numbered consistently throughout the paper (figures and tables
  are numbered separately),
\item be placed at the top or bottom of the page,
\item have an in-text reference, and
\item definitely appear within the same section as their first text
  reference.
\end{enumerate}

A figure or table should not precede its first text reference unless
they both appear on the same page spread.  When a figure or table is
an integral part of text, it may appear unnumbered in place in the
middle of text.

Figures and tables should be allowed to float according to the \latex/
defaults preset by the document class.  If you are submitting a file
that will be processed at the AMS (this applies to all journal
articles), you could introduce major problems with pagination if you
hard-set your figures and tables by using the \verb+[h!]+ option.
Production problems will be avoided when you use the appropriate
author package class file and avoid using figure or table options.
For electronic manuscripts, the final placement of inserts will be
determined by the AMS editorial staff, on the basis of the most
appropriate page layout.

\subsection*{Captions}
%%
Captions for floating inserts are usually positioned above a table and
below a figure. The following is the general structure used to specify
a figure insert, with a caption at the bottom:
\begin{verbatim}
\begin{figure}
\includegraphics{|?file name of |tt.eps|rm| file>}|quad|rm or|tt|quad|?code for the insertion>
\caption{|?caption text>}
\label{|?reference label>}|quad|rm (optional)|tt
\end{figure}
\end{verbatim}

This is the general structure for a table insert, with a caption at
the top:
\begin{verbatim}
\begin{table}
\caption{|?caption text>}
\label{|?reference label>}|quad|rm (optional)|tt
|?code for the insertion body>
\end{table}
\end{verbatim}

Caption headings (e.g., ``\textsc{Table 3.1.}''\ or
``\textsc{Figure 7.}'')\ will be supplied automatically.  The
\<caption text> is any appropriate descriptive text, and may be
omitted if no descriptive text is desired.  (The style of the caption
differs from the basic \latex/ caption style.)  A reference label
should usually be associated with the caption, and must be given
\emph{after} the caption.  A reference in text to the figure or table
has the form \cn{ref\{\<reference label>\}}.

\latex/ packages for captions may appear to work with the AMS document
classes, but the caption style doesn't usually match the AMS style.

\subsection*{Electronic graphics}\label{ss:electronicgraphics}
%%
Graphics created by applications other than \latex/ may be submitted
to the AMS in an electronic format.  This includes output from
graphics applications as well as scanned photographs or other
computer-generated images.  The AMS requires such graphics in
Encapsulated PostScript (EPS) format.

Use the \pkg{graphicx} package (part of the \pkg{graphics} bundle
of \latex/) to embed references to EPS graphics directly in a \latex/
file; see the manual, \filnam{grfguide.tex} \cite{GB}.  Another
valuable reference is the manual ``Using EPS Graphics in \latexe/
Documents'' \cite{EG}.

Other packages are not recommended, as there are numerous areas of
incompatibility which may result in the need for manual intervention,
delaying processing and increasing the cost.

The EPS files will not be physically included in the \latex/ file.
You must submit a separate file for each graphic along with the \latex/
document.  No matter what method is used to produce a graphic, it is
necessary to provide a paper copy to the AMS for verification.  Double
check to make sure the electronic and paper versions are the same.

\subsection*{Compatibility requirements for graphics}
There are two important requirements that must be observed when
embedding graphics.

\emph{Omit path names} for the included files.  Instead of
\begin{verbatim}
\includegraphics*{/usr/joe/book/figures/fig1.ps}
\end{verbatim}
use just the name of the file:
\begin{verbatim}
\includegraphics*{fig1.ps}
\end{verbatim}
The easiest way to do this is to place the graphics files in the
same directory as the file(s) being {\TeX}ed.

Do not use the \cn{graphicspath\{\<directory list>\}} command to
establish a path.  This will result in the path being included in
the DVI file, which will be impossible to process at the AMS\@.
It may be possible to specify a path using one of the environments
recognized by the {\TeX} engine (\fn{TEXINPUTS}, for example);
instructions for setting these environments should be in the
documentation for your system.

Graphics commands must be compatible with the \textit{dvips} output
driver.  This is particularly important if you are submitting a DVI
file for a camera-ready publication.  The method is specific to the
graphics package you are using; a typical method is to use a file
\filnam{graphics.cfg} containing the line \cn{ExecuteOptions\{dvips\}}
when you run \latex/ to create the DVI or PostScript file to send to
the AMS.

\subsection*{Technical requirements for graphics}
%%
If you scale a figure before submitting it, remember that line weights
also scale.  The lightest line weight that will reproduce clearly at
high resolution is 0.5pt after scaling.  Do \emph{not} specify
``hairline'' weight, as this will be nearly invisible at high
resolution and will disappear in the printing process. If a rule line
is screened, its weight should be no less than 1 point \emph{after}
scaling.

Graded line weights should increase in increments of at least 0.5pt.
Increments less than this are insufficiently distinguishable at high
resolution.

Screened fills should be screened not less than 15\% (less will print
as white) and not greater than 85\% (greater will print as black).

Ideally, text included in graphics files should be font outlines
rather than bitmaps, because the AMS typesets its publications on a
high-resolution imagesetter, not a laser printer.  Text or lines in
graphics which are 300dpi bitmaps (which look fine in the
laser-printed output of your article) may appear poor in quality next
to the high-resolution text of AMS publications.  If bitmaps of
characters are part of bitmapped line art files, output will be
satisfactory if the bitmapped characters are at 1200dpi.

\subsection*{Color graphics}
%%
AMS publications are ordinarily printed in black and white.
Use of color requires explicit permission from the AMS.

Color figures should have the resolution of at least 266 pixels per
inch when printed at 100\%, and must also be suitable for printing
as grayscale.  The files must be in EPS format


\section{Bibliographic references}
%%
\subsection*{Using \protect\bibtex/ to prepare a bibliography}%
An author may find it convenient to maintain a file of references in
\bibtex/ form, as described in \textit{The \latex/ Companion}
\cite[Section 13.2]{LC}. Two \bibtex/ styles are provided:

\begingroup
\exindent=2\parindent
\beginexample{\exboxwidth=1.05in}
\exbox{}{amsplain.bst} will produce numeric labels; preferred for articles;
\exbox{}{amsalpha.bst} will produce labels constructed from the %
  author name(s)\newline and year of publication.
\endexample
\endgroup

\noindent
Both will translate references in a \bibtex/ input (\filnam{.bib})
file to \latex/ input in a \filnam{.bbl} file in the form appropriate
for AMS publications, including all necessary formatting instructions.
This method of preparing bibliographies is therefore recommended.

To access a \bibtex/ bibliography in a paper or monograph, include these
instructions in the appropriate place in the input file:
\begin{verbatim}
\bibliographystyle{|?style>}|qquad|rm(|tt|0amsplain  |rm or|tt  amsalpha|rm)|tt
\bibliography{|?name of bibliography |bgroup|tt.bib|egroup| file>}
\end{verbatim}

Running \bibtex/ on the \filnam{.bib} file will produce a \filnam{.bbl}
file.  The \filnam{.bib} file may have any name the author finds convenient;
however, the \filnam{.bbl} file must have the same name as the source file
for a monograph from which it is input, so it may be necessary for the
author to rename it.  For an article, after the bibliography has been
completed (including processing by \bibtex/), the contents of the
\filnam{.bbl} file should be inserted into the main article input file,
replacing the \cn{bibliographystyle} and \cn{bibliography} statements.

Items in the bibliography are usually ordered alphabetically by author.
\bibtex/ processing may alter this order, especially if the
\filnam{amsalpha} style is used.

All categories of bibliographic entries listed in
\cite[Table 13.1]{LC} are supported in the two AMS \bibtex/ styles.
In addition to the fields listed in Table 13.1, a \verb+language+
field is provided for use in identifying the original language of an
item whose title has been translated.

The \pkg{harvard} package (available from CTAN) provides an
author-year style that is compatible with the AMS document classes.
This should ordinarily be used only for monographs on historical
topics.

Standard abbreviations for periodicals should be obtained from
\cite{ASMR}\@.

\subsection*{Preparing a bibliography without \protect\bibtex/}
%%
The references section of a paper is contained between the commands
\begin{verbatim}
\begin{thebibliography}{|?model label>}
\end{thebibliography}
\end{verbatim}
This environment is resolved as an unnumbered section in an article
or an unnumbered chapter in a monograph, following AMS publication
specifications.  The indentation for the labels is set to an
appropriate width using the model given with the \cn{begin} command.
Thus the widest label in the bibliography should be used as the model;
for example, \verb+99+ will provide space for a 2-digit numeric label.

For the proper order of reference elements and use of fonts and
punctuation, look at an issue or volume in the journal or book series
for which your document is intended and follow the examples you see
there.  More examples are given in \cite{MIT}.

\subsubsection*{Reference input}
%%
Begin each item with the command \cn{bibitem\{\<bibitem label>\}}.
The \latex/ default is to number references automatically; however,
other labels may be used by inserting an optional key argument in
square brackets between the command and the internal label:
\begin{verbatim}
\bibitem[ABC]{Arno1994}
\end{verbatim}
The item label and the key need not be identical.  Whatever appears
within the \verb+[ ]+ is what prints; whatever is within the curly
braces is used for linking and must not include any math or special
characters.

Give at least one full name; initials and last name is an acceptable
form.  If a subsequent reference is by the same author(s), use
\cn{bysame} instead of the name(s).

For examples, refer to \cite{MIT}.


\section{Monograph formatting}\label{s:mono}

A monograph is a long work by a single author or co-authors on a
single subject.  Each chapter should be prepared as a separate file,
as should the bibliography.  In addition, a ``driver'' file should be
used to input all the others.  These files should be given meaningful
names, so that when they are transmitted to the AMS, there will be no
question about which file represents which chapter.  For example, a
monograph by author Grey might be composed of files named
\filnam{grey.tex} (the driver file), \filnam{grey-ch1.tex},
\filnam{grey-ch2.tex}, \dots, \filnam{grey-ch12.tex},
\filnam{grey-appa.tex}, etc., and \filnam{grey-bib.tex}.
If the author name is a common one, please include something to make
it unique, such as first initials.

Information that identifies the author(s), the subject matter of the
monograph, acknowledgments of support, and so forth, will appear in the
front matter of the book.  Place this information in the driver file,
and use the tags shown below.  Most of these are the same as the tags
associated with the top matter of an article; see the \textit{Top
matter} section (page~\pageref{s:topmatter}) for explanations and an
indication of which tags are required.

\subsection*{Starting a new monograph}\label{s:newamsbook}
%%
The driver file for a monograph is generally made up of the following:
\begin{itemize}
\item \verb+\documentclass{...}+
\item preamble (where extra definitions might go)
\item \verb+\begin{document}+
\item \verb+\frontmatter+
\item title page and copyright page information
\item \verb+\maketitle+ (to set the title page and copyright page; see note)
\item \verb+\include+ files (e.g., preface, introduction)
\item \verb+\mainmatter+
\item \verb+\include+ files (e.g., main chapters, appendices)
\item \verb+\backmatter+
\item commands for bibliography, index
\item \verb+\end{document}+
\end{itemize}

\noindent
Note: The title and copyright pages are for information only, so that
a printed copy can be associated with the correct author(s).  The final
copy will be prepared at the AMS.

For a typical driver file, see the template for \cls{amsbook}, included
in the \amslatex/ collection.

Instructions for preparing a dedication page are given in the AMS
Author FAQ \cite{AF}.

The table of contents will be produced automatically from a
\filnam{.toc} file produced anew in each run of \latex/. Since there is
no \filnam{.toc} before the first run, the body of the table of contents
will be empty on the first run. It is AMS style to include only
first-level heads, chapter titles, and part titles in the table of
contents.

Chapters in AMS book series start on right-hand (odd-numbered) pages.
Blank pages will be provided as necessary at the end of earlier material
to accomplish this.


\subsection*{Chapter titles}
%%
Chapter titles should be capitalized as follows: the first and last
words of the title and all nouns, pronouns, adjectives, adverbs, and
verbs should be capitalized.  Articles, conjunctions, and prepositions
should be lowercased unless they are the first or last word of the title.

There are three common variations of the chapter title.  The form with
a chapter number is most common:
\begin{verbatim}
\chapter{Matrix Algebras}
\end{verbatim}

The second variation is an appendix, where the word ``Appendix''
replaces the word ``Chapter''. Use the command \cn{appendix} before
the first \cn{chapter} command in a sequence of appendix chapters.
\begin{verbatim}
\appendix
\chapter{Poisson Integral}
\end{verbatim}
In addition to ``Appendix'', the counter will produce letters ``A'',
``B'', ``C'', etc.,\ instead of numbers.

The third variation is used for an element such as a preface or
introduction, which has no pretitle text at all.  For this, use the
\cn{chapter*} command:
\begin{verbatim}
\chapter*{Preface}
\end{verbatim}
Unlike the \latex/ \cls{book} class, the AMS document classes will
place entries for unnumbered chapters in the table of contents, as
required by AMS style.

\subsection*{Monograph running heads}
%%
The chapter title is used for the left running head and section
headings for the right running head.  The chapter title repeats if
there are no sections.  If the text of a section heading is too long
to fit as a running head, use the square-bracket option to specify a
shortened form for use in the running heads:

\begin{verbatim}
\section[Fourier coefficients of periodic functions]
   {Fourier coefficients of continuous periodic functions
   of bounded entropy norm}
\end{verbatim}

If the chapter title is too long to fit as a running head, a shortened
form can be supplied in a similar way.  Full chapter titles and section
headings will be used in the table of contents (this is different from
basic \latex/).

\subsection*{Numbering of figures and tables}
%%
The preferred AMS style for numbering figures and tables is to start
numbering from 1 within each chapter, and to include both the chapter
and element number in the printed caption heading.  For example, the
first figure in Chapter 5 would be numbered ``\textsc{Figure} 5.1''.
The \cls{amsbook} document class resets figure and table numbers
automatically, but does not include the chapter number in the printed
caption.  To add the chapter number, place these lines in the preamble
of the driver file:
\begin{verbatim}
\numberwithin{figure}{chapter}
\numberwithin{table}{chapter}
\end{verbatim}


\section{Getting help}

Many questions raised by authors are answered in the AMS Author FAQ
\cite{AF}.  Please check that before asking for assistance.

If you encounter difficulties in preparing or submitting your
manuscript in electronic form after it has been accepted for
publication by the appropriate editorial board, you can ask for help
from the \AMS\ at:

\penalty-100
\begin{samepage}
\beginexample{\normalfont\mdseries}
Technical Support
Publications Technical Group
201 Charles Street
Providence, RI  02904-2294
\vspace{3pt}
Phone: 800-321-4267, ext.\ 4080 \quad or \quad 401-455-4080
E-mail: {\texttt{tech-support@ams.org}}
\endexample
\end{samepage}

\noindent
Please supply the name of the journal or book series, the editor's
name if the manuscript is for a proceedings volume, and a concise
explanation of the problem.


\begin{thebibliography}{[ASMR]}

\bibitem[AF]{AF} AMS Author FAQ,
\texttt{http://www.ams.org/authors/author-faq.html}

\bibitem[AFG]{AFG} \textit{AMSFonts{} version~\textup{2.2d} user's guide},
Amer. Math. Soc.,  Providence, RI, 2002.
PDF file linked from \texttt{http://www.ams.org/tex/amsfonts.html}

\bibitem[AMP]{AMP} \textit{User's guide for the \pkg{amsmath} package
\textup{(}version~\textup{2.0)}}, Amer. Math. Soc., Providence, RI, 2004.
PDF file linked from \texttt{http://www.ams.org/tex/amslatex.html}

\bibitem[AP]{AP} Author packages for publishing with the AMS,\\
\texttt{http://www.ams.org/tex/author-info.html}

\bibitem[ARC]{ARC} AMS Author Resource Center,
\texttt{http://www.ams.org/authors/}

\bibitem[ASMR]{ASMR} \textit{Abbreviations of names of serials reviewed in
Mathematical Reviews}, Amer. Math. Soc., Providence, RI,
revised annually; searchable on-line at\\
\texttt{http://www.ams.org/msnhtml/serials-list/annser\_frames.html}.

\bibitem[ATP]{ATP} \textit{Using the \pkg{amsthm} package,
version~\textup{2.0}}, Amer. Math. Soc., Providence, RI, 2004.\\
PDF file linked from \texttt{http://www.ams.org/tex/amslatex.html}

\bibitem[EG]{EG} Keith Reckdahl, ``Using EPS graphics in \latexe/
documents'', available in electronic form as \texttt{info/epslatex.ps}
or \filnam{.pdf} from\newline
\texttt{http://www.ctan.org/tex-archive/}

\bibitem[EP]{EP} Guide to AMS editor's package,
\texttt{http://www.ams.org/authors/editpkg.html}

\bibitem[GB]{GB} David Carlisle, ``Packages in the `graphics' bundle'',
available in electronic form as
\filnam{macros/latex/required/graphics/grfguide.tex} or \filnam{.ps}
from\\ \texttt{http://www.ctan.org/tex-archive/}

\bibitem[GL]{GL} Helmut Kopka and Patrick Daly, \textit{A Guide to
 \latex/}, 4th ed., Addison-Wesley, Reading, MA, 2004.

\bibitem[HBW]{HBW} Nicholas J. Higham, \textit{Handbook of Writing for
the Mathematical Sciences}, 2nd ed., SIAM, Philadelphia, PA, 1998.

\bibitem[HWM]{HWM}
Norman E. Steenrod, Paul R. Halmos, Menahem M. Schiffer, and Jean A.
Dieudonn\'e, \textit{How to Write Mathematics}, 5th printing 1995,
Amer. Math. Soc., Providence, RI, 1973.

\bibitem[Joy]{Joy} M. D. Spivak, \textit{The Joy of \tex/},
2nd revised ed., Amer. Math. Soc., Providence, RI, 1990.

\bibitem[LC]{LC} Frank Mittelbach and Michel Goossens, with
Johannes Braams, David Carlisle and Chris Rowley,
\textit{The \latex/ Companion}, 2nd ed., Addison-Wesley Co.,
Reading, MA, 2004.

\bibitem[LGC]{LGC} Michel Goossens, Sebastian Rahtz, and Frank Mittelbach,
\textit{The \latex/ graphics Companion}, Addison-Wesley Co., Reading,
MA, 1997.

\bibitem[MIT]{MIT} Ellen E. Swanson, updated by Arlene O'Sean and
Antoinette Schleyer, \textit{Mathematics into Type},
Updated ed., Amer. Math. Soc., Providence, RI, 1999.

\bibitem[ML]{ML} George Gr\"{a}tzer, \textit{Math into \latex/},
3rd ed., Birkh\"{a}user, Boston, MA/Springer Verlag, NY, 2000.

\bibitem[MSC]{MSC} \textit{\textup{2000} Mathematics subject classification},
Amer. Math. Soc., Providence, RI, 1999; searchable on-line at
\texttt{http://www.ams.org/msc/}.

\end{thebibliography}

\end{document}
