% \iffalse meta-comment
%
% File: fibnum.dtx
% Version: 2016/05/16 v1.1
% Info: Fibonacci numbers
%
% Copyright (C) 2012 by
%    Heiko Oberdiek <heiko.oberdiek at googlemail.com>
%    2016
%    https://github.com/ho-tex/oberdiek/issues
%
% This work may be distributed and/or modified under the
% conditions of the LaTeX Project Public License, either
% version 1.3c of this license or (at your option) any later
% version. This version of this license is in
%    http://www.latex-project.org/lppl/lppl-1-3c.txt
% and the latest version of this license is in
%    http://www.latex-project.org/lppl.txt
% and version 1.3 or later is part of all distributions of
% LaTeX version 2005/12/01 or later.
%
% This work has the LPPL maintenance status "maintained".
%
% This Current Maintainer of this work is Heiko Oberdiek.
%
% The Base Interpreter refers to any `TeX-Format',
% because some files are installed in TDS:tex/generic//.
%
% This work consists of the main source file fibnum.dtx
% and the derived files
%    fibnum.sty, fibnum.pdf, fibnum.ins, fibnum.drv, fibnum.bib,
%    fibnum-test1.tex, fibnum-test-calc.tex.
%
% Distribution:
%    CTAN:macros/latex/contrib/oberdiek/fibnum.dtx
%    CTAN:macros/latex/contrib/oberdiek/fibnum.pdf
%
% Unpacking:
%    (a) If fibnum.ins is present:
%           tex fibnum.ins
%    (b) Without fibnum.ins:
%           tex fibnum.dtx
%    (c) If you insist on using LaTeX
%           latex \let\install=y% \iffalse meta-comment
%
% File: fibnum.dtx
% Version: 2016/05/16 v1.1
% Info: Fibonacci numbers
%
% Copyright (C) 2012 by
%    Heiko Oberdiek <heiko.oberdiek at googlemail.com>
%    2016
%    https://github.com/ho-tex/oberdiek/issues
%
% This work may be distributed and/or modified under the
% conditions of the LaTeX Project Public License, either
% version 1.3c of this license or (at your option) any later
% version. This version of this license is in
%    http://www.latex-project.org/lppl/lppl-1-3c.txt
% and the latest version of this license is in
%    http://www.latex-project.org/lppl.txt
% and version 1.3 or later is part of all distributions of
% LaTeX version 2005/12/01 or later.
%
% This work has the LPPL maintenance status "maintained".
%
% This Current Maintainer of this work is Heiko Oberdiek.
%
% The Base Interpreter refers to any `TeX-Format',
% because some files are installed in TDS:tex/generic//.
%
% This work consists of the main source file fibnum.dtx
% and the derived files
%    fibnum.sty, fibnum.pdf, fibnum.ins, fibnum.drv, fibnum.bib,
%    fibnum-test1.tex, fibnum-test-calc.tex.
%
% Distribution:
%    CTAN:macros/latex/contrib/oberdiek/fibnum.dtx
%    CTAN:macros/latex/contrib/oberdiek/fibnum.pdf
%
% Unpacking:
%    (a) If fibnum.ins is present:
%           tex fibnum.ins
%    (b) Without fibnum.ins:
%           tex fibnum.dtx
%    (c) If you insist on using LaTeX
%           latex \let\install=y% \iffalse meta-comment
%
% File: fibnum.dtx
% Version: 2016/05/16 v1.1
% Info: Fibonacci numbers
%
% Copyright (C) 2012 by
%    Heiko Oberdiek <heiko.oberdiek at googlemail.com>
%    2016
%    https://github.com/ho-tex/oberdiek/issues
%
% This work may be distributed and/or modified under the
% conditions of the LaTeX Project Public License, either
% version 1.3c of this license or (at your option) any later
% version. This version of this license is in
%    http://www.latex-project.org/lppl/lppl-1-3c.txt
% and the latest version of this license is in
%    http://www.latex-project.org/lppl.txt
% and version 1.3 or later is part of all distributions of
% LaTeX version 2005/12/01 or later.
%
% This work has the LPPL maintenance status "maintained".
%
% This Current Maintainer of this work is Heiko Oberdiek.
%
% The Base Interpreter refers to any `TeX-Format',
% because some files are installed in TDS:tex/generic//.
%
% This work consists of the main source file fibnum.dtx
% and the derived files
%    fibnum.sty, fibnum.pdf, fibnum.ins, fibnum.drv, fibnum.bib,
%    fibnum-test1.tex, fibnum-test-calc.tex.
%
% Distribution:
%    CTAN:macros/latex/contrib/oberdiek/fibnum.dtx
%    CTAN:macros/latex/contrib/oberdiek/fibnum.pdf
%
% Unpacking:
%    (a) If fibnum.ins is present:
%           tex fibnum.ins
%    (b) Without fibnum.ins:
%           tex fibnum.dtx
%    (c) If you insist on using LaTeX
%           latex \let\install=y% \iffalse meta-comment
%
% File: fibnum.dtx
% Version: 2016/05/16 v1.1
% Info: Fibonacci numbers
%
% Copyright (C) 2012 by
%    Heiko Oberdiek <heiko.oberdiek at googlemail.com>
%    2016
%    https://github.com/ho-tex/oberdiek/issues
%
% This work may be distributed and/or modified under the
% conditions of the LaTeX Project Public License, either
% version 1.3c of this license or (at your option) any later
% version. This version of this license is in
%    http://www.latex-project.org/lppl/lppl-1-3c.txt
% and the latest version of this license is in
%    http://www.latex-project.org/lppl.txt
% and version 1.3 or later is part of all distributions of
% LaTeX version 2005/12/01 or later.
%
% This work has the LPPL maintenance status "maintained".
%
% This Current Maintainer of this work is Heiko Oberdiek.
%
% The Base Interpreter refers to any `TeX-Format',
% because some files are installed in TDS:tex/generic//.
%
% This work consists of the main source file fibnum.dtx
% and the derived files
%    fibnum.sty, fibnum.pdf, fibnum.ins, fibnum.drv, fibnum.bib,
%    fibnum-test1.tex, fibnum-test-calc.tex.
%
% Distribution:
%    CTAN:macros/latex/contrib/oberdiek/fibnum.dtx
%    CTAN:macros/latex/contrib/oberdiek/fibnum.pdf
%
% Unpacking:
%    (a) If fibnum.ins is present:
%           tex fibnum.ins
%    (b) Without fibnum.ins:
%           tex fibnum.dtx
%    (c) If you insist on using LaTeX
%           latex \let\install=y\input{fibnum.dtx}
%        (quote the arguments according to the demands of your shell)
%
% Documentation:
%    (a) If fibnum.drv is present:
%           latex fibnum.drv
%    (b) Without fibnum.drv:
%           latex fibnum.dtx; ...
%    The class ltxdoc loads the configuration file ltxdoc.cfg
%    if available. Here you can specify further options, e.g.
%    use A4 as paper format:
%       \PassOptionsToClass{a4paper}{article}
%
%    Programm calls to get the documentation (example):
%       pdflatex fibnum.dtx
%       bibtex fibnum.aux
%       makeindex -s gind.ist fibnum.idx
%       pdflatex fibnum.dtx
%       makeindex -s gind.ist fibnum.idx
%       pdflatex fibnum.dtx
%
% Installation:
%    TDS:tex/generic/oberdiek/fibnum.sty
%    TDS:doc/latex/oberdiek/fibnum.pdf
%    TDS:doc/latex/oberdiek/test/fibnum-test1.tex
%    TDS:doc/latex/oberdiek/test/fibnum-test-calc.tex
%    TDS:source/latex/oberdiek/fibnum.dtx
%
%<*ignore>
\begingroup
  \catcode123=1 %
  \catcode125=2 %
  \def\x{LaTeX2e}%
\expandafter\endgroup
\ifcase 0\ifx\install y1\fi\expandafter
         \ifx\csname processbatchFile\endcsname\relax\else1\fi
         \ifx\fmtname\x\else 1\fi\relax
\else\csname fi\endcsname
%</ignore>
%<*install>
\input docstrip.tex
\Msg{************************************************************************}
\Msg{* Installation}
\Msg{* Package: fibnum 2016/05/16 v1.1 Fibonacci numbers (HO)}
\Msg{************************************************************************}

\keepsilent
\askforoverwritefalse

\let\MetaPrefix\relax
\preamble

This is a generated file.

Project: fibnum
Version: 2016/05/16 v1.1

Copyright (C) 2012 by
   Heiko Oberdiek <heiko.oberdiek at googlemail.com>

This work may be distributed and/or modified under the
conditions of the LaTeX Project Public License, either
version 1.3c of this license or (at your option) any later
version. This version of this license is in
   http://www.latex-project.org/lppl/lppl-1-3c.txt
and the latest version of this license is in
   http://www.latex-project.org/lppl.txt
and version 1.3 or later is part of all distributions of
LaTeX version 2005/12/01 or later.

This work has the LPPL maintenance status "maintained".

This Current Maintainer of this work is Heiko Oberdiek.

The Base Interpreter refers to any `TeX-Format',
because some files are installed in TDS:tex/generic//.

This work consists of the main source file fibnum.dtx
and the derived files
   fibnum.sty, fibnum.pdf, fibnum.ins, fibnum.drv, fibnum.bib,
   fibnum-test1.tex, fibnum-test-calc.tex.

\endpreamble
\let\MetaPrefix\DoubleperCent

\generate{%
  \file{fibnum.ins}{\from{fibnum.dtx}{install}}%
  \file{fibnum.drv}{\from{fibnum.dtx}{driver}}%
  \nopreamble
  \nopostamble
  \file{fibnum.bib}{\from{fibnum.dtx}{bib}}%
  \usepreamble\defaultpreamble
  \usepostamble\defaultpostamble
  \usedir{tex/generic/oberdiek}%
  \file{fibnum.sty}{\from{fibnum.dtx}{package}}%
  \usedir{doc/latex/oberdiek/test}%
  \file{fibnum-test1.tex}{\from{fibnum.dtx}{test1}}%
  \file{fibnum-test-calc.tex}{\from{fibnum.dtx}{test-calc}}%
}

\catcode32=13\relax% active space
\let =\space%
\Msg{************************************************************************}
\Msg{*}
\Msg{* To finish the installation you have to move the following}
\Msg{* file into a directory searched by TeX:}
\Msg{*}
\Msg{*     fibnum.sty}
\Msg{*}
\Msg{* To produce the documentation run the file `fibnum.drv'}
\Msg{* through LaTeX.}
\Msg{*}
\Msg{* Happy TeXing!}
\Msg{*}
\Msg{************************************************************************}

\endbatchfile
%</install>
%<*bib>
@online{texhax:abraham,
  author={Abraham, Jan},
  title={[texhax] Beginner in TEX MACRO to compute functions},
  date={2012-04-07},
  url={http://tug.org/pipermail/texhax/2012-April/019146.html},
  urldate={2012-04-08},
}
@article{knuth:negafibonacci,
  author={Knuth, Donald E.},
  title={Negafibonacci Numbers and the Hyperbolic Plane},
  date={2008-12-11},
  url={http://research.allacademic.com/meta/p206842_index.html},
}
@online{wikipedia:negafibonacci,
  author={{Wikipedia contributors}},
  organization={{Wikipedia, The Free Encyclopedia}},
  title={Fibonacci numbers},
  language={langenglish},
  version={486266088},
  date={2012-04-08},
  url={http://en.wikipedia.org/w/index.php?title=Fibonacci_number&oldid=486266088},
  urldate={2012-04-08},
}
%</bib>
%<*ignore>
\fi
%</ignore>
%<*driver>
\NeedsTeXFormat{LaTeX2e}
\ProvidesFile{fibnum.drv}%
  [2016/05/16 v1.1 Fibonacci numbers (HO)]%
\documentclass{ltxdoc}
\usepackage{amsmath,amsfonts}
\usepackage{siunitx}
\usepackage{array}
\usepackage{tabularx}
\usepackage{fibnum}[2016/05/16]
\usepackage{holtxdoc}[2011/11/22]
\usepackage{csquotes}
\usepackage[
  bibencoding=ascii,
  alldates=iso8601,
]{biblatex}[2011/11/13]
\bibliography{oberdiek-source}
\bibliography{fibnum}
\begin{document}
  \DocInput{fibnum.dtx}%
\end{document}
%</driver>
% \fi
%
%
% \CharacterTable
%  {Upper-case    \A\B\C\D\E\F\G\H\I\J\K\L\M\N\O\P\Q\R\S\T\U\V\W\X\Y\Z
%   Lower-case    \a\b\c\d\e\f\g\h\i\j\k\l\m\n\o\p\q\r\s\t\u\v\w\x\y\z
%   Digits        \0\1\2\3\4\5\6\7\8\9
%   Exclamation   \!     Double quote  \"     Hash (number) \#
%   Dollar        \$     Percent       \%     Ampersand     \&
%   Acute accent  \'     Left paren    \(     Right paren   \)
%   Asterisk      \*     Plus          \+     Comma         \,
%   Minus         \-     Point         \.     Solidus       \/
%   Colon         \:     Semicolon     \;     Less than     \<
%   Equals        \=     Greater than  \>     Question mark \?
%   Commercial at \@     Left bracket  \[     Backslash     \\
%   Right bracket \]     Circumflex    \^     Underscore    \_
%   Grave accent  \`     Left brace    \{     Vertical bar  \|
%   Right brace   \}     Tilde         \~}
%
% \GetFileInfo{fibnum.drv}
%
% \title{The \xpackage{fibnum} package}
% \date{2016/05/16 v1.1}
% \author{Heiko Oberdiek\thanks
% {Please report any issues at https://github.com/ho-tex/oberdiek/issues}\\
% \xemail{heiko.oberdiek at googlemail.com}}
%
% \maketitle
%
% \begin{abstract}
% The package \xpackage{fibnum} provides expandable fibonacci
% numbers for both \hologo{LaTeX} and \hologo{plainTeX}.
% \end{abstract}
%
% \tableofcontents
%
% \section{Documentation}
%
% In the mailing list \textsf{texhax} Jan Abraham asked the question,
% how to get Fibonacci numbers in \hologo{TeX} \cite{texhax:abraham}:
% \begin{quote}
% Write a Macro in \hologo{TeX} that compute the function |\fib{m}|
% All fibonacci numbers from 1 to $m$ ($m < 40$).
% \end{quote}
% This packages provides an expandable implementation for the
% calculation of these numbers for a much larger set of indexes.
% For practical reasons the index is restricted to the same limitations
% that apply for \hologo{TeX} integer numbers.
% The range of the Fibonacci numbers, however, are not limited
% by the algorithm. They are only restricted to memory limitations,
% if they are hit.
%
% The package is loaded as \hologo{LaTeX} package in \hologo{LaTeX}:
% \begin{quote}
%   |\usepackage{fibnum}|
% \end{quote}
% and as file in \hologo{plainTeX}:
% \begin{quote}
%   |\input fibnum.sty|
% \end{quote}
% The package does not know any options and it provides
% the macros \cs{fibnum} and \cs{fibnumPreCalc}.
%
% \begin{declcs}{fibnum} \M{index}
% \end{declcs}
% Macro \cs{fibnum} expects a \hologo{TeX} number as \meta{index}
% in the official \hologo{TeX} number range from $-(2^{31}-1)$ up to
% $2^{31}-1$. In exact two expansion steps the macro expands to
% the Fibnoacci number $F_{\text{\meta{index}}}$. In case of a negative
% \meta{index}, the ``negafibonacci'' number \cite{wikipedia:negafibonacci}
% is used. Formally the Fibonacci number $F_n$ with integer
% index~$n$, $n\in\mathbb{Z}$ and
% $n\in[\num{-2147483647},\num{2147483647}]$ that is returned by macro
% \cs{fibnum} with numerical argument $n$ is defined the following way:
% \begin{gather}
%   \label{eq:def}
%   F_n =
%   \begin{cases}
%     0 & \text{for $n=0$}\\
%     1 & \text{for $n=1$}\\
%     F_{n-1} + F_{n-2} & \text{for $n>1$}\\
%     (-1)^{n+1}F_n & \text{for $n<0$}
%   \end{cases}
% \end{gather}
% Examples:
% \begin{quote}
%   \makeatletter
%   \def\x#1{\cs{fibnum}\{#1\}&
%     \edef\X{\fibnum{#1}}\edef\Y{\expandafter\ltx@car\X\@nil}^^A
%     \if-\Y
%       \edef\X{\expandafter\ltx@cdr\X\@nil}^^A
%       \noindent
%       \llap{-}\X
%     \else
%       \X
%     \fi
%     \tabularnewline
%   }
%   \def\y{\multicolumn{1}{@{}c@{}}{$\vdots$}\tabularnewline}
%   \DeclareUrlCommand\UrlNum{^^A
%     \urlstyle{tt}^^A
%     \def\UrlBreaks{\do\0\do\1\do\2\do\3\do\4\do\5\do\6\do\7\do\8\do\9}^^A
%   }
%   \begin{tabularx}{\dimexpr\linewidth+5.7pt\relax}{@{}>{\ttfamily}l@{ $\rightarrow$ \hphantom{\ttfamily-}}>{\ttfamily}X@{}}
%     \x{-6}
%     \x{-5}
%     \x{-4}
%     \x{-3}
%     \x{-2}
%     \x{-1}
%     \x{0}
%     \x{1}
%     \x{2}
%     \x{3}
%     \x{4}
%     \x{5}
%     \x{6}
%     \y
%     \x{10}
%     \y
%     \x{46}
%     \y
%     \cs{fibnum}\{100\} & 354224848179261915075
%     \tabularnewline
%     \y
%     \cs{fibnum}\{200\} & 280571172992510140037611932413038677189525
%     \tabularnewline
%     \y
%     \cs{fibnum}\{1000\} &
%       \raggedright
%       \UrlNum{^^A
%         434665576869374564356885276750406258025646^^A
%         605173717804024817290895365554179490518904^^A
%         038798400792551692959225930803226347752096^^A
%         896232398733224711616429964409065331879382^^A
%         98969649928516003704476137795166849228875^^A
%       }
%     \tabularnewline
%   \end{tabularx}\kern-5.7pt\mbox{}
% \end{quote}
%
% \begin{declcs}{fibnumPreCalc} \M{index}
% \end{declcs}
% The package already provides precalculated Fibonacci numbers up to
% index~46. That means that calculations are not necessary for
% Fibonacci numbers that fit into the range of \hologo{TeX}
% numbers. Because macro \cs{fibnum} is expandable, it cannot
% store calculated Fibonacci numbers for later use. Macro definitions
% are forbidden in expandable contexts. If larger Fibonacci numbers
% are used more than once, than the compilation time can be shortened
% by calculating and storing the Fibonacci numbers beforehand.
% The argument \meta{index} is a \hologo{TeX} number and macro
% \cs{fibnumPreCalc} ensures that the Fibonacci numbers
% $F_0$ up to $F_{\lvert\text{\meta{index}}\rvert}$ that are not
% already known are calculated
% and stored in internal macros. Internally only non-negative
% Fibonacci numbers are stored. If \meta{index} is negative, then
% the needed positive Fibonacci numbers are calculated and stored.
% Example:
% \begin{quote}
%   \def\x#1{\begingroup\itshape\texttt{\%} #1\endgroup}
%   |\fibnumPreCalc{50}|\\
%   \x{calculates and stores the values for indexes 47..50.}\\
%   \x{(Values for 0..46 are already stored by the package.)}\\
%   |\fibnum{49}| \x{uses the stored value}\\
%   |\fibnum{51}|
%   \x{only calculates $F_{51}$ from stored values $F_{49}$ and $F_{50}$}\\
%   |\fibnumPreCalc{100}|\\
%   \x{calculates and stores the values for indexes 51..100}\\
%   |\fibnum{100}| \x{uses the stored value for $F_{100}$}\\
%   |\fibnum{-100}|
%   \x{uses the stored value for $F_{100}$}\\
%   \x{$F_{-100}=-F_{100}$ according to equation \eqref{eq:def}.}
% \end{quote}
%
% \StopEventually{
% }
%
% \section{Implementation}
%
% \subsection{Identification}
%
%    \begin{macrocode}
%<*package>
%    \end{macrocode}
%    Reload check, especially if the package is not used with \LaTeX.
%    \begin{macrocode}
\begingroup\catcode61\catcode48\catcode32=10\relax%
  \catcode13=5 % ^^M
  \endlinechar=13 %
  \catcode35=6 % #
  \catcode39=12 % '
  \catcode44=12 % ,
  \catcode45=12 % -
  \catcode46=12 % .
  \catcode58=12 % :
  \catcode64=11 % @
  \catcode123=1 % {
  \catcode125=2 % }
  \expandafter\let\expandafter\x\csname ver@fibnum.sty\endcsname
  \ifx\x\relax % plain-TeX, first loading
  \else
    \def\empty{}%
    \ifx\x\empty % LaTeX, first loading,
      % variable is initialized, but \ProvidesPackage not yet seen
    \else
      \expandafter\ifx\csname PackageInfo\endcsname\relax
        \def\x#1#2{%
          \immediate\write-1{Package #1 Info: #2.}%
        }%
      \else
        \def\x#1#2{\PackageInfo{#1}{#2, stopped}}%
      \fi
      \x{fibnum}{The package is already loaded}%
      \aftergroup\endinput
    \fi
  \fi
\endgroup%
%    \end{macrocode}
%    Package identification:
%    \begin{macrocode}
\begingroup\catcode61\catcode48\catcode32=10\relax%
  \catcode13=5 % ^^M
  \endlinechar=13 %
  \catcode35=6 % #
  \catcode39=12 % '
  \catcode40=12 % (
  \catcode41=12 % )
  \catcode44=12 % ,
  \catcode45=12 % -
  \catcode46=12 % .
  \catcode47=12 % /
  \catcode58=12 % :
  \catcode64=11 % @
  \catcode91=12 % [
  \catcode93=12 % ]
  \catcode123=1 % {
  \catcode125=2 % }
  \expandafter\ifx\csname ProvidesPackage\endcsname\relax
    \def\x#1#2#3[#4]{\endgroup
      \immediate\write-1{Package: #3 #4}%
      \xdef#1{#4}%
    }%
  \else
    \def\x#1#2[#3]{\endgroup
      #2[{#3}]%
      \ifx#1\@undefined
        \xdef#1{#3}%
      \fi
      \ifx#1\relax
        \xdef#1{#3}%
      \fi
    }%
  \fi
\expandafter\x\csname ver@fibnum.sty\endcsname
\ProvidesPackage{fibnum}%
  [2016/05/16 v1.1 Fibonacci numbers (HO)]%
%    \end{macrocode}
%
%    \begin{macrocode}
\begingroup\catcode61\catcode48\catcode32=10\relax%
  \catcode13=5 % ^^M
  \endlinechar=13 %
  \catcode123=1 % {
  \catcode125=2 % }
  \catcode64=11 % @
  \def\x{\endgroup
    \expandafter\edef\csname FibNum@AtEnd\endcsname{%
      \endlinechar=\the\endlinechar\relax
      \catcode13=\the\catcode13\relax
      \catcode32=\the\catcode32\relax
      \catcode35=\the\catcode35\relax
      \catcode61=\the\catcode61\relax
      \catcode64=\the\catcode64\relax
      \catcode123=\the\catcode123\relax
      \catcode125=\the\catcode125\relax
    }%
  }%
\x\catcode61\catcode48\catcode32=10\relax%
\catcode13=5 % ^^M
\endlinechar=13 %
\catcode35=6 % #
\catcode64=11 % @
\catcode123=1 % {
\catcode125=2 % }
\def\TMP@EnsureCode#1#2{%
  \edef\FibNum@AtEnd{%
    \FibNum@AtEnd
    \catcode#1=\the\catcode#1\relax
  }%
  \catcode#1=#2\relax
}
\TMP@EnsureCode{33}{12}% !
%\TMP@EnsureCode{36}{3}% $
%\TMP@EnsureCode{38}{4}% &
\TMP@EnsureCode{40}{12}% (
\TMP@EnsureCode{41}{12}% )
\TMP@EnsureCode{45}{12}% -
\TMP@EnsureCode{46}{12}% .
\TMP@EnsureCode{47}{12}% /
\TMP@EnsureCode{58}{12}% :
\TMP@EnsureCode{60}{12}% <
\TMP@EnsureCode{62}{12}% >
\TMP@EnsureCode{91}{12}% [
%\TMP@EnsureCode{96}{12}% `
\TMP@EnsureCode{93}{12}% ]
%\TMP@EnsureCode{94}{12}% ^ (superscript) (!)
%\TMP@EnsureCode{124}{12}% |
\edef\FibNum@AtEnd{\FibNum@AtEnd\noexpand\endinput}
%    \end{macrocode}
%
% \subsection{Package resources}
%
%    \begin{macrocode}
\begingroup\expandafter\expandafter\expandafter\endgroup
\expandafter\ifx\csname RequirePackage\endcsname\relax
  \def\TMP@RequirePackage#1[#2]{%
    \begingroup\expandafter\expandafter\expandafter\endgroup
    \expandafter\ifx\csname ver@#1.sty\endcsname\relax
      \input #1.sty\relax
    \fi
  }%
  \TMP@RequirePackage{ltxcmds}[2011/04/18]%
  \TMP@RequirePackage{intcalc}[2007/09/27]%
  \TMP@RequirePackage{bigintcalc}[2007/11/11]%
\else
  \RequirePackage{ltxcmds}[2011/04/18]%
  \RequirePackage{intcalc}[2007/09/27]%
  \RequirePackage{bigintcalc}[2007/11/11]%
\fi
%    \end{macrocode}
%
% \subsection{Setup precalculated values}
%
%    \begin{macrocode}
\def\FibNum@temp#1{%
  \expandafter\def\csname FibNum@#1\endcsname
}
\catcode46=9 % dots are ignored
\FibNum@temp{0}{0}
\FibNum@temp{1}{1}
\FibNum@temp{2}{1}
\FibNum@temp{3}{2}
\FibNum@temp{4}{3}
\FibNum@temp{5}{5}
\FibNum@temp{6}{8}
\FibNum@temp{7}{13}
\FibNum@temp{8}{21}
\FibNum@temp{9}{34}
\FibNum@temp{10}{55}
\FibNum@temp{11}{89}
\FibNum@temp{12}{144}
\FibNum@temp{13}{233}
\FibNum@temp{14}{377}
\FibNum@temp{15}{610}
\FibNum@temp{16}{987}
\FibNum@temp{17}{1.597}
\FibNum@temp{18}{2.584}
\FibNum@temp{19}{4.181}
\FibNum@temp{20}{6.765}
\FibNum@temp{21}{10.946}
\FibNum@temp{22}{17.711}
\FibNum@temp{23}{28.657}
\FibNum@temp{24}{46.368}
\FibNum@temp{25}{75.025}
\FibNum@temp{26}{121.393}
\FibNum@temp{27}{196.418}
\FibNum@temp{28}{317.811}
\FibNum@temp{29}{514.229}
\FibNum@temp{30}{832.040}
\FibNum@temp{31}{1.346.269}
\FibNum@temp{32}{2.178.309}
\FibNum@temp{33}{3.524.578}
\FibNum@temp{34}{5.702.887}
\FibNum@temp{35}{9.227.465}
\FibNum@temp{36}{14.930.352}
\FibNum@temp{37}{24.157.817}
\FibNum@temp{38}{39.088.169}
\FibNum@temp{39}{63.245.986}
\FibNum@temp{40}{102.334.155}
\FibNum@temp{41}{165.580.141}
\FibNum@temp{42}{267.914.296}
\FibNum@temp{43}{433.494.437}
\FibNum@temp{44}{701.408.733}
\FibNum@temp{45}{1.134.903.170}
\FibNum@temp{46}{1.836.311.903}
%    \end{macrocode}
%    \begin{macro}{\FibNum@max}
%    \begin{macrocode}
\def\FibNum@max{46}
%    \end{macrocode}
%    \end{macro}
%
% \subsection{Macros for precalculating values}
%
%    \begin{macro}{\fibnumPreCalc}
%    \begin{macrocode}
\def\fibnumPreCalc#1{%
  \expandafter\expandafter\expandafter
  \FibNum@PreCalc\intcalcNum{#1}/%
}
%    \end{macrocode}
%    \end{macro}
%    \begin{macro}{\FibNum@PreCalc}
%    \begin{macrocode}
\def\FibNum@PreCalc#1/{%
  \ifnum#1<\ltx@zero
    \expandafter\FibNum@PreCalc\ltx@gobble#1/%
  \else
    \ifnum#1>\FibNum@max
      \begingroup
        \ltx@LocDimenA=#1sp\relax
        \countdef\FibNum@i=255\relax
        \FibNum@i=\FibNum@max\relax
        \edef\FibNum@temp{%
          \csname FibNum@\the\FibNum@i\endcsname/%
        }%
        \advance\FibNum@i by -1\relax
        \edef\FibNum@temp{%
          \FibNum@temp
          \csname FibNum@\the\FibNum@i\endcsname
        }%
        \advance\FibNum@i\ltx@two
        \iftrue
          \expandafter\FibNum@PreCalcAux\FibNum@temp
        \fi
      \endgroup
    \fi
  \fi
}
%    \end{macrocode}
%    \end{macro}
%    \begin{macro}{\FibNum@PreCalcAux}
%    \begin{macrocode}
\def\FibNum@PreCalcAux#1/#2\fi{%
  \fi
  \edef\FibNum@temp{\BigIntCalcAdd#1!#2!}%
  \global\expandafter
  \let\csname FibNum@\the\FibNum@i\endcsname\FibNum@temp
  \ifnum\FibNum@i=\ltx@LocDimenA
    \xdef\FibNum@max{\the\FibNum@i}%
  \else
    \advance\FibNum@i\ltx@one
    \expandafter\FibNum@PreCalcAux\FibNum@temp/#1%
  \fi
}
%    \end{macrocode}
%    \end{macro}
%
% \subsection{Expandable calculations}
%
%    \begin{macro}{\fibnum}
%    \begin{macrocode}
\def\fibnum#1{%
  \romannumeral
  \expandafter\expandafter\expandafter\FibNum@Do\intcalcNum{#1}/%
}
%    \end{macrocode}
%    \end{macro}
%    \begin{macro}{\FibNum@Do}
%    \begin{macrocode}
\def\FibNum@Do#1/{%
  \ifnum#1<\ltx@zero
    \FibNum@ReturnAfterElseFiFi{%
      \ifodd#1 %
        \expandafter\expandafter\expandafter\ltx@zero
      \else
        \expandafter\expandafter\expandafter\ltx@zero
        \expandafter\expandafter\expandafter-%
      \fi
      \romannumeral
      \expandafter\FibNum@Do\ltx@gobble#1/%
    }%
  \else
    \ifnum\FibNum@max<#1 %
      \ltx@ReturnAfterElseFi{%
        \expandafter
        \FibNum@ExpCalc\number\expandafter\IntCalcInc\FibNum@max!%
        \expandafter\expandafter\expandafter/%
        \csname FibNum@\FibNum@max
        \expandafter\expandafter\expandafter\endcsname
        \expandafter\expandafter\expandafter/%
        \csname FibNum@\expandafter\IntCalcDec\FibNum@max!%
        \endcsname/%
        #1%
      }%
    \else
      \expandafter\expandafter\expandafter\ltx@zero
      \csname FibNum@#1\expandafter\expandafter\expandafter\endcsname
    \fi
  \fi
}
%    \end{macrocode}
%    \end{macro}
%    \begin{macro}{\FibNum@ReturnAfterElseFiFi}
%    \begin{macrocode}
\def\FibNum@ReturnAfterElseFiFi#1\else#2\fi\fi{\fi#1}
%    \end{macrocode}
%    \end{macro}
%    \begin{macro}{\FibNum@ExpCalc}
%    \begin{macrocode}
\def\FibNum@ExpCalc#1/#2/#3/#4\fi{%
  \fi
  \ifnum#1=#4 %
    \ltx@ReturnAfterElseFi{%
      \expandafter\expandafter\expandafter\ltx@zero
      \BigIntCalcAdd#2!#3!%
    }%
  \else
    \expandafter\FibNum@ExpCalc
    \number\IntCalcInc#1!%
    \expandafter\expandafter\expandafter/%
    \BigIntCalcAdd#2!#3!/%
    #2/#4%
  \fi
}
%    \end{macrocode}
%    \end{macro}
%
%    \begin{macrocode}
\FibNum@AtEnd%
%</package>
%    \end{macrocode}
%
% \section{Test}
%
% \subsection{Catcode checks for loading}
%
%    \begin{macrocode}
%<*test1>
%    \end{macrocode}
%    \begin{macrocode}
\catcode`\{=1 %
\catcode`\}=2 %
\catcode`\#=6 %
\catcode`\@=11 %
\expandafter\ifx\csname count@\endcsname\relax
  \countdef\count@=255 %
\fi
\expandafter\ifx\csname @gobble\endcsname\relax
  \long\def\@gobble#1{}%
\fi
\expandafter\ifx\csname @firstofone\endcsname\relax
  \long\def\@firstofone#1{#1}%
\fi
\expandafter\ifx\csname loop\endcsname\relax
  \expandafter\@firstofone
\else
  \expandafter\@gobble
\fi
{%
  \def\loop#1\repeat{%
    \def\body{#1}%
    \iterate
  }%
  \def\iterate{%
    \body
      \let\next\iterate
    \else
      \let\next\relax
    \fi
    \next
  }%
  \let\repeat=\fi
}%
\def\RestoreCatcodes{}
\count@=0 %
\loop
  \edef\RestoreCatcodes{%
    \RestoreCatcodes
    \catcode\the\count@=\the\catcode\count@\relax
  }%
\ifnum\count@<255 %
  \advance\count@ 1 %
\repeat

\def\RangeCatcodeInvalid#1#2{%
  \count@=#1\relax
  \loop
    \catcode\count@=15 %
  \ifnum\count@<#2\relax
    \advance\count@ 1 %
  \repeat
}
\def\RangeCatcodeCheck#1#2#3{%
  \count@=#1\relax
  \loop
    \ifnum#3=\catcode\count@
    \else
      \errmessage{%
        Character \the\count@\space
        with wrong catcode \the\catcode\count@\space
        instead of \number#3%
      }%
    \fi
  \ifnum\count@<#2\relax
    \advance\count@ 1 %
  \repeat
}
\def\space{ }
\expandafter\ifx\csname LoadCommand\endcsname\relax
  \def\LoadCommand{\input fibnum.sty\relax}%
\fi
\def\Test{%
  \RangeCatcodeInvalid{0}{47}%
  \RangeCatcodeInvalid{58}{64}%
  \RangeCatcodeInvalid{91}{96}%
  \RangeCatcodeInvalid{123}{255}%
  \catcode`\@=12 %
  \catcode`\\=0 %
  \catcode`\%=14 %
  \LoadCommand
  \RangeCatcodeCheck{0}{36}{15}%
  \RangeCatcodeCheck{37}{37}{14}%
  \RangeCatcodeCheck{38}{47}{15}%
  \RangeCatcodeCheck{48}{57}{12}%
  \RangeCatcodeCheck{58}{63}{15}%
  \RangeCatcodeCheck{64}{64}{12}%
  \RangeCatcodeCheck{65}{90}{11}%
  \RangeCatcodeCheck{91}{91}{15}%
  \RangeCatcodeCheck{92}{92}{0}%
  \RangeCatcodeCheck{93}{96}{15}%
  \RangeCatcodeCheck{97}{122}{11}%
  \RangeCatcodeCheck{123}{255}{15}%
  \RestoreCatcodes
}
\Test
\csname @@end\endcsname
\end
%    \end{macrocode}
%    \begin{macrocode}
%</test1>
%    \end{macrocode}
%
% \subsection{Test calculations}
%
%    \begin{macrocode}
%<*test-calc>
\catcode`\{=1 %
\catcode`\}=2 %
\catcode`\#=6 %
\catcode`\@=11 %
\begingroup\expandafter\expandafter\expandafter\endgroup
\expandafter\ifx\csname RequirePackage\endcsname\relax
  \input fibnum.sty\relax
\else
  \RequirePackage{fibnum}[2016/05/16]%
\fi
\def\TestSet{%
  \test{0}{0}%
  \test{1}{1}%
  \test{2}{1}%
  \test{3}{2}%
  \test{4}{3}%
  \test{5}{5}%
  \test{6}{8}%
  \test{7}{13}%
  \test{8}{21}%
  \test{9}{34}%
  \test{10}{55}%
  \test{11}{89}%
  \test{12}{144}%
  \test{13}{233}%
  \test{14}{377}%
  \test{15}{610}%
  \test{16}{987}%
  \test{17}{1597}%
  \test{18}{2584}%
  \test{19}{4181}%
  \test{20}{6765}%
  \test{21}{10946}%
  \test{22}{17711}%
  \test{23}{28657}%
  \test{24}{46368}%
  \test{25}{75025}%
  \test{26}{121393}%
  \test{27}{196418}%
  \test{28}{317811}%
  \test{29}{514229}%
  \test{30}{832040}%
  \test{31}{1346269}%
  \test{32}{2178309}%
  \test{33}{3524578}%
  \test{34}{5702887}%
  \test{35}{9227465}%
  \test{36}{14930352}%
  \test{37}{24157817}%
  \test{38}{39088169}%
  \test{39}{63245986}%
  \test{40}{102334155}%
  \test{41}{165580141}%
  \test{42}{267914296}%
  \test{43}{433494437}%
  \test{44}{701408733}%
  \test{45}{1134903170}%
  \test{46}{1836311903}%
  \test{47}{2971215073}%
  \test{48}{4807526976}%
  \test{49}{7778742049}%
  \test{50}{12586269025}%
  \test{51}{20365011074}%
  \test{52}{32951280099}%
  \test{53}{53316291173}%
  \test{54}{86267571272}%
  \test{55}{139583862445}%
  \test{56}{225851433717}%
  \test{57}{365435296162}%
  \test{58}{591286729879}%
  \test{59}{956722026041}%
  \test{60}{1548008755920}%
  \test{61}{2504730781961}%
  \test{62}{4052739537881}%
  \test{63}{6557470319842}%
  \test{64}{10610209857723}%
  \test{65}{17167680177565}%
  \test{66}{27777890035288}%
  \test{67}{44945570212853}%
  \test{68}{72723460248141}%
  \test{69}{117669030460994}%
  \test{70}{190392490709135}%
  \test{71}{308061521170129}%
  \test{72}{498454011879264}%
  \test{73}{806515533049393}%
}
\def\msg#{\immediate\write16}
\def\test#1#2{%
  \TestAux{#1}{#2}%
  \ifnum#1=0 %
  \else
    \ifodd#1 %
      \TestAux{-#1}{#2}%
    \else
      \TestAux{-#1}{-#2}%
    \fi
  \fi
}
\def\TestAux#1#2{%
  \def\Expected{#2}%
  \expandafter\expandafter\expandafter\def
  \expandafter\expandafter\expandafter\Result
  \expandafter\expandafter\expandafter{%
    \fibnum{#1}%
  }%
  \ltx@onelevel@sanitize\Result
  \ifx\Result\Expected
    \msg{* #1: ok.}%
  \else
    \msg{! fib(#1) = #2}%
    \errmessage{fib(#1) <> \Result}%
  \fi
}
\TestSet
\setbox0=\hbox{%
  \msg{* PreCalc{73}}%
  \fibnumPreCalc{73}%
}
\ifdim\wd0=0pt
\else
  \errmessage{Unwanted stuff in PreCalc}%
\fi
\TestSet
\csname @@end\endcsname\end
%</test-calc>
%    \end{macrocode}
%
% \section{Installation}
%
% \subsection{Download}
%
% \paragraph{Package.} This package is available on
% CTAN\footnote{\url{http://ctan.org/pkg/fibnum}}:
% \begin{description}
% \item[\CTAN{macros/latex/contrib/oberdiek/fibnum.dtx}] The source file.
% \item[\CTAN{macros/latex/contrib/oberdiek/fibnum.pdf}] Documentation.
% \end{description}
%
%
% \paragraph{Bundle.} All the packages of the bundle `oberdiek'
% are also available in a TDS compliant ZIP archive. There
% the packages are already unpacked and the documentation files
% are generated. The files and directories obey the TDS standard.
% \begin{description}
% \item[\CTAN{install/macros/latex/contrib/oberdiek.tds.zip}]
% \end{description}
% \emph{TDS} refers to the standard ``A Directory Structure
% for \TeX\ Files'' (\CTAN{tds/tds.pdf}). Directories
% with \xfile{texmf} in their name are usually organized this way.
%
% \subsection{Bundle installation}
%
% \paragraph{Unpacking.} Unpack the \xfile{oberdiek.tds.zip} in the
% TDS tree (also known as \xfile{texmf} tree) of your choice.
% Example (linux):
% \begin{quote}
%   |unzip oberdiek.tds.zip -d ~/texmf|
% \end{quote}
%
% \paragraph{Script installation.}
% Check the directory \xfile{TDS:scripts/oberdiek/} for
% scripts that need further installation steps.
% Package \xpackage{attachfile2} comes with the Perl script
% \xfile{pdfatfi.pl} that should be installed in such a way
% that it can be called as \texttt{pdfatfi}.
% Example (linux):
% \begin{quote}
%   |chmod +x scripts/oberdiek/pdfatfi.pl|\\
%   |cp scripts/oberdiek/pdfatfi.pl /usr/local/bin/|
% \end{quote}
%
% \subsection{Package installation}
%
% \paragraph{Unpacking.} The \xfile{.dtx} file is a self-extracting
% \docstrip\ archive. The files are extracted by running the
% \xfile{.dtx} through \plainTeX:
% \begin{quote}
%   \verb|tex fibnum.dtx|
% \end{quote}
%
% \paragraph{TDS.} Now the different files must be moved into
% the different directories in your installation TDS tree
% (also known as \xfile{texmf} tree):
% \begin{quote}
% \def\t{^^A
% \begin{tabular}{@{}>{\ttfamily}l@{ $\rightarrow$ }>{\ttfamily}l@{}}
%   fibnum.sty & tex/generic/oberdiek/fibnum.sty\\
%   fibnum.pdf & doc/latex/oberdiek/fibnum.pdf\\
%   test/fibnum-test1.tex & doc/latex/oberdiek/test/fibnum-test1.tex\\
%   test/fibnum-test-calc.tex & doc/latex/oberdiek/test/fibnum-test-calc.tex\\
%   fibnum.dtx & source/latex/oberdiek/fibnum.dtx\\
% \end{tabular}^^A
% }^^A
% \sbox0{\t}^^A
% \ifdim\wd0>\linewidth
%   \begingroup
%     \advance\linewidth by\leftmargin
%     \advance\linewidth by\rightmargin
%   \edef\x{\endgroup
%     \def\noexpand\lw{\the\linewidth}^^A
%   }\x
%   \def\lwbox{^^A
%     \leavevmode
%     \hbox to \linewidth{^^A
%       \kern-\leftmargin\relax
%       \hss
%       \usebox0
%       \hss
%       \kern-\rightmargin\relax
%     }^^A
%   }^^A
%   \ifdim\wd0>\lw
%     \sbox0{\small\t}^^A
%     \ifdim\wd0>\linewidth
%       \ifdim\wd0>\lw
%         \sbox0{\footnotesize\t}^^A
%         \ifdim\wd0>\linewidth
%           \ifdim\wd0>\lw
%             \sbox0{\scriptsize\t}^^A
%             \ifdim\wd0>\linewidth
%               \ifdim\wd0>\lw
%                 \sbox0{\tiny\t}^^A
%                 \ifdim\wd0>\linewidth
%                   \lwbox
%                 \else
%                   \usebox0
%                 \fi
%               \else
%                 \lwbox
%               \fi
%             \else
%               \usebox0
%             \fi
%           \else
%             \lwbox
%           \fi
%         \else
%           \usebox0
%         \fi
%       \else
%         \lwbox
%       \fi
%     \else
%       \usebox0
%     \fi
%   \else
%     \lwbox
%   \fi
% \else
%   \usebox0
% \fi
% \end{quote}
% If you have a \xfile{docstrip.cfg} that configures and enables \docstrip's
% TDS installing feature, then some files can already be in the right
% place, see the documentation of \docstrip.
%
% \subsection{Refresh file name databases}
%
% If your \TeX~distribution
% (\teTeX, \mikTeX, \dots) relies on file name databases, you must refresh
% these. For example, \teTeX\ users run \verb|texhash| or
% \verb|mktexlsr|.
%
% \subsection{Some details for the interested}
%
% \paragraph{Attached source.}
%
% The PDF documentation on CTAN also includes the
% \xfile{.dtx} source file. It can be extracted by
% AcrobatReader 6 or higher. Another option is \textsf{pdftk},
% e.g. unpack the file into the current directory:
% \begin{quote}
%   \verb|pdftk fibnum.pdf unpack_files output .|
% \end{quote}
%
% \paragraph{Unpacking with \LaTeX.}
% The \xfile{.dtx} chooses its action depending on the format:
% \begin{description}
% \item[\plainTeX:] Run \docstrip\ and extract the files.
% \item[\LaTeX:] Generate the documentation.
% \end{description}
% If you insist on using \LaTeX\ for \docstrip\ (really,
% \docstrip\ does not need \LaTeX), then inform the autodetect routine
% about your intention:
% \begin{quote}
%   \verb|latex \let\install=y\input{fibnum.dtx}|
% \end{quote}
% Do not forget to quote the argument according to the demands
% of your shell.
%
% \paragraph{Generating the documentation.}
% You can use both the \xfile{.dtx} or the \xfile{.drv} to generate
% the documentation. The process can be configured by the
% configuration file \xfile{ltxdoc.cfg}. For instance, put this
% line into this file, if you want to have A4 as paper format:
% \begin{quote}
%   \verb|\PassOptionsToClass{a4paper}{article}|
% \end{quote}
% An example follows how to generate the
% documentation with pdf\LaTeX:
% \begin{quote}
%\begin{verbatim}
%pdflatex fibnum.dtx
%bibtex fibnum.aux
%makeindex -s gind.ist fibnum.idx
%pdflatex fibnum.dtx
%makeindex -s gind.ist fibnum.idx
%pdflatex fibnum.dtx
%\end{verbatim}
% \end{quote}
%
% \printbibliography[
%   heading=bibnumbered,
% ]
%
% \begin{History}
%   \begin{Version}{2012/04/08 v1.0}
%   \item
%     First version.
%   \end{Version}
%   \begin{Version}{2016/05/16 v1.1}
%   \item
%     Documentation updates.
%   \end{Version}
% \end{History}
%
% \PrintIndex
%
% \Finale
\endinput

%        (quote the arguments according to the demands of your shell)
%
% Documentation:
%    (a) If fibnum.drv is present:
%           latex fibnum.drv
%    (b) Without fibnum.drv:
%           latex fibnum.dtx; ...
%    The class ltxdoc loads the configuration file ltxdoc.cfg
%    if available. Here you can specify further options, e.g.
%    use A4 as paper format:
%       \PassOptionsToClass{a4paper}{article}
%
%    Programm calls to get the documentation (example):
%       pdflatex fibnum.dtx
%       bibtex fibnum.aux
%       makeindex -s gind.ist fibnum.idx
%       pdflatex fibnum.dtx
%       makeindex -s gind.ist fibnum.idx
%       pdflatex fibnum.dtx
%
% Installation:
%    TDS:tex/generic/oberdiek/fibnum.sty
%    TDS:doc/latex/oberdiek/fibnum.pdf
%    TDS:doc/latex/oberdiek/test/fibnum-test1.tex
%    TDS:doc/latex/oberdiek/test/fibnum-test-calc.tex
%    TDS:source/latex/oberdiek/fibnum.dtx
%
%<*ignore>
\begingroup
  \catcode123=1 %
  \catcode125=2 %
  \def\x{LaTeX2e}%
\expandafter\endgroup
\ifcase 0\ifx\install y1\fi\expandafter
         \ifx\csname processbatchFile\endcsname\relax\else1\fi
         \ifx\fmtname\x\else 1\fi\relax
\else\csname fi\endcsname
%</ignore>
%<*install>
\input docstrip.tex
\Msg{************************************************************************}
\Msg{* Installation}
\Msg{* Package: fibnum 2016/05/16 v1.1 Fibonacci numbers (HO)}
\Msg{************************************************************************}

\keepsilent
\askforoverwritefalse

\let\MetaPrefix\relax
\preamble

This is a generated file.

Project: fibnum
Version: 2016/05/16 v1.1

Copyright (C) 2012 by
   Heiko Oberdiek <heiko.oberdiek at googlemail.com>

This work may be distributed and/or modified under the
conditions of the LaTeX Project Public License, either
version 1.3c of this license or (at your option) any later
version. This version of this license is in
   http://www.latex-project.org/lppl/lppl-1-3c.txt
and the latest version of this license is in
   http://www.latex-project.org/lppl.txt
and version 1.3 or later is part of all distributions of
LaTeX version 2005/12/01 or later.

This work has the LPPL maintenance status "maintained".

This Current Maintainer of this work is Heiko Oberdiek.

The Base Interpreter refers to any `TeX-Format',
because some files are installed in TDS:tex/generic//.

This work consists of the main source file fibnum.dtx
and the derived files
   fibnum.sty, fibnum.pdf, fibnum.ins, fibnum.drv, fibnum.bib,
   fibnum-test1.tex, fibnum-test-calc.tex.

\endpreamble
\let\MetaPrefix\DoubleperCent

\generate{%
  \file{fibnum.ins}{\from{fibnum.dtx}{install}}%
  \file{fibnum.drv}{\from{fibnum.dtx}{driver}}%
  \nopreamble
  \nopostamble
  \file{fibnum.bib}{\from{fibnum.dtx}{bib}}%
  \usepreamble\defaultpreamble
  \usepostamble\defaultpostamble
  \usedir{tex/generic/oberdiek}%
  \file{fibnum.sty}{\from{fibnum.dtx}{package}}%
  \usedir{doc/latex/oberdiek/test}%
  \file{fibnum-test1.tex}{\from{fibnum.dtx}{test1}}%
  \file{fibnum-test-calc.tex}{\from{fibnum.dtx}{test-calc}}%
}

\catcode32=13\relax% active space
\let =\space%
\Msg{************************************************************************}
\Msg{*}
\Msg{* To finish the installation you have to move the following}
\Msg{* file into a directory searched by TeX:}
\Msg{*}
\Msg{*     fibnum.sty}
\Msg{*}
\Msg{* To produce the documentation run the file `fibnum.drv'}
\Msg{* through LaTeX.}
\Msg{*}
\Msg{* Happy TeXing!}
\Msg{*}
\Msg{************************************************************************}

\endbatchfile
%</install>
%<*bib>
@online{texhax:abraham,
  author={Abraham, Jan},
  title={[texhax] Beginner in TEX MACRO to compute functions},
  date={2012-04-07},
  url={http://tug.org/pipermail/texhax/2012-April/019146.html},
  urldate={2012-04-08},
}
@article{knuth:negafibonacci,
  author={Knuth, Donald E.},
  title={Negafibonacci Numbers and the Hyperbolic Plane},
  date={2008-12-11},
  url={http://research.allacademic.com/meta/p206842_index.html},
}
@online{wikipedia:negafibonacci,
  author={{Wikipedia contributors}},
  organization={{Wikipedia, The Free Encyclopedia}},
  title={Fibonacci numbers},
  language={langenglish},
  version={486266088},
  date={2012-04-08},
  url={http://en.wikipedia.org/w/index.php?title=Fibonacci_number&oldid=486266088},
  urldate={2012-04-08},
}
%</bib>
%<*ignore>
\fi
%</ignore>
%<*driver>
\NeedsTeXFormat{LaTeX2e}
\ProvidesFile{fibnum.drv}%
  [2016/05/16 v1.1 Fibonacci numbers (HO)]%
\documentclass{ltxdoc}
\usepackage{amsmath,amsfonts}
\usepackage{siunitx}
\usepackage{array}
\usepackage{tabularx}
\usepackage{fibnum}[2016/05/16]
\usepackage{holtxdoc}[2011/11/22]
\usepackage{csquotes}
\usepackage[
  bibencoding=ascii,
  alldates=iso8601,
]{biblatex}[2011/11/13]
\bibliography{oberdiek-source}
\bibliography{fibnum}
\begin{document}
  \DocInput{fibnum.dtx}%
\end{document}
%</driver>
% \fi
%
%
% \CharacterTable
%  {Upper-case    \A\B\C\D\E\F\G\H\I\J\K\L\M\N\O\P\Q\R\S\T\U\V\W\X\Y\Z
%   Lower-case    \a\b\c\d\e\f\g\h\i\j\k\l\m\n\o\p\q\r\s\t\u\v\w\x\y\z
%   Digits        \0\1\2\3\4\5\6\7\8\9
%   Exclamation   \!     Double quote  \"     Hash (number) \#
%   Dollar        \$     Percent       \%     Ampersand     \&
%   Acute accent  \'     Left paren    \(     Right paren   \)
%   Asterisk      \*     Plus          \+     Comma         \,
%   Minus         \-     Point         \.     Solidus       \/
%   Colon         \:     Semicolon     \;     Less than     \<
%   Equals        \=     Greater than  \>     Question mark \?
%   Commercial at \@     Left bracket  \[     Backslash     \\
%   Right bracket \]     Circumflex    \^     Underscore    \_
%   Grave accent  \`     Left brace    \{     Vertical bar  \|
%   Right brace   \}     Tilde         \~}
%
% \GetFileInfo{fibnum.drv}
%
% \title{The \xpackage{fibnum} package}
% \date{2016/05/16 v1.1}
% \author{Heiko Oberdiek\thanks
% {Please report any issues at https://github.com/ho-tex/oberdiek/issues}\\
% \xemail{heiko.oberdiek at googlemail.com}}
%
% \maketitle
%
% \begin{abstract}
% The package \xpackage{fibnum} provides expandable fibonacci
% numbers for both \hologo{LaTeX} and \hologo{plainTeX}.
% \end{abstract}
%
% \tableofcontents
%
% \section{Documentation}
%
% In the mailing list \textsf{texhax} Jan Abraham asked the question,
% how to get Fibonacci numbers in \hologo{TeX} \cite{texhax:abraham}:
% \begin{quote}
% Write a Macro in \hologo{TeX} that compute the function |\fib{m}|
% All fibonacci numbers from 1 to $m$ ($m < 40$).
% \end{quote}
% This packages provides an expandable implementation for the
% calculation of these numbers for a much larger set of indexes.
% For practical reasons the index is restricted to the same limitations
% that apply for \hologo{TeX} integer numbers.
% The range of the Fibonacci numbers, however, are not limited
% by the algorithm. They are only restricted to memory limitations,
% if they are hit.
%
% The package is loaded as \hologo{LaTeX} package in \hologo{LaTeX}:
% \begin{quote}
%   |\usepackage{fibnum}|
% \end{quote}
% and as file in \hologo{plainTeX}:
% \begin{quote}
%   |\input fibnum.sty|
% \end{quote}
% The package does not know any options and it provides
% the macros \cs{fibnum} and \cs{fibnumPreCalc}.
%
% \begin{declcs}{fibnum} \M{index}
% \end{declcs}
% Macro \cs{fibnum} expects a \hologo{TeX} number as \meta{index}
% in the official \hologo{TeX} number range from $-(2^{31}-1)$ up to
% $2^{31}-1$. In exact two expansion steps the macro expands to
% the Fibnoacci number $F_{\text{\meta{index}}}$. In case of a negative
% \meta{index}, the ``negafibonacci'' number \cite{wikipedia:negafibonacci}
% is used. Formally the Fibonacci number $F_n$ with integer
% index~$n$, $n\in\mathbb{Z}$ and
% $n\in[\num{-2147483647},\num{2147483647}]$ that is returned by macro
% \cs{fibnum} with numerical argument $n$ is defined the following way:
% \begin{gather}
%   \label{eq:def}
%   F_n =
%   \begin{cases}
%     0 & \text{for $n=0$}\\
%     1 & \text{for $n=1$}\\
%     F_{n-1} + F_{n-2} & \text{for $n>1$}\\
%     (-1)^{n+1}F_n & \text{for $n<0$}
%   \end{cases}
% \end{gather}
% Examples:
% \begin{quote}
%   \makeatletter
%   \def\x#1{\cs{fibnum}\{#1\}&
%     \edef\X{\fibnum{#1}}\edef\Y{\expandafter\ltx@car\X\@nil}^^A
%     \if-\Y
%       \edef\X{\expandafter\ltx@cdr\X\@nil}^^A
%       \noindent
%       \llap{-}\X
%     \else
%       \X
%     \fi
%     \tabularnewline
%   }
%   \def\y{\multicolumn{1}{@{}c@{}}{$\vdots$}\tabularnewline}
%   \DeclareUrlCommand\UrlNum{^^A
%     \urlstyle{tt}^^A
%     \def\UrlBreaks{\do\0\do\1\do\2\do\3\do\4\do\5\do\6\do\7\do\8\do\9}^^A
%   }
%   \begin{tabularx}{\dimexpr\linewidth+5.7pt\relax}{@{}>{\ttfamily}l@{ $\rightarrow$ \hphantom{\ttfamily-}}>{\ttfamily}X@{}}
%     \x{-6}
%     \x{-5}
%     \x{-4}
%     \x{-3}
%     \x{-2}
%     \x{-1}
%     \x{0}
%     \x{1}
%     \x{2}
%     \x{3}
%     \x{4}
%     \x{5}
%     \x{6}
%     \y
%     \x{10}
%     \y
%     \x{46}
%     \y
%     \cs{fibnum}\{100\} & 354224848179261915075
%     \tabularnewline
%     \y
%     \cs{fibnum}\{200\} & 280571172992510140037611932413038677189525
%     \tabularnewline
%     \y
%     \cs{fibnum}\{1000\} &
%       \raggedright
%       \UrlNum{^^A
%         434665576869374564356885276750406258025646^^A
%         605173717804024817290895365554179490518904^^A
%         038798400792551692959225930803226347752096^^A
%         896232398733224711616429964409065331879382^^A
%         98969649928516003704476137795166849228875^^A
%       }
%     \tabularnewline
%   \end{tabularx}\kern-5.7pt\mbox{}
% \end{quote}
%
% \begin{declcs}{fibnumPreCalc} \M{index}
% \end{declcs}
% The package already provides precalculated Fibonacci numbers up to
% index~46. That means that calculations are not necessary for
% Fibonacci numbers that fit into the range of \hologo{TeX}
% numbers. Because macro \cs{fibnum} is expandable, it cannot
% store calculated Fibonacci numbers for later use. Macro definitions
% are forbidden in expandable contexts. If larger Fibonacci numbers
% are used more than once, than the compilation time can be shortened
% by calculating and storing the Fibonacci numbers beforehand.
% The argument \meta{index} is a \hologo{TeX} number and macro
% \cs{fibnumPreCalc} ensures that the Fibonacci numbers
% $F_0$ up to $F_{\lvert\text{\meta{index}}\rvert}$ that are not
% already known are calculated
% and stored in internal macros. Internally only non-negative
% Fibonacci numbers are stored. If \meta{index} is negative, then
% the needed positive Fibonacci numbers are calculated and stored.
% Example:
% \begin{quote}
%   \def\x#1{\begingroup\itshape\texttt{\%} #1\endgroup}
%   |\fibnumPreCalc{50}|\\
%   \x{calculates and stores the values for indexes 47..50.}\\
%   \x{(Values for 0..46 are already stored by the package.)}\\
%   |\fibnum{49}| \x{uses the stored value}\\
%   |\fibnum{51}|
%   \x{only calculates $F_{51}$ from stored values $F_{49}$ and $F_{50}$}\\
%   |\fibnumPreCalc{100}|\\
%   \x{calculates and stores the values for indexes 51..100}\\
%   |\fibnum{100}| \x{uses the stored value for $F_{100}$}\\
%   |\fibnum{-100}|
%   \x{uses the stored value for $F_{100}$}\\
%   \x{$F_{-100}=-F_{100}$ according to equation \eqref{eq:def}.}
% \end{quote}
%
% \StopEventually{
% }
%
% \section{Implementation}
%
% \subsection{Identification}
%
%    \begin{macrocode}
%<*package>
%    \end{macrocode}
%    Reload check, especially if the package is not used with \LaTeX.
%    \begin{macrocode}
\begingroup\catcode61\catcode48\catcode32=10\relax%
  \catcode13=5 % ^^M
  \endlinechar=13 %
  \catcode35=6 % #
  \catcode39=12 % '
  \catcode44=12 % ,
  \catcode45=12 % -
  \catcode46=12 % .
  \catcode58=12 % :
  \catcode64=11 % @
  \catcode123=1 % {
  \catcode125=2 % }
  \expandafter\let\expandafter\x\csname ver@fibnum.sty\endcsname
  \ifx\x\relax % plain-TeX, first loading
  \else
    \def\empty{}%
    \ifx\x\empty % LaTeX, first loading,
      % variable is initialized, but \ProvidesPackage not yet seen
    \else
      \expandafter\ifx\csname PackageInfo\endcsname\relax
        \def\x#1#2{%
          \immediate\write-1{Package #1 Info: #2.}%
        }%
      \else
        \def\x#1#2{\PackageInfo{#1}{#2, stopped}}%
      \fi
      \x{fibnum}{The package is already loaded}%
      \aftergroup\endinput
    \fi
  \fi
\endgroup%
%    \end{macrocode}
%    Package identification:
%    \begin{macrocode}
\begingroup\catcode61\catcode48\catcode32=10\relax%
  \catcode13=5 % ^^M
  \endlinechar=13 %
  \catcode35=6 % #
  \catcode39=12 % '
  \catcode40=12 % (
  \catcode41=12 % )
  \catcode44=12 % ,
  \catcode45=12 % -
  \catcode46=12 % .
  \catcode47=12 % /
  \catcode58=12 % :
  \catcode64=11 % @
  \catcode91=12 % [
  \catcode93=12 % ]
  \catcode123=1 % {
  \catcode125=2 % }
  \expandafter\ifx\csname ProvidesPackage\endcsname\relax
    \def\x#1#2#3[#4]{\endgroup
      \immediate\write-1{Package: #3 #4}%
      \xdef#1{#4}%
    }%
  \else
    \def\x#1#2[#3]{\endgroup
      #2[{#3}]%
      \ifx#1\@undefined
        \xdef#1{#3}%
      \fi
      \ifx#1\relax
        \xdef#1{#3}%
      \fi
    }%
  \fi
\expandafter\x\csname ver@fibnum.sty\endcsname
\ProvidesPackage{fibnum}%
  [2016/05/16 v1.1 Fibonacci numbers (HO)]%
%    \end{macrocode}
%
%    \begin{macrocode}
\begingroup\catcode61\catcode48\catcode32=10\relax%
  \catcode13=5 % ^^M
  \endlinechar=13 %
  \catcode123=1 % {
  \catcode125=2 % }
  \catcode64=11 % @
  \def\x{\endgroup
    \expandafter\edef\csname FibNum@AtEnd\endcsname{%
      \endlinechar=\the\endlinechar\relax
      \catcode13=\the\catcode13\relax
      \catcode32=\the\catcode32\relax
      \catcode35=\the\catcode35\relax
      \catcode61=\the\catcode61\relax
      \catcode64=\the\catcode64\relax
      \catcode123=\the\catcode123\relax
      \catcode125=\the\catcode125\relax
    }%
  }%
\x\catcode61\catcode48\catcode32=10\relax%
\catcode13=5 % ^^M
\endlinechar=13 %
\catcode35=6 % #
\catcode64=11 % @
\catcode123=1 % {
\catcode125=2 % }
\def\TMP@EnsureCode#1#2{%
  \edef\FibNum@AtEnd{%
    \FibNum@AtEnd
    \catcode#1=\the\catcode#1\relax
  }%
  \catcode#1=#2\relax
}
\TMP@EnsureCode{33}{12}% !
%\TMP@EnsureCode{36}{3}% $
%\TMP@EnsureCode{38}{4}% &
\TMP@EnsureCode{40}{12}% (
\TMP@EnsureCode{41}{12}% )
\TMP@EnsureCode{45}{12}% -
\TMP@EnsureCode{46}{12}% .
\TMP@EnsureCode{47}{12}% /
\TMP@EnsureCode{58}{12}% :
\TMP@EnsureCode{60}{12}% <
\TMP@EnsureCode{62}{12}% >
\TMP@EnsureCode{91}{12}% [
%\TMP@EnsureCode{96}{12}% `
\TMP@EnsureCode{93}{12}% ]
%\TMP@EnsureCode{94}{12}% ^ (superscript) (!)
%\TMP@EnsureCode{124}{12}% |
\edef\FibNum@AtEnd{\FibNum@AtEnd\noexpand\endinput}
%    \end{macrocode}
%
% \subsection{Package resources}
%
%    \begin{macrocode}
\begingroup\expandafter\expandafter\expandafter\endgroup
\expandafter\ifx\csname RequirePackage\endcsname\relax
  \def\TMP@RequirePackage#1[#2]{%
    \begingroup\expandafter\expandafter\expandafter\endgroup
    \expandafter\ifx\csname ver@#1.sty\endcsname\relax
      \input #1.sty\relax
    \fi
  }%
  \TMP@RequirePackage{ltxcmds}[2011/04/18]%
  \TMP@RequirePackage{intcalc}[2007/09/27]%
  \TMP@RequirePackage{bigintcalc}[2007/11/11]%
\else
  \RequirePackage{ltxcmds}[2011/04/18]%
  \RequirePackage{intcalc}[2007/09/27]%
  \RequirePackage{bigintcalc}[2007/11/11]%
\fi
%    \end{macrocode}
%
% \subsection{Setup precalculated values}
%
%    \begin{macrocode}
\def\FibNum@temp#1{%
  \expandafter\def\csname FibNum@#1\endcsname
}
\catcode46=9 % dots are ignored
\FibNum@temp{0}{0}
\FibNum@temp{1}{1}
\FibNum@temp{2}{1}
\FibNum@temp{3}{2}
\FibNum@temp{4}{3}
\FibNum@temp{5}{5}
\FibNum@temp{6}{8}
\FibNum@temp{7}{13}
\FibNum@temp{8}{21}
\FibNum@temp{9}{34}
\FibNum@temp{10}{55}
\FibNum@temp{11}{89}
\FibNum@temp{12}{144}
\FibNum@temp{13}{233}
\FibNum@temp{14}{377}
\FibNum@temp{15}{610}
\FibNum@temp{16}{987}
\FibNum@temp{17}{1.597}
\FibNum@temp{18}{2.584}
\FibNum@temp{19}{4.181}
\FibNum@temp{20}{6.765}
\FibNum@temp{21}{10.946}
\FibNum@temp{22}{17.711}
\FibNum@temp{23}{28.657}
\FibNum@temp{24}{46.368}
\FibNum@temp{25}{75.025}
\FibNum@temp{26}{121.393}
\FibNum@temp{27}{196.418}
\FibNum@temp{28}{317.811}
\FibNum@temp{29}{514.229}
\FibNum@temp{30}{832.040}
\FibNum@temp{31}{1.346.269}
\FibNum@temp{32}{2.178.309}
\FibNum@temp{33}{3.524.578}
\FibNum@temp{34}{5.702.887}
\FibNum@temp{35}{9.227.465}
\FibNum@temp{36}{14.930.352}
\FibNum@temp{37}{24.157.817}
\FibNum@temp{38}{39.088.169}
\FibNum@temp{39}{63.245.986}
\FibNum@temp{40}{102.334.155}
\FibNum@temp{41}{165.580.141}
\FibNum@temp{42}{267.914.296}
\FibNum@temp{43}{433.494.437}
\FibNum@temp{44}{701.408.733}
\FibNum@temp{45}{1.134.903.170}
\FibNum@temp{46}{1.836.311.903}
%    \end{macrocode}
%    \begin{macro}{\FibNum@max}
%    \begin{macrocode}
\def\FibNum@max{46}
%    \end{macrocode}
%    \end{macro}
%
% \subsection{Macros for precalculating values}
%
%    \begin{macro}{\fibnumPreCalc}
%    \begin{macrocode}
\def\fibnumPreCalc#1{%
  \expandafter\expandafter\expandafter
  \FibNum@PreCalc\intcalcNum{#1}/%
}
%    \end{macrocode}
%    \end{macro}
%    \begin{macro}{\FibNum@PreCalc}
%    \begin{macrocode}
\def\FibNum@PreCalc#1/{%
  \ifnum#1<\ltx@zero
    \expandafter\FibNum@PreCalc\ltx@gobble#1/%
  \else
    \ifnum#1>\FibNum@max
      \begingroup
        \ltx@LocDimenA=#1sp\relax
        \countdef\FibNum@i=255\relax
        \FibNum@i=\FibNum@max\relax
        \edef\FibNum@temp{%
          \csname FibNum@\the\FibNum@i\endcsname/%
        }%
        \advance\FibNum@i by -1\relax
        \edef\FibNum@temp{%
          \FibNum@temp
          \csname FibNum@\the\FibNum@i\endcsname
        }%
        \advance\FibNum@i\ltx@two
        \iftrue
          \expandafter\FibNum@PreCalcAux\FibNum@temp
        \fi
      \endgroup
    \fi
  \fi
}
%    \end{macrocode}
%    \end{macro}
%    \begin{macro}{\FibNum@PreCalcAux}
%    \begin{macrocode}
\def\FibNum@PreCalcAux#1/#2\fi{%
  \fi
  \edef\FibNum@temp{\BigIntCalcAdd#1!#2!}%
  \global\expandafter
  \let\csname FibNum@\the\FibNum@i\endcsname\FibNum@temp
  \ifnum\FibNum@i=\ltx@LocDimenA
    \xdef\FibNum@max{\the\FibNum@i}%
  \else
    \advance\FibNum@i\ltx@one
    \expandafter\FibNum@PreCalcAux\FibNum@temp/#1%
  \fi
}
%    \end{macrocode}
%    \end{macro}
%
% \subsection{Expandable calculations}
%
%    \begin{macro}{\fibnum}
%    \begin{macrocode}
\def\fibnum#1{%
  \romannumeral
  \expandafter\expandafter\expandafter\FibNum@Do\intcalcNum{#1}/%
}
%    \end{macrocode}
%    \end{macro}
%    \begin{macro}{\FibNum@Do}
%    \begin{macrocode}
\def\FibNum@Do#1/{%
  \ifnum#1<\ltx@zero
    \FibNum@ReturnAfterElseFiFi{%
      \ifodd#1 %
        \expandafter\expandafter\expandafter\ltx@zero
      \else
        \expandafter\expandafter\expandafter\ltx@zero
        \expandafter\expandafter\expandafter-%
      \fi
      \romannumeral
      \expandafter\FibNum@Do\ltx@gobble#1/%
    }%
  \else
    \ifnum\FibNum@max<#1 %
      \ltx@ReturnAfterElseFi{%
        \expandafter
        \FibNum@ExpCalc\number\expandafter\IntCalcInc\FibNum@max!%
        \expandafter\expandafter\expandafter/%
        \csname FibNum@\FibNum@max
        \expandafter\expandafter\expandafter\endcsname
        \expandafter\expandafter\expandafter/%
        \csname FibNum@\expandafter\IntCalcDec\FibNum@max!%
        \endcsname/%
        #1%
      }%
    \else
      \expandafter\expandafter\expandafter\ltx@zero
      \csname FibNum@#1\expandafter\expandafter\expandafter\endcsname
    \fi
  \fi
}
%    \end{macrocode}
%    \end{macro}
%    \begin{macro}{\FibNum@ReturnAfterElseFiFi}
%    \begin{macrocode}
\def\FibNum@ReturnAfterElseFiFi#1\else#2\fi\fi{\fi#1}
%    \end{macrocode}
%    \end{macro}
%    \begin{macro}{\FibNum@ExpCalc}
%    \begin{macrocode}
\def\FibNum@ExpCalc#1/#2/#3/#4\fi{%
  \fi
  \ifnum#1=#4 %
    \ltx@ReturnAfterElseFi{%
      \expandafter\expandafter\expandafter\ltx@zero
      \BigIntCalcAdd#2!#3!%
    }%
  \else
    \expandafter\FibNum@ExpCalc
    \number\IntCalcInc#1!%
    \expandafter\expandafter\expandafter/%
    \BigIntCalcAdd#2!#3!/%
    #2/#4%
  \fi
}
%    \end{macrocode}
%    \end{macro}
%
%    \begin{macrocode}
\FibNum@AtEnd%
%</package>
%    \end{macrocode}
%
% \section{Test}
%
% \subsection{Catcode checks for loading}
%
%    \begin{macrocode}
%<*test1>
%    \end{macrocode}
%    \begin{macrocode}
\catcode`\{=1 %
\catcode`\}=2 %
\catcode`\#=6 %
\catcode`\@=11 %
\expandafter\ifx\csname count@\endcsname\relax
  \countdef\count@=255 %
\fi
\expandafter\ifx\csname @gobble\endcsname\relax
  \long\def\@gobble#1{}%
\fi
\expandafter\ifx\csname @firstofone\endcsname\relax
  \long\def\@firstofone#1{#1}%
\fi
\expandafter\ifx\csname loop\endcsname\relax
  \expandafter\@firstofone
\else
  \expandafter\@gobble
\fi
{%
  \def\loop#1\repeat{%
    \def\body{#1}%
    \iterate
  }%
  \def\iterate{%
    \body
      \let\next\iterate
    \else
      \let\next\relax
    \fi
    \next
  }%
  \let\repeat=\fi
}%
\def\RestoreCatcodes{}
\count@=0 %
\loop
  \edef\RestoreCatcodes{%
    \RestoreCatcodes
    \catcode\the\count@=\the\catcode\count@\relax
  }%
\ifnum\count@<255 %
  \advance\count@ 1 %
\repeat

\def\RangeCatcodeInvalid#1#2{%
  \count@=#1\relax
  \loop
    \catcode\count@=15 %
  \ifnum\count@<#2\relax
    \advance\count@ 1 %
  \repeat
}
\def\RangeCatcodeCheck#1#2#3{%
  \count@=#1\relax
  \loop
    \ifnum#3=\catcode\count@
    \else
      \errmessage{%
        Character \the\count@\space
        with wrong catcode \the\catcode\count@\space
        instead of \number#3%
      }%
    \fi
  \ifnum\count@<#2\relax
    \advance\count@ 1 %
  \repeat
}
\def\space{ }
\expandafter\ifx\csname LoadCommand\endcsname\relax
  \def\LoadCommand{\input fibnum.sty\relax}%
\fi
\def\Test{%
  \RangeCatcodeInvalid{0}{47}%
  \RangeCatcodeInvalid{58}{64}%
  \RangeCatcodeInvalid{91}{96}%
  \RangeCatcodeInvalid{123}{255}%
  \catcode`\@=12 %
  \catcode`\\=0 %
  \catcode`\%=14 %
  \LoadCommand
  \RangeCatcodeCheck{0}{36}{15}%
  \RangeCatcodeCheck{37}{37}{14}%
  \RangeCatcodeCheck{38}{47}{15}%
  \RangeCatcodeCheck{48}{57}{12}%
  \RangeCatcodeCheck{58}{63}{15}%
  \RangeCatcodeCheck{64}{64}{12}%
  \RangeCatcodeCheck{65}{90}{11}%
  \RangeCatcodeCheck{91}{91}{15}%
  \RangeCatcodeCheck{92}{92}{0}%
  \RangeCatcodeCheck{93}{96}{15}%
  \RangeCatcodeCheck{97}{122}{11}%
  \RangeCatcodeCheck{123}{255}{15}%
  \RestoreCatcodes
}
\Test
\csname @@end\endcsname
\end
%    \end{macrocode}
%    \begin{macrocode}
%</test1>
%    \end{macrocode}
%
% \subsection{Test calculations}
%
%    \begin{macrocode}
%<*test-calc>
\catcode`\{=1 %
\catcode`\}=2 %
\catcode`\#=6 %
\catcode`\@=11 %
\begingroup\expandafter\expandafter\expandafter\endgroup
\expandafter\ifx\csname RequirePackage\endcsname\relax
  \input fibnum.sty\relax
\else
  \RequirePackage{fibnum}[2016/05/16]%
\fi
\def\TestSet{%
  \test{0}{0}%
  \test{1}{1}%
  \test{2}{1}%
  \test{3}{2}%
  \test{4}{3}%
  \test{5}{5}%
  \test{6}{8}%
  \test{7}{13}%
  \test{8}{21}%
  \test{9}{34}%
  \test{10}{55}%
  \test{11}{89}%
  \test{12}{144}%
  \test{13}{233}%
  \test{14}{377}%
  \test{15}{610}%
  \test{16}{987}%
  \test{17}{1597}%
  \test{18}{2584}%
  \test{19}{4181}%
  \test{20}{6765}%
  \test{21}{10946}%
  \test{22}{17711}%
  \test{23}{28657}%
  \test{24}{46368}%
  \test{25}{75025}%
  \test{26}{121393}%
  \test{27}{196418}%
  \test{28}{317811}%
  \test{29}{514229}%
  \test{30}{832040}%
  \test{31}{1346269}%
  \test{32}{2178309}%
  \test{33}{3524578}%
  \test{34}{5702887}%
  \test{35}{9227465}%
  \test{36}{14930352}%
  \test{37}{24157817}%
  \test{38}{39088169}%
  \test{39}{63245986}%
  \test{40}{102334155}%
  \test{41}{165580141}%
  \test{42}{267914296}%
  \test{43}{433494437}%
  \test{44}{701408733}%
  \test{45}{1134903170}%
  \test{46}{1836311903}%
  \test{47}{2971215073}%
  \test{48}{4807526976}%
  \test{49}{7778742049}%
  \test{50}{12586269025}%
  \test{51}{20365011074}%
  \test{52}{32951280099}%
  \test{53}{53316291173}%
  \test{54}{86267571272}%
  \test{55}{139583862445}%
  \test{56}{225851433717}%
  \test{57}{365435296162}%
  \test{58}{591286729879}%
  \test{59}{956722026041}%
  \test{60}{1548008755920}%
  \test{61}{2504730781961}%
  \test{62}{4052739537881}%
  \test{63}{6557470319842}%
  \test{64}{10610209857723}%
  \test{65}{17167680177565}%
  \test{66}{27777890035288}%
  \test{67}{44945570212853}%
  \test{68}{72723460248141}%
  \test{69}{117669030460994}%
  \test{70}{190392490709135}%
  \test{71}{308061521170129}%
  \test{72}{498454011879264}%
  \test{73}{806515533049393}%
}
\def\msg#{\immediate\write16}
\def\test#1#2{%
  \TestAux{#1}{#2}%
  \ifnum#1=0 %
  \else
    \ifodd#1 %
      \TestAux{-#1}{#2}%
    \else
      \TestAux{-#1}{-#2}%
    \fi
  \fi
}
\def\TestAux#1#2{%
  \def\Expected{#2}%
  \expandafter\expandafter\expandafter\def
  \expandafter\expandafter\expandafter\Result
  \expandafter\expandafter\expandafter{%
    \fibnum{#1}%
  }%
  \ltx@onelevel@sanitize\Result
  \ifx\Result\Expected
    \msg{* #1: ok.}%
  \else
    \msg{! fib(#1) = #2}%
    \errmessage{fib(#1) <> \Result}%
  \fi
}
\TestSet
\setbox0=\hbox{%
  \msg{* PreCalc{73}}%
  \fibnumPreCalc{73}%
}
\ifdim\wd0=0pt
\else
  \errmessage{Unwanted stuff in PreCalc}%
\fi
\TestSet
\csname @@end\endcsname\end
%</test-calc>
%    \end{macrocode}
%
% \section{Installation}
%
% \subsection{Download}
%
% \paragraph{Package.} This package is available on
% CTAN\footnote{\url{http://ctan.org/pkg/fibnum}}:
% \begin{description}
% \item[\CTAN{macros/latex/contrib/oberdiek/fibnum.dtx}] The source file.
% \item[\CTAN{macros/latex/contrib/oberdiek/fibnum.pdf}] Documentation.
% \end{description}
%
%
% \paragraph{Bundle.} All the packages of the bundle `oberdiek'
% are also available in a TDS compliant ZIP archive. There
% the packages are already unpacked and the documentation files
% are generated. The files and directories obey the TDS standard.
% \begin{description}
% \item[\CTAN{install/macros/latex/contrib/oberdiek.tds.zip}]
% \end{description}
% \emph{TDS} refers to the standard ``A Directory Structure
% for \TeX\ Files'' (\CTAN{tds/tds.pdf}). Directories
% with \xfile{texmf} in their name are usually organized this way.
%
% \subsection{Bundle installation}
%
% \paragraph{Unpacking.} Unpack the \xfile{oberdiek.tds.zip} in the
% TDS tree (also known as \xfile{texmf} tree) of your choice.
% Example (linux):
% \begin{quote}
%   |unzip oberdiek.tds.zip -d ~/texmf|
% \end{quote}
%
% \paragraph{Script installation.}
% Check the directory \xfile{TDS:scripts/oberdiek/} for
% scripts that need further installation steps.
% Package \xpackage{attachfile2} comes with the Perl script
% \xfile{pdfatfi.pl} that should be installed in such a way
% that it can be called as \texttt{pdfatfi}.
% Example (linux):
% \begin{quote}
%   |chmod +x scripts/oberdiek/pdfatfi.pl|\\
%   |cp scripts/oberdiek/pdfatfi.pl /usr/local/bin/|
% \end{quote}
%
% \subsection{Package installation}
%
% \paragraph{Unpacking.} The \xfile{.dtx} file is a self-extracting
% \docstrip\ archive. The files are extracted by running the
% \xfile{.dtx} through \plainTeX:
% \begin{quote}
%   \verb|tex fibnum.dtx|
% \end{quote}
%
% \paragraph{TDS.} Now the different files must be moved into
% the different directories in your installation TDS tree
% (also known as \xfile{texmf} tree):
% \begin{quote}
% \def\t{^^A
% \begin{tabular}{@{}>{\ttfamily}l@{ $\rightarrow$ }>{\ttfamily}l@{}}
%   fibnum.sty & tex/generic/oberdiek/fibnum.sty\\
%   fibnum.pdf & doc/latex/oberdiek/fibnum.pdf\\
%   test/fibnum-test1.tex & doc/latex/oberdiek/test/fibnum-test1.tex\\
%   test/fibnum-test-calc.tex & doc/latex/oberdiek/test/fibnum-test-calc.tex\\
%   fibnum.dtx & source/latex/oberdiek/fibnum.dtx\\
% \end{tabular}^^A
% }^^A
% \sbox0{\t}^^A
% \ifdim\wd0>\linewidth
%   \begingroup
%     \advance\linewidth by\leftmargin
%     \advance\linewidth by\rightmargin
%   \edef\x{\endgroup
%     \def\noexpand\lw{\the\linewidth}^^A
%   }\x
%   \def\lwbox{^^A
%     \leavevmode
%     \hbox to \linewidth{^^A
%       \kern-\leftmargin\relax
%       \hss
%       \usebox0
%       \hss
%       \kern-\rightmargin\relax
%     }^^A
%   }^^A
%   \ifdim\wd0>\lw
%     \sbox0{\small\t}^^A
%     \ifdim\wd0>\linewidth
%       \ifdim\wd0>\lw
%         \sbox0{\footnotesize\t}^^A
%         \ifdim\wd0>\linewidth
%           \ifdim\wd0>\lw
%             \sbox0{\scriptsize\t}^^A
%             \ifdim\wd0>\linewidth
%               \ifdim\wd0>\lw
%                 \sbox0{\tiny\t}^^A
%                 \ifdim\wd0>\linewidth
%                   \lwbox
%                 \else
%                   \usebox0
%                 \fi
%               \else
%                 \lwbox
%               \fi
%             \else
%               \usebox0
%             \fi
%           \else
%             \lwbox
%           \fi
%         \else
%           \usebox0
%         \fi
%       \else
%         \lwbox
%       \fi
%     \else
%       \usebox0
%     \fi
%   \else
%     \lwbox
%   \fi
% \else
%   \usebox0
% \fi
% \end{quote}
% If you have a \xfile{docstrip.cfg} that configures and enables \docstrip's
% TDS installing feature, then some files can already be in the right
% place, see the documentation of \docstrip.
%
% \subsection{Refresh file name databases}
%
% If your \TeX~distribution
% (\teTeX, \mikTeX, \dots) relies on file name databases, you must refresh
% these. For example, \teTeX\ users run \verb|texhash| or
% \verb|mktexlsr|.
%
% \subsection{Some details for the interested}
%
% \paragraph{Attached source.}
%
% The PDF documentation on CTAN also includes the
% \xfile{.dtx} source file. It can be extracted by
% AcrobatReader 6 or higher. Another option is \textsf{pdftk},
% e.g. unpack the file into the current directory:
% \begin{quote}
%   \verb|pdftk fibnum.pdf unpack_files output .|
% \end{quote}
%
% \paragraph{Unpacking with \LaTeX.}
% The \xfile{.dtx} chooses its action depending on the format:
% \begin{description}
% \item[\plainTeX:] Run \docstrip\ and extract the files.
% \item[\LaTeX:] Generate the documentation.
% \end{description}
% If you insist on using \LaTeX\ for \docstrip\ (really,
% \docstrip\ does not need \LaTeX), then inform the autodetect routine
% about your intention:
% \begin{quote}
%   \verb|latex \let\install=y% \iffalse meta-comment
%
% File: fibnum.dtx
% Version: 2016/05/16 v1.1
% Info: Fibonacci numbers
%
% Copyright (C) 2012 by
%    Heiko Oberdiek <heiko.oberdiek at googlemail.com>
%    2016
%    https://github.com/ho-tex/oberdiek/issues
%
% This work may be distributed and/or modified under the
% conditions of the LaTeX Project Public License, either
% version 1.3c of this license or (at your option) any later
% version. This version of this license is in
%    http://www.latex-project.org/lppl/lppl-1-3c.txt
% and the latest version of this license is in
%    http://www.latex-project.org/lppl.txt
% and version 1.3 or later is part of all distributions of
% LaTeX version 2005/12/01 or later.
%
% This work has the LPPL maintenance status "maintained".
%
% This Current Maintainer of this work is Heiko Oberdiek.
%
% The Base Interpreter refers to any `TeX-Format',
% because some files are installed in TDS:tex/generic//.
%
% This work consists of the main source file fibnum.dtx
% and the derived files
%    fibnum.sty, fibnum.pdf, fibnum.ins, fibnum.drv, fibnum.bib,
%    fibnum-test1.tex, fibnum-test-calc.tex.
%
% Distribution:
%    CTAN:macros/latex/contrib/oberdiek/fibnum.dtx
%    CTAN:macros/latex/contrib/oberdiek/fibnum.pdf
%
% Unpacking:
%    (a) If fibnum.ins is present:
%           tex fibnum.ins
%    (b) Without fibnum.ins:
%           tex fibnum.dtx
%    (c) If you insist on using LaTeX
%           latex \let\install=y\input{fibnum.dtx}
%        (quote the arguments according to the demands of your shell)
%
% Documentation:
%    (a) If fibnum.drv is present:
%           latex fibnum.drv
%    (b) Without fibnum.drv:
%           latex fibnum.dtx; ...
%    The class ltxdoc loads the configuration file ltxdoc.cfg
%    if available. Here you can specify further options, e.g.
%    use A4 as paper format:
%       \PassOptionsToClass{a4paper}{article}
%
%    Programm calls to get the documentation (example):
%       pdflatex fibnum.dtx
%       bibtex fibnum.aux
%       makeindex -s gind.ist fibnum.idx
%       pdflatex fibnum.dtx
%       makeindex -s gind.ist fibnum.idx
%       pdflatex fibnum.dtx
%
% Installation:
%    TDS:tex/generic/oberdiek/fibnum.sty
%    TDS:doc/latex/oberdiek/fibnum.pdf
%    TDS:doc/latex/oberdiek/test/fibnum-test1.tex
%    TDS:doc/latex/oberdiek/test/fibnum-test-calc.tex
%    TDS:source/latex/oberdiek/fibnum.dtx
%
%<*ignore>
\begingroup
  \catcode123=1 %
  \catcode125=2 %
  \def\x{LaTeX2e}%
\expandafter\endgroup
\ifcase 0\ifx\install y1\fi\expandafter
         \ifx\csname processbatchFile\endcsname\relax\else1\fi
         \ifx\fmtname\x\else 1\fi\relax
\else\csname fi\endcsname
%</ignore>
%<*install>
\input docstrip.tex
\Msg{************************************************************************}
\Msg{* Installation}
\Msg{* Package: fibnum 2016/05/16 v1.1 Fibonacci numbers (HO)}
\Msg{************************************************************************}

\keepsilent
\askforoverwritefalse

\let\MetaPrefix\relax
\preamble

This is a generated file.

Project: fibnum
Version: 2016/05/16 v1.1

Copyright (C) 2012 by
   Heiko Oberdiek <heiko.oberdiek at googlemail.com>

This work may be distributed and/or modified under the
conditions of the LaTeX Project Public License, either
version 1.3c of this license or (at your option) any later
version. This version of this license is in
   http://www.latex-project.org/lppl/lppl-1-3c.txt
and the latest version of this license is in
   http://www.latex-project.org/lppl.txt
and version 1.3 or later is part of all distributions of
LaTeX version 2005/12/01 or later.

This work has the LPPL maintenance status "maintained".

This Current Maintainer of this work is Heiko Oberdiek.

The Base Interpreter refers to any `TeX-Format',
because some files are installed in TDS:tex/generic//.

This work consists of the main source file fibnum.dtx
and the derived files
   fibnum.sty, fibnum.pdf, fibnum.ins, fibnum.drv, fibnum.bib,
   fibnum-test1.tex, fibnum-test-calc.tex.

\endpreamble
\let\MetaPrefix\DoubleperCent

\generate{%
  \file{fibnum.ins}{\from{fibnum.dtx}{install}}%
  \file{fibnum.drv}{\from{fibnum.dtx}{driver}}%
  \nopreamble
  \nopostamble
  \file{fibnum.bib}{\from{fibnum.dtx}{bib}}%
  \usepreamble\defaultpreamble
  \usepostamble\defaultpostamble
  \usedir{tex/generic/oberdiek}%
  \file{fibnum.sty}{\from{fibnum.dtx}{package}}%
  \usedir{doc/latex/oberdiek/test}%
  \file{fibnum-test1.tex}{\from{fibnum.dtx}{test1}}%
  \file{fibnum-test-calc.tex}{\from{fibnum.dtx}{test-calc}}%
}

\catcode32=13\relax% active space
\let =\space%
\Msg{************************************************************************}
\Msg{*}
\Msg{* To finish the installation you have to move the following}
\Msg{* file into a directory searched by TeX:}
\Msg{*}
\Msg{*     fibnum.sty}
\Msg{*}
\Msg{* To produce the documentation run the file `fibnum.drv'}
\Msg{* through LaTeX.}
\Msg{*}
\Msg{* Happy TeXing!}
\Msg{*}
\Msg{************************************************************************}

\endbatchfile
%</install>
%<*bib>
@online{texhax:abraham,
  author={Abraham, Jan},
  title={[texhax] Beginner in TEX MACRO to compute functions},
  date={2012-04-07},
  url={http://tug.org/pipermail/texhax/2012-April/019146.html},
  urldate={2012-04-08},
}
@article{knuth:negafibonacci,
  author={Knuth, Donald E.},
  title={Negafibonacci Numbers and the Hyperbolic Plane},
  date={2008-12-11},
  url={http://research.allacademic.com/meta/p206842_index.html},
}
@online{wikipedia:negafibonacci,
  author={{Wikipedia contributors}},
  organization={{Wikipedia, The Free Encyclopedia}},
  title={Fibonacci numbers},
  language={langenglish},
  version={486266088},
  date={2012-04-08},
  url={http://en.wikipedia.org/w/index.php?title=Fibonacci_number&oldid=486266088},
  urldate={2012-04-08},
}
%</bib>
%<*ignore>
\fi
%</ignore>
%<*driver>
\NeedsTeXFormat{LaTeX2e}
\ProvidesFile{fibnum.drv}%
  [2016/05/16 v1.1 Fibonacci numbers (HO)]%
\documentclass{ltxdoc}
\usepackage{amsmath,amsfonts}
\usepackage{siunitx}
\usepackage{array}
\usepackage{tabularx}
\usepackage{fibnum}[2016/05/16]
\usepackage{holtxdoc}[2011/11/22]
\usepackage{csquotes}
\usepackage[
  bibencoding=ascii,
  alldates=iso8601,
]{biblatex}[2011/11/13]
\bibliography{oberdiek-source}
\bibliography{fibnum}
\begin{document}
  \DocInput{fibnum.dtx}%
\end{document}
%</driver>
% \fi
%
%
% \CharacterTable
%  {Upper-case    \A\B\C\D\E\F\G\H\I\J\K\L\M\N\O\P\Q\R\S\T\U\V\W\X\Y\Z
%   Lower-case    \a\b\c\d\e\f\g\h\i\j\k\l\m\n\o\p\q\r\s\t\u\v\w\x\y\z
%   Digits        \0\1\2\3\4\5\6\7\8\9
%   Exclamation   \!     Double quote  \"     Hash (number) \#
%   Dollar        \$     Percent       \%     Ampersand     \&
%   Acute accent  \'     Left paren    \(     Right paren   \)
%   Asterisk      \*     Plus          \+     Comma         \,
%   Minus         \-     Point         \.     Solidus       \/
%   Colon         \:     Semicolon     \;     Less than     \<
%   Equals        \=     Greater than  \>     Question mark \?
%   Commercial at \@     Left bracket  \[     Backslash     \\
%   Right bracket \]     Circumflex    \^     Underscore    \_
%   Grave accent  \`     Left brace    \{     Vertical bar  \|
%   Right brace   \}     Tilde         \~}
%
% \GetFileInfo{fibnum.drv}
%
% \title{The \xpackage{fibnum} package}
% \date{2016/05/16 v1.1}
% \author{Heiko Oberdiek\thanks
% {Please report any issues at https://github.com/ho-tex/oberdiek/issues}\\
% \xemail{heiko.oberdiek at googlemail.com}}
%
% \maketitle
%
% \begin{abstract}
% The package \xpackage{fibnum} provides expandable fibonacci
% numbers for both \hologo{LaTeX} and \hologo{plainTeX}.
% \end{abstract}
%
% \tableofcontents
%
% \section{Documentation}
%
% In the mailing list \textsf{texhax} Jan Abraham asked the question,
% how to get Fibonacci numbers in \hologo{TeX} \cite{texhax:abraham}:
% \begin{quote}
% Write a Macro in \hologo{TeX} that compute the function |\fib{m}|
% All fibonacci numbers from 1 to $m$ ($m < 40$).
% \end{quote}
% This packages provides an expandable implementation for the
% calculation of these numbers for a much larger set of indexes.
% For practical reasons the index is restricted to the same limitations
% that apply for \hologo{TeX} integer numbers.
% The range of the Fibonacci numbers, however, are not limited
% by the algorithm. They are only restricted to memory limitations,
% if they are hit.
%
% The package is loaded as \hologo{LaTeX} package in \hologo{LaTeX}:
% \begin{quote}
%   |\usepackage{fibnum}|
% \end{quote}
% and as file in \hologo{plainTeX}:
% \begin{quote}
%   |\input fibnum.sty|
% \end{quote}
% The package does not know any options and it provides
% the macros \cs{fibnum} and \cs{fibnumPreCalc}.
%
% \begin{declcs}{fibnum} \M{index}
% \end{declcs}
% Macro \cs{fibnum} expects a \hologo{TeX} number as \meta{index}
% in the official \hologo{TeX} number range from $-(2^{31}-1)$ up to
% $2^{31}-1$. In exact two expansion steps the macro expands to
% the Fibnoacci number $F_{\text{\meta{index}}}$. In case of a negative
% \meta{index}, the ``negafibonacci'' number \cite{wikipedia:negafibonacci}
% is used. Formally the Fibonacci number $F_n$ with integer
% index~$n$, $n\in\mathbb{Z}$ and
% $n\in[\num{-2147483647},\num{2147483647}]$ that is returned by macro
% \cs{fibnum} with numerical argument $n$ is defined the following way:
% \begin{gather}
%   \label{eq:def}
%   F_n =
%   \begin{cases}
%     0 & \text{for $n=0$}\\
%     1 & \text{for $n=1$}\\
%     F_{n-1} + F_{n-2} & \text{for $n>1$}\\
%     (-1)^{n+1}F_n & \text{for $n<0$}
%   \end{cases}
% \end{gather}
% Examples:
% \begin{quote}
%   \makeatletter
%   \def\x#1{\cs{fibnum}\{#1\}&
%     \edef\X{\fibnum{#1}}\edef\Y{\expandafter\ltx@car\X\@nil}^^A
%     \if-\Y
%       \edef\X{\expandafter\ltx@cdr\X\@nil}^^A
%       \noindent
%       \llap{-}\X
%     \else
%       \X
%     \fi
%     \tabularnewline
%   }
%   \def\y{\multicolumn{1}{@{}c@{}}{$\vdots$}\tabularnewline}
%   \DeclareUrlCommand\UrlNum{^^A
%     \urlstyle{tt}^^A
%     \def\UrlBreaks{\do\0\do\1\do\2\do\3\do\4\do\5\do\6\do\7\do\8\do\9}^^A
%   }
%   \begin{tabularx}{\dimexpr\linewidth+5.7pt\relax}{@{}>{\ttfamily}l@{ $\rightarrow$ \hphantom{\ttfamily-}}>{\ttfamily}X@{}}
%     \x{-6}
%     \x{-5}
%     \x{-4}
%     \x{-3}
%     \x{-2}
%     \x{-1}
%     \x{0}
%     \x{1}
%     \x{2}
%     \x{3}
%     \x{4}
%     \x{5}
%     \x{6}
%     \y
%     \x{10}
%     \y
%     \x{46}
%     \y
%     \cs{fibnum}\{100\} & 354224848179261915075
%     \tabularnewline
%     \y
%     \cs{fibnum}\{200\} & 280571172992510140037611932413038677189525
%     \tabularnewline
%     \y
%     \cs{fibnum}\{1000\} &
%       \raggedright
%       \UrlNum{^^A
%         434665576869374564356885276750406258025646^^A
%         605173717804024817290895365554179490518904^^A
%         038798400792551692959225930803226347752096^^A
%         896232398733224711616429964409065331879382^^A
%         98969649928516003704476137795166849228875^^A
%       }
%     \tabularnewline
%   \end{tabularx}\kern-5.7pt\mbox{}
% \end{quote}
%
% \begin{declcs}{fibnumPreCalc} \M{index}
% \end{declcs}
% The package already provides precalculated Fibonacci numbers up to
% index~46. That means that calculations are not necessary for
% Fibonacci numbers that fit into the range of \hologo{TeX}
% numbers. Because macro \cs{fibnum} is expandable, it cannot
% store calculated Fibonacci numbers for later use. Macro definitions
% are forbidden in expandable contexts. If larger Fibonacci numbers
% are used more than once, than the compilation time can be shortened
% by calculating and storing the Fibonacci numbers beforehand.
% The argument \meta{index} is a \hologo{TeX} number and macro
% \cs{fibnumPreCalc} ensures that the Fibonacci numbers
% $F_0$ up to $F_{\lvert\text{\meta{index}}\rvert}$ that are not
% already known are calculated
% and stored in internal macros. Internally only non-negative
% Fibonacci numbers are stored. If \meta{index} is negative, then
% the needed positive Fibonacci numbers are calculated and stored.
% Example:
% \begin{quote}
%   \def\x#1{\begingroup\itshape\texttt{\%} #1\endgroup}
%   |\fibnumPreCalc{50}|\\
%   \x{calculates and stores the values for indexes 47..50.}\\
%   \x{(Values for 0..46 are already stored by the package.)}\\
%   |\fibnum{49}| \x{uses the stored value}\\
%   |\fibnum{51}|
%   \x{only calculates $F_{51}$ from stored values $F_{49}$ and $F_{50}$}\\
%   |\fibnumPreCalc{100}|\\
%   \x{calculates and stores the values for indexes 51..100}\\
%   |\fibnum{100}| \x{uses the stored value for $F_{100}$}\\
%   |\fibnum{-100}|
%   \x{uses the stored value for $F_{100}$}\\
%   \x{$F_{-100}=-F_{100}$ according to equation \eqref{eq:def}.}
% \end{quote}
%
% \StopEventually{
% }
%
% \section{Implementation}
%
% \subsection{Identification}
%
%    \begin{macrocode}
%<*package>
%    \end{macrocode}
%    Reload check, especially if the package is not used with \LaTeX.
%    \begin{macrocode}
\begingroup\catcode61\catcode48\catcode32=10\relax%
  \catcode13=5 % ^^M
  \endlinechar=13 %
  \catcode35=6 % #
  \catcode39=12 % '
  \catcode44=12 % ,
  \catcode45=12 % -
  \catcode46=12 % .
  \catcode58=12 % :
  \catcode64=11 % @
  \catcode123=1 % {
  \catcode125=2 % }
  \expandafter\let\expandafter\x\csname ver@fibnum.sty\endcsname
  \ifx\x\relax % plain-TeX, first loading
  \else
    \def\empty{}%
    \ifx\x\empty % LaTeX, first loading,
      % variable is initialized, but \ProvidesPackage not yet seen
    \else
      \expandafter\ifx\csname PackageInfo\endcsname\relax
        \def\x#1#2{%
          \immediate\write-1{Package #1 Info: #2.}%
        }%
      \else
        \def\x#1#2{\PackageInfo{#1}{#2, stopped}}%
      \fi
      \x{fibnum}{The package is already loaded}%
      \aftergroup\endinput
    \fi
  \fi
\endgroup%
%    \end{macrocode}
%    Package identification:
%    \begin{macrocode}
\begingroup\catcode61\catcode48\catcode32=10\relax%
  \catcode13=5 % ^^M
  \endlinechar=13 %
  \catcode35=6 % #
  \catcode39=12 % '
  \catcode40=12 % (
  \catcode41=12 % )
  \catcode44=12 % ,
  \catcode45=12 % -
  \catcode46=12 % .
  \catcode47=12 % /
  \catcode58=12 % :
  \catcode64=11 % @
  \catcode91=12 % [
  \catcode93=12 % ]
  \catcode123=1 % {
  \catcode125=2 % }
  \expandafter\ifx\csname ProvidesPackage\endcsname\relax
    \def\x#1#2#3[#4]{\endgroup
      \immediate\write-1{Package: #3 #4}%
      \xdef#1{#4}%
    }%
  \else
    \def\x#1#2[#3]{\endgroup
      #2[{#3}]%
      \ifx#1\@undefined
        \xdef#1{#3}%
      \fi
      \ifx#1\relax
        \xdef#1{#3}%
      \fi
    }%
  \fi
\expandafter\x\csname ver@fibnum.sty\endcsname
\ProvidesPackage{fibnum}%
  [2016/05/16 v1.1 Fibonacci numbers (HO)]%
%    \end{macrocode}
%
%    \begin{macrocode}
\begingroup\catcode61\catcode48\catcode32=10\relax%
  \catcode13=5 % ^^M
  \endlinechar=13 %
  \catcode123=1 % {
  \catcode125=2 % }
  \catcode64=11 % @
  \def\x{\endgroup
    \expandafter\edef\csname FibNum@AtEnd\endcsname{%
      \endlinechar=\the\endlinechar\relax
      \catcode13=\the\catcode13\relax
      \catcode32=\the\catcode32\relax
      \catcode35=\the\catcode35\relax
      \catcode61=\the\catcode61\relax
      \catcode64=\the\catcode64\relax
      \catcode123=\the\catcode123\relax
      \catcode125=\the\catcode125\relax
    }%
  }%
\x\catcode61\catcode48\catcode32=10\relax%
\catcode13=5 % ^^M
\endlinechar=13 %
\catcode35=6 % #
\catcode64=11 % @
\catcode123=1 % {
\catcode125=2 % }
\def\TMP@EnsureCode#1#2{%
  \edef\FibNum@AtEnd{%
    \FibNum@AtEnd
    \catcode#1=\the\catcode#1\relax
  }%
  \catcode#1=#2\relax
}
\TMP@EnsureCode{33}{12}% !
%\TMP@EnsureCode{36}{3}% $
%\TMP@EnsureCode{38}{4}% &
\TMP@EnsureCode{40}{12}% (
\TMP@EnsureCode{41}{12}% )
\TMP@EnsureCode{45}{12}% -
\TMP@EnsureCode{46}{12}% .
\TMP@EnsureCode{47}{12}% /
\TMP@EnsureCode{58}{12}% :
\TMP@EnsureCode{60}{12}% <
\TMP@EnsureCode{62}{12}% >
\TMP@EnsureCode{91}{12}% [
%\TMP@EnsureCode{96}{12}% `
\TMP@EnsureCode{93}{12}% ]
%\TMP@EnsureCode{94}{12}% ^ (superscript) (!)
%\TMP@EnsureCode{124}{12}% |
\edef\FibNum@AtEnd{\FibNum@AtEnd\noexpand\endinput}
%    \end{macrocode}
%
% \subsection{Package resources}
%
%    \begin{macrocode}
\begingroup\expandafter\expandafter\expandafter\endgroup
\expandafter\ifx\csname RequirePackage\endcsname\relax
  \def\TMP@RequirePackage#1[#2]{%
    \begingroup\expandafter\expandafter\expandafter\endgroup
    \expandafter\ifx\csname ver@#1.sty\endcsname\relax
      \input #1.sty\relax
    \fi
  }%
  \TMP@RequirePackage{ltxcmds}[2011/04/18]%
  \TMP@RequirePackage{intcalc}[2007/09/27]%
  \TMP@RequirePackage{bigintcalc}[2007/11/11]%
\else
  \RequirePackage{ltxcmds}[2011/04/18]%
  \RequirePackage{intcalc}[2007/09/27]%
  \RequirePackage{bigintcalc}[2007/11/11]%
\fi
%    \end{macrocode}
%
% \subsection{Setup precalculated values}
%
%    \begin{macrocode}
\def\FibNum@temp#1{%
  \expandafter\def\csname FibNum@#1\endcsname
}
\catcode46=9 % dots are ignored
\FibNum@temp{0}{0}
\FibNum@temp{1}{1}
\FibNum@temp{2}{1}
\FibNum@temp{3}{2}
\FibNum@temp{4}{3}
\FibNum@temp{5}{5}
\FibNum@temp{6}{8}
\FibNum@temp{7}{13}
\FibNum@temp{8}{21}
\FibNum@temp{9}{34}
\FibNum@temp{10}{55}
\FibNum@temp{11}{89}
\FibNum@temp{12}{144}
\FibNum@temp{13}{233}
\FibNum@temp{14}{377}
\FibNum@temp{15}{610}
\FibNum@temp{16}{987}
\FibNum@temp{17}{1.597}
\FibNum@temp{18}{2.584}
\FibNum@temp{19}{4.181}
\FibNum@temp{20}{6.765}
\FibNum@temp{21}{10.946}
\FibNum@temp{22}{17.711}
\FibNum@temp{23}{28.657}
\FibNum@temp{24}{46.368}
\FibNum@temp{25}{75.025}
\FibNum@temp{26}{121.393}
\FibNum@temp{27}{196.418}
\FibNum@temp{28}{317.811}
\FibNum@temp{29}{514.229}
\FibNum@temp{30}{832.040}
\FibNum@temp{31}{1.346.269}
\FibNum@temp{32}{2.178.309}
\FibNum@temp{33}{3.524.578}
\FibNum@temp{34}{5.702.887}
\FibNum@temp{35}{9.227.465}
\FibNum@temp{36}{14.930.352}
\FibNum@temp{37}{24.157.817}
\FibNum@temp{38}{39.088.169}
\FibNum@temp{39}{63.245.986}
\FibNum@temp{40}{102.334.155}
\FibNum@temp{41}{165.580.141}
\FibNum@temp{42}{267.914.296}
\FibNum@temp{43}{433.494.437}
\FibNum@temp{44}{701.408.733}
\FibNum@temp{45}{1.134.903.170}
\FibNum@temp{46}{1.836.311.903}
%    \end{macrocode}
%    \begin{macro}{\FibNum@max}
%    \begin{macrocode}
\def\FibNum@max{46}
%    \end{macrocode}
%    \end{macro}
%
% \subsection{Macros for precalculating values}
%
%    \begin{macro}{\fibnumPreCalc}
%    \begin{macrocode}
\def\fibnumPreCalc#1{%
  \expandafter\expandafter\expandafter
  \FibNum@PreCalc\intcalcNum{#1}/%
}
%    \end{macrocode}
%    \end{macro}
%    \begin{macro}{\FibNum@PreCalc}
%    \begin{macrocode}
\def\FibNum@PreCalc#1/{%
  \ifnum#1<\ltx@zero
    \expandafter\FibNum@PreCalc\ltx@gobble#1/%
  \else
    \ifnum#1>\FibNum@max
      \begingroup
        \ltx@LocDimenA=#1sp\relax
        \countdef\FibNum@i=255\relax
        \FibNum@i=\FibNum@max\relax
        \edef\FibNum@temp{%
          \csname FibNum@\the\FibNum@i\endcsname/%
        }%
        \advance\FibNum@i by -1\relax
        \edef\FibNum@temp{%
          \FibNum@temp
          \csname FibNum@\the\FibNum@i\endcsname
        }%
        \advance\FibNum@i\ltx@two
        \iftrue
          \expandafter\FibNum@PreCalcAux\FibNum@temp
        \fi
      \endgroup
    \fi
  \fi
}
%    \end{macrocode}
%    \end{macro}
%    \begin{macro}{\FibNum@PreCalcAux}
%    \begin{macrocode}
\def\FibNum@PreCalcAux#1/#2\fi{%
  \fi
  \edef\FibNum@temp{\BigIntCalcAdd#1!#2!}%
  \global\expandafter
  \let\csname FibNum@\the\FibNum@i\endcsname\FibNum@temp
  \ifnum\FibNum@i=\ltx@LocDimenA
    \xdef\FibNum@max{\the\FibNum@i}%
  \else
    \advance\FibNum@i\ltx@one
    \expandafter\FibNum@PreCalcAux\FibNum@temp/#1%
  \fi
}
%    \end{macrocode}
%    \end{macro}
%
% \subsection{Expandable calculations}
%
%    \begin{macro}{\fibnum}
%    \begin{macrocode}
\def\fibnum#1{%
  \romannumeral
  \expandafter\expandafter\expandafter\FibNum@Do\intcalcNum{#1}/%
}
%    \end{macrocode}
%    \end{macro}
%    \begin{macro}{\FibNum@Do}
%    \begin{macrocode}
\def\FibNum@Do#1/{%
  \ifnum#1<\ltx@zero
    \FibNum@ReturnAfterElseFiFi{%
      \ifodd#1 %
        \expandafter\expandafter\expandafter\ltx@zero
      \else
        \expandafter\expandafter\expandafter\ltx@zero
        \expandafter\expandafter\expandafter-%
      \fi
      \romannumeral
      \expandafter\FibNum@Do\ltx@gobble#1/%
    }%
  \else
    \ifnum\FibNum@max<#1 %
      \ltx@ReturnAfterElseFi{%
        \expandafter
        \FibNum@ExpCalc\number\expandafter\IntCalcInc\FibNum@max!%
        \expandafter\expandafter\expandafter/%
        \csname FibNum@\FibNum@max
        \expandafter\expandafter\expandafter\endcsname
        \expandafter\expandafter\expandafter/%
        \csname FibNum@\expandafter\IntCalcDec\FibNum@max!%
        \endcsname/%
        #1%
      }%
    \else
      \expandafter\expandafter\expandafter\ltx@zero
      \csname FibNum@#1\expandafter\expandafter\expandafter\endcsname
    \fi
  \fi
}
%    \end{macrocode}
%    \end{macro}
%    \begin{macro}{\FibNum@ReturnAfterElseFiFi}
%    \begin{macrocode}
\def\FibNum@ReturnAfterElseFiFi#1\else#2\fi\fi{\fi#1}
%    \end{macrocode}
%    \end{macro}
%    \begin{macro}{\FibNum@ExpCalc}
%    \begin{macrocode}
\def\FibNum@ExpCalc#1/#2/#3/#4\fi{%
  \fi
  \ifnum#1=#4 %
    \ltx@ReturnAfterElseFi{%
      \expandafter\expandafter\expandafter\ltx@zero
      \BigIntCalcAdd#2!#3!%
    }%
  \else
    \expandafter\FibNum@ExpCalc
    \number\IntCalcInc#1!%
    \expandafter\expandafter\expandafter/%
    \BigIntCalcAdd#2!#3!/%
    #2/#4%
  \fi
}
%    \end{macrocode}
%    \end{macro}
%
%    \begin{macrocode}
\FibNum@AtEnd%
%</package>
%    \end{macrocode}
%
% \section{Test}
%
% \subsection{Catcode checks for loading}
%
%    \begin{macrocode}
%<*test1>
%    \end{macrocode}
%    \begin{macrocode}
\catcode`\{=1 %
\catcode`\}=2 %
\catcode`\#=6 %
\catcode`\@=11 %
\expandafter\ifx\csname count@\endcsname\relax
  \countdef\count@=255 %
\fi
\expandafter\ifx\csname @gobble\endcsname\relax
  \long\def\@gobble#1{}%
\fi
\expandafter\ifx\csname @firstofone\endcsname\relax
  \long\def\@firstofone#1{#1}%
\fi
\expandafter\ifx\csname loop\endcsname\relax
  \expandafter\@firstofone
\else
  \expandafter\@gobble
\fi
{%
  \def\loop#1\repeat{%
    \def\body{#1}%
    \iterate
  }%
  \def\iterate{%
    \body
      \let\next\iterate
    \else
      \let\next\relax
    \fi
    \next
  }%
  \let\repeat=\fi
}%
\def\RestoreCatcodes{}
\count@=0 %
\loop
  \edef\RestoreCatcodes{%
    \RestoreCatcodes
    \catcode\the\count@=\the\catcode\count@\relax
  }%
\ifnum\count@<255 %
  \advance\count@ 1 %
\repeat

\def\RangeCatcodeInvalid#1#2{%
  \count@=#1\relax
  \loop
    \catcode\count@=15 %
  \ifnum\count@<#2\relax
    \advance\count@ 1 %
  \repeat
}
\def\RangeCatcodeCheck#1#2#3{%
  \count@=#1\relax
  \loop
    \ifnum#3=\catcode\count@
    \else
      \errmessage{%
        Character \the\count@\space
        with wrong catcode \the\catcode\count@\space
        instead of \number#3%
      }%
    \fi
  \ifnum\count@<#2\relax
    \advance\count@ 1 %
  \repeat
}
\def\space{ }
\expandafter\ifx\csname LoadCommand\endcsname\relax
  \def\LoadCommand{\input fibnum.sty\relax}%
\fi
\def\Test{%
  \RangeCatcodeInvalid{0}{47}%
  \RangeCatcodeInvalid{58}{64}%
  \RangeCatcodeInvalid{91}{96}%
  \RangeCatcodeInvalid{123}{255}%
  \catcode`\@=12 %
  \catcode`\\=0 %
  \catcode`\%=14 %
  \LoadCommand
  \RangeCatcodeCheck{0}{36}{15}%
  \RangeCatcodeCheck{37}{37}{14}%
  \RangeCatcodeCheck{38}{47}{15}%
  \RangeCatcodeCheck{48}{57}{12}%
  \RangeCatcodeCheck{58}{63}{15}%
  \RangeCatcodeCheck{64}{64}{12}%
  \RangeCatcodeCheck{65}{90}{11}%
  \RangeCatcodeCheck{91}{91}{15}%
  \RangeCatcodeCheck{92}{92}{0}%
  \RangeCatcodeCheck{93}{96}{15}%
  \RangeCatcodeCheck{97}{122}{11}%
  \RangeCatcodeCheck{123}{255}{15}%
  \RestoreCatcodes
}
\Test
\csname @@end\endcsname
\end
%    \end{macrocode}
%    \begin{macrocode}
%</test1>
%    \end{macrocode}
%
% \subsection{Test calculations}
%
%    \begin{macrocode}
%<*test-calc>
\catcode`\{=1 %
\catcode`\}=2 %
\catcode`\#=6 %
\catcode`\@=11 %
\begingroup\expandafter\expandafter\expandafter\endgroup
\expandafter\ifx\csname RequirePackage\endcsname\relax
  \input fibnum.sty\relax
\else
  \RequirePackage{fibnum}[2016/05/16]%
\fi
\def\TestSet{%
  \test{0}{0}%
  \test{1}{1}%
  \test{2}{1}%
  \test{3}{2}%
  \test{4}{3}%
  \test{5}{5}%
  \test{6}{8}%
  \test{7}{13}%
  \test{8}{21}%
  \test{9}{34}%
  \test{10}{55}%
  \test{11}{89}%
  \test{12}{144}%
  \test{13}{233}%
  \test{14}{377}%
  \test{15}{610}%
  \test{16}{987}%
  \test{17}{1597}%
  \test{18}{2584}%
  \test{19}{4181}%
  \test{20}{6765}%
  \test{21}{10946}%
  \test{22}{17711}%
  \test{23}{28657}%
  \test{24}{46368}%
  \test{25}{75025}%
  \test{26}{121393}%
  \test{27}{196418}%
  \test{28}{317811}%
  \test{29}{514229}%
  \test{30}{832040}%
  \test{31}{1346269}%
  \test{32}{2178309}%
  \test{33}{3524578}%
  \test{34}{5702887}%
  \test{35}{9227465}%
  \test{36}{14930352}%
  \test{37}{24157817}%
  \test{38}{39088169}%
  \test{39}{63245986}%
  \test{40}{102334155}%
  \test{41}{165580141}%
  \test{42}{267914296}%
  \test{43}{433494437}%
  \test{44}{701408733}%
  \test{45}{1134903170}%
  \test{46}{1836311903}%
  \test{47}{2971215073}%
  \test{48}{4807526976}%
  \test{49}{7778742049}%
  \test{50}{12586269025}%
  \test{51}{20365011074}%
  \test{52}{32951280099}%
  \test{53}{53316291173}%
  \test{54}{86267571272}%
  \test{55}{139583862445}%
  \test{56}{225851433717}%
  \test{57}{365435296162}%
  \test{58}{591286729879}%
  \test{59}{956722026041}%
  \test{60}{1548008755920}%
  \test{61}{2504730781961}%
  \test{62}{4052739537881}%
  \test{63}{6557470319842}%
  \test{64}{10610209857723}%
  \test{65}{17167680177565}%
  \test{66}{27777890035288}%
  \test{67}{44945570212853}%
  \test{68}{72723460248141}%
  \test{69}{117669030460994}%
  \test{70}{190392490709135}%
  \test{71}{308061521170129}%
  \test{72}{498454011879264}%
  \test{73}{806515533049393}%
}
\def\msg#{\immediate\write16}
\def\test#1#2{%
  \TestAux{#1}{#2}%
  \ifnum#1=0 %
  \else
    \ifodd#1 %
      \TestAux{-#1}{#2}%
    \else
      \TestAux{-#1}{-#2}%
    \fi
  \fi
}
\def\TestAux#1#2{%
  \def\Expected{#2}%
  \expandafter\expandafter\expandafter\def
  \expandafter\expandafter\expandafter\Result
  \expandafter\expandafter\expandafter{%
    \fibnum{#1}%
  }%
  \ltx@onelevel@sanitize\Result
  \ifx\Result\Expected
    \msg{* #1: ok.}%
  \else
    \msg{! fib(#1) = #2}%
    \errmessage{fib(#1) <> \Result}%
  \fi
}
\TestSet
\setbox0=\hbox{%
  \msg{* PreCalc{73}}%
  \fibnumPreCalc{73}%
}
\ifdim\wd0=0pt
\else
  \errmessage{Unwanted stuff in PreCalc}%
\fi
\TestSet
\csname @@end\endcsname\end
%</test-calc>
%    \end{macrocode}
%
% \section{Installation}
%
% \subsection{Download}
%
% \paragraph{Package.} This package is available on
% CTAN\footnote{\url{http://ctan.org/pkg/fibnum}}:
% \begin{description}
% \item[\CTAN{macros/latex/contrib/oberdiek/fibnum.dtx}] The source file.
% \item[\CTAN{macros/latex/contrib/oberdiek/fibnum.pdf}] Documentation.
% \end{description}
%
%
% \paragraph{Bundle.} All the packages of the bundle `oberdiek'
% are also available in a TDS compliant ZIP archive. There
% the packages are already unpacked and the documentation files
% are generated. The files and directories obey the TDS standard.
% \begin{description}
% \item[\CTAN{install/macros/latex/contrib/oberdiek.tds.zip}]
% \end{description}
% \emph{TDS} refers to the standard ``A Directory Structure
% for \TeX\ Files'' (\CTAN{tds/tds.pdf}). Directories
% with \xfile{texmf} in their name are usually organized this way.
%
% \subsection{Bundle installation}
%
% \paragraph{Unpacking.} Unpack the \xfile{oberdiek.tds.zip} in the
% TDS tree (also known as \xfile{texmf} tree) of your choice.
% Example (linux):
% \begin{quote}
%   |unzip oberdiek.tds.zip -d ~/texmf|
% \end{quote}
%
% \paragraph{Script installation.}
% Check the directory \xfile{TDS:scripts/oberdiek/} for
% scripts that need further installation steps.
% Package \xpackage{attachfile2} comes with the Perl script
% \xfile{pdfatfi.pl} that should be installed in such a way
% that it can be called as \texttt{pdfatfi}.
% Example (linux):
% \begin{quote}
%   |chmod +x scripts/oberdiek/pdfatfi.pl|\\
%   |cp scripts/oberdiek/pdfatfi.pl /usr/local/bin/|
% \end{quote}
%
% \subsection{Package installation}
%
% \paragraph{Unpacking.} The \xfile{.dtx} file is a self-extracting
% \docstrip\ archive. The files are extracted by running the
% \xfile{.dtx} through \plainTeX:
% \begin{quote}
%   \verb|tex fibnum.dtx|
% \end{quote}
%
% \paragraph{TDS.} Now the different files must be moved into
% the different directories in your installation TDS tree
% (also known as \xfile{texmf} tree):
% \begin{quote}
% \def\t{^^A
% \begin{tabular}{@{}>{\ttfamily}l@{ $\rightarrow$ }>{\ttfamily}l@{}}
%   fibnum.sty & tex/generic/oberdiek/fibnum.sty\\
%   fibnum.pdf & doc/latex/oberdiek/fibnum.pdf\\
%   test/fibnum-test1.tex & doc/latex/oberdiek/test/fibnum-test1.tex\\
%   test/fibnum-test-calc.tex & doc/latex/oberdiek/test/fibnum-test-calc.tex\\
%   fibnum.dtx & source/latex/oberdiek/fibnum.dtx\\
% \end{tabular}^^A
% }^^A
% \sbox0{\t}^^A
% \ifdim\wd0>\linewidth
%   \begingroup
%     \advance\linewidth by\leftmargin
%     \advance\linewidth by\rightmargin
%   \edef\x{\endgroup
%     \def\noexpand\lw{\the\linewidth}^^A
%   }\x
%   \def\lwbox{^^A
%     \leavevmode
%     \hbox to \linewidth{^^A
%       \kern-\leftmargin\relax
%       \hss
%       \usebox0
%       \hss
%       \kern-\rightmargin\relax
%     }^^A
%   }^^A
%   \ifdim\wd0>\lw
%     \sbox0{\small\t}^^A
%     \ifdim\wd0>\linewidth
%       \ifdim\wd0>\lw
%         \sbox0{\footnotesize\t}^^A
%         \ifdim\wd0>\linewidth
%           \ifdim\wd0>\lw
%             \sbox0{\scriptsize\t}^^A
%             \ifdim\wd0>\linewidth
%               \ifdim\wd0>\lw
%                 \sbox0{\tiny\t}^^A
%                 \ifdim\wd0>\linewidth
%                   \lwbox
%                 \else
%                   \usebox0
%                 \fi
%               \else
%                 \lwbox
%               \fi
%             \else
%               \usebox0
%             \fi
%           \else
%             \lwbox
%           \fi
%         \else
%           \usebox0
%         \fi
%       \else
%         \lwbox
%       \fi
%     \else
%       \usebox0
%     \fi
%   \else
%     \lwbox
%   \fi
% \else
%   \usebox0
% \fi
% \end{quote}
% If you have a \xfile{docstrip.cfg} that configures and enables \docstrip's
% TDS installing feature, then some files can already be in the right
% place, see the documentation of \docstrip.
%
% \subsection{Refresh file name databases}
%
% If your \TeX~distribution
% (\teTeX, \mikTeX, \dots) relies on file name databases, you must refresh
% these. For example, \teTeX\ users run \verb|texhash| or
% \verb|mktexlsr|.
%
% \subsection{Some details for the interested}
%
% \paragraph{Attached source.}
%
% The PDF documentation on CTAN also includes the
% \xfile{.dtx} source file. It can be extracted by
% AcrobatReader 6 or higher. Another option is \textsf{pdftk},
% e.g. unpack the file into the current directory:
% \begin{quote}
%   \verb|pdftk fibnum.pdf unpack_files output .|
% \end{quote}
%
% \paragraph{Unpacking with \LaTeX.}
% The \xfile{.dtx} chooses its action depending on the format:
% \begin{description}
% \item[\plainTeX:] Run \docstrip\ and extract the files.
% \item[\LaTeX:] Generate the documentation.
% \end{description}
% If you insist on using \LaTeX\ for \docstrip\ (really,
% \docstrip\ does not need \LaTeX), then inform the autodetect routine
% about your intention:
% \begin{quote}
%   \verb|latex \let\install=y\input{fibnum.dtx}|
% \end{quote}
% Do not forget to quote the argument according to the demands
% of your shell.
%
% \paragraph{Generating the documentation.}
% You can use both the \xfile{.dtx} or the \xfile{.drv} to generate
% the documentation. The process can be configured by the
% configuration file \xfile{ltxdoc.cfg}. For instance, put this
% line into this file, if you want to have A4 as paper format:
% \begin{quote}
%   \verb|\PassOptionsToClass{a4paper}{article}|
% \end{quote}
% An example follows how to generate the
% documentation with pdf\LaTeX:
% \begin{quote}
%\begin{verbatim}
%pdflatex fibnum.dtx
%bibtex fibnum.aux
%makeindex -s gind.ist fibnum.idx
%pdflatex fibnum.dtx
%makeindex -s gind.ist fibnum.idx
%pdflatex fibnum.dtx
%\end{verbatim}
% \end{quote}
%
% \printbibliography[
%   heading=bibnumbered,
% ]
%
% \begin{History}
%   \begin{Version}{2012/04/08 v1.0}
%   \item
%     First version.
%   \end{Version}
%   \begin{Version}{2016/05/16 v1.1}
%   \item
%     Documentation updates.
%   \end{Version}
% \end{History}
%
% \PrintIndex
%
% \Finale
\endinput
|
% \end{quote}
% Do not forget to quote the argument according to the demands
% of your shell.
%
% \paragraph{Generating the documentation.}
% You can use both the \xfile{.dtx} or the \xfile{.drv} to generate
% the documentation. The process can be configured by the
% configuration file \xfile{ltxdoc.cfg}. For instance, put this
% line into this file, if you want to have A4 as paper format:
% \begin{quote}
%   \verb|\PassOptionsToClass{a4paper}{article}|
% \end{quote}
% An example follows how to generate the
% documentation with pdf\LaTeX:
% \begin{quote}
%\begin{verbatim}
%pdflatex fibnum.dtx
%bibtex fibnum.aux
%makeindex -s gind.ist fibnum.idx
%pdflatex fibnum.dtx
%makeindex -s gind.ist fibnum.idx
%pdflatex fibnum.dtx
%\end{verbatim}
% \end{quote}
%
% \printbibliography[
%   heading=bibnumbered,
% ]
%
% \begin{History}
%   \begin{Version}{2012/04/08 v1.0}
%   \item
%     First version.
%   \end{Version}
%   \begin{Version}{2016/05/16 v1.1}
%   \item
%     Documentation updates.
%   \end{Version}
% \end{History}
%
% \PrintIndex
%
% \Finale
\endinput

%        (quote the arguments according to the demands of your shell)
%
% Documentation:
%    (a) If fibnum.drv is present:
%           latex fibnum.drv
%    (b) Without fibnum.drv:
%           latex fibnum.dtx; ...
%    The class ltxdoc loads the configuration file ltxdoc.cfg
%    if available. Here you can specify further options, e.g.
%    use A4 as paper format:
%       \PassOptionsToClass{a4paper}{article}
%
%    Programm calls to get the documentation (example):
%       pdflatex fibnum.dtx
%       bibtex fibnum.aux
%       makeindex -s gind.ist fibnum.idx
%       pdflatex fibnum.dtx
%       makeindex -s gind.ist fibnum.idx
%       pdflatex fibnum.dtx
%
% Installation:
%    TDS:tex/generic/oberdiek/fibnum.sty
%    TDS:doc/latex/oberdiek/fibnum.pdf
%    TDS:doc/latex/oberdiek/test/fibnum-test1.tex
%    TDS:doc/latex/oberdiek/test/fibnum-test-calc.tex
%    TDS:source/latex/oberdiek/fibnum.dtx
%
%<*ignore>
\begingroup
  \catcode123=1 %
  \catcode125=2 %
  \def\x{LaTeX2e}%
\expandafter\endgroup
\ifcase 0\ifx\install y1\fi\expandafter
         \ifx\csname processbatchFile\endcsname\relax\else1\fi
         \ifx\fmtname\x\else 1\fi\relax
\else\csname fi\endcsname
%</ignore>
%<*install>
\input docstrip.tex
\Msg{************************************************************************}
\Msg{* Installation}
\Msg{* Package: fibnum 2016/05/16 v1.1 Fibonacci numbers (HO)}
\Msg{************************************************************************}

\keepsilent
\askforoverwritefalse

\let\MetaPrefix\relax
\preamble

This is a generated file.

Project: fibnum
Version: 2016/05/16 v1.1

Copyright (C) 2012 by
   Heiko Oberdiek <heiko.oberdiek at googlemail.com>

This work may be distributed and/or modified under the
conditions of the LaTeX Project Public License, either
version 1.3c of this license or (at your option) any later
version. This version of this license is in
   http://www.latex-project.org/lppl/lppl-1-3c.txt
and the latest version of this license is in
   http://www.latex-project.org/lppl.txt
and version 1.3 or later is part of all distributions of
LaTeX version 2005/12/01 or later.

This work has the LPPL maintenance status "maintained".

This Current Maintainer of this work is Heiko Oberdiek.

The Base Interpreter refers to any `TeX-Format',
because some files are installed in TDS:tex/generic//.

This work consists of the main source file fibnum.dtx
and the derived files
   fibnum.sty, fibnum.pdf, fibnum.ins, fibnum.drv, fibnum.bib,
   fibnum-test1.tex, fibnum-test-calc.tex.

\endpreamble
\let\MetaPrefix\DoubleperCent

\generate{%
  \file{fibnum.ins}{\from{fibnum.dtx}{install}}%
  \file{fibnum.drv}{\from{fibnum.dtx}{driver}}%
  \nopreamble
  \nopostamble
  \file{fibnum.bib}{\from{fibnum.dtx}{bib}}%
  \usepreamble\defaultpreamble
  \usepostamble\defaultpostamble
  \usedir{tex/generic/oberdiek}%
  \file{fibnum.sty}{\from{fibnum.dtx}{package}}%
  \usedir{doc/latex/oberdiek/test}%
  \file{fibnum-test1.tex}{\from{fibnum.dtx}{test1}}%
  \file{fibnum-test-calc.tex}{\from{fibnum.dtx}{test-calc}}%
}

\catcode32=13\relax% active space
\let =\space%
\Msg{************************************************************************}
\Msg{*}
\Msg{* To finish the installation you have to move the following}
\Msg{* file into a directory searched by TeX:}
\Msg{*}
\Msg{*     fibnum.sty}
\Msg{*}
\Msg{* To produce the documentation run the file `fibnum.drv'}
\Msg{* through LaTeX.}
\Msg{*}
\Msg{* Happy TeXing!}
\Msg{*}
\Msg{************************************************************************}

\endbatchfile
%</install>
%<*bib>
@online{texhax:abraham,
  author={Abraham, Jan},
  title={[texhax] Beginner in TEX MACRO to compute functions},
  date={2012-04-07},
  url={http://tug.org/pipermail/texhax/2012-April/019146.html},
  urldate={2012-04-08},
}
@article{knuth:negafibonacci,
  author={Knuth, Donald E.},
  title={Negafibonacci Numbers and the Hyperbolic Plane},
  date={2008-12-11},
  url={http://research.allacademic.com/meta/p206842_index.html},
}
@online{wikipedia:negafibonacci,
  author={{Wikipedia contributors}},
  organization={{Wikipedia, The Free Encyclopedia}},
  title={Fibonacci numbers},
  language={langenglish},
  version={486266088},
  date={2012-04-08},
  url={http://en.wikipedia.org/w/index.php?title=Fibonacci_number&oldid=486266088},
  urldate={2012-04-08},
}
%</bib>
%<*ignore>
\fi
%</ignore>
%<*driver>
\NeedsTeXFormat{LaTeX2e}
\ProvidesFile{fibnum.drv}%
  [2016/05/16 v1.1 Fibonacci numbers (HO)]%
\documentclass{ltxdoc}
\usepackage{amsmath,amsfonts}
\usepackage{siunitx}
\usepackage{array}
\usepackage{tabularx}
\usepackage{fibnum}[2016/05/16]
\usepackage{holtxdoc}[2011/11/22]
\usepackage{csquotes}
\usepackage[
  bibencoding=ascii,
  alldates=iso8601,
]{biblatex}[2011/11/13]
\bibliography{oberdiek-source}
\bibliography{fibnum}
\begin{document}
  \DocInput{fibnum.dtx}%
\end{document}
%</driver>
% \fi
%
%
% \CharacterTable
%  {Upper-case    \A\B\C\D\E\F\G\H\I\J\K\L\M\N\O\P\Q\R\S\T\U\V\W\X\Y\Z
%   Lower-case    \a\b\c\d\e\f\g\h\i\j\k\l\m\n\o\p\q\r\s\t\u\v\w\x\y\z
%   Digits        \0\1\2\3\4\5\6\7\8\9
%   Exclamation   \!     Double quote  \"     Hash (number) \#
%   Dollar        \$     Percent       \%     Ampersand     \&
%   Acute accent  \'     Left paren    \(     Right paren   \)
%   Asterisk      \*     Plus          \+     Comma         \,
%   Minus         \-     Point         \.     Solidus       \/
%   Colon         \:     Semicolon     \;     Less than     \<
%   Equals        \=     Greater than  \>     Question mark \?
%   Commercial at \@     Left bracket  \[     Backslash     \\
%   Right bracket \]     Circumflex    \^     Underscore    \_
%   Grave accent  \`     Left brace    \{     Vertical bar  \|
%   Right brace   \}     Tilde         \~}
%
% \GetFileInfo{fibnum.drv}
%
% \title{The \xpackage{fibnum} package}
% \date{2016/05/16 v1.1}
% \author{Heiko Oberdiek\thanks
% {Please report any issues at https://github.com/ho-tex/oberdiek/issues}\\
% \xemail{heiko.oberdiek at googlemail.com}}
%
% \maketitle
%
% \begin{abstract}
% The package \xpackage{fibnum} provides expandable fibonacci
% numbers for both \hologo{LaTeX} and \hologo{plainTeX}.
% \end{abstract}
%
% \tableofcontents
%
% \section{Documentation}
%
% In the mailing list \textsf{texhax} Jan Abraham asked the question,
% how to get Fibonacci numbers in \hologo{TeX} \cite{texhax:abraham}:
% \begin{quote}
% Write a Macro in \hologo{TeX} that compute the function |\fib{m}|
% All fibonacci numbers from 1 to $m$ ($m < 40$).
% \end{quote}
% This packages provides an expandable implementation for the
% calculation of these numbers for a much larger set of indexes.
% For practical reasons the index is restricted to the same limitations
% that apply for \hologo{TeX} integer numbers.
% The range of the Fibonacci numbers, however, are not limited
% by the algorithm. They are only restricted to memory limitations,
% if they are hit.
%
% The package is loaded as \hologo{LaTeX} package in \hologo{LaTeX}:
% \begin{quote}
%   |\usepackage{fibnum}|
% \end{quote}
% and as file in \hologo{plainTeX}:
% \begin{quote}
%   |\input fibnum.sty|
% \end{quote}
% The package does not know any options and it provides
% the macros \cs{fibnum} and \cs{fibnumPreCalc}.
%
% \begin{declcs}{fibnum} \M{index}
% \end{declcs}
% Macro \cs{fibnum} expects a \hologo{TeX} number as \meta{index}
% in the official \hologo{TeX} number range from $-(2^{31}-1)$ up to
% $2^{31}-1$. In exact two expansion steps the macro expands to
% the Fibnoacci number $F_{\text{\meta{index}}}$. In case of a negative
% \meta{index}, the ``negafibonacci'' number \cite{wikipedia:negafibonacci}
% is used. Formally the Fibonacci number $F_n$ with integer
% index~$n$, $n\in\mathbb{Z}$ and
% $n\in[\num{-2147483647},\num{2147483647}]$ that is returned by macro
% \cs{fibnum} with numerical argument $n$ is defined the following way:
% \begin{gather}
%   \label{eq:def}
%   F_n =
%   \begin{cases}
%     0 & \text{for $n=0$}\\
%     1 & \text{for $n=1$}\\
%     F_{n-1} + F_{n-2} & \text{for $n>1$}\\
%     (-1)^{n+1}F_n & \text{for $n<0$}
%   \end{cases}
% \end{gather}
% Examples:
% \begin{quote}
%   \makeatletter
%   \def\x#1{\cs{fibnum}\{#1\}&
%     \edef\X{\fibnum{#1}}\edef\Y{\expandafter\ltx@car\X\@nil}^^A
%     \if-\Y
%       \edef\X{\expandafter\ltx@cdr\X\@nil}^^A
%       \noindent
%       \llap{-}\X
%     \else
%       \X
%     \fi
%     \tabularnewline
%   }
%   \def\y{\multicolumn{1}{@{}c@{}}{$\vdots$}\tabularnewline}
%   \DeclareUrlCommand\UrlNum{^^A
%     \urlstyle{tt}^^A
%     \def\UrlBreaks{\do\0\do\1\do\2\do\3\do\4\do\5\do\6\do\7\do\8\do\9}^^A
%   }
%   \begin{tabularx}{\dimexpr\linewidth+5.7pt\relax}{@{}>{\ttfamily}l@{ $\rightarrow$ \hphantom{\ttfamily-}}>{\ttfamily}X@{}}
%     \x{-6}
%     \x{-5}
%     \x{-4}
%     \x{-3}
%     \x{-2}
%     \x{-1}
%     \x{0}
%     \x{1}
%     \x{2}
%     \x{3}
%     \x{4}
%     \x{5}
%     \x{6}
%     \y
%     \x{10}
%     \y
%     \x{46}
%     \y
%     \cs{fibnum}\{100\} & 354224848179261915075
%     \tabularnewline
%     \y
%     \cs{fibnum}\{200\} & 280571172992510140037611932413038677189525
%     \tabularnewline
%     \y
%     \cs{fibnum}\{1000\} &
%       \raggedright
%       \UrlNum{^^A
%         434665576869374564356885276750406258025646^^A
%         605173717804024817290895365554179490518904^^A
%         038798400792551692959225930803226347752096^^A
%         896232398733224711616429964409065331879382^^A
%         98969649928516003704476137795166849228875^^A
%       }
%     \tabularnewline
%   \end{tabularx}\kern-5.7pt\mbox{}
% \end{quote}
%
% \begin{declcs}{fibnumPreCalc} \M{index}
% \end{declcs}
% The package already provides precalculated Fibonacci numbers up to
% index~46. That means that calculations are not necessary for
% Fibonacci numbers that fit into the range of \hologo{TeX}
% numbers. Because macro \cs{fibnum} is expandable, it cannot
% store calculated Fibonacci numbers for later use. Macro definitions
% are forbidden in expandable contexts. If larger Fibonacci numbers
% are used more than once, than the compilation time can be shortened
% by calculating and storing the Fibonacci numbers beforehand.
% The argument \meta{index} is a \hologo{TeX} number and macro
% \cs{fibnumPreCalc} ensures that the Fibonacci numbers
% $F_0$ up to $F_{\lvert\text{\meta{index}}\rvert}$ that are not
% already known are calculated
% and stored in internal macros. Internally only non-negative
% Fibonacci numbers are stored. If \meta{index} is negative, then
% the needed positive Fibonacci numbers are calculated and stored.
% Example:
% \begin{quote}
%   \def\x#1{\begingroup\itshape\texttt{\%} #1\endgroup}
%   |\fibnumPreCalc{50}|\\
%   \x{calculates and stores the values for indexes 47..50.}\\
%   \x{(Values for 0..46 are already stored by the package.)}\\
%   |\fibnum{49}| \x{uses the stored value}\\
%   |\fibnum{51}|
%   \x{only calculates $F_{51}$ from stored values $F_{49}$ and $F_{50}$}\\
%   |\fibnumPreCalc{100}|\\
%   \x{calculates and stores the values for indexes 51..100}\\
%   |\fibnum{100}| \x{uses the stored value for $F_{100}$}\\
%   |\fibnum{-100}|
%   \x{uses the stored value for $F_{100}$}\\
%   \x{$F_{-100}=-F_{100}$ according to equation \eqref{eq:def}.}
% \end{quote}
%
% \StopEventually{
% }
%
% \section{Implementation}
%
% \subsection{Identification}
%
%    \begin{macrocode}
%<*package>
%    \end{macrocode}
%    Reload check, especially if the package is not used with \LaTeX.
%    \begin{macrocode}
\begingroup\catcode61\catcode48\catcode32=10\relax%
  \catcode13=5 % ^^M
  \endlinechar=13 %
  \catcode35=6 % #
  \catcode39=12 % '
  \catcode44=12 % ,
  \catcode45=12 % -
  \catcode46=12 % .
  \catcode58=12 % :
  \catcode64=11 % @
  \catcode123=1 % {
  \catcode125=2 % }
  \expandafter\let\expandafter\x\csname ver@fibnum.sty\endcsname
  \ifx\x\relax % plain-TeX, first loading
  \else
    \def\empty{}%
    \ifx\x\empty % LaTeX, first loading,
      % variable is initialized, but \ProvidesPackage not yet seen
    \else
      \expandafter\ifx\csname PackageInfo\endcsname\relax
        \def\x#1#2{%
          \immediate\write-1{Package #1 Info: #2.}%
        }%
      \else
        \def\x#1#2{\PackageInfo{#1}{#2, stopped}}%
      \fi
      \x{fibnum}{The package is already loaded}%
      \aftergroup\endinput
    \fi
  \fi
\endgroup%
%    \end{macrocode}
%    Package identification:
%    \begin{macrocode}
\begingroup\catcode61\catcode48\catcode32=10\relax%
  \catcode13=5 % ^^M
  \endlinechar=13 %
  \catcode35=6 % #
  \catcode39=12 % '
  \catcode40=12 % (
  \catcode41=12 % )
  \catcode44=12 % ,
  \catcode45=12 % -
  \catcode46=12 % .
  \catcode47=12 % /
  \catcode58=12 % :
  \catcode64=11 % @
  \catcode91=12 % [
  \catcode93=12 % ]
  \catcode123=1 % {
  \catcode125=2 % }
  \expandafter\ifx\csname ProvidesPackage\endcsname\relax
    \def\x#1#2#3[#4]{\endgroup
      \immediate\write-1{Package: #3 #4}%
      \xdef#1{#4}%
    }%
  \else
    \def\x#1#2[#3]{\endgroup
      #2[{#3}]%
      \ifx#1\@undefined
        \xdef#1{#3}%
      \fi
      \ifx#1\relax
        \xdef#1{#3}%
      \fi
    }%
  \fi
\expandafter\x\csname ver@fibnum.sty\endcsname
\ProvidesPackage{fibnum}%
  [2016/05/16 v1.1 Fibonacci numbers (HO)]%
%    \end{macrocode}
%
%    \begin{macrocode}
\begingroup\catcode61\catcode48\catcode32=10\relax%
  \catcode13=5 % ^^M
  \endlinechar=13 %
  \catcode123=1 % {
  \catcode125=2 % }
  \catcode64=11 % @
  \def\x{\endgroup
    \expandafter\edef\csname FibNum@AtEnd\endcsname{%
      \endlinechar=\the\endlinechar\relax
      \catcode13=\the\catcode13\relax
      \catcode32=\the\catcode32\relax
      \catcode35=\the\catcode35\relax
      \catcode61=\the\catcode61\relax
      \catcode64=\the\catcode64\relax
      \catcode123=\the\catcode123\relax
      \catcode125=\the\catcode125\relax
    }%
  }%
\x\catcode61\catcode48\catcode32=10\relax%
\catcode13=5 % ^^M
\endlinechar=13 %
\catcode35=6 % #
\catcode64=11 % @
\catcode123=1 % {
\catcode125=2 % }
\def\TMP@EnsureCode#1#2{%
  \edef\FibNum@AtEnd{%
    \FibNum@AtEnd
    \catcode#1=\the\catcode#1\relax
  }%
  \catcode#1=#2\relax
}
\TMP@EnsureCode{33}{12}% !
%\TMP@EnsureCode{36}{3}% $
%\TMP@EnsureCode{38}{4}% &
\TMP@EnsureCode{40}{12}% (
\TMP@EnsureCode{41}{12}% )
\TMP@EnsureCode{45}{12}% -
\TMP@EnsureCode{46}{12}% .
\TMP@EnsureCode{47}{12}% /
\TMP@EnsureCode{58}{12}% :
\TMP@EnsureCode{60}{12}% <
\TMP@EnsureCode{62}{12}% >
\TMP@EnsureCode{91}{12}% [
%\TMP@EnsureCode{96}{12}% `
\TMP@EnsureCode{93}{12}% ]
%\TMP@EnsureCode{94}{12}% ^ (superscript) (!)
%\TMP@EnsureCode{124}{12}% |
\edef\FibNum@AtEnd{\FibNum@AtEnd\noexpand\endinput}
%    \end{macrocode}
%
% \subsection{Package resources}
%
%    \begin{macrocode}
\begingroup\expandafter\expandafter\expandafter\endgroup
\expandafter\ifx\csname RequirePackage\endcsname\relax
  \def\TMP@RequirePackage#1[#2]{%
    \begingroup\expandafter\expandafter\expandafter\endgroup
    \expandafter\ifx\csname ver@#1.sty\endcsname\relax
      \input #1.sty\relax
    \fi
  }%
  \TMP@RequirePackage{ltxcmds}[2011/04/18]%
  \TMP@RequirePackage{intcalc}[2007/09/27]%
  \TMP@RequirePackage{bigintcalc}[2007/11/11]%
\else
  \RequirePackage{ltxcmds}[2011/04/18]%
  \RequirePackage{intcalc}[2007/09/27]%
  \RequirePackage{bigintcalc}[2007/11/11]%
\fi
%    \end{macrocode}
%
% \subsection{Setup precalculated values}
%
%    \begin{macrocode}
\def\FibNum@temp#1{%
  \expandafter\def\csname FibNum@#1\endcsname
}
\catcode46=9 % dots are ignored
\FibNum@temp{0}{0}
\FibNum@temp{1}{1}
\FibNum@temp{2}{1}
\FibNum@temp{3}{2}
\FibNum@temp{4}{3}
\FibNum@temp{5}{5}
\FibNum@temp{6}{8}
\FibNum@temp{7}{13}
\FibNum@temp{8}{21}
\FibNum@temp{9}{34}
\FibNum@temp{10}{55}
\FibNum@temp{11}{89}
\FibNum@temp{12}{144}
\FibNum@temp{13}{233}
\FibNum@temp{14}{377}
\FibNum@temp{15}{610}
\FibNum@temp{16}{987}
\FibNum@temp{17}{1.597}
\FibNum@temp{18}{2.584}
\FibNum@temp{19}{4.181}
\FibNum@temp{20}{6.765}
\FibNum@temp{21}{10.946}
\FibNum@temp{22}{17.711}
\FibNum@temp{23}{28.657}
\FibNum@temp{24}{46.368}
\FibNum@temp{25}{75.025}
\FibNum@temp{26}{121.393}
\FibNum@temp{27}{196.418}
\FibNum@temp{28}{317.811}
\FibNum@temp{29}{514.229}
\FibNum@temp{30}{832.040}
\FibNum@temp{31}{1.346.269}
\FibNum@temp{32}{2.178.309}
\FibNum@temp{33}{3.524.578}
\FibNum@temp{34}{5.702.887}
\FibNum@temp{35}{9.227.465}
\FibNum@temp{36}{14.930.352}
\FibNum@temp{37}{24.157.817}
\FibNum@temp{38}{39.088.169}
\FibNum@temp{39}{63.245.986}
\FibNum@temp{40}{102.334.155}
\FibNum@temp{41}{165.580.141}
\FibNum@temp{42}{267.914.296}
\FibNum@temp{43}{433.494.437}
\FibNum@temp{44}{701.408.733}
\FibNum@temp{45}{1.134.903.170}
\FibNum@temp{46}{1.836.311.903}
%    \end{macrocode}
%    \begin{macro}{\FibNum@max}
%    \begin{macrocode}
\def\FibNum@max{46}
%    \end{macrocode}
%    \end{macro}
%
% \subsection{Macros for precalculating values}
%
%    \begin{macro}{\fibnumPreCalc}
%    \begin{macrocode}
\def\fibnumPreCalc#1{%
  \expandafter\expandafter\expandafter
  \FibNum@PreCalc\intcalcNum{#1}/%
}
%    \end{macrocode}
%    \end{macro}
%    \begin{macro}{\FibNum@PreCalc}
%    \begin{macrocode}
\def\FibNum@PreCalc#1/{%
  \ifnum#1<\ltx@zero
    \expandafter\FibNum@PreCalc\ltx@gobble#1/%
  \else
    \ifnum#1>\FibNum@max
      \begingroup
        \ltx@LocDimenA=#1sp\relax
        \countdef\FibNum@i=255\relax
        \FibNum@i=\FibNum@max\relax
        \edef\FibNum@temp{%
          \csname FibNum@\the\FibNum@i\endcsname/%
        }%
        \advance\FibNum@i by -1\relax
        \edef\FibNum@temp{%
          \FibNum@temp
          \csname FibNum@\the\FibNum@i\endcsname
        }%
        \advance\FibNum@i\ltx@two
        \iftrue
          \expandafter\FibNum@PreCalcAux\FibNum@temp
        \fi
      \endgroup
    \fi
  \fi
}
%    \end{macrocode}
%    \end{macro}
%    \begin{macro}{\FibNum@PreCalcAux}
%    \begin{macrocode}
\def\FibNum@PreCalcAux#1/#2\fi{%
  \fi
  \edef\FibNum@temp{\BigIntCalcAdd#1!#2!}%
  \global\expandafter
  \let\csname FibNum@\the\FibNum@i\endcsname\FibNum@temp
  \ifnum\FibNum@i=\ltx@LocDimenA
    \xdef\FibNum@max{\the\FibNum@i}%
  \else
    \advance\FibNum@i\ltx@one
    \expandafter\FibNum@PreCalcAux\FibNum@temp/#1%
  \fi
}
%    \end{macrocode}
%    \end{macro}
%
% \subsection{Expandable calculations}
%
%    \begin{macro}{\fibnum}
%    \begin{macrocode}
\def\fibnum#1{%
  \romannumeral
  \expandafter\expandafter\expandafter\FibNum@Do\intcalcNum{#1}/%
}
%    \end{macrocode}
%    \end{macro}
%    \begin{macro}{\FibNum@Do}
%    \begin{macrocode}
\def\FibNum@Do#1/{%
  \ifnum#1<\ltx@zero
    \FibNum@ReturnAfterElseFiFi{%
      \ifodd#1 %
        \expandafter\expandafter\expandafter\ltx@zero
      \else
        \expandafter\expandafter\expandafter\ltx@zero
        \expandafter\expandafter\expandafter-%
      \fi
      \romannumeral
      \expandafter\FibNum@Do\ltx@gobble#1/%
    }%
  \else
    \ifnum\FibNum@max<#1 %
      \ltx@ReturnAfterElseFi{%
        \expandafter
        \FibNum@ExpCalc\number\expandafter\IntCalcInc\FibNum@max!%
        \expandafter\expandafter\expandafter/%
        \csname FibNum@\FibNum@max
        \expandafter\expandafter\expandafter\endcsname
        \expandafter\expandafter\expandafter/%
        \csname FibNum@\expandafter\IntCalcDec\FibNum@max!%
        \endcsname/%
        #1%
      }%
    \else
      \expandafter\expandafter\expandafter\ltx@zero
      \csname FibNum@#1\expandafter\expandafter\expandafter\endcsname
    \fi
  \fi
}
%    \end{macrocode}
%    \end{macro}
%    \begin{macro}{\FibNum@ReturnAfterElseFiFi}
%    \begin{macrocode}
\def\FibNum@ReturnAfterElseFiFi#1\else#2\fi\fi{\fi#1}
%    \end{macrocode}
%    \end{macro}
%    \begin{macro}{\FibNum@ExpCalc}
%    \begin{macrocode}
\def\FibNum@ExpCalc#1/#2/#3/#4\fi{%
  \fi
  \ifnum#1=#4 %
    \ltx@ReturnAfterElseFi{%
      \expandafter\expandafter\expandafter\ltx@zero
      \BigIntCalcAdd#2!#3!%
    }%
  \else
    \expandafter\FibNum@ExpCalc
    \number\IntCalcInc#1!%
    \expandafter\expandafter\expandafter/%
    \BigIntCalcAdd#2!#3!/%
    #2/#4%
  \fi
}
%    \end{macrocode}
%    \end{macro}
%
%    \begin{macrocode}
\FibNum@AtEnd%
%</package>
%    \end{macrocode}
%
% \section{Test}
%
% \subsection{Catcode checks for loading}
%
%    \begin{macrocode}
%<*test1>
%    \end{macrocode}
%    \begin{macrocode}
\catcode`\{=1 %
\catcode`\}=2 %
\catcode`\#=6 %
\catcode`\@=11 %
\expandafter\ifx\csname count@\endcsname\relax
  \countdef\count@=255 %
\fi
\expandafter\ifx\csname @gobble\endcsname\relax
  \long\def\@gobble#1{}%
\fi
\expandafter\ifx\csname @firstofone\endcsname\relax
  \long\def\@firstofone#1{#1}%
\fi
\expandafter\ifx\csname loop\endcsname\relax
  \expandafter\@firstofone
\else
  \expandafter\@gobble
\fi
{%
  \def\loop#1\repeat{%
    \def\body{#1}%
    \iterate
  }%
  \def\iterate{%
    \body
      \let\next\iterate
    \else
      \let\next\relax
    \fi
    \next
  }%
  \let\repeat=\fi
}%
\def\RestoreCatcodes{}
\count@=0 %
\loop
  \edef\RestoreCatcodes{%
    \RestoreCatcodes
    \catcode\the\count@=\the\catcode\count@\relax
  }%
\ifnum\count@<255 %
  \advance\count@ 1 %
\repeat

\def\RangeCatcodeInvalid#1#2{%
  \count@=#1\relax
  \loop
    \catcode\count@=15 %
  \ifnum\count@<#2\relax
    \advance\count@ 1 %
  \repeat
}
\def\RangeCatcodeCheck#1#2#3{%
  \count@=#1\relax
  \loop
    \ifnum#3=\catcode\count@
    \else
      \errmessage{%
        Character \the\count@\space
        with wrong catcode \the\catcode\count@\space
        instead of \number#3%
      }%
    \fi
  \ifnum\count@<#2\relax
    \advance\count@ 1 %
  \repeat
}
\def\space{ }
\expandafter\ifx\csname LoadCommand\endcsname\relax
  \def\LoadCommand{\input fibnum.sty\relax}%
\fi
\def\Test{%
  \RangeCatcodeInvalid{0}{47}%
  \RangeCatcodeInvalid{58}{64}%
  \RangeCatcodeInvalid{91}{96}%
  \RangeCatcodeInvalid{123}{255}%
  \catcode`\@=12 %
  \catcode`\\=0 %
  \catcode`\%=14 %
  \LoadCommand
  \RangeCatcodeCheck{0}{36}{15}%
  \RangeCatcodeCheck{37}{37}{14}%
  \RangeCatcodeCheck{38}{47}{15}%
  \RangeCatcodeCheck{48}{57}{12}%
  \RangeCatcodeCheck{58}{63}{15}%
  \RangeCatcodeCheck{64}{64}{12}%
  \RangeCatcodeCheck{65}{90}{11}%
  \RangeCatcodeCheck{91}{91}{15}%
  \RangeCatcodeCheck{92}{92}{0}%
  \RangeCatcodeCheck{93}{96}{15}%
  \RangeCatcodeCheck{97}{122}{11}%
  \RangeCatcodeCheck{123}{255}{15}%
  \RestoreCatcodes
}
\Test
\csname @@end\endcsname
\end
%    \end{macrocode}
%    \begin{macrocode}
%</test1>
%    \end{macrocode}
%
% \subsection{Test calculations}
%
%    \begin{macrocode}
%<*test-calc>
\catcode`\{=1 %
\catcode`\}=2 %
\catcode`\#=6 %
\catcode`\@=11 %
\begingroup\expandafter\expandafter\expandafter\endgroup
\expandafter\ifx\csname RequirePackage\endcsname\relax
  \input fibnum.sty\relax
\else
  \RequirePackage{fibnum}[2016/05/16]%
\fi
\def\TestSet{%
  \test{0}{0}%
  \test{1}{1}%
  \test{2}{1}%
  \test{3}{2}%
  \test{4}{3}%
  \test{5}{5}%
  \test{6}{8}%
  \test{7}{13}%
  \test{8}{21}%
  \test{9}{34}%
  \test{10}{55}%
  \test{11}{89}%
  \test{12}{144}%
  \test{13}{233}%
  \test{14}{377}%
  \test{15}{610}%
  \test{16}{987}%
  \test{17}{1597}%
  \test{18}{2584}%
  \test{19}{4181}%
  \test{20}{6765}%
  \test{21}{10946}%
  \test{22}{17711}%
  \test{23}{28657}%
  \test{24}{46368}%
  \test{25}{75025}%
  \test{26}{121393}%
  \test{27}{196418}%
  \test{28}{317811}%
  \test{29}{514229}%
  \test{30}{832040}%
  \test{31}{1346269}%
  \test{32}{2178309}%
  \test{33}{3524578}%
  \test{34}{5702887}%
  \test{35}{9227465}%
  \test{36}{14930352}%
  \test{37}{24157817}%
  \test{38}{39088169}%
  \test{39}{63245986}%
  \test{40}{102334155}%
  \test{41}{165580141}%
  \test{42}{267914296}%
  \test{43}{433494437}%
  \test{44}{701408733}%
  \test{45}{1134903170}%
  \test{46}{1836311903}%
  \test{47}{2971215073}%
  \test{48}{4807526976}%
  \test{49}{7778742049}%
  \test{50}{12586269025}%
  \test{51}{20365011074}%
  \test{52}{32951280099}%
  \test{53}{53316291173}%
  \test{54}{86267571272}%
  \test{55}{139583862445}%
  \test{56}{225851433717}%
  \test{57}{365435296162}%
  \test{58}{591286729879}%
  \test{59}{956722026041}%
  \test{60}{1548008755920}%
  \test{61}{2504730781961}%
  \test{62}{4052739537881}%
  \test{63}{6557470319842}%
  \test{64}{10610209857723}%
  \test{65}{17167680177565}%
  \test{66}{27777890035288}%
  \test{67}{44945570212853}%
  \test{68}{72723460248141}%
  \test{69}{117669030460994}%
  \test{70}{190392490709135}%
  \test{71}{308061521170129}%
  \test{72}{498454011879264}%
  \test{73}{806515533049393}%
}
\def\msg#{\immediate\write16}
\def\test#1#2{%
  \TestAux{#1}{#2}%
  \ifnum#1=0 %
  \else
    \ifodd#1 %
      \TestAux{-#1}{#2}%
    \else
      \TestAux{-#1}{-#2}%
    \fi
  \fi
}
\def\TestAux#1#2{%
  \def\Expected{#2}%
  \expandafter\expandafter\expandafter\def
  \expandafter\expandafter\expandafter\Result
  \expandafter\expandafter\expandafter{%
    \fibnum{#1}%
  }%
  \ltx@onelevel@sanitize\Result
  \ifx\Result\Expected
    \msg{* #1: ok.}%
  \else
    \msg{! fib(#1) = #2}%
    \errmessage{fib(#1) <> \Result}%
  \fi
}
\TestSet
\setbox0=\hbox{%
  \msg{* PreCalc{73}}%
  \fibnumPreCalc{73}%
}
\ifdim\wd0=0pt
\else
  \errmessage{Unwanted stuff in PreCalc}%
\fi
\TestSet
\csname @@end\endcsname\end
%</test-calc>
%    \end{macrocode}
%
% \section{Installation}
%
% \subsection{Download}
%
% \paragraph{Package.} This package is available on
% CTAN\footnote{\url{http://ctan.org/pkg/fibnum}}:
% \begin{description}
% \item[\CTAN{macros/latex/contrib/oberdiek/fibnum.dtx}] The source file.
% \item[\CTAN{macros/latex/contrib/oberdiek/fibnum.pdf}] Documentation.
% \end{description}
%
%
% \paragraph{Bundle.} All the packages of the bundle `oberdiek'
% are also available in a TDS compliant ZIP archive. There
% the packages are already unpacked and the documentation files
% are generated. The files and directories obey the TDS standard.
% \begin{description}
% \item[\CTAN{install/macros/latex/contrib/oberdiek.tds.zip}]
% \end{description}
% \emph{TDS} refers to the standard ``A Directory Structure
% for \TeX\ Files'' (\CTAN{tds/tds.pdf}). Directories
% with \xfile{texmf} in their name are usually organized this way.
%
% \subsection{Bundle installation}
%
% \paragraph{Unpacking.} Unpack the \xfile{oberdiek.tds.zip} in the
% TDS tree (also known as \xfile{texmf} tree) of your choice.
% Example (linux):
% \begin{quote}
%   |unzip oberdiek.tds.zip -d ~/texmf|
% \end{quote}
%
% \paragraph{Script installation.}
% Check the directory \xfile{TDS:scripts/oberdiek/} for
% scripts that need further installation steps.
% Package \xpackage{attachfile2} comes with the Perl script
% \xfile{pdfatfi.pl} that should be installed in such a way
% that it can be called as \texttt{pdfatfi}.
% Example (linux):
% \begin{quote}
%   |chmod +x scripts/oberdiek/pdfatfi.pl|\\
%   |cp scripts/oberdiek/pdfatfi.pl /usr/local/bin/|
% \end{quote}
%
% \subsection{Package installation}
%
% \paragraph{Unpacking.} The \xfile{.dtx} file is a self-extracting
% \docstrip\ archive. The files are extracted by running the
% \xfile{.dtx} through \plainTeX:
% \begin{quote}
%   \verb|tex fibnum.dtx|
% \end{quote}
%
% \paragraph{TDS.} Now the different files must be moved into
% the different directories in your installation TDS tree
% (also known as \xfile{texmf} tree):
% \begin{quote}
% \def\t{^^A
% \begin{tabular}{@{}>{\ttfamily}l@{ $\rightarrow$ }>{\ttfamily}l@{}}
%   fibnum.sty & tex/generic/oberdiek/fibnum.sty\\
%   fibnum.pdf & doc/latex/oberdiek/fibnum.pdf\\
%   test/fibnum-test1.tex & doc/latex/oberdiek/test/fibnum-test1.tex\\
%   test/fibnum-test-calc.tex & doc/latex/oberdiek/test/fibnum-test-calc.tex\\
%   fibnum.dtx & source/latex/oberdiek/fibnum.dtx\\
% \end{tabular}^^A
% }^^A
% \sbox0{\t}^^A
% \ifdim\wd0>\linewidth
%   \begingroup
%     \advance\linewidth by\leftmargin
%     \advance\linewidth by\rightmargin
%   \edef\x{\endgroup
%     \def\noexpand\lw{\the\linewidth}^^A
%   }\x
%   \def\lwbox{^^A
%     \leavevmode
%     \hbox to \linewidth{^^A
%       \kern-\leftmargin\relax
%       \hss
%       \usebox0
%       \hss
%       \kern-\rightmargin\relax
%     }^^A
%   }^^A
%   \ifdim\wd0>\lw
%     \sbox0{\small\t}^^A
%     \ifdim\wd0>\linewidth
%       \ifdim\wd0>\lw
%         \sbox0{\footnotesize\t}^^A
%         \ifdim\wd0>\linewidth
%           \ifdim\wd0>\lw
%             \sbox0{\scriptsize\t}^^A
%             \ifdim\wd0>\linewidth
%               \ifdim\wd0>\lw
%                 \sbox0{\tiny\t}^^A
%                 \ifdim\wd0>\linewidth
%                   \lwbox
%                 \else
%                   \usebox0
%                 \fi
%               \else
%                 \lwbox
%               \fi
%             \else
%               \usebox0
%             \fi
%           \else
%             \lwbox
%           \fi
%         \else
%           \usebox0
%         \fi
%       \else
%         \lwbox
%       \fi
%     \else
%       \usebox0
%     \fi
%   \else
%     \lwbox
%   \fi
% \else
%   \usebox0
% \fi
% \end{quote}
% If you have a \xfile{docstrip.cfg} that configures and enables \docstrip's
% TDS installing feature, then some files can already be in the right
% place, see the documentation of \docstrip.
%
% \subsection{Refresh file name databases}
%
% If your \TeX~distribution
% (\teTeX, \mikTeX, \dots) relies on file name databases, you must refresh
% these. For example, \teTeX\ users run \verb|texhash| or
% \verb|mktexlsr|.
%
% \subsection{Some details for the interested}
%
% \paragraph{Attached source.}
%
% The PDF documentation on CTAN also includes the
% \xfile{.dtx} source file. It can be extracted by
% AcrobatReader 6 or higher. Another option is \textsf{pdftk},
% e.g. unpack the file into the current directory:
% \begin{quote}
%   \verb|pdftk fibnum.pdf unpack_files output .|
% \end{quote}
%
% \paragraph{Unpacking with \LaTeX.}
% The \xfile{.dtx} chooses its action depending on the format:
% \begin{description}
% \item[\plainTeX:] Run \docstrip\ and extract the files.
% \item[\LaTeX:] Generate the documentation.
% \end{description}
% If you insist on using \LaTeX\ for \docstrip\ (really,
% \docstrip\ does not need \LaTeX), then inform the autodetect routine
% about your intention:
% \begin{quote}
%   \verb|latex \let\install=y% \iffalse meta-comment
%
% File: fibnum.dtx
% Version: 2016/05/16 v1.1
% Info: Fibonacci numbers
%
% Copyright (C) 2012 by
%    Heiko Oberdiek <heiko.oberdiek at googlemail.com>
%    2016
%    https://github.com/ho-tex/oberdiek/issues
%
% This work may be distributed and/or modified under the
% conditions of the LaTeX Project Public License, either
% version 1.3c of this license or (at your option) any later
% version. This version of this license is in
%    http://www.latex-project.org/lppl/lppl-1-3c.txt
% and the latest version of this license is in
%    http://www.latex-project.org/lppl.txt
% and version 1.3 or later is part of all distributions of
% LaTeX version 2005/12/01 or later.
%
% This work has the LPPL maintenance status "maintained".
%
% This Current Maintainer of this work is Heiko Oberdiek.
%
% The Base Interpreter refers to any `TeX-Format',
% because some files are installed in TDS:tex/generic//.
%
% This work consists of the main source file fibnum.dtx
% and the derived files
%    fibnum.sty, fibnum.pdf, fibnum.ins, fibnum.drv, fibnum.bib,
%    fibnum-test1.tex, fibnum-test-calc.tex.
%
% Distribution:
%    CTAN:macros/latex/contrib/oberdiek/fibnum.dtx
%    CTAN:macros/latex/contrib/oberdiek/fibnum.pdf
%
% Unpacking:
%    (a) If fibnum.ins is present:
%           tex fibnum.ins
%    (b) Without fibnum.ins:
%           tex fibnum.dtx
%    (c) If you insist on using LaTeX
%           latex \let\install=y% \iffalse meta-comment
%
% File: fibnum.dtx
% Version: 2016/05/16 v1.1
% Info: Fibonacci numbers
%
% Copyright (C) 2012 by
%    Heiko Oberdiek <heiko.oberdiek at googlemail.com>
%    2016
%    https://github.com/ho-tex/oberdiek/issues
%
% This work may be distributed and/or modified under the
% conditions of the LaTeX Project Public License, either
% version 1.3c of this license or (at your option) any later
% version. This version of this license is in
%    http://www.latex-project.org/lppl/lppl-1-3c.txt
% and the latest version of this license is in
%    http://www.latex-project.org/lppl.txt
% and version 1.3 or later is part of all distributions of
% LaTeX version 2005/12/01 or later.
%
% This work has the LPPL maintenance status "maintained".
%
% This Current Maintainer of this work is Heiko Oberdiek.
%
% The Base Interpreter refers to any `TeX-Format',
% because some files are installed in TDS:tex/generic//.
%
% This work consists of the main source file fibnum.dtx
% and the derived files
%    fibnum.sty, fibnum.pdf, fibnum.ins, fibnum.drv, fibnum.bib,
%    fibnum-test1.tex, fibnum-test-calc.tex.
%
% Distribution:
%    CTAN:macros/latex/contrib/oberdiek/fibnum.dtx
%    CTAN:macros/latex/contrib/oberdiek/fibnum.pdf
%
% Unpacking:
%    (a) If fibnum.ins is present:
%           tex fibnum.ins
%    (b) Without fibnum.ins:
%           tex fibnum.dtx
%    (c) If you insist on using LaTeX
%           latex \let\install=y\input{fibnum.dtx}
%        (quote the arguments according to the demands of your shell)
%
% Documentation:
%    (a) If fibnum.drv is present:
%           latex fibnum.drv
%    (b) Without fibnum.drv:
%           latex fibnum.dtx; ...
%    The class ltxdoc loads the configuration file ltxdoc.cfg
%    if available. Here you can specify further options, e.g.
%    use A4 as paper format:
%       \PassOptionsToClass{a4paper}{article}
%
%    Programm calls to get the documentation (example):
%       pdflatex fibnum.dtx
%       bibtex fibnum.aux
%       makeindex -s gind.ist fibnum.idx
%       pdflatex fibnum.dtx
%       makeindex -s gind.ist fibnum.idx
%       pdflatex fibnum.dtx
%
% Installation:
%    TDS:tex/generic/oberdiek/fibnum.sty
%    TDS:doc/latex/oberdiek/fibnum.pdf
%    TDS:doc/latex/oberdiek/test/fibnum-test1.tex
%    TDS:doc/latex/oberdiek/test/fibnum-test-calc.tex
%    TDS:source/latex/oberdiek/fibnum.dtx
%
%<*ignore>
\begingroup
  \catcode123=1 %
  \catcode125=2 %
  \def\x{LaTeX2e}%
\expandafter\endgroup
\ifcase 0\ifx\install y1\fi\expandafter
         \ifx\csname processbatchFile\endcsname\relax\else1\fi
         \ifx\fmtname\x\else 1\fi\relax
\else\csname fi\endcsname
%</ignore>
%<*install>
\input docstrip.tex
\Msg{************************************************************************}
\Msg{* Installation}
\Msg{* Package: fibnum 2016/05/16 v1.1 Fibonacci numbers (HO)}
\Msg{************************************************************************}

\keepsilent
\askforoverwritefalse

\let\MetaPrefix\relax
\preamble

This is a generated file.

Project: fibnum
Version: 2016/05/16 v1.1

Copyright (C) 2012 by
   Heiko Oberdiek <heiko.oberdiek at googlemail.com>

This work may be distributed and/or modified under the
conditions of the LaTeX Project Public License, either
version 1.3c of this license or (at your option) any later
version. This version of this license is in
   http://www.latex-project.org/lppl/lppl-1-3c.txt
and the latest version of this license is in
   http://www.latex-project.org/lppl.txt
and version 1.3 or later is part of all distributions of
LaTeX version 2005/12/01 or later.

This work has the LPPL maintenance status "maintained".

This Current Maintainer of this work is Heiko Oberdiek.

The Base Interpreter refers to any `TeX-Format',
because some files are installed in TDS:tex/generic//.

This work consists of the main source file fibnum.dtx
and the derived files
   fibnum.sty, fibnum.pdf, fibnum.ins, fibnum.drv, fibnum.bib,
   fibnum-test1.tex, fibnum-test-calc.tex.

\endpreamble
\let\MetaPrefix\DoubleperCent

\generate{%
  \file{fibnum.ins}{\from{fibnum.dtx}{install}}%
  \file{fibnum.drv}{\from{fibnum.dtx}{driver}}%
  \nopreamble
  \nopostamble
  \file{fibnum.bib}{\from{fibnum.dtx}{bib}}%
  \usepreamble\defaultpreamble
  \usepostamble\defaultpostamble
  \usedir{tex/generic/oberdiek}%
  \file{fibnum.sty}{\from{fibnum.dtx}{package}}%
  \usedir{doc/latex/oberdiek/test}%
  \file{fibnum-test1.tex}{\from{fibnum.dtx}{test1}}%
  \file{fibnum-test-calc.tex}{\from{fibnum.dtx}{test-calc}}%
}

\catcode32=13\relax% active space
\let =\space%
\Msg{************************************************************************}
\Msg{*}
\Msg{* To finish the installation you have to move the following}
\Msg{* file into a directory searched by TeX:}
\Msg{*}
\Msg{*     fibnum.sty}
\Msg{*}
\Msg{* To produce the documentation run the file `fibnum.drv'}
\Msg{* through LaTeX.}
\Msg{*}
\Msg{* Happy TeXing!}
\Msg{*}
\Msg{************************************************************************}

\endbatchfile
%</install>
%<*bib>
@online{texhax:abraham,
  author={Abraham, Jan},
  title={[texhax] Beginner in TEX MACRO to compute functions},
  date={2012-04-07},
  url={http://tug.org/pipermail/texhax/2012-April/019146.html},
  urldate={2012-04-08},
}
@article{knuth:negafibonacci,
  author={Knuth, Donald E.},
  title={Negafibonacci Numbers and the Hyperbolic Plane},
  date={2008-12-11},
  url={http://research.allacademic.com/meta/p206842_index.html},
}
@online{wikipedia:negafibonacci,
  author={{Wikipedia contributors}},
  organization={{Wikipedia, The Free Encyclopedia}},
  title={Fibonacci numbers},
  language={langenglish},
  version={486266088},
  date={2012-04-08},
  url={http://en.wikipedia.org/w/index.php?title=Fibonacci_number&oldid=486266088},
  urldate={2012-04-08},
}
%</bib>
%<*ignore>
\fi
%</ignore>
%<*driver>
\NeedsTeXFormat{LaTeX2e}
\ProvidesFile{fibnum.drv}%
  [2016/05/16 v1.1 Fibonacci numbers (HO)]%
\documentclass{ltxdoc}
\usepackage{amsmath,amsfonts}
\usepackage{siunitx}
\usepackage{array}
\usepackage{tabularx}
\usepackage{fibnum}[2016/05/16]
\usepackage{holtxdoc}[2011/11/22]
\usepackage{csquotes}
\usepackage[
  bibencoding=ascii,
  alldates=iso8601,
]{biblatex}[2011/11/13]
\bibliography{oberdiek-source}
\bibliography{fibnum}
\begin{document}
  \DocInput{fibnum.dtx}%
\end{document}
%</driver>
% \fi
%
%
% \CharacterTable
%  {Upper-case    \A\B\C\D\E\F\G\H\I\J\K\L\M\N\O\P\Q\R\S\T\U\V\W\X\Y\Z
%   Lower-case    \a\b\c\d\e\f\g\h\i\j\k\l\m\n\o\p\q\r\s\t\u\v\w\x\y\z
%   Digits        \0\1\2\3\4\5\6\7\8\9
%   Exclamation   \!     Double quote  \"     Hash (number) \#
%   Dollar        \$     Percent       \%     Ampersand     \&
%   Acute accent  \'     Left paren    \(     Right paren   \)
%   Asterisk      \*     Plus          \+     Comma         \,
%   Minus         \-     Point         \.     Solidus       \/
%   Colon         \:     Semicolon     \;     Less than     \<
%   Equals        \=     Greater than  \>     Question mark \?
%   Commercial at \@     Left bracket  \[     Backslash     \\
%   Right bracket \]     Circumflex    \^     Underscore    \_
%   Grave accent  \`     Left brace    \{     Vertical bar  \|
%   Right brace   \}     Tilde         \~}
%
% \GetFileInfo{fibnum.drv}
%
% \title{The \xpackage{fibnum} package}
% \date{2016/05/16 v1.1}
% \author{Heiko Oberdiek\thanks
% {Please report any issues at https://github.com/ho-tex/oberdiek/issues}\\
% \xemail{heiko.oberdiek at googlemail.com}}
%
% \maketitle
%
% \begin{abstract}
% The package \xpackage{fibnum} provides expandable fibonacci
% numbers for both \hologo{LaTeX} and \hologo{plainTeX}.
% \end{abstract}
%
% \tableofcontents
%
% \section{Documentation}
%
% In the mailing list \textsf{texhax} Jan Abraham asked the question,
% how to get Fibonacci numbers in \hologo{TeX} \cite{texhax:abraham}:
% \begin{quote}
% Write a Macro in \hologo{TeX} that compute the function |\fib{m}|
% All fibonacci numbers from 1 to $m$ ($m < 40$).
% \end{quote}
% This packages provides an expandable implementation for the
% calculation of these numbers for a much larger set of indexes.
% For practical reasons the index is restricted to the same limitations
% that apply for \hologo{TeX} integer numbers.
% The range of the Fibonacci numbers, however, are not limited
% by the algorithm. They are only restricted to memory limitations,
% if they are hit.
%
% The package is loaded as \hologo{LaTeX} package in \hologo{LaTeX}:
% \begin{quote}
%   |\usepackage{fibnum}|
% \end{quote}
% and as file in \hologo{plainTeX}:
% \begin{quote}
%   |\input fibnum.sty|
% \end{quote}
% The package does not know any options and it provides
% the macros \cs{fibnum} and \cs{fibnumPreCalc}.
%
% \begin{declcs}{fibnum} \M{index}
% \end{declcs}
% Macro \cs{fibnum} expects a \hologo{TeX} number as \meta{index}
% in the official \hologo{TeX} number range from $-(2^{31}-1)$ up to
% $2^{31}-1$. In exact two expansion steps the macro expands to
% the Fibnoacci number $F_{\text{\meta{index}}}$. In case of a negative
% \meta{index}, the ``negafibonacci'' number \cite{wikipedia:negafibonacci}
% is used. Formally the Fibonacci number $F_n$ with integer
% index~$n$, $n\in\mathbb{Z}$ and
% $n\in[\num{-2147483647},\num{2147483647}]$ that is returned by macro
% \cs{fibnum} with numerical argument $n$ is defined the following way:
% \begin{gather}
%   \label{eq:def}
%   F_n =
%   \begin{cases}
%     0 & \text{for $n=0$}\\
%     1 & \text{for $n=1$}\\
%     F_{n-1} + F_{n-2} & \text{for $n>1$}\\
%     (-1)^{n+1}F_n & \text{for $n<0$}
%   \end{cases}
% \end{gather}
% Examples:
% \begin{quote}
%   \makeatletter
%   \def\x#1{\cs{fibnum}\{#1\}&
%     \edef\X{\fibnum{#1}}\edef\Y{\expandafter\ltx@car\X\@nil}^^A
%     \if-\Y
%       \edef\X{\expandafter\ltx@cdr\X\@nil}^^A
%       \noindent
%       \llap{-}\X
%     \else
%       \X
%     \fi
%     \tabularnewline
%   }
%   \def\y{\multicolumn{1}{@{}c@{}}{$\vdots$}\tabularnewline}
%   \DeclareUrlCommand\UrlNum{^^A
%     \urlstyle{tt}^^A
%     \def\UrlBreaks{\do\0\do\1\do\2\do\3\do\4\do\5\do\6\do\7\do\8\do\9}^^A
%   }
%   \begin{tabularx}{\dimexpr\linewidth+5.7pt\relax}{@{}>{\ttfamily}l@{ $\rightarrow$ \hphantom{\ttfamily-}}>{\ttfamily}X@{}}
%     \x{-6}
%     \x{-5}
%     \x{-4}
%     \x{-3}
%     \x{-2}
%     \x{-1}
%     \x{0}
%     \x{1}
%     \x{2}
%     \x{3}
%     \x{4}
%     \x{5}
%     \x{6}
%     \y
%     \x{10}
%     \y
%     \x{46}
%     \y
%     \cs{fibnum}\{100\} & 354224848179261915075
%     \tabularnewline
%     \y
%     \cs{fibnum}\{200\} & 280571172992510140037611932413038677189525
%     \tabularnewline
%     \y
%     \cs{fibnum}\{1000\} &
%       \raggedright
%       \UrlNum{^^A
%         434665576869374564356885276750406258025646^^A
%         605173717804024817290895365554179490518904^^A
%         038798400792551692959225930803226347752096^^A
%         896232398733224711616429964409065331879382^^A
%         98969649928516003704476137795166849228875^^A
%       }
%     \tabularnewline
%   \end{tabularx}\kern-5.7pt\mbox{}
% \end{quote}
%
% \begin{declcs}{fibnumPreCalc} \M{index}
% \end{declcs}
% The package already provides precalculated Fibonacci numbers up to
% index~46. That means that calculations are not necessary for
% Fibonacci numbers that fit into the range of \hologo{TeX}
% numbers. Because macro \cs{fibnum} is expandable, it cannot
% store calculated Fibonacci numbers for later use. Macro definitions
% are forbidden in expandable contexts. If larger Fibonacci numbers
% are used more than once, than the compilation time can be shortened
% by calculating and storing the Fibonacci numbers beforehand.
% The argument \meta{index} is a \hologo{TeX} number and macro
% \cs{fibnumPreCalc} ensures that the Fibonacci numbers
% $F_0$ up to $F_{\lvert\text{\meta{index}}\rvert}$ that are not
% already known are calculated
% and stored in internal macros. Internally only non-negative
% Fibonacci numbers are stored. If \meta{index} is negative, then
% the needed positive Fibonacci numbers are calculated and stored.
% Example:
% \begin{quote}
%   \def\x#1{\begingroup\itshape\texttt{\%} #1\endgroup}
%   |\fibnumPreCalc{50}|\\
%   \x{calculates and stores the values for indexes 47..50.}\\
%   \x{(Values for 0..46 are already stored by the package.)}\\
%   |\fibnum{49}| \x{uses the stored value}\\
%   |\fibnum{51}|
%   \x{only calculates $F_{51}$ from stored values $F_{49}$ and $F_{50}$}\\
%   |\fibnumPreCalc{100}|\\
%   \x{calculates and stores the values for indexes 51..100}\\
%   |\fibnum{100}| \x{uses the stored value for $F_{100}$}\\
%   |\fibnum{-100}|
%   \x{uses the stored value for $F_{100}$}\\
%   \x{$F_{-100}=-F_{100}$ according to equation \eqref{eq:def}.}
% \end{quote}
%
% \StopEventually{
% }
%
% \section{Implementation}
%
% \subsection{Identification}
%
%    \begin{macrocode}
%<*package>
%    \end{macrocode}
%    Reload check, especially if the package is not used with \LaTeX.
%    \begin{macrocode}
\begingroup\catcode61\catcode48\catcode32=10\relax%
  \catcode13=5 % ^^M
  \endlinechar=13 %
  \catcode35=6 % #
  \catcode39=12 % '
  \catcode44=12 % ,
  \catcode45=12 % -
  \catcode46=12 % .
  \catcode58=12 % :
  \catcode64=11 % @
  \catcode123=1 % {
  \catcode125=2 % }
  \expandafter\let\expandafter\x\csname ver@fibnum.sty\endcsname
  \ifx\x\relax % plain-TeX, first loading
  \else
    \def\empty{}%
    \ifx\x\empty % LaTeX, first loading,
      % variable is initialized, but \ProvidesPackage not yet seen
    \else
      \expandafter\ifx\csname PackageInfo\endcsname\relax
        \def\x#1#2{%
          \immediate\write-1{Package #1 Info: #2.}%
        }%
      \else
        \def\x#1#2{\PackageInfo{#1}{#2, stopped}}%
      \fi
      \x{fibnum}{The package is already loaded}%
      \aftergroup\endinput
    \fi
  \fi
\endgroup%
%    \end{macrocode}
%    Package identification:
%    \begin{macrocode}
\begingroup\catcode61\catcode48\catcode32=10\relax%
  \catcode13=5 % ^^M
  \endlinechar=13 %
  \catcode35=6 % #
  \catcode39=12 % '
  \catcode40=12 % (
  \catcode41=12 % )
  \catcode44=12 % ,
  \catcode45=12 % -
  \catcode46=12 % .
  \catcode47=12 % /
  \catcode58=12 % :
  \catcode64=11 % @
  \catcode91=12 % [
  \catcode93=12 % ]
  \catcode123=1 % {
  \catcode125=2 % }
  \expandafter\ifx\csname ProvidesPackage\endcsname\relax
    \def\x#1#2#3[#4]{\endgroup
      \immediate\write-1{Package: #3 #4}%
      \xdef#1{#4}%
    }%
  \else
    \def\x#1#2[#3]{\endgroup
      #2[{#3}]%
      \ifx#1\@undefined
        \xdef#1{#3}%
      \fi
      \ifx#1\relax
        \xdef#1{#3}%
      \fi
    }%
  \fi
\expandafter\x\csname ver@fibnum.sty\endcsname
\ProvidesPackage{fibnum}%
  [2016/05/16 v1.1 Fibonacci numbers (HO)]%
%    \end{macrocode}
%
%    \begin{macrocode}
\begingroup\catcode61\catcode48\catcode32=10\relax%
  \catcode13=5 % ^^M
  \endlinechar=13 %
  \catcode123=1 % {
  \catcode125=2 % }
  \catcode64=11 % @
  \def\x{\endgroup
    \expandafter\edef\csname FibNum@AtEnd\endcsname{%
      \endlinechar=\the\endlinechar\relax
      \catcode13=\the\catcode13\relax
      \catcode32=\the\catcode32\relax
      \catcode35=\the\catcode35\relax
      \catcode61=\the\catcode61\relax
      \catcode64=\the\catcode64\relax
      \catcode123=\the\catcode123\relax
      \catcode125=\the\catcode125\relax
    }%
  }%
\x\catcode61\catcode48\catcode32=10\relax%
\catcode13=5 % ^^M
\endlinechar=13 %
\catcode35=6 % #
\catcode64=11 % @
\catcode123=1 % {
\catcode125=2 % }
\def\TMP@EnsureCode#1#2{%
  \edef\FibNum@AtEnd{%
    \FibNum@AtEnd
    \catcode#1=\the\catcode#1\relax
  }%
  \catcode#1=#2\relax
}
\TMP@EnsureCode{33}{12}% !
%\TMP@EnsureCode{36}{3}% $
%\TMP@EnsureCode{38}{4}% &
\TMP@EnsureCode{40}{12}% (
\TMP@EnsureCode{41}{12}% )
\TMP@EnsureCode{45}{12}% -
\TMP@EnsureCode{46}{12}% .
\TMP@EnsureCode{47}{12}% /
\TMP@EnsureCode{58}{12}% :
\TMP@EnsureCode{60}{12}% <
\TMP@EnsureCode{62}{12}% >
\TMP@EnsureCode{91}{12}% [
%\TMP@EnsureCode{96}{12}% `
\TMP@EnsureCode{93}{12}% ]
%\TMP@EnsureCode{94}{12}% ^ (superscript) (!)
%\TMP@EnsureCode{124}{12}% |
\edef\FibNum@AtEnd{\FibNum@AtEnd\noexpand\endinput}
%    \end{macrocode}
%
% \subsection{Package resources}
%
%    \begin{macrocode}
\begingroup\expandafter\expandafter\expandafter\endgroup
\expandafter\ifx\csname RequirePackage\endcsname\relax
  \def\TMP@RequirePackage#1[#2]{%
    \begingroup\expandafter\expandafter\expandafter\endgroup
    \expandafter\ifx\csname ver@#1.sty\endcsname\relax
      \input #1.sty\relax
    \fi
  }%
  \TMP@RequirePackage{ltxcmds}[2011/04/18]%
  \TMP@RequirePackage{intcalc}[2007/09/27]%
  \TMP@RequirePackage{bigintcalc}[2007/11/11]%
\else
  \RequirePackage{ltxcmds}[2011/04/18]%
  \RequirePackage{intcalc}[2007/09/27]%
  \RequirePackage{bigintcalc}[2007/11/11]%
\fi
%    \end{macrocode}
%
% \subsection{Setup precalculated values}
%
%    \begin{macrocode}
\def\FibNum@temp#1{%
  \expandafter\def\csname FibNum@#1\endcsname
}
\catcode46=9 % dots are ignored
\FibNum@temp{0}{0}
\FibNum@temp{1}{1}
\FibNum@temp{2}{1}
\FibNum@temp{3}{2}
\FibNum@temp{4}{3}
\FibNum@temp{5}{5}
\FibNum@temp{6}{8}
\FibNum@temp{7}{13}
\FibNum@temp{8}{21}
\FibNum@temp{9}{34}
\FibNum@temp{10}{55}
\FibNum@temp{11}{89}
\FibNum@temp{12}{144}
\FibNum@temp{13}{233}
\FibNum@temp{14}{377}
\FibNum@temp{15}{610}
\FibNum@temp{16}{987}
\FibNum@temp{17}{1.597}
\FibNum@temp{18}{2.584}
\FibNum@temp{19}{4.181}
\FibNum@temp{20}{6.765}
\FibNum@temp{21}{10.946}
\FibNum@temp{22}{17.711}
\FibNum@temp{23}{28.657}
\FibNum@temp{24}{46.368}
\FibNum@temp{25}{75.025}
\FibNum@temp{26}{121.393}
\FibNum@temp{27}{196.418}
\FibNum@temp{28}{317.811}
\FibNum@temp{29}{514.229}
\FibNum@temp{30}{832.040}
\FibNum@temp{31}{1.346.269}
\FibNum@temp{32}{2.178.309}
\FibNum@temp{33}{3.524.578}
\FibNum@temp{34}{5.702.887}
\FibNum@temp{35}{9.227.465}
\FibNum@temp{36}{14.930.352}
\FibNum@temp{37}{24.157.817}
\FibNum@temp{38}{39.088.169}
\FibNum@temp{39}{63.245.986}
\FibNum@temp{40}{102.334.155}
\FibNum@temp{41}{165.580.141}
\FibNum@temp{42}{267.914.296}
\FibNum@temp{43}{433.494.437}
\FibNum@temp{44}{701.408.733}
\FibNum@temp{45}{1.134.903.170}
\FibNum@temp{46}{1.836.311.903}
%    \end{macrocode}
%    \begin{macro}{\FibNum@max}
%    \begin{macrocode}
\def\FibNum@max{46}
%    \end{macrocode}
%    \end{macro}
%
% \subsection{Macros for precalculating values}
%
%    \begin{macro}{\fibnumPreCalc}
%    \begin{macrocode}
\def\fibnumPreCalc#1{%
  \expandafter\expandafter\expandafter
  \FibNum@PreCalc\intcalcNum{#1}/%
}
%    \end{macrocode}
%    \end{macro}
%    \begin{macro}{\FibNum@PreCalc}
%    \begin{macrocode}
\def\FibNum@PreCalc#1/{%
  \ifnum#1<\ltx@zero
    \expandafter\FibNum@PreCalc\ltx@gobble#1/%
  \else
    \ifnum#1>\FibNum@max
      \begingroup
        \ltx@LocDimenA=#1sp\relax
        \countdef\FibNum@i=255\relax
        \FibNum@i=\FibNum@max\relax
        \edef\FibNum@temp{%
          \csname FibNum@\the\FibNum@i\endcsname/%
        }%
        \advance\FibNum@i by -1\relax
        \edef\FibNum@temp{%
          \FibNum@temp
          \csname FibNum@\the\FibNum@i\endcsname
        }%
        \advance\FibNum@i\ltx@two
        \iftrue
          \expandafter\FibNum@PreCalcAux\FibNum@temp
        \fi
      \endgroup
    \fi
  \fi
}
%    \end{macrocode}
%    \end{macro}
%    \begin{macro}{\FibNum@PreCalcAux}
%    \begin{macrocode}
\def\FibNum@PreCalcAux#1/#2\fi{%
  \fi
  \edef\FibNum@temp{\BigIntCalcAdd#1!#2!}%
  \global\expandafter
  \let\csname FibNum@\the\FibNum@i\endcsname\FibNum@temp
  \ifnum\FibNum@i=\ltx@LocDimenA
    \xdef\FibNum@max{\the\FibNum@i}%
  \else
    \advance\FibNum@i\ltx@one
    \expandafter\FibNum@PreCalcAux\FibNum@temp/#1%
  \fi
}
%    \end{macrocode}
%    \end{macro}
%
% \subsection{Expandable calculations}
%
%    \begin{macro}{\fibnum}
%    \begin{macrocode}
\def\fibnum#1{%
  \romannumeral
  \expandafter\expandafter\expandafter\FibNum@Do\intcalcNum{#1}/%
}
%    \end{macrocode}
%    \end{macro}
%    \begin{macro}{\FibNum@Do}
%    \begin{macrocode}
\def\FibNum@Do#1/{%
  \ifnum#1<\ltx@zero
    \FibNum@ReturnAfterElseFiFi{%
      \ifodd#1 %
        \expandafter\expandafter\expandafter\ltx@zero
      \else
        \expandafter\expandafter\expandafter\ltx@zero
        \expandafter\expandafter\expandafter-%
      \fi
      \romannumeral
      \expandafter\FibNum@Do\ltx@gobble#1/%
    }%
  \else
    \ifnum\FibNum@max<#1 %
      \ltx@ReturnAfterElseFi{%
        \expandafter
        \FibNum@ExpCalc\number\expandafter\IntCalcInc\FibNum@max!%
        \expandafter\expandafter\expandafter/%
        \csname FibNum@\FibNum@max
        \expandafter\expandafter\expandafter\endcsname
        \expandafter\expandafter\expandafter/%
        \csname FibNum@\expandafter\IntCalcDec\FibNum@max!%
        \endcsname/%
        #1%
      }%
    \else
      \expandafter\expandafter\expandafter\ltx@zero
      \csname FibNum@#1\expandafter\expandafter\expandafter\endcsname
    \fi
  \fi
}
%    \end{macrocode}
%    \end{macro}
%    \begin{macro}{\FibNum@ReturnAfterElseFiFi}
%    \begin{macrocode}
\def\FibNum@ReturnAfterElseFiFi#1\else#2\fi\fi{\fi#1}
%    \end{macrocode}
%    \end{macro}
%    \begin{macro}{\FibNum@ExpCalc}
%    \begin{macrocode}
\def\FibNum@ExpCalc#1/#2/#3/#4\fi{%
  \fi
  \ifnum#1=#4 %
    \ltx@ReturnAfterElseFi{%
      \expandafter\expandafter\expandafter\ltx@zero
      \BigIntCalcAdd#2!#3!%
    }%
  \else
    \expandafter\FibNum@ExpCalc
    \number\IntCalcInc#1!%
    \expandafter\expandafter\expandafter/%
    \BigIntCalcAdd#2!#3!/%
    #2/#4%
  \fi
}
%    \end{macrocode}
%    \end{macro}
%
%    \begin{macrocode}
\FibNum@AtEnd%
%</package>
%    \end{macrocode}
%
% \section{Test}
%
% \subsection{Catcode checks for loading}
%
%    \begin{macrocode}
%<*test1>
%    \end{macrocode}
%    \begin{macrocode}
\catcode`\{=1 %
\catcode`\}=2 %
\catcode`\#=6 %
\catcode`\@=11 %
\expandafter\ifx\csname count@\endcsname\relax
  \countdef\count@=255 %
\fi
\expandafter\ifx\csname @gobble\endcsname\relax
  \long\def\@gobble#1{}%
\fi
\expandafter\ifx\csname @firstofone\endcsname\relax
  \long\def\@firstofone#1{#1}%
\fi
\expandafter\ifx\csname loop\endcsname\relax
  \expandafter\@firstofone
\else
  \expandafter\@gobble
\fi
{%
  \def\loop#1\repeat{%
    \def\body{#1}%
    \iterate
  }%
  \def\iterate{%
    \body
      \let\next\iterate
    \else
      \let\next\relax
    \fi
    \next
  }%
  \let\repeat=\fi
}%
\def\RestoreCatcodes{}
\count@=0 %
\loop
  \edef\RestoreCatcodes{%
    \RestoreCatcodes
    \catcode\the\count@=\the\catcode\count@\relax
  }%
\ifnum\count@<255 %
  \advance\count@ 1 %
\repeat

\def\RangeCatcodeInvalid#1#2{%
  \count@=#1\relax
  \loop
    \catcode\count@=15 %
  \ifnum\count@<#2\relax
    \advance\count@ 1 %
  \repeat
}
\def\RangeCatcodeCheck#1#2#3{%
  \count@=#1\relax
  \loop
    \ifnum#3=\catcode\count@
    \else
      \errmessage{%
        Character \the\count@\space
        with wrong catcode \the\catcode\count@\space
        instead of \number#3%
      }%
    \fi
  \ifnum\count@<#2\relax
    \advance\count@ 1 %
  \repeat
}
\def\space{ }
\expandafter\ifx\csname LoadCommand\endcsname\relax
  \def\LoadCommand{\input fibnum.sty\relax}%
\fi
\def\Test{%
  \RangeCatcodeInvalid{0}{47}%
  \RangeCatcodeInvalid{58}{64}%
  \RangeCatcodeInvalid{91}{96}%
  \RangeCatcodeInvalid{123}{255}%
  \catcode`\@=12 %
  \catcode`\\=0 %
  \catcode`\%=14 %
  \LoadCommand
  \RangeCatcodeCheck{0}{36}{15}%
  \RangeCatcodeCheck{37}{37}{14}%
  \RangeCatcodeCheck{38}{47}{15}%
  \RangeCatcodeCheck{48}{57}{12}%
  \RangeCatcodeCheck{58}{63}{15}%
  \RangeCatcodeCheck{64}{64}{12}%
  \RangeCatcodeCheck{65}{90}{11}%
  \RangeCatcodeCheck{91}{91}{15}%
  \RangeCatcodeCheck{92}{92}{0}%
  \RangeCatcodeCheck{93}{96}{15}%
  \RangeCatcodeCheck{97}{122}{11}%
  \RangeCatcodeCheck{123}{255}{15}%
  \RestoreCatcodes
}
\Test
\csname @@end\endcsname
\end
%    \end{macrocode}
%    \begin{macrocode}
%</test1>
%    \end{macrocode}
%
% \subsection{Test calculations}
%
%    \begin{macrocode}
%<*test-calc>
\catcode`\{=1 %
\catcode`\}=2 %
\catcode`\#=6 %
\catcode`\@=11 %
\begingroup\expandafter\expandafter\expandafter\endgroup
\expandafter\ifx\csname RequirePackage\endcsname\relax
  \input fibnum.sty\relax
\else
  \RequirePackage{fibnum}[2016/05/16]%
\fi
\def\TestSet{%
  \test{0}{0}%
  \test{1}{1}%
  \test{2}{1}%
  \test{3}{2}%
  \test{4}{3}%
  \test{5}{5}%
  \test{6}{8}%
  \test{7}{13}%
  \test{8}{21}%
  \test{9}{34}%
  \test{10}{55}%
  \test{11}{89}%
  \test{12}{144}%
  \test{13}{233}%
  \test{14}{377}%
  \test{15}{610}%
  \test{16}{987}%
  \test{17}{1597}%
  \test{18}{2584}%
  \test{19}{4181}%
  \test{20}{6765}%
  \test{21}{10946}%
  \test{22}{17711}%
  \test{23}{28657}%
  \test{24}{46368}%
  \test{25}{75025}%
  \test{26}{121393}%
  \test{27}{196418}%
  \test{28}{317811}%
  \test{29}{514229}%
  \test{30}{832040}%
  \test{31}{1346269}%
  \test{32}{2178309}%
  \test{33}{3524578}%
  \test{34}{5702887}%
  \test{35}{9227465}%
  \test{36}{14930352}%
  \test{37}{24157817}%
  \test{38}{39088169}%
  \test{39}{63245986}%
  \test{40}{102334155}%
  \test{41}{165580141}%
  \test{42}{267914296}%
  \test{43}{433494437}%
  \test{44}{701408733}%
  \test{45}{1134903170}%
  \test{46}{1836311903}%
  \test{47}{2971215073}%
  \test{48}{4807526976}%
  \test{49}{7778742049}%
  \test{50}{12586269025}%
  \test{51}{20365011074}%
  \test{52}{32951280099}%
  \test{53}{53316291173}%
  \test{54}{86267571272}%
  \test{55}{139583862445}%
  \test{56}{225851433717}%
  \test{57}{365435296162}%
  \test{58}{591286729879}%
  \test{59}{956722026041}%
  \test{60}{1548008755920}%
  \test{61}{2504730781961}%
  \test{62}{4052739537881}%
  \test{63}{6557470319842}%
  \test{64}{10610209857723}%
  \test{65}{17167680177565}%
  \test{66}{27777890035288}%
  \test{67}{44945570212853}%
  \test{68}{72723460248141}%
  \test{69}{117669030460994}%
  \test{70}{190392490709135}%
  \test{71}{308061521170129}%
  \test{72}{498454011879264}%
  \test{73}{806515533049393}%
}
\def\msg#{\immediate\write16}
\def\test#1#2{%
  \TestAux{#1}{#2}%
  \ifnum#1=0 %
  \else
    \ifodd#1 %
      \TestAux{-#1}{#2}%
    \else
      \TestAux{-#1}{-#2}%
    \fi
  \fi
}
\def\TestAux#1#2{%
  \def\Expected{#2}%
  \expandafter\expandafter\expandafter\def
  \expandafter\expandafter\expandafter\Result
  \expandafter\expandafter\expandafter{%
    \fibnum{#1}%
  }%
  \ltx@onelevel@sanitize\Result
  \ifx\Result\Expected
    \msg{* #1: ok.}%
  \else
    \msg{! fib(#1) = #2}%
    \errmessage{fib(#1) <> \Result}%
  \fi
}
\TestSet
\setbox0=\hbox{%
  \msg{* PreCalc{73}}%
  \fibnumPreCalc{73}%
}
\ifdim\wd0=0pt
\else
  \errmessage{Unwanted stuff in PreCalc}%
\fi
\TestSet
\csname @@end\endcsname\end
%</test-calc>
%    \end{macrocode}
%
% \section{Installation}
%
% \subsection{Download}
%
% \paragraph{Package.} This package is available on
% CTAN\footnote{\url{http://ctan.org/pkg/fibnum}}:
% \begin{description}
% \item[\CTAN{macros/latex/contrib/oberdiek/fibnum.dtx}] The source file.
% \item[\CTAN{macros/latex/contrib/oberdiek/fibnum.pdf}] Documentation.
% \end{description}
%
%
% \paragraph{Bundle.} All the packages of the bundle `oberdiek'
% are also available in a TDS compliant ZIP archive. There
% the packages are already unpacked and the documentation files
% are generated. The files and directories obey the TDS standard.
% \begin{description}
% \item[\CTAN{install/macros/latex/contrib/oberdiek.tds.zip}]
% \end{description}
% \emph{TDS} refers to the standard ``A Directory Structure
% for \TeX\ Files'' (\CTAN{tds/tds.pdf}). Directories
% with \xfile{texmf} in their name are usually organized this way.
%
% \subsection{Bundle installation}
%
% \paragraph{Unpacking.} Unpack the \xfile{oberdiek.tds.zip} in the
% TDS tree (also known as \xfile{texmf} tree) of your choice.
% Example (linux):
% \begin{quote}
%   |unzip oberdiek.tds.zip -d ~/texmf|
% \end{quote}
%
% \paragraph{Script installation.}
% Check the directory \xfile{TDS:scripts/oberdiek/} for
% scripts that need further installation steps.
% Package \xpackage{attachfile2} comes with the Perl script
% \xfile{pdfatfi.pl} that should be installed in such a way
% that it can be called as \texttt{pdfatfi}.
% Example (linux):
% \begin{quote}
%   |chmod +x scripts/oberdiek/pdfatfi.pl|\\
%   |cp scripts/oberdiek/pdfatfi.pl /usr/local/bin/|
% \end{quote}
%
% \subsection{Package installation}
%
% \paragraph{Unpacking.} The \xfile{.dtx} file is a self-extracting
% \docstrip\ archive. The files are extracted by running the
% \xfile{.dtx} through \plainTeX:
% \begin{quote}
%   \verb|tex fibnum.dtx|
% \end{quote}
%
% \paragraph{TDS.} Now the different files must be moved into
% the different directories in your installation TDS tree
% (also known as \xfile{texmf} tree):
% \begin{quote}
% \def\t{^^A
% \begin{tabular}{@{}>{\ttfamily}l@{ $\rightarrow$ }>{\ttfamily}l@{}}
%   fibnum.sty & tex/generic/oberdiek/fibnum.sty\\
%   fibnum.pdf & doc/latex/oberdiek/fibnum.pdf\\
%   test/fibnum-test1.tex & doc/latex/oberdiek/test/fibnum-test1.tex\\
%   test/fibnum-test-calc.tex & doc/latex/oberdiek/test/fibnum-test-calc.tex\\
%   fibnum.dtx & source/latex/oberdiek/fibnum.dtx\\
% \end{tabular}^^A
% }^^A
% \sbox0{\t}^^A
% \ifdim\wd0>\linewidth
%   \begingroup
%     \advance\linewidth by\leftmargin
%     \advance\linewidth by\rightmargin
%   \edef\x{\endgroup
%     \def\noexpand\lw{\the\linewidth}^^A
%   }\x
%   \def\lwbox{^^A
%     \leavevmode
%     \hbox to \linewidth{^^A
%       \kern-\leftmargin\relax
%       \hss
%       \usebox0
%       \hss
%       \kern-\rightmargin\relax
%     }^^A
%   }^^A
%   \ifdim\wd0>\lw
%     \sbox0{\small\t}^^A
%     \ifdim\wd0>\linewidth
%       \ifdim\wd0>\lw
%         \sbox0{\footnotesize\t}^^A
%         \ifdim\wd0>\linewidth
%           \ifdim\wd0>\lw
%             \sbox0{\scriptsize\t}^^A
%             \ifdim\wd0>\linewidth
%               \ifdim\wd0>\lw
%                 \sbox0{\tiny\t}^^A
%                 \ifdim\wd0>\linewidth
%                   \lwbox
%                 \else
%                   \usebox0
%                 \fi
%               \else
%                 \lwbox
%               \fi
%             \else
%               \usebox0
%             \fi
%           \else
%             \lwbox
%           \fi
%         \else
%           \usebox0
%         \fi
%       \else
%         \lwbox
%       \fi
%     \else
%       \usebox0
%     \fi
%   \else
%     \lwbox
%   \fi
% \else
%   \usebox0
% \fi
% \end{quote}
% If you have a \xfile{docstrip.cfg} that configures and enables \docstrip's
% TDS installing feature, then some files can already be in the right
% place, see the documentation of \docstrip.
%
% \subsection{Refresh file name databases}
%
% If your \TeX~distribution
% (\teTeX, \mikTeX, \dots) relies on file name databases, you must refresh
% these. For example, \teTeX\ users run \verb|texhash| or
% \verb|mktexlsr|.
%
% \subsection{Some details for the interested}
%
% \paragraph{Attached source.}
%
% The PDF documentation on CTAN also includes the
% \xfile{.dtx} source file. It can be extracted by
% AcrobatReader 6 or higher. Another option is \textsf{pdftk},
% e.g. unpack the file into the current directory:
% \begin{quote}
%   \verb|pdftk fibnum.pdf unpack_files output .|
% \end{quote}
%
% \paragraph{Unpacking with \LaTeX.}
% The \xfile{.dtx} chooses its action depending on the format:
% \begin{description}
% \item[\plainTeX:] Run \docstrip\ and extract the files.
% \item[\LaTeX:] Generate the documentation.
% \end{description}
% If you insist on using \LaTeX\ for \docstrip\ (really,
% \docstrip\ does not need \LaTeX), then inform the autodetect routine
% about your intention:
% \begin{quote}
%   \verb|latex \let\install=y\input{fibnum.dtx}|
% \end{quote}
% Do not forget to quote the argument according to the demands
% of your shell.
%
% \paragraph{Generating the documentation.}
% You can use both the \xfile{.dtx} or the \xfile{.drv} to generate
% the documentation. The process can be configured by the
% configuration file \xfile{ltxdoc.cfg}. For instance, put this
% line into this file, if you want to have A4 as paper format:
% \begin{quote}
%   \verb|\PassOptionsToClass{a4paper}{article}|
% \end{quote}
% An example follows how to generate the
% documentation with pdf\LaTeX:
% \begin{quote}
%\begin{verbatim}
%pdflatex fibnum.dtx
%bibtex fibnum.aux
%makeindex -s gind.ist fibnum.idx
%pdflatex fibnum.dtx
%makeindex -s gind.ist fibnum.idx
%pdflatex fibnum.dtx
%\end{verbatim}
% \end{quote}
%
% \printbibliography[
%   heading=bibnumbered,
% ]
%
% \begin{History}
%   \begin{Version}{2012/04/08 v1.0}
%   \item
%     First version.
%   \end{Version}
%   \begin{Version}{2016/05/16 v1.1}
%   \item
%     Documentation updates.
%   \end{Version}
% \end{History}
%
% \PrintIndex
%
% \Finale
\endinput

%        (quote the arguments according to the demands of your shell)
%
% Documentation:
%    (a) If fibnum.drv is present:
%           latex fibnum.drv
%    (b) Without fibnum.drv:
%           latex fibnum.dtx; ...
%    The class ltxdoc loads the configuration file ltxdoc.cfg
%    if available. Here you can specify further options, e.g.
%    use A4 as paper format:
%       \PassOptionsToClass{a4paper}{article}
%
%    Programm calls to get the documentation (example):
%       pdflatex fibnum.dtx
%       bibtex fibnum.aux
%       makeindex -s gind.ist fibnum.idx
%       pdflatex fibnum.dtx
%       makeindex -s gind.ist fibnum.idx
%       pdflatex fibnum.dtx
%
% Installation:
%    TDS:tex/generic/oberdiek/fibnum.sty
%    TDS:doc/latex/oberdiek/fibnum.pdf
%    TDS:doc/latex/oberdiek/test/fibnum-test1.tex
%    TDS:doc/latex/oberdiek/test/fibnum-test-calc.tex
%    TDS:source/latex/oberdiek/fibnum.dtx
%
%<*ignore>
\begingroup
  \catcode123=1 %
  \catcode125=2 %
  \def\x{LaTeX2e}%
\expandafter\endgroup
\ifcase 0\ifx\install y1\fi\expandafter
         \ifx\csname processbatchFile\endcsname\relax\else1\fi
         \ifx\fmtname\x\else 1\fi\relax
\else\csname fi\endcsname
%</ignore>
%<*install>
\input docstrip.tex
\Msg{************************************************************************}
\Msg{* Installation}
\Msg{* Package: fibnum 2016/05/16 v1.1 Fibonacci numbers (HO)}
\Msg{************************************************************************}

\keepsilent
\askforoverwritefalse

\let\MetaPrefix\relax
\preamble

This is a generated file.

Project: fibnum
Version: 2016/05/16 v1.1

Copyright (C) 2012 by
   Heiko Oberdiek <heiko.oberdiek at googlemail.com>

This work may be distributed and/or modified under the
conditions of the LaTeX Project Public License, either
version 1.3c of this license or (at your option) any later
version. This version of this license is in
   http://www.latex-project.org/lppl/lppl-1-3c.txt
and the latest version of this license is in
   http://www.latex-project.org/lppl.txt
and version 1.3 or later is part of all distributions of
LaTeX version 2005/12/01 or later.

This work has the LPPL maintenance status "maintained".

This Current Maintainer of this work is Heiko Oberdiek.

The Base Interpreter refers to any `TeX-Format',
because some files are installed in TDS:tex/generic//.

This work consists of the main source file fibnum.dtx
and the derived files
   fibnum.sty, fibnum.pdf, fibnum.ins, fibnum.drv, fibnum.bib,
   fibnum-test1.tex, fibnum-test-calc.tex.

\endpreamble
\let\MetaPrefix\DoubleperCent

\generate{%
  \file{fibnum.ins}{\from{fibnum.dtx}{install}}%
  \file{fibnum.drv}{\from{fibnum.dtx}{driver}}%
  \nopreamble
  \nopostamble
  \file{fibnum.bib}{\from{fibnum.dtx}{bib}}%
  \usepreamble\defaultpreamble
  \usepostamble\defaultpostamble
  \usedir{tex/generic/oberdiek}%
  \file{fibnum.sty}{\from{fibnum.dtx}{package}}%
  \usedir{doc/latex/oberdiek/test}%
  \file{fibnum-test1.tex}{\from{fibnum.dtx}{test1}}%
  \file{fibnum-test-calc.tex}{\from{fibnum.dtx}{test-calc}}%
}

\catcode32=13\relax% active space
\let =\space%
\Msg{************************************************************************}
\Msg{*}
\Msg{* To finish the installation you have to move the following}
\Msg{* file into a directory searched by TeX:}
\Msg{*}
\Msg{*     fibnum.sty}
\Msg{*}
\Msg{* To produce the documentation run the file `fibnum.drv'}
\Msg{* through LaTeX.}
\Msg{*}
\Msg{* Happy TeXing!}
\Msg{*}
\Msg{************************************************************************}

\endbatchfile
%</install>
%<*bib>
@online{texhax:abraham,
  author={Abraham, Jan},
  title={[texhax] Beginner in TEX MACRO to compute functions},
  date={2012-04-07},
  url={http://tug.org/pipermail/texhax/2012-April/019146.html},
  urldate={2012-04-08},
}
@article{knuth:negafibonacci,
  author={Knuth, Donald E.},
  title={Negafibonacci Numbers and the Hyperbolic Plane},
  date={2008-12-11},
  url={http://research.allacademic.com/meta/p206842_index.html},
}
@online{wikipedia:negafibonacci,
  author={{Wikipedia contributors}},
  organization={{Wikipedia, The Free Encyclopedia}},
  title={Fibonacci numbers},
  language={langenglish},
  version={486266088},
  date={2012-04-08},
  url={http://en.wikipedia.org/w/index.php?title=Fibonacci_number&oldid=486266088},
  urldate={2012-04-08},
}
%</bib>
%<*ignore>
\fi
%</ignore>
%<*driver>
\NeedsTeXFormat{LaTeX2e}
\ProvidesFile{fibnum.drv}%
  [2016/05/16 v1.1 Fibonacci numbers (HO)]%
\documentclass{ltxdoc}
\usepackage{amsmath,amsfonts}
\usepackage{siunitx}
\usepackage{array}
\usepackage{tabularx}
\usepackage{fibnum}[2016/05/16]
\usepackage{holtxdoc}[2011/11/22]
\usepackage{csquotes}
\usepackage[
  bibencoding=ascii,
  alldates=iso8601,
]{biblatex}[2011/11/13]
\bibliography{oberdiek-source}
\bibliography{fibnum}
\begin{document}
  \DocInput{fibnum.dtx}%
\end{document}
%</driver>
% \fi
%
%
% \CharacterTable
%  {Upper-case    \A\B\C\D\E\F\G\H\I\J\K\L\M\N\O\P\Q\R\S\T\U\V\W\X\Y\Z
%   Lower-case    \a\b\c\d\e\f\g\h\i\j\k\l\m\n\o\p\q\r\s\t\u\v\w\x\y\z
%   Digits        \0\1\2\3\4\5\6\7\8\9
%   Exclamation   \!     Double quote  \"     Hash (number) \#
%   Dollar        \$     Percent       \%     Ampersand     \&
%   Acute accent  \'     Left paren    \(     Right paren   \)
%   Asterisk      \*     Plus          \+     Comma         \,
%   Minus         \-     Point         \.     Solidus       \/
%   Colon         \:     Semicolon     \;     Less than     \<
%   Equals        \=     Greater than  \>     Question mark \?
%   Commercial at \@     Left bracket  \[     Backslash     \\
%   Right bracket \]     Circumflex    \^     Underscore    \_
%   Grave accent  \`     Left brace    \{     Vertical bar  \|
%   Right brace   \}     Tilde         \~}
%
% \GetFileInfo{fibnum.drv}
%
% \title{The \xpackage{fibnum} package}
% \date{2016/05/16 v1.1}
% \author{Heiko Oberdiek\thanks
% {Please report any issues at https://github.com/ho-tex/oberdiek/issues}\\
% \xemail{heiko.oberdiek at googlemail.com}}
%
% \maketitle
%
% \begin{abstract}
% The package \xpackage{fibnum} provides expandable fibonacci
% numbers for both \hologo{LaTeX} and \hologo{plainTeX}.
% \end{abstract}
%
% \tableofcontents
%
% \section{Documentation}
%
% In the mailing list \textsf{texhax} Jan Abraham asked the question,
% how to get Fibonacci numbers in \hologo{TeX} \cite{texhax:abraham}:
% \begin{quote}
% Write a Macro in \hologo{TeX} that compute the function |\fib{m}|
% All fibonacci numbers from 1 to $m$ ($m < 40$).
% \end{quote}
% This packages provides an expandable implementation for the
% calculation of these numbers for a much larger set of indexes.
% For practical reasons the index is restricted to the same limitations
% that apply for \hologo{TeX} integer numbers.
% The range of the Fibonacci numbers, however, are not limited
% by the algorithm. They are only restricted to memory limitations,
% if they are hit.
%
% The package is loaded as \hologo{LaTeX} package in \hologo{LaTeX}:
% \begin{quote}
%   |\usepackage{fibnum}|
% \end{quote}
% and as file in \hologo{plainTeX}:
% \begin{quote}
%   |\input fibnum.sty|
% \end{quote}
% The package does not know any options and it provides
% the macros \cs{fibnum} and \cs{fibnumPreCalc}.
%
% \begin{declcs}{fibnum} \M{index}
% \end{declcs}
% Macro \cs{fibnum} expects a \hologo{TeX} number as \meta{index}
% in the official \hologo{TeX} number range from $-(2^{31}-1)$ up to
% $2^{31}-1$. In exact two expansion steps the macro expands to
% the Fibnoacci number $F_{\text{\meta{index}}}$. In case of a negative
% \meta{index}, the ``negafibonacci'' number \cite{wikipedia:negafibonacci}
% is used. Formally the Fibonacci number $F_n$ with integer
% index~$n$, $n\in\mathbb{Z}$ and
% $n\in[\num{-2147483647},\num{2147483647}]$ that is returned by macro
% \cs{fibnum} with numerical argument $n$ is defined the following way:
% \begin{gather}
%   \label{eq:def}
%   F_n =
%   \begin{cases}
%     0 & \text{for $n=0$}\\
%     1 & \text{for $n=1$}\\
%     F_{n-1} + F_{n-2} & \text{for $n>1$}\\
%     (-1)^{n+1}F_n & \text{for $n<0$}
%   \end{cases}
% \end{gather}
% Examples:
% \begin{quote}
%   \makeatletter
%   \def\x#1{\cs{fibnum}\{#1\}&
%     \edef\X{\fibnum{#1}}\edef\Y{\expandafter\ltx@car\X\@nil}^^A
%     \if-\Y
%       \edef\X{\expandafter\ltx@cdr\X\@nil}^^A
%       \noindent
%       \llap{-}\X
%     \else
%       \X
%     \fi
%     \tabularnewline
%   }
%   \def\y{\multicolumn{1}{@{}c@{}}{$\vdots$}\tabularnewline}
%   \DeclareUrlCommand\UrlNum{^^A
%     \urlstyle{tt}^^A
%     \def\UrlBreaks{\do\0\do\1\do\2\do\3\do\4\do\5\do\6\do\7\do\8\do\9}^^A
%   }
%   \begin{tabularx}{\dimexpr\linewidth+5.7pt\relax}{@{}>{\ttfamily}l@{ $\rightarrow$ \hphantom{\ttfamily-}}>{\ttfamily}X@{}}
%     \x{-6}
%     \x{-5}
%     \x{-4}
%     \x{-3}
%     \x{-2}
%     \x{-1}
%     \x{0}
%     \x{1}
%     \x{2}
%     \x{3}
%     \x{4}
%     \x{5}
%     \x{6}
%     \y
%     \x{10}
%     \y
%     \x{46}
%     \y
%     \cs{fibnum}\{100\} & 354224848179261915075
%     \tabularnewline
%     \y
%     \cs{fibnum}\{200\} & 280571172992510140037611932413038677189525
%     \tabularnewline
%     \y
%     \cs{fibnum}\{1000\} &
%       \raggedright
%       \UrlNum{^^A
%         434665576869374564356885276750406258025646^^A
%         605173717804024817290895365554179490518904^^A
%         038798400792551692959225930803226347752096^^A
%         896232398733224711616429964409065331879382^^A
%         98969649928516003704476137795166849228875^^A
%       }
%     \tabularnewline
%   \end{tabularx}\kern-5.7pt\mbox{}
% \end{quote}
%
% \begin{declcs}{fibnumPreCalc} \M{index}
% \end{declcs}
% The package already provides precalculated Fibonacci numbers up to
% index~46. That means that calculations are not necessary for
% Fibonacci numbers that fit into the range of \hologo{TeX}
% numbers. Because macro \cs{fibnum} is expandable, it cannot
% store calculated Fibonacci numbers for later use. Macro definitions
% are forbidden in expandable contexts. If larger Fibonacci numbers
% are used more than once, than the compilation time can be shortened
% by calculating and storing the Fibonacci numbers beforehand.
% The argument \meta{index} is a \hologo{TeX} number and macro
% \cs{fibnumPreCalc} ensures that the Fibonacci numbers
% $F_0$ up to $F_{\lvert\text{\meta{index}}\rvert}$ that are not
% already known are calculated
% and stored in internal macros. Internally only non-negative
% Fibonacci numbers are stored. If \meta{index} is negative, then
% the needed positive Fibonacci numbers are calculated and stored.
% Example:
% \begin{quote}
%   \def\x#1{\begingroup\itshape\texttt{\%} #1\endgroup}
%   |\fibnumPreCalc{50}|\\
%   \x{calculates and stores the values for indexes 47..50.}\\
%   \x{(Values for 0..46 are already stored by the package.)}\\
%   |\fibnum{49}| \x{uses the stored value}\\
%   |\fibnum{51}|
%   \x{only calculates $F_{51}$ from stored values $F_{49}$ and $F_{50}$}\\
%   |\fibnumPreCalc{100}|\\
%   \x{calculates and stores the values for indexes 51..100}\\
%   |\fibnum{100}| \x{uses the stored value for $F_{100}$}\\
%   |\fibnum{-100}|
%   \x{uses the stored value for $F_{100}$}\\
%   \x{$F_{-100}=-F_{100}$ according to equation \eqref{eq:def}.}
% \end{quote}
%
% \StopEventually{
% }
%
% \section{Implementation}
%
% \subsection{Identification}
%
%    \begin{macrocode}
%<*package>
%    \end{macrocode}
%    Reload check, especially if the package is not used with \LaTeX.
%    \begin{macrocode}
\begingroup\catcode61\catcode48\catcode32=10\relax%
  \catcode13=5 % ^^M
  \endlinechar=13 %
  \catcode35=6 % #
  \catcode39=12 % '
  \catcode44=12 % ,
  \catcode45=12 % -
  \catcode46=12 % .
  \catcode58=12 % :
  \catcode64=11 % @
  \catcode123=1 % {
  \catcode125=2 % }
  \expandafter\let\expandafter\x\csname ver@fibnum.sty\endcsname
  \ifx\x\relax % plain-TeX, first loading
  \else
    \def\empty{}%
    \ifx\x\empty % LaTeX, first loading,
      % variable is initialized, but \ProvidesPackage not yet seen
    \else
      \expandafter\ifx\csname PackageInfo\endcsname\relax
        \def\x#1#2{%
          \immediate\write-1{Package #1 Info: #2.}%
        }%
      \else
        \def\x#1#2{\PackageInfo{#1}{#2, stopped}}%
      \fi
      \x{fibnum}{The package is already loaded}%
      \aftergroup\endinput
    \fi
  \fi
\endgroup%
%    \end{macrocode}
%    Package identification:
%    \begin{macrocode}
\begingroup\catcode61\catcode48\catcode32=10\relax%
  \catcode13=5 % ^^M
  \endlinechar=13 %
  \catcode35=6 % #
  \catcode39=12 % '
  \catcode40=12 % (
  \catcode41=12 % )
  \catcode44=12 % ,
  \catcode45=12 % -
  \catcode46=12 % .
  \catcode47=12 % /
  \catcode58=12 % :
  \catcode64=11 % @
  \catcode91=12 % [
  \catcode93=12 % ]
  \catcode123=1 % {
  \catcode125=2 % }
  \expandafter\ifx\csname ProvidesPackage\endcsname\relax
    \def\x#1#2#3[#4]{\endgroup
      \immediate\write-1{Package: #3 #4}%
      \xdef#1{#4}%
    }%
  \else
    \def\x#1#2[#3]{\endgroup
      #2[{#3}]%
      \ifx#1\@undefined
        \xdef#1{#3}%
      \fi
      \ifx#1\relax
        \xdef#1{#3}%
      \fi
    }%
  \fi
\expandafter\x\csname ver@fibnum.sty\endcsname
\ProvidesPackage{fibnum}%
  [2016/05/16 v1.1 Fibonacci numbers (HO)]%
%    \end{macrocode}
%
%    \begin{macrocode}
\begingroup\catcode61\catcode48\catcode32=10\relax%
  \catcode13=5 % ^^M
  \endlinechar=13 %
  \catcode123=1 % {
  \catcode125=2 % }
  \catcode64=11 % @
  \def\x{\endgroup
    \expandafter\edef\csname FibNum@AtEnd\endcsname{%
      \endlinechar=\the\endlinechar\relax
      \catcode13=\the\catcode13\relax
      \catcode32=\the\catcode32\relax
      \catcode35=\the\catcode35\relax
      \catcode61=\the\catcode61\relax
      \catcode64=\the\catcode64\relax
      \catcode123=\the\catcode123\relax
      \catcode125=\the\catcode125\relax
    }%
  }%
\x\catcode61\catcode48\catcode32=10\relax%
\catcode13=5 % ^^M
\endlinechar=13 %
\catcode35=6 % #
\catcode64=11 % @
\catcode123=1 % {
\catcode125=2 % }
\def\TMP@EnsureCode#1#2{%
  \edef\FibNum@AtEnd{%
    \FibNum@AtEnd
    \catcode#1=\the\catcode#1\relax
  }%
  \catcode#1=#2\relax
}
\TMP@EnsureCode{33}{12}% !
%\TMP@EnsureCode{36}{3}% $
%\TMP@EnsureCode{38}{4}% &
\TMP@EnsureCode{40}{12}% (
\TMP@EnsureCode{41}{12}% )
\TMP@EnsureCode{45}{12}% -
\TMP@EnsureCode{46}{12}% .
\TMP@EnsureCode{47}{12}% /
\TMP@EnsureCode{58}{12}% :
\TMP@EnsureCode{60}{12}% <
\TMP@EnsureCode{62}{12}% >
\TMP@EnsureCode{91}{12}% [
%\TMP@EnsureCode{96}{12}% `
\TMP@EnsureCode{93}{12}% ]
%\TMP@EnsureCode{94}{12}% ^ (superscript) (!)
%\TMP@EnsureCode{124}{12}% |
\edef\FibNum@AtEnd{\FibNum@AtEnd\noexpand\endinput}
%    \end{macrocode}
%
% \subsection{Package resources}
%
%    \begin{macrocode}
\begingroup\expandafter\expandafter\expandafter\endgroup
\expandafter\ifx\csname RequirePackage\endcsname\relax
  \def\TMP@RequirePackage#1[#2]{%
    \begingroup\expandafter\expandafter\expandafter\endgroup
    \expandafter\ifx\csname ver@#1.sty\endcsname\relax
      \input #1.sty\relax
    \fi
  }%
  \TMP@RequirePackage{ltxcmds}[2011/04/18]%
  \TMP@RequirePackage{intcalc}[2007/09/27]%
  \TMP@RequirePackage{bigintcalc}[2007/11/11]%
\else
  \RequirePackage{ltxcmds}[2011/04/18]%
  \RequirePackage{intcalc}[2007/09/27]%
  \RequirePackage{bigintcalc}[2007/11/11]%
\fi
%    \end{macrocode}
%
% \subsection{Setup precalculated values}
%
%    \begin{macrocode}
\def\FibNum@temp#1{%
  \expandafter\def\csname FibNum@#1\endcsname
}
\catcode46=9 % dots are ignored
\FibNum@temp{0}{0}
\FibNum@temp{1}{1}
\FibNum@temp{2}{1}
\FibNum@temp{3}{2}
\FibNum@temp{4}{3}
\FibNum@temp{5}{5}
\FibNum@temp{6}{8}
\FibNum@temp{7}{13}
\FibNum@temp{8}{21}
\FibNum@temp{9}{34}
\FibNum@temp{10}{55}
\FibNum@temp{11}{89}
\FibNum@temp{12}{144}
\FibNum@temp{13}{233}
\FibNum@temp{14}{377}
\FibNum@temp{15}{610}
\FibNum@temp{16}{987}
\FibNum@temp{17}{1.597}
\FibNum@temp{18}{2.584}
\FibNum@temp{19}{4.181}
\FibNum@temp{20}{6.765}
\FibNum@temp{21}{10.946}
\FibNum@temp{22}{17.711}
\FibNum@temp{23}{28.657}
\FibNum@temp{24}{46.368}
\FibNum@temp{25}{75.025}
\FibNum@temp{26}{121.393}
\FibNum@temp{27}{196.418}
\FibNum@temp{28}{317.811}
\FibNum@temp{29}{514.229}
\FibNum@temp{30}{832.040}
\FibNum@temp{31}{1.346.269}
\FibNum@temp{32}{2.178.309}
\FibNum@temp{33}{3.524.578}
\FibNum@temp{34}{5.702.887}
\FibNum@temp{35}{9.227.465}
\FibNum@temp{36}{14.930.352}
\FibNum@temp{37}{24.157.817}
\FibNum@temp{38}{39.088.169}
\FibNum@temp{39}{63.245.986}
\FibNum@temp{40}{102.334.155}
\FibNum@temp{41}{165.580.141}
\FibNum@temp{42}{267.914.296}
\FibNum@temp{43}{433.494.437}
\FibNum@temp{44}{701.408.733}
\FibNum@temp{45}{1.134.903.170}
\FibNum@temp{46}{1.836.311.903}
%    \end{macrocode}
%    \begin{macro}{\FibNum@max}
%    \begin{macrocode}
\def\FibNum@max{46}
%    \end{macrocode}
%    \end{macro}
%
% \subsection{Macros for precalculating values}
%
%    \begin{macro}{\fibnumPreCalc}
%    \begin{macrocode}
\def\fibnumPreCalc#1{%
  \expandafter\expandafter\expandafter
  \FibNum@PreCalc\intcalcNum{#1}/%
}
%    \end{macrocode}
%    \end{macro}
%    \begin{macro}{\FibNum@PreCalc}
%    \begin{macrocode}
\def\FibNum@PreCalc#1/{%
  \ifnum#1<\ltx@zero
    \expandafter\FibNum@PreCalc\ltx@gobble#1/%
  \else
    \ifnum#1>\FibNum@max
      \begingroup
        \ltx@LocDimenA=#1sp\relax
        \countdef\FibNum@i=255\relax
        \FibNum@i=\FibNum@max\relax
        \edef\FibNum@temp{%
          \csname FibNum@\the\FibNum@i\endcsname/%
        }%
        \advance\FibNum@i by -1\relax
        \edef\FibNum@temp{%
          \FibNum@temp
          \csname FibNum@\the\FibNum@i\endcsname
        }%
        \advance\FibNum@i\ltx@two
        \iftrue
          \expandafter\FibNum@PreCalcAux\FibNum@temp
        \fi
      \endgroup
    \fi
  \fi
}
%    \end{macrocode}
%    \end{macro}
%    \begin{macro}{\FibNum@PreCalcAux}
%    \begin{macrocode}
\def\FibNum@PreCalcAux#1/#2\fi{%
  \fi
  \edef\FibNum@temp{\BigIntCalcAdd#1!#2!}%
  \global\expandafter
  \let\csname FibNum@\the\FibNum@i\endcsname\FibNum@temp
  \ifnum\FibNum@i=\ltx@LocDimenA
    \xdef\FibNum@max{\the\FibNum@i}%
  \else
    \advance\FibNum@i\ltx@one
    \expandafter\FibNum@PreCalcAux\FibNum@temp/#1%
  \fi
}
%    \end{macrocode}
%    \end{macro}
%
% \subsection{Expandable calculations}
%
%    \begin{macro}{\fibnum}
%    \begin{macrocode}
\def\fibnum#1{%
  \romannumeral
  \expandafter\expandafter\expandafter\FibNum@Do\intcalcNum{#1}/%
}
%    \end{macrocode}
%    \end{macro}
%    \begin{macro}{\FibNum@Do}
%    \begin{macrocode}
\def\FibNum@Do#1/{%
  \ifnum#1<\ltx@zero
    \FibNum@ReturnAfterElseFiFi{%
      \ifodd#1 %
        \expandafter\expandafter\expandafter\ltx@zero
      \else
        \expandafter\expandafter\expandafter\ltx@zero
        \expandafter\expandafter\expandafter-%
      \fi
      \romannumeral
      \expandafter\FibNum@Do\ltx@gobble#1/%
    }%
  \else
    \ifnum\FibNum@max<#1 %
      \ltx@ReturnAfterElseFi{%
        \expandafter
        \FibNum@ExpCalc\number\expandafter\IntCalcInc\FibNum@max!%
        \expandafter\expandafter\expandafter/%
        \csname FibNum@\FibNum@max
        \expandafter\expandafter\expandafter\endcsname
        \expandafter\expandafter\expandafter/%
        \csname FibNum@\expandafter\IntCalcDec\FibNum@max!%
        \endcsname/%
        #1%
      }%
    \else
      \expandafter\expandafter\expandafter\ltx@zero
      \csname FibNum@#1\expandafter\expandafter\expandafter\endcsname
    \fi
  \fi
}
%    \end{macrocode}
%    \end{macro}
%    \begin{macro}{\FibNum@ReturnAfterElseFiFi}
%    \begin{macrocode}
\def\FibNum@ReturnAfterElseFiFi#1\else#2\fi\fi{\fi#1}
%    \end{macrocode}
%    \end{macro}
%    \begin{macro}{\FibNum@ExpCalc}
%    \begin{macrocode}
\def\FibNum@ExpCalc#1/#2/#3/#4\fi{%
  \fi
  \ifnum#1=#4 %
    \ltx@ReturnAfterElseFi{%
      \expandafter\expandafter\expandafter\ltx@zero
      \BigIntCalcAdd#2!#3!%
    }%
  \else
    \expandafter\FibNum@ExpCalc
    \number\IntCalcInc#1!%
    \expandafter\expandafter\expandafter/%
    \BigIntCalcAdd#2!#3!/%
    #2/#4%
  \fi
}
%    \end{macrocode}
%    \end{macro}
%
%    \begin{macrocode}
\FibNum@AtEnd%
%</package>
%    \end{macrocode}
%
% \section{Test}
%
% \subsection{Catcode checks for loading}
%
%    \begin{macrocode}
%<*test1>
%    \end{macrocode}
%    \begin{macrocode}
\catcode`\{=1 %
\catcode`\}=2 %
\catcode`\#=6 %
\catcode`\@=11 %
\expandafter\ifx\csname count@\endcsname\relax
  \countdef\count@=255 %
\fi
\expandafter\ifx\csname @gobble\endcsname\relax
  \long\def\@gobble#1{}%
\fi
\expandafter\ifx\csname @firstofone\endcsname\relax
  \long\def\@firstofone#1{#1}%
\fi
\expandafter\ifx\csname loop\endcsname\relax
  \expandafter\@firstofone
\else
  \expandafter\@gobble
\fi
{%
  \def\loop#1\repeat{%
    \def\body{#1}%
    \iterate
  }%
  \def\iterate{%
    \body
      \let\next\iterate
    \else
      \let\next\relax
    \fi
    \next
  }%
  \let\repeat=\fi
}%
\def\RestoreCatcodes{}
\count@=0 %
\loop
  \edef\RestoreCatcodes{%
    \RestoreCatcodes
    \catcode\the\count@=\the\catcode\count@\relax
  }%
\ifnum\count@<255 %
  \advance\count@ 1 %
\repeat

\def\RangeCatcodeInvalid#1#2{%
  \count@=#1\relax
  \loop
    \catcode\count@=15 %
  \ifnum\count@<#2\relax
    \advance\count@ 1 %
  \repeat
}
\def\RangeCatcodeCheck#1#2#3{%
  \count@=#1\relax
  \loop
    \ifnum#3=\catcode\count@
    \else
      \errmessage{%
        Character \the\count@\space
        with wrong catcode \the\catcode\count@\space
        instead of \number#3%
      }%
    \fi
  \ifnum\count@<#2\relax
    \advance\count@ 1 %
  \repeat
}
\def\space{ }
\expandafter\ifx\csname LoadCommand\endcsname\relax
  \def\LoadCommand{\input fibnum.sty\relax}%
\fi
\def\Test{%
  \RangeCatcodeInvalid{0}{47}%
  \RangeCatcodeInvalid{58}{64}%
  \RangeCatcodeInvalid{91}{96}%
  \RangeCatcodeInvalid{123}{255}%
  \catcode`\@=12 %
  \catcode`\\=0 %
  \catcode`\%=14 %
  \LoadCommand
  \RangeCatcodeCheck{0}{36}{15}%
  \RangeCatcodeCheck{37}{37}{14}%
  \RangeCatcodeCheck{38}{47}{15}%
  \RangeCatcodeCheck{48}{57}{12}%
  \RangeCatcodeCheck{58}{63}{15}%
  \RangeCatcodeCheck{64}{64}{12}%
  \RangeCatcodeCheck{65}{90}{11}%
  \RangeCatcodeCheck{91}{91}{15}%
  \RangeCatcodeCheck{92}{92}{0}%
  \RangeCatcodeCheck{93}{96}{15}%
  \RangeCatcodeCheck{97}{122}{11}%
  \RangeCatcodeCheck{123}{255}{15}%
  \RestoreCatcodes
}
\Test
\csname @@end\endcsname
\end
%    \end{macrocode}
%    \begin{macrocode}
%</test1>
%    \end{macrocode}
%
% \subsection{Test calculations}
%
%    \begin{macrocode}
%<*test-calc>
\catcode`\{=1 %
\catcode`\}=2 %
\catcode`\#=6 %
\catcode`\@=11 %
\begingroup\expandafter\expandafter\expandafter\endgroup
\expandafter\ifx\csname RequirePackage\endcsname\relax
  \input fibnum.sty\relax
\else
  \RequirePackage{fibnum}[2016/05/16]%
\fi
\def\TestSet{%
  \test{0}{0}%
  \test{1}{1}%
  \test{2}{1}%
  \test{3}{2}%
  \test{4}{3}%
  \test{5}{5}%
  \test{6}{8}%
  \test{7}{13}%
  \test{8}{21}%
  \test{9}{34}%
  \test{10}{55}%
  \test{11}{89}%
  \test{12}{144}%
  \test{13}{233}%
  \test{14}{377}%
  \test{15}{610}%
  \test{16}{987}%
  \test{17}{1597}%
  \test{18}{2584}%
  \test{19}{4181}%
  \test{20}{6765}%
  \test{21}{10946}%
  \test{22}{17711}%
  \test{23}{28657}%
  \test{24}{46368}%
  \test{25}{75025}%
  \test{26}{121393}%
  \test{27}{196418}%
  \test{28}{317811}%
  \test{29}{514229}%
  \test{30}{832040}%
  \test{31}{1346269}%
  \test{32}{2178309}%
  \test{33}{3524578}%
  \test{34}{5702887}%
  \test{35}{9227465}%
  \test{36}{14930352}%
  \test{37}{24157817}%
  \test{38}{39088169}%
  \test{39}{63245986}%
  \test{40}{102334155}%
  \test{41}{165580141}%
  \test{42}{267914296}%
  \test{43}{433494437}%
  \test{44}{701408733}%
  \test{45}{1134903170}%
  \test{46}{1836311903}%
  \test{47}{2971215073}%
  \test{48}{4807526976}%
  \test{49}{7778742049}%
  \test{50}{12586269025}%
  \test{51}{20365011074}%
  \test{52}{32951280099}%
  \test{53}{53316291173}%
  \test{54}{86267571272}%
  \test{55}{139583862445}%
  \test{56}{225851433717}%
  \test{57}{365435296162}%
  \test{58}{591286729879}%
  \test{59}{956722026041}%
  \test{60}{1548008755920}%
  \test{61}{2504730781961}%
  \test{62}{4052739537881}%
  \test{63}{6557470319842}%
  \test{64}{10610209857723}%
  \test{65}{17167680177565}%
  \test{66}{27777890035288}%
  \test{67}{44945570212853}%
  \test{68}{72723460248141}%
  \test{69}{117669030460994}%
  \test{70}{190392490709135}%
  \test{71}{308061521170129}%
  \test{72}{498454011879264}%
  \test{73}{806515533049393}%
}
\def\msg#{\immediate\write16}
\def\test#1#2{%
  \TestAux{#1}{#2}%
  \ifnum#1=0 %
  \else
    \ifodd#1 %
      \TestAux{-#1}{#2}%
    \else
      \TestAux{-#1}{-#2}%
    \fi
  \fi
}
\def\TestAux#1#2{%
  \def\Expected{#2}%
  \expandafter\expandafter\expandafter\def
  \expandafter\expandafter\expandafter\Result
  \expandafter\expandafter\expandafter{%
    \fibnum{#1}%
  }%
  \ltx@onelevel@sanitize\Result
  \ifx\Result\Expected
    \msg{* #1: ok.}%
  \else
    \msg{! fib(#1) = #2}%
    \errmessage{fib(#1) <> \Result}%
  \fi
}
\TestSet
\setbox0=\hbox{%
  \msg{* PreCalc{73}}%
  \fibnumPreCalc{73}%
}
\ifdim\wd0=0pt
\else
  \errmessage{Unwanted stuff in PreCalc}%
\fi
\TestSet
\csname @@end\endcsname\end
%</test-calc>
%    \end{macrocode}
%
% \section{Installation}
%
% \subsection{Download}
%
% \paragraph{Package.} This package is available on
% CTAN\footnote{\url{http://ctan.org/pkg/fibnum}}:
% \begin{description}
% \item[\CTAN{macros/latex/contrib/oberdiek/fibnum.dtx}] The source file.
% \item[\CTAN{macros/latex/contrib/oberdiek/fibnum.pdf}] Documentation.
% \end{description}
%
%
% \paragraph{Bundle.} All the packages of the bundle `oberdiek'
% are also available in a TDS compliant ZIP archive. There
% the packages are already unpacked and the documentation files
% are generated. The files and directories obey the TDS standard.
% \begin{description}
% \item[\CTAN{install/macros/latex/contrib/oberdiek.tds.zip}]
% \end{description}
% \emph{TDS} refers to the standard ``A Directory Structure
% for \TeX\ Files'' (\CTAN{tds/tds.pdf}). Directories
% with \xfile{texmf} in their name are usually organized this way.
%
% \subsection{Bundle installation}
%
% \paragraph{Unpacking.} Unpack the \xfile{oberdiek.tds.zip} in the
% TDS tree (also known as \xfile{texmf} tree) of your choice.
% Example (linux):
% \begin{quote}
%   |unzip oberdiek.tds.zip -d ~/texmf|
% \end{quote}
%
% \paragraph{Script installation.}
% Check the directory \xfile{TDS:scripts/oberdiek/} for
% scripts that need further installation steps.
% Package \xpackage{attachfile2} comes with the Perl script
% \xfile{pdfatfi.pl} that should be installed in such a way
% that it can be called as \texttt{pdfatfi}.
% Example (linux):
% \begin{quote}
%   |chmod +x scripts/oberdiek/pdfatfi.pl|\\
%   |cp scripts/oberdiek/pdfatfi.pl /usr/local/bin/|
% \end{quote}
%
% \subsection{Package installation}
%
% \paragraph{Unpacking.} The \xfile{.dtx} file is a self-extracting
% \docstrip\ archive. The files are extracted by running the
% \xfile{.dtx} through \plainTeX:
% \begin{quote}
%   \verb|tex fibnum.dtx|
% \end{quote}
%
% \paragraph{TDS.} Now the different files must be moved into
% the different directories in your installation TDS tree
% (also known as \xfile{texmf} tree):
% \begin{quote}
% \def\t{^^A
% \begin{tabular}{@{}>{\ttfamily}l@{ $\rightarrow$ }>{\ttfamily}l@{}}
%   fibnum.sty & tex/generic/oberdiek/fibnum.sty\\
%   fibnum.pdf & doc/latex/oberdiek/fibnum.pdf\\
%   test/fibnum-test1.tex & doc/latex/oberdiek/test/fibnum-test1.tex\\
%   test/fibnum-test-calc.tex & doc/latex/oberdiek/test/fibnum-test-calc.tex\\
%   fibnum.dtx & source/latex/oberdiek/fibnum.dtx\\
% \end{tabular}^^A
% }^^A
% \sbox0{\t}^^A
% \ifdim\wd0>\linewidth
%   \begingroup
%     \advance\linewidth by\leftmargin
%     \advance\linewidth by\rightmargin
%   \edef\x{\endgroup
%     \def\noexpand\lw{\the\linewidth}^^A
%   }\x
%   \def\lwbox{^^A
%     \leavevmode
%     \hbox to \linewidth{^^A
%       \kern-\leftmargin\relax
%       \hss
%       \usebox0
%       \hss
%       \kern-\rightmargin\relax
%     }^^A
%   }^^A
%   \ifdim\wd0>\lw
%     \sbox0{\small\t}^^A
%     \ifdim\wd0>\linewidth
%       \ifdim\wd0>\lw
%         \sbox0{\footnotesize\t}^^A
%         \ifdim\wd0>\linewidth
%           \ifdim\wd0>\lw
%             \sbox0{\scriptsize\t}^^A
%             \ifdim\wd0>\linewidth
%               \ifdim\wd0>\lw
%                 \sbox0{\tiny\t}^^A
%                 \ifdim\wd0>\linewidth
%                   \lwbox
%                 \else
%                   \usebox0
%                 \fi
%               \else
%                 \lwbox
%               \fi
%             \else
%               \usebox0
%             \fi
%           \else
%             \lwbox
%           \fi
%         \else
%           \usebox0
%         \fi
%       \else
%         \lwbox
%       \fi
%     \else
%       \usebox0
%     \fi
%   \else
%     \lwbox
%   \fi
% \else
%   \usebox0
% \fi
% \end{quote}
% If you have a \xfile{docstrip.cfg} that configures and enables \docstrip's
% TDS installing feature, then some files can already be in the right
% place, see the documentation of \docstrip.
%
% \subsection{Refresh file name databases}
%
% If your \TeX~distribution
% (\teTeX, \mikTeX, \dots) relies on file name databases, you must refresh
% these. For example, \teTeX\ users run \verb|texhash| or
% \verb|mktexlsr|.
%
% \subsection{Some details for the interested}
%
% \paragraph{Attached source.}
%
% The PDF documentation on CTAN also includes the
% \xfile{.dtx} source file. It can be extracted by
% AcrobatReader 6 or higher. Another option is \textsf{pdftk},
% e.g. unpack the file into the current directory:
% \begin{quote}
%   \verb|pdftk fibnum.pdf unpack_files output .|
% \end{quote}
%
% \paragraph{Unpacking with \LaTeX.}
% The \xfile{.dtx} chooses its action depending on the format:
% \begin{description}
% \item[\plainTeX:] Run \docstrip\ and extract the files.
% \item[\LaTeX:] Generate the documentation.
% \end{description}
% If you insist on using \LaTeX\ for \docstrip\ (really,
% \docstrip\ does not need \LaTeX), then inform the autodetect routine
% about your intention:
% \begin{quote}
%   \verb|latex \let\install=y% \iffalse meta-comment
%
% File: fibnum.dtx
% Version: 2016/05/16 v1.1
% Info: Fibonacci numbers
%
% Copyright (C) 2012 by
%    Heiko Oberdiek <heiko.oberdiek at googlemail.com>
%    2016
%    https://github.com/ho-tex/oberdiek/issues
%
% This work may be distributed and/or modified under the
% conditions of the LaTeX Project Public License, either
% version 1.3c of this license or (at your option) any later
% version. This version of this license is in
%    http://www.latex-project.org/lppl/lppl-1-3c.txt
% and the latest version of this license is in
%    http://www.latex-project.org/lppl.txt
% and version 1.3 or later is part of all distributions of
% LaTeX version 2005/12/01 or later.
%
% This work has the LPPL maintenance status "maintained".
%
% This Current Maintainer of this work is Heiko Oberdiek.
%
% The Base Interpreter refers to any `TeX-Format',
% because some files are installed in TDS:tex/generic//.
%
% This work consists of the main source file fibnum.dtx
% and the derived files
%    fibnum.sty, fibnum.pdf, fibnum.ins, fibnum.drv, fibnum.bib,
%    fibnum-test1.tex, fibnum-test-calc.tex.
%
% Distribution:
%    CTAN:macros/latex/contrib/oberdiek/fibnum.dtx
%    CTAN:macros/latex/contrib/oberdiek/fibnum.pdf
%
% Unpacking:
%    (a) If fibnum.ins is present:
%           tex fibnum.ins
%    (b) Without fibnum.ins:
%           tex fibnum.dtx
%    (c) If you insist on using LaTeX
%           latex \let\install=y\input{fibnum.dtx}
%        (quote the arguments according to the demands of your shell)
%
% Documentation:
%    (a) If fibnum.drv is present:
%           latex fibnum.drv
%    (b) Without fibnum.drv:
%           latex fibnum.dtx; ...
%    The class ltxdoc loads the configuration file ltxdoc.cfg
%    if available. Here you can specify further options, e.g.
%    use A4 as paper format:
%       \PassOptionsToClass{a4paper}{article}
%
%    Programm calls to get the documentation (example):
%       pdflatex fibnum.dtx
%       bibtex fibnum.aux
%       makeindex -s gind.ist fibnum.idx
%       pdflatex fibnum.dtx
%       makeindex -s gind.ist fibnum.idx
%       pdflatex fibnum.dtx
%
% Installation:
%    TDS:tex/generic/oberdiek/fibnum.sty
%    TDS:doc/latex/oberdiek/fibnum.pdf
%    TDS:doc/latex/oberdiek/test/fibnum-test1.tex
%    TDS:doc/latex/oberdiek/test/fibnum-test-calc.tex
%    TDS:source/latex/oberdiek/fibnum.dtx
%
%<*ignore>
\begingroup
  \catcode123=1 %
  \catcode125=2 %
  \def\x{LaTeX2e}%
\expandafter\endgroup
\ifcase 0\ifx\install y1\fi\expandafter
         \ifx\csname processbatchFile\endcsname\relax\else1\fi
         \ifx\fmtname\x\else 1\fi\relax
\else\csname fi\endcsname
%</ignore>
%<*install>
\input docstrip.tex
\Msg{************************************************************************}
\Msg{* Installation}
\Msg{* Package: fibnum 2016/05/16 v1.1 Fibonacci numbers (HO)}
\Msg{************************************************************************}

\keepsilent
\askforoverwritefalse

\let\MetaPrefix\relax
\preamble

This is a generated file.

Project: fibnum
Version: 2016/05/16 v1.1

Copyright (C) 2012 by
   Heiko Oberdiek <heiko.oberdiek at googlemail.com>

This work may be distributed and/or modified under the
conditions of the LaTeX Project Public License, either
version 1.3c of this license or (at your option) any later
version. This version of this license is in
   http://www.latex-project.org/lppl/lppl-1-3c.txt
and the latest version of this license is in
   http://www.latex-project.org/lppl.txt
and version 1.3 or later is part of all distributions of
LaTeX version 2005/12/01 or later.

This work has the LPPL maintenance status "maintained".

This Current Maintainer of this work is Heiko Oberdiek.

The Base Interpreter refers to any `TeX-Format',
because some files are installed in TDS:tex/generic//.

This work consists of the main source file fibnum.dtx
and the derived files
   fibnum.sty, fibnum.pdf, fibnum.ins, fibnum.drv, fibnum.bib,
   fibnum-test1.tex, fibnum-test-calc.tex.

\endpreamble
\let\MetaPrefix\DoubleperCent

\generate{%
  \file{fibnum.ins}{\from{fibnum.dtx}{install}}%
  \file{fibnum.drv}{\from{fibnum.dtx}{driver}}%
  \nopreamble
  \nopostamble
  \file{fibnum.bib}{\from{fibnum.dtx}{bib}}%
  \usepreamble\defaultpreamble
  \usepostamble\defaultpostamble
  \usedir{tex/generic/oberdiek}%
  \file{fibnum.sty}{\from{fibnum.dtx}{package}}%
  \usedir{doc/latex/oberdiek/test}%
  \file{fibnum-test1.tex}{\from{fibnum.dtx}{test1}}%
  \file{fibnum-test-calc.tex}{\from{fibnum.dtx}{test-calc}}%
}

\catcode32=13\relax% active space
\let =\space%
\Msg{************************************************************************}
\Msg{*}
\Msg{* To finish the installation you have to move the following}
\Msg{* file into a directory searched by TeX:}
\Msg{*}
\Msg{*     fibnum.sty}
\Msg{*}
\Msg{* To produce the documentation run the file `fibnum.drv'}
\Msg{* through LaTeX.}
\Msg{*}
\Msg{* Happy TeXing!}
\Msg{*}
\Msg{************************************************************************}

\endbatchfile
%</install>
%<*bib>
@online{texhax:abraham,
  author={Abraham, Jan},
  title={[texhax] Beginner in TEX MACRO to compute functions},
  date={2012-04-07},
  url={http://tug.org/pipermail/texhax/2012-April/019146.html},
  urldate={2012-04-08},
}
@article{knuth:negafibonacci,
  author={Knuth, Donald E.},
  title={Negafibonacci Numbers and the Hyperbolic Plane},
  date={2008-12-11},
  url={http://research.allacademic.com/meta/p206842_index.html},
}
@online{wikipedia:negafibonacci,
  author={{Wikipedia contributors}},
  organization={{Wikipedia, The Free Encyclopedia}},
  title={Fibonacci numbers},
  language={langenglish},
  version={486266088},
  date={2012-04-08},
  url={http://en.wikipedia.org/w/index.php?title=Fibonacci_number&oldid=486266088},
  urldate={2012-04-08},
}
%</bib>
%<*ignore>
\fi
%</ignore>
%<*driver>
\NeedsTeXFormat{LaTeX2e}
\ProvidesFile{fibnum.drv}%
  [2016/05/16 v1.1 Fibonacci numbers (HO)]%
\documentclass{ltxdoc}
\usepackage{amsmath,amsfonts}
\usepackage{siunitx}
\usepackage{array}
\usepackage{tabularx}
\usepackage{fibnum}[2016/05/16]
\usepackage{holtxdoc}[2011/11/22]
\usepackage{csquotes}
\usepackage[
  bibencoding=ascii,
  alldates=iso8601,
]{biblatex}[2011/11/13]
\bibliography{oberdiek-source}
\bibliography{fibnum}
\begin{document}
  \DocInput{fibnum.dtx}%
\end{document}
%</driver>
% \fi
%
%
% \CharacterTable
%  {Upper-case    \A\B\C\D\E\F\G\H\I\J\K\L\M\N\O\P\Q\R\S\T\U\V\W\X\Y\Z
%   Lower-case    \a\b\c\d\e\f\g\h\i\j\k\l\m\n\o\p\q\r\s\t\u\v\w\x\y\z
%   Digits        \0\1\2\3\4\5\6\7\8\9
%   Exclamation   \!     Double quote  \"     Hash (number) \#
%   Dollar        \$     Percent       \%     Ampersand     \&
%   Acute accent  \'     Left paren    \(     Right paren   \)
%   Asterisk      \*     Plus          \+     Comma         \,
%   Minus         \-     Point         \.     Solidus       \/
%   Colon         \:     Semicolon     \;     Less than     \<
%   Equals        \=     Greater than  \>     Question mark \?
%   Commercial at \@     Left bracket  \[     Backslash     \\
%   Right bracket \]     Circumflex    \^     Underscore    \_
%   Grave accent  \`     Left brace    \{     Vertical bar  \|
%   Right brace   \}     Tilde         \~}
%
% \GetFileInfo{fibnum.drv}
%
% \title{The \xpackage{fibnum} package}
% \date{2016/05/16 v1.1}
% \author{Heiko Oberdiek\thanks
% {Please report any issues at https://github.com/ho-tex/oberdiek/issues}\\
% \xemail{heiko.oberdiek at googlemail.com}}
%
% \maketitle
%
% \begin{abstract}
% The package \xpackage{fibnum} provides expandable fibonacci
% numbers for both \hologo{LaTeX} and \hologo{plainTeX}.
% \end{abstract}
%
% \tableofcontents
%
% \section{Documentation}
%
% In the mailing list \textsf{texhax} Jan Abraham asked the question,
% how to get Fibonacci numbers in \hologo{TeX} \cite{texhax:abraham}:
% \begin{quote}
% Write a Macro in \hologo{TeX} that compute the function |\fib{m}|
% All fibonacci numbers from 1 to $m$ ($m < 40$).
% \end{quote}
% This packages provides an expandable implementation for the
% calculation of these numbers for a much larger set of indexes.
% For practical reasons the index is restricted to the same limitations
% that apply for \hologo{TeX} integer numbers.
% The range of the Fibonacci numbers, however, are not limited
% by the algorithm. They are only restricted to memory limitations,
% if they are hit.
%
% The package is loaded as \hologo{LaTeX} package in \hologo{LaTeX}:
% \begin{quote}
%   |\usepackage{fibnum}|
% \end{quote}
% and as file in \hologo{plainTeX}:
% \begin{quote}
%   |\input fibnum.sty|
% \end{quote}
% The package does not know any options and it provides
% the macros \cs{fibnum} and \cs{fibnumPreCalc}.
%
% \begin{declcs}{fibnum} \M{index}
% \end{declcs}
% Macro \cs{fibnum} expects a \hologo{TeX} number as \meta{index}
% in the official \hologo{TeX} number range from $-(2^{31}-1)$ up to
% $2^{31}-1$. In exact two expansion steps the macro expands to
% the Fibnoacci number $F_{\text{\meta{index}}}$. In case of a negative
% \meta{index}, the ``negafibonacci'' number \cite{wikipedia:negafibonacci}
% is used. Formally the Fibonacci number $F_n$ with integer
% index~$n$, $n\in\mathbb{Z}$ and
% $n\in[\num{-2147483647},\num{2147483647}]$ that is returned by macro
% \cs{fibnum} with numerical argument $n$ is defined the following way:
% \begin{gather}
%   \label{eq:def}
%   F_n =
%   \begin{cases}
%     0 & \text{for $n=0$}\\
%     1 & \text{for $n=1$}\\
%     F_{n-1} + F_{n-2} & \text{for $n>1$}\\
%     (-1)^{n+1}F_n & \text{for $n<0$}
%   \end{cases}
% \end{gather}
% Examples:
% \begin{quote}
%   \makeatletter
%   \def\x#1{\cs{fibnum}\{#1\}&
%     \edef\X{\fibnum{#1}}\edef\Y{\expandafter\ltx@car\X\@nil}^^A
%     \if-\Y
%       \edef\X{\expandafter\ltx@cdr\X\@nil}^^A
%       \noindent
%       \llap{-}\X
%     \else
%       \X
%     \fi
%     \tabularnewline
%   }
%   \def\y{\multicolumn{1}{@{}c@{}}{$\vdots$}\tabularnewline}
%   \DeclareUrlCommand\UrlNum{^^A
%     \urlstyle{tt}^^A
%     \def\UrlBreaks{\do\0\do\1\do\2\do\3\do\4\do\5\do\6\do\7\do\8\do\9}^^A
%   }
%   \begin{tabularx}{\dimexpr\linewidth+5.7pt\relax}{@{}>{\ttfamily}l@{ $\rightarrow$ \hphantom{\ttfamily-}}>{\ttfamily}X@{}}
%     \x{-6}
%     \x{-5}
%     \x{-4}
%     \x{-3}
%     \x{-2}
%     \x{-1}
%     \x{0}
%     \x{1}
%     \x{2}
%     \x{3}
%     \x{4}
%     \x{5}
%     \x{6}
%     \y
%     \x{10}
%     \y
%     \x{46}
%     \y
%     \cs{fibnum}\{100\} & 354224848179261915075
%     \tabularnewline
%     \y
%     \cs{fibnum}\{200\} & 280571172992510140037611932413038677189525
%     \tabularnewline
%     \y
%     \cs{fibnum}\{1000\} &
%       \raggedright
%       \UrlNum{^^A
%         434665576869374564356885276750406258025646^^A
%         605173717804024817290895365554179490518904^^A
%         038798400792551692959225930803226347752096^^A
%         896232398733224711616429964409065331879382^^A
%         98969649928516003704476137795166849228875^^A
%       }
%     \tabularnewline
%   \end{tabularx}\kern-5.7pt\mbox{}
% \end{quote}
%
% \begin{declcs}{fibnumPreCalc} \M{index}
% \end{declcs}
% The package already provides precalculated Fibonacci numbers up to
% index~46. That means that calculations are not necessary for
% Fibonacci numbers that fit into the range of \hologo{TeX}
% numbers. Because macro \cs{fibnum} is expandable, it cannot
% store calculated Fibonacci numbers for later use. Macro definitions
% are forbidden in expandable contexts. If larger Fibonacci numbers
% are used more than once, than the compilation time can be shortened
% by calculating and storing the Fibonacci numbers beforehand.
% The argument \meta{index} is a \hologo{TeX} number and macro
% \cs{fibnumPreCalc} ensures that the Fibonacci numbers
% $F_0$ up to $F_{\lvert\text{\meta{index}}\rvert}$ that are not
% already known are calculated
% and stored in internal macros. Internally only non-negative
% Fibonacci numbers are stored. If \meta{index} is negative, then
% the needed positive Fibonacci numbers are calculated and stored.
% Example:
% \begin{quote}
%   \def\x#1{\begingroup\itshape\texttt{\%} #1\endgroup}
%   |\fibnumPreCalc{50}|\\
%   \x{calculates and stores the values for indexes 47..50.}\\
%   \x{(Values for 0..46 are already stored by the package.)}\\
%   |\fibnum{49}| \x{uses the stored value}\\
%   |\fibnum{51}|
%   \x{only calculates $F_{51}$ from stored values $F_{49}$ and $F_{50}$}\\
%   |\fibnumPreCalc{100}|\\
%   \x{calculates and stores the values for indexes 51..100}\\
%   |\fibnum{100}| \x{uses the stored value for $F_{100}$}\\
%   |\fibnum{-100}|
%   \x{uses the stored value for $F_{100}$}\\
%   \x{$F_{-100}=-F_{100}$ according to equation \eqref{eq:def}.}
% \end{quote}
%
% \StopEventually{
% }
%
% \section{Implementation}
%
% \subsection{Identification}
%
%    \begin{macrocode}
%<*package>
%    \end{macrocode}
%    Reload check, especially if the package is not used with \LaTeX.
%    \begin{macrocode}
\begingroup\catcode61\catcode48\catcode32=10\relax%
  \catcode13=5 % ^^M
  \endlinechar=13 %
  \catcode35=6 % #
  \catcode39=12 % '
  \catcode44=12 % ,
  \catcode45=12 % -
  \catcode46=12 % .
  \catcode58=12 % :
  \catcode64=11 % @
  \catcode123=1 % {
  \catcode125=2 % }
  \expandafter\let\expandafter\x\csname ver@fibnum.sty\endcsname
  \ifx\x\relax % plain-TeX, first loading
  \else
    \def\empty{}%
    \ifx\x\empty % LaTeX, first loading,
      % variable is initialized, but \ProvidesPackage not yet seen
    \else
      \expandafter\ifx\csname PackageInfo\endcsname\relax
        \def\x#1#2{%
          \immediate\write-1{Package #1 Info: #2.}%
        }%
      \else
        \def\x#1#2{\PackageInfo{#1}{#2, stopped}}%
      \fi
      \x{fibnum}{The package is already loaded}%
      \aftergroup\endinput
    \fi
  \fi
\endgroup%
%    \end{macrocode}
%    Package identification:
%    \begin{macrocode}
\begingroup\catcode61\catcode48\catcode32=10\relax%
  \catcode13=5 % ^^M
  \endlinechar=13 %
  \catcode35=6 % #
  \catcode39=12 % '
  \catcode40=12 % (
  \catcode41=12 % )
  \catcode44=12 % ,
  \catcode45=12 % -
  \catcode46=12 % .
  \catcode47=12 % /
  \catcode58=12 % :
  \catcode64=11 % @
  \catcode91=12 % [
  \catcode93=12 % ]
  \catcode123=1 % {
  \catcode125=2 % }
  \expandafter\ifx\csname ProvidesPackage\endcsname\relax
    \def\x#1#2#3[#4]{\endgroup
      \immediate\write-1{Package: #3 #4}%
      \xdef#1{#4}%
    }%
  \else
    \def\x#1#2[#3]{\endgroup
      #2[{#3}]%
      \ifx#1\@undefined
        \xdef#1{#3}%
      \fi
      \ifx#1\relax
        \xdef#1{#3}%
      \fi
    }%
  \fi
\expandafter\x\csname ver@fibnum.sty\endcsname
\ProvidesPackage{fibnum}%
  [2016/05/16 v1.1 Fibonacci numbers (HO)]%
%    \end{macrocode}
%
%    \begin{macrocode}
\begingroup\catcode61\catcode48\catcode32=10\relax%
  \catcode13=5 % ^^M
  \endlinechar=13 %
  \catcode123=1 % {
  \catcode125=2 % }
  \catcode64=11 % @
  \def\x{\endgroup
    \expandafter\edef\csname FibNum@AtEnd\endcsname{%
      \endlinechar=\the\endlinechar\relax
      \catcode13=\the\catcode13\relax
      \catcode32=\the\catcode32\relax
      \catcode35=\the\catcode35\relax
      \catcode61=\the\catcode61\relax
      \catcode64=\the\catcode64\relax
      \catcode123=\the\catcode123\relax
      \catcode125=\the\catcode125\relax
    }%
  }%
\x\catcode61\catcode48\catcode32=10\relax%
\catcode13=5 % ^^M
\endlinechar=13 %
\catcode35=6 % #
\catcode64=11 % @
\catcode123=1 % {
\catcode125=2 % }
\def\TMP@EnsureCode#1#2{%
  \edef\FibNum@AtEnd{%
    \FibNum@AtEnd
    \catcode#1=\the\catcode#1\relax
  }%
  \catcode#1=#2\relax
}
\TMP@EnsureCode{33}{12}% !
%\TMP@EnsureCode{36}{3}% $
%\TMP@EnsureCode{38}{4}% &
\TMP@EnsureCode{40}{12}% (
\TMP@EnsureCode{41}{12}% )
\TMP@EnsureCode{45}{12}% -
\TMP@EnsureCode{46}{12}% .
\TMP@EnsureCode{47}{12}% /
\TMP@EnsureCode{58}{12}% :
\TMP@EnsureCode{60}{12}% <
\TMP@EnsureCode{62}{12}% >
\TMP@EnsureCode{91}{12}% [
%\TMP@EnsureCode{96}{12}% `
\TMP@EnsureCode{93}{12}% ]
%\TMP@EnsureCode{94}{12}% ^ (superscript) (!)
%\TMP@EnsureCode{124}{12}% |
\edef\FibNum@AtEnd{\FibNum@AtEnd\noexpand\endinput}
%    \end{macrocode}
%
% \subsection{Package resources}
%
%    \begin{macrocode}
\begingroup\expandafter\expandafter\expandafter\endgroup
\expandafter\ifx\csname RequirePackage\endcsname\relax
  \def\TMP@RequirePackage#1[#2]{%
    \begingroup\expandafter\expandafter\expandafter\endgroup
    \expandafter\ifx\csname ver@#1.sty\endcsname\relax
      \input #1.sty\relax
    \fi
  }%
  \TMP@RequirePackage{ltxcmds}[2011/04/18]%
  \TMP@RequirePackage{intcalc}[2007/09/27]%
  \TMP@RequirePackage{bigintcalc}[2007/11/11]%
\else
  \RequirePackage{ltxcmds}[2011/04/18]%
  \RequirePackage{intcalc}[2007/09/27]%
  \RequirePackage{bigintcalc}[2007/11/11]%
\fi
%    \end{macrocode}
%
% \subsection{Setup precalculated values}
%
%    \begin{macrocode}
\def\FibNum@temp#1{%
  \expandafter\def\csname FibNum@#1\endcsname
}
\catcode46=9 % dots are ignored
\FibNum@temp{0}{0}
\FibNum@temp{1}{1}
\FibNum@temp{2}{1}
\FibNum@temp{3}{2}
\FibNum@temp{4}{3}
\FibNum@temp{5}{5}
\FibNum@temp{6}{8}
\FibNum@temp{7}{13}
\FibNum@temp{8}{21}
\FibNum@temp{9}{34}
\FibNum@temp{10}{55}
\FibNum@temp{11}{89}
\FibNum@temp{12}{144}
\FibNum@temp{13}{233}
\FibNum@temp{14}{377}
\FibNum@temp{15}{610}
\FibNum@temp{16}{987}
\FibNum@temp{17}{1.597}
\FibNum@temp{18}{2.584}
\FibNum@temp{19}{4.181}
\FibNum@temp{20}{6.765}
\FibNum@temp{21}{10.946}
\FibNum@temp{22}{17.711}
\FibNum@temp{23}{28.657}
\FibNum@temp{24}{46.368}
\FibNum@temp{25}{75.025}
\FibNum@temp{26}{121.393}
\FibNum@temp{27}{196.418}
\FibNum@temp{28}{317.811}
\FibNum@temp{29}{514.229}
\FibNum@temp{30}{832.040}
\FibNum@temp{31}{1.346.269}
\FibNum@temp{32}{2.178.309}
\FibNum@temp{33}{3.524.578}
\FibNum@temp{34}{5.702.887}
\FibNum@temp{35}{9.227.465}
\FibNum@temp{36}{14.930.352}
\FibNum@temp{37}{24.157.817}
\FibNum@temp{38}{39.088.169}
\FibNum@temp{39}{63.245.986}
\FibNum@temp{40}{102.334.155}
\FibNum@temp{41}{165.580.141}
\FibNum@temp{42}{267.914.296}
\FibNum@temp{43}{433.494.437}
\FibNum@temp{44}{701.408.733}
\FibNum@temp{45}{1.134.903.170}
\FibNum@temp{46}{1.836.311.903}
%    \end{macrocode}
%    \begin{macro}{\FibNum@max}
%    \begin{macrocode}
\def\FibNum@max{46}
%    \end{macrocode}
%    \end{macro}
%
% \subsection{Macros for precalculating values}
%
%    \begin{macro}{\fibnumPreCalc}
%    \begin{macrocode}
\def\fibnumPreCalc#1{%
  \expandafter\expandafter\expandafter
  \FibNum@PreCalc\intcalcNum{#1}/%
}
%    \end{macrocode}
%    \end{macro}
%    \begin{macro}{\FibNum@PreCalc}
%    \begin{macrocode}
\def\FibNum@PreCalc#1/{%
  \ifnum#1<\ltx@zero
    \expandafter\FibNum@PreCalc\ltx@gobble#1/%
  \else
    \ifnum#1>\FibNum@max
      \begingroup
        \ltx@LocDimenA=#1sp\relax
        \countdef\FibNum@i=255\relax
        \FibNum@i=\FibNum@max\relax
        \edef\FibNum@temp{%
          \csname FibNum@\the\FibNum@i\endcsname/%
        }%
        \advance\FibNum@i by -1\relax
        \edef\FibNum@temp{%
          \FibNum@temp
          \csname FibNum@\the\FibNum@i\endcsname
        }%
        \advance\FibNum@i\ltx@two
        \iftrue
          \expandafter\FibNum@PreCalcAux\FibNum@temp
        \fi
      \endgroup
    \fi
  \fi
}
%    \end{macrocode}
%    \end{macro}
%    \begin{macro}{\FibNum@PreCalcAux}
%    \begin{macrocode}
\def\FibNum@PreCalcAux#1/#2\fi{%
  \fi
  \edef\FibNum@temp{\BigIntCalcAdd#1!#2!}%
  \global\expandafter
  \let\csname FibNum@\the\FibNum@i\endcsname\FibNum@temp
  \ifnum\FibNum@i=\ltx@LocDimenA
    \xdef\FibNum@max{\the\FibNum@i}%
  \else
    \advance\FibNum@i\ltx@one
    \expandafter\FibNum@PreCalcAux\FibNum@temp/#1%
  \fi
}
%    \end{macrocode}
%    \end{macro}
%
% \subsection{Expandable calculations}
%
%    \begin{macro}{\fibnum}
%    \begin{macrocode}
\def\fibnum#1{%
  \romannumeral
  \expandafter\expandafter\expandafter\FibNum@Do\intcalcNum{#1}/%
}
%    \end{macrocode}
%    \end{macro}
%    \begin{macro}{\FibNum@Do}
%    \begin{macrocode}
\def\FibNum@Do#1/{%
  \ifnum#1<\ltx@zero
    \FibNum@ReturnAfterElseFiFi{%
      \ifodd#1 %
        \expandafter\expandafter\expandafter\ltx@zero
      \else
        \expandafter\expandafter\expandafter\ltx@zero
        \expandafter\expandafter\expandafter-%
      \fi
      \romannumeral
      \expandafter\FibNum@Do\ltx@gobble#1/%
    }%
  \else
    \ifnum\FibNum@max<#1 %
      \ltx@ReturnAfterElseFi{%
        \expandafter
        \FibNum@ExpCalc\number\expandafter\IntCalcInc\FibNum@max!%
        \expandafter\expandafter\expandafter/%
        \csname FibNum@\FibNum@max
        \expandafter\expandafter\expandafter\endcsname
        \expandafter\expandafter\expandafter/%
        \csname FibNum@\expandafter\IntCalcDec\FibNum@max!%
        \endcsname/%
        #1%
      }%
    \else
      \expandafter\expandafter\expandafter\ltx@zero
      \csname FibNum@#1\expandafter\expandafter\expandafter\endcsname
    \fi
  \fi
}
%    \end{macrocode}
%    \end{macro}
%    \begin{macro}{\FibNum@ReturnAfterElseFiFi}
%    \begin{macrocode}
\def\FibNum@ReturnAfterElseFiFi#1\else#2\fi\fi{\fi#1}
%    \end{macrocode}
%    \end{macro}
%    \begin{macro}{\FibNum@ExpCalc}
%    \begin{macrocode}
\def\FibNum@ExpCalc#1/#2/#3/#4\fi{%
  \fi
  \ifnum#1=#4 %
    \ltx@ReturnAfterElseFi{%
      \expandafter\expandafter\expandafter\ltx@zero
      \BigIntCalcAdd#2!#3!%
    }%
  \else
    \expandafter\FibNum@ExpCalc
    \number\IntCalcInc#1!%
    \expandafter\expandafter\expandafter/%
    \BigIntCalcAdd#2!#3!/%
    #2/#4%
  \fi
}
%    \end{macrocode}
%    \end{macro}
%
%    \begin{macrocode}
\FibNum@AtEnd%
%</package>
%    \end{macrocode}
%
% \section{Test}
%
% \subsection{Catcode checks for loading}
%
%    \begin{macrocode}
%<*test1>
%    \end{macrocode}
%    \begin{macrocode}
\catcode`\{=1 %
\catcode`\}=2 %
\catcode`\#=6 %
\catcode`\@=11 %
\expandafter\ifx\csname count@\endcsname\relax
  \countdef\count@=255 %
\fi
\expandafter\ifx\csname @gobble\endcsname\relax
  \long\def\@gobble#1{}%
\fi
\expandafter\ifx\csname @firstofone\endcsname\relax
  \long\def\@firstofone#1{#1}%
\fi
\expandafter\ifx\csname loop\endcsname\relax
  \expandafter\@firstofone
\else
  \expandafter\@gobble
\fi
{%
  \def\loop#1\repeat{%
    \def\body{#1}%
    \iterate
  }%
  \def\iterate{%
    \body
      \let\next\iterate
    \else
      \let\next\relax
    \fi
    \next
  }%
  \let\repeat=\fi
}%
\def\RestoreCatcodes{}
\count@=0 %
\loop
  \edef\RestoreCatcodes{%
    \RestoreCatcodes
    \catcode\the\count@=\the\catcode\count@\relax
  }%
\ifnum\count@<255 %
  \advance\count@ 1 %
\repeat

\def\RangeCatcodeInvalid#1#2{%
  \count@=#1\relax
  \loop
    \catcode\count@=15 %
  \ifnum\count@<#2\relax
    \advance\count@ 1 %
  \repeat
}
\def\RangeCatcodeCheck#1#2#3{%
  \count@=#1\relax
  \loop
    \ifnum#3=\catcode\count@
    \else
      \errmessage{%
        Character \the\count@\space
        with wrong catcode \the\catcode\count@\space
        instead of \number#3%
      }%
    \fi
  \ifnum\count@<#2\relax
    \advance\count@ 1 %
  \repeat
}
\def\space{ }
\expandafter\ifx\csname LoadCommand\endcsname\relax
  \def\LoadCommand{\input fibnum.sty\relax}%
\fi
\def\Test{%
  \RangeCatcodeInvalid{0}{47}%
  \RangeCatcodeInvalid{58}{64}%
  \RangeCatcodeInvalid{91}{96}%
  \RangeCatcodeInvalid{123}{255}%
  \catcode`\@=12 %
  \catcode`\\=0 %
  \catcode`\%=14 %
  \LoadCommand
  \RangeCatcodeCheck{0}{36}{15}%
  \RangeCatcodeCheck{37}{37}{14}%
  \RangeCatcodeCheck{38}{47}{15}%
  \RangeCatcodeCheck{48}{57}{12}%
  \RangeCatcodeCheck{58}{63}{15}%
  \RangeCatcodeCheck{64}{64}{12}%
  \RangeCatcodeCheck{65}{90}{11}%
  \RangeCatcodeCheck{91}{91}{15}%
  \RangeCatcodeCheck{92}{92}{0}%
  \RangeCatcodeCheck{93}{96}{15}%
  \RangeCatcodeCheck{97}{122}{11}%
  \RangeCatcodeCheck{123}{255}{15}%
  \RestoreCatcodes
}
\Test
\csname @@end\endcsname
\end
%    \end{macrocode}
%    \begin{macrocode}
%</test1>
%    \end{macrocode}
%
% \subsection{Test calculations}
%
%    \begin{macrocode}
%<*test-calc>
\catcode`\{=1 %
\catcode`\}=2 %
\catcode`\#=6 %
\catcode`\@=11 %
\begingroup\expandafter\expandafter\expandafter\endgroup
\expandafter\ifx\csname RequirePackage\endcsname\relax
  \input fibnum.sty\relax
\else
  \RequirePackage{fibnum}[2016/05/16]%
\fi
\def\TestSet{%
  \test{0}{0}%
  \test{1}{1}%
  \test{2}{1}%
  \test{3}{2}%
  \test{4}{3}%
  \test{5}{5}%
  \test{6}{8}%
  \test{7}{13}%
  \test{8}{21}%
  \test{9}{34}%
  \test{10}{55}%
  \test{11}{89}%
  \test{12}{144}%
  \test{13}{233}%
  \test{14}{377}%
  \test{15}{610}%
  \test{16}{987}%
  \test{17}{1597}%
  \test{18}{2584}%
  \test{19}{4181}%
  \test{20}{6765}%
  \test{21}{10946}%
  \test{22}{17711}%
  \test{23}{28657}%
  \test{24}{46368}%
  \test{25}{75025}%
  \test{26}{121393}%
  \test{27}{196418}%
  \test{28}{317811}%
  \test{29}{514229}%
  \test{30}{832040}%
  \test{31}{1346269}%
  \test{32}{2178309}%
  \test{33}{3524578}%
  \test{34}{5702887}%
  \test{35}{9227465}%
  \test{36}{14930352}%
  \test{37}{24157817}%
  \test{38}{39088169}%
  \test{39}{63245986}%
  \test{40}{102334155}%
  \test{41}{165580141}%
  \test{42}{267914296}%
  \test{43}{433494437}%
  \test{44}{701408733}%
  \test{45}{1134903170}%
  \test{46}{1836311903}%
  \test{47}{2971215073}%
  \test{48}{4807526976}%
  \test{49}{7778742049}%
  \test{50}{12586269025}%
  \test{51}{20365011074}%
  \test{52}{32951280099}%
  \test{53}{53316291173}%
  \test{54}{86267571272}%
  \test{55}{139583862445}%
  \test{56}{225851433717}%
  \test{57}{365435296162}%
  \test{58}{591286729879}%
  \test{59}{956722026041}%
  \test{60}{1548008755920}%
  \test{61}{2504730781961}%
  \test{62}{4052739537881}%
  \test{63}{6557470319842}%
  \test{64}{10610209857723}%
  \test{65}{17167680177565}%
  \test{66}{27777890035288}%
  \test{67}{44945570212853}%
  \test{68}{72723460248141}%
  \test{69}{117669030460994}%
  \test{70}{190392490709135}%
  \test{71}{308061521170129}%
  \test{72}{498454011879264}%
  \test{73}{806515533049393}%
}
\def\msg#{\immediate\write16}
\def\test#1#2{%
  \TestAux{#1}{#2}%
  \ifnum#1=0 %
  \else
    \ifodd#1 %
      \TestAux{-#1}{#2}%
    \else
      \TestAux{-#1}{-#2}%
    \fi
  \fi
}
\def\TestAux#1#2{%
  \def\Expected{#2}%
  \expandafter\expandafter\expandafter\def
  \expandafter\expandafter\expandafter\Result
  \expandafter\expandafter\expandafter{%
    \fibnum{#1}%
  }%
  \ltx@onelevel@sanitize\Result
  \ifx\Result\Expected
    \msg{* #1: ok.}%
  \else
    \msg{! fib(#1) = #2}%
    \errmessage{fib(#1) <> \Result}%
  \fi
}
\TestSet
\setbox0=\hbox{%
  \msg{* PreCalc{73}}%
  \fibnumPreCalc{73}%
}
\ifdim\wd0=0pt
\else
  \errmessage{Unwanted stuff in PreCalc}%
\fi
\TestSet
\csname @@end\endcsname\end
%</test-calc>
%    \end{macrocode}
%
% \section{Installation}
%
% \subsection{Download}
%
% \paragraph{Package.} This package is available on
% CTAN\footnote{\url{http://ctan.org/pkg/fibnum}}:
% \begin{description}
% \item[\CTAN{macros/latex/contrib/oberdiek/fibnum.dtx}] The source file.
% \item[\CTAN{macros/latex/contrib/oberdiek/fibnum.pdf}] Documentation.
% \end{description}
%
%
% \paragraph{Bundle.} All the packages of the bundle `oberdiek'
% are also available in a TDS compliant ZIP archive. There
% the packages are already unpacked and the documentation files
% are generated. The files and directories obey the TDS standard.
% \begin{description}
% \item[\CTAN{install/macros/latex/contrib/oberdiek.tds.zip}]
% \end{description}
% \emph{TDS} refers to the standard ``A Directory Structure
% for \TeX\ Files'' (\CTAN{tds/tds.pdf}). Directories
% with \xfile{texmf} in their name are usually organized this way.
%
% \subsection{Bundle installation}
%
% \paragraph{Unpacking.} Unpack the \xfile{oberdiek.tds.zip} in the
% TDS tree (also known as \xfile{texmf} tree) of your choice.
% Example (linux):
% \begin{quote}
%   |unzip oberdiek.tds.zip -d ~/texmf|
% \end{quote}
%
% \paragraph{Script installation.}
% Check the directory \xfile{TDS:scripts/oberdiek/} for
% scripts that need further installation steps.
% Package \xpackage{attachfile2} comes with the Perl script
% \xfile{pdfatfi.pl} that should be installed in such a way
% that it can be called as \texttt{pdfatfi}.
% Example (linux):
% \begin{quote}
%   |chmod +x scripts/oberdiek/pdfatfi.pl|\\
%   |cp scripts/oberdiek/pdfatfi.pl /usr/local/bin/|
% \end{quote}
%
% \subsection{Package installation}
%
% \paragraph{Unpacking.} The \xfile{.dtx} file is a self-extracting
% \docstrip\ archive. The files are extracted by running the
% \xfile{.dtx} through \plainTeX:
% \begin{quote}
%   \verb|tex fibnum.dtx|
% \end{quote}
%
% \paragraph{TDS.} Now the different files must be moved into
% the different directories in your installation TDS tree
% (also known as \xfile{texmf} tree):
% \begin{quote}
% \def\t{^^A
% \begin{tabular}{@{}>{\ttfamily}l@{ $\rightarrow$ }>{\ttfamily}l@{}}
%   fibnum.sty & tex/generic/oberdiek/fibnum.sty\\
%   fibnum.pdf & doc/latex/oberdiek/fibnum.pdf\\
%   test/fibnum-test1.tex & doc/latex/oberdiek/test/fibnum-test1.tex\\
%   test/fibnum-test-calc.tex & doc/latex/oberdiek/test/fibnum-test-calc.tex\\
%   fibnum.dtx & source/latex/oberdiek/fibnum.dtx\\
% \end{tabular}^^A
% }^^A
% \sbox0{\t}^^A
% \ifdim\wd0>\linewidth
%   \begingroup
%     \advance\linewidth by\leftmargin
%     \advance\linewidth by\rightmargin
%   \edef\x{\endgroup
%     \def\noexpand\lw{\the\linewidth}^^A
%   }\x
%   \def\lwbox{^^A
%     \leavevmode
%     \hbox to \linewidth{^^A
%       \kern-\leftmargin\relax
%       \hss
%       \usebox0
%       \hss
%       \kern-\rightmargin\relax
%     }^^A
%   }^^A
%   \ifdim\wd0>\lw
%     \sbox0{\small\t}^^A
%     \ifdim\wd0>\linewidth
%       \ifdim\wd0>\lw
%         \sbox0{\footnotesize\t}^^A
%         \ifdim\wd0>\linewidth
%           \ifdim\wd0>\lw
%             \sbox0{\scriptsize\t}^^A
%             \ifdim\wd0>\linewidth
%               \ifdim\wd0>\lw
%                 \sbox0{\tiny\t}^^A
%                 \ifdim\wd0>\linewidth
%                   \lwbox
%                 \else
%                   \usebox0
%                 \fi
%               \else
%                 \lwbox
%               \fi
%             \else
%               \usebox0
%             \fi
%           \else
%             \lwbox
%           \fi
%         \else
%           \usebox0
%         \fi
%       \else
%         \lwbox
%       \fi
%     \else
%       \usebox0
%     \fi
%   \else
%     \lwbox
%   \fi
% \else
%   \usebox0
% \fi
% \end{quote}
% If you have a \xfile{docstrip.cfg} that configures and enables \docstrip's
% TDS installing feature, then some files can already be in the right
% place, see the documentation of \docstrip.
%
% \subsection{Refresh file name databases}
%
% If your \TeX~distribution
% (\teTeX, \mikTeX, \dots) relies on file name databases, you must refresh
% these. For example, \teTeX\ users run \verb|texhash| or
% \verb|mktexlsr|.
%
% \subsection{Some details for the interested}
%
% \paragraph{Attached source.}
%
% The PDF documentation on CTAN also includes the
% \xfile{.dtx} source file. It can be extracted by
% AcrobatReader 6 or higher. Another option is \textsf{pdftk},
% e.g. unpack the file into the current directory:
% \begin{quote}
%   \verb|pdftk fibnum.pdf unpack_files output .|
% \end{quote}
%
% \paragraph{Unpacking with \LaTeX.}
% The \xfile{.dtx} chooses its action depending on the format:
% \begin{description}
% \item[\plainTeX:] Run \docstrip\ and extract the files.
% \item[\LaTeX:] Generate the documentation.
% \end{description}
% If you insist on using \LaTeX\ for \docstrip\ (really,
% \docstrip\ does not need \LaTeX), then inform the autodetect routine
% about your intention:
% \begin{quote}
%   \verb|latex \let\install=y\input{fibnum.dtx}|
% \end{quote}
% Do not forget to quote the argument according to the demands
% of your shell.
%
% \paragraph{Generating the documentation.}
% You can use both the \xfile{.dtx} or the \xfile{.drv} to generate
% the documentation. The process can be configured by the
% configuration file \xfile{ltxdoc.cfg}. For instance, put this
% line into this file, if you want to have A4 as paper format:
% \begin{quote}
%   \verb|\PassOptionsToClass{a4paper}{article}|
% \end{quote}
% An example follows how to generate the
% documentation with pdf\LaTeX:
% \begin{quote}
%\begin{verbatim}
%pdflatex fibnum.dtx
%bibtex fibnum.aux
%makeindex -s gind.ist fibnum.idx
%pdflatex fibnum.dtx
%makeindex -s gind.ist fibnum.idx
%pdflatex fibnum.dtx
%\end{verbatim}
% \end{quote}
%
% \printbibliography[
%   heading=bibnumbered,
% ]
%
% \begin{History}
%   \begin{Version}{2012/04/08 v1.0}
%   \item
%     First version.
%   \end{Version}
%   \begin{Version}{2016/05/16 v1.1}
%   \item
%     Documentation updates.
%   \end{Version}
% \end{History}
%
% \PrintIndex
%
% \Finale
\endinput
|
% \end{quote}
% Do not forget to quote the argument according to the demands
% of your shell.
%
% \paragraph{Generating the documentation.}
% You can use both the \xfile{.dtx} or the \xfile{.drv} to generate
% the documentation. The process can be configured by the
% configuration file \xfile{ltxdoc.cfg}. For instance, put this
% line into this file, if you want to have A4 as paper format:
% \begin{quote}
%   \verb|\PassOptionsToClass{a4paper}{article}|
% \end{quote}
% An example follows how to generate the
% documentation with pdf\LaTeX:
% \begin{quote}
%\begin{verbatim}
%pdflatex fibnum.dtx
%bibtex fibnum.aux
%makeindex -s gind.ist fibnum.idx
%pdflatex fibnum.dtx
%makeindex -s gind.ist fibnum.idx
%pdflatex fibnum.dtx
%\end{verbatim}
% \end{quote}
%
% \printbibliography[
%   heading=bibnumbered,
% ]
%
% \begin{History}
%   \begin{Version}{2012/04/08 v1.0}
%   \item
%     First version.
%   \end{Version}
%   \begin{Version}{2016/05/16 v1.1}
%   \item
%     Documentation updates.
%   \end{Version}
% \end{History}
%
% \PrintIndex
%
% \Finale
\endinput
|
% \end{quote}
% Do not forget to quote the argument according to the demands
% of your shell.
%
% \paragraph{Generating the documentation.}
% You can use both the \xfile{.dtx} or the \xfile{.drv} to generate
% the documentation. The process can be configured by the
% configuration file \xfile{ltxdoc.cfg}. For instance, put this
% line into this file, if you want to have A4 as paper format:
% \begin{quote}
%   \verb|\PassOptionsToClass{a4paper}{article}|
% \end{quote}
% An example follows how to generate the
% documentation with pdf\LaTeX:
% \begin{quote}
%\begin{verbatim}
%pdflatex fibnum.dtx
%bibtex fibnum.aux
%makeindex -s gind.ist fibnum.idx
%pdflatex fibnum.dtx
%makeindex -s gind.ist fibnum.idx
%pdflatex fibnum.dtx
%\end{verbatim}
% \end{quote}
%
% \printbibliography[
%   heading=bibnumbered,
% ]
%
% \begin{History}
%   \begin{Version}{2012/04/08 v1.0}
%   \item
%     First version.
%   \end{Version}
%   \begin{Version}{2016/05/16 v1.1}
%   \item
%     Documentation updates.
%   \end{Version}
% \end{History}
%
% \PrintIndex
%
% \Finale
\endinput

%        (quote the arguments according to the demands of your shell)
%
% Documentation:
%    (a) If fibnum.drv is present:
%           latex fibnum.drv
%    (b) Without fibnum.drv:
%           latex fibnum.dtx; ...
%    The class ltxdoc loads the configuration file ltxdoc.cfg
%    if available. Here you can specify further options, e.g.
%    use A4 as paper format:
%       \PassOptionsToClass{a4paper}{article}
%
%    Programm calls to get the documentation (example):
%       pdflatex fibnum.dtx
%       bibtex fibnum.aux
%       makeindex -s gind.ist fibnum.idx
%       pdflatex fibnum.dtx
%       makeindex -s gind.ist fibnum.idx
%       pdflatex fibnum.dtx
%
% Installation:
%    TDS:tex/generic/oberdiek/fibnum.sty
%    TDS:doc/latex/oberdiek/fibnum.pdf
%    TDS:doc/latex/oberdiek/test/fibnum-test1.tex
%    TDS:doc/latex/oberdiek/test/fibnum-test-calc.tex
%    TDS:source/latex/oberdiek/fibnum.dtx
%
%<*ignore>
\begingroup
  \catcode123=1 %
  \catcode125=2 %
  \def\x{LaTeX2e}%
\expandafter\endgroup
\ifcase 0\ifx\install y1\fi\expandafter
         \ifx\csname processbatchFile\endcsname\relax\else1\fi
         \ifx\fmtname\x\else 1\fi\relax
\else\csname fi\endcsname
%</ignore>
%<*install>
\input docstrip.tex
\Msg{************************************************************************}
\Msg{* Installation}
\Msg{* Package: fibnum 2016/05/16 v1.1 Fibonacci numbers (HO)}
\Msg{************************************************************************}

\keepsilent
\askforoverwritefalse

\let\MetaPrefix\relax
\preamble

This is a generated file.

Project: fibnum
Version: 2016/05/16 v1.1

Copyright (C) 2012 by
   Heiko Oberdiek <heiko.oberdiek at googlemail.com>

This work may be distributed and/or modified under the
conditions of the LaTeX Project Public License, either
version 1.3c of this license or (at your option) any later
version. This version of this license is in
   http://www.latex-project.org/lppl/lppl-1-3c.txt
and the latest version of this license is in
   http://www.latex-project.org/lppl.txt
and version 1.3 or later is part of all distributions of
LaTeX version 2005/12/01 or later.

This work has the LPPL maintenance status "maintained".

This Current Maintainer of this work is Heiko Oberdiek.

The Base Interpreter refers to any `TeX-Format',
because some files are installed in TDS:tex/generic//.

This work consists of the main source file fibnum.dtx
and the derived files
   fibnum.sty, fibnum.pdf, fibnum.ins, fibnum.drv, fibnum.bib,
   fibnum-test1.tex, fibnum-test-calc.tex.

\endpreamble
\let\MetaPrefix\DoubleperCent

\generate{%
  \file{fibnum.ins}{\from{fibnum.dtx}{install}}%
  \file{fibnum.drv}{\from{fibnum.dtx}{driver}}%
  \nopreamble
  \nopostamble
  \file{fibnum.bib}{\from{fibnum.dtx}{bib}}%
  \usepreamble\defaultpreamble
  \usepostamble\defaultpostamble
  \usedir{tex/generic/oberdiek}%
  \file{fibnum.sty}{\from{fibnum.dtx}{package}}%
  \usedir{doc/latex/oberdiek/test}%
  \file{fibnum-test1.tex}{\from{fibnum.dtx}{test1}}%
  \file{fibnum-test-calc.tex}{\from{fibnum.dtx}{test-calc}}%
}

\catcode32=13\relax% active space
\let =\space%
\Msg{************************************************************************}
\Msg{*}
\Msg{* To finish the installation you have to move the following}
\Msg{* file into a directory searched by TeX:}
\Msg{*}
\Msg{*     fibnum.sty}
\Msg{*}
\Msg{* To produce the documentation run the file `fibnum.drv'}
\Msg{* through LaTeX.}
\Msg{*}
\Msg{* Happy TeXing!}
\Msg{*}
\Msg{************************************************************************}

\endbatchfile
%</install>
%<*bib>
@online{texhax:abraham,
  author={Abraham, Jan},
  title={[texhax] Beginner in TEX MACRO to compute functions},
  date={2012-04-07},
  url={http://tug.org/pipermail/texhax/2012-April/019146.html},
  urldate={2012-04-08},
}
@article{knuth:negafibonacci,
  author={Knuth, Donald E.},
  title={Negafibonacci Numbers and the Hyperbolic Plane},
  date={2008-12-11},
  url={http://research.allacademic.com/meta/p206842_index.html},
}
@online{wikipedia:negafibonacci,
  author={{Wikipedia contributors}},
  organization={{Wikipedia, The Free Encyclopedia}},
  title={Fibonacci numbers},
  language={langenglish},
  version={486266088},
  date={2012-04-08},
  url={http://en.wikipedia.org/w/index.php?title=Fibonacci_number&oldid=486266088},
  urldate={2012-04-08},
}
%</bib>
%<*ignore>
\fi
%</ignore>
%<*driver>
\NeedsTeXFormat{LaTeX2e}
\ProvidesFile{fibnum.drv}%
  [2016/05/16 v1.1 Fibonacci numbers (HO)]%
\documentclass{ltxdoc}
\usepackage{amsmath,amsfonts}
\usepackage{siunitx}
\usepackage{array}
\usepackage{tabularx}
\usepackage{fibnum}[2016/05/16]
\usepackage{holtxdoc}[2011/11/22]
\usepackage{csquotes}
\usepackage[
  bibencoding=ascii,
  alldates=iso8601,
]{biblatex}[2011/11/13]
\bibliography{oberdiek-source}
\bibliography{fibnum}
\begin{document}
  \DocInput{fibnum.dtx}%
\end{document}
%</driver>
% \fi
%
%
% \CharacterTable
%  {Upper-case    \A\B\C\D\E\F\G\H\I\J\K\L\M\N\O\P\Q\R\S\T\U\V\W\X\Y\Z
%   Lower-case    \a\b\c\d\e\f\g\h\i\j\k\l\m\n\o\p\q\r\s\t\u\v\w\x\y\z
%   Digits        \0\1\2\3\4\5\6\7\8\9
%   Exclamation   \!     Double quote  \"     Hash (number) \#
%   Dollar        \$     Percent       \%     Ampersand     \&
%   Acute accent  \'     Left paren    \(     Right paren   \)
%   Asterisk      \*     Plus          \+     Comma         \,
%   Minus         \-     Point         \.     Solidus       \/
%   Colon         \:     Semicolon     \;     Less than     \<
%   Equals        \=     Greater than  \>     Question mark \?
%   Commercial at \@     Left bracket  \[     Backslash     \\
%   Right bracket \]     Circumflex    \^     Underscore    \_
%   Grave accent  \`     Left brace    \{     Vertical bar  \|
%   Right brace   \}     Tilde         \~}
%
% \GetFileInfo{fibnum.drv}
%
% \title{The \xpackage{fibnum} package}
% \date{2016/05/16 v1.1}
% \author{Heiko Oberdiek\thanks
% {Please report any issues at https://github.com/ho-tex/oberdiek/issues}\\
% \xemail{heiko.oberdiek at googlemail.com}}
%
% \maketitle
%
% \begin{abstract}
% The package \xpackage{fibnum} provides expandable fibonacci
% numbers for both \hologo{LaTeX} and \hologo{plainTeX}.
% \end{abstract}
%
% \tableofcontents
%
% \section{Documentation}
%
% In the mailing list \textsf{texhax} Jan Abraham asked the question,
% how to get Fibonacci numbers in \hologo{TeX} \cite{texhax:abraham}:
% \begin{quote}
% Write a Macro in \hologo{TeX} that compute the function |\fib{m}|
% All fibonacci numbers from 1 to $m$ ($m < 40$).
% \end{quote}
% This packages provides an expandable implementation for the
% calculation of these numbers for a much larger set of indexes.
% For practical reasons the index is restricted to the same limitations
% that apply for \hologo{TeX} integer numbers.
% The range of the Fibonacci numbers, however, are not limited
% by the algorithm. They are only restricted to memory limitations,
% if they are hit.
%
% The package is loaded as \hologo{LaTeX} package in \hologo{LaTeX}:
% \begin{quote}
%   |\usepackage{fibnum}|
% \end{quote}
% and as file in \hologo{plainTeX}:
% \begin{quote}
%   |\input fibnum.sty|
% \end{quote}
% The package does not know any options and it provides
% the macros \cs{fibnum} and \cs{fibnumPreCalc}.
%
% \begin{declcs}{fibnum} \M{index}
% \end{declcs}
% Macro \cs{fibnum} expects a \hologo{TeX} number as \meta{index}
% in the official \hologo{TeX} number range from $-(2^{31}-1)$ up to
% $2^{31}-1$. In exact two expansion steps the macro expands to
% the Fibnoacci number $F_{\text{\meta{index}}}$. In case of a negative
% \meta{index}, the ``negafibonacci'' number \cite{wikipedia:negafibonacci}
% is used. Formally the Fibonacci number $F_n$ with integer
% index~$n$, $n\in\mathbb{Z}$ and
% $n\in[\num{-2147483647},\num{2147483647}]$ that is returned by macro
% \cs{fibnum} with numerical argument $n$ is defined the following way:
% \begin{gather}
%   \label{eq:def}
%   F_n =
%   \begin{cases}
%     0 & \text{for $n=0$}\\
%     1 & \text{for $n=1$}\\
%     F_{n-1} + F_{n-2} & \text{for $n>1$}\\
%     (-1)^{n+1}F_n & \text{for $n<0$}
%   \end{cases}
% \end{gather}
% Examples:
% \begin{quote}
%   \makeatletter
%   \def\x#1{\cs{fibnum}\{#1\}&
%     \edef\X{\fibnum{#1}}\edef\Y{\expandafter\ltx@car\X\@nil}^^A
%     \if-\Y
%       \edef\X{\expandafter\ltx@cdr\X\@nil}^^A
%       \noindent
%       \llap{-}\X
%     \else
%       \X
%     \fi
%     \tabularnewline
%   }
%   \def\y{\multicolumn{1}{@{}c@{}}{$\vdots$}\tabularnewline}
%   \DeclareUrlCommand\UrlNum{^^A
%     \urlstyle{tt}^^A
%     \def\UrlBreaks{\do\0\do\1\do\2\do\3\do\4\do\5\do\6\do\7\do\8\do\9}^^A
%   }
%   \begin{tabularx}{\dimexpr\linewidth+5.7pt\relax}{@{}>{\ttfamily}l@{ $\rightarrow$ \hphantom{\ttfamily-}}>{\ttfamily}X@{}}
%     \x{-6}
%     \x{-5}
%     \x{-4}
%     \x{-3}
%     \x{-2}
%     \x{-1}
%     \x{0}
%     \x{1}
%     \x{2}
%     \x{3}
%     \x{4}
%     \x{5}
%     \x{6}
%     \y
%     \x{10}
%     \y
%     \x{46}
%     \y
%     \cs{fibnum}\{100\} & 354224848179261915075
%     \tabularnewline
%     \y
%     \cs{fibnum}\{200\} & 280571172992510140037611932413038677189525
%     \tabularnewline
%     \y
%     \cs{fibnum}\{1000\} &
%       \raggedright
%       \UrlNum{^^A
%         434665576869374564356885276750406258025646^^A
%         605173717804024817290895365554179490518904^^A
%         038798400792551692959225930803226347752096^^A
%         896232398733224711616429964409065331879382^^A
%         98969649928516003704476137795166849228875^^A
%       }
%     \tabularnewline
%   \end{tabularx}\kern-5.7pt\mbox{}
% \end{quote}
%
% \begin{declcs}{fibnumPreCalc} \M{index}
% \end{declcs}
% The package already provides precalculated Fibonacci numbers up to
% index~46. That means that calculations are not necessary for
% Fibonacci numbers that fit into the range of \hologo{TeX}
% numbers. Because macro \cs{fibnum} is expandable, it cannot
% store calculated Fibonacci numbers for later use. Macro definitions
% are forbidden in expandable contexts. If larger Fibonacci numbers
% are used more than once, than the compilation time can be shortened
% by calculating and storing the Fibonacci numbers beforehand.
% The argument \meta{index} is a \hologo{TeX} number and macro
% \cs{fibnumPreCalc} ensures that the Fibonacci numbers
% $F_0$ up to $F_{\lvert\text{\meta{index}}\rvert}$ that are not
% already known are calculated
% and stored in internal macros. Internally only non-negative
% Fibonacci numbers are stored. If \meta{index} is negative, then
% the needed positive Fibonacci numbers are calculated and stored.
% Example:
% \begin{quote}
%   \def\x#1{\begingroup\itshape\texttt{\%} #1\endgroup}
%   |\fibnumPreCalc{50}|\\
%   \x{calculates and stores the values for indexes 47..50.}\\
%   \x{(Values for 0..46 are already stored by the package.)}\\
%   |\fibnum{49}| \x{uses the stored value}\\
%   |\fibnum{51}|
%   \x{only calculates $F_{51}$ from stored values $F_{49}$ and $F_{50}$}\\
%   |\fibnumPreCalc{100}|\\
%   \x{calculates and stores the values for indexes 51..100}\\
%   |\fibnum{100}| \x{uses the stored value for $F_{100}$}\\
%   |\fibnum{-100}|
%   \x{uses the stored value for $F_{100}$}\\
%   \x{$F_{-100}=-F_{100}$ according to equation \eqref{eq:def}.}
% \end{quote}
%
% \StopEventually{
% }
%
% \section{Implementation}
%
% \subsection{Identification}
%
%    \begin{macrocode}
%<*package>
%    \end{macrocode}
%    Reload check, especially if the package is not used with \LaTeX.
%    \begin{macrocode}
\begingroup\catcode61\catcode48\catcode32=10\relax%
  \catcode13=5 % ^^M
  \endlinechar=13 %
  \catcode35=6 % #
  \catcode39=12 % '
  \catcode44=12 % ,
  \catcode45=12 % -
  \catcode46=12 % .
  \catcode58=12 % :
  \catcode64=11 % @
  \catcode123=1 % {
  \catcode125=2 % }
  \expandafter\let\expandafter\x\csname ver@fibnum.sty\endcsname
  \ifx\x\relax % plain-TeX, first loading
  \else
    \def\empty{}%
    \ifx\x\empty % LaTeX, first loading,
      % variable is initialized, but \ProvidesPackage not yet seen
    \else
      \expandafter\ifx\csname PackageInfo\endcsname\relax
        \def\x#1#2{%
          \immediate\write-1{Package #1 Info: #2.}%
        }%
      \else
        \def\x#1#2{\PackageInfo{#1}{#2, stopped}}%
      \fi
      \x{fibnum}{The package is already loaded}%
      \aftergroup\endinput
    \fi
  \fi
\endgroup%
%    \end{macrocode}
%    Package identification:
%    \begin{macrocode}
\begingroup\catcode61\catcode48\catcode32=10\relax%
  \catcode13=5 % ^^M
  \endlinechar=13 %
  \catcode35=6 % #
  \catcode39=12 % '
  \catcode40=12 % (
  \catcode41=12 % )
  \catcode44=12 % ,
  \catcode45=12 % -
  \catcode46=12 % .
  \catcode47=12 % /
  \catcode58=12 % :
  \catcode64=11 % @
  \catcode91=12 % [
  \catcode93=12 % ]
  \catcode123=1 % {
  \catcode125=2 % }
  \expandafter\ifx\csname ProvidesPackage\endcsname\relax
    \def\x#1#2#3[#4]{\endgroup
      \immediate\write-1{Package: #3 #4}%
      \xdef#1{#4}%
    }%
  \else
    \def\x#1#2[#3]{\endgroup
      #2[{#3}]%
      \ifx#1\@undefined
        \xdef#1{#3}%
      \fi
      \ifx#1\relax
        \xdef#1{#3}%
      \fi
    }%
  \fi
\expandafter\x\csname ver@fibnum.sty\endcsname
\ProvidesPackage{fibnum}%
  [2016/05/16 v1.1 Fibonacci numbers (HO)]%
%    \end{macrocode}
%
%    \begin{macrocode}
\begingroup\catcode61\catcode48\catcode32=10\relax%
  \catcode13=5 % ^^M
  \endlinechar=13 %
  \catcode123=1 % {
  \catcode125=2 % }
  \catcode64=11 % @
  \def\x{\endgroup
    \expandafter\edef\csname FibNum@AtEnd\endcsname{%
      \endlinechar=\the\endlinechar\relax
      \catcode13=\the\catcode13\relax
      \catcode32=\the\catcode32\relax
      \catcode35=\the\catcode35\relax
      \catcode61=\the\catcode61\relax
      \catcode64=\the\catcode64\relax
      \catcode123=\the\catcode123\relax
      \catcode125=\the\catcode125\relax
    }%
  }%
\x\catcode61\catcode48\catcode32=10\relax%
\catcode13=5 % ^^M
\endlinechar=13 %
\catcode35=6 % #
\catcode64=11 % @
\catcode123=1 % {
\catcode125=2 % }
\def\TMP@EnsureCode#1#2{%
  \edef\FibNum@AtEnd{%
    \FibNum@AtEnd
    \catcode#1=\the\catcode#1\relax
  }%
  \catcode#1=#2\relax
}
\TMP@EnsureCode{33}{12}% !
%\TMP@EnsureCode{36}{3}% $
%\TMP@EnsureCode{38}{4}% &
\TMP@EnsureCode{40}{12}% (
\TMP@EnsureCode{41}{12}% )
\TMP@EnsureCode{45}{12}% -
\TMP@EnsureCode{46}{12}% .
\TMP@EnsureCode{47}{12}% /
\TMP@EnsureCode{58}{12}% :
\TMP@EnsureCode{60}{12}% <
\TMP@EnsureCode{62}{12}% >
\TMP@EnsureCode{91}{12}% [
%\TMP@EnsureCode{96}{12}% `
\TMP@EnsureCode{93}{12}% ]
%\TMP@EnsureCode{94}{12}% ^ (superscript) (!)
%\TMP@EnsureCode{124}{12}% |
\edef\FibNum@AtEnd{\FibNum@AtEnd\noexpand\endinput}
%    \end{macrocode}
%
% \subsection{Package resources}
%
%    \begin{macrocode}
\begingroup\expandafter\expandafter\expandafter\endgroup
\expandafter\ifx\csname RequirePackage\endcsname\relax
  \def\TMP@RequirePackage#1[#2]{%
    \begingroup\expandafter\expandafter\expandafter\endgroup
    \expandafter\ifx\csname ver@#1.sty\endcsname\relax
      \input #1.sty\relax
    \fi
  }%
  \TMP@RequirePackage{ltxcmds}[2011/04/18]%
  \TMP@RequirePackage{intcalc}[2007/09/27]%
  \TMP@RequirePackage{bigintcalc}[2007/11/11]%
\else
  \RequirePackage{ltxcmds}[2011/04/18]%
  \RequirePackage{intcalc}[2007/09/27]%
  \RequirePackage{bigintcalc}[2007/11/11]%
\fi
%    \end{macrocode}
%
% \subsection{Setup precalculated values}
%
%    \begin{macrocode}
\def\FibNum@temp#1{%
  \expandafter\def\csname FibNum@#1\endcsname
}
\catcode46=9 % dots are ignored
\FibNum@temp{0}{0}
\FibNum@temp{1}{1}
\FibNum@temp{2}{1}
\FibNum@temp{3}{2}
\FibNum@temp{4}{3}
\FibNum@temp{5}{5}
\FibNum@temp{6}{8}
\FibNum@temp{7}{13}
\FibNum@temp{8}{21}
\FibNum@temp{9}{34}
\FibNum@temp{10}{55}
\FibNum@temp{11}{89}
\FibNum@temp{12}{144}
\FibNum@temp{13}{233}
\FibNum@temp{14}{377}
\FibNum@temp{15}{610}
\FibNum@temp{16}{987}
\FibNum@temp{17}{1.597}
\FibNum@temp{18}{2.584}
\FibNum@temp{19}{4.181}
\FibNum@temp{20}{6.765}
\FibNum@temp{21}{10.946}
\FibNum@temp{22}{17.711}
\FibNum@temp{23}{28.657}
\FibNum@temp{24}{46.368}
\FibNum@temp{25}{75.025}
\FibNum@temp{26}{121.393}
\FibNum@temp{27}{196.418}
\FibNum@temp{28}{317.811}
\FibNum@temp{29}{514.229}
\FibNum@temp{30}{832.040}
\FibNum@temp{31}{1.346.269}
\FibNum@temp{32}{2.178.309}
\FibNum@temp{33}{3.524.578}
\FibNum@temp{34}{5.702.887}
\FibNum@temp{35}{9.227.465}
\FibNum@temp{36}{14.930.352}
\FibNum@temp{37}{24.157.817}
\FibNum@temp{38}{39.088.169}
\FibNum@temp{39}{63.245.986}
\FibNum@temp{40}{102.334.155}
\FibNum@temp{41}{165.580.141}
\FibNum@temp{42}{267.914.296}
\FibNum@temp{43}{433.494.437}
\FibNum@temp{44}{701.408.733}
\FibNum@temp{45}{1.134.903.170}
\FibNum@temp{46}{1.836.311.903}
%    \end{macrocode}
%    \begin{macro}{\FibNum@max}
%    \begin{macrocode}
\def\FibNum@max{46}
%    \end{macrocode}
%    \end{macro}
%
% \subsection{Macros for precalculating values}
%
%    \begin{macro}{\fibnumPreCalc}
%    \begin{macrocode}
\def\fibnumPreCalc#1{%
  \expandafter\expandafter\expandafter
  \FibNum@PreCalc\intcalcNum{#1}/%
}
%    \end{macrocode}
%    \end{macro}
%    \begin{macro}{\FibNum@PreCalc}
%    \begin{macrocode}
\def\FibNum@PreCalc#1/{%
  \ifnum#1<\ltx@zero
    \expandafter\FibNum@PreCalc\ltx@gobble#1/%
  \else
    \ifnum#1>\FibNum@max
      \begingroup
        \ltx@LocDimenA=#1sp\relax
        \countdef\FibNum@i=255\relax
        \FibNum@i=\FibNum@max\relax
        \edef\FibNum@temp{%
          \csname FibNum@\the\FibNum@i\endcsname/%
        }%
        \advance\FibNum@i by -1\relax
        \edef\FibNum@temp{%
          \FibNum@temp
          \csname FibNum@\the\FibNum@i\endcsname
        }%
        \advance\FibNum@i\ltx@two
        \iftrue
          \expandafter\FibNum@PreCalcAux\FibNum@temp
        \fi
      \endgroup
    \fi
  \fi
}
%    \end{macrocode}
%    \end{macro}
%    \begin{macro}{\FibNum@PreCalcAux}
%    \begin{macrocode}
\def\FibNum@PreCalcAux#1/#2\fi{%
  \fi
  \edef\FibNum@temp{\BigIntCalcAdd#1!#2!}%
  \global\expandafter
  \let\csname FibNum@\the\FibNum@i\endcsname\FibNum@temp
  \ifnum\FibNum@i=\ltx@LocDimenA
    \xdef\FibNum@max{\the\FibNum@i}%
  \else
    \advance\FibNum@i\ltx@one
    \expandafter\FibNum@PreCalcAux\FibNum@temp/#1%
  \fi
}
%    \end{macrocode}
%    \end{macro}
%
% \subsection{Expandable calculations}
%
%    \begin{macro}{\fibnum}
%    \begin{macrocode}
\def\fibnum#1{%
  \romannumeral
  \expandafter\expandafter\expandafter\FibNum@Do\intcalcNum{#1}/%
}
%    \end{macrocode}
%    \end{macro}
%    \begin{macro}{\FibNum@Do}
%    \begin{macrocode}
\def\FibNum@Do#1/{%
  \ifnum#1<\ltx@zero
    \FibNum@ReturnAfterElseFiFi{%
      \ifodd#1 %
        \expandafter\expandafter\expandafter\ltx@zero
      \else
        \expandafter\expandafter\expandafter\ltx@zero
        \expandafter\expandafter\expandafter-%
      \fi
      \romannumeral
      \expandafter\FibNum@Do\ltx@gobble#1/%
    }%
  \else
    \ifnum\FibNum@max<#1 %
      \ltx@ReturnAfterElseFi{%
        \expandafter
        \FibNum@ExpCalc\number\expandafter\IntCalcInc\FibNum@max!%
        \expandafter\expandafter\expandafter/%
        \csname FibNum@\FibNum@max
        \expandafter\expandafter\expandafter\endcsname
        \expandafter\expandafter\expandafter/%
        \csname FibNum@\expandafter\IntCalcDec\FibNum@max!%
        \endcsname/%
        #1%
      }%
    \else
      \expandafter\expandafter\expandafter\ltx@zero
      \csname FibNum@#1\expandafter\expandafter\expandafter\endcsname
    \fi
  \fi
}
%    \end{macrocode}
%    \end{macro}
%    \begin{macro}{\FibNum@ReturnAfterElseFiFi}
%    \begin{macrocode}
\def\FibNum@ReturnAfterElseFiFi#1\else#2\fi\fi{\fi#1}
%    \end{macrocode}
%    \end{macro}
%    \begin{macro}{\FibNum@ExpCalc}
%    \begin{macrocode}
\def\FibNum@ExpCalc#1/#2/#3/#4\fi{%
  \fi
  \ifnum#1=#4 %
    \ltx@ReturnAfterElseFi{%
      \expandafter\expandafter\expandafter\ltx@zero
      \BigIntCalcAdd#2!#3!%
    }%
  \else
    \expandafter\FibNum@ExpCalc
    \number\IntCalcInc#1!%
    \expandafter\expandafter\expandafter/%
    \BigIntCalcAdd#2!#3!/%
    #2/#4%
  \fi
}
%    \end{macrocode}
%    \end{macro}
%
%    \begin{macrocode}
\FibNum@AtEnd%
%</package>
%    \end{macrocode}
%
% \section{Test}
%
% \subsection{Catcode checks for loading}
%
%    \begin{macrocode}
%<*test1>
%    \end{macrocode}
%    \begin{macrocode}
\catcode`\{=1 %
\catcode`\}=2 %
\catcode`\#=6 %
\catcode`\@=11 %
\expandafter\ifx\csname count@\endcsname\relax
  \countdef\count@=255 %
\fi
\expandafter\ifx\csname @gobble\endcsname\relax
  \long\def\@gobble#1{}%
\fi
\expandafter\ifx\csname @firstofone\endcsname\relax
  \long\def\@firstofone#1{#1}%
\fi
\expandafter\ifx\csname loop\endcsname\relax
  \expandafter\@firstofone
\else
  \expandafter\@gobble
\fi
{%
  \def\loop#1\repeat{%
    \def\body{#1}%
    \iterate
  }%
  \def\iterate{%
    \body
      \let\next\iterate
    \else
      \let\next\relax
    \fi
    \next
  }%
  \let\repeat=\fi
}%
\def\RestoreCatcodes{}
\count@=0 %
\loop
  \edef\RestoreCatcodes{%
    \RestoreCatcodes
    \catcode\the\count@=\the\catcode\count@\relax
  }%
\ifnum\count@<255 %
  \advance\count@ 1 %
\repeat

\def\RangeCatcodeInvalid#1#2{%
  \count@=#1\relax
  \loop
    \catcode\count@=15 %
  \ifnum\count@<#2\relax
    \advance\count@ 1 %
  \repeat
}
\def\RangeCatcodeCheck#1#2#3{%
  \count@=#1\relax
  \loop
    \ifnum#3=\catcode\count@
    \else
      \errmessage{%
        Character \the\count@\space
        with wrong catcode \the\catcode\count@\space
        instead of \number#3%
      }%
    \fi
  \ifnum\count@<#2\relax
    \advance\count@ 1 %
  \repeat
}
\def\space{ }
\expandafter\ifx\csname LoadCommand\endcsname\relax
  \def\LoadCommand{\input fibnum.sty\relax}%
\fi
\def\Test{%
  \RangeCatcodeInvalid{0}{47}%
  \RangeCatcodeInvalid{58}{64}%
  \RangeCatcodeInvalid{91}{96}%
  \RangeCatcodeInvalid{123}{255}%
  \catcode`\@=12 %
  \catcode`\\=0 %
  \catcode`\%=14 %
  \LoadCommand
  \RangeCatcodeCheck{0}{36}{15}%
  \RangeCatcodeCheck{37}{37}{14}%
  \RangeCatcodeCheck{38}{47}{15}%
  \RangeCatcodeCheck{48}{57}{12}%
  \RangeCatcodeCheck{58}{63}{15}%
  \RangeCatcodeCheck{64}{64}{12}%
  \RangeCatcodeCheck{65}{90}{11}%
  \RangeCatcodeCheck{91}{91}{15}%
  \RangeCatcodeCheck{92}{92}{0}%
  \RangeCatcodeCheck{93}{96}{15}%
  \RangeCatcodeCheck{97}{122}{11}%
  \RangeCatcodeCheck{123}{255}{15}%
  \RestoreCatcodes
}
\Test
\csname @@end\endcsname
\end
%    \end{macrocode}
%    \begin{macrocode}
%</test1>
%    \end{macrocode}
%
% \subsection{Test calculations}
%
%    \begin{macrocode}
%<*test-calc>
\catcode`\{=1 %
\catcode`\}=2 %
\catcode`\#=6 %
\catcode`\@=11 %
\begingroup\expandafter\expandafter\expandafter\endgroup
\expandafter\ifx\csname RequirePackage\endcsname\relax
  \input fibnum.sty\relax
\else
  \RequirePackage{fibnum}[2016/05/16]%
\fi
\def\TestSet{%
  \test{0}{0}%
  \test{1}{1}%
  \test{2}{1}%
  \test{3}{2}%
  \test{4}{3}%
  \test{5}{5}%
  \test{6}{8}%
  \test{7}{13}%
  \test{8}{21}%
  \test{9}{34}%
  \test{10}{55}%
  \test{11}{89}%
  \test{12}{144}%
  \test{13}{233}%
  \test{14}{377}%
  \test{15}{610}%
  \test{16}{987}%
  \test{17}{1597}%
  \test{18}{2584}%
  \test{19}{4181}%
  \test{20}{6765}%
  \test{21}{10946}%
  \test{22}{17711}%
  \test{23}{28657}%
  \test{24}{46368}%
  \test{25}{75025}%
  \test{26}{121393}%
  \test{27}{196418}%
  \test{28}{317811}%
  \test{29}{514229}%
  \test{30}{832040}%
  \test{31}{1346269}%
  \test{32}{2178309}%
  \test{33}{3524578}%
  \test{34}{5702887}%
  \test{35}{9227465}%
  \test{36}{14930352}%
  \test{37}{24157817}%
  \test{38}{39088169}%
  \test{39}{63245986}%
  \test{40}{102334155}%
  \test{41}{165580141}%
  \test{42}{267914296}%
  \test{43}{433494437}%
  \test{44}{701408733}%
  \test{45}{1134903170}%
  \test{46}{1836311903}%
  \test{47}{2971215073}%
  \test{48}{4807526976}%
  \test{49}{7778742049}%
  \test{50}{12586269025}%
  \test{51}{20365011074}%
  \test{52}{32951280099}%
  \test{53}{53316291173}%
  \test{54}{86267571272}%
  \test{55}{139583862445}%
  \test{56}{225851433717}%
  \test{57}{365435296162}%
  \test{58}{591286729879}%
  \test{59}{956722026041}%
  \test{60}{1548008755920}%
  \test{61}{2504730781961}%
  \test{62}{4052739537881}%
  \test{63}{6557470319842}%
  \test{64}{10610209857723}%
  \test{65}{17167680177565}%
  \test{66}{27777890035288}%
  \test{67}{44945570212853}%
  \test{68}{72723460248141}%
  \test{69}{117669030460994}%
  \test{70}{190392490709135}%
  \test{71}{308061521170129}%
  \test{72}{498454011879264}%
  \test{73}{806515533049393}%
}
\def\msg#{\immediate\write16}
\def\test#1#2{%
  \TestAux{#1}{#2}%
  \ifnum#1=0 %
  \else
    \ifodd#1 %
      \TestAux{-#1}{#2}%
    \else
      \TestAux{-#1}{-#2}%
    \fi
  \fi
}
\def\TestAux#1#2{%
  \def\Expected{#2}%
  \expandafter\expandafter\expandafter\def
  \expandafter\expandafter\expandafter\Result
  \expandafter\expandafter\expandafter{%
    \fibnum{#1}%
  }%
  \ltx@onelevel@sanitize\Result
  \ifx\Result\Expected
    \msg{* #1: ok.}%
  \else
    \msg{! fib(#1) = #2}%
    \errmessage{fib(#1) <> \Result}%
  \fi
}
\TestSet
\setbox0=\hbox{%
  \msg{* PreCalc{73}}%
  \fibnumPreCalc{73}%
}
\ifdim\wd0=0pt
\else
  \errmessage{Unwanted stuff in PreCalc}%
\fi
\TestSet
\csname @@end\endcsname\end
%</test-calc>
%    \end{macrocode}
%
% \section{Installation}
%
% \subsection{Download}
%
% \paragraph{Package.} This package is available on
% CTAN\footnote{\url{http://ctan.org/pkg/fibnum}}:
% \begin{description}
% \item[\CTAN{macros/latex/contrib/oberdiek/fibnum.dtx}] The source file.
% \item[\CTAN{macros/latex/contrib/oberdiek/fibnum.pdf}] Documentation.
% \end{description}
%
%
% \paragraph{Bundle.} All the packages of the bundle `oberdiek'
% are also available in a TDS compliant ZIP archive. There
% the packages are already unpacked and the documentation files
% are generated. The files and directories obey the TDS standard.
% \begin{description}
% \item[\CTAN{install/macros/latex/contrib/oberdiek.tds.zip}]
% \end{description}
% \emph{TDS} refers to the standard ``A Directory Structure
% for \TeX\ Files'' (\CTAN{tds/tds.pdf}). Directories
% with \xfile{texmf} in their name are usually organized this way.
%
% \subsection{Bundle installation}
%
% \paragraph{Unpacking.} Unpack the \xfile{oberdiek.tds.zip} in the
% TDS tree (also known as \xfile{texmf} tree) of your choice.
% Example (linux):
% \begin{quote}
%   |unzip oberdiek.tds.zip -d ~/texmf|
% \end{quote}
%
% \paragraph{Script installation.}
% Check the directory \xfile{TDS:scripts/oberdiek/} for
% scripts that need further installation steps.
% Package \xpackage{attachfile2} comes with the Perl script
% \xfile{pdfatfi.pl} that should be installed in such a way
% that it can be called as \texttt{pdfatfi}.
% Example (linux):
% \begin{quote}
%   |chmod +x scripts/oberdiek/pdfatfi.pl|\\
%   |cp scripts/oberdiek/pdfatfi.pl /usr/local/bin/|
% \end{quote}
%
% \subsection{Package installation}
%
% \paragraph{Unpacking.} The \xfile{.dtx} file is a self-extracting
% \docstrip\ archive. The files are extracted by running the
% \xfile{.dtx} through \plainTeX:
% \begin{quote}
%   \verb|tex fibnum.dtx|
% \end{quote}
%
% \paragraph{TDS.} Now the different files must be moved into
% the different directories in your installation TDS tree
% (also known as \xfile{texmf} tree):
% \begin{quote}
% \def\t{^^A
% \begin{tabular}{@{}>{\ttfamily}l@{ $\rightarrow$ }>{\ttfamily}l@{}}
%   fibnum.sty & tex/generic/oberdiek/fibnum.sty\\
%   fibnum.pdf & doc/latex/oberdiek/fibnum.pdf\\
%   test/fibnum-test1.tex & doc/latex/oberdiek/test/fibnum-test1.tex\\
%   test/fibnum-test-calc.tex & doc/latex/oberdiek/test/fibnum-test-calc.tex\\
%   fibnum.dtx & source/latex/oberdiek/fibnum.dtx\\
% \end{tabular}^^A
% }^^A
% \sbox0{\t}^^A
% \ifdim\wd0>\linewidth
%   \begingroup
%     \advance\linewidth by\leftmargin
%     \advance\linewidth by\rightmargin
%   \edef\x{\endgroup
%     \def\noexpand\lw{\the\linewidth}^^A
%   }\x
%   \def\lwbox{^^A
%     \leavevmode
%     \hbox to \linewidth{^^A
%       \kern-\leftmargin\relax
%       \hss
%       \usebox0
%       \hss
%       \kern-\rightmargin\relax
%     }^^A
%   }^^A
%   \ifdim\wd0>\lw
%     \sbox0{\small\t}^^A
%     \ifdim\wd0>\linewidth
%       \ifdim\wd0>\lw
%         \sbox0{\footnotesize\t}^^A
%         \ifdim\wd0>\linewidth
%           \ifdim\wd0>\lw
%             \sbox0{\scriptsize\t}^^A
%             \ifdim\wd0>\linewidth
%               \ifdim\wd0>\lw
%                 \sbox0{\tiny\t}^^A
%                 \ifdim\wd0>\linewidth
%                   \lwbox
%                 \else
%                   \usebox0
%                 \fi
%               \else
%                 \lwbox
%               \fi
%             \else
%               \usebox0
%             \fi
%           \else
%             \lwbox
%           \fi
%         \else
%           \usebox0
%         \fi
%       \else
%         \lwbox
%       \fi
%     \else
%       \usebox0
%     \fi
%   \else
%     \lwbox
%   \fi
% \else
%   \usebox0
% \fi
% \end{quote}
% If you have a \xfile{docstrip.cfg} that configures and enables \docstrip's
% TDS installing feature, then some files can already be in the right
% place, see the documentation of \docstrip.
%
% \subsection{Refresh file name databases}
%
% If your \TeX~distribution
% (\teTeX, \mikTeX, \dots) relies on file name databases, you must refresh
% these. For example, \teTeX\ users run \verb|texhash| or
% \verb|mktexlsr|.
%
% \subsection{Some details for the interested}
%
% \paragraph{Attached source.}
%
% The PDF documentation on CTAN also includes the
% \xfile{.dtx} source file. It can be extracted by
% AcrobatReader 6 or higher. Another option is \textsf{pdftk},
% e.g. unpack the file into the current directory:
% \begin{quote}
%   \verb|pdftk fibnum.pdf unpack_files output .|
% \end{quote}
%
% \paragraph{Unpacking with \LaTeX.}
% The \xfile{.dtx} chooses its action depending on the format:
% \begin{description}
% \item[\plainTeX:] Run \docstrip\ and extract the files.
% \item[\LaTeX:] Generate the documentation.
% \end{description}
% If you insist on using \LaTeX\ for \docstrip\ (really,
% \docstrip\ does not need \LaTeX), then inform the autodetect routine
% about your intention:
% \begin{quote}
%   \verb|latex \let\install=y% \iffalse meta-comment
%
% File: fibnum.dtx
% Version: 2016/05/16 v1.1
% Info: Fibonacci numbers
%
% Copyright (C) 2012 by
%    Heiko Oberdiek <heiko.oberdiek at googlemail.com>
%    2016
%    https://github.com/ho-tex/oberdiek/issues
%
% This work may be distributed and/or modified under the
% conditions of the LaTeX Project Public License, either
% version 1.3c of this license or (at your option) any later
% version. This version of this license is in
%    http://www.latex-project.org/lppl/lppl-1-3c.txt
% and the latest version of this license is in
%    http://www.latex-project.org/lppl.txt
% and version 1.3 or later is part of all distributions of
% LaTeX version 2005/12/01 or later.
%
% This work has the LPPL maintenance status "maintained".
%
% This Current Maintainer of this work is Heiko Oberdiek.
%
% The Base Interpreter refers to any `TeX-Format',
% because some files are installed in TDS:tex/generic//.
%
% This work consists of the main source file fibnum.dtx
% and the derived files
%    fibnum.sty, fibnum.pdf, fibnum.ins, fibnum.drv, fibnum.bib,
%    fibnum-test1.tex, fibnum-test-calc.tex.
%
% Distribution:
%    CTAN:macros/latex/contrib/oberdiek/fibnum.dtx
%    CTAN:macros/latex/contrib/oberdiek/fibnum.pdf
%
% Unpacking:
%    (a) If fibnum.ins is present:
%           tex fibnum.ins
%    (b) Without fibnum.ins:
%           tex fibnum.dtx
%    (c) If you insist on using LaTeX
%           latex \let\install=y% \iffalse meta-comment
%
% File: fibnum.dtx
% Version: 2016/05/16 v1.1
% Info: Fibonacci numbers
%
% Copyright (C) 2012 by
%    Heiko Oberdiek <heiko.oberdiek at googlemail.com>
%    2016
%    https://github.com/ho-tex/oberdiek/issues
%
% This work may be distributed and/or modified under the
% conditions of the LaTeX Project Public License, either
% version 1.3c of this license or (at your option) any later
% version. This version of this license is in
%    http://www.latex-project.org/lppl/lppl-1-3c.txt
% and the latest version of this license is in
%    http://www.latex-project.org/lppl.txt
% and version 1.3 or later is part of all distributions of
% LaTeX version 2005/12/01 or later.
%
% This work has the LPPL maintenance status "maintained".
%
% This Current Maintainer of this work is Heiko Oberdiek.
%
% The Base Interpreter refers to any `TeX-Format',
% because some files are installed in TDS:tex/generic//.
%
% This work consists of the main source file fibnum.dtx
% and the derived files
%    fibnum.sty, fibnum.pdf, fibnum.ins, fibnum.drv, fibnum.bib,
%    fibnum-test1.tex, fibnum-test-calc.tex.
%
% Distribution:
%    CTAN:macros/latex/contrib/oberdiek/fibnum.dtx
%    CTAN:macros/latex/contrib/oberdiek/fibnum.pdf
%
% Unpacking:
%    (a) If fibnum.ins is present:
%           tex fibnum.ins
%    (b) Without fibnum.ins:
%           tex fibnum.dtx
%    (c) If you insist on using LaTeX
%           latex \let\install=y% \iffalse meta-comment
%
% File: fibnum.dtx
% Version: 2016/05/16 v1.1
% Info: Fibonacci numbers
%
% Copyright (C) 2012 by
%    Heiko Oberdiek <heiko.oberdiek at googlemail.com>
%    2016
%    https://github.com/ho-tex/oberdiek/issues
%
% This work may be distributed and/or modified under the
% conditions of the LaTeX Project Public License, either
% version 1.3c of this license or (at your option) any later
% version. This version of this license is in
%    http://www.latex-project.org/lppl/lppl-1-3c.txt
% and the latest version of this license is in
%    http://www.latex-project.org/lppl.txt
% and version 1.3 or later is part of all distributions of
% LaTeX version 2005/12/01 or later.
%
% This work has the LPPL maintenance status "maintained".
%
% This Current Maintainer of this work is Heiko Oberdiek.
%
% The Base Interpreter refers to any `TeX-Format',
% because some files are installed in TDS:tex/generic//.
%
% This work consists of the main source file fibnum.dtx
% and the derived files
%    fibnum.sty, fibnum.pdf, fibnum.ins, fibnum.drv, fibnum.bib,
%    fibnum-test1.tex, fibnum-test-calc.tex.
%
% Distribution:
%    CTAN:macros/latex/contrib/oberdiek/fibnum.dtx
%    CTAN:macros/latex/contrib/oberdiek/fibnum.pdf
%
% Unpacking:
%    (a) If fibnum.ins is present:
%           tex fibnum.ins
%    (b) Without fibnum.ins:
%           tex fibnum.dtx
%    (c) If you insist on using LaTeX
%           latex \let\install=y\input{fibnum.dtx}
%        (quote the arguments according to the demands of your shell)
%
% Documentation:
%    (a) If fibnum.drv is present:
%           latex fibnum.drv
%    (b) Without fibnum.drv:
%           latex fibnum.dtx; ...
%    The class ltxdoc loads the configuration file ltxdoc.cfg
%    if available. Here you can specify further options, e.g.
%    use A4 as paper format:
%       \PassOptionsToClass{a4paper}{article}
%
%    Programm calls to get the documentation (example):
%       pdflatex fibnum.dtx
%       bibtex fibnum.aux
%       makeindex -s gind.ist fibnum.idx
%       pdflatex fibnum.dtx
%       makeindex -s gind.ist fibnum.idx
%       pdflatex fibnum.dtx
%
% Installation:
%    TDS:tex/generic/oberdiek/fibnum.sty
%    TDS:doc/latex/oberdiek/fibnum.pdf
%    TDS:doc/latex/oberdiek/test/fibnum-test1.tex
%    TDS:doc/latex/oberdiek/test/fibnum-test-calc.tex
%    TDS:source/latex/oberdiek/fibnum.dtx
%
%<*ignore>
\begingroup
  \catcode123=1 %
  \catcode125=2 %
  \def\x{LaTeX2e}%
\expandafter\endgroup
\ifcase 0\ifx\install y1\fi\expandafter
         \ifx\csname processbatchFile\endcsname\relax\else1\fi
         \ifx\fmtname\x\else 1\fi\relax
\else\csname fi\endcsname
%</ignore>
%<*install>
\input docstrip.tex
\Msg{************************************************************************}
\Msg{* Installation}
\Msg{* Package: fibnum 2016/05/16 v1.1 Fibonacci numbers (HO)}
\Msg{************************************************************************}

\keepsilent
\askforoverwritefalse

\let\MetaPrefix\relax
\preamble

This is a generated file.

Project: fibnum
Version: 2016/05/16 v1.1

Copyright (C) 2012 by
   Heiko Oberdiek <heiko.oberdiek at googlemail.com>

This work may be distributed and/or modified under the
conditions of the LaTeX Project Public License, either
version 1.3c of this license or (at your option) any later
version. This version of this license is in
   http://www.latex-project.org/lppl/lppl-1-3c.txt
and the latest version of this license is in
   http://www.latex-project.org/lppl.txt
and version 1.3 or later is part of all distributions of
LaTeX version 2005/12/01 or later.

This work has the LPPL maintenance status "maintained".

This Current Maintainer of this work is Heiko Oberdiek.

The Base Interpreter refers to any `TeX-Format',
because some files are installed in TDS:tex/generic//.

This work consists of the main source file fibnum.dtx
and the derived files
   fibnum.sty, fibnum.pdf, fibnum.ins, fibnum.drv, fibnum.bib,
   fibnum-test1.tex, fibnum-test-calc.tex.

\endpreamble
\let\MetaPrefix\DoubleperCent

\generate{%
  \file{fibnum.ins}{\from{fibnum.dtx}{install}}%
  \file{fibnum.drv}{\from{fibnum.dtx}{driver}}%
  \nopreamble
  \nopostamble
  \file{fibnum.bib}{\from{fibnum.dtx}{bib}}%
  \usepreamble\defaultpreamble
  \usepostamble\defaultpostamble
  \usedir{tex/generic/oberdiek}%
  \file{fibnum.sty}{\from{fibnum.dtx}{package}}%
  \usedir{doc/latex/oberdiek/test}%
  \file{fibnum-test1.tex}{\from{fibnum.dtx}{test1}}%
  \file{fibnum-test-calc.tex}{\from{fibnum.dtx}{test-calc}}%
}

\catcode32=13\relax% active space
\let =\space%
\Msg{************************************************************************}
\Msg{*}
\Msg{* To finish the installation you have to move the following}
\Msg{* file into a directory searched by TeX:}
\Msg{*}
\Msg{*     fibnum.sty}
\Msg{*}
\Msg{* To produce the documentation run the file `fibnum.drv'}
\Msg{* through LaTeX.}
\Msg{*}
\Msg{* Happy TeXing!}
\Msg{*}
\Msg{************************************************************************}

\endbatchfile
%</install>
%<*bib>
@online{texhax:abraham,
  author={Abraham, Jan},
  title={[texhax] Beginner in TEX MACRO to compute functions},
  date={2012-04-07},
  url={http://tug.org/pipermail/texhax/2012-April/019146.html},
  urldate={2012-04-08},
}
@article{knuth:negafibonacci,
  author={Knuth, Donald E.},
  title={Negafibonacci Numbers and the Hyperbolic Plane},
  date={2008-12-11},
  url={http://research.allacademic.com/meta/p206842_index.html},
}
@online{wikipedia:negafibonacci,
  author={{Wikipedia contributors}},
  organization={{Wikipedia, The Free Encyclopedia}},
  title={Fibonacci numbers},
  language={langenglish},
  version={486266088},
  date={2012-04-08},
  url={http://en.wikipedia.org/w/index.php?title=Fibonacci_number&oldid=486266088},
  urldate={2012-04-08},
}
%</bib>
%<*ignore>
\fi
%</ignore>
%<*driver>
\NeedsTeXFormat{LaTeX2e}
\ProvidesFile{fibnum.drv}%
  [2016/05/16 v1.1 Fibonacci numbers (HO)]%
\documentclass{ltxdoc}
\usepackage{amsmath,amsfonts}
\usepackage{siunitx}
\usepackage{array}
\usepackage{tabularx}
\usepackage{fibnum}[2016/05/16]
\usepackage{holtxdoc}[2011/11/22]
\usepackage{csquotes}
\usepackage[
  bibencoding=ascii,
  alldates=iso8601,
]{biblatex}[2011/11/13]
\bibliography{oberdiek-source}
\bibliography{fibnum}
\begin{document}
  \DocInput{fibnum.dtx}%
\end{document}
%</driver>
% \fi
%
%
% \CharacterTable
%  {Upper-case    \A\B\C\D\E\F\G\H\I\J\K\L\M\N\O\P\Q\R\S\T\U\V\W\X\Y\Z
%   Lower-case    \a\b\c\d\e\f\g\h\i\j\k\l\m\n\o\p\q\r\s\t\u\v\w\x\y\z
%   Digits        \0\1\2\3\4\5\6\7\8\9
%   Exclamation   \!     Double quote  \"     Hash (number) \#
%   Dollar        \$     Percent       \%     Ampersand     \&
%   Acute accent  \'     Left paren    \(     Right paren   \)
%   Asterisk      \*     Plus          \+     Comma         \,
%   Minus         \-     Point         \.     Solidus       \/
%   Colon         \:     Semicolon     \;     Less than     \<
%   Equals        \=     Greater than  \>     Question mark \?
%   Commercial at \@     Left bracket  \[     Backslash     \\
%   Right bracket \]     Circumflex    \^     Underscore    \_
%   Grave accent  \`     Left brace    \{     Vertical bar  \|
%   Right brace   \}     Tilde         \~}
%
% \GetFileInfo{fibnum.drv}
%
% \title{The \xpackage{fibnum} package}
% \date{2016/05/16 v1.1}
% \author{Heiko Oberdiek\thanks
% {Please report any issues at https://github.com/ho-tex/oberdiek/issues}\\
% \xemail{heiko.oberdiek at googlemail.com}}
%
% \maketitle
%
% \begin{abstract}
% The package \xpackage{fibnum} provides expandable fibonacci
% numbers for both \hologo{LaTeX} and \hologo{plainTeX}.
% \end{abstract}
%
% \tableofcontents
%
% \section{Documentation}
%
% In the mailing list \textsf{texhax} Jan Abraham asked the question,
% how to get Fibonacci numbers in \hologo{TeX} \cite{texhax:abraham}:
% \begin{quote}
% Write a Macro in \hologo{TeX} that compute the function |\fib{m}|
% All fibonacci numbers from 1 to $m$ ($m < 40$).
% \end{quote}
% This packages provides an expandable implementation for the
% calculation of these numbers for a much larger set of indexes.
% For practical reasons the index is restricted to the same limitations
% that apply for \hologo{TeX} integer numbers.
% The range of the Fibonacci numbers, however, are not limited
% by the algorithm. They are only restricted to memory limitations,
% if they are hit.
%
% The package is loaded as \hologo{LaTeX} package in \hologo{LaTeX}:
% \begin{quote}
%   |\usepackage{fibnum}|
% \end{quote}
% and as file in \hologo{plainTeX}:
% \begin{quote}
%   |\input fibnum.sty|
% \end{quote}
% The package does not know any options and it provides
% the macros \cs{fibnum} and \cs{fibnumPreCalc}.
%
% \begin{declcs}{fibnum} \M{index}
% \end{declcs}
% Macro \cs{fibnum} expects a \hologo{TeX} number as \meta{index}
% in the official \hologo{TeX} number range from $-(2^{31}-1)$ up to
% $2^{31}-1$. In exact two expansion steps the macro expands to
% the Fibnoacci number $F_{\text{\meta{index}}}$. In case of a negative
% \meta{index}, the ``negafibonacci'' number \cite{wikipedia:negafibonacci}
% is used. Formally the Fibonacci number $F_n$ with integer
% index~$n$, $n\in\mathbb{Z}$ and
% $n\in[\num{-2147483647},\num{2147483647}]$ that is returned by macro
% \cs{fibnum} with numerical argument $n$ is defined the following way:
% \begin{gather}
%   \label{eq:def}
%   F_n =
%   \begin{cases}
%     0 & \text{for $n=0$}\\
%     1 & \text{for $n=1$}\\
%     F_{n-1} + F_{n-2} & \text{for $n>1$}\\
%     (-1)^{n+1}F_n & \text{for $n<0$}
%   \end{cases}
% \end{gather}
% Examples:
% \begin{quote}
%   \makeatletter
%   \def\x#1{\cs{fibnum}\{#1\}&
%     \edef\X{\fibnum{#1}}\edef\Y{\expandafter\ltx@car\X\@nil}^^A
%     \if-\Y
%       \edef\X{\expandafter\ltx@cdr\X\@nil}^^A
%       \noindent
%       \llap{-}\X
%     \else
%       \X
%     \fi
%     \tabularnewline
%   }
%   \def\y{\multicolumn{1}{@{}c@{}}{$\vdots$}\tabularnewline}
%   \DeclareUrlCommand\UrlNum{^^A
%     \urlstyle{tt}^^A
%     \def\UrlBreaks{\do\0\do\1\do\2\do\3\do\4\do\5\do\6\do\7\do\8\do\9}^^A
%   }
%   \begin{tabularx}{\dimexpr\linewidth+5.7pt\relax}{@{}>{\ttfamily}l@{ $\rightarrow$ \hphantom{\ttfamily-}}>{\ttfamily}X@{}}
%     \x{-6}
%     \x{-5}
%     \x{-4}
%     \x{-3}
%     \x{-2}
%     \x{-1}
%     \x{0}
%     \x{1}
%     \x{2}
%     \x{3}
%     \x{4}
%     \x{5}
%     \x{6}
%     \y
%     \x{10}
%     \y
%     \x{46}
%     \y
%     \cs{fibnum}\{100\} & 354224848179261915075
%     \tabularnewline
%     \y
%     \cs{fibnum}\{200\} & 280571172992510140037611932413038677189525
%     \tabularnewline
%     \y
%     \cs{fibnum}\{1000\} &
%       \raggedright
%       \UrlNum{^^A
%         434665576869374564356885276750406258025646^^A
%         605173717804024817290895365554179490518904^^A
%         038798400792551692959225930803226347752096^^A
%         896232398733224711616429964409065331879382^^A
%         98969649928516003704476137795166849228875^^A
%       }
%     \tabularnewline
%   \end{tabularx}\kern-5.7pt\mbox{}
% \end{quote}
%
% \begin{declcs}{fibnumPreCalc} \M{index}
% \end{declcs}
% The package already provides precalculated Fibonacci numbers up to
% index~46. That means that calculations are not necessary for
% Fibonacci numbers that fit into the range of \hologo{TeX}
% numbers. Because macro \cs{fibnum} is expandable, it cannot
% store calculated Fibonacci numbers for later use. Macro definitions
% are forbidden in expandable contexts. If larger Fibonacci numbers
% are used more than once, than the compilation time can be shortened
% by calculating and storing the Fibonacci numbers beforehand.
% The argument \meta{index} is a \hologo{TeX} number and macro
% \cs{fibnumPreCalc} ensures that the Fibonacci numbers
% $F_0$ up to $F_{\lvert\text{\meta{index}}\rvert}$ that are not
% already known are calculated
% and stored in internal macros. Internally only non-negative
% Fibonacci numbers are stored. If \meta{index} is negative, then
% the needed positive Fibonacci numbers are calculated and stored.
% Example:
% \begin{quote}
%   \def\x#1{\begingroup\itshape\texttt{\%} #1\endgroup}
%   |\fibnumPreCalc{50}|\\
%   \x{calculates and stores the values for indexes 47..50.}\\
%   \x{(Values for 0..46 are already stored by the package.)}\\
%   |\fibnum{49}| \x{uses the stored value}\\
%   |\fibnum{51}|
%   \x{only calculates $F_{51}$ from stored values $F_{49}$ and $F_{50}$}\\
%   |\fibnumPreCalc{100}|\\
%   \x{calculates and stores the values for indexes 51..100}\\
%   |\fibnum{100}| \x{uses the stored value for $F_{100}$}\\
%   |\fibnum{-100}|
%   \x{uses the stored value for $F_{100}$}\\
%   \x{$F_{-100}=-F_{100}$ according to equation \eqref{eq:def}.}
% \end{quote}
%
% \StopEventually{
% }
%
% \section{Implementation}
%
% \subsection{Identification}
%
%    \begin{macrocode}
%<*package>
%    \end{macrocode}
%    Reload check, especially if the package is not used with \LaTeX.
%    \begin{macrocode}
\begingroup\catcode61\catcode48\catcode32=10\relax%
  \catcode13=5 % ^^M
  \endlinechar=13 %
  \catcode35=6 % #
  \catcode39=12 % '
  \catcode44=12 % ,
  \catcode45=12 % -
  \catcode46=12 % .
  \catcode58=12 % :
  \catcode64=11 % @
  \catcode123=1 % {
  \catcode125=2 % }
  \expandafter\let\expandafter\x\csname ver@fibnum.sty\endcsname
  \ifx\x\relax % plain-TeX, first loading
  \else
    \def\empty{}%
    \ifx\x\empty % LaTeX, first loading,
      % variable is initialized, but \ProvidesPackage not yet seen
    \else
      \expandafter\ifx\csname PackageInfo\endcsname\relax
        \def\x#1#2{%
          \immediate\write-1{Package #1 Info: #2.}%
        }%
      \else
        \def\x#1#2{\PackageInfo{#1}{#2, stopped}}%
      \fi
      \x{fibnum}{The package is already loaded}%
      \aftergroup\endinput
    \fi
  \fi
\endgroup%
%    \end{macrocode}
%    Package identification:
%    \begin{macrocode}
\begingroup\catcode61\catcode48\catcode32=10\relax%
  \catcode13=5 % ^^M
  \endlinechar=13 %
  \catcode35=6 % #
  \catcode39=12 % '
  \catcode40=12 % (
  \catcode41=12 % )
  \catcode44=12 % ,
  \catcode45=12 % -
  \catcode46=12 % .
  \catcode47=12 % /
  \catcode58=12 % :
  \catcode64=11 % @
  \catcode91=12 % [
  \catcode93=12 % ]
  \catcode123=1 % {
  \catcode125=2 % }
  \expandafter\ifx\csname ProvidesPackage\endcsname\relax
    \def\x#1#2#3[#4]{\endgroup
      \immediate\write-1{Package: #3 #4}%
      \xdef#1{#4}%
    }%
  \else
    \def\x#1#2[#3]{\endgroup
      #2[{#3}]%
      \ifx#1\@undefined
        \xdef#1{#3}%
      \fi
      \ifx#1\relax
        \xdef#1{#3}%
      \fi
    }%
  \fi
\expandafter\x\csname ver@fibnum.sty\endcsname
\ProvidesPackage{fibnum}%
  [2016/05/16 v1.1 Fibonacci numbers (HO)]%
%    \end{macrocode}
%
%    \begin{macrocode}
\begingroup\catcode61\catcode48\catcode32=10\relax%
  \catcode13=5 % ^^M
  \endlinechar=13 %
  \catcode123=1 % {
  \catcode125=2 % }
  \catcode64=11 % @
  \def\x{\endgroup
    \expandafter\edef\csname FibNum@AtEnd\endcsname{%
      \endlinechar=\the\endlinechar\relax
      \catcode13=\the\catcode13\relax
      \catcode32=\the\catcode32\relax
      \catcode35=\the\catcode35\relax
      \catcode61=\the\catcode61\relax
      \catcode64=\the\catcode64\relax
      \catcode123=\the\catcode123\relax
      \catcode125=\the\catcode125\relax
    }%
  }%
\x\catcode61\catcode48\catcode32=10\relax%
\catcode13=5 % ^^M
\endlinechar=13 %
\catcode35=6 % #
\catcode64=11 % @
\catcode123=1 % {
\catcode125=2 % }
\def\TMP@EnsureCode#1#2{%
  \edef\FibNum@AtEnd{%
    \FibNum@AtEnd
    \catcode#1=\the\catcode#1\relax
  }%
  \catcode#1=#2\relax
}
\TMP@EnsureCode{33}{12}% !
%\TMP@EnsureCode{36}{3}% $
%\TMP@EnsureCode{38}{4}% &
\TMP@EnsureCode{40}{12}% (
\TMP@EnsureCode{41}{12}% )
\TMP@EnsureCode{45}{12}% -
\TMP@EnsureCode{46}{12}% .
\TMP@EnsureCode{47}{12}% /
\TMP@EnsureCode{58}{12}% :
\TMP@EnsureCode{60}{12}% <
\TMP@EnsureCode{62}{12}% >
\TMP@EnsureCode{91}{12}% [
%\TMP@EnsureCode{96}{12}% `
\TMP@EnsureCode{93}{12}% ]
%\TMP@EnsureCode{94}{12}% ^ (superscript) (!)
%\TMP@EnsureCode{124}{12}% |
\edef\FibNum@AtEnd{\FibNum@AtEnd\noexpand\endinput}
%    \end{macrocode}
%
% \subsection{Package resources}
%
%    \begin{macrocode}
\begingroup\expandafter\expandafter\expandafter\endgroup
\expandafter\ifx\csname RequirePackage\endcsname\relax
  \def\TMP@RequirePackage#1[#2]{%
    \begingroup\expandafter\expandafter\expandafter\endgroup
    \expandafter\ifx\csname ver@#1.sty\endcsname\relax
      \input #1.sty\relax
    \fi
  }%
  \TMP@RequirePackage{ltxcmds}[2011/04/18]%
  \TMP@RequirePackage{intcalc}[2007/09/27]%
  \TMP@RequirePackage{bigintcalc}[2007/11/11]%
\else
  \RequirePackage{ltxcmds}[2011/04/18]%
  \RequirePackage{intcalc}[2007/09/27]%
  \RequirePackage{bigintcalc}[2007/11/11]%
\fi
%    \end{macrocode}
%
% \subsection{Setup precalculated values}
%
%    \begin{macrocode}
\def\FibNum@temp#1{%
  \expandafter\def\csname FibNum@#1\endcsname
}
\catcode46=9 % dots are ignored
\FibNum@temp{0}{0}
\FibNum@temp{1}{1}
\FibNum@temp{2}{1}
\FibNum@temp{3}{2}
\FibNum@temp{4}{3}
\FibNum@temp{5}{5}
\FibNum@temp{6}{8}
\FibNum@temp{7}{13}
\FibNum@temp{8}{21}
\FibNum@temp{9}{34}
\FibNum@temp{10}{55}
\FibNum@temp{11}{89}
\FibNum@temp{12}{144}
\FibNum@temp{13}{233}
\FibNum@temp{14}{377}
\FibNum@temp{15}{610}
\FibNum@temp{16}{987}
\FibNum@temp{17}{1.597}
\FibNum@temp{18}{2.584}
\FibNum@temp{19}{4.181}
\FibNum@temp{20}{6.765}
\FibNum@temp{21}{10.946}
\FibNum@temp{22}{17.711}
\FibNum@temp{23}{28.657}
\FibNum@temp{24}{46.368}
\FibNum@temp{25}{75.025}
\FibNum@temp{26}{121.393}
\FibNum@temp{27}{196.418}
\FibNum@temp{28}{317.811}
\FibNum@temp{29}{514.229}
\FibNum@temp{30}{832.040}
\FibNum@temp{31}{1.346.269}
\FibNum@temp{32}{2.178.309}
\FibNum@temp{33}{3.524.578}
\FibNum@temp{34}{5.702.887}
\FibNum@temp{35}{9.227.465}
\FibNum@temp{36}{14.930.352}
\FibNum@temp{37}{24.157.817}
\FibNum@temp{38}{39.088.169}
\FibNum@temp{39}{63.245.986}
\FibNum@temp{40}{102.334.155}
\FibNum@temp{41}{165.580.141}
\FibNum@temp{42}{267.914.296}
\FibNum@temp{43}{433.494.437}
\FibNum@temp{44}{701.408.733}
\FibNum@temp{45}{1.134.903.170}
\FibNum@temp{46}{1.836.311.903}
%    \end{macrocode}
%    \begin{macro}{\FibNum@max}
%    \begin{macrocode}
\def\FibNum@max{46}
%    \end{macrocode}
%    \end{macro}
%
% \subsection{Macros for precalculating values}
%
%    \begin{macro}{\fibnumPreCalc}
%    \begin{macrocode}
\def\fibnumPreCalc#1{%
  \expandafter\expandafter\expandafter
  \FibNum@PreCalc\intcalcNum{#1}/%
}
%    \end{macrocode}
%    \end{macro}
%    \begin{macro}{\FibNum@PreCalc}
%    \begin{macrocode}
\def\FibNum@PreCalc#1/{%
  \ifnum#1<\ltx@zero
    \expandafter\FibNum@PreCalc\ltx@gobble#1/%
  \else
    \ifnum#1>\FibNum@max
      \begingroup
        \ltx@LocDimenA=#1sp\relax
        \countdef\FibNum@i=255\relax
        \FibNum@i=\FibNum@max\relax
        \edef\FibNum@temp{%
          \csname FibNum@\the\FibNum@i\endcsname/%
        }%
        \advance\FibNum@i by -1\relax
        \edef\FibNum@temp{%
          \FibNum@temp
          \csname FibNum@\the\FibNum@i\endcsname
        }%
        \advance\FibNum@i\ltx@two
        \iftrue
          \expandafter\FibNum@PreCalcAux\FibNum@temp
        \fi
      \endgroup
    \fi
  \fi
}
%    \end{macrocode}
%    \end{macro}
%    \begin{macro}{\FibNum@PreCalcAux}
%    \begin{macrocode}
\def\FibNum@PreCalcAux#1/#2\fi{%
  \fi
  \edef\FibNum@temp{\BigIntCalcAdd#1!#2!}%
  \global\expandafter
  \let\csname FibNum@\the\FibNum@i\endcsname\FibNum@temp
  \ifnum\FibNum@i=\ltx@LocDimenA
    \xdef\FibNum@max{\the\FibNum@i}%
  \else
    \advance\FibNum@i\ltx@one
    \expandafter\FibNum@PreCalcAux\FibNum@temp/#1%
  \fi
}
%    \end{macrocode}
%    \end{macro}
%
% \subsection{Expandable calculations}
%
%    \begin{macro}{\fibnum}
%    \begin{macrocode}
\def\fibnum#1{%
  \romannumeral
  \expandafter\expandafter\expandafter\FibNum@Do\intcalcNum{#1}/%
}
%    \end{macrocode}
%    \end{macro}
%    \begin{macro}{\FibNum@Do}
%    \begin{macrocode}
\def\FibNum@Do#1/{%
  \ifnum#1<\ltx@zero
    \FibNum@ReturnAfterElseFiFi{%
      \ifodd#1 %
        \expandafter\expandafter\expandafter\ltx@zero
      \else
        \expandafter\expandafter\expandafter\ltx@zero
        \expandafter\expandafter\expandafter-%
      \fi
      \romannumeral
      \expandafter\FibNum@Do\ltx@gobble#1/%
    }%
  \else
    \ifnum\FibNum@max<#1 %
      \ltx@ReturnAfterElseFi{%
        \expandafter
        \FibNum@ExpCalc\number\expandafter\IntCalcInc\FibNum@max!%
        \expandafter\expandafter\expandafter/%
        \csname FibNum@\FibNum@max
        \expandafter\expandafter\expandafter\endcsname
        \expandafter\expandafter\expandafter/%
        \csname FibNum@\expandafter\IntCalcDec\FibNum@max!%
        \endcsname/%
        #1%
      }%
    \else
      \expandafter\expandafter\expandafter\ltx@zero
      \csname FibNum@#1\expandafter\expandafter\expandafter\endcsname
    \fi
  \fi
}
%    \end{macrocode}
%    \end{macro}
%    \begin{macro}{\FibNum@ReturnAfterElseFiFi}
%    \begin{macrocode}
\def\FibNum@ReturnAfterElseFiFi#1\else#2\fi\fi{\fi#1}
%    \end{macrocode}
%    \end{macro}
%    \begin{macro}{\FibNum@ExpCalc}
%    \begin{macrocode}
\def\FibNum@ExpCalc#1/#2/#3/#4\fi{%
  \fi
  \ifnum#1=#4 %
    \ltx@ReturnAfterElseFi{%
      \expandafter\expandafter\expandafter\ltx@zero
      \BigIntCalcAdd#2!#3!%
    }%
  \else
    \expandafter\FibNum@ExpCalc
    \number\IntCalcInc#1!%
    \expandafter\expandafter\expandafter/%
    \BigIntCalcAdd#2!#3!/%
    #2/#4%
  \fi
}
%    \end{macrocode}
%    \end{macro}
%
%    \begin{macrocode}
\FibNum@AtEnd%
%</package>
%    \end{macrocode}
%
% \section{Test}
%
% \subsection{Catcode checks for loading}
%
%    \begin{macrocode}
%<*test1>
%    \end{macrocode}
%    \begin{macrocode}
\catcode`\{=1 %
\catcode`\}=2 %
\catcode`\#=6 %
\catcode`\@=11 %
\expandafter\ifx\csname count@\endcsname\relax
  \countdef\count@=255 %
\fi
\expandafter\ifx\csname @gobble\endcsname\relax
  \long\def\@gobble#1{}%
\fi
\expandafter\ifx\csname @firstofone\endcsname\relax
  \long\def\@firstofone#1{#1}%
\fi
\expandafter\ifx\csname loop\endcsname\relax
  \expandafter\@firstofone
\else
  \expandafter\@gobble
\fi
{%
  \def\loop#1\repeat{%
    \def\body{#1}%
    \iterate
  }%
  \def\iterate{%
    \body
      \let\next\iterate
    \else
      \let\next\relax
    \fi
    \next
  }%
  \let\repeat=\fi
}%
\def\RestoreCatcodes{}
\count@=0 %
\loop
  \edef\RestoreCatcodes{%
    \RestoreCatcodes
    \catcode\the\count@=\the\catcode\count@\relax
  }%
\ifnum\count@<255 %
  \advance\count@ 1 %
\repeat

\def\RangeCatcodeInvalid#1#2{%
  \count@=#1\relax
  \loop
    \catcode\count@=15 %
  \ifnum\count@<#2\relax
    \advance\count@ 1 %
  \repeat
}
\def\RangeCatcodeCheck#1#2#3{%
  \count@=#1\relax
  \loop
    \ifnum#3=\catcode\count@
    \else
      \errmessage{%
        Character \the\count@\space
        with wrong catcode \the\catcode\count@\space
        instead of \number#3%
      }%
    \fi
  \ifnum\count@<#2\relax
    \advance\count@ 1 %
  \repeat
}
\def\space{ }
\expandafter\ifx\csname LoadCommand\endcsname\relax
  \def\LoadCommand{\input fibnum.sty\relax}%
\fi
\def\Test{%
  \RangeCatcodeInvalid{0}{47}%
  \RangeCatcodeInvalid{58}{64}%
  \RangeCatcodeInvalid{91}{96}%
  \RangeCatcodeInvalid{123}{255}%
  \catcode`\@=12 %
  \catcode`\\=0 %
  \catcode`\%=14 %
  \LoadCommand
  \RangeCatcodeCheck{0}{36}{15}%
  \RangeCatcodeCheck{37}{37}{14}%
  \RangeCatcodeCheck{38}{47}{15}%
  \RangeCatcodeCheck{48}{57}{12}%
  \RangeCatcodeCheck{58}{63}{15}%
  \RangeCatcodeCheck{64}{64}{12}%
  \RangeCatcodeCheck{65}{90}{11}%
  \RangeCatcodeCheck{91}{91}{15}%
  \RangeCatcodeCheck{92}{92}{0}%
  \RangeCatcodeCheck{93}{96}{15}%
  \RangeCatcodeCheck{97}{122}{11}%
  \RangeCatcodeCheck{123}{255}{15}%
  \RestoreCatcodes
}
\Test
\csname @@end\endcsname
\end
%    \end{macrocode}
%    \begin{macrocode}
%</test1>
%    \end{macrocode}
%
% \subsection{Test calculations}
%
%    \begin{macrocode}
%<*test-calc>
\catcode`\{=1 %
\catcode`\}=2 %
\catcode`\#=6 %
\catcode`\@=11 %
\begingroup\expandafter\expandafter\expandafter\endgroup
\expandafter\ifx\csname RequirePackage\endcsname\relax
  \input fibnum.sty\relax
\else
  \RequirePackage{fibnum}[2016/05/16]%
\fi
\def\TestSet{%
  \test{0}{0}%
  \test{1}{1}%
  \test{2}{1}%
  \test{3}{2}%
  \test{4}{3}%
  \test{5}{5}%
  \test{6}{8}%
  \test{7}{13}%
  \test{8}{21}%
  \test{9}{34}%
  \test{10}{55}%
  \test{11}{89}%
  \test{12}{144}%
  \test{13}{233}%
  \test{14}{377}%
  \test{15}{610}%
  \test{16}{987}%
  \test{17}{1597}%
  \test{18}{2584}%
  \test{19}{4181}%
  \test{20}{6765}%
  \test{21}{10946}%
  \test{22}{17711}%
  \test{23}{28657}%
  \test{24}{46368}%
  \test{25}{75025}%
  \test{26}{121393}%
  \test{27}{196418}%
  \test{28}{317811}%
  \test{29}{514229}%
  \test{30}{832040}%
  \test{31}{1346269}%
  \test{32}{2178309}%
  \test{33}{3524578}%
  \test{34}{5702887}%
  \test{35}{9227465}%
  \test{36}{14930352}%
  \test{37}{24157817}%
  \test{38}{39088169}%
  \test{39}{63245986}%
  \test{40}{102334155}%
  \test{41}{165580141}%
  \test{42}{267914296}%
  \test{43}{433494437}%
  \test{44}{701408733}%
  \test{45}{1134903170}%
  \test{46}{1836311903}%
  \test{47}{2971215073}%
  \test{48}{4807526976}%
  \test{49}{7778742049}%
  \test{50}{12586269025}%
  \test{51}{20365011074}%
  \test{52}{32951280099}%
  \test{53}{53316291173}%
  \test{54}{86267571272}%
  \test{55}{139583862445}%
  \test{56}{225851433717}%
  \test{57}{365435296162}%
  \test{58}{591286729879}%
  \test{59}{956722026041}%
  \test{60}{1548008755920}%
  \test{61}{2504730781961}%
  \test{62}{4052739537881}%
  \test{63}{6557470319842}%
  \test{64}{10610209857723}%
  \test{65}{17167680177565}%
  \test{66}{27777890035288}%
  \test{67}{44945570212853}%
  \test{68}{72723460248141}%
  \test{69}{117669030460994}%
  \test{70}{190392490709135}%
  \test{71}{308061521170129}%
  \test{72}{498454011879264}%
  \test{73}{806515533049393}%
}
\def\msg#{\immediate\write16}
\def\test#1#2{%
  \TestAux{#1}{#2}%
  \ifnum#1=0 %
  \else
    \ifodd#1 %
      \TestAux{-#1}{#2}%
    \else
      \TestAux{-#1}{-#2}%
    \fi
  \fi
}
\def\TestAux#1#2{%
  \def\Expected{#2}%
  \expandafter\expandafter\expandafter\def
  \expandafter\expandafter\expandafter\Result
  \expandafter\expandafter\expandafter{%
    \fibnum{#1}%
  }%
  \ltx@onelevel@sanitize\Result
  \ifx\Result\Expected
    \msg{* #1: ok.}%
  \else
    \msg{! fib(#1) = #2}%
    \errmessage{fib(#1) <> \Result}%
  \fi
}
\TestSet
\setbox0=\hbox{%
  \msg{* PreCalc{73}}%
  \fibnumPreCalc{73}%
}
\ifdim\wd0=0pt
\else
  \errmessage{Unwanted stuff in PreCalc}%
\fi
\TestSet
\csname @@end\endcsname\end
%</test-calc>
%    \end{macrocode}
%
% \section{Installation}
%
% \subsection{Download}
%
% \paragraph{Package.} This package is available on
% CTAN\footnote{\url{http://ctan.org/pkg/fibnum}}:
% \begin{description}
% \item[\CTAN{macros/latex/contrib/oberdiek/fibnum.dtx}] The source file.
% \item[\CTAN{macros/latex/contrib/oberdiek/fibnum.pdf}] Documentation.
% \end{description}
%
%
% \paragraph{Bundle.} All the packages of the bundle `oberdiek'
% are also available in a TDS compliant ZIP archive. There
% the packages are already unpacked and the documentation files
% are generated. The files and directories obey the TDS standard.
% \begin{description}
% \item[\CTAN{install/macros/latex/contrib/oberdiek.tds.zip}]
% \end{description}
% \emph{TDS} refers to the standard ``A Directory Structure
% for \TeX\ Files'' (\CTAN{tds/tds.pdf}). Directories
% with \xfile{texmf} in their name are usually organized this way.
%
% \subsection{Bundle installation}
%
% \paragraph{Unpacking.} Unpack the \xfile{oberdiek.tds.zip} in the
% TDS tree (also known as \xfile{texmf} tree) of your choice.
% Example (linux):
% \begin{quote}
%   |unzip oberdiek.tds.zip -d ~/texmf|
% \end{quote}
%
% \paragraph{Script installation.}
% Check the directory \xfile{TDS:scripts/oberdiek/} for
% scripts that need further installation steps.
% Package \xpackage{attachfile2} comes with the Perl script
% \xfile{pdfatfi.pl} that should be installed in such a way
% that it can be called as \texttt{pdfatfi}.
% Example (linux):
% \begin{quote}
%   |chmod +x scripts/oberdiek/pdfatfi.pl|\\
%   |cp scripts/oberdiek/pdfatfi.pl /usr/local/bin/|
% \end{quote}
%
% \subsection{Package installation}
%
% \paragraph{Unpacking.} The \xfile{.dtx} file is a self-extracting
% \docstrip\ archive. The files are extracted by running the
% \xfile{.dtx} through \plainTeX:
% \begin{quote}
%   \verb|tex fibnum.dtx|
% \end{quote}
%
% \paragraph{TDS.} Now the different files must be moved into
% the different directories in your installation TDS tree
% (also known as \xfile{texmf} tree):
% \begin{quote}
% \def\t{^^A
% \begin{tabular}{@{}>{\ttfamily}l@{ $\rightarrow$ }>{\ttfamily}l@{}}
%   fibnum.sty & tex/generic/oberdiek/fibnum.sty\\
%   fibnum.pdf & doc/latex/oberdiek/fibnum.pdf\\
%   test/fibnum-test1.tex & doc/latex/oberdiek/test/fibnum-test1.tex\\
%   test/fibnum-test-calc.tex & doc/latex/oberdiek/test/fibnum-test-calc.tex\\
%   fibnum.dtx & source/latex/oberdiek/fibnum.dtx\\
% \end{tabular}^^A
% }^^A
% \sbox0{\t}^^A
% \ifdim\wd0>\linewidth
%   \begingroup
%     \advance\linewidth by\leftmargin
%     \advance\linewidth by\rightmargin
%   \edef\x{\endgroup
%     \def\noexpand\lw{\the\linewidth}^^A
%   }\x
%   \def\lwbox{^^A
%     \leavevmode
%     \hbox to \linewidth{^^A
%       \kern-\leftmargin\relax
%       \hss
%       \usebox0
%       \hss
%       \kern-\rightmargin\relax
%     }^^A
%   }^^A
%   \ifdim\wd0>\lw
%     \sbox0{\small\t}^^A
%     \ifdim\wd0>\linewidth
%       \ifdim\wd0>\lw
%         \sbox0{\footnotesize\t}^^A
%         \ifdim\wd0>\linewidth
%           \ifdim\wd0>\lw
%             \sbox0{\scriptsize\t}^^A
%             \ifdim\wd0>\linewidth
%               \ifdim\wd0>\lw
%                 \sbox0{\tiny\t}^^A
%                 \ifdim\wd0>\linewidth
%                   \lwbox
%                 \else
%                   \usebox0
%                 \fi
%               \else
%                 \lwbox
%               \fi
%             \else
%               \usebox0
%             \fi
%           \else
%             \lwbox
%           \fi
%         \else
%           \usebox0
%         \fi
%       \else
%         \lwbox
%       \fi
%     \else
%       \usebox0
%     \fi
%   \else
%     \lwbox
%   \fi
% \else
%   \usebox0
% \fi
% \end{quote}
% If you have a \xfile{docstrip.cfg} that configures and enables \docstrip's
% TDS installing feature, then some files can already be in the right
% place, see the documentation of \docstrip.
%
% \subsection{Refresh file name databases}
%
% If your \TeX~distribution
% (\teTeX, \mikTeX, \dots) relies on file name databases, you must refresh
% these. For example, \teTeX\ users run \verb|texhash| or
% \verb|mktexlsr|.
%
% \subsection{Some details for the interested}
%
% \paragraph{Attached source.}
%
% The PDF documentation on CTAN also includes the
% \xfile{.dtx} source file. It can be extracted by
% AcrobatReader 6 or higher. Another option is \textsf{pdftk},
% e.g. unpack the file into the current directory:
% \begin{quote}
%   \verb|pdftk fibnum.pdf unpack_files output .|
% \end{quote}
%
% \paragraph{Unpacking with \LaTeX.}
% The \xfile{.dtx} chooses its action depending on the format:
% \begin{description}
% \item[\plainTeX:] Run \docstrip\ and extract the files.
% \item[\LaTeX:] Generate the documentation.
% \end{description}
% If you insist on using \LaTeX\ for \docstrip\ (really,
% \docstrip\ does not need \LaTeX), then inform the autodetect routine
% about your intention:
% \begin{quote}
%   \verb|latex \let\install=y\input{fibnum.dtx}|
% \end{quote}
% Do not forget to quote the argument according to the demands
% of your shell.
%
% \paragraph{Generating the documentation.}
% You can use both the \xfile{.dtx} or the \xfile{.drv} to generate
% the documentation. The process can be configured by the
% configuration file \xfile{ltxdoc.cfg}. For instance, put this
% line into this file, if you want to have A4 as paper format:
% \begin{quote}
%   \verb|\PassOptionsToClass{a4paper}{article}|
% \end{quote}
% An example follows how to generate the
% documentation with pdf\LaTeX:
% \begin{quote}
%\begin{verbatim}
%pdflatex fibnum.dtx
%bibtex fibnum.aux
%makeindex -s gind.ist fibnum.idx
%pdflatex fibnum.dtx
%makeindex -s gind.ist fibnum.idx
%pdflatex fibnum.dtx
%\end{verbatim}
% \end{quote}
%
% \printbibliography[
%   heading=bibnumbered,
% ]
%
% \begin{History}
%   \begin{Version}{2012/04/08 v1.0}
%   \item
%     First version.
%   \end{Version}
%   \begin{Version}{2016/05/16 v1.1}
%   \item
%     Documentation updates.
%   \end{Version}
% \end{History}
%
% \PrintIndex
%
% \Finale
\endinput

%        (quote the arguments according to the demands of your shell)
%
% Documentation:
%    (a) If fibnum.drv is present:
%           latex fibnum.drv
%    (b) Without fibnum.drv:
%           latex fibnum.dtx; ...
%    The class ltxdoc loads the configuration file ltxdoc.cfg
%    if available. Here you can specify further options, e.g.
%    use A4 as paper format:
%       \PassOptionsToClass{a4paper}{article}
%
%    Programm calls to get the documentation (example):
%       pdflatex fibnum.dtx
%       bibtex fibnum.aux
%       makeindex -s gind.ist fibnum.idx
%       pdflatex fibnum.dtx
%       makeindex -s gind.ist fibnum.idx
%       pdflatex fibnum.dtx
%
% Installation:
%    TDS:tex/generic/oberdiek/fibnum.sty
%    TDS:doc/latex/oberdiek/fibnum.pdf
%    TDS:doc/latex/oberdiek/test/fibnum-test1.tex
%    TDS:doc/latex/oberdiek/test/fibnum-test-calc.tex
%    TDS:source/latex/oberdiek/fibnum.dtx
%
%<*ignore>
\begingroup
  \catcode123=1 %
  \catcode125=2 %
  \def\x{LaTeX2e}%
\expandafter\endgroup
\ifcase 0\ifx\install y1\fi\expandafter
         \ifx\csname processbatchFile\endcsname\relax\else1\fi
         \ifx\fmtname\x\else 1\fi\relax
\else\csname fi\endcsname
%</ignore>
%<*install>
\input docstrip.tex
\Msg{************************************************************************}
\Msg{* Installation}
\Msg{* Package: fibnum 2016/05/16 v1.1 Fibonacci numbers (HO)}
\Msg{************************************************************************}

\keepsilent
\askforoverwritefalse

\let\MetaPrefix\relax
\preamble

This is a generated file.

Project: fibnum
Version: 2016/05/16 v1.1

Copyright (C) 2012 by
   Heiko Oberdiek <heiko.oberdiek at googlemail.com>

This work may be distributed and/or modified under the
conditions of the LaTeX Project Public License, either
version 1.3c of this license or (at your option) any later
version. This version of this license is in
   http://www.latex-project.org/lppl/lppl-1-3c.txt
and the latest version of this license is in
   http://www.latex-project.org/lppl.txt
and version 1.3 or later is part of all distributions of
LaTeX version 2005/12/01 or later.

This work has the LPPL maintenance status "maintained".

This Current Maintainer of this work is Heiko Oberdiek.

The Base Interpreter refers to any `TeX-Format',
because some files are installed in TDS:tex/generic//.

This work consists of the main source file fibnum.dtx
and the derived files
   fibnum.sty, fibnum.pdf, fibnum.ins, fibnum.drv, fibnum.bib,
   fibnum-test1.tex, fibnum-test-calc.tex.

\endpreamble
\let\MetaPrefix\DoubleperCent

\generate{%
  \file{fibnum.ins}{\from{fibnum.dtx}{install}}%
  \file{fibnum.drv}{\from{fibnum.dtx}{driver}}%
  \nopreamble
  \nopostamble
  \file{fibnum.bib}{\from{fibnum.dtx}{bib}}%
  \usepreamble\defaultpreamble
  \usepostamble\defaultpostamble
  \usedir{tex/generic/oberdiek}%
  \file{fibnum.sty}{\from{fibnum.dtx}{package}}%
  \usedir{doc/latex/oberdiek/test}%
  \file{fibnum-test1.tex}{\from{fibnum.dtx}{test1}}%
  \file{fibnum-test-calc.tex}{\from{fibnum.dtx}{test-calc}}%
}

\catcode32=13\relax% active space
\let =\space%
\Msg{************************************************************************}
\Msg{*}
\Msg{* To finish the installation you have to move the following}
\Msg{* file into a directory searched by TeX:}
\Msg{*}
\Msg{*     fibnum.sty}
\Msg{*}
\Msg{* To produce the documentation run the file `fibnum.drv'}
\Msg{* through LaTeX.}
\Msg{*}
\Msg{* Happy TeXing!}
\Msg{*}
\Msg{************************************************************************}

\endbatchfile
%</install>
%<*bib>
@online{texhax:abraham,
  author={Abraham, Jan},
  title={[texhax] Beginner in TEX MACRO to compute functions},
  date={2012-04-07},
  url={http://tug.org/pipermail/texhax/2012-April/019146.html},
  urldate={2012-04-08},
}
@article{knuth:negafibonacci,
  author={Knuth, Donald E.},
  title={Negafibonacci Numbers and the Hyperbolic Plane},
  date={2008-12-11},
  url={http://research.allacademic.com/meta/p206842_index.html},
}
@online{wikipedia:negafibonacci,
  author={{Wikipedia contributors}},
  organization={{Wikipedia, The Free Encyclopedia}},
  title={Fibonacci numbers},
  language={langenglish},
  version={486266088},
  date={2012-04-08},
  url={http://en.wikipedia.org/w/index.php?title=Fibonacci_number&oldid=486266088},
  urldate={2012-04-08},
}
%</bib>
%<*ignore>
\fi
%</ignore>
%<*driver>
\NeedsTeXFormat{LaTeX2e}
\ProvidesFile{fibnum.drv}%
  [2016/05/16 v1.1 Fibonacci numbers (HO)]%
\documentclass{ltxdoc}
\usepackage{amsmath,amsfonts}
\usepackage{siunitx}
\usepackage{array}
\usepackage{tabularx}
\usepackage{fibnum}[2016/05/16]
\usepackage{holtxdoc}[2011/11/22]
\usepackage{csquotes}
\usepackage[
  bibencoding=ascii,
  alldates=iso8601,
]{biblatex}[2011/11/13]
\bibliography{oberdiek-source}
\bibliography{fibnum}
\begin{document}
  \DocInput{fibnum.dtx}%
\end{document}
%</driver>
% \fi
%
%
% \CharacterTable
%  {Upper-case    \A\B\C\D\E\F\G\H\I\J\K\L\M\N\O\P\Q\R\S\T\U\V\W\X\Y\Z
%   Lower-case    \a\b\c\d\e\f\g\h\i\j\k\l\m\n\o\p\q\r\s\t\u\v\w\x\y\z
%   Digits        \0\1\2\3\4\5\6\7\8\9
%   Exclamation   \!     Double quote  \"     Hash (number) \#
%   Dollar        \$     Percent       \%     Ampersand     \&
%   Acute accent  \'     Left paren    \(     Right paren   \)
%   Asterisk      \*     Plus          \+     Comma         \,
%   Minus         \-     Point         \.     Solidus       \/
%   Colon         \:     Semicolon     \;     Less than     \<
%   Equals        \=     Greater than  \>     Question mark \?
%   Commercial at \@     Left bracket  \[     Backslash     \\
%   Right bracket \]     Circumflex    \^     Underscore    \_
%   Grave accent  \`     Left brace    \{     Vertical bar  \|
%   Right brace   \}     Tilde         \~}
%
% \GetFileInfo{fibnum.drv}
%
% \title{The \xpackage{fibnum} package}
% \date{2016/05/16 v1.1}
% \author{Heiko Oberdiek\thanks
% {Please report any issues at https://github.com/ho-tex/oberdiek/issues}\\
% \xemail{heiko.oberdiek at googlemail.com}}
%
% \maketitle
%
% \begin{abstract}
% The package \xpackage{fibnum} provides expandable fibonacci
% numbers for both \hologo{LaTeX} and \hologo{plainTeX}.
% \end{abstract}
%
% \tableofcontents
%
% \section{Documentation}
%
% In the mailing list \textsf{texhax} Jan Abraham asked the question,
% how to get Fibonacci numbers in \hologo{TeX} \cite{texhax:abraham}:
% \begin{quote}
% Write a Macro in \hologo{TeX} that compute the function |\fib{m}|
% All fibonacci numbers from 1 to $m$ ($m < 40$).
% \end{quote}
% This packages provides an expandable implementation for the
% calculation of these numbers for a much larger set of indexes.
% For practical reasons the index is restricted to the same limitations
% that apply for \hologo{TeX} integer numbers.
% The range of the Fibonacci numbers, however, are not limited
% by the algorithm. They are only restricted to memory limitations,
% if they are hit.
%
% The package is loaded as \hologo{LaTeX} package in \hologo{LaTeX}:
% \begin{quote}
%   |\usepackage{fibnum}|
% \end{quote}
% and as file in \hologo{plainTeX}:
% \begin{quote}
%   |\input fibnum.sty|
% \end{quote}
% The package does not know any options and it provides
% the macros \cs{fibnum} and \cs{fibnumPreCalc}.
%
% \begin{declcs}{fibnum} \M{index}
% \end{declcs}
% Macro \cs{fibnum} expects a \hologo{TeX} number as \meta{index}
% in the official \hologo{TeX} number range from $-(2^{31}-1)$ up to
% $2^{31}-1$. In exact two expansion steps the macro expands to
% the Fibnoacci number $F_{\text{\meta{index}}}$. In case of a negative
% \meta{index}, the ``negafibonacci'' number \cite{wikipedia:negafibonacci}
% is used. Formally the Fibonacci number $F_n$ with integer
% index~$n$, $n\in\mathbb{Z}$ and
% $n\in[\num{-2147483647},\num{2147483647}]$ that is returned by macro
% \cs{fibnum} with numerical argument $n$ is defined the following way:
% \begin{gather}
%   \label{eq:def}
%   F_n =
%   \begin{cases}
%     0 & \text{for $n=0$}\\
%     1 & \text{for $n=1$}\\
%     F_{n-1} + F_{n-2} & \text{for $n>1$}\\
%     (-1)^{n+1}F_n & \text{for $n<0$}
%   \end{cases}
% \end{gather}
% Examples:
% \begin{quote}
%   \makeatletter
%   \def\x#1{\cs{fibnum}\{#1\}&
%     \edef\X{\fibnum{#1}}\edef\Y{\expandafter\ltx@car\X\@nil}^^A
%     \if-\Y
%       \edef\X{\expandafter\ltx@cdr\X\@nil}^^A
%       \noindent
%       \llap{-}\X
%     \else
%       \X
%     \fi
%     \tabularnewline
%   }
%   \def\y{\multicolumn{1}{@{}c@{}}{$\vdots$}\tabularnewline}
%   \DeclareUrlCommand\UrlNum{^^A
%     \urlstyle{tt}^^A
%     \def\UrlBreaks{\do\0\do\1\do\2\do\3\do\4\do\5\do\6\do\7\do\8\do\9}^^A
%   }
%   \begin{tabularx}{\dimexpr\linewidth+5.7pt\relax}{@{}>{\ttfamily}l@{ $\rightarrow$ \hphantom{\ttfamily-}}>{\ttfamily}X@{}}
%     \x{-6}
%     \x{-5}
%     \x{-4}
%     \x{-3}
%     \x{-2}
%     \x{-1}
%     \x{0}
%     \x{1}
%     \x{2}
%     \x{3}
%     \x{4}
%     \x{5}
%     \x{6}
%     \y
%     \x{10}
%     \y
%     \x{46}
%     \y
%     \cs{fibnum}\{100\} & 354224848179261915075
%     \tabularnewline
%     \y
%     \cs{fibnum}\{200\} & 280571172992510140037611932413038677189525
%     \tabularnewline
%     \y
%     \cs{fibnum}\{1000\} &
%       \raggedright
%       \UrlNum{^^A
%         434665576869374564356885276750406258025646^^A
%         605173717804024817290895365554179490518904^^A
%         038798400792551692959225930803226347752096^^A
%         896232398733224711616429964409065331879382^^A
%         98969649928516003704476137795166849228875^^A
%       }
%     \tabularnewline
%   \end{tabularx}\kern-5.7pt\mbox{}
% \end{quote}
%
% \begin{declcs}{fibnumPreCalc} \M{index}
% \end{declcs}
% The package already provides precalculated Fibonacci numbers up to
% index~46. That means that calculations are not necessary for
% Fibonacci numbers that fit into the range of \hologo{TeX}
% numbers. Because macro \cs{fibnum} is expandable, it cannot
% store calculated Fibonacci numbers for later use. Macro definitions
% are forbidden in expandable contexts. If larger Fibonacci numbers
% are used more than once, than the compilation time can be shortened
% by calculating and storing the Fibonacci numbers beforehand.
% The argument \meta{index} is a \hologo{TeX} number and macro
% \cs{fibnumPreCalc} ensures that the Fibonacci numbers
% $F_0$ up to $F_{\lvert\text{\meta{index}}\rvert}$ that are not
% already known are calculated
% and stored in internal macros. Internally only non-negative
% Fibonacci numbers are stored. If \meta{index} is negative, then
% the needed positive Fibonacci numbers are calculated and stored.
% Example:
% \begin{quote}
%   \def\x#1{\begingroup\itshape\texttt{\%} #1\endgroup}
%   |\fibnumPreCalc{50}|\\
%   \x{calculates and stores the values for indexes 47..50.}\\
%   \x{(Values for 0..46 are already stored by the package.)}\\
%   |\fibnum{49}| \x{uses the stored value}\\
%   |\fibnum{51}|
%   \x{only calculates $F_{51}$ from stored values $F_{49}$ and $F_{50}$}\\
%   |\fibnumPreCalc{100}|\\
%   \x{calculates and stores the values for indexes 51..100}\\
%   |\fibnum{100}| \x{uses the stored value for $F_{100}$}\\
%   |\fibnum{-100}|
%   \x{uses the stored value for $F_{100}$}\\
%   \x{$F_{-100}=-F_{100}$ according to equation \eqref{eq:def}.}
% \end{quote}
%
% \StopEventually{
% }
%
% \section{Implementation}
%
% \subsection{Identification}
%
%    \begin{macrocode}
%<*package>
%    \end{macrocode}
%    Reload check, especially if the package is not used with \LaTeX.
%    \begin{macrocode}
\begingroup\catcode61\catcode48\catcode32=10\relax%
  \catcode13=5 % ^^M
  \endlinechar=13 %
  \catcode35=6 % #
  \catcode39=12 % '
  \catcode44=12 % ,
  \catcode45=12 % -
  \catcode46=12 % .
  \catcode58=12 % :
  \catcode64=11 % @
  \catcode123=1 % {
  \catcode125=2 % }
  \expandafter\let\expandafter\x\csname ver@fibnum.sty\endcsname
  \ifx\x\relax % plain-TeX, first loading
  \else
    \def\empty{}%
    \ifx\x\empty % LaTeX, first loading,
      % variable is initialized, but \ProvidesPackage not yet seen
    \else
      \expandafter\ifx\csname PackageInfo\endcsname\relax
        \def\x#1#2{%
          \immediate\write-1{Package #1 Info: #2.}%
        }%
      \else
        \def\x#1#2{\PackageInfo{#1}{#2, stopped}}%
      \fi
      \x{fibnum}{The package is already loaded}%
      \aftergroup\endinput
    \fi
  \fi
\endgroup%
%    \end{macrocode}
%    Package identification:
%    \begin{macrocode}
\begingroup\catcode61\catcode48\catcode32=10\relax%
  \catcode13=5 % ^^M
  \endlinechar=13 %
  \catcode35=6 % #
  \catcode39=12 % '
  \catcode40=12 % (
  \catcode41=12 % )
  \catcode44=12 % ,
  \catcode45=12 % -
  \catcode46=12 % .
  \catcode47=12 % /
  \catcode58=12 % :
  \catcode64=11 % @
  \catcode91=12 % [
  \catcode93=12 % ]
  \catcode123=1 % {
  \catcode125=2 % }
  \expandafter\ifx\csname ProvidesPackage\endcsname\relax
    \def\x#1#2#3[#4]{\endgroup
      \immediate\write-1{Package: #3 #4}%
      \xdef#1{#4}%
    }%
  \else
    \def\x#1#2[#3]{\endgroup
      #2[{#3}]%
      \ifx#1\@undefined
        \xdef#1{#3}%
      \fi
      \ifx#1\relax
        \xdef#1{#3}%
      \fi
    }%
  \fi
\expandafter\x\csname ver@fibnum.sty\endcsname
\ProvidesPackage{fibnum}%
  [2016/05/16 v1.1 Fibonacci numbers (HO)]%
%    \end{macrocode}
%
%    \begin{macrocode}
\begingroup\catcode61\catcode48\catcode32=10\relax%
  \catcode13=5 % ^^M
  \endlinechar=13 %
  \catcode123=1 % {
  \catcode125=2 % }
  \catcode64=11 % @
  \def\x{\endgroup
    \expandafter\edef\csname FibNum@AtEnd\endcsname{%
      \endlinechar=\the\endlinechar\relax
      \catcode13=\the\catcode13\relax
      \catcode32=\the\catcode32\relax
      \catcode35=\the\catcode35\relax
      \catcode61=\the\catcode61\relax
      \catcode64=\the\catcode64\relax
      \catcode123=\the\catcode123\relax
      \catcode125=\the\catcode125\relax
    }%
  }%
\x\catcode61\catcode48\catcode32=10\relax%
\catcode13=5 % ^^M
\endlinechar=13 %
\catcode35=6 % #
\catcode64=11 % @
\catcode123=1 % {
\catcode125=2 % }
\def\TMP@EnsureCode#1#2{%
  \edef\FibNum@AtEnd{%
    \FibNum@AtEnd
    \catcode#1=\the\catcode#1\relax
  }%
  \catcode#1=#2\relax
}
\TMP@EnsureCode{33}{12}% !
%\TMP@EnsureCode{36}{3}% $
%\TMP@EnsureCode{38}{4}% &
\TMP@EnsureCode{40}{12}% (
\TMP@EnsureCode{41}{12}% )
\TMP@EnsureCode{45}{12}% -
\TMP@EnsureCode{46}{12}% .
\TMP@EnsureCode{47}{12}% /
\TMP@EnsureCode{58}{12}% :
\TMP@EnsureCode{60}{12}% <
\TMP@EnsureCode{62}{12}% >
\TMP@EnsureCode{91}{12}% [
%\TMP@EnsureCode{96}{12}% `
\TMP@EnsureCode{93}{12}% ]
%\TMP@EnsureCode{94}{12}% ^ (superscript) (!)
%\TMP@EnsureCode{124}{12}% |
\edef\FibNum@AtEnd{\FibNum@AtEnd\noexpand\endinput}
%    \end{macrocode}
%
% \subsection{Package resources}
%
%    \begin{macrocode}
\begingroup\expandafter\expandafter\expandafter\endgroup
\expandafter\ifx\csname RequirePackage\endcsname\relax
  \def\TMP@RequirePackage#1[#2]{%
    \begingroup\expandafter\expandafter\expandafter\endgroup
    \expandafter\ifx\csname ver@#1.sty\endcsname\relax
      \input #1.sty\relax
    \fi
  }%
  \TMP@RequirePackage{ltxcmds}[2011/04/18]%
  \TMP@RequirePackage{intcalc}[2007/09/27]%
  \TMP@RequirePackage{bigintcalc}[2007/11/11]%
\else
  \RequirePackage{ltxcmds}[2011/04/18]%
  \RequirePackage{intcalc}[2007/09/27]%
  \RequirePackage{bigintcalc}[2007/11/11]%
\fi
%    \end{macrocode}
%
% \subsection{Setup precalculated values}
%
%    \begin{macrocode}
\def\FibNum@temp#1{%
  \expandafter\def\csname FibNum@#1\endcsname
}
\catcode46=9 % dots are ignored
\FibNum@temp{0}{0}
\FibNum@temp{1}{1}
\FibNum@temp{2}{1}
\FibNum@temp{3}{2}
\FibNum@temp{4}{3}
\FibNum@temp{5}{5}
\FibNum@temp{6}{8}
\FibNum@temp{7}{13}
\FibNum@temp{8}{21}
\FibNum@temp{9}{34}
\FibNum@temp{10}{55}
\FibNum@temp{11}{89}
\FibNum@temp{12}{144}
\FibNum@temp{13}{233}
\FibNum@temp{14}{377}
\FibNum@temp{15}{610}
\FibNum@temp{16}{987}
\FibNum@temp{17}{1.597}
\FibNum@temp{18}{2.584}
\FibNum@temp{19}{4.181}
\FibNum@temp{20}{6.765}
\FibNum@temp{21}{10.946}
\FibNum@temp{22}{17.711}
\FibNum@temp{23}{28.657}
\FibNum@temp{24}{46.368}
\FibNum@temp{25}{75.025}
\FibNum@temp{26}{121.393}
\FibNum@temp{27}{196.418}
\FibNum@temp{28}{317.811}
\FibNum@temp{29}{514.229}
\FibNum@temp{30}{832.040}
\FibNum@temp{31}{1.346.269}
\FibNum@temp{32}{2.178.309}
\FibNum@temp{33}{3.524.578}
\FibNum@temp{34}{5.702.887}
\FibNum@temp{35}{9.227.465}
\FibNum@temp{36}{14.930.352}
\FibNum@temp{37}{24.157.817}
\FibNum@temp{38}{39.088.169}
\FibNum@temp{39}{63.245.986}
\FibNum@temp{40}{102.334.155}
\FibNum@temp{41}{165.580.141}
\FibNum@temp{42}{267.914.296}
\FibNum@temp{43}{433.494.437}
\FibNum@temp{44}{701.408.733}
\FibNum@temp{45}{1.134.903.170}
\FibNum@temp{46}{1.836.311.903}
%    \end{macrocode}
%    \begin{macro}{\FibNum@max}
%    \begin{macrocode}
\def\FibNum@max{46}
%    \end{macrocode}
%    \end{macro}
%
% \subsection{Macros for precalculating values}
%
%    \begin{macro}{\fibnumPreCalc}
%    \begin{macrocode}
\def\fibnumPreCalc#1{%
  \expandafter\expandafter\expandafter
  \FibNum@PreCalc\intcalcNum{#1}/%
}
%    \end{macrocode}
%    \end{macro}
%    \begin{macro}{\FibNum@PreCalc}
%    \begin{macrocode}
\def\FibNum@PreCalc#1/{%
  \ifnum#1<\ltx@zero
    \expandafter\FibNum@PreCalc\ltx@gobble#1/%
  \else
    \ifnum#1>\FibNum@max
      \begingroup
        \ltx@LocDimenA=#1sp\relax
        \countdef\FibNum@i=255\relax
        \FibNum@i=\FibNum@max\relax
        \edef\FibNum@temp{%
          \csname FibNum@\the\FibNum@i\endcsname/%
        }%
        \advance\FibNum@i by -1\relax
        \edef\FibNum@temp{%
          \FibNum@temp
          \csname FibNum@\the\FibNum@i\endcsname
        }%
        \advance\FibNum@i\ltx@two
        \iftrue
          \expandafter\FibNum@PreCalcAux\FibNum@temp
        \fi
      \endgroup
    \fi
  \fi
}
%    \end{macrocode}
%    \end{macro}
%    \begin{macro}{\FibNum@PreCalcAux}
%    \begin{macrocode}
\def\FibNum@PreCalcAux#1/#2\fi{%
  \fi
  \edef\FibNum@temp{\BigIntCalcAdd#1!#2!}%
  \global\expandafter
  \let\csname FibNum@\the\FibNum@i\endcsname\FibNum@temp
  \ifnum\FibNum@i=\ltx@LocDimenA
    \xdef\FibNum@max{\the\FibNum@i}%
  \else
    \advance\FibNum@i\ltx@one
    \expandafter\FibNum@PreCalcAux\FibNum@temp/#1%
  \fi
}
%    \end{macrocode}
%    \end{macro}
%
% \subsection{Expandable calculations}
%
%    \begin{macro}{\fibnum}
%    \begin{macrocode}
\def\fibnum#1{%
  \romannumeral
  \expandafter\expandafter\expandafter\FibNum@Do\intcalcNum{#1}/%
}
%    \end{macrocode}
%    \end{macro}
%    \begin{macro}{\FibNum@Do}
%    \begin{macrocode}
\def\FibNum@Do#1/{%
  \ifnum#1<\ltx@zero
    \FibNum@ReturnAfterElseFiFi{%
      \ifodd#1 %
        \expandafter\expandafter\expandafter\ltx@zero
      \else
        \expandafter\expandafter\expandafter\ltx@zero
        \expandafter\expandafter\expandafter-%
      \fi
      \romannumeral
      \expandafter\FibNum@Do\ltx@gobble#1/%
    }%
  \else
    \ifnum\FibNum@max<#1 %
      \ltx@ReturnAfterElseFi{%
        \expandafter
        \FibNum@ExpCalc\number\expandafter\IntCalcInc\FibNum@max!%
        \expandafter\expandafter\expandafter/%
        \csname FibNum@\FibNum@max
        \expandafter\expandafter\expandafter\endcsname
        \expandafter\expandafter\expandafter/%
        \csname FibNum@\expandafter\IntCalcDec\FibNum@max!%
        \endcsname/%
        #1%
      }%
    \else
      \expandafter\expandafter\expandafter\ltx@zero
      \csname FibNum@#1\expandafter\expandafter\expandafter\endcsname
    \fi
  \fi
}
%    \end{macrocode}
%    \end{macro}
%    \begin{macro}{\FibNum@ReturnAfterElseFiFi}
%    \begin{macrocode}
\def\FibNum@ReturnAfterElseFiFi#1\else#2\fi\fi{\fi#1}
%    \end{macrocode}
%    \end{macro}
%    \begin{macro}{\FibNum@ExpCalc}
%    \begin{macrocode}
\def\FibNum@ExpCalc#1/#2/#3/#4\fi{%
  \fi
  \ifnum#1=#4 %
    \ltx@ReturnAfterElseFi{%
      \expandafter\expandafter\expandafter\ltx@zero
      \BigIntCalcAdd#2!#3!%
    }%
  \else
    \expandafter\FibNum@ExpCalc
    \number\IntCalcInc#1!%
    \expandafter\expandafter\expandafter/%
    \BigIntCalcAdd#2!#3!/%
    #2/#4%
  \fi
}
%    \end{macrocode}
%    \end{macro}
%
%    \begin{macrocode}
\FibNum@AtEnd%
%</package>
%    \end{macrocode}
%
% \section{Test}
%
% \subsection{Catcode checks for loading}
%
%    \begin{macrocode}
%<*test1>
%    \end{macrocode}
%    \begin{macrocode}
\catcode`\{=1 %
\catcode`\}=2 %
\catcode`\#=6 %
\catcode`\@=11 %
\expandafter\ifx\csname count@\endcsname\relax
  \countdef\count@=255 %
\fi
\expandafter\ifx\csname @gobble\endcsname\relax
  \long\def\@gobble#1{}%
\fi
\expandafter\ifx\csname @firstofone\endcsname\relax
  \long\def\@firstofone#1{#1}%
\fi
\expandafter\ifx\csname loop\endcsname\relax
  \expandafter\@firstofone
\else
  \expandafter\@gobble
\fi
{%
  \def\loop#1\repeat{%
    \def\body{#1}%
    \iterate
  }%
  \def\iterate{%
    \body
      \let\next\iterate
    \else
      \let\next\relax
    \fi
    \next
  }%
  \let\repeat=\fi
}%
\def\RestoreCatcodes{}
\count@=0 %
\loop
  \edef\RestoreCatcodes{%
    \RestoreCatcodes
    \catcode\the\count@=\the\catcode\count@\relax
  }%
\ifnum\count@<255 %
  \advance\count@ 1 %
\repeat

\def\RangeCatcodeInvalid#1#2{%
  \count@=#1\relax
  \loop
    \catcode\count@=15 %
  \ifnum\count@<#2\relax
    \advance\count@ 1 %
  \repeat
}
\def\RangeCatcodeCheck#1#2#3{%
  \count@=#1\relax
  \loop
    \ifnum#3=\catcode\count@
    \else
      \errmessage{%
        Character \the\count@\space
        with wrong catcode \the\catcode\count@\space
        instead of \number#3%
      }%
    \fi
  \ifnum\count@<#2\relax
    \advance\count@ 1 %
  \repeat
}
\def\space{ }
\expandafter\ifx\csname LoadCommand\endcsname\relax
  \def\LoadCommand{\input fibnum.sty\relax}%
\fi
\def\Test{%
  \RangeCatcodeInvalid{0}{47}%
  \RangeCatcodeInvalid{58}{64}%
  \RangeCatcodeInvalid{91}{96}%
  \RangeCatcodeInvalid{123}{255}%
  \catcode`\@=12 %
  \catcode`\\=0 %
  \catcode`\%=14 %
  \LoadCommand
  \RangeCatcodeCheck{0}{36}{15}%
  \RangeCatcodeCheck{37}{37}{14}%
  \RangeCatcodeCheck{38}{47}{15}%
  \RangeCatcodeCheck{48}{57}{12}%
  \RangeCatcodeCheck{58}{63}{15}%
  \RangeCatcodeCheck{64}{64}{12}%
  \RangeCatcodeCheck{65}{90}{11}%
  \RangeCatcodeCheck{91}{91}{15}%
  \RangeCatcodeCheck{92}{92}{0}%
  \RangeCatcodeCheck{93}{96}{15}%
  \RangeCatcodeCheck{97}{122}{11}%
  \RangeCatcodeCheck{123}{255}{15}%
  \RestoreCatcodes
}
\Test
\csname @@end\endcsname
\end
%    \end{macrocode}
%    \begin{macrocode}
%</test1>
%    \end{macrocode}
%
% \subsection{Test calculations}
%
%    \begin{macrocode}
%<*test-calc>
\catcode`\{=1 %
\catcode`\}=2 %
\catcode`\#=6 %
\catcode`\@=11 %
\begingroup\expandafter\expandafter\expandafter\endgroup
\expandafter\ifx\csname RequirePackage\endcsname\relax
  \input fibnum.sty\relax
\else
  \RequirePackage{fibnum}[2016/05/16]%
\fi
\def\TestSet{%
  \test{0}{0}%
  \test{1}{1}%
  \test{2}{1}%
  \test{3}{2}%
  \test{4}{3}%
  \test{5}{5}%
  \test{6}{8}%
  \test{7}{13}%
  \test{8}{21}%
  \test{9}{34}%
  \test{10}{55}%
  \test{11}{89}%
  \test{12}{144}%
  \test{13}{233}%
  \test{14}{377}%
  \test{15}{610}%
  \test{16}{987}%
  \test{17}{1597}%
  \test{18}{2584}%
  \test{19}{4181}%
  \test{20}{6765}%
  \test{21}{10946}%
  \test{22}{17711}%
  \test{23}{28657}%
  \test{24}{46368}%
  \test{25}{75025}%
  \test{26}{121393}%
  \test{27}{196418}%
  \test{28}{317811}%
  \test{29}{514229}%
  \test{30}{832040}%
  \test{31}{1346269}%
  \test{32}{2178309}%
  \test{33}{3524578}%
  \test{34}{5702887}%
  \test{35}{9227465}%
  \test{36}{14930352}%
  \test{37}{24157817}%
  \test{38}{39088169}%
  \test{39}{63245986}%
  \test{40}{102334155}%
  \test{41}{165580141}%
  \test{42}{267914296}%
  \test{43}{433494437}%
  \test{44}{701408733}%
  \test{45}{1134903170}%
  \test{46}{1836311903}%
  \test{47}{2971215073}%
  \test{48}{4807526976}%
  \test{49}{7778742049}%
  \test{50}{12586269025}%
  \test{51}{20365011074}%
  \test{52}{32951280099}%
  \test{53}{53316291173}%
  \test{54}{86267571272}%
  \test{55}{139583862445}%
  \test{56}{225851433717}%
  \test{57}{365435296162}%
  \test{58}{591286729879}%
  \test{59}{956722026041}%
  \test{60}{1548008755920}%
  \test{61}{2504730781961}%
  \test{62}{4052739537881}%
  \test{63}{6557470319842}%
  \test{64}{10610209857723}%
  \test{65}{17167680177565}%
  \test{66}{27777890035288}%
  \test{67}{44945570212853}%
  \test{68}{72723460248141}%
  \test{69}{117669030460994}%
  \test{70}{190392490709135}%
  \test{71}{308061521170129}%
  \test{72}{498454011879264}%
  \test{73}{806515533049393}%
}
\def\msg#{\immediate\write16}
\def\test#1#2{%
  \TestAux{#1}{#2}%
  \ifnum#1=0 %
  \else
    \ifodd#1 %
      \TestAux{-#1}{#2}%
    \else
      \TestAux{-#1}{-#2}%
    \fi
  \fi
}
\def\TestAux#1#2{%
  \def\Expected{#2}%
  \expandafter\expandafter\expandafter\def
  \expandafter\expandafter\expandafter\Result
  \expandafter\expandafter\expandafter{%
    \fibnum{#1}%
  }%
  \ltx@onelevel@sanitize\Result
  \ifx\Result\Expected
    \msg{* #1: ok.}%
  \else
    \msg{! fib(#1) = #2}%
    \errmessage{fib(#1) <> \Result}%
  \fi
}
\TestSet
\setbox0=\hbox{%
  \msg{* PreCalc{73}}%
  \fibnumPreCalc{73}%
}
\ifdim\wd0=0pt
\else
  \errmessage{Unwanted stuff in PreCalc}%
\fi
\TestSet
\csname @@end\endcsname\end
%</test-calc>
%    \end{macrocode}
%
% \section{Installation}
%
% \subsection{Download}
%
% \paragraph{Package.} This package is available on
% CTAN\footnote{\url{http://ctan.org/pkg/fibnum}}:
% \begin{description}
% \item[\CTAN{macros/latex/contrib/oberdiek/fibnum.dtx}] The source file.
% \item[\CTAN{macros/latex/contrib/oberdiek/fibnum.pdf}] Documentation.
% \end{description}
%
%
% \paragraph{Bundle.} All the packages of the bundle `oberdiek'
% are also available in a TDS compliant ZIP archive. There
% the packages are already unpacked and the documentation files
% are generated. The files and directories obey the TDS standard.
% \begin{description}
% \item[\CTAN{install/macros/latex/contrib/oberdiek.tds.zip}]
% \end{description}
% \emph{TDS} refers to the standard ``A Directory Structure
% for \TeX\ Files'' (\CTAN{tds/tds.pdf}). Directories
% with \xfile{texmf} in their name are usually organized this way.
%
% \subsection{Bundle installation}
%
% \paragraph{Unpacking.} Unpack the \xfile{oberdiek.tds.zip} in the
% TDS tree (also known as \xfile{texmf} tree) of your choice.
% Example (linux):
% \begin{quote}
%   |unzip oberdiek.tds.zip -d ~/texmf|
% \end{quote}
%
% \paragraph{Script installation.}
% Check the directory \xfile{TDS:scripts/oberdiek/} for
% scripts that need further installation steps.
% Package \xpackage{attachfile2} comes with the Perl script
% \xfile{pdfatfi.pl} that should be installed in such a way
% that it can be called as \texttt{pdfatfi}.
% Example (linux):
% \begin{quote}
%   |chmod +x scripts/oberdiek/pdfatfi.pl|\\
%   |cp scripts/oberdiek/pdfatfi.pl /usr/local/bin/|
% \end{quote}
%
% \subsection{Package installation}
%
% \paragraph{Unpacking.} The \xfile{.dtx} file is a self-extracting
% \docstrip\ archive. The files are extracted by running the
% \xfile{.dtx} through \plainTeX:
% \begin{quote}
%   \verb|tex fibnum.dtx|
% \end{quote}
%
% \paragraph{TDS.} Now the different files must be moved into
% the different directories in your installation TDS tree
% (also known as \xfile{texmf} tree):
% \begin{quote}
% \def\t{^^A
% \begin{tabular}{@{}>{\ttfamily}l@{ $\rightarrow$ }>{\ttfamily}l@{}}
%   fibnum.sty & tex/generic/oberdiek/fibnum.sty\\
%   fibnum.pdf & doc/latex/oberdiek/fibnum.pdf\\
%   test/fibnum-test1.tex & doc/latex/oberdiek/test/fibnum-test1.tex\\
%   test/fibnum-test-calc.tex & doc/latex/oberdiek/test/fibnum-test-calc.tex\\
%   fibnum.dtx & source/latex/oberdiek/fibnum.dtx\\
% \end{tabular}^^A
% }^^A
% \sbox0{\t}^^A
% \ifdim\wd0>\linewidth
%   \begingroup
%     \advance\linewidth by\leftmargin
%     \advance\linewidth by\rightmargin
%   \edef\x{\endgroup
%     \def\noexpand\lw{\the\linewidth}^^A
%   }\x
%   \def\lwbox{^^A
%     \leavevmode
%     \hbox to \linewidth{^^A
%       \kern-\leftmargin\relax
%       \hss
%       \usebox0
%       \hss
%       \kern-\rightmargin\relax
%     }^^A
%   }^^A
%   \ifdim\wd0>\lw
%     \sbox0{\small\t}^^A
%     \ifdim\wd0>\linewidth
%       \ifdim\wd0>\lw
%         \sbox0{\footnotesize\t}^^A
%         \ifdim\wd0>\linewidth
%           \ifdim\wd0>\lw
%             \sbox0{\scriptsize\t}^^A
%             \ifdim\wd0>\linewidth
%               \ifdim\wd0>\lw
%                 \sbox0{\tiny\t}^^A
%                 \ifdim\wd0>\linewidth
%                   \lwbox
%                 \else
%                   \usebox0
%                 \fi
%               \else
%                 \lwbox
%               \fi
%             \else
%               \usebox0
%             \fi
%           \else
%             \lwbox
%           \fi
%         \else
%           \usebox0
%         \fi
%       \else
%         \lwbox
%       \fi
%     \else
%       \usebox0
%     \fi
%   \else
%     \lwbox
%   \fi
% \else
%   \usebox0
% \fi
% \end{quote}
% If you have a \xfile{docstrip.cfg} that configures and enables \docstrip's
% TDS installing feature, then some files can already be in the right
% place, see the documentation of \docstrip.
%
% \subsection{Refresh file name databases}
%
% If your \TeX~distribution
% (\teTeX, \mikTeX, \dots) relies on file name databases, you must refresh
% these. For example, \teTeX\ users run \verb|texhash| or
% \verb|mktexlsr|.
%
% \subsection{Some details for the interested}
%
% \paragraph{Attached source.}
%
% The PDF documentation on CTAN also includes the
% \xfile{.dtx} source file. It can be extracted by
% AcrobatReader 6 or higher. Another option is \textsf{pdftk},
% e.g. unpack the file into the current directory:
% \begin{quote}
%   \verb|pdftk fibnum.pdf unpack_files output .|
% \end{quote}
%
% \paragraph{Unpacking with \LaTeX.}
% The \xfile{.dtx} chooses its action depending on the format:
% \begin{description}
% \item[\plainTeX:] Run \docstrip\ and extract the files.
% \item[\LaTeX:] Generate the documentation.
% \end{description}
% If you insist on using \LaTeX\ for \docstrip\ (really,
% \docstrip\ does not need \LaTeX), then inform the autodetect routine
% about your intention:
% \begin{quote}
%   \verb|latex \let\install=y% \iffalse meta-comment
%
% File: fibnum.dtx
% Version: 2016/05/16 v1.1
% Info: Fibonacci numbers
%
% Copyright (C) 2012 by
%    Heiko Oberdiek <heiko.oberdiek at googlemail.com>
%    2016
%    https://github.com/ho-tex/oberdiek/issues
%
% This work may be distributed and/or modified under the
% conditions of the LaTeX Project Public License, either
% version 1.3c of this license or (at your option) any later
% version. This version of this license is in
%    http://www.latex-project.org/lppl/lppl-1-3c.txt
% and the latest version of this license is in
%    http://www.latex-project.org/lppl.txt
% and version 1.3 or later is part of all distributions of
% LaTeX version 2005/12/01 or later.
%
% This work has the LPPL maintenance status "maintained".
%
% This Current Maintainer of this work is Heiko Oberdiek.
%
% The Base Interpreter refers to any `TeX-Format',
% because some files are installed in TDS:tex/generic//.
%
% This work consists of the main source file fibnum.dtx
% and the derived files
%    fibnum.sty, fibnum.pdf, fibnum.ins, fibnum.drv, fibnum.bib,
%    fibnum-test1.tex, fibnum-test-calc.tex.
%
% Distribution:
%    CTAN:macros/latex/contrib/oberdiek/fibnum.dtx
%    CTAN:macros/latex/contrib/oberdiek/fibnum.pdf
%
% Unpacking:
%    (a) If fibnum.ins is present:
%           tex fibnum.ins
%    (b) Without fibnum.ins:
%           tex fibnum.dtx
%    (c) If you insist on using LaTeX
%           latex \let\install=y\input{fibnum.dtx}
%        (quote the arguments according to the demands of your shell)
%
% Documentation:
%    (a) If fibnum.drv is present:
%           latex fibnum.drv
%    (b) Without fibnum.drv:
%           latex fibnum.dtx; ...
%    The class ltxdoc loads the configuration file ltxdoc.cfg
%    if available. Here you can specify further options, e.g.
%    use A4 as paper format:
%       \PassOptionsToClass{a4paper}{article}
%
%    Programm calls to get the documentation (example):
%       pdflatex fibnum.dtx
%       bibtex fibnum.aux
%       makeindex -s gind.ist fibnum.idx
%       pdflatex fibnum.dtx
%       makeindex -s gind.ist fibnum.idx
%       pdflatex fibnum.dtx
%
% Installation:
%    TDS:tex/generic/oberdiek/fibnum.sty
%    TDS:doc/latex/oberdiek/fibnum.pdf
%    TDS:doc/latex/oberdiek/test/fibnum-test1.tex
%    TDS:doc/latex/oberdiek/test/fibnum-test-calc.tex
%    TDS:source/latex/oberdiek/fibnum.dtx
%
%<*ignore>
\begingroup
  \catcode123=1 %
  \catcode125=2 %
  \def\x{LaTeX2e}%
\expandafter\endgroup
\ifcase 0\ifx\install y1\fi\expandafter
         \ifx\csname processbatchFile\endcsname\relax\else1\fi
         \ifx\fmtname\x\else 1\fi\relax
\else\csname fi\endcsname
%</ignore>
%<*install>
\input docstrip.tex
\Msg{************************************************************************}
\Msg{* Installation}
\Msg{* Package: fibnum 2016/05/16 v1.1 Fibonacci numbers (HO)}
\Msg{************************************************************************}

\keepsilent
\askforoverwritefalse

\let\MetaPrefix\relax
\preamble

This is a generated file.

Project: fibnum
Version: 2016/05/16 v1.1

Copyright (C) 2012 by
   Heiko Oberdiek <heiko.oberdiek at googlemail.com>

This work may be distributed and/or modified under the
conditions of the LaTeX Project Public License, either
version 1.3c of this license or (at your option) any later
version. This version of this license is in
   http://www.latex-project.org/lppl/lppl-1-3c.txt
and the latest version of this license is in
   http://www.latex-project.org/lppl.txt
and version 1.3 or later is part of all distributions of
LaTeX version 2005/12/01 or later.

This work has the LPPL maintenance status "maintained".

This Current Maintainer of this work is Heiko Oberdiek.

The Base Interpreter refers to any `TeX-Format',
because some files are installed in TDS:tex/generic//.

This work consists of the main source file fibnum.dtx
and the derived files
   fibnum.sty, fibnum.pdf, fibnum.ins, fibnum.drv, fibnum.bib,
   fibnum-test1.tex, fibnum-test-calc.tex.

\endpreamble
\let\MetaPrefix\DoubleperCent

\generate{%
  \file{fibnum.ins}{\from{fibnum.dtx}{install}}%
  \file{fibnum.drv}{\from{fibnum.dtx}{driver}}%
  \nopreamble
  \nopostamble
  \file{fibnum.bib}{\from{fibnum.dtx}{bib}}%
  \usepreamble\defaultpreamble
  \usepostamble\defaultpostamble
  \usedir{tex/generic/oberdiek}%
  \file{fibnum.sty}{\from{fibnum.dtx}{package}}%
  \usedir{doc/latex/oberdiek/test}%
  \file{fibnum-test1.tex}{\from{fibnum.dtx}{test1}}%
  \file{fibnum-test-calc.tex}{\from{fibnum.dtx}{test-calc}}%
}

\catcode32=13\relax% active space
\let =\space%
\Msg{************************************************************************}
\Msg{*}
\Msg{* To finish the installation you have to move the following}
\Msg{* file into a directory searched by TeX:}
\Msg{*}
\Msg{*     fibnum.sty}
\Msg{*}
\Msg{* To produce the documentation run the file `fibnum.drv'}
\Msg{* through LaTeX.}
\Msg{*}
\Msg{* Happy TeXing!}
\Msg{*}
\Msg{************************************************************************}

\endbatchfile
%</install>
%<*bib>
@online{texhax:abraham,
  author={Abraham, Jan},
  title={[texhax] Beginner in TEX MACRO to compute functions},
  date={2012-04-07},
  url={http://tug.org/pipermail/texhax/2012-April/019146.html},
  urldate={2012-04-08},
}
@article{knuth:negafibonacci,
  author={Knuth, Donald E.},
  title={Negafibonacci Numbers and the Hyperbolic Plane},
  date={2008-12-11},
  url={http://research.allacademic.com/meta/p206842_index.html},
}
@online{wikipedia:negafibonacci,
  author={{Wikipedia contributors}},
  organization={{Wikipedia, The Free Encyclopedia}},
  title={Fibonacci numbers},
  language={langenglish},
  version={486266088},
  date={2012-04-08},
  url={http://en.wikipedia.org/w/index.php?title=Fibonacci_number&oldid=486266088},
  urldate={2012-04-08},
}
%</bib>
%<*ignore>
\fi
%</ignore>
%<*driver>
\NeedsTeXFormat{LaTeX2e}
\ProvidesFile{fibnum.drv}%
  [2016/05/16 v1.1 Fibonacci numbers (HO)]%
\documentclass{ltxdoc}
\usepackage{amsmath,amsfonts}
\usepackage{siunitx}
\usepackage{array}
\usepackage{tabularx}
\usepackage{fibnum}[2016/05/16]
\usepackage{holtxdoc}[2011/11/22]
\usepackage{csquotes}
\usepackage[
  bibencoding=ascii,
  alldates=iso8601,
]{biblatex}[2011/11/13]
\bibliography{oberdiek-source}
\bibliography{fibnum}
\begin{document}
  \DocInput{fibnum.dtx}%
\end{document}
%</driver>
% \fi
%
%
% \CharacterTable
%  {Upper-case    \A\B\C\D\E\F\G\H\I\J\K\L\M\N\O\P\Q\R\S\T\U\V\W\X\Y\Z
%   Lower-case    \a\b\c\d\e\f\g\h\i\j\k\l\m\n\o\p\q\r\s\t\u\v\w\x\y\z
%   Digits        \0\1\2\3\4\5\6\7\8\9
%   Exclamation   \!     Double quote  \"     Hash (number) \#
%   Dollar        \$     Percent       \%     Ampersand     \&
%   Acute accent  \'     Left paren    \(     Right paren   \)
%   Asterisk      \*     Plus          \+     Comma         \,
%   Minus         \-     Point         \.     Solidus       \/
%   Colon         \:     Semicolon     \;     Less than     \<
%   Equals        \=     Greater than  \>     Question mark \?
%   Commercial at \@     Left bracket  \[     Backslash     \\
%   Right bracket \]     Circumflex    \^     Underscore    \_
%   Grave accent  \`     Left brace    \{     Vertical bar  \|
%   Right brace   \}     Tilde         \~}
%
% \GetFileInfo{fibnum.drv}
%
% \title{The \xpackage{fibnum} package}
% \date{2016/05/16 v1.1}
% \author{Heiko Oberdiek\thanks
% {Please report any issues at https://github.com/ho-tex/oberdiek/issues}\\
% \xemail{heiko.oberdiek at googlemail.com}}
%
% \maketitle
%
% \begin{abstract}
% The package \xpackage{fibnum} provides expandable fibonacci
% numbers for both \hologo{LaTeX} and \hologo{plainTeX}.
% \end{abstract}
%
% \tableofcontents
%
% \section{Documentation}
%
% In the mailing list \textsf{texhax} Jan Abraham asked the question,
% how to get Fibonacci numbers in \hologo{TeX} \cite{texhax:abraham}:
% \begin{quote}
% Write a Macro in \hologo{TeX} that compute the function |\fib{m}|
% All fibonacci numbers from 1 to $m$ ($m < 40$).
% \end{quote}
% This packages provides an expandable implementation for the
% calculation of these numbers for a much larger set of indexes.
% For practical reasons the index is restricted to the same limitations
% that apply for \hologo{TeX} integer numbers.
% The range of the Fibonacci numbers, however, are not limited
% by the algorithm. They are only restricted to memory limitations,
% if they are hit.
%
% The package is loaded as \hologo{LaTeX} package in \hologo{LaTeX}:
% \begin{quote}
%   |\usepackage{fibnum}|
% \end{quote}
% and as file in \hologo{plainTeX}:
% \begin{quote}
%   |\input fibnum.sty|
% \end{quote}
% The package does not know any options and it provides
% the macros \cs{fibnum} and \cs{fibnumPreCalc}.
%
% \begin{declcs}{fibnum} \M{index}
% \end{declcs}
% Macro \cs{fibnum} expects a \hologo{TeX} number as \meta{index}
% in the official \hologo{TeX} number range from $-(2^{31}-1)$ up to
% $2^{31}-1$. In exact two expansion steps the macro expands to
% the Fibnoacci number $F_{\text{\meta{index}}}$. In case of a negative
% \meta{index}, the ``negafibonacci'' number \cite{wikipedia:negafibonacci}
% is used. Formally the Fibonacci number $F_n$ with integer
% index~$n$, $n\in\mathbb{Z}$ and
% $n\in[\num{-2147483647},\num{2147483647}]$ that is returned by macro
% \cs{fibnum} with numerical argument $n$ is defined the following way:
% \begin{gather}
%   \label{eq:def}
%   F_n =
%   \begin{cases}
%     0 & \text{for $n=0$}\\
%     1 & \text{for $n=1$}\\
%     F_{n-1} + F_{n-2} & \text{for $n>1$}\\
%     (-1)^{n+1}F_n & \text{for $n<0$}
%   \end{cases}
% \end{gather}
% Examples:
% \begin{quote}
%   \makeatletter
%   \def\x#1{\cs{fibnum}\{#1\}&
%     \edef\X{\fibnum{#1}}\edef\Y{\expandafter\ltx@car\X\@nil}^^A
%     \if-\Y
%       \edef\X{\expandafter\ltx@cdr\X\@nil}^^A
%       \noindent
%       \llap{-}\X
%     \else
%       \X
%     \fi
%     \tabularnewline
%   }
%   \def\y{\multicolumn{1}{@{}c@{}}{$\vdots$}\tabularnewline}
%   \DeclareUrlCommand\UrlNum{^^A
%     \urlstyle{tt}^^A
%     \def\UrlBreaks{\do\0\do\1\do\2\do\3\do\4\do\5\do\6\do\7\do\8\do\9}^^A
%   }
%   \begin{tabularx}{\dimexpr\linewidth+5.7pt\relax}{@{}>{\ttfamily}l@{ $\rightarrow$ \hphantom{\ttfamily-}}>{\ttfamily}X@{}}
%     \x{-6}
%     \x{-5}
%     \x{-4}
%     \x{-3}
%     \x{-2}
%     \x{-1}
%     \x{0}
%     \x{1}
%     \x{2}
%     \x{3}
%     \x{4}
%     \x{5}
%     \x{6}
%     \y
%     \x{10}
%     \y
%     \x{46}
%     \y
%     \cs{fibnum}\{100\} & 354224848179261915075
%     \tabularnewline
%     \y
%     \cs{fibnum}\{200\} & 280571172992510140037611932413038677189525
%     \tabularnewline
%     \y
%     \cs{fibnum}\{1000\} &
%       \raggedright
%       \UrlNum{^^A
%         434665576869374564356885276750406258025646^^A
%         605173717804024817290895365554179490518904^^A
%         038798400792551692959225930803226347752096^^A
%         896232398733224711616429964409065331879382^^A
%         98969649928516003704476137795166849228875^^A
%       }
%     \tabularnewline
%   \end{tabularx}\kern-5.7pt\mbox{}
% \end{quote}
%
% \begin{declcs}{fibnumPreCalc} \M{index}
% \end{declcs}
% The package already provides precalculated Fibonacci numbers up to
% index~46. That means that calculations are not necessary for
% Fibonacci numbers that fit into the range of \hologo{TeX}
% numbers. Because macro \cs{fibnum} is expandable, it cannot
% store calculated Fibonacci numbers for later use. Macro definitions
% are forbidden in expandable contexts. If larger Fibonacci numbers
% are used more than once, than the compilation time can be shortened
% by calculating and storing the Fibonacci numbers beforehand.
% The argument \meta{index} is a \hologo{TeX} number and macro
% \cs{fibnumPreCalc} ensures that the Fibonacci numbers
% $F_0$ up to $F_{\lvert\text{\meta{index}}\rvert}$ that are not
% already known are calculated
% and stored in internal macros. Internally only non-negative
% Fibonacci numbers are stored. If \meta{index} is negative, then
% the needed positive Fibonacci numbers are calculated and stored.
% Example:
% \begin{quote}
%   \def\x#1{\begingroup\itshape\texttt{\%} #1\endgroup}
%   |\fibnumPreCalc{50}|\\
%   \x{calculates and stores the values for indexes 47..50.}\\
%   \x{(Values for 0..46 are already stored by the package.)}\\
%   |\fibnum{49}| \x{uses the stored value}\\
%   |\fibnum{51}|
%   \x{only calculates $F_{51}$ from stored values $F_{49}$ and $F_{50}$}\\
%   |\fibnumPreCalc{100}|\\
%   \x{calculates and stores the values for indexes 51..100}\\
%   |\fibnum{100}| \x{uses the stored value for $F_{100}$}\\
%   |\fibnum{-100}|
%   \x{uses the stored value for $F_{100}$}\\
%   \x{$F_{-100}=-F_{100}$ according to equation \eqref{eq:def}.}
% \end{quote}
%
% \StopEventually{
% }
%
% \section{Implementation}
%
% \subsection{Identification}
%
%    \begin{macrocode}
%<*package>
%    \end{macrocode}
%    Reload check, especially if the package is not used with \LaTeX.
%    \begin{macrocode}
\begingroup\catcode61\catcode48\catcode32=10\relax%
  \catcode13=5 % ^^M
  \endlinechar=13 %
  \catcode35=6 % #
  \catcode39=12 % '
  \catcode44=12 % ,
  \catcode45=12 % -
  \catcode46=12 % .
  \catcode58=12 % :
  \catcode64=11 % @
  \catcode123=1 % {
  \catcode125=2 % }
  \expandafter\let\expandafter\x\csname ver@fibnum.sty\endcsname
  \ifx\x\relax % plain-TeX, first loading
  \else
    \def\empty{}%
    \ifx\x\empty % LaTeX, first loading,
      % variable is initialized, but \ProvidesPackage not yet seen
    \else
      \expandafter\ifx\csname PackageInfo\endcsname\relax
        \def\x#1#2{%
          \immediate\write-1{Package #1 Info: #2.}%
        }%
      \else
        \def\x#1#2{\PackageInfo{#1}{#2, stopped}}%
      \fi
      \x{fibnum}{The package is already loaded}%
      \aftergroup\endinput
    \fi
  \fi
\endgroup%
%    \end{macrocode}
%    Package identification:
%    \begin{macrocode}
\begingroup\catcode61\catcode48\catcode32=10\relax%
  \catcode13=5 % ^^M
  \endlinechar=13 %
  \catcode35=6 % #
  \catcode39=12 % '
  \catcode40=12 % (
  \catcode41=12 % )
  \catcode44=12 % ,
  \catcode45=12 % -
  \catcode46=12 % .
  \catcode47=12 % /
  \catcode58=12 % :
  \catcode64=11 % @
  \catcode91=12 % [
  \catcode93=12 % ]
  \catcode123=1 % {
  \catcode125=2 % }
  \expandafter\ifx\csname ProvidesPackage\endcsname\relax
    \def\x#1#2#3[#4]{\endgroup
      \immediate\write-1{Package: #3 #4}%
      \xdef#1{#4}%
    }%
  \else
    \def\x#1#2[#3]{\endgroup
      #2[{#3}]%
      \ifx#1\@undefined
        \xdef#1{#3}%
      \fi
      \ifx#1\relax
        \xdef#1{#3}%
      \fi
    }%
  \fi
\expandafter\x\csname ver@fibnum.sty\endcsname
\ProvidesPackage{fibnum}%
  [2016/05/16 v1.1 Fibonacci numbers (HO)]%
%    \end{macrocode}
%
%    \begin{macrocode}
\begingroup\catcode61\catcode48\catcode32=10\relax%
  \catcode13=5 % ^^M
  \endlinechar=13 %
  \catcode123=1 % {
  \catcode125=2 % }
  \catcode64=11 % @
  \def\x{\endgroup
    \expandafter\edef\csname FibNum@AtEnd\endcsname{%
      \endlinechar=\the\endlinechar\relax
      \catcode13=\the\catcode13\relax
      \catcode32=\the\catcode32\relax
      \catcode35=\the\catcode35\relax
      \catcode61=\the\catcode61\relax
      \catcode64=\the\catcode64\relax
      \catcode123=\the\catcode123\relax
      \catcode125=\the\catcode125\relax
    }%
  }%
\x\catcode61\catcode48\catcode32=10\relax%
\catcode13=5 % ^^M
\endlinechar=13 %
\catcode35=6 % #
\catcode64=11 % @
\catcode123=1 % {
\catcode125=2 % }
\def\TMP@EnsureCode#1#2{%
  \edef\FibNum@AtEnd{%
    \FibNum@AtEnd
    \catcode#1=\the\catcode#1\relax
  }%
  \catcode#1=#2\relax
}
\TMP@EnsureCode{33}{12}% !
%\TMP@EnsureCode{36}{3}% $
%\TMP@EnsureCode{38}{4}% &
\TMP@EnsureCode{40}{12}% (
\TMP@EnsureCode{41}{12}% )
\TMP@EnsureCode{45}{12}% -
\TMP@EnsureCode{46}{12}% .
\TMP@EnsureCode{47}{12}% /
\TMP@EnsureCode{58}{12}% :
\TMP@EnsureCode{60}{12}% <
\TMP@EnsureCode{62}{12}% >
\TMP@EnsureCode{91}{12}% [
%\TMP@EnsureCode{96}{12}% `
\TMP@EnsureCode{93}{12}% ]
%\TMP@EnsureCode{94}{12}% ^ (superscript) (!)
%\TMP@EnsureCode{124}{12}% |
\edef\FibNum@AtEnd{\FibNum@AtEnd\noexpand\endinput}
%    \end{macrocode}
%
% \subsection{Package resources}
%
%    \begin{macrocode}
\begingroup\expandafter\expandafter\expandafter\endgroup
\expandafter\ifx\csname RequirePackage\endcsname\relax
  \def\TMP@RequirePackage#1[#2]{%
    \begingroup\expandafter\expandafter\expandafter\endgroup
    \expandafter\ifx\csname ver@#1.sty\endcsname\relax
      \input #1.sty\relax
    \fi
  }%
  \TMP@RequirePackage{ltxcmds}[2011/04/18]%
  \TMP@RequirePackage{intcalc}[2007/09/27]%
  \TMP@RequirePackage{bigintcalc}[2007/11/11]%
\else
  \RequirePackage{ltxcmds}[2011/04/18]%
  \RequirePackage{intcalc}[2007/09/27]%
  \RequirePackage{bigintcalc}[2007/11/11]%
\fi
%    \end{macrocode}
%
% \subsection{Setup precalculated values}
%
%    \begin{macrocode}
\def\FibNum@temp#1{%
  \expandafter\def\csname FibNum@#1\endcsname
}
\catcode46=9 % dots are ignored
\FibNum@temp{0}{0}
\FibNum@temp{1}{1}
\FibNum@temp{2}{1}
\FibNum@temp{3}{2}
\FibNum@temp{4}{3}
\FibNum@temp{5}{5}
\FibNum@temp{6}{8}
\FibNum@temp{7}{13}
\FibNum@temp{8}{21}
\FibNum@temp{9}{34}
\FibNum@temp{10}{55}
\FibNum@temp{11}{89}
\FibNum@temp{12}{144}
\FibNum@temp{13}{233}
\FibNum@temp{14}{377}
\FibNum@temp{15}{610}
\FibNum@temp{16}{987}
\FibNum@temp{17}{1.597}
\FibNum@temp{18}{2.584}
\FibNum@temp{19}{4.181}
\FibNum@temp{20}{6.765}
\FibNum@temp{21}{10.946}
\FibNum@temp{22}{17.711}
\FibNum@temp{23}{28.657}
\FibNum@temp{24}{46.368}
\FibNum@temp{25}{75.025}
\FibNum@temp{26}{121.393}
\FibNum@temp{27}{196.418}
\FibNum@temp{28}{317.811}
\FibNum@temp{29}{514.229}
\FibNum@temp{30}{832.040}
\FibNum@temp{31}{1.346.269}
\FibNum@temp{32}{2.178.309}
\FibNum@temp{33}{3.524.578}
\FibNum@temp{34}{5.702.887}
\FibNum@temp{35}{9.227.465}
\FibNum@temp{36}{14.930.352}
\FibNum@temp{37}{24.157.817}
\FibNum@temp{38}{39.088.169}
\FibNum@temp{39}{63.245.986}
\FibNum@temp{40}{102.334.155}
\FibNum@temp{41}{165.580.141}
\FibNum@temp{42}{267.914.296}
\FibNum@temp{43}{433.494.437}
\FibNum@temp{44}{701.408.733}
\FibNum@temp{45}{1.134.903.170}
\FibNum@temp{46}{1.836.311.903}
%    \end{macrocode}
%    \begin{macro}{\FibNum@max}
%    \begin{macrocode}
\def\FibNum@max{46}
%    \end{macrocode}
%    \end{macro}
%
% \subsection{Macros for precalculating values}
%
%    \begin{macro}{\fibnumPreCalc}
%    \begin{macrocode}
\def\fibnumPreCalc#1{%
  \expandafter\expandafter\expandafter
  \FibNum@PreCalc\intcalcNum{#1}/%
}
%    \end{macrocode}
%    \end{macro}
%    \begin{macro}{\FibNum@PreCalc}
%    \begin{macrocode}
\def\FibNum@PreCalc#1/{%
  \ifnum#1<\ltx@zero
    \expandafter\FibNum@PreCalc\ltx@gobble#1/%
  \else
    \ifnum#1>\FibNum@max
      \begingroup
        \ltx@LocDimenA=#1sp\relax
        \countdef\FibNum@i=255\relax
        \FibNum@i=\FibNum@max\relax
        \edef\FibNum@temp{%
          \csname FibNum@\the\FibNum@i\endcsname/%
        }%
        \advance\FibNum@i by -1\relax
        \edef\FibNum@temp{%
          \FibNum@temp
          \csname FibNum@\the\FibNum@i\endcsname
        }%
        \advance\FibNum@i\ltx@two
        \iftrue
          \expandafter\FibNum@PreCalcAux\FibNum@temp
        \fi
      \endgroup
    \fi
  \fi
}
%    \end{macrocode}
%    \end{macro}
%    \begin{macro}{\FibNum@PreCalcAux}
%    \begin{macrocode}
\def\FibNum@PreCalcAux#1/#2\fi{%
  \fi
  \edef\FibNum@temp{\BigIntCalcAdd#1!#2!}%
  \global\expandafter
  \let\csname FibNum@\the\FibNum@i\endcsname\FibNum@temp
  \ifnum\FibNum@i=\ltx@LocDimenA
    \xdef\FibNum@max{\the\FibNum@i}%
  \else
    \advance\FibNum@i\ltx@one
    \expandafter\FibNum@PreCalcAux\FibNum@temp/#1%
  \fi
}
%    \end{macrocode}
%    \end{macro}
%
% \subsection{Expandable calculations}
%
%    \begin{macro}{\fibnum}
%    \begin{macrocode}
\def\fibnum#1{%
  \romannumeral
  \expandafter\expandafter\expandafter\FibNum@Do\intcalcNum{#1}/%
}
%    \end{macrocode}
%    \end{macro}
%    \begin{macro}{\FibNum@Do}
%    \begin{macrocode}
\def\FibNum@Do#1/{%
  \ifnum#1<\ltx@zero
    \FibNum@ReturnAfterElseFiFi{%
      \ifodd#1 %
        \expandafter\expandafter\expandafter\ltx@zero
      \else
        \expandafter\expandafter\expandafter\ltx@zero
        \expandafter\expandafter\expandafter-%
      \fi
      \romannumeral
      \expandafter\FibNum@Do\ltx@gobble#1/%
    }%
  \else
    \ifnum\FibNum@max<#1 %
      \ltx@ReturnAfterElseFi{%
        \expandafter
        \FibNum@ExpCalc\number\expandafter\IntCalcInc\FibNum@max!%
        \expandafter\expandafter\expandafter/%
        \csname FibNum@\FibNum@max
        \expandafter\expandafter\expandafter\endcsname
        \expandafter\expandafter\expandafter/%
        \csname FibNum@\expandafter\IntCalcDec\FibNum@max!%
        \endcsname/%
        #1%
      }%
    \else
      \expandafter\expandafter\expandafter\ltx@zero
      \csname FibNum@#1\expandafter\expandafter\expandafter\endcsname
    \fi
  \fi
}
%    \end{macrocode}
%    \end{macro}
%    \begin{macro}{\FibNum@ReturnAfterElseFiFi}
%    \begin{macrocode}
\def\FibNum@ReturnAfterElseFiFi#1\else#2\fi\fi{\fi#1}
%    \end{macrocode}
%    \end{macro}
%    \begin{macro}{\FibNum@ExpCalc}
%    \begin{macrocode}
\def\FibNum@ExpCalc#1/#2/#3/#4\fi{%
  \fi
  \ifnum#1=#4 %
    \ltx@ReturnAfterElseFi{%
      \expandafter\expandafter\expandafter\ltx@zero
      \BigIntCalcAdd#2!#3!%
    }%
  \else
    \expandafter\FibNum@ExpCalc
    \number\IntCalcInc#1!%
    \expandafter\expandafter\expandafter/%
    \BigIntCalcAdd#2!#3!/%
    #2/#4%
  \fi
}
%    \end{macrocode}
%    \end{macro}
%
%    \begin{macrocode}
\FibNum@AtEnd%
%</package>
%    \end{macrocode}
%
% \section{Test}
%
% \subsection{Catcode checks for loading}
%
%    \begin{macrocode}
%<*test1>
%    \end{macrocode}
%    \begin{macrocode}
\catcode`\{=1 %
\catcode`\}=2 %
\catcode`\#=6 %
\catcode`\@=11 %
\expandafter\ifx\csname count@\endcsname\relax
  \countdef\count@=255 %
\fi
\expandafter\ifx\csname @gobble\endcsname\relax
  \long\def\@gobble#1{}%
\fi
\expandafter\ifx\csname @firstofone\endcsname\relax
  \long\def\@firstofone#1{#1}%
\fi
\expandafter\ifx\csname loop\endcsname\relax
  \expandafter\@firstofone
\else
  \expandafter\@gobble
\fi
{%
  \def\loop#1\repeat{%
    \def\body{#1}%
    \iterate
  }%
  \def\iterate{%
    \body
      \let\next\iterate
    \else
      \let\next\relax
    \fi
    \next
  }%
  \let\repeat=\fi
}%
\def\RestoreCatcodes{}
\count@=0 %
\loop
  \edef\RestoreCatcodes{%
    \RestoreCatcodes
    \catcode\the\count@=\the\catcode\count@\relax
  }%
\ifnum\count@<255 %
  \advance\count@ 1 %
\repeat

\def\RangeCatcodeInvalid#1#2{%
  \count@=#1\relax
  \loop
    \catcode\count@=15 %
  \ifnum\count@<#2\relax
    \advance\count@ 1 %
  \repeat
}
\def\RangeCatcodeCheck#1#2#3{%
  \count@=#1\relax
  \loop
    \ifnum#3=\catcode\count@
    \else
      \errmessage{%
        Character \the\count@\space
        with wrong catcode \the\catcode\count@\space
        instead of \number#3%
      }%
    \fi
  \ifnum\count@<#2\relax
    \advance\count@ 1 %
  \repeat
}
\def\space{ }
\expandafter\ifx\csname LoadCommand\endcsname\relax
  \def\LoadCommand{\input fibnum.sty\relax}%
\fi
\def\Test{%
  \RangeCatcodeInvalid{0}{47}%
  \RangeCatcodeInvalid{58}{64}%
  \RangeCatcodeInvalid{91}{96}%
  \RangeCatcodeInvalid{123}{255}%
  \catcode`\@=12 %
  \catcode`\\=0 %
  \catcode`\%=14 %
  \LoadCommand
  \RangeCatcodeCheck{0}{36}{15}%
  \RangeCatcodeCheck{37}{37}{14}%
  \RangeCatcodeCheck{38}{47}{15}%
  \RangeCatcodeCheck{48}{57}{12}%
  \RangeCatcodeCheck{58}{63}{15}%
  \RangeCatcodeCheck{64}{64}{12}%
  \RangeCatcodeCheck{65}{90}{11}%
  \RangeCatcodeCheck{91}{91}{15}%
  \RangeCatcodeCheck{92}{92}{0}%
  \RangeCatcodeCheck{93}{96}{15}%
  \RangeCatcodeCheck{97}{122}{11}%
  \RangeCatcodeCheck{123}{255}{15}%
  \RestoreCatcodes
}
\Test
\csname @@end\endcsname
\end
%    \end{macrocode}
%    \begin{macrocode}
%</test1>
%    \end{macrocode}
%
% \subsection{Test calculations}
%
%    \begin{macrocode}
%<*test-calc>
\catcode`\{=1 %
\catcode`\}=2 %
\catcode`\#=6 %
\catcode`\@=11 %
\begingroup\expandafter\expandafter\expandafter\endgroup
\expandafter\ifx\csname RequirePackage\endcsname\relax
  \input fibnum.sty\relax
\else
  \RequirePackage{fibnum}[2016/05/16]%
\fi
\def\TestSet{%
  \test{0}{0}%
  \test{1}{1}%
  \test{2}{1}%
  \test{3}{2}%
  \test{4}{3}%
  \test{5}{5}%
  \test{6}{8}%
  \test{7}{13}%
  \test{8}{21}%
  \test{9}{34}%
  \test{10}{55}%
  \test{11}{89}%
  \test{12}{144}%
  \test{13}{233}%
  \test{14}{377}%
  \test{15}{610}%
  \test{16}{987}%
  \test{17}{1597}%
  \test{18}{2584}%
  \test{19}{4181}%
  \test{20}{6765}%
  \test{21}{10946}%
  \test{22}{17711}%
  \test{23}{28657}%
  \test{24}{46368}%
  \test{25}{75025}%
  \test{26}{121393}%
  \test{27}{196418}%
  \test{28}{317811}%
  \test{29}{514229}%
  \test{30}{832040}%
  \test{31}{1346269}%
  \test{32}{2178309}%
  \test{33}{3524578}%
  \test{34}{5702887}%
  \test{35}{9227465}%
  \test{36}{14930352}%
  \test{37}{24157817}%
  \test{38}{39088169}%
  \test{39}{63245986}%
  \test{40}{102334155}%
  \test{41}{165580141}%
  \test{42}{267914296}%
  \test{43}{433494437}%
  \test{44}{701408733}%
  \test{45}{1134903170}%
  \test{46}{1836311903}%
  \test{47}{2971215073}%
  \test{48}{4807526976}%
  \test{49}{7778742049}%
  \test{50}{12586269025}%
  \test{51}{20365011074}%
  \test{52}{32951280099}%
  \test{53}{53316291173}%
  \test{54}{86267571272}%
  \test{55}{139583862445}%
  \test{56}{225851433717}%
  \test{57}{365435296162}%
  \test{58}{591286729879}%
  \test{59}{956722026041}%
  \test{60}{1548008755920}%
  \test{61}{2504730781961}%
  \test{62}{4052739537881}%
  \test{63}{6557470319842}%
  \test{64}{10610209857723}%
  \test{65}{17167680177565}%
  \test{66}{27777890035288}%
  \test{67}{44945570212853}%
  \test{68}{72723460248141}%
  \test{69}{117669030460994}%
  \test{70}{190392490709135}%
  \test{71}{308061521170129}%
  \test{72}{498454011879264}%
  \test{73}{806515533049393}%
}
\def\msg#{\immediate\write16}
\def\test#1#2{%
  \TestAux{#1}{#2}%
  \ifnum#1=0 %
  \else
    \ifodd#1 %
      \TestAux{-#1}{#2}%
    \else
      \TestAux{-#1}{-#2}%
    \fi
  \fi
}
\def\TestAux#1#2{%
  \def\Expected{#2}%
  \expandafter\expandafter\expandafter\def
  \expandafter\expandafter\expandafter\Result
  \expandafter\expandafter\expandafter{%
    \fibnum{#1}%
  }%
  \ltx@onelevel@sanitize\Result
  \ifx\Result\Expected
    \msg{* #1: ok.}%
  \else
    \msg{! fib(#1) = #2}%
    \errmessage{fib(#1) <> \Result}%
  \fi
}
\TestSet
\setbox0=\hbox{%
  \msg{* PreCalc{73}}%
  \fibnumPreCalc{73}%
}
\ifdim\wd0=0pt
\else
  \errmessage{Unwanted stuff in PreCalc}%
\fi
\TestSet
\csname @@end\endcsname\end
%</test-calc>
%    \end{macrocode}
%
% \section{Installation}
%
% \subsection{Download}
%
% \paragraph{Package.} This package is available on
% CTAN\footnote{\url{http://ctan.org/pkg/fibnum}}:
% \begin{description}
% \item[\CTAN{macros/latex/contrib/oberdiek/fibnum.dtx}] The source file.
% \item[\CTAN{macros/latex/contrib/oberdiek/fibnum.pdf}] Documentation.
% \end{description}
%
%
% \paragraph{Bundle.} All the packages of the bundle `oberdiek'
% are also available in a TDS compliant ZIP archive. There
% the packages are already unpacked and the documentation files
% are generated. The files and directories obey the TDS standard.
% \begin{description}
% \item[\CTAN{install/macros/latex/contrib/oberdiek.tds.zip}]
% \end{description}
% \emph{TDS} refers to the standard ``A Directory Structure
% for \TeX\ Files'' (\CTAN{tds/tds.pdf}). Directories
% with \xfile{texmf} in their name are usually organized this way.
%
% \subsection{Bundle installation}
%
% \paragraph{Unpacking.} Unpack the \xfile{oberdiek.tds.zip} in the
% TDS tree (also known as \xfile{texmf} tree) of your choice.
% Example (linux):
% \begin{quote}
%   |unzip oberdiek.tds.zip -d ~/texmf|
% \end{quote}
%
% \paragraph{Script installation.}
% Check the directory \xfile{TDS:scripts/oberdiek/} for
% scripts that need further installation steps.
% Package \xpackage{attachfile2} comes with the Perl script
% \xfile{pdfatfi.pl} that should be installed in such a way
% that it can be called as \texttt{pdfatfi}.
% Example (linux):
% \begin{quote}
%   |chmod +x scripts/oberdiek/pdfatfi.pl|\\
%   |cp scripts/oberdiek/pdfatfi.pl /usr/local/bin/|
% \end{quote}
%
% \subsection{Package installation}
%
% \paragraph{Unpacking.} The \xfile{.dtx} file is a self-extracting
% \docstrip\ archive. The files are extracted by running the
% \xfile{.dtx} through \plainTeX:
% \begin{quote}
%   \verb|tex fibnum.dtx|
% \end{quote}
%
% \paragraph{TDS.} Now the different files must be moved into
% the different directories in your installation TDS tree
% (also known as \xfile{texmf} tree):
% \begin{quote}
% \def\t{^^A
% \begin{tabular}{@{}>{\ttfamily}l@{ $\rightarrow$ }>{\ttfamily}l@{}}
%   fibnum.sty & tex/generic/oberdiek/fibnum.sty\\
%   fibnum.pdf & doc/latex/oberdiek/fibnum.pdf\\
%   test/fibnum-test1.tex & doc/latex/oberdiek/test/fibnum-test1.tex\\
%   test/fibnum-test-calc.tex & doc/latex/oberdiek/test/fibnum-test-calc.tex\\
%   fibnum.dtx & source/latex/oberdiek/fibnum.dtx\\
% \end{tabular}^^A
% }^^A
% \sbox0{\t}^^A
% \ifdim\wd0>\linewidth
%   \begingroup
%     \advance\linewidth by\leftmargin
%     \advance\linewidth by\rightmargin
%   \edef\x{\endgroup
%     \def\noexpand\lw{\the\linewidth}^^A
%   }\x
%   \def\lwbox{^^A
%     \leavevmode
%     \hbox to \linewidth{^^A
%       \kern-\leftmargin\relax
%       \hss
%       \usebox0
%       \hss
%       \kern-\rightmargin\relax
%     }^^A
%   }^^A
%   \ifdim\wd0>\lw
%     \sbox0{\small\t}^^A
%     \ifdim\wd0>\linewidth
%       \ifdim\wd0>\lw
%         \sbox0{\footnotesize\t}^^A
%         \ifdim\wd0>\linewidth
%           \ifdim\wd0>\lw
%             \sbox0{\scriptsize\t}^^A
%             \ifdim\wd0>\linewidth
%               \ifdim\wd0>\lw
%                 \sbox0{\tiny\t}^^A
%                 \ifdim\wd0>\linewidth
%                   \lwbox
%                 \else
%                   \usebox0
%                 \fi
%               \else
%                 \lwbox
%               \fi
%             \else
%               \usebox0
%             \fi
%           \else
%             \lwbox
%           \fi
%         \else
%           \usebox0
%         \fi
%       \else
%         \lwbox
%       \fi
%     \else
%       \usebox0
%     \fi
%   \else
%     \lwbox
%   \fi
% \else
%   \usebox0
% \fi
% \end{quote}
% If you have a \xfile{docstrip.cfg} that configures and enables \docstrip's
% TDS installing feature, then some files can already be in the right
% place, see the documentation of \docstrip.
%
% \subsection{Refresh file name databases}
%
% If your \TeX~distribution
% (\teTeX, \mikTeX, \dots) relies on file name databases, you must refresh
% these. For example, \teTeX\ users run \verb|texhash| or
% \verb|mktexlsr|.
%
% \subsection{Some details for the interested}
%
% \paragraph{Attached source.}
%
% The PDF documentation on CTAN also includes the
% \xfile{.dtx} source file. It can be extracted by
% AcrobatReader 6 or higher. Another option is \textsf{pdftk},
% e.g. unpack the file into the current directory:
% \begin{quote}
%   \verb|pdftk fibnum.pdf unpack_files output .|
% \end{quote}
%
% \paragraph{Unpacking with \LaTeX.}
% The \xfile{.dtx} chooses its action depending on the format:
% \begin{description}
% \item[\plainTeX:] Run \docstrip\ and extract the files.
% \item[\LaTeX:] Generate the documentation.
% \end{description}
% If you insist on using \LaTeX\ for \docstrip\ (really,
% \docstrip\ does not need \LaTeX), then inform the autodetect routine
% about your intention:
% \begin{quote}
%   \verb|latex \let\install=y\input{fibnum.dtx}|
% \end{quote}
% Do not forget to quote the argument according to the demands
% of your shell.
%
% \paragraph{Generating the documentation.}
% You can use both the \xfile{.dtx} or the \xfile{.drv} to generate
% the documentation. The process can be configured by the
% configuration file \xfile{ltxdoc.cfg}. For instance, put this
% line into this file, if you want to have A4 as paper format:
% \begin{quote}
%   \verb|\PassOptionsToClass{a4paper}{article}|
% \end{quote}
% An example follows how to generate the
% documentation with pdf\LaTeX:
% \begin{quote}
%\begin{verbatim}
%pdflatex fibnum.dtx
%bibtex fibnum.aux
%makeindex -s gind.ist fibnum.idx
%pdflatex fibnum.dtx
%makeindex -s gind.ist fibnum.idx
%pdflatex fibnum.dtx
%\end{verbatim}
% \end{quote}
%
% \printbibliography[
%   heading=bibnumbered,
% ]
%
% \begin{History}
%   \begin{Version}{2012/04/08 v1.0}
%   \item
%     First version.
%   \end{Version}
%   \begin{Version}{2016/05/16 v1.1}
%   \item
%     Documentation updates.
%   \end{Version}
% \end{History}
%
% \PrintIndex
%
% \Finale
\endinput
|
% \end{quote}
% Do not forget to quote the argument according to the demands
% of your shell.
%
% \paragraph{Generating the documentation.}
% You can use both the \xfile{.dtx} or the \xfile{.drv} to generate
% the documentation. The process can be configured by the
% configuration file \xfile{ltxdoc.cfg}. For instance, put this
% line into this file, if you want to have A4 as paper format:
% \begin{quote}
%   \verb|\PassOptionsToClass{a4paper}{article}|
% \end{quote}
% An example follows how to generate the
% documentation with pdf\LaTeX:
% \begin{quote}
%\begin{verbatim}
%pdflatex fibnum.dtx
%bibtex fibnum.aux
%makeindex -s gind.ist fibnum.idx
%pdflatex fibnum.dtx
%makeindex -s gind.ist fibnum.idx
%pdflatex fibnum.dtx
%\end{verbatim}
% \end{quote}
%
% \printbibliography[
%   heading=bibnumbered,
% ]
%
% \begin{History}
%   \begin{Version}{2012/04/08 v1.0}
%   \item
%     First version.
%   \end{Version}
%   \begin{Version}{2016/05/16 v1.1}
%   \item
%     Documentation updates.
%   \end{Version}
% \end{History}
%
% \PrintIndex
%
% \Finale
\endinput

%        (quote the arguments according to the demands of your shell)
%
% Documentation:
%    (a) If fibnum.drv is present:
%           latex fibnum.drv
%    (b) Without fibnum.drv:
%           latex fibnum.dtx; ...
%    The class ltxdoc loads the configuration file ltxdoc.cfg
%    if available. Here you can specify further options, e.g.
%    use A4 as paper format:
%       \PassOptionsToClass{a4paper}{article}
%
%    Programm calls to get the documentation (example):
%       pdflatex fibnum.dtx
%       bibtex fibnum.aux
%       makeindex -s gind.ist fibnum.idx
%       pdflatex fibnum.dtx
%       makeindex -s gind.ist fibnum.idx
%       pdflatex fibnum.dtx
%
% Installation:
%    TDS:tex/generic/oberdiek/fibnum.sty
%    TDS:doc/latex/oberdiek/fibnum.pdf
%    TDS:doc/latex/oberdiek/test/fibnum-test1.tex
%    TDS:doc/latex/oberdiek/test/fibnum-test-calc.tex
%    TDS:source/latex/oberdiek/fibnum.dtx
%
%<*ignore>
\begingroup
  \catcode123=1 %
  \catcode125=2 %
  \def\x{LaTeX2e}%
\expandafter\endgroup
\ifcase 0\ifx\install y1\fi\expandafter
         \ifx\csname processbatchFile\endcsname\relax\else1\fi
         \ifx\fmtname\x\else 1\fi\relax
\else\csname fi\endcsname
%</ignore>
%<*install>
\input docstrip.tex
\Msg{************************************************************************}
\Msg{* Installation}
\Msg{* Package: fibnum 2016/05/16 v1.1 Fibonacci numbers (HO)}
\Msg{************************************************************************}

\keepsilent
\askforoverwritefalse

\let\MetaPrefix\relax
\preamble

This is a generated file.

Project: fibnum
Version: 2016/05/16 v1.1

Copyright (C) 2012 by
   Heiko Oberdiek <heiko.oberdiek at googlemail.com>

This work may be distributed and/or modified under the
conditions of the LaTeX Project Public License, either
version 1.3c of this license or (at your option) any later
version. This version of this license is in
   http://www.latex-project.org/lppl/lppl-1-3c.txt
and the latest version of this license is in
   http://www.latex-project.org/lppl.txt
and version 1.3 or later is part of all distributions of
LaTeX version 2005/12/01 or later.

This work has the LPPL maintenance status "maintained".

This Current Maintainer of this work is Heiko Oberdiek.

The Base Interpreter refers to any `TeX-Format',
because some files are installed in TDS:tex/generic//.

This work consists of the main source file fibnum.dtx
and the derived files
   fibnum.sty, fibnum.pdf, fibnum.ins, fibnum.drv, fibnum.bib,
   fibnum-test1.tex, fibnum-test-calc.tex.

\endpreamble
\let\MetaPrefix\DoubleperCent

\generate{%
  \file{fibnum.ins}{\from{fibnum.dtx}{install}}%
  \file{fibnum.drv}{\from{fibnum.dtx}{driver}}%
  \nopreamble
  \nopostamble
  \file{fibnum.bib}{\from{fibnum.dtx}{bib}}%
  \usepreamble\defaultpreamble
  \usepostamble\defaultpostamble
  \usedir{tex/generic/oberdiek}%
  \file{fibnum.sty}{\from{fibnum.dtx}{package}}%
  \usedir{doc/latex/oberdiek/test}%
  \file{fibnum-test1.tex}{\from{fibnum.dtx}{test1}}%
  \file{fibnum-test-calc.tex}{\from{fibnum.dtx}{test-calc}}%
}

\catcode32=13\relax% active space
\let =\space%
\Msg{************************************************************************}
\Msg{*}
\Msg{* To finish the installation you have to move the following}
\Msg{* file into a directory searched by TeX:}
\Msg{*}
\Msg{*     fibnum.sty}
\Msg{*}
\Msg{* To produce the documentation run the file `fibnum.drv'}
\Msg{* through LaTeX.}
\Msg{*}
\Msg{* Happy TeXing!}
\Msg{*}
\Msg{************************************************************************}

\endbatchfile
%</install>
%<*bib>
@online{texhax:abraham,
  author={Abraham, Jan},
  title={[texhax] Beginner in TEX MACRO to compute functions},
  date={2012-04-07},
  url={http://tug.org/pipermail/texhax/2012-April/019146.html},
  urldate={2012-04-08},
}
@article{knuth:negafibonacci,
  author={Knuth, Donald E.},
  title={Negafibonacci Numbers and the Hyperbolic Plane},
  date={2008-12-11},
  url={http://research.allacademic.com/meta/p206842_index.html},
}
@online{wikipedia:negafibonacci,
  author={{Wikipedia contributors}},
  organization={{Wikipedia, The Free Encyclopedia}},
  title={Fibonacci numbers},
  language={langenglish},
  version={486266088},
  date={2012-04-08},
  url={http://en.wikipedia.org/w/index.php?title=Fibonacci_number&oldid=486266088},
  urldate={2012-04-08},
}
%</bib>
%<*ignore>
\fi
%</ignore>
%<*driver>
\NeedsTeXFormat{LaTeX2e}
\ProvidesFile{fibnum.drv}%
  [2016/05/16 v1.1 Fibonacci numbers (HO)]%
\documentclass{ltxdoc}
\usepackage{amsmath,amsfonts}
\usepackage{siunitx}
\usepackage{array}
\usepackage{tabularx}
\usepackage{fibnum}[2016/05/16]
\usepackage{holtxdoc}[2011/11/22]
\usepackage{csquotes}
\usepackage[
  bibencoding=ascii,
  alldates=iso8601,
]{biblatex}[2011/11/13]
\bibliography{oberdiek-source}
\bibliography{fibnum}
\begin{document}
  \DocInput{fibnum.dtx}%
\end{document}
%</driver>
% \fi
%
%
% \CharacterTable
%  {Upper-case    \A\B\C\D\E\F\G\H\I\J\K\L\M\N\O\P\Q\R\S\T\U\V\W\X\Y\Z
%   Lower-case    \a\b\c\d\e\f\g\h\i\j\k\l\m\n\o\p\q\r\s\t\u\v\w\x\y\z
%   Digits        \0\1\2\3\4\5\6\7\8\9
%   Exclamation   \!     Double quote  \"     Hash (number) \#
%   Dollar        \$     Percent       \%     Ampersand     \&
%   Acute accent  \'     Left paren    \(     Right paren   \)
%   Asterisk      \*     Plus          \+     Comma         \,
%   Minus         \-     Point         \.     Solidus       \/
%   Colon         \:     Semicolon     \;     Less than     \<
%   Equals        \=     Greater than  \>     Question mark \?
%   Commercial at \@     Left bracket  \[     Backslash     \\
%   Right bracket \]     Circumflex    \^     Underscore    \_
%   Grave accent  \`     Left brace    \{     Vertical bar  \|
%   Right brace   \}     Tilde         \~}
%
% \GetFileInfo{fibnum.drv}
%
% \title{The \xpackage{fibnum} package}
% \date{2016/05/16 v1.1}
% \author{Heiko Oberdiek\thanks
% {Please report any issues at https://github.com/ho-tex/oberdiek/issues}\\
% \xemail{heiko.oberdiek at googlemail.com}}
%
% \maketitle
%
% \begin{abstract}
% The package \xpackage{fibnum} provides expandable fibonacci
% numbers for both \hologo{LaTeX} and \hologo{plainTeX}.
% \end{abstract}
%
% \tableofcontents
%
% \section{Documentation}
%
% In the mailing list \textsf{texhax} Jan Abraham asked the question,
% how to get Fibonacci numbers in \hologo{TeX} \cite{texhax:abraham}:
% \begin{quote}
% Write a Macro in \hologo{TeX} that compute the function |\fib{m}|
% All fibonacci numbers from 1 to $m$ ($m < 40$).
% \end{quote}
% This packages provides an expandable implementation for the
% calculation of these numbers for a much larger set of indexes.
% For practical reasons the index is restricted to the same limitations
% that apply for \hologo{TeX} integer numbers.
% The range of the Fibonacci numbers, however, are not limited
% by the algorithm. They are only restricted to memory limitations,
% if they are hit.
%
% The package is loaded as \hologo{LaTeX} package in \hologo{LaTeX}:
% \begin{quote}
%   |\usepackage{fibnum}|
% \end{quote}
% and as file in \hologo{plainTeX}:
% \begin{quote}
%   |\input fibnum.sty|
% \end{quote}
% The package does not know any options and it provides
% the macros \cs{fibnum} and \cs{fibnumPreCalc}.
%
% \begin{declcs}{fibnum} \M{index}
% \end{declcs}
% Macro \cs{fibnum} expects a \hologo{TeX} number as \meta{index}
% in the official \hologo{TeX} number range from $-(2^{31}-1)$ up to
% $2^{31}-1$. In exact two expansion steps the macro expands to
% the Fibnoacci number $F_{\text{\meta{index}}}$. In case of a negative
% \meta{index}, the ``negafibonacci'' number \cite{wikipedia:negafibonacci}
% is used. Formally the Fibonacci number $F_n$ with integer
% index~$n$, $n\in\mathbb{Z}$ and
% $n\in[\num{-2147483647},\num{2147483647}]$ that is returned by macro
% \cs{fibnum} with numerical argument $n$ is defined the following way:
% \begin{gather}
%   \label{eq:def}
%   F_n =
%   \begin{cases}
%     0 & \text{for $n=0$}\\
%     1 & \text{for $n=1$}\\
%     F_{n-1} + F_{n-2} & \text{for $n>1$}\\
%     (-1)^{n+1}F_n & \text{for $n<0$}
%   \end{cases}
% \end{gather}
% Examples:
% \begin{quote}
%   \makeatletter
%   \def\x#1{\cs{fibnum}\{#1\}&
%     \edef\X{\fibnum{#1}}\edef\Y{\expandafter\ltx@car\X\@nil}^^A
%     \if-\Y
%       \edef\X{\expandafter\ltx@cdr\X\@nil}^^A
%       \noindent
%       \llap{-}\X
%     \else
%       \X
%     \fi
%     \tabularnewline
%   }
%   \def\y{\multicolumn{1}{@{}c@{}}{$\vdots$}\tabularnewline}
%   \DeclareUrlCommand\UrlNum{^^A
%     \urlstyle{tt}^^A
%     \def\UrlBreaks{\do\0\do\1\do\2\do\3\do\4\do\5\do\6\do\7\do\8\do\9}^^A
%   }
%   \begin{tabularx}{\dimexpr\linewidth+5.7pt\relax}{@{}>{\ttfamily}l@{ $\rightarrow$ \hphantom{\ttfamily-}}>{\ttfamily}X@{}}
%     \x{-6}
%     \x{-5}
%     \x{-4}
%     \x{-3}
%     \x{-2}
%     \x{-1}
%     \x{0}
%     \x{1}
%     \x{2}
%     \x{3}
%     \x{4}
%     \x{5}
%     \x{6}
%     \y
%     \x{10}
%     \y
%     \x{46}
%     \y
%     \cs{fibnum}\{100\} & 354224848179261915075
%     \tabularnewline
%     \y
%     \cs{fibnum}\{200\} & 280571172992510140037611932413038677189525
%     \tabularnewline
%     \y
%     \cs{fibnum}\{1000\} &
%       \raggedright
%       \UrlNum{^^A
%         434665576869374564356885276750406258025646^^A
%         605173717804024817290895365554179490518904^^A
%         038798400792551692959225930803226347752096^^A
%         896232398733224711616429964409065331879382^^A
%         98969649928516003704476137795166849228875^^A
%       }
%     \tabularnewline
%   \end{tabularx}\kern-5.7pt\mbox{}
% \end{quote}
%
% \begin{declcs}{fibnumPreCalc} \M{index}
% \end{declcs}
% The package already provides precalculated Fibonacci numbers up to
% index~46. That means that calculations are not necessary for
% Fibonacci numbers that fit into the range of \hologo{TeX}
% numbers. Because macro \cs{fibnum} is expandable, it cannot
% store calculated Fibonacci numbers for later use. Macro definitions
% are forbidden in expandable contexts. If larger Fibonacci numbers
% are used more than once, than the compilation time can be shortened
% by calculating and storing the Fibonacci numbers beforehand.
% The argument \meta{index} is a \hologo{TeX} number and macro
% \cs{fibnumPreCalc} ensures that the Fibonacci numbers
% $F_0$ up to $F_{\lvert\text{\meta{index}}\rvert}$ that are not
% already known are calculated
% and stored in internal macros. Internally only non-negative
% Fibonacci numbers are stored. If \meta{index} is negative, then
% the needed positive Fibonacci numbers are calculated and stored.
% Example:
% \begin{quote}
%   \def\x#1{\begingroup\itshape\texttt{\%} #1\endgroup}
%   |\fibnumPreCalc{50}|\\
%   \x{calculates and stores the values for indexes 47..50.}\\
%   \x{(Values for 0..46 are already stored by the package.)}\\
%   |\fibnum{49}| \x{uses the stored value}\\
%   |\fibnum{51}|
%   \x{only calculates $F_{51}$ from stored values $F_{49}$ and $F_{50}$}\\
%   |\fibnumPreCalc{100}|\\
%   \x{calculates and stores the values for indexes 51..100}\\
%   |\fibnum{100}| \x{uses the stored value for $F_{100}$}\\
%   |\fibnum{-100}|
%   \x{uses the stored value for $F_{100}$}\\
%   \x{$F_{-100}=-F_{100}$ according to equation \eqref{eq:def}.}
% \end{quote}
%
% \StopEventually{
% }
%
% \section{Implementation}
%
% \subsection{Identification}
%
%    \begin{macrocode}
%<*package>
%    \end{macrocode}
%    Reload check, especially if the package is not used with \LaTeX.
%    \begin{macrocode}
\begingroup\catcode61\catcode48\catcode32=10\relax%
  \catcode13=5 % ^^M
  \endlinechar=13 %
  \catcode35=6 % #
  \catcode39=12 % '
  \catcode44=12 % ,
  \catcode45=12 % -
  \catcode46=12 % .
  \catcode58=12 % :
  \catcode64=11 % @
  \catcode123=1 % {
  \catcode125=2 % }
  \expandafter\let\expandafter\x\csname ver@fibnum.sty\endcsname
  \ifx\x\relax % plain-TeX, first loading
  \else
    \def\empty{}%
    \ifx\x\empty % LaTeX, first loading,
      % variable is initialized, but \ProvidesPackage not yet seen
    \else
      \expandafter\ifx\csname PackageInfo\endcsname\relax
        \def\x#1#2{%
          \immediate\write-1{Package #1 Info: #2.}%
        }%
      \else
        \def\x#1#2{\PackageInfo{#1}{#2, stopped}}%
      \fi
      \x{fibnum}{The package is already loaded}%
      \aftergroup\endinput
    \fi
  \fi
\endgroup%
%    \end{macrocode}
%    Package identification:
%    \begin{macrocode}
\begingroup\catcode61\catcode48\catcode32=10\relax%
  \catcode13=5 % ^^M
  \endlinechar=13 %
  \catcode35=6 % #
  \catcode39=12 % '
  \catcode40=12 % (
  \catcode41=12 % )
  \catcode44=12 % ,
  \catcode45=12 % -
  \catcode46=12 % .
  \catcode47=12 % /
  \catcode58=12 % :
  \catcode64=11 % @
  \catcode91=12 % [
  \catcode93=12 % ]
  \catcode123=1 % {
  \catcode125=2 % }
  \expandafter\ifx\csname ProvidesPackage\endcsname\relax
    \def\x#1#2#3[#4]{\endgroup
      \immediate\write-1{Package: #3 #4}%
      \xdef#1{#4}%
    }%
  \else
    \def\x#1#2[#3]{\endgroup
      #2[{#3}]%
      \ifx#1\@undefined
        \xdef#1{#3}%
      \fi
      \ifx#1\relax
        \xdef#1{#3}%
      \fi
    }%
  \fi
\expandafter\x\csname ver@fibnum.sty\endcsname
\ProvidesPackage{fibnum}%
  [2016/05/16 v1.1 Fibonacci numbers (HO)]%
%    \end{macrocode}
%
%    \begin{macrocode}
\begingroup\catcode61\catcode48\catcode32=10\relax%
  \catcode13=5 % ^^M
  \endlinechar=13 %
  \catcode123=1 % {
  \catcode125=2 % }
  \catcode64=11 % @
  \def\x{\endgroup
    \expandafter\edef\csname FibNum@AtEnd\endcsname{%
      \endlinechar=\the\endlinechar\relax
      \catcode13=\the\catcode13\relax
      \catcode32=\the\catcode32\relax
      \catcode35=\the\catcode35\relax
      \catcode61=\the\catcode61\relax
      \catcode64=\the\catcode64\relax
      \catcode123=\the\catcode123\relax
      \catcode125=\the\catcode125\relax
    }%
  }%
\x\catcode61\catcode48\catcode32=10\relax%
\catcode13=5 % ^^M
\endlinechar=13 %
\catcode35=6 % #
\catcode64=11 % @
\catcode123=1 % {
\catcode125=2 % }
\def\TMP@EnsureCode#1#2{%
  \edef\FibNum@AtEnd{%
    \FibNum@AtEnd
    \catcode#1=\the\catcode#1\relax
  }%
  \catcode#1=#2\relax
}
\TMP@EnsureCode{33}{12}% !
%\TMP@EnsureCode{36}{3}% $
%\TMP@EnsureCode{38}{4}% &
\TMP@EnsureCode{40}{12}% (
\TMP@EnsureCode{41}{12}% )
\TMP@EnsureCode{45}{12}% -
\TMP@EnsureCode{46}{12}% .
\TMP@EnsureCode{47}{12}% /
\TMP@EnsureCode{58}{12}% :
\TMP@EnsureCode{60}{12}% <
\TMP@EnsureCode{62}{12}% >
\TMP@EnsureCode{91}{12}% [
%\TMP@EnsureCode{96}{12}% `
\TMP@EnsureCode{93}{12}% ]
%\TMP@EnsureCode{94}{12}% ^ (superscript) (!)
%\TMP@EnsureCode{124}{12}% |
\edef\FibNum@AtEnd{\FibNum@AtEnd\noexpand\endinput}
%    \end{macrocode}
%
% \subsection{Package resources}
%
%    \begin{macrocode}
\begingroup\expandafter\expandafter\expandafter\endgroup
\expandafter\ifx\csname RequirePackage\endcsname\relax
  \def\TMP@RequirePackage#1[#2]{%
    \begingroup\expandafter\expandafter\expandafter\endgroup
    \expandafter\ifx\csname ver@#1.sty\endcsname\relax
      \input #1.sty\relax
    \fi
  }%
  \TMP@RequirePackage{ltxcmds}[2011/04/18]%
  \TMP@RequirePackage{intcalc}[2007/09/27]%
  \TMP@RequirePackage{bigintcalc}[2007/11/11]%
\else
  \RequirePackage{ltxcmds}[2011/04/18]%
  \RequirePackage{intcalc}[2007/09/27]%
  \RequirePackage{bigintcalc}[2007/11/11]%
\fi
%    \end{macrocode}
%
% \subsection{Setup precalculated values}
%
%    \begin{macrocode}
\def\FibNum@temp#1{%
  \expandafter\def\csname FibNum@#1\endcsname
}
\catcode46=9 % dots are ignored
\FibNum@temp{0}{0}
\FibNum@temp{1}{1}
\FibNum@temp{2}{1}
\FibNum@temp{3}{2}
\FibNum@temp{4}{3}
\FibNum@temp{5}{5}
\FibNum@temp{6}{8}
\FibNum@temp{7}{13}
\FibNum@temp{8}{21}
\FibNum@temp{9}{34}
\FibNum@temp{10}{55}
\FibNum@temp{11}{89}
\FibNum@temp{12}{144}
\FibNum@temp{13}{233}
\FibNum@temp{14}{377}
\FibNum@temp{15}{610}
\FibNum@temp{16}{987}
\FibNum@temp{17}{1.597}
\FibNum@temp{18}{2.584}
\FibNum@temp{19}{4.181}
\FibNum@temp{20}{6.765}
\FibNum@temp{21}{10.946}
\FibNum@temp{22}{17.711}
\FibNum@temp{23}{28.657}
\FibNum@temp{24}{46.368}
\FibNum@temp{25}{75.025}
\FibNum@temp{26}{121.393}
\FibNum@temp{27}{196.418}
\FibNum@temp{28}{317.811}
\FibNum@temp{29}{514.229}
\FibNum@temp{30}{832.040}
\FibNum@temp{31}{1.346.269}
\FibNum@temp{32}{2.178.309}
\FibNum@temp{33}{3.524.578}
\FibNum@temp{34}{5.702.887}
\FibNum@temp{35}{9.227.465}
\FibNum@temp{36}{14.930.352}
\FibNum@temp{37}{24.157.817}
\FibNum@temp{38}{39.088.169}
\FibNum@temp{39}{63.245.986}
\FibNum@temp{40}{102.334.155}
\FibNum@temp{41}{165.580.141}
\FibNum@temp{42}{267.914.296}
\FibNum@temp{43}{433.494.437}
\FibNum@temp{44}{701.408.733}
\FibNum@temp{45}{1.134.903.170}
\FibNum@temp{46}{1.836.311.903}
%    \end{macrocode}
%    \begin{macro}{\FibNum@max}
%    \begin{macrocode}
\def\FibNum@max{46}
%    \end{macrocode}
%    \end{macro}
%
% \subsection{Macros for precalculating values}
%
%    \begin{macro}{\fibnumPreCalc}
%    \begin{macrocode}
\def\fibnumPreCalc#1{%
  \expandafter\expandafter\expandafter
  \FibNum@PreCalc\intcalcNum{#1}/%
}
%    \end{macrocode}
%    \end{macro}
%    \begin{macro}{\FibNum@PreCalc}
%    \begin{macrocode}
\def\FibNum@PreCalc#1/{%
  \ifnum#1<\ltx@zero
    \expandafter\FibNum@PreCalc\ltx@gobble#1/%
  \else
    \ifnum#1>\FibNum@max
      \begingroup
        \ltx@LocDimenA=#1sp\relax
        \countdef\FibNum@i=255\relax
        \FibNum@i=\FibNum@max\relax
        \edef\FibNum@temp{%
          \csname FibNum@\the\FibNum@i\endcsname/%
        }%
        \advance\FibNum@i by -1\relax
        \edef\FibNum@temp{%
          \FibNum@temp
          \csname FibNum@\the\FibNum@i\endcsname
        }%
        \advance\FibNum@i\ltx@two
        \iftrue
          \expandafter\FibNum@PreCalcAux\FibNum@temp
        \fi
      \endgroup
    \fi
  \fi
}
%    \end{macrocode}
%    \end{macro}
%    \begin{macro}{\FibNum@PreCalcAux}
%    \begin{macrocode}
\def\FibNum@PreCalcAux#1/#2\fi{%
  \fi
  \edef\FibNum@temp{\BigIntCalcAdd#1!#2!}%
  \global\expandafter
  \let\csname FibNum@\the\FibNum@i\endcsname\FibNum@temp
  \ifnum\FibNum@i=\ltx@LocDimenA
    \xdef\FibNum@max{\the\FibNum@i}%
  \else
    \advance\FibNum@i\ltx@one
    \expandafter\FibNum@PreCalcAux\FibNum@temp/#1%
  \fi
}
%    \end{macrocode}
%    \end{macro}
%
% \subsection{Expandable calculations}
%
%    \begin{macro}{\fibnum}
%    \begin{macrocode}
\def\fibnum#1{%
  \romannumeral
  \expandafter\expandafter\expandafter\FibNum@Do\intcalcNum{#1}/%
}
%    \end{macrocode}
%    \end{macro}
%    \begin{macro}{\FibNum@Do}
%    \begin{macrocode}
\def\FibNum@Do#1/{%
  \ifnum#1<\ltx@zero
    \FibNum@ReturnAfterElseFiFi{%
      \ifodd#1 %
        \expandafter\expandafter\expandafter\ltx@zero
      \else
        \expandafter\expandafter\expandafter\ltx@zero
        \expandafter\expandafter\expandafter-%
      \fi
      \romannumeral
      \expandafter\FibNum@Do\ltx@gobble#1/%
    }%
  \else
    \ifnum\FibNum@max<#1 %
      \ltx@ReturnAfterElseFi{%
        \expandafter
        \FibNum@ExpCalc\number\expandafter\IntCalcInc\FibNum@max!%
        \expandafter\expandafter\expandafter/%
        \csname FibNum@\FibNum@max
        \expandafter\expandafter\expandafter\endcsname
        \expandafter\expandafter\expandafter/%
        \csname FibNum@\expandafter\IntCalcDec\FibNum@max!%
        \endcsname/%
        #1%
      }%
    \else
      \expandafter\expandafter\expandafter\ltx@zero
      \csname FibNum@#1\expandafter\expandafter\expandafter\endcsname
    \fi
  \fi
}
%    \end{macrocode}
%    \end{macro}
%    \begin{macro}{\FibNum@ReturnAfterElseFiFi}
%    \begin{macrocode}
\def\FibNum@ReturnAfterElseFiFi#1\else#2\fi\fi{\fi#1}
%    \end{macrocode}
%    \end{macro}
%    \begin{macro}{\FibNum@ExpCalc}
%    \begin{macrocode}
\def\FibNum@ExpCalc#1/#2/#3/#4\fi{%
  \fi
  \ifnum#1=#4 %
    \ltx@ReturnAfterElseFi{%
      \expandafter\expandafter\expandafter\ltx@zero
      \BigIntCalcAdd#2!#3!%
    }%
  \else
    \expandafter\FibNum@ExpCalc
    \number\IntCalcInc#1!%
    \expandafter\expandafter\expandafter/%
    \BigIntCalcAdd#2!#3!/%
    #2/#4%
  \fi
}
%    \end{macrocode}
%    \end{macro}
%
%    \begin{macrocode}
\FibNum@AtEnd%
%</package>
%    \end{macrocode}
%
% \section{Test}
%
% \subsection{Catcode checks for loading}
%
%    \begin{macrocode}
%<*test1>
%    \end{macrocode}
%    \begin{macrocode}
\catcode`\{=1 %
\catcode`\}=2 %
\catcode`\#=6 %
\catcode`\@=11 %
\expandafter\ifx\csname count@\endcsname\relax
  \countdef\count@=255 %
\fi
\expandafter\ifx\csname @gobble\endcsname\relax
  \long\def\@gobble#1{}%
\fi
\expandafter\ifx\csname @firstofone\endcsname\relax
  \long\def\@firstofone#1{#1}%
\fi
\expandafter\ifx\csname loop\endcsname\relax
  \expandafter\@firstofone
\else
  \expandafter\@gobble
\fi
{%
  \def\loop#1\repeat{%
    \def\body{#1}%
    \iterate
  }%
  \def\iterate{%
    \body
      \let\next\iterate
    \else
      \let\next\relax
    \fi
    \next
  }%
  \let\repeat=\fi
}%
\def\RestoreCatcodes{}
\count@=0 %
\loop
  \edef\RestoreCatcodes{%
    \RestoreCatcodes
    \catcode\the\count@=\the\catcode\count@\relax
  }%
\ifnum\count@<255 %
  \advance\count@ 1 %
\repeat

\def\RangeCatcodeInvalid#1#2{%
  \count@=#1\relax
  \loop
    \catcode\count@=15 %
  \ifnum\count@<#2\relax
    \advance\count@ 1 %
  \repeat
}
\def\RangeCatcodeCheck#1#2#3{%
  \count@=#1\relax
  \loop
    \ifnum#3=\catcode\count@
    \else
      \errmessage{%
        Character \the\count@\space
        with wrong catcode \the\catcode\count@\space
        instead of \number#3%
      }%
    \fi
  \ifnum\count@<#2\relax
    \advance\count@ 1 %
  \repeat
}
\def\space{ }
\expandafter\ifx\csname LoadCommand\endcsname\relax
  \def\LoadCommand{\input fibnum.sty\relax}%
\fi
\def\Test{%
  \RangeCatcodeInvalid{0}{47}%
  \RangeCatcodeInvalid{58}{64}%
  \RangeCatcodeInvalid{91}{96}%
  \RangeCatcodeInvalid{123}{255}%
  \catcode`\@=12 %
  \catcode`\\=0 %
  \catcode`\%=14 %
  \LoadCommand
  \RangeCatcodeCheck{0}{36}{15}%
  \RangeCatcodeCheck{37}{37}{14}%
  \RangeCatcodeCheck{38}{47}{15}%
  \RangeCatcodeCheck{48}{57}{12}%
  \RangeCatcodeCheck{58}{63}{15}%
  \RangeCatcodeCheck{64}{64}{12}%
  \RangeCatcodeCheck{65}{90}{11}%
  \RangeCatcodeCheck{91}{91}{15}%
  \RangeCatcodeCheck{92}{92}{0}%
  \RangeCatcodeCheck{93}{96}{15}%
  \RangeCatcodeCheck{97}{122}{11}%
  \RangeCatcodeCheck{123}{255}{15}%
  \RestoreCatcodes
}
\Test
\csname @@end\endcsname
\end
%    \end{macrocode}
%    \begin{macrocode}
%</test1>
%    \end{macrocode}
%
% \subsection{Test calculations}
%
%    \begin{macrocode}
%<*test-calc>
\catcode`\{=1 %
\catcode`\}=2 %
\catcode`\#=6 %
\catcode`\@=11 %
\begingroup\expandafter\expandafter\expandafter\endgroup
\expandafter\ifx\csname RequirePackage\endcsname\relax
  \input fibnum.sty\relax
\else
  \RequirePackage{fibnum}[2016/05/16]%
\fi
\def\TestSet{%
  \test{0}{0}%
  \test{1}{1}%
  \test{2}{1}%
  \test{3}{2}%
  \test{4}{3}%
  \test{5}{5}%
  \test{6}{8}%
  \test{7}{13}%
  \test{8}{21}%
  \test{9}{34}%
  \test{10}{55}%
  \test{11}{89}%
  \test{12}{144}%
  \test{13}{233}%
  \test{14}{377}%
  \test{15}{610}%
  \test{16}{987}%
  \test{17}{1597}%
  \test{18}{2584}%
  \test{19}{4181}%
  \test{20}{6765}%
  \test{21}{10946}%
  \test{22}{17711}%
  \test{23}{28657}%
  \test{24}{46368}%
  \test{25}{75025}%
  \test{26}{121393}%
  \test{27}{196418}%
  \test{28}{317811}%
  \test{29}{514229}%
  \test{30}{832040}%
  \test{31}{1346269}%
  \test{32}{2178309}%
  \test{33}{3524578}%
  \test{34}{5702887}%
  \test{35}{9227465}%
  \test{36}{14930352}%
  \test{37}{24157817}%
  \test{38}{39088169}%
  \test{39}{63245986}%
  \test{40}{102334155}%
  \test{41}{165580141}%
  \test{42}{267914296}%
  \test{43}{433494437}%
  \test{44}{701408733}%
  \test{45}{1134903170}%
  \test{46}{1836311903}%
  \test{47}{2971215073}%
  \test{48}{4807526976}%
  \test{49}{7778742049}%
  \test{50}{12586269025}%
  \test{51}{20365011074}%
  \test{52}{32951280099}%
  \test{53}{53316291173}%
  \test{54}{86267571272}%
  \test{55}{139583862445}%
  \test{56}{225851433717}%
  \test{57}{365435296162}%
  \test{58}{591286729879}%
  \test{59}{956722026041}%
  \test{60}{1548008755920}%
  \test{61}{2504730781961}%
  \test{62}{4052739537881}%
  \test{63}{6557470319842}%
  \test{64}{10610209857723}%
  \test{65}{17167680177565}%
  \test{66}{27777890035288}%
  \test{67}{44945570212853}%
  \test{68}{72723460248141}%
  \test{69}{117669030460994}%
  \test{70}{190392490709135}%
  \test{71}{308061521170129}%
  \test{72}{498454011879264}%
  \test{73}{806515533049393}%
}
\def\msg#{\immediate\write16}
\def\test#1#2{%
  \TestAux{#1}{#2}%
  \ifnum#1=0 %
  \else
    \ifodd#1 %
      \TestAux{-#1}{#2}%
    \else
      \TestAux{-#1}{-#2}%
    \fi
  \fi
}
\def\TestAux#1#2{%
  \def\Expected{#2}%
  \expandafter\expandafter\expandafter\def
  \expandafter\expandafter\expandafter\Result
  \expandafter\expandafter\expandafter{%
    \fibnum{#1}%
  }%
  \ltx@onelevel@sanitize\Result
  \ifx\Result\Expected
    \msg{* #1: ok.}%
  \else
    \msg{! fib(#1) = #2}%
    \errmessage{fib(#1) <> \Result}%
  \fi
}
\TestSet
\setbox0=\hbox{%
  \msg{* PreCalc{73}}%
  \fibnumPreCalc{73}%
}
\ifdim\wd0=0pt
\else
  \errmessage{Unwanted stuff in PreCalc}%
\fi
\TestSet
\csname @@end\endcsname\end
%</test-calc>
%    \end{macrocode}
%
% \section{Installation}
%
% \subsection{Download}
%
% \paragraph{Package.} This package is available on
% CTAN\footnote{\url{http://ctan.org/pkg/fibnum}}:
% \begin{description}
% \item[\CTAN{macros/latex/contrib/oberdiek/fibnum.dtx}] The source file.
% \item[\CTAN{macros/latex/contrib/oberdiek/fibnum.pdf}] Documentation.
% \end{description}
%
%
% \paragraph{Bundle.} All the packages of the bundle `oberdiek'
% are also available in a TDS compliant ZIP archive. There
% the packages are already unpacked and the documentation files
% are generated. The files and directories obey the TDS standard.
% \begin{description}
% \item[\CTAN{install/macros/latex/contrib/oberdiek.tds.zip}]
% \end{description}
% \emph{TDS} refers to the standard ``A Directory Structure
% for \TeX\ Files'' (\CTAN{tds/tds.pdf}). Directories
% with \xfile{texmf} in their name are usually organized this way.
%
% \subsection{Bundle installation}
%
% \paragraph{Unpacking.} Unpack the \xfile{oberdiek.tds.zip} in the
% TDS tree (also known as \xfile{texmf} tree) of your choice.
% Example (linux):
% \begin{quote}
%   |unzip oberdiek.tds.zip -d ~/texmf|
% \end{quote}
%
% \paragraph{Script installation.}
% Check the directory \xfile{TDS:scripts/oberdiek/} for
% scripts that need further installation steps.
% Package \xpackage{attachfile2} comes with the Perl script
% \xfile{pdfatfi.pl} that should be installed in such a way
% that it can be called as \texttt{pdfatfi}.
% Example (linux):
% \begin{quote}
%   |chmod +x scripts/oberdiek/pdfatfi.pl|\\
%   |cp scripts/oberdiek/pdfatfi.pl /usr/local/bin/|
% \end{quote}
%
% \subsection{Package installation}
%
% \paragraph{Unpacking.} The \xfile{.dtx} file is a self-extracting
% \docstrip\ archive. The files are extracted by running the
% \xfile{.dtx} through \plainTeX:
% \begin{quote}
%   \verb|tex fibnum.dtx|
% \end{quote}
%
% \paragraph{TDS.} Now the different files must be moved into
% the different directories in your installation TDS tree
% (also known as \xfile{texmf} tree):
% \begin{quote}
% \def\t{^^A
% \begin{tabular}{@{}>{\ttfamily}l@{ $\rightarrow$ }>{\ttfamily}l@{}}
%   fibnum.sty & tex/generic/oberdiek/fibnum.sty\\
%   fibnum.pdf & doc/latex/oberdiek/fibnum.pdf\\
%   test/fibnum-test1.tex & doc/latex/oberdiek/test/fibnum-test1.tex\\
%   test/fibnum-test-calc.tex & doc/latex/oberdiek/test/fibnum-test-calc.tex\\
%   fibnum.dtx & source/latex/oberdiek/fibnum.dtx\\
% \end{tabular}^^A
% }^^A
% \sbox0{\t}^^A
% \ifdim\wd0>\linewidth
%   \begingroup
%     \advance\linewidth by\leftmargin
%     \advance\linewidth by\rightmargin
%   \edef\x{\endgroup
%     \def\noexpand\lw{\the\linewidth}^^A
%   }\x
%   \def\lwbox{^^A
%     \leavevmode
%     \hbox to \linewidth{^^A
%       \kern-\leftmargin\relax
%       \hss
%       \usebox0
%       \hss
%       \kern-\rightmargin\relax
%     }^^A
%   }^^A
%   \ifdim\wd0>\lw
%     \sbox0{\small\t}^^A
%     \ifdim\wd0>\linewidth
%       \ifdim\wd0>\lw
%         \sbox0{\footnotesize\t}^^A
%         \ifdim\wd0>\linewidth
%           \ifdim\wd0>\lw
%             \sbox0{\scriptsize\t}^^A
%             \ifdim\wd0>\linewidth
%               \ifdim\wd0>\lw
%                 \sbox0{\tiny\t}^^A
%                 \ifdim\wd0>\linewidth
%                   \lwbox
%                 \else
%                   \usebox0
%                 \fi
%               \else
%                 \lwbox
%               \fi
%             \else
%               \usebox0
%             \fi
%           \else
%             \lwbox
%           \fi
%         \else
%           \usebox0
%         \fi
%       \else
%         \lwbox
%       \fi
%     \else
%       \usebox0
%     \fi
%   \else
%     \lwbox
%   \fi
% \else
%   \usebox0
% \fi
% \end{quote}
% If you have a \xfile{docstrip.cfg} that configures and enables \docstrip's
% TDS installing feature, then some files can already be in the right
% place, see the documentation of \docstrip.
%
% \subsection{Refresh file name databases}
%
% If your \TeX~distribution
% (\teTeX, \mikTeX, \dots) relies on file name databases, you must refresh
% these. For example, \teTeX\ users run \verb|texhash| or
% \verb|mktexlsr|.
%
% \subsection{Some details for the interested}
%
% \paragraph{Attached source.}
%
% The PDF documentation on CTAN also includes the
% \xfile{.dtx} source file. It can be extracted by
% AcrobatReader 6 or higher. Another option is \textsf{pdftk},
% e.g. unpack the file into the current directory:
% \begin{quote}
%   \verb|pdftk fibnum.pdf unpack_files output .|
% \end{quote}
%
% \paragraph{Unpacking with \LaTeX.}
% The \xfile{.dtx} chooses its action depending on the format:
% \begin{description}
% \item[\plainTeX:] Run \docstrip\ and extract the files.
% \item[\LaTeX:] Generate the documentation.
% \end{description}
% If you insist on using \LaTeX\ for \docstrip\ (really,
% \docstrip\ does not need \LaTeX), then inform the autodetect routine
% about your intention:
% \begin{quote}
%   \verb|latex \let\install=y% \iffalse meta-comment
%
% File: fibnum.dtx
% Version: 2016/05/16 v1.1
% Info: Fibonacci numbers
%
% Copyright (C) 2012 by
%    Heiko Oberdiek <heiko.oberdiek at googlemail.com>
%    2016
%    https://github.com/ho-tex/oberdiek/issues
%
% This work may be distributed and/or modified under the
% conditions of the LaTeX Project Public License, either
% version 1.3c of this license or (at your option) any later
% version. This version of this license is in
%    http://www.latex-project.org/lppl/lppl-1-3c.txt
% and the latest version of this license is in
%    http://www.latex-project.org/lppl.txt
% and version 1.3 or later is part of all distributions of
% LaTeX version 2005/12/01 or later.
%
% This work has the LPPL maintenance status "maintained".
%
% This Current Maintainer of this work is Heiko Oberdiek.
%
% The Base Interpreter refers to any `TeX-Format',
% because some files are installed in TDS:tex/generic//.
%
% This work consists of the main source file fibnum.dtx
% and the derived files
%    fibnum.sty, fibnum.pdf, fibnum.ins, fibnum.drv, fibnum.bib,
%    fibnum-test1.tex, fibnum-test-calc.tex.
%
% Distribution:
%    CTAN:macros/latex/contrib/oberdiek/fibnum.dtx
%    CTAN:macros/latex/contrib/oberdiek/fibnum.pdf
%
% Unpacking:
%    (a) If fibnum.ins is present:
%           tex fibnum.ins
%    (b) Without fibnum.ins:
%           tex fibnum.dtx
%    (c) If you insist on using LaTeX
%           latex \let\install=y% \iffalse meta-comment
%
% File: fibnum.dtx
% Version: 2016/05/16 v1.1
% Info: Fibonacci numbers
%
% Copyright (C) 2012 by
%    Heiko Oberdiek <heiko.oberdiek at googlemail.com>
%    2016
%    https://github.com/ho-tex/oberdiek/issues
%
% This work may be distributed and/or modified under the
% conditions of the LaTeX Project Public License, either
% version 1.3c of this license or (at your option) any later
% version. This version of this license is in
%    http://www.latex-project.org/lppl/lppl-1-3c.txt
% and the latest version of this license is in
%    http://www.latex-project.org/lppl.txt
% and version 1.3 or later is part of all distributions of
% LaTeX version 2005/12/01 or later.
%
% This work has the LPPL maintenance status "maintained".
%
% This Current Maintainer of this work is Heiko Oberdiek.
%
% The Base Interpreter refers to any `TeX-Format',
% because some files are installed in TDS:tex/generic//.
%
% This work consists of the main source file fibnum.dtx
% and the derived files
%    fibnum.sty, fibnum.pdf, fibnum.ins, fibnum.drv, fibnum.bib,
%    fibnum-test1.tex, fibnum-test-calc.tex.
%
% Distribution:
%    CTAN:macros/latex/contrib/oberdiek/fibnum.dtx
%    CTAN:macros/latex/contrib/oberdiek/fibnum.pdf
%
% Unpacking:
%    (a) If fibnum.ins is present:
%           tex fibnum.ins
%    (b) Without fibnum.ins:
%           tex fibnum.dtx
%    (c) If you insist on using LaTeX
%           latex \let\install=y\input{fibnum.dtx}
%        (quote the arguments according to the demands of your shell)
%
% Documentation:
%    (a) If fibnum.drv is present:
%           latex fibnum.drv
%    (b) Without fibnum.drv:
%           latex fibnum.dtx; ...
%    The class ltxdoc loads the configuration file ltxdoc.cfg
%    if available. Here you can specify further options, e.g.
%    use A4 as paper format:
%       \PassOptionsToClass{a4paper}{article}
%
%    Programm calls to get the documentation (example):
%       pdflatex fibnum.dtx
%       bibtex fibnum.aux
%       makeindex -s gind.ist fibnum.idx
%       pdflatex fibnum.dtx
%       makeindex -s gind.ist fibnum.idx
%       pdflatex fibnum.dtx
%
% Installation:
%    TDS:tex/generic/oberdiek/fibnum.sty
%    TDS:doc/latex/oberdiek/fibnum.pdf
%    TDS:doc/latex/oberdiek/test/fibnum-test1.tex
%    TDS:doc/latex/oberdiek/test/fibnum-test-calc.tex
%    TDS:source/latex/oberdiek/fibnum.dtx
%
%<*ignore>
\begingroup
  \catcode123=1 %
  \catcode125=2 %
  \def\x{LaTeX2e}%
\expandafter\endgroup
\ifcase 0\ifx\install y1\fi\expandafter
         \ifx\csname processbatchFile\endcsname\relax\else1\fi
         \ifx\fmtname\x\else 1\fi\relax
\else\csname fi\endcsname
%</ignore>
%<*install>
\input docstrip.tex
\Msg{************************************************************************}
\Msg{* Installation}
\Msg{* Package: fibnum 2016/05/16 v1.1 Fibonacci numbers (HO)}
\Msg{************************************************************************}

\keepsilent
\askforoverwritefalse

\let\MetaPrefix\relax
\preamble

This is a generated file.

Project: fibnum
Version: 2016/05/16 v1.1

Copyright (C) 2012 by
   Heiko Oberdiek <heiko.oberdiek at googlemail.com>

This work may be distributed and/or modified under the
conditions of the LaTeX Project Public License, either
version 1.3c of this license or (at your option) any later
version. This version of this license is in
   http://www.latex-project.org/lppl/lppl-1-3c.txt
and the latest version of this license is in
   http://www.latex-project.org/lppl.txt
and version 1.3 or later is part of all distributions of
LaTeX version 2005/12/01 or later.

This work has the LPPL maintenance status "maintained".

This Current Maintainer of this work is Heiko Oberdiek.

The Base Interpreter refers to any `TeX-Format',
because some files are installed in TDS:tex/generic//.

This work consists of the main source file fibnum.dtx
and the derived files
   fibnum.sty, fibnum.pdf, fibnum.ins, fibnum.drv, fibnum.bib,
   fibnum-test1.tex, fibnum-test-calc.tex.

\endpreamble
\let\MetaPrefix\DoubleperCent

\generate{%
  \file{fibnum.ins}{\from{fibnum.dtx}{install}}%
  \file{fibnum.drv}{\from{fibnum.dtx}{driver}}%
  \nopreamble
  \nopostamble
  \file{fibnum.bib}{\from{fibnum.dtx}{bib}}%
  \usepreamble\defaultpreamble
  \usepostamble\defaultpostamble
  \usedir{tex/generic/oberdiek}%
  \file{fibnum.sty}{\from{fibnum.dtx}{package}}%
  \usedir{doc/latex/oberdiek/test}%
  \file{fibnum-test1.tex}{\from{fibnum.dtx}{test1}}%
  \file{fibnum-test-calc.tex}{\from{fibnum.dtx}{test-calc}}%
}

\catcode32=13\relax% active space
\let =\space%
\Msg{************************************************************************}
\Msg{*}
\Msg{* To finish the installation you have to move the following}
\Msg{* file into a directory searched by TeX:}
\Msg{*}
\Msg{*     fibnum.sty}
\Msg{*}
\Msg{* To produce the documentation run the file `fibnum.drv'}
\Msg{* through LaTeX.}
\Msg{*}
\Msg{* Happy TeXing!}
\Msg{*}
\Msg{************************************************************************}

\endbatchfile
%</install>
%<*bib>
@online{texhax:abraham,
  author={Abraham, Jan},
  title={[texhax] Beginner in TEX MACRO to compute functions},
  date={2012-04-07},
  url={http://tug.org/pipermail/texhax/2012-April/019146.html},
  urldate={2012-04-08},
}
@article{knuth:negafibonacci,
  author={Knuth, Donald E.},
  title={Negafibonacci Numbers and the Hyperbolic Plane},
  date={2008-12-11},
  url={http://research.allacademic.com/meta/p206842_index.html},
}
@online{wikipedia:negafibonacci,
  author={{Wikipedia contributors}},
  organization={{Wikipedia, The Free Encyclopedia}},
  title={Fibonacci numbers},
  language={langenglish},
  version={486266088},
  date={2012-04-08},
  url={http://en.wikipedia.org/w/index.php?title=Fibonacci_number&oldid=486266088},
  urldate={2012-04-08},
}
%</bib>
%<*ignore>
\fi
%</ignore>
%<*driver>
\NeedsTeXFormat{LaTeX2e}
\ProvidesFile{fibnum.drv}%
  [2016/05/16 v1.1 Fibonacci numbers (HO)]%
\documentclass{ltxdoc}
\usepackage{amsmath,amsfonts}
\usepackage{siunitx}
\usepackage{array}
\usepackage{tabularx}
\usepackage{fibnum}[2016/05/16]
\usepackage{holtxdoc}[2011/11/22]
\usepackage{csquotes}
\usepackage[
  bibencoding=ascii,
  alldates=iso8601,
]{biblatex}[2011/11/13]
\bibliography{oberdiek-source}
\bibliography{fibnum}
\begin{document}
  \DocInput{fibnum.dtx}%
\end{document}
%</driver>
% \fi
%
%
% \CharacterTable
%  {Upper-case    \A\B\C\D\E\F\G\H\I\J\K\L\M\N\O\P\Q\R\S\T\U\V\W\X\Y\Z
%   Lower-case    \a\b\c\d\e\f\g\h\i\j\k\l\m\n\o\p\q\r\s\t\u\v\w\x\y\z
%   Digits        \0\1\2\3\4\5\6\7\8\9
%   Exclamation   \!     Double quote  \"     Hash (number) \#
%   Dollar        \$     Percent       \%     Ampersand     \&
%   Acute accent  \'     Left paren    \(     Right paren   \)
%   Asterisk      \*     Plus          \+     Comma         \,
%   Minus         \-     Point         \.     Solidus       \/
%   Colon         \:     Semicolon     \;     Less than     \<
%   Equals        \=     Greater than  \>     Question mark \?
%   Commercial at \@     Left bracket  \[     Backslash     \\
%   Right bracket \]     Circumflex    \^     Underscore    \_
%   Grave accent  \`     Left brace    \{     Vertical bar  \|
%   Right brace   \}     Tilde         \~}
%
% \GetFileInfo{fibnum.drv}
%
% \title{The \xpackage{fibnum} package}
% \date{2016/05/16 v1.1}
% \author{Heiko Oberdiek\thanks
% {Please report any issues at https://github.com/ho-tex/oberdiek/issues}\\
% \xemail{heiko.oberdiek at googlemail.com}}
%
% \maketitle
%
% \begin{abstract}
% The package \xpackage{fibnum} provides expandable fibonacci
% numbers for both \hologo{LaTeX} and \hologo{plainTeX}.
% \end{abstract}
%
% \tableofcontents
%
% \section{Documentation}
%
% In the mailing list \textsf{texhax} Jan Abraham asked the question,
% how to get Fibonacci numbers in \hologo{TeX} \cite{texhax:abraham}:
% \begin{quote}
% Write a Macro in \hologo{TeX} that compute the function |\fib{m}|
% All fibonacci numbers from 1 to $m$ ($m < 40$).
% \end{quote}
% This packages provides an expandable implementation for the
% calculation of these numbers for a much larger set of indexes.
% For practical reasons the index is restricted to the same limitations
% that apply for \hologo{TeX} integer numbers.
% The range of the Fibonacci numbers, however, are not limited
% by the algorithm. They are only restricted to memory limitations,
% if they are hit.
%
% The package is loaded as \hologo{LaTeX} package in \hologo{LaTeX}:
% \begin{quote}
%   |\usepackage{fibnum}|
% \end{quote}
% and as file in \hologo{plainTeX}:
% \begin{quote}
%   |\input fibnum.sty|
% \end{quote}
% The package does not know any options and it provides
% the macros \cs{fibnum} and \cs{fibnumPreCalc}.
%
% \begin{declcs}{fibnum} \M{index}
% \end{declcs}
% Macro \cs{fibnum} expects a \hologo{TeX} number as \meta{index}
% in the official \hologo{TeX} number range from $-(2^{31}-1)$ up to
% $2^{31}-1$. In exact two expansion steps the macro expands to
% the Fibnoacci number $F_{\text{\meta{index}}}$. In case of a negative
% \meta{index}, the ``negafibonacci'' number \cite{wikipedia:negafibonacci}
% is used. Formally the Fibonacci number $F_n$ with integer
% index~$n$, $n\in\mathbb{Z}$ and
% $n\in[\num{-2147483647},\num{2147483647}]$ that is returned by macro
% \cs{fibnum} with numerical argument $n$ is defined the following way:
% \begin{gather}
%   \label{eq:def}
%   F_n =
%   \begin{cases}
%     0 & \text{for $n=0$}\\
%     1 & \text{for $n=1$}\\
%     F_{n-1} + F_{n-2} & \text{for $n>1$}\\
%     (-1)^{n+1}F_n & \text{for $n<0$}
%   \end{cases}
% \end{gather}
% Examples:
% \begin{quote}
%   \makeatletter
%   \def\x#1{\cs{fibnum}\{#1\}&
%     \edef\X{\fibnum{#1}}\edef\Y{\expandafter\ltx@car\X\@nil}^^A
%     \if-\Y
%       \edef\X{\expandafter\ltx@cdr\X\@nil}^^A
%       \noindent
%       \llap{-}\X
%     \else
%       \X
%     \fi
%     \tabularnewline
%   }
%   \def\y{\multicolumn{1}{@{}c@{}}{$\vdots$}\tabularnewline}
%   \DeclareUrlCommand\UrlNum{^^A
%     \urlstyle{tt}^^A
%     \def\UrlBreaks{\do\0\do\1\do\2\do\3\do\4\do\5\do\6\do\7\do\8\do\9}^^A
%   }
%   \begin{tabularx}{\dimexpr\linewidth+5.7pt\relax}{@{}>{\ttfamily}l@{ $\rightarrow$ \hphantom{\ttfamily-}}>{\ttfamily}X@{}}
%     \x{-6}
%     \x{-5}
%     \x{-4}
%     \x{-3}
%     \x{-2}
%     \x{-1}
%     \x{0}
%     \x{1}
%     \x{2}
%     \x{3}
%     \x{4}
%     \x{5}
%     \x{6}
%     \y
%     \x{10}
%     \y
%     \x{46}
%     \y
%     \cs{fibnum}\{100\} & 354224848179261915075
%     \tabularnewline
%     \y
%     \cs{fibnum}\{200\} & 280571172992510140037611932413038677189525
%     \tabularnewline
%     \y
%     \cs{fibnum}\{1000\} &
%       \raggedright
%       \UrlNum{^^A
%         434665576869374564356885276750406258025646^^A
%         605173717804024817290895365554179490518904^^A
%         038798400792551692959225930803226347752096^^A
%         896232398733224711616429964409065331879382^^A
%         98969649928516003704476137795166849228875^^A
%       }
%     \tabularnewline
%   \end{tabularx}\kern-5.7pt\mbox{}
% \end{quote}
%
% \begin{declcs}{fibnumPreCalc} \M{index}
% \end{declcs}
% The package already provides precalculated Fibonacci numbers up to
% index~46. That means that calculations are not necessary for
% Fibonacci numbers that fit into the range of \hologo{TeX}
% numbers. Because macro \cs{fibnum} is expandable, it cannot
% store calculated Fibonacci numbers for later use. Macro definitions
% are forbidden in expandable contexts. If larger Fibonacci numbers
% are used more than once, than the compilation time can be shortened
% by calculating and storing the Fibonacci numbers beforehand.
% The argument \meta{index} is a \hologo{TeX} number and macro
% \cs{fibnumPreCalc} ensures that the Fibonacci numbers
% $F_0$ up to $F_{\lvert\text{\meta{index}}\rvert}$ that are not
% already known are calculated
% and stored in internal macros. Internally only non-negative
% Fibonacci numbers are stored. If \meta{index} is negative, then
% the needed positive Fibonacci numbers are calculated and stored.
% Example:
% \begin{quote}
%   \def\x#1{\begingroup\itshape\texttt{\%} #1\endgroup}
%   |\fibnumPreCalc{50}|\\
%   \x{calculates and stores the values for indexes 47..50.}\\
%   \x{(Values for 0..46 are already stored by the package.)}\\
%   |\fibnum{49}| \x{uses the stored value}\\
%   |\fibnum{51}|
%   \x{only calculates $F_{51}$ from stored values $F_{49}$ and $F_{50}$}\\
%   |\fibnumPreCalc{100}|\\
%   \x{calculates and stores the values for indexes 51..100}\\
%   |\fibnum{100}| \x{uses the stored value for $F_{100}$}\\
%   |\fibnum{-100}|
%   \x{uses the stored value for $F_{100}$}\\
%   \x{$F_{-100}=-F_{100}$ according to equation \eqref{eq:def}.}
% \end{quote}
%
% \StopEventually{
% }
%
% \section{Implementation}
%
% \subsection{Identification}
%
%    \begin{macrocode}
%<*package>
%    \end{macrocode}
%    Reload check, especially if the package is not used with \LaTeX.
%    \begin{macrocode}
\begingroup\catcode61\catcode48\catcode32=10\relax%
  \catcode13=5 % ^^M
  \endlinechar=13 %
  \catcode35=6 % #
  \catcode39=12 % '
  \catcode44=12 % ,
  \catcode45=12 % -
  \catcode46=12 % .
  \catcode58=12 % :
  \catcode64=11 % @
  \catcode123=1 % {
  \catcode125=2 % }
  \expandafter\let\expandafter\x\csname ver@fibnum.sty\endcsname
  \ifx\x\relax % plain-TeX, first loading
  \else
    \def\empty{}%
    \ifx\x\empty % LaTeX, first loading,
      % variable is initialized, but \ProvidesPackage not yet seen
    \else
      \expandafter\ifx\csname PackageInfo\endcsname\relax
        \def\x#1#2{%
          \immediate\write-1{Package #1 Info: #2.}%
        }%
      \else
        \def\x#1#2{\PackageInfo{#1}{#2, stopped}}%
      \fi
      \x{fibnum}{The package is already loaded}%
      \aftergroup\endinput
    \fi
  \fi
\endgroup%
%    \end{macrocode}
%    Package identification:
%    \begin{macrocode}
\begingroup\catcode61\catcode48\catcode32=10\relax%
  \catcode13=5 % ^^M
  \endlinechar=13 %
  \catcode35=6 % #
  \catcode39=12 % '
  \catcode40=12 % (
  \catcode41=12 % )
  \catcode44=12 % ,
  \catcode45=12 % -
  \catcode46=12 % .
  \catcode47=12 % /
  \catcode58=12 % :
  \catcode64=11 % @
  \catcode91=12 % [
  \catcode93=12 % ]
  \catcode123=1 % {
  \catcode125=2 % }
  \expandafter\ifx\csname ProvidesPackage\endcsname\relax
    \def\x#1#2#3[#4]{\endgroup
      \immediate\write-1{Package: #3 #4}%
      \xdef#1{#4}%
    }%
  \else
    \def\x#1#2[#3]{\endgroup
      #2[{#3}]%
      \ifx#1\@undefined
        \xdef#1{#3}%
      \fi
      \ifx#1\relax
        \xdef#1{#3}%
      \fi
    }%
  \fi
\expandafter\x\csname ver@fibnum.sty\endcsname
\ProvidesPackage{fibnum}%
  [2016/05/16 v1.1 Fibonacci numbers (HO)]%
%    \end{macrocode}
%
%    \begin{macrocode}
\begingroup\catcode61\catcode48\catcode32=10\relax%
  \catcode13=5 % ^^M
  \endlinechar=13 %
  \catcode123=1 % {
  \catcode125=2 % }
  \catcode64=11 % @
  \def\x{\endgroup
    \expandafter\edef\csname FibNum@AtEnd\endcsname{%
      \endlinechar=\the\endlinechar\relax
      \catcode13=\the\catcode13\relax
      \catcode32=\the\catcode32\relax
      \catcode35=\the\catcode35\relax
      \catcode61=\the\catcode61\relax
      \catcode64=\the\catcode64\relax
      \catcode123=\the\catcode123\relax
      \catcode125=\the\catcode125\relax
    }%
  }%
\x\catcode61\catcode48\catcode32=10\relax%
\catcode13=5 % ^^M
\endlinechar=13 %
\catcode35=6 % #
\catcode64=11 % @
\catcode123=1 % {
\catcode125=2 % }
\def\TMP@EnsureCode#1#2{%
  \edef\FibNum@AtEnd{%
    \FibNum@AtEnd
    \catcode#1=\the\catcode#1\relax
  }%
  \catcode#1=#2\relax
}
\TMP@EnsureCode{33}{12}% !
%\TMP@EnsureCode{36}{3}% $
%\TMP@EnsureCode{38}{4}% &
\TMP@EnsureCode{40}{12}% (
\TMP@EnsureCode{41}{12}% )
\TMP@EnsureCode{45}{12}% -
\TMP@EnsureCode{46}{12}% .
\TMP@EnsureCode{47}{12}% /
\TMP@EnsureCode{58}{12}% :
\TMP@EnsureCode{60}{12}% <
\TMP@EnsureCode{62}{12}% >
\TMP@EnsureCode{91}{12}% [
%\TMP@EnsureCode{96}{12}% `
\TMP@EnsureCode{93}{12}% ]
%\TMP@EnsureCode{94}{12}% ^ (superscript) (!)
%\TMP@EnsureCode{124}{12}% |
\edef\FibNum@AtEnd{\FibNum@AtEnd\noexpand\endinput}
%    \end{macrocode}
%
% \subsection{Package resources}
%
%    \begin{macrocode}
\begingroup\expandafter\expandafter\expandafter\endgroup
\expandafter\ifx\csname RequirePackage\endcsname\relax
  \def\TMP@RequirePackage#1[#2]{%
    \begingroup\expandafter\expandafter\expandafter\endgroup
    \expandafter\ifx\csname ver@#1.sty\endcsname\relax
      \input #1.sty\relax
    \fi
  }%
  \TMP@RequirePackage{ltxcmds}[2011/04/18]%
  \TMP@RequirePackage{intcalc}[2007/09/27]%
  \TMP@RequirePackage{bigintcalc}[2007/11/11]%
\else
  \RequirePackage{ltxcmds}[2011/04/18]%
  \RequirePackage{intcalc}[2007/09/27]%
  \RequirePackage{bigintcalc}[2007/11/11]%
\fi
%    \end{macrocode}
%
% \subsection{Setup precalculated values}
%
%    \begin{macrocode}
\def\FibNum@temp#1{%
  \expandafter\def\csname FibNum@#1\endcsname
}
\catcode46=9 % dots are ignored
\FibNum@temp{0}{0}
\FibNum@temp{1}{1}
\FibNum@temp{2}{1}
\FibNum@temp{3}{2}
\FibNum@temp{4}{3}
\FibNum@temp{5}{5}
\FibNum@temp{6}{8}
\FibNum@temp{7}{13}
\FibNum@temp{8}{21}
\FibNum@temp{9}{34}
\FibNum@temp{10}{55}
\FibNum@temp{11}{89}
\FibNum@temp{12}{144}
\FibNum@temp{13}{233}
\FibNum@temp{14}{377}
\FibNum@temp{15}{610}
\FibNum@temp{16}{987}
\FibNum@temp{17}{1.597}
\FibNum@temp{18}{2.584}
\FibNum@temp{19}{4.181}
\FibNum@temp{20}{6.765}
\FibNum@temp{21}{10.946}
\FibNum@temp{22}{17.711}
\FibNum@temp{23}{28.657}
\FibNum@temp{24}{46.368}
\FibNum@temp{25}{75.025}
\FibNum@temp{26}{121.393}
\FibNum@temp{27}{196.418}
\FibNum@temp{28}{317.811}
\FibNum@temp{29}{514.229}
\FibNum@temp{30}{832.040}
\FibNum@temp{31}{1.346.269}
\FibNum@temp{32}{2.178.309}
\FibNum@temp{33}{3.524.578}
\FibNum@temp{34}{5.702.887}
\FibNum@temp{35}{9.227.465}
\FibNum@temp{36}{14.930.352}
\FibNum@temp{37}{24.157.817}
\FibNum@temp{38}{39.088.169}
\FibNum@temp{39}{63.245.986}
\FibNum@temp{40}{102.334.155}
\FibNum@temp{41}{165.580.141}
\FibNum@temp{42}{267.914.296}
\FibNum@temp{43}{433.494.437}
\FibNum@temp{44}{701.408.733}
\FibNum@temp{45}{1.134.903.170}
\FibNum@temp{46}{1.836.311.903}
%    \end{macrocode}
%    \begin{macro}{\FibNum@max}
%    \begin{macrocode}
\def\FibNum@max{46}
%    \end{macrocode}
%    \end{macro}
%
% \subsection{Macros for precalculating values}
%
%    \begin{macro}{\fibnumPreCalc}
%    \begin{macrocode}
\def\fibnumPreCalc#1{%
  \expandafter\expandafter\expandafter
  \FibNum@PreCalc\intcalcNum{#1}/%
}
%    \end{macrocode}
%    \end{macro}
%    \begin{macro}{\FibNum@PreCalc}
%    \begin{macrocode}
\def\FibNum@PreCalc#1/{%
  \ifnum#1<\ltx@zero
    \expandafter\FibNum@PreCalc\ltx@gobble#1/%
  \else
    \ifnum#1>\FibNum@max
      \begingroup
        \ltx@LocDimenA=#1sp\relax
        \countdef\FibNum@i=255\relax
        \FibNum@i=\FibNum@max\relax
        \edef\FibNum@temp{%
          \csname FibNum@\the\FibNum@i\endcsname/%
        }%
        \advance\FibNum@i by -1\relax
        \edef\FibNum@temp{%
          \FibNum@temp
          \csname FibNum@\the\FibNum@i\endcsname
        }%
        \advance\FibNum@i\ltx@two
        \iftrue
          \expandafter\FibNum@PreCalcAux\FibNum@temp
        \fi
      \endgroup
    \fi
  \fi
}
%    \end{macrocode}
%    \end{macro}
%    \begin{macro}{\FibNum@PreCalcAux}
%    \begin{macrocode}
\def\FibNum@PreCalcAux#1/#2\fi{%
  \fi
  \edef\FibNum@temp{\BigIntCalcAdd#1!#2!}%
  \global\expandafter
  \let\csname FibNum@\the\FibNum@i\endcsname\FibNum@temp
  \ifnum\FibNum@i=\ltx@LocDimenA
    \xdef\FibNum@max{\the\FibNum@i}%
  \else
    \advance\FibNum@i\ltx@one
    \expandafter\FibNum@PreCalcAux\FibNum@temp/#1%
  \fi
}
%    \end{macrocode}
%    \end{macro}
%
% \subsection{Expandable calculations}
%
%    \begin{macro}{\fibnum}
%    \begin{macrocode}
\def\fibnum#1{%
  \romannumeral
  \expandafter\expandafter\expandafter\FibNum@Do\intcalcNum{#1}/%
}
%    \end{macrocode}
%    \end{macro}
%    \begin{macro}{\FibNum@Do}
%    \begin{macrocode}
\def\FibNum@Do#1/{%
  \ifnum#1<\ltx@zero
    \FibNum@ReturnAfterElseFiFi{%
      \ifodd#1 %
        \expandafter\expandafter\expandafter\ltx@zero
      \else
        \expandafter\expandafter\expandafter\ltx@zero
        \expandafter\expandafter\expandafter-%
      \fi
      \romannumeral
      \expandafter\FibNum@Do\ltx@gobble#1/%
    }%
  \else
    \ifnum\FibNum@max<#1 %
      \ltx@ReturnAfterElseFi{%
        \expandafter
        \FibNum@ExpCalc\number\expandafter\IntCalcInc\FibNum@max!%
        \expandafter\expandafter\expandafter/%
        \csname FibNum@\FibNum@max
        \expandafter\expandafter\expandafter\endcsname
        \expandafter\expandafter\expandafter/%
        \csname FibNum@\expandafter\IntCalcDec\FibNum@max!%
        \endcsname/%
        #1%
      }%
    \else
      \expandafter\expandafter\expandafter\ltx@zero
      \csname FibNum@#1\expandafter\expandafter\expandafter\endcsname
    \fi
  \fi
}
%    \end{macrocode}
%    \end{macro}
%    \begin{macro}{\FibNum@ReturnAfterElseFiFi}
%    \begin{macrocode}
\def\FibNum@ReturnAfterElseFiFi#1\else#2\fi\fi{\fi#1}
%    \end{macrocode}
%    \end{macro}
%    \begin{macro}{\FibNum@ExpCalc}
%    \begin{macrocode}
\def\FibNum@ExpCalc#1/#2/#3/#4\fi{%
  \fi
  \ifnum#1=#4 %
    \ltx@ReturnAfterElseFi{%
      \expandafter\expandafter\expandafter\ltx@zero
      \BigIntCalcAdd#2!#3!%
    }%
  \else
    \expandafter\FibNum@ExpCalc
    \number\IntCalcInc#1!%
    \expandafter\expandafter\expandafter/%
    \BigIntCalcAdd#2!#3!/%
    #2/#4%
  \fi
}
%    \end{macrocode}
%    \end{macro}
%
%    \begin{macrocode}
\FibNum@AtEnd%
%</package>
%    \end{macrocode}
%
% \section{Test}
%
% \subsection{Catcode checks for loading}
%
%    \begin{macrocode}
%<*test1>
%    \end{macrocode}
%    \begin{macrocode}
\catcode`\{=1 %
\catcode`\}=2 %
\catcode`\#=6 %
\catcode`\@=11 %
\expandafter\ifx\csname count@\endcsname\relax
  \countdef\count@=255 %
\fi
\expandafter\ifx\csname @gobble\endcsname\relax
  \long\def\@gobble#1{}%
\fi
\expandafter\ifx\csname @firstofone\endcsname\relax
  \long\def\@firstofone#1{#1}%
\fi
\expandafter\ifx\csname loop\endcsname\relax
  \expandafter\@firstofone
\else
  \expandafter\@gobble
\fi
{%
  \def\loop#1\repeat{%
    \def\body{#1}%
    \iterate
  }%
  \def\iterate{%
    \body
      \let\next\iterate
    \else
      \let\next\relax
    \fi
    \next
  }%
  \let\repeat=\fi
}%
\def\RestoreCatcodes{}
\count@=0 %
\loop
  \edef\RestoreCatcodes{%
    \RestoreCatcodes
    \catcode\the\count@=\the\catcode\count@\relax
  }%
\ifnum\count@<255 %
  \advance\count@ 1 %
\repeat

\def\RangeCatcodeInvalid#1#2{%
  \count@=#1\relax
  \loop
    \catcode\count@=15 %
  \ifnum\count@<#2\relax
    \advance\count@ 1 %
  \repeat
}
\def\RangeCatcodeCheck#1#2#3{%
  \count@=#1\relax
  \loop
    \ifnum#3=\catcode\count@
    \else
      \errmessage{%
        Character \the\count@\space
        with wrong catcode \the\catcode\count@\space
        instead of \number#3%
      }%
    \fi
  \ifnum\count@<#2\relax
    \advance\count@ 1 %
  \repeat
}
\def\space{ }
\expandafter\ifx\csname LoadCommand\endcsname\relax
  \def\LoadCommand{\input fibnum.sty\relax}%
\fi
\def\Test{%
  \RangeCatcodeInvalid{0}{47}%
  \RangeCatcodeInvalid{58}{64}%
  \RangeCatcodeInvalid{91}{96}%
  \RangeCatcodeInvalid{123}{255}%
  \catcode`\@=12 %
  \catcode`\\=0 %
  \catcode`\%=14 %
  \LoadCommand
  \RangeCatcodeCheck{0}{36}{15}%
  \RangeCatcodeCheck{37}{37}{14}%
  \RangeCatcodeCheck{38}{47}{15}%
  \RangeCatcodeCheck{48}{57}{12}%
  \RangeCatcodeCheck{58}{63}{15}%
  \RangeCatcodeCheck{64}{64}{12}%
  \RangeCatcodeCheck{65}{90}{11}%
  \RangeCatcodeCheck{91}{91}{15}%
  \RangeCatcodeCheck{92}{92}{0}%
  \RangeCatcodeCheck{93}{96}{15}%
  \RangeCatcodeCheck{97}{122}{11}%
  \RangeCatcodeCheck{123}{255}{15}%
  \RestoreCatcodes
}
\Test
\csname @@end\endcsname
\end
%    \end{macrocode}
%    \begin{macrocode}
%</test1>
%    \end{macrocode}
%
% \subsection{Test calculations}
%
%    \begin{macrocode}
%<*test-calc>
\catcode`\{=1 %
\catcode`\}=2 %
\catcode`\#=6 %
\catcode`\@=11 %
\begingroup\expandafter\expandafter\expandafter\endgroup
\expandafter\ifx\csname RequirePackage\endcsname\relax
  \input fibnum.sty\relax
\else
  \RequirePackage{fibnum}[2016/05/16]%
\fi
\def\TestSet{%
  \test{0}{0}%
  \test{1}{1}%
  \test{2}{1}%
  \test{3}{2}%
  \test{4}{3}%
  \test{5}{5}%
  \test{6}{8}%
  \test{7}{13}%
  \test{8}{21}%
  \test{9}{34}%
  \test{10}{55}%
  \test{11}{89}%
  \test{12}{144}%
  \test{13}{233}%
  \test{14}{377}%
  \test{15}{610}%
  \test{16}{987}%
  \test{17}{1597}%
  \test{18}{2584}%
  \test{19}{4181}%
  \test{20}{6765}%
  \test{21}{10946}%
  \test{22}{17711}%
  \test{23}{28657}%
  \test{24}{46368}%
  \test{25}{75025}%
  \test{26}{121393}%
  \test{27}{196418}%
  \test{28}{317811}%
  \test{29}{514229}%
  \test{30}{832040}%
  \test{31}{1346269}%
  \test{32}{2178309}%
  \test{33}{3524578}%
  \test{34}{5702887}%
  \test{35}{9227465}%
  \test{36}{14930352}%
  \test{37}{24157817}%
  \test{38}{39088169}%
  \test{39}{63245986}%
  \test{40}{102334155}%
  \test{41}{165580141}%
  \test{42}{267914296}%
  \test{43}{433494437}%
  \test{44}{701408733}%
  \test{45}{1134903170}%
  \test{46}{1836311903}%
  \test{47}{2971215073}%
  \test{48}{4807526976}%
  \test{49}{7778742049}%
  \test{50}{12586269025}%
  \test{51}{20365011074}%
  \test{52}{32951280099}%
  \test{53}{53316291173}%
  \test{54}{86267571272}%
  \test{55}{139583862445}%
  \test{56}{225851433717}%
  \test{57}{365435296162}%
  \test{58}{591286729879}%
  \test{59}{956722026041}%
  \test{60}{1548008755920}%
  \test{61}{2504730781961}%
  \test{62}{4052739537881}%
  \test{63}{6557470319842}%
  \test{64}{10610209857723}%
  \test{65}{17167680177565}%
  \test{66}{27777890035288}%
  \test{67}{44945570212853}%
  \test{68}{72723460248141}%
  \test{69}{117669030460994}%
  \test{70}{190392490709135}%
  \test{71}{308061521170129}%
  \test{72}{498454011879264}%
  \test{73}{806515533049393}%
}
\def\msg#{\immediate\write16}
\def\test#1#2{%
  \TestAux{#1}{#2}%
  \ifnum#1=0 %
  \else
    \ifodd#1 %
      \TestAux{-#1}{#2}%
    \else
      \TestAux{-#1}{-#2}%
    \fi
  \fi
}
\def\TestAux#1#2{%
  \def\Expected{#2}%
  \expandafter\expandafter\expandafter\def
  \expandafter\expandafter\expandafter\Result
  \expandafter\expandafter\expandafter{%
    \fibnum{#1}%
  }%
  \ltx@onelevel@sanitize\Result
  \ifx\Result\Expected
    \msg{* #1: ok.}%
  \else
    \msg{! fib(#1) = #2}%
    \errmessage{fib(#1) <> \Result}%
  \fi
}
\TestSet
\setbox0=\hbox{%
  \msg{* PreCalc{73}}%
  \fibnumPreCalc{73}%
}
\ifdim\wd0=0pt
\else
  \errmessage{Unwanted stuff in PreCalc}%
\fi
\TestSet
\csname @@end\endcsname\end
%</test-calc>
%    \end{macrocode}
%
% \section{Installation}
%
% \subsection{Download}
%
% \paragraph{Package.} This package is available on
% CTAN\footnote{\url{http://ctan.org/pkg/fibnum}}:
% \begin{description}
% \item[\CTAN{macros/latex/contrib/oberdiek/fibnum.dtx}] The source file.
% \item[\CTAN{macros/latex/contrib/oberdiek/fibnum.pdf}] Documentation.
% \end{description}
%
%
% \paragraph{Bundle.} All the packages of the bundle `oberdiek'
% are also available in a TDS compliant ZIP archive. There
% the packages are already unpacked and the documentation files
% are generated. The files and directories obey the TDS standard.
% \begin{description}
% \item[\CTAN{install/macros/latex/contrib/oberdiek.tds.zip}]
% \end{description}
% \emph{TDS} refers to the standard ``A Directory Structure
% for \TeX\ Files'' (\CTAN{tds/tds.pdf}). Directories
% with \xfile{texmf} in their name are usually organized this way.
%
% \subsection{Bundle installation}
%
% \paragraph{Unpacking.} Unpack the \xfile{oberdiek.tds.zip} in the
% TDS tree (also known as \xfile{texmf} tree) of your choice.
% Example (linux):
% \begin{quote}
%   |unzip oberdiek.tds.zip -d ~/texmf|
% \end{quote}
%
% \paragraph{Script installation.}
% Check the directory \xfile{TDS:scripts/oberdiek/} for
% scripts that need further installation steps.
% Package \xpackage{attachfile2} comes with the Perl script
% \xfile{pdfatfi.pl} that should be installed in such a way
% that it can be called as \texttt{pdfatfi}.
% Example (linux):
% \begin{quote}
%   |chmod +x scripts/oberdiek/pdfatfi.pl|\\
%   |cp scripts/oberdiek/pdfatfi.pl /usr/local/bin/|
% \end{quote}
%
% \subsection{Package installation}
%
% \paragraph{Unpacking.} The \xfile{.dtx} file is a self-extracting
% \docstrip\ archive. The files are extracted by running the
% \xfile{.dtx} through \plainTeX:
% \begin{quote}
%   \verb|tex fibnum.dtx|
% \end{quote}
%
% \paragraph{TDS.} Now the different files must be moved into
% the different directories in your installation TDS tree
% (also known as \xfile{texmf} tree):
% \begin{quote}
% \def\t{^^A
% \begin{tabular}{@{}>{\ttfamily}l@{ $\rightarrow$ }>{\ttfamily}l@{}}
%   fibnum.sty & tex/generic/oberdiek/fibnum.sty\\
%   fibnum.pdf & doc/latex/oberdiek/fibnum.pdf\\
%   test/fibnum-test1.tex & doc/latex/oberdiek/test/fibnum-test1.tex\\
%   test/fibnum-test-calc.tex & doc/latex/oberdiek/test/fibnum-test-calc.tex\\
%   fibnum.dtx & source/latex/oberdiek/fibnum.dtx\\
% \end{tabular}^^A
% }^^A
% \sbox0{\t}^^A
% \ifdim\wd0>\linewidth
%   \begingroup
%     \advance\linewidth by\leftmargin
%     \advance\linewidth by\rightmargin
%   \edef\x{\endgroup
%     \def\noexpand\lw{\the\linewidth}^^A
%   }\x
%   \def\lwbox{^^A
%     \leavevmode
%     \hbox to \linewidth{^^A
%       \kern-\leftmargin\relax
%       \hss
%       \usebox0
%       \hss
%       \kern-\rightmargin\relax
%     }^^A
%   }^^A
%   \ifdim\wd0>\lw
%     \sbox0{\small\t}^^A
%     \ifdim\wd0>\linewidth
%       \ifdim\wd0>\lw
%         \sbox0{\footnotesize\t}^^A
%         \ifdim\wd0>\linewidth
%           \ifdim\wd0>\lw
%             \sbox0{\scriptsize\t}^^A
%             \ifdim\wd0>\linewidth
%               \ifdim\wd0>\lw
%                 \sbox0{\tiny\t}^^A
%                 \ifdim\wd0>\linewidth
%                   \lwbox
%                 \else
%                   \usebox0
%                 \fi
%               \else
%                 \lwbox
%               \fi
%             \else
%               \usebox0
%             \fi
%           \else
%             \lwbox
%           \fi
%         \else
%           \usebox0
%         \fi
%       \else
%         \lwbox
%       \fi
%     \else
%       \usebox0
%     \fi
%   \else
%     \lwbox
%   \fi
% \else
%   \usebox0
% \fi
% \end{quote}
% If you have a \xfile{docstrip.cfg} that configures and enables \docstrip's
% TDS installing feature, then some files can already be in the right
% place, see the documentation of \docstrip.
%
% \subsection{Refresh file name databases}
%
% If your \TeX~distribution
% (\teTeX, \mikTeX, \dots) relies on file name databases, you must refresh
% these. For example, \teTeX\ users run \verb|texhash| or
% \verb|mktexlsr|.
%
% \subsection{Some details for the interested}
%
% \paragraph{Attached source.}
%
% The PDF documentation on CTAN also includes the
% \xfile{.dtx} source file. It can be extracted by
% AcrobatReader 6 or higher. Another option is \textsf{pdftk},
% e.g. unpack the file into the current directory:
% \begin{quote}
%   \verb|pdftk fibnum.pdf unpack_files output .|
% \end{quote}
%
% \paragraph{Unpacking with \LaTeX.}
% The \xfile{.dtx} chooses its action depending on the format:
% \begin{description}
% \item[\plainTeX:] Run \docstrip\ and extract the files.
% \item[\LaTeX:] Generate the documentation.
% \end{description}
% If you insist on using \LaTeX\ for \docstrip\ (really,
% \docstrip\ does not need \LaTeX), then inform the autodetect routine
% about your intention:
% \begin{quote}
%   \verb|latex \let\install=y\input{fibnum.dtx}|
% \end{quote}
% Do not forget to quote the argument according to the demands
% of your shell.
%
% \paragraph{Generating the documentation.}
% You can use both the \xfile{.dtx} or the \xfile{.drv} to generate
% the documentation. The process can be configured by the
% configuration file \xfile{ltxdoc.cfg}. For instance, put this
% line into this file, if you want to have A4 as paper format:
% \begin{quote}
%   \verb|\PassOptionsToClass{a4paper}{article}|
% \end{quote}
% An example follows how to generate the
% documentation with pdf\LaTeX:
% \begin{quote}
%\begin{verbatim}
%pdflatex fibnum.dtx
%bibtex fibnum.aux
%makeindex -s gind.ist fibnum.idx
%pdflatex fibnum.dtx
%makeindex -s gind.ist fibnum.idx
%pdflatex fibnum.dtx
%\end{verbatim}
% \end{quote}
%
% \printbibliography[
%   heading=bibnumbered,
% ]
%
% \begin{History}
%   \begin{Version}{2012/04/08 v1.0}
%   \item
%     First version.
%   \end{Version}
%   \begin{Version}{2016/05/16 v1.1}
%   \item
%     Documentation updates.
%   \end{Version}
% \end{History}
%
% \PrintIndex
%
% \Finale
\endinput

%        (quote the arguments according to the demands of your shell)
%
% Documentation:
%    (a) If fibnum.drv is present:
%           latex fibnum.drv
%    (b) Without fibnum.drv:
%           latex fibnum.dtx; ...
%    The class ltxdoc loads the configuration file ltxdoc.cfg
%    if available. Here you can specify further options, e.g.
%    use A4 as paper format:
%       \PassOptionsToClass{a4paper}{article}
%
%    Programm calls to get the documentation (example):
%       pdflatex fibnum.dtx
%       bibtex fibnum.aux
%       makeindex -s gind.ist fibnum.idx
%       pdflatex fibnum.dtx
%       makeindex -s gind.ist fibnum.idx
%       pdflatex fibnum.dtx
%
% Installation:
%    TDS:tex/generic/oberdiek/fibnum.sty
%    TDS:doc/latex/oberdiek/fibnum.pdf
%    TDS:doc/latex/oberdiek/test/fibnum-test1.tex
%    TDS:doc/latex/oberdiek/test/fibnum-test-calc.tex
%    TDS:source/latex/oberdiek/fibnum.dtx
%
%<*ignore>
\begingroup
  \catcode123=1 %
  \catcode125=2 %
  \def\x{LaTeX2e}%
\expandafter\endgroup
\ifcase 0\ifx\install y1\fi\expandafter
         \ifx\csname processbatchFile\endcsname\relax\else1\fi
         \ifx\fmtname\x\else 1\fi\relax
\else\csname fi\endcsname
%</ignore>
%<*install>
\input docstrip.tex
\Msg{************************************************************************}
\Msg{* Installation}
\Msg{* Package: fibnum 2016/05/16 v1.1 Fibonacci numbers (HO)}
\Msg{************************************************************************}

\keepsilent
\askforoverwritefalse

\let\MetaPrefix\relax
\preamble

This is a generated file.

Project: fibnum
Version: 2016/05/16 v1.1

Copyright (C) 2012 by
   Heiko Oberdiek <heiko.oberdiek at googlemail.com>

This work may be distributed and/or modified under the
conditions of the LaTeX Project Public License, either
version 1.3c of this license or (at your option) any later
version. This version of this license is in
   http://www.latex-project.org/lppl/lppl-1-3c.txt
and the latest version of this license is in
   http://www.latex-project.org/lppl.txt
and version 1.3 or later is part of all distributions of
LaTeX version 2005/12/01 or later.

This work has the LPPL maintenance status "maintained".

This Current Maintainer of this work is Heiko Oberdiek.

The Base Interpreter refers to any `TeX-Format',
because some files are installed in TDS:tex/generic//.

This work consists of the main source file fibnum.dtx
and the derived files
   fibnum.sty, fibnum.pdf, fibnum.ins, fibnum.drv, fibnum.bib,
   fibnum-test1.tex, fibnum-test-calc.tex.

\endpreamble
\let\MetaPrefix\DoubleperCent

\generate{%
  \file{fibnum.ins}{\from{fibnum.dtx}{install}}%
  \file{fibnum.drv}{\from{fibnum.dtx}{driver}}%
  \nopreamble
  \nopostamble
  \file{fibnum.bib}{\from{fibnum.dtx}{bib}}%
  \usepreamble\defaultpreamble
  \usepostamble\defaultpostamble
  \usedir{tex/generic/oberdiek}%
  \file{fibnum.sty}{\from{fibnum.dtx}{package}}%
  \usedir{doc/latex/oberdiek/test}%
  \file{fibnum-test1.tex}{\from{fibnum.dtx}{test1}}%
  \file{fibnum-test-calc.tex}{\from{fibnum.dtx}{test-calc}}%
}

\catcode32=13\relax% active space
\let =\space%
\Msg{************************************************************************}
\Msg{*}
\Msg{* To finish the installation you have to move the following}
\Msg{* file into a directory searched by TeX:}
\Msg{*}
\Msg{*     fibnum.sty}
\Msg{*}
\Msg{* To produce the documentation run the file `fibnum.drv'}
\Msg{* through LaTeX.}
\Msg{*}
\Msg{* Happy TeXing!}
\Msg{*}
\Msg{************************************************************************}

\endbatchfile
%</install>
%<*bib>
@online{texhax:abraham,
  author={Abraham, Jan},
  title={[texhax] Beginner in TEX MACRO to compute functions},
  date={2012-04-07},
  url={http://tug.org/pipermail/texhax/2012-April/019146.html},
  urldate={2012-04-08},
}
@article{knuth:negafibonacci,
  author={Knuth, Donald E.},
  title={Negafibonacci Numbers and the Hyperbolic Plane},
  date={2008-12-11},
  url={http://research.allacademic.com/meta/p206842_index.html},
}
@online{wikipedia:negafibonacci,
  author={{Wikipedia contributors}},
  organization={{Wikipedia, The Free Encyclopedia}},
  title={Fibonacci numbers},
  language={langenglish},
  version={486266088},
  date={2012-04-08},
  url={http://en.wikipedia.org/w/index.php?title=Fibonacci_number&oldid=486266088},
  urldate={2012-04-08},
}
%</bib>
%<*ignore>
\fi
%</ignore>
%<*driver>
\NeedsTeXFormat{LaTeX2e}
\ProvidesFile{fibnum.drv}%
  [2016/05/16 v1.1 Fibonacci numbers (HO)]%
\documentclass{ltxdoc}
\usepackage{amsmath,amsfonts}
\usepackage{siunitx}
\usepackage{array}
\usepackage{tabularx}
\usepackage{fibnum}[2016/05/16]
\usepackage{holtxdoc}[2011/11/22]
\usepackage{csquotes}
\usepackage[
  bibencoding=ascii,
  alldates=iso8601,
]{biblatex}[2011/11/13]
\bibliography{oberdiek-source}
\bibliography{fibnum}
\begin{document}
  \DocInput{fibnum.dtx}%
\end{document}
%</driver>
% \fi
%
%
% \CharacterTable
%  {Upper-case    \A\B\C\D\E\F\G\H\I\J\K\L\M\N\O\P\Q\R\S\T\U\V\W\X\Y\Z
%   Lower-case    \a\b\c\d\e\f\g\h\i\j\k\l\m\n\o\p\q\r\s\t\u\v\w\x\y\z
%   Digits        \0\1\2\3\4\5\6\7\8\9
%   Exclamation   \!     Double quote  \"     Hash (number) \#
%   Dollar        \$     Percent       \%     Ampersand     \&
%   Acute accent  \'     Left paren    \(     Right paren   \)
%   Asterisk      \*     Plus          \+     Comma         \,
%   Minus         \-     Point         \.     Solidus       \/
%   Colon         \:     Semicolon     \;     Less than     \<
%   Equals        \=     Greater than  \>     Question mark \?
%   Commercial at \@     Left bracket  \[     Backslash     \\
%   Right bracket \]     Circumflex    \^     Underscore    \_
%   Grave accent  \`     Left brace    \{     Vertical bar  \|
%   Right brace   \}     Tilde         \~}
%
% \GetFileInfo{fibnum.drv}
%
% \title{The \xpackage{fibnum} package}
% \date{2016/05/16 v1.1}
% \author{Heiko Oberdiek\thanks
% {Please report any issues at https://github.com/ho-tex/oberdiek/issues}\\
% \xemail{heiko.oberdiek at googlemail.com}}
%
% \maketitle
%
% \begin{abstract}
% The package \xpackage{fibnum} provides expandable fibonacci
% numbers for both \hologo{LaTeX} and \hologo{plainTeX}.
% \end{abstract}
%
% \tableofcontents
%
% \section{Documentation}
%
% In the mailing list \textsf{texhax} Jan Abraham asked the question,
% how to get Fibonacci numbers in \hologo{TeX} \cite{texhax:abraham}:
% \begin{quote}
% Write a Macro in \hologo{TeX} that compute the function |\fib{m}|
% All fibonacci numbers from 1 to $m$ ($m < 40$).
% \end{quote}
% This packages provides an expandable implementation for the
% calculation of these numbers for a much larger set of indexes.
% For practical reasons the index is restricted to the same limitations
% that apply for \hologo{TeX} integer numbers.
% The range of the Fibonacci numbers, however, are not limited
% by the algorithm. They are only restricted to memory limitations,
% if they are hit.
%
% The package is loaded as \hologo{LaTeX} package in \hologo{LaTeX}:
% \begin{quote}
%   |\usepackage{fibnum}|
% \end{quote}
% and as file in \hologo{plainTeX}:
% \begin{quote}
%   |\input fibnum.sty|
% \end{quote}
% The package does not know any options and it provides
% the macros \cs{fibnum} and \cs{fibnumPreCalc}.
%
% \begin{declcs}{fibnum} \M{index}
% \end{declcs}
% Macro \cs{fibnum} expects a \hologo{TeX} number as \meta{index}
% in the official \hologo{TeX} number range from $-(2^{31}-1)$ up to
% $2^{31}-1$. In exact two expansion steps the macro expands to
% the Fibnoacci number $F_{\text{\meta{index}}}$. In case of a negative
% \meta{index}, the ``negafibonacci'' number \cite{wikipedia:negafibonacci}
% is used. Formally the Fibonacci number $F_n$ with integer
% index~$n$, $n\in\mathbb{Z}$ and
% $n\in[\num{-2147483647},\num{2147483647}]$ that is returned by macro
% \cs{fibnum} with numerical argument $n$ is defined the following way:
% \begin{gather}
%   \label{eq:def}
%   F_n =
%   \begin{cases}
%     0 & \text{for $n=0$}\\
%     1 & \text{for $n=1$}\\
%     F_{n-1} + F_{n-2} & \text{for $n>1$}\\
%     (-1)^{n+1}F_n & \text{for $n<0$}
%   \end{cases}
% \end{gather}
% Examples:
% \begin{quote}
%   \makeatletter
%   \def\x#1{\cs{fibnum}\{#1\}&
%     \edef\X{\fibnum{#1}}\edef\Y{\expandafter\ltx@car\X\@nil}^^A
%     \if-\Y
%       \edef\X{\expandafter\ltx@cdr\X\@nil}^^A
%       \noindent
%       \llap{-}\X
%     \else
%       \X
%     \fi
%     \tabularnewline
%   }
%   \def\y{\multicolumn{1}{@{}c@{}}{$\vdots$}\tabularnewline}
%   \DeclareUrlCommand\UrlNum{^^A
%     \urlstyle{tt}^^A
%     \def\UrlBreaks{\do\0\do\1\do\2\do\3\do\4\do\5\do\6\do\7\do\8\do\9}^^A
%   }
%   \begin{tabularx}{\dimexpr\linewidth+5.7pt\relax}{@{}>{\ttfamily}l@{ $\rightarrow$ \hphantom{\ttfamily-}}>{\ttfamily}X@{}}
%     \x{-6}
%     \x{-5}
%     \x{-4}
%     \x{-3}
%     \x{-2}
%     \x{-1}
%     \x{0}
%     \x{1}
%     \x{2}
%     \x{3}
%     \x{4}
%     \x{5}
%     \x{6}
%     \y
%     \x{10}
%     \y
%     \x{46}
%     \y
%     \cs{fibnum}\{100\} & 354224848179261915075
%     \tabularnewline
%     \y
%     \cs{fibnum}\{200\} & 280571172992510140037611932413038677189525
%     \tabularnewline
%     \y
%     \cs{fibnum}\{1000\} &
%       \raggedright
%       \UrlNum{^^A
%         434665576869374564356885276750406258025646^^A
%         605173717804024817290895365554179490518904^^A
%         038798400792551692959225930803226347752096^^A
%         896232398733224711616429964409065331879382^^A
%         98969649928516003704476137795166849228875^^A
%       }
%     \tabularnewline
%   \end{tabularx}\kern-5.7pt\mbox{}
% \end{quote}
%
% \begin{declcs}{fibnumPreCalc} \M{index}
% \end{declcs}
% The package already provides precalculated Fibonacci numbers up to
% index~46. That means that calculations are not necessary for
% Fibonacci numbers that fit into the range of \hologo{TeX}
% numbers. Because macro \cs{fibnum} is expandable, it cannot
% store calculated Fibonacci numbers for later use. Macro definitions
% are forbidden in expandable contexts. If larger Fibonacci numbers
% are used more than once, than the compilation time can be shortened
% by calculating and storing the Fibonacci numbers beforehand.
% The argument \meta{index} is a \hologo{TeX} number and macro
% \cs{fibnumPreCalc} ensures that the Fibonacci numbers
% $F_0$ up to $F_{\lvert\text{\meta{index}}\rvert}$ that are not
% already known are calculated
% and stored in internal macros. Internally only non-negative
% Fibonacci numbers are stored. If \meta{index} is negative, then
% the needed positive Fibonacci numbers are calculated and stored.
% Example:
% \begin{quote}
%   \def\x#1{\begingroup\itshape\texttt{\%} #1\endgroup}
%   |\fibnumPreCalc{50}|\\
%   \x{calculates and stores the values for indexes 47..50.}\\
%   \x{(Values for 0..46 are already stored by the package.)}\\
%   |\fibnum{49}| \x{uses the stored value}\\
%   |\fibnum{51}|
%   \x{only calculates $F_{51}$ from stored values $F_{49}$ and $F_{50}$}\\
%   |\fibnumPreCalc{100}|\\
%   \x{calculates and stores the values for indexes 51..100}\\
%   |\fibnum{100}| \x{uses the stored value for $F_{100}$}\\
%   |\fibnum{-100}|
%   \x{uses the stored value for $F_{100}$}\\
%   \x{$F_{-100}=-F_{100}$ according to equation \eqref{eq:def}.}
% \end{quote}
%
% \StopEventually{
% }
%
% \section{Implementation}
%
% \subsection{Identification}
%
%    \begin{macrocode}
%<*package>
%    \end{macrocode}
%    Reload check, especially if the package is not used with \LaTeX.
%    \begin{macrocode}
\begingroup\catcode61\catcode48\catcode32=10\relax%
  \catcode13=5 % ^^M
  \endlinechar=13 %
  \catcode35=6 % #
  \catcode39=12 % '
  \catcode44=12 % ,
  \catcode45=12 % -
  \catcode46=12 % .
  \catcode58=12 % :
  \catcode64=11 % @
  \catcode123=1 % {
  \catcode125=2 % }
  \expandafter\let\expandafter\x\csname ver@fibnum.sty\endcsname
  \ifx\x\relax % plain-TeX, first loading
  \else
    \def\empty{}%
    \ifx\x\empty % LaTeX, first loading,
      % variable is initialized, but \ProvidesPackage not yet seen
    \else
      \expandafter\ifx\csname PackageInfo\endcsname\relax
        \def\x#1#2{%
          \immediate\write-1{Package #1 Info: #2.}%
        }%
      \else
        \def\x#1#2{\PackageInfo{#1}{#2, stopped}}%
      \fi
      \x{fibnum}{The package is already loaded}%
      \aftergroup\endinput
    \fi
  \fi
\endgroup%
%    \end{macrocode}
%    Package identification:
%    \begin{macrocode}
\begingroup\catcode61\catcode48\catcode32=10\relax%
  \catcode13=5 % ^^M
  \endlinechar=13 %
  \catcode35=6 % #
  \catcode39=12 % '
  \catcode40=12 % (
  \catcode41=12 % )
  \catcode44=12 % ,
  \catcode45=12 % -
  \catcode46=12 % .
  \catcode47=12 % /
  \catcode58=12 % :
  \catcode64=11 % @
  \catcode91=12 % [
  \catcode93=12 % ]
  \catcode123=1 % {
  \catcode125=2 % }
  \expandafter\ifx\csname ProvidesPackage\endcsname\relax
    \def\x#1#2#3[#4]{\endgroup
      \immediate\write-1{Package: #3 #4}%
      \xdef#1{#4}%
    }%
  \else
    \def\x#1#2[#3]{\endgroup
      #2[{#3}]%
      \ifx#1\@undefined
        \xdef#1{#3}%
      \fi
      \ifx#1\relax
        \xdef#1{#3}%
      \fi
    }%
  \fi
\expandafter\x\csname ver@fibnum.sty\endcsname
\ProvidesPackage{fibnum}%
  [2016/05/16 v1.1 Fibonacci numbers (HO)]%
%    \end{macrocode}
%
%    \begin{macrocode}
\begingroup\catcode61\catcode48\catcode32=10\relax%
  \catcode13=5 % ^^M
  \endlinechar=13 %
  \catcode123=1 % {
  \catcode125=2 % }
  \catcode64=11 % @
  \def\x{\endgroup
    \expandafter\edef\csname FibNum@AtEnd\endcsname{%
      \endlinechar=\the\endlinechar\relax
      \catcode13=\the\catcode13\relax
      \catcode32=\the\catcode32\relax
      \catcode35=\the\catcode35\relax
      \catcode61=\the\catcode61\relax
      \catcode64=\the\catcode64\relax
      \catcode123=\the\catcode123\relax
      \catcode125=\the\catcode125\relax
    }%
  }%
\x\catcode61\catcode48\catcode32=10\relax%
\catcode13=5 % ^^M
\endlinechar=13 %
\catcode35=6 % #
\catcode64=11 % @
\catcode123=1 % {
\catcode125=2 % }
\def\TMP@EnsureCode#1#2{%
  \edef\FibNum@AtEnd{%
    \FibNum@AtEnd
    \catcode#1=\the\catcode#1\relax
  }%
  \catcode#1=#2\relax
}
\TMP@EnsureCode{33}{12}% !
%\TMP@EnsureCode{36}{3}% $
%\TMP@EnsureCode{38}{4}% &
\TMP@EnsureCode{40}{12}% (
\TMP@EnsureCode{41}{12}% )
\TMP@EnsureCode{45}{12}% -
\TMP@EnsureCode{46}{12}% .
\TMP@EnsureCode{47}{12}% /
\TMP@EnsureCode{58}{12}% :
\TMP@EnsureCode{60}{12}% <
\TMP@EnsureCode{62}{12}% >
\TMP@EnsureCode{91}{12}% [
%\TMP@EnsureCode{96}{12}% `
\TMP@EnsureCode{93}{12}% ]
%\TMP@EnsureCode{94}{12}% ^ (superscript) (!)
%\TMP@EnsureCode{124}{12}% |
\edef\FibNum@AtEnd{\FibNum@AtEnd\noexpand\endinput}
%    \end{macrocode}
%
% \subsection{Package resources}
%
%    \begin{macrocode}
\begingroup\expandafter\expandafter\expandafter\endgroup
\expandafter\ifx\csname RequirePackage\endcsname\relax
  \def\TMP@RequirePackage#1[#2]{%
    \begingroup\expandafter\expandafter\expandafter\endgroup
    \expandafter\ifx\csname ver@#1.sty\endcsname\relax
      \input #1.sty\relax
    \fi
  }%
  \TMP@RequirePackage{ltxcmds}[2011/04/18]%
  \TMP@RequirePackage{intcalc}[2007/09/27]%
  \TMP@RequirePackage{bigintcalc}[2007/11/11]%
\else
  \RequirePackage{ltxcmds}[2011/04/18]%
  \RequirePackage{intcalc}[2007/09/27]%
  \RequirePackage{bigintcalc}[2007/11/11]%
\fi
%    \end{macrocode}
%
% \subsection{Setup precalculated values}
%
%    \begin{macrocode}
\def\FibNum@temp#1{%
  \expandafter\def\csname FibNum@#1\endcsname
}
\catcode46=9 % dots are ignored
\FibNum@temp{0}{0}
\FibNum@temp{1}{1}
\FibNum@temp{2}{1}
\FibNum@temp{3}{2}
\FibNum@temp{4}{3}
\FibNum@temp{5}{5}
\FibNum@temp{6}{8}
\FibNum@temp{7}{13}
\FibNum@temp{8}{21}
\FibNum@temp{9}{34}
\FibNum@temp{10}{55}
\FibNum@temp{11}{89}
\FibNum@temp{12}{144}
\FibNum@temp{13}{233}
\FibNum@temp{14}{377}
\FibNum@temp{15}{610}
\FibNum@temp{16}{987}
\FibNum@temp{17}{1.597}
\FibNum@temp{18}{2.584}
\FibNum@temp{19}{4.181}
\FibNum@temp{20}{6.765}
\FibNum@temp{21}{10.946}
\FibNum@temp{22}{17.711}
\FibNum@temp{23}{28.657}
\FibNum@temp{24}{46.368}
\FibNum@temp{25}{75.025}
\FibNum@temp{26}{121.393}
\FibNum@temp{27}{196.418}
\FibNum@temp{28}{317.811}
\FibNum@temp{29}{514.229}
\FibNum@temp{30}{832.040}
\FibNum@temp{31}{1.346.269}
\FibNum@temp{32}{2.178.309}
\FibNum@temp{33}{3.524.578}
\FibNum@temp{34}{5.702.887}
\FibNum@temp{35}{9.227.465}
\FibNum@temp{36}{14.930.352}
\FibNum@temp{37}{24.157.817}
\FibNum@temp{38}{39.088.169}
\FibNum@temp{39}{63.245.986}
\FibNum@temp{40}{102.334.155}
\FibNum@temp{41}{165.580.141}
\FibNum@temp{42}{267.914.296}
\FibNum@temp{43}{433.494.437}
\FibNum@temp{44}{701.408.733}
\FibNum@temp{45}{1.134.903.170}
\FibNum@temp{46}{1.836.311.903}
%    \end{macrocode}
%    \begin{macro}{\FibNum@max}
%    \begin{macrocode}
\def\FibNum@max{46}
%    \end{macrocode}
%    \end{macro}
%
% \subsection{Macros for precalculating values}
%
%    \begin{macro}{\fibnumPreCalc}
%    \begin{macrocode}
\def\fibnumPreCalc#1{%
  \expandafter\expandafter\expandafter
  \FibNum@PreCalc\intcalcNum{#1}/%
}
%    \end{macrocode}
%    \end{macro}
%    \begin{macro}{\FibNum@PreCalc}
%    \begin{macrocode}
\def\FibNum@PreCalc#1/{%
  \ifnum#1<\ltx@zero
    \expandafter\FibNum@PreCalc\ltx@gobble#1/%
  \else
    \ifnum#1>\FibNum@max
      \begingroup
        \ltx@LocDimenA=#1sp\relax
        \countdef\FibNum@i=255\relax
        \FibNum@i=\FibNum@max\relax
        \edef\FibNum@temp{%
          \csname FibNum@\the\FibNum@i\endcsname/%
        }%
        \advance\FibNum@i by -1\relax
        \edef\FibNum@temp{%
          \FibNum@temp
          \csname FibNum@\the\FibNum@i\endcsname
        }%
        \advance\FibNum@i\ltx@two
        \iftrue
          \expandafter\FibNum@PreCalcAux\FibNum@temp
        \fi
      \endgroup
    \fi
  \fi
}
%    \end{macrocode}
%    \end{macro}
%    \begin{macro}{\FibNum@PreCalcAux}
%    \begin{macrocode}
\def\FibNum@PreCalcAux#1/#2\fi{%
  \fi
  \edef\FibNum@temp{\BigIntCalcAdd#1!#2!}%
  \global\expandafter
  \let\csname FibNum@\the\FibNum@i\endcsname\FibNum@temp
  \ifnum\FibNum@i=\ltx@LocDimenA
    \xdef\FibNum@max{\the\FibNum@i}%
  \else
    \advance\FibNum@i\ltx@one
    \expandafter\FibNum@PreCalcAux\FibNum@temp/#1%
  \fi
}
%    \end{macrocode}
%    \end{macro}
%
% \subsection{Expandable calculations}
%
%    \begin{macro}{\fibnum}
%    \begin{macrocode}
\def\fibnum#1{%
  \romannumeral
  \expandafter\expandafter\expandafter\FibNum@Do\intcalcNum{#1}/%
}
%    \end{macrocode}
%    \end{macro}
%    \begin{macro}{\FibNum@Do}
%    \begin{macrocode}
\def\FibNum@Do#1/{%
  \ifnum#1<\ltx@zero
    \FibNum@ReturnAfterElseFiFi{%
      \ifodd#1 %
        \expandafter\expandafter\expandafter\ltx@zero
      \else
        \expandafter\expandafter\expandafter\ltx@zero
        \expandafter\expandafter\expandafter-%
      \fi
      \romannumeral
      \expandafter\FibNum@Do\ltx@gobble#1/%
    }%
  \else
    \ifnum\FibNum@max<#1 %
      \ltx@ReturnAfterElseFi{%
        \expandafter
        \FibNum@ExpCalc\number\expandafter\IntCalcInc\FibNum@max!%
        \expandafter\expandafter\expandafter/%
        \csname FibNum@\FibNum@max
        \expandafter\expandafter\expandafter\endcsname
        \expandafter\expandafter\expandafter/%
        \csname FibNum@\expandafter\IntCalcDec\FibNum@max!%
        \endcsname/%
        #1%
      }%
    \else
      \expandafter\expandafter\expandafter\ltx@zero
      \csname FibNum@#1\expandafter\expandafter\expandafter\endcsname
    \fi
  \fi
}
%    \end{macrocode}
%    \end{macro}
%    \begin{macro}{\FibNum@ReturnAfterElseFiFi}
%    \begin{macrocode}
\def\FibNum@ReturnAfterElseFiFi#1\else#2\fi\fi{\fi#1}
%    \end{macrocode}
%    \end{macro}
%    \begin{macro}{\FibNum@ExpCalc}
%    \begin{macrocode}
\def\FibNum@ExpCalc#1/#2/#3/#4\fi{%
  \fi
  \ifnum#1=#4 %
    \ltx@ReturnAfterElseFi{%
      \expandafter\expandafter\expandafter\ltx@zero
      \BigIntCalcAdd#2!#3!%
    }%
  \else
    \expandafter\FibNum@ExpCalc
    \number\IntCalcInc#1!%
    \expandafter\expandafter\expandafter/%
    \BigIntCalcAdd#2!#3!/%
    #2/#4%
  \fi
}
%    \end{macrocode}
%    \end{macro}
%
%    \begin{macrocode}
\FibNum@AtEnd%
%</package>
%    \end{macrocode}
%
% \section{Test}
%
% \subsection{Catcode checks for loading}
%
%    \begin{macrocode}
%<*test1>
%    \end{macrocode}
%    \begin{macrocode}
\catcode`\{=1 %
\catcode`\}=2 %
\catcode`\#=6 %
\catcode`\@=11 %
\expandafter\ifx\csname count@\endcsname\relax
  \countdef\count@=255 %
\fi
\expandafter\ifx\csname @gobble\endcsname\relax
  \long\def\@gobble#1{}%
\fi
\expandafter\ifx\csname @firstofone\endcsname\relax
  \long\def\@firstofone#1{#1}%
\fi
\expandafter\ifx\csname loop\endcsname\relax
  \expandafter\@firstofone
\else
  \expandafter\@gobble
\fi
{%
  \def\loop#1\repeat{%
    \def\body{#1}%
    \iterate
  }%
  \def\iterate{%
    \body
      \let\next\iterate
    \else
      \let\next\relax
    \fi
    \next
  }%
  \let\repeat=\fi
}%
\def\RestoreCatcodes{}
\count@=0 %
\loop
  \edef\RestoreCatcodes{%
    \RestoreCatcodes
    \catcode\the\count@=\the\catcode\count@\relax
  }%
\ifnum\count@<255 %
  \advance\count@ 1 %
\repeat

\def\RangeCatcodeInvalid#1#2{%
  \count@=#1\relax
  \loop
    \catcode\count@=15 %
  \ifnum\count@<#2\relax
    \advance\count@ 1 %
  \repeat
}
\def\RangeCatcodeCheck#1#2#3{%
  \count@=#1\relax
  \loop
    \ifnum#3=\catcode\count@
    \else
      \errmessage{%
        Character \the\count@\space
        with wrong catcode \the\catcode\count@\space
        instead of \number#3%
      }%
    \fi
  \ifnum\count@<#2\relax
    \advance\count@ 1 %
  \repeat
}
\def\space{ }
\expandafter\ifx\csname LoadCommand\endcsname\relax
  \def\LoadCommand{\input fibnum.sty\relax}%
\fi
\def\Test{%
  \RangeCatcodeInvalid{0}{47}%
  \RangeCatcodeInvalid{58}{64}%
  \RangeCatcodeInvalid{91}{96}%
  \RangeCatcodeInvalid{123}{255}%
  \catcode`\@=12 %
  \catcode`\\=0 %
  \catcode`\%=14 %
  \LoadCommand
  \RangeCatcodeCheck{0}{36}{15}%
  \RangeCatcodeCheck{37}{37}{14}%
  \RangeCatcodeCheck{38}{47}{15}%
  \RangeCatcodeCheck{48}{57}{12}%
  \RangeCatcodeCheck{58}{63}{15}%
  \RangeCatcodeCheck{64}{64}{12}%
  \RangeCatcodeCheck{65}{90}{11}%
  \RangeCatcodeCheck{91}{91}{15}%
  \RangeCatcodeCheck{92}{92}{0}%
  \RangeCatcodeCheck{93}{96}{15}%
  \RangeCatcodeCheck{97}{122}{11}%
  \RangeCatcodeCheck{123}{255}{15}%
  \RestoreCatcodes
}
\Test
\csname @@end\endcsname
\end
%    \end{macrocode}
%    \begin{macrocode}
%</test1>
%    \end{macrocode}
%
% \subsection{Test calculations}
%
%    \begin{macrocode}
%<*test-calc>
\catcode`\{=1 %
\catcode`\}=2 %
\catcode`\#=6 %
\catcode`\@=11 %
\begingroup\expandafter\expandafter\expandafter\endgroup
\expandafter\ifx\csname RequirePackage\endcsname\relax
  \input fibnum.sty\relax
\else
  \RequirePackage{fibnum}[2016/05/16]%
\fi
\def\TestSet{%
  \test{0}{0}%
  \test{1}{1}%
  \test{2}{1}%
  \test{3}{2}%
  \test{4}{3}%
  \test{5}{5}%
  \test{6}{8}%
  \test{7}{13}%
  \test{8}{21}%
  \test{9}{34}%
  \test{10}{55}%
  \test{11}{89}%
  \test{12}{144}%
  \test{13}{233}%
  \test{14}{377}%
  \test{15}{610}%
  \test{16}{987}%
  \test{17}{1597}%
  \test{18}{2584}%
  \test{19}{4181}%
  \test{20}{6765}%
  \test{21}{10946}%
  \test{22}{17711}%
  \test{23}{28657}%
  \test{24}{46368}%
  \test{25}{75025}%
  \test{26}{121393}%
  \test{27}{196418}%
  \test{28}{317811}%
  \test{29}{514229}%
  \test{30}{832040}%
  \test{31}{1346269}%
  \test{32}{2178309}%
  \test{33}{3524578}%
  \test{34}{5702887}%
  \test{35}{9227465}%
  \test{36}{14930352}%
  \test{37}{24157817}%
  \test{38}{39088169}%
  \test{39}{63245986}%
  \test{40}{102334155}%
  \test{41}{165580141}%
  \test{42}{267914296}%
  \test{43}{433494437}%
  \test{44}{701408733}%
  \test{45}{1134903170}%
  \test{46}{1836311903}%
  \test{47}{2971215073}%
  \test{48}{4807526976}%
  \test{49}{7778742049}%
  \test{50}{12586269025}%
  \test{51}{20365011074}%
  \test{52}{32951280099}%
  \test{53}{53316291173}%
  \test{54}{86267571272}%
  \test{55}{139583862445}%
  \test{56}{225851433717}%
  \test{57}{365435296162}%
  \test{58}{591286729879}%
  \test{59}{956722026041}%
  \test{60}{1548008755920}%
  \test{61}{2504730781961}%
  \test{62}{4052739537881}%
  \test{63}{6557470319842}%
  \test{64}{10610209857723}%
  \test{65}{17167680177565}%
  \test{66}{27777890035288}%
  \test{67}{44945570212853}%
  \test{68}{72723460248141}%
  \test{69}{117669030460994}%
  \test{70}{190392490709135}%
  \test{71}{308061521170129}%
  \test{72}{498454011879264}%
  \test{73}{806515533049393}%
}
\def\msg#{\immediate\write16}
\def\test#1#2{%
  \TestAux{#1}{#2}%
  \ifnum#1=0 %
  \else
    \ifodd#1 %
      \TestAux{-#1}{#2}%
    \else
      \TestAux{-#1}{-#2}%
    \fi
  \fi
}
\def\TestAux#1#2{%
  \def\Expected{#2}%
  \expandafter\expandafter\expandafter\def
  \expandafter\expandafter\expandafter\Result
  \expandafter\expandafter\expandafter{%
    \fibnum{#1}%
  }%
  \ltx@onelevel@sanitize\Result
  \ifx\Result\Expected
    \msg{* #1: ok.}%
  \else
    \msg{! fib(#1) = #2}%
    \errmessage{fib(#1) <> \Result}%
  \fi
}
\TestSet
\setbox0=\hbox{%
  \msg{* PreCalc{73}}%
  \fibnumPreCalc{73}%
}
\ifdim\wd0=0pt
\else
  \errmessage{Unwanted stuff in PreCalc}%
\fi
\TestSet
\csname @@end\endcsname\end
%</test-calc>
%    \end{macrocode}
%
% \section{Installation}
%
% \subsection{Download}
%
% \paragraph{Package.} This package is available on
% CTAN\footnote{\url{http://ctan.org/pkg/fibnum}}:
% \begin{description}
% \item[\CTAN{macros/latex/contrib/oberdiek/fibnum.dtx}] The source file.
% \item[\CTAN{macros/latex/contrib/oberdiek/fibnum.pdf}] Documentation.
% \end{description}
%
%
% \paragraph{Bundle.} All the packages of the bundle `oberdiek'
% are also available in a TDS compliant ZIP archive. There
% the packages are already unpacked and the documentation files
% are generated. The files and directories obey the TDS standard.
% \begin{description}
% \item[\CTAN{install/macros/latex/contrib/oberdiek.tds.zip}]
% \end{description}
% \emph{TDS} refers to the standard ``A Directory Structure
% for \TeX\ Files'' (\CTAN{tds/tds.pdf}). Directories
% with \xfile{texmf} in their name are usually organized this way.
%
% \subsection{Bundle installation}
%
% \paragraph{Unpacking.} Unpack the \xfile{oberdiek.tds.zip} in the
% TDS tree (also known as \xfile{texmf} tree) of your choice.
% Example (linux):
% \begin{quote}
%   |unzip oberdiek.tds.zip -d ~/texmf|
% \end{quote}
%
% \paragraph{Script installation.}
% Check the directory \xfile{TDS:scripts/oberdiek/} for
% scripts that need further installation steps.
% Package \xpackage{attachfile2} comes with the Perl script
% \xfile{pdfatfi.pl} that should be installed in such a way
% that it can be called as \texttt{pdfatfi}.
% Example (linux):
% \begin{quote}
%   |chmod +x scripts/oberdiek/pdfatfi.pl|\\
%   |cp scripts/oberdiek/pdfatfi.pl /usr/local/bin/|
% \end{quote}
%
% \subsection{Package installation}
%
% \paragraph{Unpacking.} The \xfile{.dtx} file is a self-extracting
% \docstrip\ archive. The files are extracted by running the
% \xfile{.dtx} through \plainTeX:
% \begin{quote}
%   \verb|tex fibnum.dtx|
% \end{quote}
%
% \paragraph{TDS.} Now the different files must be moved into
% the different directories in your installation TDS tree
% (also known as \xfile{texmf} tree):
% \begin{quote}
% \def\t{^^A
% \begin{tabular}{@{}>{\ttfamily}l@{ $\rightarrow$ }>{\ttfamily}l@{}}
%   fibnum.sty & tex/generic/oberdiek/fibnum.sty\\
%   fibnum.pdf & doc/latex/oberdiek/fibnum.pdf\\
%   test/fibnum-test1.tex & doc/latex/oberdiek/test/fibnum-test1.tex\\
%   test/fibnum-test-calc.tex & doc/latex/oberdiek/test/fibnum-test-calc.tex\\
%   fibnum.dtx & source/latex/oberdiek/fibnum.dtx\\
% \end{tabular}^^A
% }^^A
% \sbox0{\t}^^A
% \ifdim\wd0>\linewidth
%   \begingroup
%     \advance\linewidth by\leftmargin
%     \advance\linewidth by\rightmargin
%   \edef\x{\endgroup
%     \def\noexpand\lw{\the\linewidth}^^A
%   }\x
%   \def\lwbox{^^A
%     \leavevmode
%     \hbox to \linewidth{^^A
%       \kern-\leftmargin\relax
%       \hss
%       \usebox0
%       \hss
%       \kern-\rightmargin\relax
%     }^^A
%   }^^A
%   \ifdim\wd0>\lw
%     \sbox0{\small\t}^^A
%     \ifdim\wd0>\linewidth
%       \ifdim\wd0>\lw
%         \sbox0{\footnotesize\t}^^A
%         \ifdim\wd0>\linewidth
%           \ifdim\wd0>\lw
%             \sbox0{\scriptsize\t}^^A
%             \ifdim\wd0>\linewidth
%               \ifdim\wd0>\lw
%                 \sbox0{\tiny\t}^^A
%                 \ifdim\wd0>\linewidth
%                   \lwbox
%                 \else
%                   \usebox0
%                 \fi
%               \else
%                 \lwbox
%               \fi
%             \else
%               \usebox0
%             \fi
%           \else
%             \lwbox
%           \fi
%         \else
%           \usebox0
%         \fi
%       \else
%         \lwbox
%       \fi
%     \else
%       \usebox0
%     \fi
%   \else
%     \lwbox
%   \fi
% \else
%   \usebox0
% \fi
% \end{quote}
% If you have a \xfile{docstrip.cfg} that configures and enables \docstrip's
% TDS installing feature, then some files can already be in the right
% place, see the documentation of \docstrip.
%
% \subsection{Refresh file name databases}
%
% If your \TeX~distribution
% (\teTeX, \mikTeX, \dots) relies on file name databases, you must refresh
% these. For example, \teTeX\ users run \verb|texhash| or
% \verb|mktexlsr|.
%
% \subsection{Some details for the interested}
%
% \paragraph{Attached source.}
%
% The PDF documentation on CTAN also includes the
% \xfile{.dtx} source file. It can be extracted by
% AcrobatReader 6 or higher. Another option is \textsf{pdftk},
% e.g. unpack the file into the current directory:
% \begin{quote}
%   \verb|pdftk fibnum.pdf unpack_files output .|
% \end{quote}
%
% \paragraph{Unpacking with \LaTeX.}
% The \xfile{.dtx} chooses its action depending on the format:
% \begin{description}
% \item[\plainTeX:] Run \docstrip\ and extract the files.
% \item[\LaTeX:] Generate the documentation.
% \end{description}
% If you insist on using \LaTeX\ for \docstrip\ (really,
% \docstrip\ does not need \LaTeX), then inform the autodetect routine
% about your intention:
% \begin{quote}
%   \verb|latex \let\install=y% \iffalse meta-comment
%
% File: fibnum.dtx
% Version: 2016/05/16 v1.1
% Info: Fibonacci numbers
%
% Copyright (C) 2012 by
%    Heiko Oberdiek <heiko.oberdiek at googlemail.com>
%    2016
%    https://github.com/ho-tex/oberdiek/issues
%
% This work may be distributed and/or modified under the
% conditions of the LaTeX Project Public License, either
% version 1.3c of this license or (at your option) any later
% version. This version of this license is in
%    http://www.latex-project.org/lppl/lppl-1-3c.txt
% and the latest version of this license is in
%    http://www.latex-project.org/lppl.txt
% and version 1.3 or later is part of all distributions of
% LaTeX version 2005/12/01 or later.
%
% This work has the LPPL maintenance status "maintained".
%
% This Current Maintainer of this work is Heiko Oberdiek.
%
% The Base Interpreter refers to any `TeX-Format',
% because some files are installed in TDS:tex/generic//.
%
% This work consists of the main source file fibnum.dtx
% and the derived files
%    fibnum.sty, fibnum.pdf, fibnum.ins, fibnum.drv, fibnum.bib,
%    fibnum-test1.tex, fibnum-test-calc.tex.
%
% Distribution:
%    CTAN:macros/latex/contrib/oberdiek/fibnum.dtx
%    CTAN:macros/latex/contrib/oberdiek/fibnum.pdf
%
% Unpacking:
%    (a) If fibnum.ins is present:
%           tex fibnum.ins
%    (b) Without fibnum.ins:
%           tex fibnum.dtx
%    (c) If you insist on using LaTeX
%           latex \let\install=y\input{fibnum.dtx}
%        (quote the arguments according to the demands of your shell)
%
% Documentation:
%    (a) If fibnum.drv is present:
%           latex fibnum.drv
%    (b) Without fibnum.drv:
%           latex fibnum.dtx; ...
%    The class ltxdoc loads the configuration file ltxdoc.cfg
%    if available. Here you can specify further options, e.g.
%    use A4 as paper format:
%       \PassOptionsToClass{a4paper}{article}
%
%    Programm calls to get the documentation (example):
%       pdflatex fibnum.dtx
%       bibtex fibnum.aux
%       makeindex -s gind.ist fibnum.idx
%       pdflatex fibnum.dtx
%       makeindex -s gind.ist fibnum.idx
%       pdflatex fibnum.dtx
%
% Installation:
%    TDS:tex/generic/oberdiek/fibnum.sty
%    TDS:doc/latex/oberdiek/fibnum.pdf
%    TDS:doc/latex/oberdiek/test/fibnum-test1.tex
%    TDS:doc/latex/oberdiek/test/fibnum-test-calc.tex
%    TDS:source/latex/oberdiek/fibnum.dtx
%
%<*ignore>
\begingroup
  \catcode123=1 %
  \catcode125=2 %
  \def\x{LaTeX2e}%
\expandafter\endgroup
\ifcase 0\ifx\install y1\fi\expandafter
         \ifx\csname processbatchFile\endcsname\relax\else1\fi
         \ifx\fmtname\x\else 1\fi\relax
\else\csname fi\endcsname
%</ignore>
%<*install>
\input docstrip.tex
\Msg{************************************************************************}
\Msg{* Installation}
\Msg{* Package: fibnum 2016/05/16 v1.1 Fibonacci numbers (HO)}
\Msg{************************************************************************}

\keepsilent
\askforoverwritefalse

\let\MetaPrefix\relax
\preamble

This is a generated file.

Project: fibnum
Version: 2016/05/16 v1.1

Copyright (C) 2012 by
   Heiko Oberdiek <heiko.oberdiek at googlemail.com>

This work may be distributed and/or modified under the
conditions of the LaTeX Project Public License, either
version 1.3c of this license or (at your option) any later
version. This version of this license is in
   http://www.latex-project.org/lppl/lppl-1-3c.txt
and the latest version of this license is in
   http://www.latex-project.org/lppl.txt
and version 1.3 or later is part of all distributions of
LaTeX version 2005/12/01 or later.

This work has the LPPL maintenance status "maintained".

This Current Maintainer of this work is Heiko Oberdiek.

The Base Interpreter refers to any `TeX-Format',
because some files are installed in TDS:tex/generic//.

This work consists of the main source file fibnum.dtx
and the derived files
   fibnum.sty, fibnum.pdf, fibnum.ins, fibnum.drv, fibnum.bib,
   fibnum-test1.tex, fibnum-test-calc.tex.

\endpreamble
\let\MetaPrefix\DoubleperCent

\generate{%
  \file{fibnum.ins}{\from{fibnum.dtx}{install}}%
  \file{fibnum.drv}{\from{fibnum.dtx}{driver}}%
  \nopreamble
  \nopostamble
  \file{fibnum.bib}{\from{fibnum.dtx}{bib}}%
  \usepreamble\defaultpreamble
  \usepostamble\defaultpostamble
  \usedir{tex/generic/oberdiek}%
  \file{fibnum.sty}{\from{fibnum.dtx}{package}}%
  \usedir{doc/latex/oberdiek/test}%
  \file{fibnum-test1.tex}{\from{fibnum.dtx}{test1}}%
  \file{fibnum-test-calc.tex}{\from{fibnum.dtx}{test-calc}}%
}

\catcode32=13\relax% active space
\let =\space%
\Msg{************************************************************************}
\Msg{*}
\Msg{* To finish the installation you have to move the following}
\Msg{* file into a directory searched by TeX:}
\Msg{*}
\Msg{*     fibnum.sty}
\Msg{*}
\Msg{* To produce the documentation run the file `fibnum.drv'}
\Msg{* through LaTeX.}
\Msg{*}
\Msg{* Happy TeXing!}
\Msg{*}
\Msg{************************************************************************}

\endbatchfile
%</install>
%<*bib>
@online{texhax:abraham,
  author={Abraham, Jan},
  title={[texhax] Beginner in TEX MACRO to compute functions},
  date={2012-04-07},
  url={http://tug.org/pipermail/texhax/2012-April/019146.html},
  urldate={2012-04-08},
}
@article{knuth:negafibonacci,
  author={Knuth, Donald E.},
  title={Negafibonacci Numbers and the Hyperbolic Plane},
  date={2008-12-11},
  url={http://research.allacademic.com/meta/p206842_index.html},
}
@online{wikipedia:negafibonacci,
  author={{Wikipedia contributors}},
  organization={{Wikipedia, The Free Encyclopedia}},
  title={Fibonacci numbers},
  language={langenglish},
  version={486266088},
  date={2012-04-08},
  url={http://en.wikipedia.org/w/index.php?title=Fibonacci_number&oldid=486266088},
  urldate={2012-04-08},
}
%</bib>
%<*ignore>
\fi
%</ignore>
%<*driver>
\NeedsTeXFormat{LaTeX2e}
\ProvidesFile{fibnum.drv}%
  [2016/05/16 v1.1 Fibonacci numbers (HO)]%
\documentclass{ltxdoc}
\usepackage{amsmath,amsfonts}
\usepackage{siunitx}
\usepackage{array}
\usepackage{tabularx}
\usepackage{fibnum}[2016/05/16]
\usepackage{holtxdoc}[2011/11/22]
\usepackage{csquotes}
\usepackage[
  bibencoding=ascii,
  alldates=iso8601,
]{biblatex}[2011/11/13]
\bibliography{oberdiek-source}
\bibliography{fibnum}
\begin{document}
  \DocInput{fibnum.dtx}%
\end{document}
%</driver>
% \fi
%
%
% \CharacterTable
%  {Upper-case    \A\B\C\D\E\F\G\H\I\J\K\L\M\N\O\P\Q\R\S\T\U\V\W\X\Y\Z
%   Lower-case    \a\b\c\d\e\f\g\h\i\j\k\l\m\n\o\p\q\r\s\t\u\v\w\x\y\z
%   Digits        \0\1\2\3\4\5\6\7\8\9
%   Exclamation   \!     Double quote  \"     Hash (number) \#
%   Dollar        \$     Percent       \%     Ampersand     \&
%   Acute accent  \'     Left paren    \(     Right paren   \)
%   Asterisk      \*     Plus          \+     Comma         \,
%   Minus         \-     Point         \.     Solidus       \/
%   Colon         \:     Semicolon     \;     Less than     \<
%   Equals        \=     Greater than  \>     Question mark \?
%   Commercial at \@     Left bracket  \[     Backslash     \\
%   Right bracket \]     Circumflex    \^     Underscore    \_
%   Grave accent  \`     Left brace    \{     Vertical bar  \|
%   Right brace   \}     Tilde         \~}
%
% \GetFileInfo{fibnum.drv}
%
% \title{The \xpackage{fibnum} package}
% \date{2016/05/16 v1.1}
% \author{Heiko Oberdiek\thanks
% {Please report any issues at https://github.com/ho-tex/oberdiek/issues}\\
% \xemail{heiko.oberdiek at googlemail.com}}
%
% \maketitle
%
% \begin{abstract}
% The package \xpackage{fibnum} provides expandable fibonacci
% numbers for both \hologo{LaTeX} and \hologo{plainTeX}.
% \end{abstract}
%
% \tableofcontents
%
% \section{Documentation}
%
% In the mailing list \textsf{texhax} Jan Abraham asked the question,
% how to get Fibonacci numbers in \hologo{TeX} \cite{texhax:abraham}:
% \begin{quote}
% Write a Macro in \hologo{TeX} that compute the function |\fib{m}|
% All fibonacci numbers from 1 to $m$ ($m < 40$).
% \end{quote}
% This packages provides an expandable implementation for the
% calculation of these numbers for a much larger set of indexes.
% For practical reasons the index is restricted to the same limitations
% that apply for \hologo{TeX} integer numbers.
% The range of the Fibonacci numbers, however, are not limited
% by the algorithm. They are only restricted to memory limitations,
% if they are hit.
%
% The package is loaded as \hologo{LaTeX} package in \hologo{LaTeX}:
% \begin{quote}
%   |\usepackage{fibnum}|
% \end{quote}
% and as file in \hologo{plainTeX}:
% \begin{quote}
%   |\input fibnum.sty|
% \end{quote}
% The package does not know any options and it provides
% the macros \cs{fibnum} and \cs{fibnumPreCalc}.
%
% \begin{declcs}{fibnum} \M{index}
% \end{declcs}
% Macro \cs{fibnum} expects a \hologo{TeX} number as \meta{index}
% in the official \hologo{TeX} number range from $-(2^{31}-1)$ up to
% $2^{31}-1$. In exact two expansion steps the macro expands to
% the Fibnoacci number $F_{\text{\meta{index}}}$. In case of a negative
% \meta{index}, the ``negafibonacci'' number \cite{wikipedia:negafibonacci}
% is used. Formally the Fibonacci number $F_n$ with integer
% index~$n$, $n\in\mathbb{Z}$ and
% $n\in[\num{-2147483647},\num{2147483647}]$ that is returned by macro
% \cs{fibnum} with numerical argument $n$ is defined the following way:
% \begin{gather}
%   \label{eq:def}
%   F_n =
%   \begin{cases}
%     0 & \text{for $n=0$}\\
%     1 & \text{for $n=1$}\\
%     F_{n-1} + F_{n-2} & \text{for $n>1$}\\
%     (-1)^{n+1}F_n & \text{for $n<0$}
%   \end{cases}
% \end{gather}
% Examples:
% \begin{quote}
%   \makeatletter
%   \def\x#1{\cs{fibnum}\{#1\}&
%     \edef\X{\fibnum{#1}}\edef\Y{\expandafter\ltx@car\X\@nil}^^A
%     \if-\Y
%       \edef\X{\expandafter\ltx@cdr\X\@nil}^^A
%       \noindent
%       \llap{-}\X
%     \else
%       \X
%     \fi
%     \tabularnewline
%   }
%   \def\y{\multicolumn{1}{@{}c@{}}{$\vdots$}\tabularnewline}
%   \DeclareUrlCommand\UrlNum{^^A
%     \urlstyle{tt}^^A
%     \def\UrlBreaks{\do\0\do\1\do\2\do\3\do\4\do\5\do\6\do\7\do\8\do\9}^^A
%   }
%   \begin{tabularx}{\dimexpr\linewidth+5.7pt\relax}{@{}>{\ttfamily}l@{ $\rightarrow$ \hphantom{\ttfamily-}}>{\ttfamily}X@{}}
%     \x{-6}
%     \x{-5}
%     \x{-4}
%     \x{-3}
%     \x{-2}
%     \x{-1}
%     \x{0}
%     \x{1}
%     \x{2}
%     \x{3}
%     \x{4}
%     \x{5}
%     \x{6}
%     \y
%     \x{10}
%     \y
%     \x{46}
%     \y
%     \cs{fibnum}\{100\} & 354224848179261915075
%     \tabularnewline
%     \y
%     \cs{fibnum}\{200\} & 280571172992510140037611932413038677189525
%     \tabularnewline
%     \y
%     \cs{fibnum}\{1000\} &
%       \raggedright
%       \UrlNum{^^A
%         434665576869374564356885276750406258025646^^A
%         605173717804024817290895365554179490518904^^A
%         038798400792551692959225930803226347752096^^A
%         896232398733224711616429964409065331879382^^A
%         98969649928516003704476137795166849228875^^A
%       }
%     \tabularnewline
%   \end{tabularx}\kern-5.7pt\mbox{}
% \end{quote}
%
% \begin{declcs}{fibnumPreCalc} \M{index}
% \end{declcs}
% The package already provides precalculated Fibonacci numbers up to
% index~46. That means that calculations are not necessary for
% Fibonacci numbers that fit into the range of \hologo{TeX}
% numbers. Because macro \cs{fibnum} is expandable, it cannot
% store calculated Fibonacci numbers for later use. Macro definitions
% are forbidden in expandable contexts. If larger Fibonacci numbers
% are used more than once, than the compilation time can be shortened
% by calculating and storing the Fibonacci numbers beforehand.
% The argument \meta{index} is a \hologo{TeX} number and macro
% \cs{fibnumPreCalc} ensures that the Fibonacci numbers
% $F_0$ up to $F_{\lvert\text{\meta{index}}\rvert}$ that are not
% already known are calculated
% and stored in internal macros. Internally only non-negative
% Fibonacci numbers are stored. If \meta{index} is negative, then
% the needed positive Fibonacci numbers are calculated and stored.
% Example:
% \begin{quote}
%   \def\x#1{\begingroup\itshape\texttt{\%} #1\endgroup}
%   |\fibnumPreCalc{50}|\\
%   \x{calculates and stores the values for indexes 47..50.}\\
%   \x{(Values for 0..46 are already stored by the package.)}\\
%   |\fibnum{49}| \x{uses the stored value}\\
%   |\fibnum{51}|
%   \x{only calculates $F_{51}$ from stored values $F_{49}$ and $F_{50}$}\\
%   |\fibnumPreCalc{100}|\\
%   \x{calculates and stores the values for indexes 51..100}\\
%   |\fibnum{100}| \x{uses the stored value for $F_{100}$}\\
%   |\fibnum{-100}|
%   \x{uses the stored value for $F_{100}$}\\
%   \x{$F_{-100}=-F_{100}$ according to equation \eqref{eq:def}.}
% \end{quote}
%
% \StopEventually{
% }
%
% \section{Implementation}
%
% \subsection{Identification}
%
%    \begin{macrocode}
%<*package>
%    \end{macrocode}
%    Reload check, especially if the package is not used with \LaTeX.
%    \begin{macrocode}
\begingroup\catcode61\catcode48\catcode32=10\relax%
  \catcode13=5 % ^^M
  \endlinechar=13 %
  \catcode35=6 % #
  \catcode39=12 % '
  \catcode44=12 % ,
  \catcode45=12 % -
  \catcode46=12 % .
  \catcode58=12 % :
  \catcode64=11 % @
  \catcode123=1 % {
  \catcode125=2 % }
  \expandafter\let\expandafter\x\csname ver@fibnum.sty\endcsname
  \ifx\x\relax % plain-TeX, first loading
  \else
    \def\empty{}%
    \ifx\x\empty % LaTeX, first loading,
      % variable is initialized, but \ProvidesPackage not yet seen
    \else
      \expandafter\ifx\csname PackageInfo\endcsname\relax
        \def\x#1#2{%
          \immediate\write-1{Package #1 Info: #2.}%
        }%
      \else
        \def\x#1#2{\PackageInfo{#1}{#2, stopped}}%
      \fi
      \x{fibnum}{The package is already loaded}%
      \aftergroup\endinput
    \fi
  \fi
\endgroup%
%    \end{macrocode}
%    Package identification:
%    \begin{macrocode}
\begingroup\catcode61\catcode48\catcode32=10\relax%
  \catcode13=5 % ^^M
  \endlinechar=13 %
  \catcode35=6 % #
  \catcode39=12 % '
  \catcode40=12 % (
  \catcode41=12 % )
  \catcode44=12 % ,
  \catcode45=12 % -
  \catcode46=12 % .
  \catcode47=12 % /
  \catcode58=12 % :
  \catcode64=11 % @
  \catcode91=12 % [
  \catcode93=12 % ]
  \catcode123=1 % {
  \catcode125=2 % }
  \expandafter\ifx\csname ProvidesPackage\endcsname\relax
    \def\x#1#2#3[#4]{\endgroup
      \immediate\write-1{Package: #3 #4}%
      \xdef#1{#4}%
    }%
  \else
    \def\x#1#2[#3]{\endgroup
      #2[{#3}]%
      \ifx#1\@undefined
        \xdef#1{#3}%
      \fi
      \ifx#1\relax
        \xdef#1{#3}%
      \fi
    }%
  \fi
\expandafter\x\csname ver@fibnum.sty\endcsname
\ProvidesPackage{fibnum}%
  [2016/05/16 v1.1 Fibonacci numbers (HO)]%
%    \end{macrocode}
%
%    \begin{macrocode}
\begingroup\catcode61\catcode48\catcode32=10\relax%
  \catcode13=5 % ^^M
  \endlinechar=13 %
  \catcode123=1 % {
  \catcode125=2 % }
  \catcode64=11 % @
  \def\x{\endgroup
    \expandafter\edef\csname FibNum@AtEnd\endcsname{%
      \endlinechar=\the\endlinechar\relax
      \catcode13=\the\catcode13\relax
      \catcode32=\the\catcode32\relax
      \catcode35=\the\catcode35\relax
      \catcode61=\the\catcode61\relax
      \catcode64=\the\catcode64\relax
      \catcode123=\the\catcode123\relax
      \catcode125=\the\catcode125\relax
    }%
  }%
\x\catcode61\catcode48\catcode32=10\relax%
\catcode13=5 % ^^M
\endlinechar=13 %
\catcode35=6 % #
\catcode64=11 % @
\catcode123=1 % {
\catcode125=2 % }
\def\TMP@EnsureCode#1#2{%
  \edef\FibNum@AtEnd{%
    \FibNum@AtEnd
    \catcode#1=\the\catcode#1\relax
  }%
  \catcode#1=#2\relax
}
\TMP@EnsureCode{33}{12}% !
%\TMP@EnsureCode{36}{3}% $
%\TMP@EnsureCode{38}{4}% &
\TMP@EnsureCode{40}{12}% (
\TMP@EnsureCode{41}{12}% )
\TMP@EnsureCode{45}{12}% -
\TMP@EnsureCode{46}{12}% .
\TMP@EnsureCode{47}{12}% /
\TMP@EnsureCode{58}{12}% :
\TMP@EnsureCode{60}{12}% <
\TMP@EnsureCode{62}{12}% >
\TMP@EnsureCode{91}{12}% [
%\TMP@EnsureCode{96}{12}% `
\TMP@EnsureCode{93}{12}% ]
%\TMP@EnsureCode{94}{12}% ^ (superscript) (!)
%\TMP@EnsureCode{124}{12}% |
\edef\FibNum@AtEnd{\FibNum@AtEnd\noexpand\endinput}
%    \end{macrocode}
%
% \subsection{Package resources}
%
%    \begin{macrocode}
\begingroup\expandafter\expandafter\expandafter\endgroup
\expandafter\ifx\csname RequirePackage\endcsname\relax
  \def\TMP@RequirePackage#1[#2]{%
    \begingroup\expandafter\expandafter\expandafter\endgroup
    \expandafter\ifx\csname ver@#1.sty\endcsname\relax
      \input #1.sty\relax
    \fi
  }%
  \TMP@RequirePackage{ltxcmds}[2011/04/18]%
  \TMP@RequirePackage{intcalc}[2007/09/27]%
  \TMP@RequirePackage{bigintcalc}[2007/11/11]%
\else
  \RequirePackage{ltxcmds}[2011/04/18]%
  \RequirePackage{intcalc}[2007/09/27]%
  \RequirePackage{bigintcalc}[2007/11/11]%
\fi
%    \end{macrocode}
%
% \subsection{Setup precalculated values}
%
%    \begin{macrocode}
\def\FibNum@temp#1{%
  \expandafter\def\csname FibNum@#1\endcsname
}
\catcode46=9 % dots are ignored
\FibNum@temp{0}{0}
\FibNum@temp{1}{1}
\FibNum@temp{2}{1}
\FibNum@temp{3}{2}
\FibNum@temp{4}{3}
\FibNum@temp{5}{5}
\FibNum@temp{6}{8}
\FibNum@temp{7}{13}
\FibNum@temp{8}{21}
\FibNum@temp{9}{34}
\FibNum@temp{10}{55}
\FibNum@temp{11}{89}
\FibNum@temp{12}{144}
\FibNum@temp{13}{233}
\FibNum@temp{14}{377}
\FibNum@temp{15}{610}
\FibNum@temp{16}{987}
\FibNum@temp{17}{1.597}
\FibNum@temp{18}{2.584}
\FibNum@temp{19}{4.181}
\FibNum@temp{20}{6.765}
\FibNum@temp{21}{10.946}
\FibNum@temp{22}{17.711}
\FibNum@temp{23}{28.657}
\FibNum@temp{24}{46.368}
\FibNum@temp{25}{75.025}
\FibNum@temp{26}{121.393}
\FibNum@temp{27}{196.418}
\FibNum@temp{28}{317.811}
\FibNum@temp{29}{514.229}
\FibNum@temp{30}{832.040}
\FibNum@temp{31}{1.346.269}
\FibNum@temp{32}{2.178.309}
\FibNum@temp{33}{3.524.578}
\FibNum@temp{34}{5.702.887}
\FibNum@temp{35}{9.227.465}
\FibNum@temp{36}{14.930.352}
\FibNum@temp{37}{24.157.817}
\FibNum@temp{38}{39.088.169}
\FibNum@temp{39}{63.245.986}
\FibNum@temp{40}{102.334.155}
\FibNum@temp{41}{165.580.141}
\FibNum@temp{42}{267.914.296}
\FibNum@temp{43}{433.494.437}
\FibNum@temp{44}{701.408.733}
\FibNum@temp{45}{1.134.903.170}
\FibNum@temp{46}{1.836.311.903}
%    \end{macrocode}
%    \begin{macro}{\FibNum@max}
%    \begin{macrocode}
\def\FibNum@max{46}
%    \end{macrocode}
%    \end{macro}
%
% \subsection{Macros for precalculating values}
%
%    \begin{macro}{\fibnumPreCalc}
%    \begin{macrocode}
\def\fibnumPreCalc#1{%
  \expandafter\expandafter\expandafter
  \FibNum@PreCalc\intcalcNum{#1}/%
}
%    \end{macrocode}
%    \end{macro}
%    \begin{macro}{\FibNum@PreCalc}
%    \begin{macrocode}
\def\FibNum@PreCalc#1/{%
  \ifnum#1<\ltx@zero
    \expandafter\FibNum@PreCalc\ltx@gobble#1/%
  \else
    \ifnum#1>\FibNum@max
      \begingroup
        \ltx@LocDimenA=#1sp\relax
        \countdef\FibNum@i=255\relax
        \FibNum@i=\FibNum@max\relax
        \edef\FibNum@temp{%
          \csname FibNum@\the\FibNum@i\endcsname/%
        }%
        \advance\FibNum@i by -1\relax
        \edef\FibNum@temp{%
          \FibNum@temp
          \csname FibNum@\the\FibNum@i\endcsname
        }%
        \advance\FibNum@i\ltx@two
        \iftrue
          \expandafter\FibNum@PreCalcAux\FibNum@temp
        \fi
      \endgroup
    \fi
  \fi
}
%    \end{macrocode}
%    \end{macro}
%    \begin{macro}{\FibNum@PreCalcAux}
%    \begin{macrocode}
\def\FibNum@PreCalcAux#1/#2\fi{%
  \fi
  \edef\FibNum@temp{\BigIntCalcAdd#1!#2!}%
  \global\expandafter
  \let\csname FibNum@\the\FibNum@i\endcsname\FibNum@temp
  \ifnum\FibNum@i=\ltx@LocDimenA
    \xdef\FibNum@max{\the\FibNum@i}%
  \else
    \advance\FibNum@i\ltx@one
    \expandafter\FibNum@PreCalcAux\FibNum@temp/#1%
  \fi
}
%    \end{macrocode}
%    \end{macro}
%
% \subsection{Expandable calculations}
%
%    \begin{macro}{\fibnum}
%    \begin{macrocode}
\def\fibnum#1{%
  \romannumeral
  \expandafter\expandafter\expandafter\FibNum@Do\intcalcNum{#1}/%
}
%    \end{macrocode}
%    \end{macro}
%    \begin{macro}{\FibNum@Do}
%    \begin{macrocode}
\def\FibNum@Do#1/{%
  \ifnum#1<\ltx@zero
    \FibNum@ReturnAfterElseFiFi{%
      \ifodd#1 %
        \expandafter\expandafter\expandafter\ltx@zero
      \else
        \expandafter\expandafter\expandafter\ltx@zero
        \expandafter\expandafter\expandafter-%
      \fi
      \romannumeral
      \expandafter\FibNum@Do\ltx@gobble#1/%
    }%
  \else
    \ifnum\FibNum@max<#1 %
      \ltx@ReturnAfterElseFi{%
        \expandafter
        \FibNum@ExpCalc\number\expandafter\IntCalcInc\FibNum@max!%
        \expandafter\expandafter\expandafter/%
        \csname FibNum@\FibNum@max
        \expandafter\expandafter\expandafter\endcsname
        \expandafter\expandafter\expandafter/%
        \csname FibNum@\expandafter\IntCalcDec\FibNum@max!%
        \endcsname/%
        #1%
      }%
    \else
      \expandafter\expandafter\expandafter\ltx@zero
      \csname FibNum@#1\expandafter\expandafter\expandafter\endcsname
    \fi
  \fi
}
%    \end{macrocode}
%    \end{macro}
%    \begin{macro}{\FibNum@ReturnAfterElseFiFi}
%    \begin{macrocode}
\def\FibNum@ReturnAfterElseFiFi#1\else#2\fi\fi{\fi#1}
%    \end{macrocode}
%    \end{macro}
%    \begin{macro}{\FibNum@ExpCalc}
%    \begin{macrocode}
\def\FibNum@ExpCalc#1/#2/#3/#4\fi{%
  \fi
  \ifnum#1=#4 %
    \ltx@ReturnAfterElseFi{%
      \expandafter\expandafter\expandafter\ltx@zero
      \BigIntCalcAdd#2!#3!%
    }%
  \else
    \expandafter\FibNum@ExpCalc
    \number\IntCalcInc#1!%
    \expandafter\expandafter\expandafter/%
    \BigIntCalcAdd#2!#3!/%
    #2/#4%
  \fi
}
%    \end{macrocode}
%    \end{macro}
%
%    \begin{macrocode}
\FibNum@AtEnd%
%</package>
%    \end{macrocode}
%
% \section{Test}
%
% \subsection{Catcode checks for loading}
%
%    \begin{macrocode}
%<*test1>
%    \end{macrocode}
%    \begin{macrocode}
\catcode`\{=1 %
\catcode`\}=2 %
\catcode`\#=6 %
\catcode`\@=11 %
\expandafter\ifx\csname count@\endcsname\relax
  \countdef\count@=255 %
\fi
\expandafter\ifx\csname @gobble\endcsname\relax
  \long\def\@gobble#1{}%
\fi
\expandafter\ifx\csname @firstofone\endcsname\relax
  \long\def\@firstofone#1{#1}%
\fi
\expandafter\ifx\csname loop\endcsname\relax
  \expandafter\@firstofone
\else
  \expandafter\@gobble
\fi
{%
  \def\loop#1\repeat{%
    \def\body{#1}%
    \iterate
  }%
  \def\iterate{%
    \body
      \let\next\iterate
    \else
      \let\next\relax
    \fi
    \next
  }%
  \let\repeat=\fi
}%
\def\RestoreCatcodes{}
\count@=0 %
\loop
  \edef\RestoreCatcodes{%
    \RestoreCatcodes
    \catcode\the\count@=\the\catcode\count@\relax
  }%
\ifnum\count@<255 %
  \advance\count@ 1 %
\repeat

\def\RangeCatcodeInvalid#1#2{%
  \count@=#1\relax
  \loop
    \catcode\count@=15 %
  \ifnum\count@<#2\relax
    \advance\count@ 1 %
  \repeat
}
\def\RangeCatcodeCheck#1#2#3{%
  \count@=#1\relax
  \loop
    \ifnum#3=\catcode\count@
    \else
      \errmessage{%
        Character \the\count@\space
        with wrong catcode \the\catcode\count@\space
        instead of \number#3%
      }%
    \fi
  \ifnum\count@<#2\relax
    \advance\count@ 1 %
  \repeat
}
\def\space{ }
\expandafter\ifx\csname LoadCommand\endcsname\relax
  \def\LoadCommand{\input fibnum.sty\relax}%
\fi
\def\Test{%
  \RangeCatcodeInvalid{0}{47}%
  \RangeCatcodeInvalid{58}{64}%
  \RangeCatcodeInvalid{91}{96}%
  \RangeCatcodeInvalid{123}{255}%
  \catcode`\@=12 %
  \catcode`\\=0 %
  \catcode`\%=14 %
  \LoadCommand
  \RangeCatcodeCheck{0}{36}{15}%
  \RangeCatcodeCheck{37}{37}{14}%
  \RangeCatcodeCheck{38}{47}{15}%
  \RangeCatcodeCheck{48}{57}{12}%
  \RangeCatcodeCheck{58}{63}{15}%
  \RangeCatcodeCheck{64}{64}{12}%
  \RangeCatcodeCheck{65}{90}{11}%
  \RangeCatcodeCheck{91}{91}{15}%
  \RangeCatcodeCheck{92}{92}{0}%
  \RangeCatcodeCheck{93}{96}{15}%
  \RangeCatcodeCheck{97}{122}{11}%
  \RangeCatcodeCheck{123}{255}{15}%
  \RestoreCatcodes
}
\Test
\csname @@end\endcsname
\end
%    \end{macrocode}
%    \begin{macrocode}
%</test1>
%    \end{macrocode}
%
% \subsection{Test calculations}
%
%    \begin{macrocode}
%<*test-calc>
\catcode`\{=1 %
\catcode`\}=2 %
\catcode`\#=6 %
\catcode`\@=11 %
\begingroup\expandafter\expandafter\expandafter\endgroup
\expandafter\ifx\csname RequirePackage\endcsname\relax
  \input fibnum.sty\relax
\else
  \RequirePackage{fibnum}[2016/05/16]%
\fi
\def\TestSet{%
  \test{0}{0}%
  \test{1}{1}%
  \test{2}{1}%
  \test{3}{2}%
  \test{4}{3}%
  \test{5}{5}%
  \test{6}{8}%
  \test{7}{13}%
  \test{8}{21}%
  \test{9}{34}%
  \test{10}{55}%
  \test{11}{89}%
  \test{12}{144}%
  \test{13}{233}%
  \test{14}{377}%
  \test{15}{610}%
  \test{16}{987}%
  \test{17}{1597}%
  \test{18}{2584}%
  \test{19}{4181}%
  \test{20}{6765}%
  \test{21}{10946}%
  \test{22}{17711}%
  \test{23}{28657}%
  \test{24}{46368}%
  \test{25}{75025}%
  \test{26}{121393}%
  \test{27}{196418}%
  \test{28}{317811}%
  \test{29}{514229}%
  \test{30}{832040}%
  \test{31}{1346269}%
  \test{32}{2178309}%
  \test{33}{3524578}%
  \test{34}{5702887}%
  \test{35}{9227465}%
  \test{36}{14930352}%
  \test{37}{24157817}%
  \test{38}{39088169}%
  \test{39}{63245986}%
  \test{40}{102334155}%
  \test{41}{165580141}%
  \test{42}{267914296}%
  \test{43}{433494437}%
  \test{44}{701408733}%
  \test{45}{1134903170}%
  \test{46}{1836311903}%
  \test{47}{2971215073}%
  \test{48}{4807526976}%
  \test{49}{7778742049}%
  \test{50}{12586269025}%
  \test{51}{20365011074}%
  \test{52}{32951280099}%
  \test{53}{53316291173}%
  \test{54}{86267571272}%
  \test{55}{139583862445}%
  \test{56}{225851433717}%
  \test{57}{365435296162}%
  \test{58}{591286729879}%
  \test{59}{956722026041}%
  \test{60}{1548008755920}%
  \test{61}{2504730781961}%
  \test{62}{4052739537881}%
  \test{63}{6557470319842}%
  \test{64}{10610209857723}%
  \test{65}{17167680177565}%
  \test{66}{27777890035288}%
  \test{67}{44945570212853}%
  \test{68}{72723460248141}%
  \test{69}{117669030460994}%
  \test{70}{190392490709135}%
  \test{71}{308061521170129}%
  \test{72}{498454011879264}%
  \test{73}{806515533049393}%
}
\def\msg#{\immediate\write16}
\def\test#1#2{%
  \TestAux{#1}{#2}%
  \ifnum#1=0 %
  \else
    \ifodd#1 %
      \TestAux{-#1}{#2}%
    \else
      \TestAux{-#1}{-#2}%
    \fi
  \fi
}
\def\TestAux#1#2{%
  \def\Expected{#2}%
  \expandafter\expandafter\expandafter\def
  \expandafter\expandafter\expandafter\Result
  \expandafter\expandafter\expandafter{%
    \fibnum{#1}%
  }%
  \ltx@onelevel@sanitize\Result
  \ifx\Result\Expected
    \msg{* #1: ok.}%
  \else
    \msg{! fib(#1) = #2}%
    \errmessage{fib(#1) <> \Result}%
  \fi
}
\TestSet
\setbox0=\hbox{%
  \msg{* PreCalc{73}}%
  \fibnumPreCalc{73}%
}
\ifdim\wd0=0pt
\else
  \errmessage{Unwanted stuff in PreCalc}%
\fi
\TestSet
\csname @@end\endcsname\end
%</test-calc>
%    \end{macrocode}
%
% \section{Installation}
%
% \subsection{Download}
%
% \paragraph{Package.} This package is available on
% CTAN\footnote{\url{http://ctan.org/pkg/fibnum}}:
% \begin{description}
% \item[\CTAN{macros/latex/contrib/oberdiek/fibnum.dtx}] The source file.
% \item[\CTAN{macros/latex/contrib/oberdiek/fibnum.pdf}] Documentation.
% \end{description}
%
%
% \paragraph{Bundle.} All the packages of the bundle `oberdiek'
% are also available in a TDS compliant ZIP archive. There
% the packages are already unpacked and the documentation files
% are generated. The files and directories obey the TDS standard.
% \begin{description}
% \item[\CTAN{install/macros/latex/contrib/oberdiek.tds.zip}]
% \end{description}
% \emph{TDS} refers to the standard ``A Directory Structure
% for \TeX\ Files'' (\CTAN{tds/tds.pdf}). Directories
% with \xfile{texmf} in their name are usually organized this way.
%
% \subsection{Bundle installation}
%
% \paragraph{Unpacking.} Unpack the \xfile{oberdiek.tds.zip} in the
% TDS tree (also known as \xfile{texmf} tree) of your choice.
% Example (linux):
% \begin{quote}
%   |unzip oberdiek.tds.zip -d ~/texmf|
% \end{quote}
%
% \paragraph{Script installation.}
% Check the directory \xfile{TDS:scripts/oberdiek/} for
% scripts that need further installation steps.
% Package \xpackage{attachfile2} comes with the Perl script
% \xfile{pdfatfi.pl} that should be installed in such a way
% that it can be called as \texttt{pdfatfi}.
% Example (linux):
% \begin{quote}
%   |chmod +x scripts/oberdiek/pdfatfi.pl|\\
%   |cp scripts/oberdiek/pdfatfi.pl /usr/local/bin/|
% \end{quote}
%
% \subsection{Package installation}
%
% \paragraph{Unpacking.} The \xfile{.dtx} file is a self-extracting
% \docstrip\ archive. The files are extracted by running the
% \xfile{.dtx} through \plainTeX:
% \begin{quote}
%   \verb|tex fibnum.dtx|
% \end{quote}
%
% \paragraph{TDS.} Now the different files must be moved into
% the different directories in your installation TDS tree
% (also known as \xfile{texmf} tree):
% \begin{quote}
% \def\t{^^A
% \begin{tabular}{@{}>{\ttfamily}l@{ $\rightarrow$ }>{\ttfamily}l@{}}
%   fibnum.sty & tex/generic/oberdiek/fibnum.sty\\
%   fibnum.pdf & doc/latex/oberdiek/fibnum.pdf\\
%   test/fibnum-test1.tex & doc/latex/oberdiek/test/fibnum-test1.tex\\
%   test/fibnum-test-calc.tex & doc/latex/oberdiek/test/fibnum-test-calc.tex\\
%   fibnum.dtx & source/latex/oberdiek/fibnum.dtx\\
% \end{tabular}^^A
% }^^A
% \sbox0{\t}^^A
% \ifdim\wd0>\linewidth
%   \begingroup
%     \advance\linewidth by\leftmargin
%     \advance\linewidth by\rightmargin
%   \edef\x{\endgroup
%     \def\noexpand\lw{\the\linewidth}^^A
%   }\x
%   \def\lwbox{^^A
%     \leavevmode
%     \hbox to \linewidth{^^A
%       \kern-\leftmargin\relax
%       \hss
%       \usebox0
%       \hss
%       \kern-\rightmargin\relax
%     }^^A
%   }^^A
%   \ifdim\wd0>\lw
%     \sbox0{\small\t}^^A
%     \ifdim\wd0>\linewidth
%       \ifdim\wd0>\lw
%         \sbox0{\footnotesize\t}^^A
%         \ifdim\wd0>\linewidth
%           \ifdim\wd0>\lw
%             \sbox0{\scriptsize\t}^^A
%             \ifdim\wd0>\linewidth
%               \ifdim\wd0>\lw
%                 \sbox0{\tiny\t}^^A
%                 \ifdim\wd0>\linewidth
%                   \lwbox
%                 \else
%                   \usebox0
%                 \fi
%               \else
%                 \lwbox
%               \fi
%             \else
%               \usebox0
%             \fi
%           \else
%             \lwbox
%           \fi
%         \else
%           \usebox0
%         \fi
%       \else
%         \lwbox
%       \fi
%     \else
%       \usebox0
%     \fi
%   \else
%     \lwbox
%   \fi
% \else
%   \usebox0
% \fi
% \end{quote}
% If you have a \xfile{docstrip.cfg} that configures and enables \docstrip's
% TDS installing feature, then some files can already be in the right
% place, see the documentation of \docstrip.
%
% \subsection{Refresh file name databases}
%
% If your \TeX~distribution
% (\teTeX, \mikTeX, \dots) relies on file name databases, you must refresh
% these. For example, \teTeX\ users run \verb|texhash| or
% \verb|mktexlsr|.
%
% \subsection{Some details for the interested}
%
% \paragraph{Attached source.}
%
% The PDF documentation on CTAN also includes the
% \xfile{.dtx} source file. It can be extracted by
% AcrobatReader 6 or higher. Another option is \textsf{pdftk},
% e.g. unpack the file into the current directory:
% \begin{quote}
%   \verb|pdftk fibnum.pdf unpack_files output .|
% \end{quote}
%
% \paragraph{Unpacking with \LaTeX.}
% The \xfile{.dtx} chooses its action depending on the format:
% \begin{description}
% \item[\plainTeX:] Run \docstrip\ and extract the files.
% \item[\LaTeX:] Generate the documentation.
% \end{description}
% If you insist on using \LaTeX\ for \docstrip\ (really,
% \docstrip\ does not need \LaTeX), then inform the autodetect routine
% about your intention:
% \begin{quote}
%   \verb|latex \let\install=y\input{fibnum.dtx}|
% \end{quote}
% Do not forget to quote the argument according to the demands
% of your shell.
%
% \paragraph{Generating the documentation.}
% You can use both the \xfile{.dtx} or the \xfile{.drv} to generate
% the documentation. The process can be configured by the
% configuration file \xfile{ltxdoc.cfg}. For instance, put this
% line into this file, if you want to have A4 as paper format:
% \begin{quote}
%   \verb|\PassOptionsToClass{a4paper}{article}|
% \end{quote}
% An example follows how to generate the
% documentation with pdf\LaTeX:
% \begin{quote}
%\begin{verbatim}
%pdflatex fibnum.dtx
%bibtex fibnum.aux
%makeindex -s gind.ist fibnum.idx
%pdflatex fibnum.dtx
%makeindex -s gind.ist fibnum.idx
%pdflatex fibnum.dtx
%\end{verbatim}
% \end{quote}
%
% \printbibliography[
%   heading=bibnumbered,
% ]
%
% \begin{History}
%   \begin{Version}{2012/04/08 v1.0}
%   \item
%     First version.
%   \end{Version}
%   \begin{Version}{2016/05/16 v1.1}
%   \item
%     Documentation updates.
%   \end{Version}
% \end{History}
%
% \PrintIndex
%
% \Finale
\endinput
|
% \end{quote}
% Do not forget to quote the argument according to the demands
% of your shell.
%
% \paragraph{Generating the documentation.}
% You can use both the \xfile{.dtx} or the \xfile{.drv} to generate
% the documentation. The process can be configured by the
% configuration file \xfile{ltxdoc.cfg}. For instance, put this
% line into this file, if you want to have A4 as paper format:
% \begin{quote}
%   \verb|\PassOptionsToClass{a4paper}{article}|
% \end{quote}
% An example follows how to generate the
% documentation with pdf\LaTeX:
% \begin{quote}
%\begin{verbatim}
%pdflatex fibnum.dtx
%bibtex fibnum.aux
%makeindex -s gind.ist fibnum.idx
%pdflatex fibnum.dtx
%makeindex -s gind.ist fibnum.idx
%pdflatex fibnum.dtx
%\end{verbatim}
% \end{quote}
%
% \printbibliography[
%   heading=bibnumbered,
% ]
%
% \begin{History}
%   \begin{Version}{2012/04/08 v1.0}
%   \item
%     First version.
%   \end{Version}
%   \begin{Version}{2016/05/16 v1.1}
%   \item
%     Documentation updates.
%   \end{Version}
% \end{History}
%
% \PrintIndex
%
% \Finale
\endinput
|
% \end{quote}
% Do not forget to quote the argument according to the demands
% of your shell.
%
% \paragraph{Generating the documentation.}
% You can use both the \xfile{.dtx} or the \xfile{.drv} to generate
% the documentation. The process can be configured by the
% configuration file \xfile{ltxdoc.cfg}. For instance, put this
% line into this file, if you want to have A4 as paper format:
% \begin{quote}
%   \verb|\PassOptionsToClass{a4paper}{article}|
% \end{quote}
% An example follows how to generate the
% documentation with pdf\LaTeX:
% \begin{quote}
%\begin{verbatim}
%pdflatex fibnum.dtx
%bibtex fibnum.aux
%makeindex -s gind.ist fibnum.idx
%pdflatex fibnum.dtx
%makeindex -s gind.ist fibnum.idx
%pdflatex fibnum.dtx
%\end{verbatim}
% \end{quote}
%
% \printbibliography[
%   heading=bibnumbered,
% ]
%
% \begin{History}
%   \begin{Version}{2012/04/08 v1.0}
%   \item
%     First version.
%   \end{Version}
%   \begin{Version}{2016/05/16 v1.1}
%   \item
%     Documentation updates.
%   \end{Version}
% \end{History}
%
% \PrintIndex
%
% \Finale
\endinput
|
% \end{quote}
% Do not forget to quote the argument according to the demands
% of your shell.
%
% \paragraph{Generating the documentation.}
% You can use both the \xfile{.dtx} or the \xfile{.drv} to generate
% the documentation. The process can be configured by the
% configuration file \xfile{ltxdoc.cfg}. For instance, put this
% line into this file, if you want to have A4 as paper format:
% \begin{quote}
%   \verb|\PassOptionsToClass{a4paper}{article}|
% \end{quote}
% An example follows how to generate the
% documentation with pdf\LaTeX:
% \begin{quote}
%\begin{verbatim}
%pdflatex fibnum.dtx
%bibtex fibnum.aux
%makeindex -s gind.ist fibnum.idx
%pdflatex fibnum.dtx
%makeindex -s gind.ist fibnum.idx
%pdflatex fibnum.dtx
%\end{verbatim}
% \end{quote}
%
% \printbibliography[
%   heading=bibnumbered,
% ]
%
% \begin{History}
%   \begin{Version}{2012/04/08 v1.0}
%   \item
%     First version.
%   \end{Version}
%   \begin{Version}{2016/05/16 v1.1}
%   \item
%     Documentation updates.
%   \end{Version}
% \end{History}
%
% \PrintIndex
%
% \Finale
\endinput
