% \iffalse meta-comment
%
% File: kvdefinekeys.dtx
% Version: 2016/05/16 v1.4
% Info: Define keys
%
% Copyright (C) 2010, 2011 by
%    Heiko Oberdiek <heiko.oberdiek at googlemail.com>
%    2016
%    https://github.com/ho-tex/oberdiek/issues
%
% This work may be distributed and/or modified under the
% conditions of the LaTeX Project Public License, either
% version 1.3c of this license or (at your option) any later
% version. This version of this license is in
%    http://www.latex-project.org/lppl/lppl-1-3c.txt
% and the latest version of this license is in
%    http://www.latex-project.org/lppl.txt
% and version 1.3 or later is part of all distributions of
% LaTeX version 2005/12/01 or later.
%
% This work has the LPPL maintenance status "maintained".
%
% This Current Maintainer of this work is Heiko Oberdiek.
%
% The Base Interpreter refers to any `TeX-Format',
% because some files are installed in TDS:tex/generic//.
%
% This work consists of the main source file kvdefinekeys.dtx
% and the derived files
%    kvdefinekeys.sty, kvdefinekeys.pdf, kvdefinekeys.ins, kvdefinekeys.drv,
%    kvdefinekeys-test1.tex.
%
% Distribution:
%    CTAN:macros/latex/contrib/oberdiek/kvdefinekeys.dtx
%    CTAN:macros/latex/contrib/oberdiek/kvdefinekeys.pdf
%
% Unpacking:
%    (a) If kvdefinekeys.ins is present:
%           tex kvdefinekeys.ins
%    (b) Without kvdefinekeys.ins:
%           tex kvdefinekeys.dtx
%    (c) If you insist on using LaTeX
%           latex \let\install=y% \iffalse meta-comment
%
% File: kvdefinekeys.dtx
% Version: 2016/05/16 v1.4
% Info: Define keys
%
% Copyright (C) 2010, 2011 by
%    Heiko Oberdiek <heiko.oberdiek at googlemail.com>
%    2016
%    https://github.com/ho-tex/oberdiek/issues
%
% This work may be distributed and/or modified under the
% conditions of the LaTeX Project Public License, either
% version 1.3c of this license or (at your option) any later
% version. This version of this license is in
%    http://www.latex-project.org/lppl/lppl-1-3c.txt
% and the latest version of this license is in
%    http://www.latex-project.org/lppl.txt
% and version 1.3 or later is part of all distributions of
% LaTeX version 2005/12/01 or later.
%
% This work has the LPPL maintenance status "maintained".
%
% This Current Maintainer of this work is Heiko Oberdiek.
%
% The Base Interpreter refers to any `TeX-Format',
% because some files are installed in TDS:tex/generic//.
%
% This work consists of the main source file kvdefinekeys.dtx
% and the derived files
%    kvdefinekeys.sty, kvdefinekeys.pdf, kvdefinekeys.ins, kvdefinekeys.drv,
%    kvdefinekeys-test1.tex.
%
% Distribution:
%    CTAN:macros/latex/contrib/oberdiek/kvdefinekeys.dtx
%    CTAN:macros/latex/contrib/oberdiek/kvdefinekeys.pdf
%
% Unpacking:
%    (a) If kvdefinekeys.ins is present:
%           tex kvdefinekeys.ins
%    (b) Without kvdefinekeys.ins:
%           tex kvdefinekeys.dtx
%    (c) If you insist on using LaTeX
%           latex \let\install=y% \iffalse meta-comment
%
% File: kvdefinekeys.dtx
% Version: 2016/05/16 v1.4
% Info: Define keys
%
% Copyright (C) 2010, 2011 by
%    Heiko Oberdiek <heiko.oberdiek at googlemail.com>
%    2016
%    https://github.com/ho-tex/oberdiek/issues
%
% This work may be distributed and/or modified under the
% conditions of the LaTeX Project Public License, either
% version 1.3c of this license or (at your option) any later
% version. This version of this license is in
%    http://www.latex-project.org/lppl/lppl-1-3c.txt
% and the latest version of this license is in
%    http://www.latex-project.org/lppl.txt
% and version 1.3 or later is part of all distributions of
% LaTeX version 2005/12/01 or later.
%
% This work has the LPPL maintenance status "maintained".
%
% This Current Maintainer of this work is Heiko Oberdiek.
%
% The Base Interpreter refers to any `TeX-Format',
% because some files are installed in TDS:tex/generic//.
%
% This work consists of the main source file kvdefinekeys.dtx
% and the derived files
%    kvdefinekeys.sty, kvdefinekeys.pdf, kvdefinekeys.ins, kvdefinekeys.drv,
%    kvdefinekeys-test1.tex.
%
% Distribution:
%    CTAN:macros/latex/contrib/oberdiek/kvdefinekeys.dtx
%    CTAN:macros/latex/contrib/oberdiek/kvdefinekeys.pdf
%
% Unpacking:
%    (a) If kvdefinekeys.ins is present:
%           tex kvdefinekeys.ins
%    (b) Without kvdefinekeys.ins:
%           tex kvdefinekeys.dtx
%    (c) If you insist on using LaTeX
%           latex \let\install=y% \iffalse meta-comment
%
% File: kvdefinekeys.dtx
% Version: 2016/05/16 v1.4
% Info: Define keys
%
% Copyright (C) 2010, 2011 by
%    Heiko Oberdiek <heiko.oberdiek at googlemail.com>
%    2016
%    https://github.com/ho-tex/oberdiek/issues
%
% This work may be distributed and/or modified under the
% conditions of the LaTeX Project Public License, either
% version 1.3c of this license or (at your option) any later
% version. This version of this license is in
%    http://www.latex-project.org/lppl/lppl-1-3c.txt
% and the latest version of this license is in
%    http://www.latex-project.org/lppl.txt
% and version 1.3 or later is part of all distributions of
% LaTeX version 2005/12/01 or later.
%
% This work has the LPPL maintenance status "maintained".
%
% This Current Maintainer of this work is Heiko Oberdiek.
%
% The Base Interpreter refers to any `TeX-Format',
% because some files are installed in TDS:tex/generic//.
%
% This work consists of the main source file kvdefinekeys.dtx
% and the derived files
%    kvdefinekeys.sty, kvdefinekeys.pdf, kvdefinekeys.ins, kvdefinekeys.drv,
%    kvdefinekeys-test1.tex.
%
% Distribution:
%    CTAN:macros/latex/contrib/oberdiek/kvdefinekeys.dtx
%    CTAN:macros/latex/contrib/oberdiek/kvdefinekeys.pdf
%
% Unpacking:
%    (a) If kvdefinekeys.ins is present:
%           tex kvdefinekeys.ins
%    (b) Without kvdefinekeys.ins:
%           tex kvdefinekeys.dtx
%    (c) If you insist on using LaTeX
%           latex \let\install=y\input{kvdefinekeys.dtx}
%        (quote the arguments according to the demands of your shell)
%
% Documentation:
%    (a) If kvdefinekeys.drv is present:
%           latex kvdefinekeys.drv
%    (b) Without kvdefinekeys.drv:
%           latex kvdefinekeys.dtx; ...
%    The class ltxdoc loads the configuration file ltxdoc.cfg
%    if available. Here you can specify further options, e.g.
%    use A4 as paper format:
%       \PassOptionsToClass{a4paper}{article}
%
%    Programm calls to get the documentation (example):
%       pdflatex kvdefinekeys.dtx
%       makeindex -s gind.ist kvdefinekeys.idx
%       pdflatex kvdefinekeys.dtx
%       makeindex -s gind.ist kvdefinekeys.idx
%       pdflatex kvdefinekeys.dtx
%
% Installation:
%    TDS:tex/generic/oberdiek/kvdefinekeys.sty
%    TDS:doc/latex/oberdiek/kvdefinekeys.pdf
%    TDS:doc/latex/oberdiek/test/kvdefinekeys-test1.tex
%    TDS:source/latex/oberdiek/kvdefinekeys.dtx
%
%<*ignore>
\begingroup
  \catcode123=1 %
  \catcode125=2 %
  \def\x{LaTeX2e}%
\expandafter\endgroup
\ifcase 0\ifx\install y1\fi\expandafter
         \ifx\csname processbatchFile\endcsname\relax\else1\fi
         \ifx\fmtname\x\else 1\fi\relax
\else\csname fi\endcsname
%</ignore>
%<*install>
\input docstrip.tex
\Msg{************************************************************************}
\Msg{* Installation}
\Msg{* Package: kvdefinekeys 2016/05/16 v1.4 Define keys (HO)}
\Msg{************************************************************************}

\keepsilent
\askforoverwritefalse

\let\MetaPrefix\relax
\preamble

This is a generated file.

Project: kvdefinekeys
Version: 2016/05/16 v1.4

Copyright (C) 2010, 2011 by
   Heiko Oberdiek <heiko.oberdiek at googlemail.com>

This work may be distributed and/or modified under the
conditions of the LaTeX Project Public License, either
version 1.3c of this license or (at your option) any later
version. This version of this license is in
   http://www.latex-project.org/lppl/lppl-1-3c.txt
and the latest version of this license is in
   http://www.latex-project.org/lppl.txt
and version 1.3 or later is part of all distributions of
LaTeX version 2005/12/01 or later.

This work has the LPPL maintenance status "maintained".

This Current Maintainer of this work is Heiko Oberdiek.

The Base Interpreter refers to any `TeX-Format',
because some files are installed in TDS:tex/generic//.

This work consists of the main source file kvdefinekeys.dtx
and the derived files
   kvdefinekeys.sty, kvdefinekeys.pdf, kvdefinekeys.ins, kvdefinekeys.drv,
   kvdefinekeys-test1.tex.

\endpreamble
\let\MetaPrefix\DoubleperCent

\generate{%
  \file{kvdefinekeys.ins}{\from{kvdefinekeys.dtx}{install}}%
  \file{kvdefinekeys.drv}{\from{kvdefinekeys.dtx}{driver}}%
  \usedir{tex/generic/oberdiek}%
  \file{kvdefinekeys.sty}{\from{kvdefinekeys.dtx}{package}}%
  \usedir{doc/latex/oberdiek/test}%
  \file{kvdefinekeys-test1.tex}{\from{kvdefinekeys.dtx}{test1}}%
  \nopreamble
  \nopostamble
  \usedir{source/latex/oberdiek/catalogue}%
  \file{kvdefinekeys.xml}{\from{kvdefinekeys.dtx}{catalogue}}%
}

\catcode32=13\relax% active space
\let =\space%
\Msg{************************************************************************}
\Msg{*}
\Msg{* To finish the installation you have to move the following}
\Msg{* file into a directory searched by TeX:}
\Msg{*}
\Msg{*     kvdefinekeys.sty}
\Msg{*}
\Msg{* To produce the documentation run the file `kvdefinekeys.drv'}
\Msg{* through LaTeX.}
\Msg{*}
\Msg{* Happy TeXing!}
\Msg{*}
\Msg{************************************************************************}

\endbatchfile
%</install>
%<*ignore>
\fi
%</ignore>
%<*driver>
\NeedsTeXFormat{LaTeX2e}
\ProvidesFile{kvdefinekeys.drv}%
  [2016/05/16 v1.4 Define keys (HO)]%
\documentclass{ltxdoc}
\usepackage{holtxdoc}[2011/11/22]
\begin{document}
  \DocInput{kvdefinekeys.dtx}%
\end{document}
%</driver>
% \fi
%
%
% \CharacterTable
%  {Upper-case    \A\B\C\D\E\F\G\H\I\J\K\L\M\N\O\P\Q\R\S\T\U\V\W\X\Y\Z
%   Lower-case    \a\b\c\d\e\f\g\h\i\j\k\l\m\n\o\p\q\r\s\t\u\v\w\x\y\z
%   Digits        \0\1\2\3\4\5\6\7\8\9
%   Exclamation   \!     Double quote  \"     Hash (number) \#
%   Dollar        \$     Percent       \%     Ampersand     \&
%   Acute accent  \'     Left paren    \(     Right paren   \)
%   Asterisk      \*     Plus          \+     Comma         \,
%   Minus         \-     Point         \.     Solidus       \/
%   Colon         \:     Semicolon     \;     Less than     \<
%   Equals        \=     Greater than  \>     Question mark \?
%   Commercial at \@     Left bracket  \[     Backslash     \\
%   Right bracket \]     Circumflex    \^     Underscore    \_
%   Grave accent  \`     Left brace    \{     Vertical bar  \|
%   Right brace   \}     Tilde         \~}
%
% \GetFileInfo{kvdefinekeys.drv}
%
% \title{The \xpackage{kvdefinekeys} package}
% \date{2016/05/16 v1.4}
% \author{Heiko Oberdiek\thanks
% {Please report any issues at https://github.com/ho-tex/oberdiek/issues}\\
% \xemail{heiko.oberdiek at googlemail.com}}
%
% \maketitle
%
% \begin{abstract}
% Package \xpackage{kvdefinekeys} provides \cs{kv@define@key} to define
% keys the same way as \xpackage{keyval}'s \cs{define@key}. However, it
% works also using \iniTeX.
% \end{abstract}
%
% \tableofcontents
%
% \def\M#1{\texttt{\{}\meta{#1}\texttt{\}}}
%
% \section{Documentation}
%
% \subsection{Motivation}
%
% \cs{kvsetkeys} serves as replacement for \xpackage{keyval}'s
% \cs{setkeys}. This package adds macros to define keys, closing
% the gap \cs{kvsetkeys} leaves.
%
% \begin{declcs}{kv@define@key}\,\M{family}\,\M{key}\,[\meta{default}]\,^^A
%   \M{definition}
% \end{declcs}
% Macro \cs{kv@define@key} reimplements \xpackage{keyval}'s
% \cs{define@key}. Differences to the original:
% \begin{itemize}
% \item The defined keys also allow \cs{par} inside values.
% \item Shorthands of package \xpackage{babel} are supported in
%   family and key names.
% \item Macro \cs{kv@define@key} is made robust if
%   \hologo{eTeX}'s \cs{protected} or \hologo{LaTeX}'s
%   \cs{DeclareRobustCommand} are found.
% \end{itemize}
%
% \StopEventually{
% }
%
% \section{Implementation}
%
% \subsection{Identification}
%
%    \begin{macrocode}
%<*package>
%    \end{macrocode}
%    Reload check, especially if the package is not used with \LaTeX.
%    \begin{macrocode}
\begingroup\catcode61\catcode48\catcode32=10\relax%
  \catcode13=5 % ^^M
  \endlinechar=13 %
  \catcode35=6 % #
  \catcode39=12 % '
  \catcode44=12 % ,
  \catcode45=12 % -
  \catcode46=12 % .
  \catcode58=12 % :
  \catcode64=11 % @
  \catcode123=1 % {
  \catcode125=2 % }
  \expandafter\let\expandafter\x\csname ver@kvdefinekeys.sty\endcsname
  \ifx\x\relax % plain-TeX, first loading
  \else
    \def\empty{}%
    \ifx\x\empty % LaTeX, first loading,
      % variable is initialized, but \ProvidesPackage not yet seen
    \else
      \expandafter\ifx\csname PackageInfo\endcsname\relax
        \def\x#1#2{%
          \immediate\write-1{Package #1 Info: #2.}%
        }%
      \else
        \def\x#1#2{\PackageInfo{#1}{#2, stopped}}%
      \fi
      \x{kvdefinekeys}{The package is already loaded}%
      \aftergroup\endinput
    \fi
  \fi
\endgroup%
%    \end{macrocode}
%    Package identification:
%    \begin{macrocode}
\begingroup\catcode61\catcode48\catcode32=10\relax%
  \catcode13=5 % ^^M
  \endlinechar=13 %
  \catcode35=6 % #
  \catcode39=12 % '
  \catcode40=12 % (
  \catcode41=12 % )
  \catcode44=12 % ,
  \catcode45=12 % -
  \catcode46=12 % .
  \catcode47=12 % /
  \catcode58=12 % :
  \catcode64=11 % @
  \catcode91=12 % [
  \catcode93=12 % ]
  \catcode123=1 % {
  \catcode125=2 % }
  \expandafter\ifx\csname ProvidesPackage\endcsname\relax
    \def\x#1#2#3[#4]{\endgroup
      \immediate\write-1{Package: #3 #4}%
      \xdef#1{#4}%
    }%
  \else
    \def\x#1#2[#3]{\endgroup
      #2[{#3}]%
      \ifx#1\@undefined
        \xdef#1{#3}%
      \fi
      \ifx#1\relax
        \xdef#1{#3}%
      \fi
    }%
  \fi
\expandafter\x\csname ver@kvdefinekeys.sty\endcsname
\ProvidesPackage{kvdefinekeys}%
  [2016/05/16 v1.4 Define keys (HO)]%
%    \end{macrocode}
%
%    \begin{macrocode}
\begingroup\catcode61\catcode48\catcode32=10\relax%
  \catcode13=5 % ^^M
  \endlinechar=13 %
  \catcode123=1 % {
  \catcode125=2 % }
  \catcode64=11 % @
  \def\x{\endgroup
    \expandafter\edef\csname KVD@AtEnd\endcsname{%
      \endlinechar=\the\endlinechar\relax
      \catcode13=\the\catcode13\relax
      \catcode32=\the\catcode32\relax
      \catcode35=\the\catcode35\relax
      \catcode61=\the\catcode61\relax
      \catcode64=\the\catcode64\relax
      \catcode123=\the\catcode123\relax
      \catcode125=\the\catcode125\relax
    }%
  }%
\x\catcode61\catcode48\catcode32=10\relax%
\catcode13=5 % ^^M
\endlinechar=13 %
\catcode35=6 % #
\catcode64=11 % @
\catcode123=1 % {
\catcode125=2 % }
\def\TMP@EnsureCode#1#2{%
  \edef\KVD@AtEnd{%
    \KVD@AtEnd
    \catcode#1=\the\catcode#1\relax
  }%
  \catcode#1=#2\relax
}
\TMP@EnsureCode{42}{12}% *
\TMP@EnsureCode{46}{12}% .
\TMP@EnsureCode{47}{12}% /
\TMP@EnsureCode{91}{12}% [
\TMP@EnsureCode{93}{12}% ]
\edef\KVD@AtEnd{\KVD@AtEnd\noexpand\endinput}
%    \end{macrocode}
%
% \subsection{Package loading}
%
%    \begin{macrocode}
\begingroup\expandafter\expandafter\expandafter\endgroup
\expandafter\ifx\csname RequirePackage\endcsname\relax
  \def\TMP@RequirePackage#1[#2]{%
    \begingroup\expandafter\expandafter\expandafter\endgroup
    \expandafter\ifx\csname ver@#1.sty\endcsname\relax
      \input #1.sty\relax
    \fi
  }%
  \TMP@RequirePackage{ltxcmds}[2010/03/01]%
\else
  \RequirePackage{ltxcmds}[2010/03/01]%
\fi
%    \end{macrocode}
%
%
% \subsection{Provide key defining macro}
%
%    \begin{macro}{\kv@define@key}
%    \begin{macrocode}
\ltx@IfUndefined{protected}{%
  \ltx@IfUndefined{DeclareRobustCommand}{%
    \def\kv@define@key#1#2%
  }{%
    \DeclareRobustCommand*{\kv@define@key}[2]%
  }%
}{%
  \protected\def\kv@define@key#1#2%
}%
{%
  \begingroup
    \csname @safe@activestrue\endcsname
    \let\ifincsname\iftrue
    \edef\KVD@temp{\endgroup
      \noexpand\KVD@DefineKey{#1}{#2}%
    }%
  \KVD@temp
}
%    \end{macrocode}
%    \end{macro}
%    \begin{macro}{\KVD@DefineKey}
%    \begin{macrocode}
\def\KVD@DefineKey#1#2{%
  \ltx@ifnextchar[{%
    \KVD@DefineKeyWithDefault{#1}{#2}%
  }{%
    \long\expandafter\def\csname KV@#1@#2\endcsname##1%
  }%
}
%    \end{macrocode}
%    \end{macro}
%    \begin{macro}{\KVD@DefineKeyWithDefault}
%    \begin{macrocode}
\long\def\KVD@DefineKeyWithDefault#1#2[#3]{%
  \expandafter\def\csname KV@#1@#2@default\expandafter\endcsname
  \expandafter{%
    \csname KV@#1@#2\endcsname{#3}%
  }%
  \long\expandafter\def\csname KV@#1@#2\endcsname##1%
}
%    \end{macrocode}
%    \end{macro}
%
%    \begin{macrocode}
\KVD@AtEnd%
%</package>
%    \end{macrocode}
%
% \section{Test}
%
% \subsection{Catcode checks for loading}
%
%    \begin{macrocode}
%<*test1>
%    \end{macrocode}
%    \begin{macrocode}
\catcode`\{=1 %
\catcode`\}=2 %
\catcode`\#=6 %
\catcode`\@=11 %
\expandafter\ifx\csname count@\endcsname\relax
  \countdef\count@=255 %
\fi
\expandafter\ifx\csname @gobble\endcsname\relax
  \long\def\@gobble#1{}%
\fi
\expandafter\ifx\csname @firstofone\endcsname\relax
  \long\def\@firstofone#1{#1}%
\fi
\expandafter\ifx\csname loop\endcsname\relax
  \expandafter\@firstofone
\else
  \expandafter\@gobble
\fi
{%
  \def\loop#1\repeat{%
    \def\body{#1}%
    \iterate
  }%
  \def\iterate{%
    \body
      \let\next\iterate
    \else
      \let\next\relax
    \fi
    \next
  }%
  \let\repeat=\fi
}%
\def\RestoreCatcodes{}
\count@=0 %
\loop
  \edef\RestoreCatcodes{%
    \RestoreCatcodes
    \catcode\the\count@=\the\catcode\count@\relax
  }%
\ifnum\count@<255 %
  \advance\count@ 1 %
\repeat

\def\RangeCatcodeInvalid#1#2{%
  \count@=#1\relax
  \loop
    \catcode\count@=15 %
  \ifnum\count@<#2\relax
    \advance\count@ 1 %
  \repeat
}
\def\RangeCatcodeCheck#1#2#3{%
  \count@=#1\relax
  \loop
    \ifnum#3=\catcode\count@
    \else
      \errmessage{%
        Character \the\count@\space
        with wrong catcode \the\catcode\count@\space
        instead of \number#3%
      }%
    \fi
  \ifnum\count@<#2\relax
    \advance\count@ 1 %
  \repeat
}
\def\space{ }
\expandafter\ifx\csname LoadCommand\endcsname\relax
  \def\LoadCommand{\input kvdefinekeys.sty\relax}%
\fi
\def\Test{%
  \RangeCatcodeInvalid{0}{47}%
  \RangeCatcodeInvalid{58}{64}%
  \RangeCatcodeInvalid{91}{96}%
  \RangeCatcodeInvalid{123}{255}%
  \catcode`\@=12 %
  \catcode`\\=0 %
  \catcode`\%=14 %
  \LoadCommand
  \RangeCatcodeCheck{0}{36}{15}%
  \RangeCatcodeCheck{37}{37}{14}%
  \RangeCatcodeCheck{38}{47}{15}%
  \RangeCatcodeCheck{48}{57}{12}%
  \RangeCatcodeCheck{58}{63}{15}%
  \RangeCatcodeCheck{64}{64}{12}%
  \RangeCatcodeCheck{65}{90}{11}%
  \RangeCatcodeCheck{91}{91}{15}%
  \RangeCatcodeCheck{92}{92}{0}%
  \RangeCatcodeCheck{93}{96}{15}%
  \RangeCatcodeCheck{97}{122}{11}%
  \RangeCatcodeCheck{123}{255}{15}%
  \RestoreCatcodes
}
\Test
\csname @@end\endcsname
\end
%    \end{macrocode}
%    \begin{macrocode}
%</test1>
%    \end{macrocode}
%
% \section{Installation}
%
% \subsection{Download}
%
% \paragraph{Package.} This package is available on
% CTAN\footnote{\url{http://ctan.org/pkg/kvdefinekeys}}:
% \begin{description}
% \item[\CTAN{macros/latex/contrib/oberdiek/kvdefinekeys.dtx}] The source file.
% \item[\CTAN{macros/latex/contrib/oberdiek/kvdefinekeys.pdf}] Documentation.
% \end{description}
%
%
% \paragraph{Bundle.} All the packages of the bundle `oberdiek'
% are also available in a TDS compliant ZIP archive. There
% the packages are already unpacked and the documentation files
% are generated. The files and directories obey the TDS standard.
% \begin{description}
% \item[\CTAN{install/macros/latex/contrib/oberdiek.tds.zip}]
% \end{description}
% \emph{TDS} refers to the standard ``A Directory Structure
% for \TeX\ Files'' (\CTAN{tds/tds.pdf}). Directories
% with \xfile{texmf} in their name are usually organized this way.
%
% \subsection{Bundle installation}
%
% \paragraph{Unpacking.} Unpack the \xfile{oberdiek.tds.zip} in the
% TDS tree (also known as \xfile{texmf} tree) of your choice.
% Example (linux):
% \begin{quote}
%   |unzip oberdiek.tds.zip -d ~/texmf|
% \end{quote}
%
% \paragraph{Script installation.}
% Check the directory \xfile{TDS:scripts/oberdiek/} for
% scripts that need further installation steps.
% Package \xpackage{attachfile2} comes with the Perl script
% \xfile{pdfatfi.pl} that should be installed in such a way
% that it can be called as \texttt{pdfatfi}.
% Example (linux):
% \begin{quote}
%   |chmod +x scripts/oberdiek/pdfatfi.pl|\\
%   |cp scripts/oberdiek/pdfatfi.pl /usr/local/bin/|
% \end{quote}
%
% \subsection{Package installation}
%
% \paragraph{Unpacking.} The \xfile{.dtx} file is a self-extracting
% \docstrip\ archive. The files are extracted by running the
% \xfile{.dtx} through \plainTeX:
% \begin{quote}
%   \verb|tex kvdefinekeys.dtx|
% \end{quote}
%
% \paragraph{TDS.} Now the different files must be moved into
% the different directories in your installation TDS tree
% (also known as \xfile{texmf} tree):
% \begin{quote}
% \def\t{^^A
% \begin{tabular}{@{}>{\ttfamily}l@{ $\rightarrow$ }>{\ttfamily}l@{}}
%   kvdefinekeys.sty & tex/generic/oberdiek/kvdefinekeys.sty\\
%   kvdefinekeys.pdf & doc/latex/oberdiek/kvdefinekeys.pdf\\
%   test/kvdefinekeys-test1.tex & doc/latex/oberdiek/test/kvdefinekeys-test1.tex\\
%   kvdefinekeys.dtx & source/latex/oberdiek/kvdefinekeys.dtx\\
% \end{tabular}^^A
% }^^A
% \sbox0{\t}^^A
% \ifdim\wd0>\linewidth
%   \begingroup
%     \advance\linewidth by\leftmargin
%     \advance\linewidth by\rightmargin
%   \edef\x{\endgroup
%     \def\noexpand\lw{\the\linewidth}^^A
%   }\x
%   \def\lwbox{^^A
%     \leavevmode
%     \hbox to \linewidth{^^A
%       \kern-\leftmargin\relax
%       \hss
%       \usebox0
%       \hss
%       \kern-\rightmargin\relax
%     }^^A
%   }^^A
%   \ifdim\wd0>\lw
%     \sbox0{\small\t}^^A
%     \ifdim\wd0>\linewidth
%       \ifdim\wd0>\lw
%         \sbox0{\footnotesize\t}^^A
%         \ifdim\wd0>\linewidth
%           \ifdim\wd0>\lw
%             \sbox0{\scriptsize\t}^^A
%             \ifdim\wd0>\linewidth
%               \ifdim\wd0>\lw
%                 \sbox0{\tiny\t}^^A
%                 \ifdim\wd0>\linewidth
%                   \lwbox
%                 \else
%                   \usebox0
%                 \fi
%               \else
%                 \lwbox
%               \fi
%             \else
%               \usebox0
%             \fi
%           \else
%             \lwbox
%           \fi
%         \else
%           \usebox0
%         \fi
%       \else
%         \lwbox
%       \fi
%     \else
%       \usebox0
%     \fi
%   \else
%     \lwbox
%   \fi
% \else
%   \usebox0
% \fi
% \end{quote}
% If you have a \xfile{docstrip.cfg} that configures and enables \docstrip's
% TDS installing feature, then some files can already be in the right
% place, see the documentation of \docstrip.
%
% \subsection{Refresh file name databases}
%
% If your \TeX~distribution
% (\teTeX, \mikTeX, \dots) relies on file name databases, you must refresh
% these. For example, \teTeX\ users run \verb|texhash| or
% \verb|mktexlsr|.
%
% \subsection{Some details for the interested}
%
% \paragraph{Attached source.}
%
% The PDF documentation on CTAN also includes the
% \xfile{.dtx} source file. It can be extracted by
% AcrobatReader 6 or higher. Another option is \textsf{pdftk},
% e.g. unpack the file into the current directory:
% \begin{quote}
%   \verb|pdftk kvdefinekeys.pdf unpack_files output .|
% \end{quote}
%
% \paragraph{Unpacking with \LaTeX.}
% The \xfile{.dtx} chooses its action depending on the format:
% \begin{description}
% \item[\plainTeX:] Run \docstrip\ and extract the files.
% \item[\LaTeX:] Generate the documentation.
% \end{description}
% If you insist on using \LaTeX\ for \docstrip\ (really,
% \docstrip\ does not need \LaTeX), then inform the autodetect routine
% about your intention:
% \begin{quote}
%   \verb|latex \let\install=y\input{kvdefinekeys.dtx}|
% \end{quote}
% Do not forget to quote the argument according to the demands
% of your shell.
%
% \paragraph{Generating the documentation.}
% You can use both the \xfile{.dtx} or the \xfile{.drv} to generate
% the documentation. The process can be configured by the
% configuration file \xfile{ltxdoc.cfg}. For instance, put this
% line into this file, if you want to have A4 as paper format:
% \begin{quote}
%   \verb|\PassOptionsToClass{a4paper}{article}|
% \end{quote}
% An example follows how to generate the
% documentation with pdf\LaTeX:
% \begin{quote}
%\begin{verbatim}
%pdflatex kvdefinekeys.dtx
%makeindex -s gind.ist kvdefinekeys.idx
%pdflatex kvdefinekeys.dtx
%makeindex -s gind.ist kvdefinekeys.idx
%pdflatex kvdefinekeys.dtx
%\end{verbatim}
% \end{quote}
%
% \section{Catalogue}
%
% The following XML file can be used as source for the
% \href{http://mirror.ctan.org/help/Catalogue/catalogue.html}{\TeX\ Catalogue}.
% The elements \texttt{caption} and \texttt{description} are imported
% from the original XML file from the Catalogue.
% The name of the XML file in the Catalogue is \xfile{kvdefinekeys.xml}.
%    \begin{macrocode}
%<*catalogue>
<?xml version='1.0' encoding='us-ascii'?>
<!DOCTYPE entry SYSTEM 'catalogue.dtd'>
<entry datestamp='$Date$' modifier='$Author$' id='kvdefinekeys'>
  <name>kvdefinekeys</name>
  <caption>Define keys for use in the kvsetkeys package.</caption>
  <authorref id='auth:oberdiek'/>
  <copyright owner='Heiko Oberdiek' year='2010,2011'/>
  <license type='lppl1.3'/>
  <version number='1.4'/>
  <description>
    The package provides a macro <tt>\kv@define@key</tt> (analogous to
    <xref refid='keyval'>keyval&#x2019;s</xref> <tt>\define@key</tt>, to
    define keys for use by <xref refid='kvsetkeys'>kvsetkeys</xref>.
    <p/>
    The package is part of the <xref refid='oberdiek'>oberdiek</xref>
    bundle.
  </description>
  <documentation details='Package documentation'
      href='ctan:/macros/latex/contrib/oberdiek/kvdefinekeys.pdf'/>
  <ctan file='true' path='/macros/latex/contrib/oberdiek/kvdefinekeys.dtx'/>
  <miktex location='oberdiek'/>
  <texlive location='oberdiek'/>
  <install path='/macros/latex/contrib/oberdiek/oberdiek.tds.zip'/>
</entry>
%</catalogue>
%    \end{macrocode}
%
% \begin{thebibliography}{9}
% \bibitem{keyval}
%   David Carlisle:
%   \textit{The \xpackage{keyval} package};
%   1999/03/16 v1.13;
%   \CTAN{macros/latex/required/graphics/keyval.dtx}.
%
% \end{thebibliography}
%
% \begin{History}
%   \begin{Version}{2010/03/01 v1.0}
%   \item
%     First version.
%   \end{Version}
%   \begin{Version}{2010/08/19 v1.1}
%   \item
%     Documentation fix, no code change.
%   \end{Version}
%   \begin{Version}{2011/01/30 v1.2}
%   \item
%     Already loaded package files are not input in \hologo{plainTeX}.
%   \end{Version}
%   \begin{Version}{2011/04/07 v1.3}
%   \item
%     Support for package \xpackage{babel}'s shorthands added.
%   \item
%     \cs{kv@define@key} is made robust if available.
%   \end{Version}
%   \begin{Version}{2016/05/16 v1.4}
%   \item
%     Documentation updates.
%   \end{Version}
% \end{History}
%
% \PrintIndex
%
% \Finale
\endinput

%        (quote the arguments according to the demands of your shell)
%
% Documentation:
%    (a) If kvdefinekeys.drv is present:
%           latex kvdefinekeys.drv
%    (b) Without kvdefinekeys.drv:
%           latex kvdefinekeys.dtx; ...
%    The class ltxdoc loads the configuration file ltxdoc.cfg
%    if available. Here you can specify further options, e.g.
%    use A4 as paper format:
%       \PassOptionsToClass{a4paper}{article}
%
%    Programm calls to get the documentation (example):
%       pdflatex kvdefinekeys.dtx
%       makeindex -s gind.ist kvdefinekeys.idx
%       pdflatex kvdefinekeys.dtx
%       makeindex -s gind.ist kvdefinekeys.idx
%       pdflatex kvdefinekeys.dtx
%
% Installation:
%    TDS:tex/generic/oberdiek/kvdefinekeys.sty
%    TDS:doc/latex/oberdiek/kvdefinekeys.pdf
%    TDS:doc/latex/oberdiek/test/kvdefinekeys-test1.tex
%    TDS:source/latex/oberdiek/kvdefinekeys.dtx
%
%<*ignore>
\begingroup
  \catcode123=1 %
  \catcode125=2 %
  \def\x{LaTeX2e}%
\expandafter\endgroup
\ifcase 0\ifx\install y1\fi\expandafter
         \ifx\csname processbatchFile\endcsname\relax\else1\fi
         \ifx\fmtname\x\else 1\fi\relax
\else\csname fi\endcsname
%</ignore>
%<*install>
\input docstrip.tex
\Msg{************************************************************************}
\Msg{* Installation}
\Msg{* Package: kvdefinekeys 2016/05/16 v1.4 Define keys (HO)}
\Msg{************************************************************************}

\keepsilent
\askforoverwritefalse

\let\MetaPrefix\relax
\preamble

This is a generated file.

Project: kvdefinekeys
Version: 2016/05/16 v1.4

Copyright (C) 2010, 2011 by
   Heiko Oberdiek <heiko.oberdiek at googlemail.com>

This work may be distributed and/or modified under the
conditions of the LaTeX Project Public License, either
version 1.3c of this license or (at your option) any later
version. This version of this license is in
   http://www.latex-project.org/lppl/lppl-1-3c.txt
and the latest version of this license is in
   http://www.latex-project.org/lppl.txt
and version 1.3 or later is part of all distributions of
LaTeX version 2005/12/01 or later.

This work has the LPPL maintenance status "maintained".

This Current Maintainer of this work is Heiko Oberdiek.

The Base Interpreter refers to any `TeX-Format',
because some files are installed in TDS:tex/generic//.

This work consists of the main source file kvdefinekeys.dtx
and the derived files
   kvdefinekeys.sty, kvdefinekeys.pdf, kvdefinekeys.ins, kvdefinekeys.drv,
   kvdefinekeys-test1.tex.

\endpreamble
\let\MetaPrefix\DoubleperCent

\generate{%
  \file{kvdefinekeys.ins}{\from{kvdefinekeys.dtx}{install}}%
  \file{kvdefinekeys.drv}{\from{kvdefinekeys.dtx}{driver}}%
  \usedir{tex/generic/oberdiek}%
  \file{kvdefinekeys.sty}{\from{kvdefinekeys.dtx}{package}}%
  \usedir{doc/latex/oberdiek/test}%
  \file{kvdefinekeys-test1.tex}{\from{kvdefinekeys.dtx}{test1}}%
  \nopreamble
  \nopostamble
  \usedir{source/latex/oberdiek/catalogue}%
  \file{kvdefinekeys.xml}{\from{kvdefinekeys.dtx}{catalogue}}%
}

\catcode32=13\relax% active space
\let =\space%
\Msg{************************************************************************}
\Msg{*}
\Msg{* To finish the installation you have to move the following}
\Msg{* file into a directory searched by TeX:}
\Msg{*}
\Msg{*     kvdefinekeys.sty}
\Msg{*}
\Msg{* To produce the documentation run the file `kvdefinekeys.drv'}
\Msg{* through LaTeX.}
\Msg{*}
\Msg{* Happy TeXing!}
\Msg{*}
\Msg{************************************************************************}

\endbatchfile
%</install>
%<*ignore>
\fi
%</ignore>
%<*driver>
\NeedsTeXFormat{LaTeX2e}
\ProvidesFile{kvdefinekeys.drv}%
  [2016/05/16 v1.4 Define keys (HO)]%
\documentclass{ltxdoc}
\usepackage{holtxdoc}[2011/11/22]
\begin{document}
  \DocInput{kvdefinekeys.dtx}%
\end{document}
%</driver>
% \fi
%
%
% \CharacterTable
%  {Upper-case    \A\B\C\D\E\F\G\H\I\J\K\L\M\N\O\P\Q\R\S\T\U\V\W\X\Y\Z
%   Lower-case    \a\b\c\d\e\f\g\h\i\j\k\l\m\n\o\p\q\r\s\t\u\v\w\x\y\z
%   Digits        \0\1\2\3\4\5\6\7\8\9
%   Exclamation   \!     Double quote  \"     Hash (number) \#
%   Dollar        \$     Percent       \%     Ampersand     \&
%   Acute accent  \'     Left paren    \(     Right paren   \)
%   Asterisk      \*     Plus          \+     Comma         \,
%   Minus         \-     Point         \.     Solidus       \/
%   Colon         \:     Semicolon     \;     Less than     \<
%   Equals        \=     Greater than  \>     Question mark \?
%   Commercial at \@     Left bracket  \[     Backslash     \\
%   Right bracket \]     Circumflex    \^     Underscore    \_
%   Grave accent  \`     Left brace    \{     Vertical bar  \|
%   Right brace   \}     Tilde         \~}
%
% \GetFileInfo{kvdefinekeys.drv}
%
% \title{The \xpackage{kvdefinekeys} package}
% \date{2016/05/16 v1.4}
% \author{Heiko Oberdiek\thanks
% {Please report any issues at https://github.com/ho-tex/oberdiek/issues}\\
% \xemail{heiko.oberdiek at googlemail.com}}
%
% \maketitle
%
% \begin{abstract}
% Package \xpackage{kvdefinekeys} provides \cs{kv@define@key} to define
% keys the same way as \xpackage{keyval}'s \cs{define@key}. However, it
% works also using \iniTeX.
% \end{abstract}
%
% \tableofcontents
%
% \def\M#1{\texttt{\{}\meta{#1}\texttt{\}}}
%
% \section{Documentation}
%
% \subsection{Motivation}
%
% \cs{kvsetkeys} serves as replacement for \xpackage{keyval}'s
% \cs{setkeys}. This package adds macros to define keys, closing
% the gap \cs{kvsetkeys} leaves.
%
% \begin{declcs}{kv@define@key}\,\M{family}\,\M{key}\,[\meta{default}]\,^^A
%   \M{definition}
% \end{declcs}
% Macro \cs{kv@define@key} reimplements \xpackage{keyval}'s
% \cs{define@key}. Differences to the original:
% \begin{itemize}
% \item The defined keys also allow \cs{par} inside values.
% \item Shorthands of package \xpackage{babel} are supported in
%   family and key names.
% \item Macro \cs{kv@define@key} is made robust if
%   \hologo{eTeX}'s \cs{protected} or \hologo{LaTeX}'s
%   \cs{DeclareRobustCommand} are found.
% \end{itemize}
%
% \StopEventually{
% }
%
% \section{Implementation}
%
% \subsection{Identification}
%
%    \begin{macrocode}
%<*package>
%    \end{macrocode}
%    Reload check, especially if the package is not used with \LaTeX.
%    \begin{macrocode}
\begingroup\catcode61\catcode48\catcode32=10\relax%
  \catcode13=5 % ^^M
  \endlinechar=13 %
  \catcode35=6 % #
  \catcode39=12 % '
  \catcode44=12 % ,
  \catcode45=12 % -
  \catcode46=12 % .
  \catcode58=12 % :
  \catcode64=11 % @
  \catcode123=1 % {
  \catcode125=2 % }
  \expandafter\let\expandafter\x\csname ver@kvdefinekeys.sty\endcsname
  \ifx\x\relax % plain-TeX, first loading
  \else
    \def\empty{}%
    \ifx\x\empty % LaTeX, first loading,
      % variable is initialized, but \ProvidesPackage not yet seen
    \else
      \expandafter\ifx\csname PackageInfo\endcsname\relax
        \def\x#1#2{%
          \immediate\write-1{Package #1 Info: #2.}%
        }%
      \else
        \def\x#1#2{\PackageInfo{#1}{#2, stopped}}%
      \fi
      \x{kvdefinekeys}{The package is already loaded}%
      \aftergroup\endinput
    \fi
  \fi
\endgroup%
%    \end{macrocode}
%    Package identification:
%    \begin{macrocode}
\begingroup\catcode61\catcode48\catcode32=10\relax%
  \catcode13=5 % ^^M
  \endlinechar=13 %
  \catcode35=6 % #
  \catcode39=12 % '
  \catcode40=12 % (
  \catcode41=12 % )
  \catcode44=12 % ,
  \catcode45=12 % -
  \catcode46=12 % .
  \catcode47=12 % /
  \catcode58=12 % :
  \catcode64=11 % @
  \catcode91=12 % [
  \catcode93=12 % ]
  \catcode123=1 % {
  \catcode125=2 % }
  \expandafter\ifx\csname ProvidesPackage\endcsname\relax
    \def\x#1#2#3[#4]{\endgroup
      \immediate\write-1{Package: #3 #4}%
      \xdef#1{#4}%
    }%
  \else
    \def\x#1#2[#3]{\endgroup
      #2[{#3}]%
      \ifx#1\@undefined
        \xdef#1{#3}%
      \fi
      \ifx#1\relax
        \xdef#1{#3}%
      \fi
    }%
  \fi
\expandafter\x\csname ver@kvdefinekeys.sty\endcsname
\ProvidesPackage{kvdefinekeys}%
  [2016/05/16 v1.4 Define keys (HO)]%
%    \end{macrocode}
%
%    \begin{macrocode}
\begingroup\catcode61\catcode48\catcode32=10\relax%
  \catcode13=5 % ^^M
  \endlinechar=13 %
  \catcode123=1 % {
  \catcode125=2 % }
  \catcode64=11 % @
  \def\x{\endgroup
    \expandafter\edef\csname KVD@AtEnd\endcsname{%
      \endlinechar=\the\endlinechar\relax
      \catcode13=\the\catcode13\relax
      \catcode32=\the\catcode32\relax
      \catcode35=\the\catcode35\relax
      \catcode61=\the\catcode61\relax
      \catcode64=\the\catcode64\relax
      \catcode123=\the\catcode123\relax
      \catcode125=\the\catcode125\relax
    }%
  }%
\x\catcode61\catcode48\catcode32=10\relax%
\catcode13=5 % ^^M
\endlinechar=13 %
\catcode35=6 % #
\catcode64=11 % @
\catcode123=1 % {
\catcode125=2 % }
\def\TMP@EnsureCode#1#2{%
  \edef\KVD@AtEnd{%
    \KVD@AtEnd
    \catcode#1=\the\catcode#1\relax
  }%
  \catcode#1=#2\relax
}
\TMP@EnsureCode{42}{12}% *
\TMP@EnsureCode{46}{12}% .
\TMP@EnsureCode{47}{12}% /
\TMP@EnsureCode{91}{12}% [
\TMP@EnsureCode{93}{12}% ]
\edef\KVD@AtEnd{\KVD@AtEnd\noexpand\endinput}
%    \end{macrocode}
%
% \subsection{Package loading}
%
%    \begin{macrocode}
\begingroup\expandafter\expandafter\expandafter\endgroup
\expandafter\ifx\csname RequirePackage\endcsname\relax
  \def\TMP@RequirePackage#1[#2]{%
    \begingroup\expandafter\expandafter\expandafter\endgroup
    \expandafter\ifx\csname ver@#1.sty\endcsname\relax
      \input #1.sty\relax
    \fi
  }%
  \TMP@RequirePackage{ltxcmds}[2010/03/01]%
\else
  \RequirePackage{ltxcmds}[2010/03/01]%
\fi
%    \end{macrocode}
%
%
% \subsection{Provide key defining macro}
%
%    \begin{macro}{\kv@define@key}
%    \begin{macrocode}
\ltx@IfUndefined{protected}{%
  \ltx@IfUndefined{DeclareRobustCommand}{%
    \def\kv@define@key#1#2%
  }{%
    \DeclareRobustCommand*{\kv@define@key}[2]%
  }%
}{%
  \protected\def\kv@define@key#1#2%
}%
{%
  \begingroup
    \csname @safe@activestrue\endcsname
    \let\ifincsname\iftrue
    \edef\KVD@temp{\endgroup
      \noexpand\KVD@DefineKey{#1}{#2}%
    }%
  \KVD@temp
}
%    \end{macrocode}
%    \end{macro}
%    \begin{macro}{\KVD@DefineKey}
%    \begin{macrocode}
\def\KVD@DefineKey#1#2{%
  \ltx@ifnextchar[{%
    \KVD@DefineKeyWithDefault{#1}{#2}%
  }{%
    \long\expandafter\def\csname KV@#1@#2\endcsname##1%
  }%
}
%    \end{macrocode}
%    \end{macro}
%    \begin{macro}{\KVD@DefineKeyWithDefault}
%    \begin{macrocode}
\long\def\KVD@DefineKeyWithDefault#1#2[#3]{%
  \expandafter\def\csname KV@#1@#2@default\expandafter\endcsname
  \expandafter{%
    \csname KV@#1@#2\endcsname{#3}%
  }%
  \long\expandafter\def\csname KV@#1@#2\endcsname##1%
}
%    \end{macrocode}
%    \end{macro}
%
%    \begin{macrocode}
\KVD@AtEnd%
%</package>
%    \end{macrocode}
%
% \section{Test}
%
% \subsection{Catcode checks for loading}
%
%    \begin{macrocode}
%<*test1>
%    \end{macrocode}
%    \begin{macrocode}
\catcode`\{=1 %
\catcode`\}=2 %
\catcode`\#=6 %
\catcode`\@=11 %
\expandafter\ifx\csname count@\endcsname\relax
  \countdef\count@=255 %
\fi
\expandafter\ifx\csname @gobble\endcsname\relax
  \long\def\@gobble#1{}%
\fi
\expandafter\ifx\csname @firstofone\endcsname\relax
  \long\def\@firstofone#1{#1}%
\fi
\expandafter\ifx\csname loop\endcsname\relax
  \expandafter\@firstofone
\else
  \expandafter\@gobble
\fi
{%
  \def\loop#1\repeat{%
    \def\body{#1}%
    \iterate
  }%
  \def\iterate{%
    \body
      \let\next\iterate
    \else
      \let\next\relax
    \fi
    \next
  }%
  \let\repeat=\fi
}%
\def\RestoreCatcodes{}
\count@=0 %
\loop
  \edef\RestoreCatcodes{%
    \RestoreCatcodes
    \catcode\the\count@=\the\catcode\count@\relax
  }%
\ifnum\count@<255 %
  \advance\count@ 1 %
\repeat

\def\RangeCatcodeInvalid#1#2{%
  \count@=#1\relax
  \loop
    \catcode\count@=15 %
  \ifnum\count@<#2\relax
    \advance\count@ 1 %
  \repeat
}
\def\RangeCatcodeCheck#1#2#3{%
  \count@=#1\relax
  \loop
    \ifnum#3=\catcode\count@
    \else
      \errmessage{%
        Character \the\count@\space
        with wrong catcode \the\catcode\count@\space
        instead of \number#3%
      }%
    \fi
  \ifnum\count@<#2\relax
    \advance\count@ 1 %
  \repeat
}
\def\space{ }
\expandafter\ifx\csname LoadCommand\endcsname\relax
  \def\LoadCommand{\input kvdefinekeys.sty\relax}%
\fi
\def\Test{%
  \RangeCatcodeInvalid{0}{47}%
  \RangeCatcodeInvalid{58}{64}%
  \RangeCatcodeInvalid{91}{96}%
  \RangeCatcodeInvalid{123}{255}%
  \catcode`\@=12 %
  \catcode`\\=0 %
  \catcode`\%=14 %
  \LoadCommand
  \RangeCatcodeCheck{0}{36}{15}%
  \RangeCatcodeCheck{37}{37}{14}%
  \RangeCatcodeCheck{38}{47}{15}%
  \RangeCatcodeCheck{48}{57}{12}%
  \RangeCatcodeCheck{58}{63}{15}%
  \RangeCatcodeCheck{64}{64}{12}%
  \RangeCatcodeCheck{65}{90}{11}%
  \RangeCatcodeCheck{91}{91}{15}%
  \RangeCatcodeCheck{92}{92}{0}%
  \RangeCatcodeCheck{93}{96}{15}%
  \RangeCatcodeCheck{97}{122}{11}%
  \RangeCatcodeCheck{123}{255}{15}%
  \RestoreCatcodes
}
\Test
\csname @@end\endcsname
\end
%    \end{macrocode}
%    \begin{macrocode}
%</test1>
%    \end{macrocode}
%
% \section{Installation}
%
% \subsection{Download}
%
% \paragraph{Package.} This package is available on
% CTAN\footnote{\url{http://ctan.org/pkg/kvdefinekeys}}:
% \begin{description}
% \item[\CTAN{macros/latex/contrib/oberdiek/kvdefinekeys.dtx}] The source file.
% \item[\CTAN{macros/latex/contrib/oberdiek/kvdefinekeys.pdf}] Documentation.
% \end{description}
%
%
% \paragraph{Bundle.} All the packages of the bundle `oberdiek'
% are also available in a TDS compliant ZIP archive. There
% the packages are already unpacked and the documentation files
% are generated. The files and directories obey the TDS standard.
% \begin{description}
% \item[\CTAN{install/macros/latex/contrib/oberdiek.tds.zip}]
% \end{description}
% \emph{TDS} refers to the standard ``A Directory Structure
% for \TeX\ Files'' (\CTAN{tds/tds.pdf}). Directories
% with \xfile{texmf} in their name are usually organized this way.
%
% \subsection{Bundle installation}
%
% \paragraph{Unpacking.} Unpack the \xfile{oberdiek.tds.zip} in the
% TDS tree (also known as \xfile{texmf} tree) of your choice.
% Example (linux):
% \begin{quote}
%   |unzip oberdiek.tds.zip -d ~/texmf|
% \end{quote}
%
% \paragraph{Script installation.}
% Check the directory \xfile{TDS:scripts/oberdiek/} for
% scripts that need further installation steps.
% Package \xpackage{attachfile2} comes with the Perl script
% \xfile{pdfatfi.pl} that should be installed in such a way
% that it can be called as \texttt{pdfatfi}.
% Example (linux):
% \begin{quote}
%   |chmod +x scripts/oberdiek/pdfatfi.pl|\\
%   |cp scripts/oberdiek/pdfatfi.pl /usr/local/bin/|
% \end{quote}
%
% \subsection{Package installation}
%
% \paragraph{Unpacking.} The \xfile{.dtx} file is a self-extracting
% \docstrip\ archive. The files are extracted by running the
% \xfile{.dtx} through \plainTeX:
% \begin{quote}
%   \verb|tex kvdefinekeys.dtx|
% \end{quote}
%
% \paragraph{TDS.} Now the different files must be moved into
% the different directories in your installation TDS tree
% (also known as \xfile{texmf} tree):
% \begin{quote}
% \def\t{^^A
% \begin{tabular}{@{}>{\ttfamily}l@{ $\rightarrow$ }>{\ttfamily}l@{}}
%   kvdefinekeys.sty & tex/generic/oberdiek/kvdefinekeys.sty\\
%   kvdefinekeys.pdf & doc/latex/oberdiek/kvdefinekeys.pdf\\
%   test/kvdefinekeys-test1.tex & doc/latex/oberdiek/test/kvdefinekeys-test1.tex\\
%   kvdefinekeys.dtx & source/latex/oberdiek/kvdefinekeys.dtx\\
% \end{tabular}^^A
% }^^A
% \sbox0{\t}^^A
% \ifdim\wd0>\linewidth
%   \begingroup
%     \advance\linewidth by\leftmargin
%     \advance\linewidth by\rightmargin
%   \edef\x{\endgroup
%     \def\noexpand\lw{\the\linewidth}^^A
%   }\x
%   \def\lwbox{^^A
%     \leavevmode
%     \hbox to \linewidth{^^A
%       \kern-\leftmargin\relax
%       \hss
%       \usebox0
%       \hss
%       \kern-\rightmargin\relax
%     }^^A
%   }^^A
%   \ifdim\wd0>\lw
%     \sbox0{\small\t}^^A
%     \ifdim\wd0>\linewidth
%       \ifdim\wd0>\lw
%         \sbox0{\footnotesize\t}^^A
%         \ifdim\wd0>\linewidth
%           \ifdim\wd0>\lw
%             \sbox0{\scriptsize\t}^^A
%             \ifdim\wd0>\linewidth
%               \ifdim\wd0>\lw
%                 \sbox0{\tiny\t}^^A
%                 \ifdim\wd0>\linewidth
%                   \lwbox
%                 \else
%                   \usebox0
%                 \fi
%               \else
%                 \lwbox
%               \fi
%             \else
%               \usebox0
%             \fi
%           \else
%             \lwbox
%           \fi
%         \else
%           \usebox0
%         \fi
%       \else
%         \lwbox
%       \fi
%     \else
%       \usebox0
%     \fi
%   \else
%     \lwbox
%   \fi
% \else
%   \usebox0
% \fi
% \end{quote}
% If you have a \xfile{docstrip.cfg} that configures and enables \docstrip's
% TDS installing feature, then some files can already be in the right
% place, see the documentation of \docstrip.
%
% \subsection{Refresh file name databases}
%
% If your \TeX~distribution
% (\teTeX, \mikTeX, \dots) relies on file name databases, you must refresh
% these. For example, \teTeX\ users run \verb|texhash| or
% \verb|mktexlsr|.
%
% \subsection{Some details for the interested}
%
% \paragraph{Attached source.}
%
% The PDF documentation on CTAN also includes the
% \xfile{.dtx} source file. It can be extracted by
% AcrobatReader 6 or higher. Another option is \textsf{pdftk},
% e.g. unpack the file into the current directory:
% \begin{quote}
%   \verb|pdftk kvdefinekeys.pdf unpack_files output .|
% \end{quote}
%
% \paragraph{Unpacking with \LaTeX.}
% The \xfile{.dtx} chooses its action depending on the format:
% \begin{description}
% \item[\plainTeX:] Run \docstrip\ and extract the files.
% \item[\LaTeX:] Generate the documentation.
% \end{description}
% If you insist on using \LaTeX\ for \docstrip\ (really,
% \docstrip\ does not need \LaTeX), then inform the autodetect routine
% about your intention:
% \begin{quote}
%   \verb|latex \let\install=y% \iffalse meta-comment
%
% File: kvdefinekeys.dtx
% Version: 2016/05/16 v1.4
% Info: Define keys
%
% Copyright (C) 2010, 2011 by
%    Heiko Oberdiek <heiko.oberdiek at googlemail.com>
%    2016
%    https://github.com/ho-tex/oberdiek/issues
%
% This work may be distributed and/or modified under the
% conditions of the LaTeX Project Public License, either
% version 1.3c of this license or (at your option) any later
% version. This version of this license is in
%    http://www.latex-project.org/lppl/lppl-1-3c.txt
% and the latest version of this license is in
%    http://www.latex-project.org/lppl.txt
% and version 1.3 or later is part of all distributions of
% LaTeX version 2005/12/01 or later.
%
% This work has the LPPL maintenance status "maintained".
%
% This Current Maintainer of this work is Heiko Oberdiek.
%
% The Base Interpreter refers to any `TeX-Format',
% because some files are installed in TDS:tex/generic//.
%
% This work consists of the main source file kvdefinekeys.dtx
% and the derived files
%    kvdefinekeys.sty, kvdefinekeys.pdf, kvdefinekeys.ins, kvdefinekeys.drv,
%    kvdefinekeys-test1.tex.
%
% Distribution:
%    CTAN:macros/latex/contrib/oberdiek/kvdefinekeys.dtx
%    CTAN:macros/latex/contrib/oberdiek/kvdefinekeys.pdf
%
% Unpacking:
%    (a) If kvdefinekeys.ins is present:
%           tex kvdefinekeys.ins
%    (b) Without kvdefinekeys.ins:
%           tex kvdefinekeys.dtx
%    (c) If you insist on using LaTeX
%           latex \let\install=y\input{kvdefinekeys.dtx}
%        (quote the arguments according to the demands of your shell)
%
% Documentation:
%    (a) If kvdefinekeys.drv is present:
%           latex kvdefinekeys.drv
%    (b) Without kvdefinekeys.drv:
%           latex kvdefinekeys.dtx; ...
%    The class ltxdoc loads the configuration file ltxdoc.cfg
%    if available. Here you can specify further options, e.g.
%    use A4 as paper format:
%       \PassOptionsToClass{a4paper}{article}
%
%    Programm calls to get the documentation (example):
%       pdflatex kvdefinekeys.dtx
%       makeindex -s gind.ist kvdefinekeys.idx
%       pdflatex kvdefinekeys.dtx
%       makeindex -s gind.ist kvdefinekeys.idx
%       pdflatex kvdefinekeys.dtx
%
% Installation:
%    TDS:tex/generic/oberdiek/kvdefinekeys.sty
%    TDS:doc/latex/oberdiek/kvdefinekeys.pdf
%    TDS:doc/latex/oberdiek/test/kvdefinekeys-test1.tex
%    TDS:source/latex/oberdiek/kvdefinekeys.dtx
%
%<*ignore>
\begingroup
  \catcode123=1 %
  \catcode125=2 %
  \def\x{LaTeX2e}%
\expandafter\endgroup
\ifcase 0\ifx\install y1\fi\expandafter
         \ifx\csname processbatchFile\endcsname\relax\else1\fi
         \ifx\fmtname\x\else 1\fi\relax
\else\csname fi\endcsname
%</ignore>
%<*install>
\input docstrip.tex
\Msg{************************************************************************}
\Msg{* Installation}
\Msg{* Package: kvdefinekeys 2016/05/16 v1.4 Define keys (HO)}
\Msg{************************************************************************}

\keepsilent
\askforoverwritefalse

\let\MetaPrefix\relax
\preamble

This is a generated file.

Project: kvdefinekeys
Version: 2016/05/16 v1.4

Copyright (C) 2010, 2011 by
   Heiko Oberdiek <heiko.oberdiek at googlemail.com>

This work may be distributed and/or modified under the
conditions of the LaTeX Project Public License, either
version 1.3c of this license or (at your option) any later
version. This version of this license is in
   http://www.latex-project.org/lppl/lppl-1-3c.txt
and the latest version of this license is in
   http://www.latex-project.org/lppl.txt
and version 1.3 or later is part of all distributions of
LaTeX version 2005/12/01 or later.

This work has the LPPL maintenance status "maintained".

This Current Maintainer of this work is Heiko Oberdiek.

The Base Interpreter refers to any `TeX-Format',
because some files are installed in TDS:tex/generic//.

This work consists of the main source file kvdefinekeys.dtx
and the derived files
   kvdefinekeys.sty, kvdefinekeys.pdf, kvdefinekeys.ins, kvdefinekeys.drv,
   kvdefinekeys-test1.tex.

\endpreamble
\let\MetaPrefix\DoubleperCent

\generate{%
  \file{kvdefinekeys.ins}{\from{kvdefinekeys.dtx}{install}}%
  \file{kvdefinekeys.drv}{\from{kvdefinekeys.dtx}{driver}}%
  \usedir{tex/generic/oberdiek}%
  \file{kvdefinekeys.sty}{\from{kvdefinekeys.dtx}{package}}%
  \usedir{doc/latex/oberdiek/test}%
  \file{kvdefinekeys-test1.tex}{\from{kvdefinekeys.dtx}{test1}}%
  \nopreamble
  \nopostamble
  \usedir{source/latex/oberdiek/catalogue}%
  \file{kvdefinekeys.xml}{\from{kvdefinekeys.dtx}{catalogue}}%
}

\catcode32=13\relax% active space
\let =\space%
\Msg{************************************************************************}
\Msg{*}
\Msg{* To finish the installation you have to move the following}
\Msg{* file into a directory searched by TeX:}
\Msg{*}
\Msg{*     kvdefinekeys.sty}
\Msg{*}
\Msg{* To produce the documentation run the file `kvdefinekeys.drv'}
\Msg{* through LaTeX.}
\Msg{*}
\Msg{* Happy TeXing!}
\Msg{*}
\Msg{************************************************************************}

\endbatchfile
%</install>
%<*ignore>
\fi
%</ignore>
%<*driver>
\NeedsTeXFormat{LaTeX2e}
\ProvidesFile{kvdefinekeys.drv}%
  [2016/05/16 v1.4 Define keys (HO)]%
\documentclass{ltxdoc}
\usepackage{holtxdoc}[2011/11/22]
\begin{document}
  \DocInput{kvdefinekeys.dtx}%
\end{document}
%</driver>
% \fi
%
%
% \CharacterTable
%  {Upper-case    \A\B\C\D\E\F\G\H\I\J\K\L\M\N\O\P\Q\R\S\T\U\V\W\X\Y\Z
%   Lower-case    \a\b\c\d\e\f\g\h\i\j\k\l\m\n\o\p\q\r\s\t\u\v\w\x\y\z
%   Digits        \0\1\2\3\4\5\6\7\8\9
%   Exclamation   \!     Double quote  \"     Hash (number) \#
%   Dollar        \$     Percent       \%     Ampersand     \&
%   Acute accent  \'     Left paren    \(     Right paren   \)
%   Asterisk      \*     Plus          \+     Comma         \,
%   Minus         \-     Point         \.     Solidus       \/
%   Colon         \:     Semicolon     \;     Less than     \<
%   Equals        \=     Greater than  \>     Question mark \?
%   Commercial at \@     Left bracket  \[     Backslash     \\
%   Right bracket \]     Circumflex    \^     Underscore    \_
%   Grave accent  \`     Left brace    \{     Vertical bar  \|
%   Right brace   \}     Tilde         \~}
%
% \GetFileInfo{kvdefinekeys.drv}
%
% \title{The \xpackage{kvdefinekeys} package}
% \date{2016/05/16 v1.4}
% \author{Heiko Oberdiek\thanks
% {Please report any issues at https://github.com/ho-tex/oberdiek/issues}\\
% \xemail{heiko.oberdiek at googlemail.com}}
%
% \maketitle
%
% \begin{abstract}
% Package \xpackage{kvdefinekeys} provides \cs{kv@define@key} to define
% keys the same way as \xpackage{keyval}'s \cs{define@key}. However, it
% works also using \iniTeX.
% \end{abstract}
%
% \tableofcontents
%
% \def\M#1{\texttt{\{}\meta{#1}\texttt{\}}}
%
% \section{Documentation}
%
% \subsection{Motivation}
%
% \cs{kvsetkeys} serves as replacement for \xpackage{keyval}'s
% \cs{setkeys}. This package adds macros to define keys, closing
% the gap \cs{kvsetkeys} leaves.
%
% \begin{declcs}{kv@define@key}\,\M{family}\,\M{key}\,[\meta{default}]\,^^A
%   \M{definition}
% \end{declcs}
% Macro \cs{kv@define@key} reimplements \xpackage{keyval}'s
% \cs{define@key}. Differences to the original:
% \begin{itemize}
% \item The defined keys also allow \cs{par} inside values.
% \item Shorthands of package \xpackage{babel} are supported in
%   family and key names.
% \item Macro \cs{kv@define@key} is made robust if
%   \hologo{eTeX}'s \cs{protected} or \hologo{LaTeX}'s
%   \cs{DeclareRobustCommand} are found.
% \end{itemize}
%
% \StopEventually{
% }
%
% \section{Implementation}
%
% \subsection{Identification}
%
%    \begin{macrocode}
%<*package>
%    \end{macrocode}
%    Reload check, especially if the package is not used with \LaTeX.
%    \begin{macrocode}
\begingroup\catcode61\catcode48\catcode32=10\relax%
  \catcode13=5 % ^^M
  \endlinechar=13 %
  \catcode35=6 % #
  \catcode39=12 % '
  \catcode44=12 % ,
  \catcode45=12 % -
  \catcode46=12 % .
  \catcode58=12 % :
  \catcode64=11 % @
  \catcode123=1 % {
  \catcode125=2 % }
  \expandafter\let\expandafter\x\csname ver@kvdefinekeys.sty\endcsname
  \ifx\x\relax % plain-TeX, first loading
  \else
    \def\empty{}%
    \ifx\x\empty % LaTeX, first loading,
      % variable is initialized, but \ProvidesPackage not yet seen
    \else
      \expandafter\ifx\csname PackageInfo\endcsname\relax
        \def\x#1#2{%
          \immediate\write-1{Package #1 Info: #2.}%
        }%
      \else
        \def\x#1#2{\PackageInfo{#1}{#2, stopped}}%
      \fi
      \x{kvdefinekeys}{The package is already loaded}%
      \aftergroup\endinput
    \fi
  \fi
\endgroup%
%    \end{macrocode}
%    Package identification:
%    \begin{macrocode}
\begingroup\catcode61\catcode48\catcode32=10\relax%
  \catcode13=5 % ^^M
  \endlinechar=13 %
  \catcode35=6 % #
  \catcode39=12 % '
  \catcode40=12 % (
  \catcode41=12 % )
  \catcode44=12 % ,
  \catcode45=12 % -
  \catcode46=12 % .
  \catcode47=12 % /
  \catcode58=12 % :
  \catcode64=11 % @
  \catcode91=12 % [
  \catcode93=12 % ]
  \catcode123=1 % {
  \catcode125=2 % }
  \expandafter\ifx\csname ProvidesPackage\endcsname\relax
    \def\x#1#2#3[#4]{\endgroup
      \immediate\write-1{Package: #3 #4}%
      \xdef#1{#4}%
    }%
  \else
    \def\x#1#2[#3]{\endgroup
      #2[{#3}]%
      \ifx#1\@undefined
        \xdef#1{#3}%
      \fi
      \ifx#1\relax
        \xdef#1{#3}%
      \fi
    }%
  \fi
\expandafter\x\csname ver@kvdefinekeys.sty\endcsname
\ProvidesPackage{kvdefinekeys}%
  [2016/05/16 v1.4 Define keys (HO)]%
%    \end{macrocode}
%
%    \begin{macrocode}
\begingroup\catcode61\catcode48\catcode32=10\relax%
  \catcode13=5 % ^^M
  \endlinechar=13 %
  \catcode123=1 % {
  \catcode125=2 % }
  \catcode64=11 % @
  \def\x{\endgroup
    \expandafter\edef\csname KVD@AtEnd\endcsname{%
      \endlinechar=\the\endlinechar\relax
      \catcode13=\the\catcode13\relax
      \catcode32=\the\catcode32\relax
      \catcode35=\the\catcode35\relax
      \catcode61=\the\catcode61\relax
      \catcode64=\the\catcode64\relax
      \catcode123=\the\catcode123\relax
      \catcode125=\the\catcode125\relax
    }%
  }%
\x\catcode61\catcode48\catcode32=10\relax%
\catcode13=5 % ^^M
\endlinechar=13 %
\catcode35=6 % #
\catcode64=11 % @
\catcode123=1 % {
\catcode125=2 % }
\def\TMP@EnsureCode#1#2{%
  \edef\KVD@AtEnd{%
    \KVD@AtEnd
    \catcode#1=\the\catcode#1\relax
  }%
  \catcode#1=#2\relax
}
\TMP@EnsureCode{42}{12}% *
\TMP@EnsureCode{46}{12}% .
\TMP@EnsureCode{47}{12}% /
\TMP@EnsureCode{91}{12}% [
\TMP@EnsureCode{93}{12}% ]
\edef\KVD@AtEnd{\KVD@AtEnd\noexpand\endinput}
%    \end{macrocode}
%
% \subsection{Package loading}
%
%    \begin{macrocode}
\begingroup\expandafter\expandafter\expandafter\endgroup
\expandafter\ifx\csname RequirePackage\endcsname\relax
  \def\TMP@RequirePackage#1[#2]{%
    \begingroup\expandafter\expandafter\expandafter\endgroup
    \expandafter\ifx\csname ver@#1.sty\endcsname\relax
      \input #1.sty\relax
    \fi
  }%
  \TMP@RequirePackage{ltxcmds}[2010/03/01]%
\else
  \RequirePackage{ltxcmds}[2010/03/01]%
\fi
%    \end{macrocode}
%
%
% \subsection{Provide key defining macro}
%
%    \begin{macro}{\kv@define@key}
%    \begin{macrocode}
\ltx@IfUndefined{protected}{%
  \ltx@IfUndefined{DeclareRobustCommand}{%
    \def\kv@define@key#1#2%
  }{%
    \DeclareRobustCommand*{\kv@define@key}[2]%
  }%
}{%
  \protected\def\kv@define@key#1#2%
}%
{%
  \begingroup
    \csname @safe@activestrue\endcsname
    \let\ifincsname\iftrue
    \edef\KVD@temp{\endgroup
      \noexpand\KVD@DefineKey{#1}{#2}%
    }%
  \KVD@temp
}
%    \end{macrocode}
%    \end{macro}
%    \begin{macro}{\KVD@DefineKey}
%    \begin{macrocode}
\def\KVD@DefineKey#1#2{%
  \ltx@ifnextchar[{%
    \KVD@DefineKeyWithDefault{#1}{#2}%
  }{%
    \long\expandafter\def\csname KV@#1@#2\endcsname##1%
  }%
}
%    \end{macrocode}
%    \end{macro}
%    \begin{macro}{\KVD@DefineKeyWithDefault}
%    \begin{macrocode}
\long\def\KVD@DefineKeyWithDefault#1#2[#3]{%
  \expandafter\def\csname KV@#1@#2@default\expandafter\endcsname
  \expandafter{%
    \csname KV@#1@#2\endcsname{#3}%
  }%
  \long\expandafter\def\csname KV@#1@#2\endcsname##1%
}
%    \end{macrocode}
%    \end{macro}
%
%    \begin{macrocode}
\KVD@AtEnd%
%</package>
%    \end{macrocode}
%
% \section{Test}
%
% \subsection{Catcode checks for loading}
%
%    \begin{macrocode}
%<*test1>
%    \end{macrocode}
%    \begin{macrocode}
\catcode`\{=1 %
\catcode`\}=2 %
\catcode`\#=6 %
\catcode`\@=11 %
\expandafter\ifx\csname count@\endcsname\relax
  \countdef\count@=255 %
\fi
\expandafter\ifx\csname @gobble\endcsname\relax
  \long\def\@gobble#1{}%
\fi
\expandafter\ifx\csname @firstofone\endcsname\relax
  \long\def\@firstofone#1{#1}%
\fi
\expandafter\ifx\csname loop\endcsname\relax
  \expandafter\@firstofone
\else
  \expandafter\@gobble
\fi
{%
  \def\loop#1\repeat{%
    \def\body{#1}%
    \iterate
  }%
  \def\iterate{%
    \body
      \let\next\iterate
    \else
      \let\next\relax
    \fi
    \next
  }%
  \let\repeat=\fi
}%
\def\RestoreCatcodes{}
\count@=0 %
\loop
  \edef\RestoreCatcodes{%
    \RestoreCatcodes
    \catcode\the\count@=\the\catcode\count@\relax
  }%
\ifnum\count@<255 %
  \advance\count@ 1 %
\repeat

\def\RangeCatcodeInvalid#1#2{%
  \count@=#1\relax
  \loop
    \catcode\count@=15 %
  \ifnum\count@<#2\relax
    \advance\count@ 1 %
  \repeat
}
\def\RangeCatcodeCheck#1#2#3{%
  \count@=#1\relax
  \loop
    \ifnum#3=\catcode\count@
    \else
      \errmessage{%
        Character \the\count@\space
        with wrong catcode \the\catcode\count@\space
        instead of \number#3%
      }%
    \fi
  \ifnum\count@<#2\relax
    \advance\count@ 1 %
  \repeat
}
\def\space{ }
\expandafter\ifx\csname LoadCommand\endcsname\relax
  \def\LoadCommand{\input kvdefinekeys.sty\relax}%
\fi
\def\Test{%
  \RangeCatcodeInvalid{0}{47}%
  \RangeCatcodeInvalid{58}{64}%
  \RangeCatcodeInvalid{91}{96}%
  \RangeCatcodeInvalid{123}{255}%
  \catcode`\@=12 %
  \catcode`\\=0 %
  \catcode`\%=14 %
  \LoadCommand
  \RangeCatcodeCheck{0}{36}{15}%
  \RangeCatcodeCheck{37}{37}{14}%
  \RangeCatcodeCheck{38}{47}{15}%
  \RangeCatcodeCheck{48}{57}{12}%
  \RangeCatcodeCheck{58}{63}{15}%
  \RangeCatcodeCheck{64}{64}{12}%
  \RangeCatcodeCheck{65}{90}{11}%
  \RangeCatcodeCheck{91}{91}{15}%
  \RangeCatcodeCheck{92}{92}{0}%
  \RangeCatcodeCheck{93}{96}{15}%
  \RangeCatcodeCheck{97}{122}{11}%
  \RangeCatcodeCheck{123}{255}{15}%
  \RestoreCatcodes
}
\Test
\csname @@end\endcsname
\end
%    \end{macrocode}
%    \begin{macrocode}
%</test1>
%    \end{macrocode}
%
% \section{Installation}
%
% \subsection{Download}
%
% \paragraph{Package.} This package is available on
% CTAN\footnote{\url{http://ctan.org/pkg/kvdefinekeys}}:
% \begin{description}
% \item[\CTAN{macros/latex/contrib/oberdiek/kvdefinekeys.dtx}] The source file.
% \item[\CTAN{macros/latex/contrib/oberdiek/kvdefinekeys.pdf}] Documentation.
% \end{description}
%
%
% \paragraph{Bundle.} All the packages of the bundle `oberdiek'
% are also available in a TDS compliant ZIP archive. There
% the packages are already unpacked and the documentation files
% are generated. The files and directories obey the TDS standard.
% \begin{description}
% \item[\CTAN{install/macros/latex/contrib/oberdiek.tds.zip}]
% \end{description}
% \emph{TDS} refers to the standard ``A Directory Structure
% for \TeX\ Files'' (\CTAN{tds/tds.pdf}). Directories
% with \xfile{texmf} in their name are usually organized this way.
%
% \subsection{Bundle installation}
%
% \paragraph{Unpacking.} Unpack the \xfile{oberdiek.tds.zip} in the
% TDS tree (also known as \xfile{texmf} tree) of your choice.
% Example (linux):
% \begin{quote}
%   |unzip oberdiek.tds.zip -d ~/texmf|
% \end{quote}
%
% \paragraph{Script installation.}
% Check the directory \xfile{TDS:scripts/oberdiek/} for
% scripts that need further installation steps.
% Package \xpackage{attachfile2} comes with the Perl script
% \xfile{pdfatfi.pl} that should be installed in such a way
% that it can be called as \texttt{pdfatfi}.
% Example (linux):
% \begin{quote}
%   |chmod +x scripts/oberdiek/pdfatfi.pl|\\
%   |cp scripts/oberdiek/pdfatfi.pl /usr/local/bin/|
% \end{quote}
%
% \subsection{Package installation}
%
% \paragraph{Unpacking.} The \xfile{.dtx} file is a self-extracting
% \docstrip\ archive. The files are extracted by running the
% \xfile{.dtx} through \plainTeX:
% \begin{quote}
%   \verb|tex kvdefinekeys.dtx|
% \end{quote}
%
% \paragraph{TDS.} Now the different files must be moved into
% the different directories in your installation TDS tree
% (also known as \xfile{texmf} tree):
% \begin{quote}
% \def\t{^^A
% \begin{tabular}{@{}>{\ttfamily}l@{ $\rightarrow$ }>{\ttfamily}l@{}}
%   kvdefinekeys.sty & tex/generic/oberdiek/kvdefinekeys.sty\\
%   kvdefinekeys.pdf & doc/latex/oberdiek/kvdefinekeys.pdf\\
%   test/kvdefinekeys-test1.tex & doc/latex/oberdiek/test/kvdefinekeys-test1.tex\\
%   kvdefinekeys.dtx & source/latex/oberdiek/kvdefinekeys.dtx\\
% \end{tabular}^^A
% }^^A
% \sbox0{\t}^^A
% \ifdim\wd0>\linewidth
%   \begingroup
%     \advance\linewidth by\leftmargin
%     \advance\linewidth by\rightmargin
%   \edef\x{\endgroup
%     \def\noexpand\lw{\the\linewidth}^^A
%   }\x
%   \def\lwbox{^^A
%     \leavevmode
%     \hbox to \linewidth{^^A
%       \kern-\leftmargin\relax
%       \hss
%       \usebox0
%       \hss
%       \kern-\rightmargin\relax
%     }^^A
%   }^^A
%   \ifdim\wd0>\lw
%     \sbox0{\small\t}^^A
%     \ifdim\wd0>\linewidth
%       \ifdim\wd0>\lw
%         \sbox0{\footnotesize\t}^^A
%         \ifdim\wd0>\linewidth
%           \ifdim\wd0>\lw
%             \sbox0{\scriptsize\t}^^A
%             \ifdim\wd0>\linewidth
%               \ifdim\wd0>\lw
%                 \sbox0{\tiny\t}^^A
%                 \ifdim\wd0>\linewidth
%                   \lwbox
%                 \else
%                   \usebox0
%                 \fi
%               \else
%                 \lwbox
%               \fi
%             \else
%               \usebox0
%             \fi
%           \else
%             \lwbox
%           \fi
%         \else
%           \usebox0
%         \fi
%       \else
%         \lwbox
%       \fi
%     \else
%       \usebox0
%     \fi
%   \else
%     \lwbox
%   \fi
% \else
%   \usebox0
% \fi
% \end{quote}
% If you have a \xfile{docstrip.cfg} that configures and enables \docstrip's
% TDS installing feature, then some files can already be in the right
% place, see the documentation of \docstrip.
%
% \subsection{Refresh file name databases}
%
% If your \TeX~distribution
% (\teTeX, \mikTeX, \dots) relies on file name databases, you must refresh
% these. For example, \teTeX\ users run \verb|texhash| or
% \verb|mktexlsr|.
%
% \subsection{Some details for the interested}
%
% \paragraph{Attached source.}
%
% The PDF documentation on CTAN also includes the
% \xfile{.dtx} source file. It can be extracted by
% AcrobatReader 6 or higher. Another option is \textsf{pdftk},
% e.g. unpack the file into the current directory:
% \begin{quote}
%   \verb|pdftk kvdefinekeys.pdf unpack_files output .|
% \end{quote}
%
% \paragraph{Unpacking with \LaTeX.}
% The \xfile{.dtx} chooses its action depending on the format:
% \begin{description}
% \item[\plainTeX:] Run \docstrip\ and extract the files.
% \item[\LaTeX:] Generate the documentation.
% \end{description}
% If you insist on using \LaTeX\ for \docstrip\ (really,
% \docstrip\ does not need \LaTeX), then inform the autodetect routine
% about your intention:
% \begin{quote}
%   \verb|latex \let\install=y\input{kvdefinekeys.dtx}|
% \end{quote}
% Do not forget to quote the argument according to the demands
% of your shell.
%
% \paragraph{Generating the documentation.}
% You can use both the \xfile{.dtx} or the \xfile{.drv} to generate
% the documentation. The process can be configured by the
% configuration file \xfile{ltxdoc.cfg}. For instance, put this
% line into this file, if you want to have A4 as paper format:
% \begin{quote}
%   \verb|\PassOptionsToClass{a4paper}{article}|
% \end{quote}
% An example follows how to generate the
% documentation with pdf\LaTeX:
% \begin{quote}
%\begin{verbatim}
%pdflatex kvdefinekeys.dtx
%makeindex -s gind.ist kvdefinekeys.idx
%pdflatex kvdefinekeys.dtx
%makeindex -s gind.ist kvdefinekeys.idx
%pdflatex kvdefinekeys.dtx
%\end{verbatim}
% \end{quote}
%
% \section{Catalogue}
%
% The following XML file can be used as source for the
% \href{http://mirror.ctan.org/help/Catalogue/catalogue.html}{\TeX\ Catalogue}.
% The elements \texttt{caption} and \texttt{description} are imported
% from the original XML file from the Catalogue.
% The name of the XML file in the Catalogue is \xfile{kvdefinekeys.xml}.
%    \begin{macrocode}
%<*catalogue>
<?xml version='1.0' encoding='us-ascii'?>
<!DOCTYPE entry SYSTEM 'catalogue.dtd'>
<entry datestamp='$Date$' modifier='$Author$' id='kvdefinekeys'>
  <name>kvdefinekeys</name>
  <caption>Define keys for use in the kvsetkeys package.</caption>
  <authorref id='auth:oberdiek'/>
  <copyright owner='Heiko Oberdiek' year='2010,2011'/>
  <license type='lppl1.3'/>
  <version number='1.4'/>
  <description>
    The package provides a macro <tt>\kv@define@key</tt> (analogous to
    <xref refid='keyval'>keyval&#x2019;s</xref> <tt>\define@key</tt>, to
    define keys for use by <xref refid='kvsetkeys'>kvsetkeys</xref>.
    <p/>
    The package is part of the <xref refid='oberdiek'>oberdiek</xref>
    bundle.
  </description>
  <documentation details='Package documentation'
      href='ctan:/macros/latex/contrib/oberdiek/kvdefinekeys.pdf'/>
  <ctan file='true' path='/macros/latex/contrib/oberdiek/kvdefinekeys.dtx'/>
  <miktex location='oberdiek'/>
  <texlive location='oberdiek'/>
  <install path='/macros/latex/contrib/oberdiek/oberdiek.tds.zip'/>
</entry>
%</catalogue>
%    \end{macrocode}
%
% \begin{thebibliography}{9}
% \bibitem{keyval}
%   David Carlisle:
%   \textit{The \xpackage{keyval} package};
%   1999/03/16 v1.13;
%   \CTAN{macros/latex/required/graphics/keyval.dtx}.
%
% \end{thebibliography}
%
% \begin{History}
%   \begin{Version}{2010/03/01 v1.0}
%   \item
%     First version.
%   \end{Version}
%   \begin{Version}{2010/08/19 v1.1}
%   \item
%     Documentation fix, no code change.
%   \end{Version}
%   \begin{Version}{2011/01/30 v1.2}
%   \item
%     Already loaded package files are not input in \hologo{plainTeX}.
%   \end{Version}
%   \begin{Version}{2011/04/07 v1.3}
%   \item
%     Support for package \xpackage{babel}'s shorthands added.
%   \item
%     \cs{kv@define@key} is made robust if available.
%   \end{Version}
%   \begin{Version}{2016/05/16 v1.4}
%   \item
%     Documentation updates.
%   \end{Version}
% \end{History}
%
% \PrintIndex
%
% \Finale
\endinput
|
% \end{quote}
% Do not forget to quote the argument according to the demands
% of your shell.
%
% \paragraph{Generating the documentation.}
% You can use both the \xfile{.dtx} or the \xfile{.drv} to generate
% the documentation. The process can be configured by the
% configuration file \xfile{ltxdoc.cfg}. For instance, put this
% line into this file, if you want to have A4 as paper format:
% \begin{quote}
%   \verb|\PassOptionsToClass{a4paper}{article}|
% \end{quote}
% An example follows how to generate the
% documentation with pdf\LaTeX:
% \begin{quote}
%\begin{verbatim}
%pdflatex kvdefinekeys.dtx
%makeindex -s gind.ist kvdefinekeys.idx
%pdflatex kvdefinekeys.dtx
%makeindex -s gind.ist kvdefinekeys.idx
%pdflatex kvdefinekeys.dtx
%\end{verbatim}
% \end{quote}
%
% \section{Catalogue}
%
% The following XML file can be used as source for the
% \href{http://mirror.ctan.org/help/Catalogue/catalogue.html}{\TeX\ Catalogue}.
% The elements \texttt{caption} and \texttt{description} are imported
% from the original XML file from the Catalogue.
% The name of the XML file in the Catalogue is \xfile{kvdefinekeys.xml}.
%    \begin{macrocode}
%<*catalogue>
<?xml version='1.0' encoding='us-ascii'?>
<!DOCTYPE entry SYSTEM 'catalogue.dtd'>
<entry datestamp='$Date$' modifier='$Author$' id='kvdefinekeys'>
  <name>kvdefinekeys</name>
  <caption>Define keys for use in the kvsetkeys package.</caption>
  <authorref id='auth:oberdiek'/>
  <copyright owner='Heiko Oberdiek' year='2010,2011'/>
  <license type='lppl1.3'/>
  <version number='1.4'/>
  <description>
    The package provides a macro <tt>\kv@define@key</tt> (analogous to
    <xref refid='keyval'>keyval&#x2019;s</xref> <tt>\define@key</tt>, to
    define keys for use by <xref refid='kvsetkeys'>kvsetkeys</xref>.
    <p/>
    The package is part of the <xref refid='oberdiek'>oberdiek</xref>
    bundle.
  </description>
  <documentation details='Package documentation'
      href='ctan:/macros/latex/contrib/oberdiek/kvdefinekeys.pdf'/>
  <ctan file='true' path='/macros/latex/contrib/oberdiek/kvdefinekeys.dtx'/>
  <miktex location='oberdiek'/>
  <texlive location='oberdiek'/>
  <install path='/macros/latex/contrib/oberdiek/oberdiek.tds.zip'/>
</entry>
%</catalogue>
%    \end{macrocode}
%
% \begin{thebibliography}{9}
% \bibitem{keyval}
%   David Carlisle:
%   \textit{The \xpackage{keyval} package};
%   1999/03/16 v1.13;
%   \CTAN{macros/latex/required/graphics/keyval.dtx}.
%
% \end{thebibliography}
%
% \begin{History}
%   \begin{Version}{2010/03/01 v1.0}
%   \item
%     First version.
%   \end{Version}
%   \begin{Version}{2010/08/19 v1.1}
%   \item
%     Documentation fix, no code change.
%   \end{Version}
%   \begin{Version}{2011/01/30 v1.2}
%   \item
%     Already loaded package files are not input in \hologo{plainTeX}.
%   \end{Version}
%   \begin{Version}{2011/04/07 v1.3}
%   \item
%     Support for package \xpackage{babel}'s shorthands added.
%   \item
%     \cs{kv@define@key} is made robust if available.
%   \end{Version}
%   \begin{Version}{2016/05/16 v1.4}
%   \item
%     Documentation updates.
%   \end{Version}
% \end{History}
%
% \PrintIndex
%
% \Finale
\endinput

%        (quote the arguments according to the demands of your shell)
%
% Documentation:
%    (a) If kvdefinekeys.drv is present:
%           latex kvdefinekeys.drv
%    (b) Without kvdefinekeys.drv:
%           latex kvdefinekeys.dtx; ...
%    The class ltxdoc loads the configuration file ltxdoc.cfg
%    if available. Here you can specify further options, e.g.
%    use A4 as paper format:
%       \PassOptionsToClass{a4paper}{article}
%
%    Programm calls to get the documentation (example):
%       pdflatex kvdefinekeys.dtx
%       makeindex -s gind.ist kvdefinekeys.idx
%       pdflatex kvdefinekeys.dtx
%       makeindex -s gind.ist kvdefinekeys.idx
%       pdflatex kvdefinekeys.dtx
%
% Installation:
%    TDS:tex/generic/oberdiek/kvdefinekeys.sty
%    TDS:doc/latex/oberdiek/kvdefinekeys.pdf
%    TDS:doc/latex/oberdiek/test/kvdefinekeys-test1.tex
%    TDS:source/latex/oberdiek/kvdefinekeys.dtx
%
%<*ignore>
\begingroup
  \catcode123=1 %
  \catcode125=2 %
  \def\x{LaTeX2e}%
\expandafter\endgroup
\ifcase 0\ifx\install y1\fi\expandafter
         \ifx\csname processbatchFile\endcsname\relax\else1\fi
         \ifx\fmtname\x\else 1\fi\relax
\else\csname fi\endcsname
%</ignore>
%<*install>
\input docstrip.tex
\Msg{************************************************************************}
\Msg{* Installation}
\Msg{* Package: kvdefinekeys 2016/05/16 v1.4 Define keys (HO)}
\Msg{************************************************************************}

\keepsilent
\askforoverwritefalse

\let\MetaPrefix\relax
\preamble

This is a generated file.

Project: kvdefinekeys
Version: 2016/05/16 v1.4

Copyright (C) 2010, 2011 by
   Heiko Oberdiek <heiko.oberdiek at googlemail.com>

This work may be distributed and/or modified under the
conditions of the LaTeX Project Public License, either
version 1.3c of this license or (at your option) any later
version. This version of this license is in
   http://www.latex-project.org/lppl/lppl-1-3c.txt
and the latest version of this license is in
   http://www.latex-project.org/lppl.txt
and version 1.3 or later is part of all distributions of
LaTeX version 2005/12/01 or later.

This work has the LPPL maintenance status "maintained".

This Current Maintainer of this work is Heiko Oberdiek.

The Base Interpreter refers to any `TeX-Format',
because some files are installed in TDS:tex/generic//.

This work consists of the main source file kvdefinekeys.dtx
and the derived files
   kvdefinekeys.sty, kvdefinekeys.pdf, kvdefinekeys.ins, kvdefinekeys.drv,
   kvdefinekeys-test1.tex.

\endpreamble
\let\MetaPrefix\DoubleperCent

\generate{%
  \file{kvdefinekeys.ins}{\from{kvdefinekeys.dtx}{install}}%
  \file{kvdefinekeys.drv}{\from{kvdefinekeys.dtx}{driver}}%
  \usedir{tex/generic/oberdiek}%
  \file{kvdefinekeys.sty}{\from{kvdefinekeys.dtx}{package}}%
  \usedir{doc/latex/oberdiek/test}%
  \file{kvdefinekeys-test1.tex}{\from{kvdefinekeys.dtx}{test1}}%
  \nopreamble
  \nopostamble
  \usedir{source/latex/oberdiek/catalogue}%
  \file{kvdefinekeys.xml}{\from{kvdefinekeys.dtx}{catalogue}}%
}

\catcode32=13\relax% active space
\let =\space%
\Msg{************************************************************************}
\Msg{*}
\Msg{* To finish the installation you have to move the following}
\Msg{* file into a directory searched by TeX:}
\Msg{*}
\Msg{*     kvdefinekeys.sty}
\Msg{*}
\Msg{* To produce the documentation run the file `kvdefinekeys.drv'}
\Msg{* through LaTeX.}
\Msg{*}
\Msg{* Happy TeXing!}
\Msg{*}
\Msg{************************************************************************}

\endbatchfile
%</install>
%<*ignore>
\fi
%</ignore>
%<*driver>
\NeedsTeXFormat{LaTeX2e}
\ProvidesFile{kvdefinekeys.drv}%
  [2016/05/16 v1.4 Define keys (HO)]%
\documentclass{ltxdoc}
\usepackage{holtxdoc}[2011/11/22]
\begin{document}
  \DocInput{kvdefinekeys.dtx}%
\end{document}
%</driver>
% \fi
%
%
% \CharacterTable
%  {Upper-case    \A\B\C\D\E\F\G\H\I\J\K\L\M\N\O\P\Q\R\S\T\U\V\W\X\Y\Z
%   Lower-case    \a\b\c\d\e\f\g\h\i\j\k\l\m\n\o\p\q\r\s\t\u\v\w\x\y\z
%   Digits        \0\1\2\3\4\5\6\7\8\9
%   Exclamation   \!     Double quote  \"     Hash (number) \#
%   Dollar        \$     Percent       \%     Ampersand     \&
%   Acute accent  \'     Left paren    \(     Right paren   \)
%   Asterisk      \*     Plus          \+     Comma         \,
%   Minus         \-     Point         \.     Solidus       \/
%   Colon         \:     Semicolon     \;     Less than     \<
%   Equals        \=     Greater than  \>     Question mark \?
%   Commercial at \@     Left bracket  \[     Backslash     \\
%   Right bracket \]     Circumflex    \^     Underscore    \_
%   Grave accent  \`     Left brace    \{     Vertical bar  \|
%   Right brace   \}     Tilde         \~}
%
% \GetFileInfo{kvdefinekeys.drv}
%
% \title{The \xpackage{kvdefinekeys} package}
% \date{2016/05/16 v1.4}
% \author{Heiko Oberdiek\thanks
% {Please report any issues at https://github.com/ho-tex/oberdiek/issues}\\
% \xemail{heiko.oberdiek at googlemail.com}}
%
% \maketitle
%
% \begin{abstract}
% Package \xpackage{kvdefinekeys} provides \cs{kv@define@key} to define
% keys the same way as \xpackage{keyval}'s \cs{define@key}. However, it
% works also using \iniTeX.
% \end{abstract}
%
% \tableofcontents
%
% \def\M#1{\texttt{\{}\meta{#1}\texttt{\}}}
%
% \section{Documentation}
%
% \subsection{Motivation}
%
% \cs{kvsetkeys} serves as replacement for \xpackage{keyval}'s
% \cs{setkeys}. This package adds macros to define keys, closing
% the gap \cs{kvsetkeys} leaves.
%
% \begin{declcs}{kv@define@key}\,\M{family}\,\M{key}\,[\meta{default}]\,^^A
%   \M{definition}
% \end{declcs}
% Macro \cs{kv@define@key} reimplements \xpackage{keyval}'s
% \cs{define@key}. Differences to the original:
% \begin{itemize}
% \item The defined keys also allow \cs{par} inside values.
% \item Shorthands of package \xpackage{babel} are supported in
%   family and key names.
% \item Macro \cs{kv@define@key} is made robust if
%   \hologo{eTeX}'s \cs{protected} or \hologo{LaTeX}'s
%   \cs{DeclareRobustCommand} are found.
% \end{itemize}
%
% \StopEventually{
% }
%
% \section{Implementation}
%
% \subsection{Identification}
%
%    \begin{macrocode}
%<*package>
%    \end{macrocode}
%    Reload check, especially if the package is not used with \LaTeX.
%    \begin{macrocode}
\begingroup\catcode61\catcode48\catcode32=10\relax%
  \catcode13=5 % ^^M
  \endlinechar=13 %
  \catcode35=6 % #
  \catcode39=12 % '
  \catcode44=12 % ,
  \catcode45=12 % -
  \catcode46=12 % .
  \catcode58=12 % :
  \catcode64=11 % @
  \catcode123=1 % {
  \catcode125=2 % }
  \expandafter\let\expandafter\x\csname ver@kvdefinekeys.sty\endcsname
  \ifx\x\relax % plain-TeX, first loading
  \else
    \def\empty{}%
    \ifx\x\empty % LaTeX, first loading,
      % variable is initialized, but \ProvidesPackage not yet seen
    \else
      \expandafter\ifx\csname PackageInfo\endcsname\relax
        \def\x#1#2{%
          \immediate\write-1{Package #1 Info: #2.}%
        }%
      \else
        \def\x#1#2{\PackageInfo{#1}{#2, stopped}}%
      \fi
      \x{kvdefinekeys}{The package is already loaded}%
      \aftergroup\endinput
    \fi
  \fi
\endgroup%
%    \end{macrocode}
%    Package identification:
%    \begin{macrocode}
\begingroup\catcode61\catcode48\catcode32=10\relax%
  \catcode13=5 % ^^M
  \endlinechar=13 %
  \catcode35=6 % #
  \catcode39=12 % '
  \catcode40=12 % (
  \catcode41=12 % )
  \catcode44=12 % ,
  \catcode45=12 % -
  \catcode46=12 % .
  \catcode47=12 % /
  \catcode58=12 % :
  \catcode64=11 % @
  \catcode91=12 % [
  \catcode93=12 % ]
  \catcode123=1 % {
  \catcode125=2 % }
  \expandafter\ifx\csname ProvidesPackage\endcsname\relax
    \def\x#1#2#3[#4]{\endgroup
      \immediate\write-1{Package: #3 #4}%
      \xdef#1{#4}%
    }%
  \else
    \def\x#1#2[#3]{\endgroup
      #2[{#3}]%
      \ifx#1\@undefined
        \xdef#1{#3}%
      \fi
      \ifx#1\relax
        \xdef#1{#3}%
      \fi
    }%
  \fi
\expandafter\x\csname ver@kvdefinekeys.sty\endcsname
\ProvidesPackage{kvdefinekeys}%
  [2016/05/16 v1.4 Define keys (HO)]%
%    \end{macrocode}
%
%    \begin{macrocode}
\begingroup\catcode61\catcode48\catcode32=10\relax%
  \catcode13=5 % ^^M
  \endlinechar=13 %
  \catcode123=1 % {
  \catcode125=2 % }
  \catcode64=11 % @
  \def\x{\endgroup
    \expandafter\edef\csname KVD@AtEnd\endcsname{%
      \endlinechar=\the\endlinechar\relax
      \catcode13=\the\catcode13\relax
      \catcode32=\the\catcode32\relax
      \catcode35=\the\catcode35\relax
      \catcode61=\the\catcode61\relax
      \catcode64=\the\catcode64\relax
      \catcode123=\the\catcode123\relax
      \catcode125=\the\catcode125\relax
    }%
  }%
\x\catcode61\catcode48\catcode32=10\relax%
\catcode13=5 % ^^M
\endlinechar=13 %
\catcode35=6 % #
\catcode64=11 % @
\catcode123=1 % {
\catcode125=2 % }
\def\TMP@EnsureCode#1#2{%
  \edef\KVD@AtEnd{%
    \KVD@AtEnd
    \catcode#1=\the\catcode#1\relax
  }%
  \catcode#1=#2\relax
}
\TMP@EnsureCode{42}{12}% *
\TMP@EnsureCode{46}{12}% .
\TMP@EnsureCode{47}{12}% /
\TMP@EnsureCode{91}{12}% [
\TMP@EnsureCode{93}{12}% ]
\edef\KVD@AtEnd{\KVD@AtEnd\noexpand\endinput}
%    \end{macrocode}
%
% \subsection{Package loading}
%
%    \begin{macrocode}
\begingroup\expandafter\expandafter\expandafter\endgroup
\expandafter\ifx\csname RequirePackage\endcsname\relax
  \def\TMP@RequirePackage#1[#2]{%
    \begingroup\expandafter\expandafter\expandafter\endgroup
    \expandafter\ifx\csname ver@#1.sty\endcsname\relax
      \input #1.sty\relax
    \fi
  }%
  \TMP@RequirePackage{ltxcmds}[2010/03/01]%
\else
  \RequirePackage{ltxcmds}[2010/03/01]%
\fi
%    \end{macrocode}
%
%
% \subsection{Provide key defining macro}
%
%    \begin{macro}{\kv@define@key}
%    \begin{macrocode}
\ltx@IfUndefined{protected}{%
  \ltx@IfUndefined{DeclareRobustCommand}{%
    \def\kv@define@key#1#2%
  }{%
    \DeclareRobustCommand*{\kv@define@key}[2]%
  }%
}{%
  \protected\def\kv@define@key#1#2%
}%
{%
  \begingroup
    \csname @safe@activestrue\endcsname
    \let\ifincsname\iftrue
    \edef\KVD@temp{\endgroup
      \noexpand\KVD@DefineKey{#1}{#2}%
    }%
  \KVD@temp
}
%    \end{macrocode}
%    \end{macro}
%    \begin{macro}{\KVD@DefineKey}
%    \begin{macrocode}
\def\KVD@DefineKey#1#2{%
  \ltx@ifnextchar[{%
    \KVD@DefineKeyWithDefault{#1}{#2}%
  }{%
    \long\expandafter\def\csname KV@#1@#2\endcsname##1%
  }%
}
%    \end{macrocode}
%    \end{macro}
%    \begin{macro}{\KVD@DefineKeyWithDefault}
%    \begin{macrocode}
\long\def\KVD@DefineKeyWithDefault#1#2[#3]{%
  \expandafter\def\csname KV@#1@#2@default\expandafter\endcsname
  \expandafter{%
    \csname KV@#1@#2\endcsname{#3}%
  }%
  \long\expandafter\def\csname KV@#1@#2\endcsname##1%
}
%    \end{macrocode}
%    \end{macro}
%
%    \begin{macrocode}
\KVD@AtEnd%
%</package>
%    \end{macrocode}
%
% \section{Test}
%
% \subsection{Catcode checks for loading}
%
%    \begin{macrocode}
%<*test1>
%    \end{macrocode}
%    \begin{macrocode}
\catcode`\{=1 %
\catcode`\}=2 %
\catcode`\#=6 %
\catcode`\@=11 %
\expandafter\ifx\csname count@\endcsname\relax
  \countdef\count@=255 %
\fi
\expandafter\ifx\csname @gobble\endcsname\relax
  \long\def\@gobble#1{}%
\fi
\expandafter\ifx\csname @firstofone\endcsname\relax
  \long\def\@firstofone#1{#1}%
\fi
\expandafter\ifx\csname loop\endcsname\relax
  \expandafter\@firstofone
\else
  \expandafter\@gobble
\fi
{%
  \def\loop#1\repeat{%
    \def\body{#1}%
    \iterate
  }%
  \def\iterate{%
    \body
      \let\next\iterate
    \else
      \let\next\relax
    \fi
    \next
  }%
  \let\repeat=\fi
}%
\def\RestoreCatcodes{}
\count@=0 %
\loop
  \edef\RestoreCatcodes{%
    \RestoreCatcodes
    \catcode\the\count@=\the\catcode\count@\relax
  }%
\ifnum\count@<255 %
  \advance\count@ 1 %
\repeat

\def\RangeCatcodeInvalid#1#2{%
  \count@=#1\relax
  \loop
    \catcode\count@=15 %
  \ifnum\count@<#2\relax
    \advance\count@ 1 %
  \repeat
}
\def\RangeCatcodeCheck#1#2#3{%
  \count@=#1\relax
  \loop
    \ifnum#3=\catcode\count@
    \else
      \errmessage{%
        Character \the\count@\space
        with wrong catcode \the\catcode\count@\space
        instead of \number#3%
      }%
    \fi
  \ifnum\count@<#2\relax
    \advance\count@ 1 %
  \repeat
}
\def\space{ }
\expandafter\ifx\csname LoadCommand\endcsname\relax
  \def\LoadCommand{\input kvdefinekeys.sty\relax}%
\fi
\def\Test{%
  \RangeCatcodeInvalid{0}{47}%
  \RangeCatcodeInvalid{58}{64}%
  \RangeCatcodeInvalid{91}{96}%
  \RangeCatcodeInvalid{123}{255}%
  \catcode`\@=12 %
  \catcode`\\=0 %
  \catcode`\%=14 %
  \LoadCommand
  \RangeCatcodeCheck{0}{36}{15}%
  \RangeCatcodeCheck{37}{37}{14}%
  \RangeCatcodeCheck{38}{47}{15}%
  \RangeCatcodeCheck{48}{57}{12}%
  \RangeCatcodeCheck{58}{63}{15}%
  \RangeCatcodeCheck{64}{64}{12}%
  \RangeCatcodeCheck{65}{90}{11}%
  \RangeCatcodeCheck{91}{91}{15}%
  \RangeCatcodeCheck{92}{92}{0}%
  \RangeCatcodeCheck{93}{96}{15}%
  \RangeCatcodeCheck{97}{122}{11}%
  \RangeCatcodeCheck{123}{255}{15}%
  \RestoreCatcodes
}
\Test
\csname @@end\endcsname
\end
%    \end{macrocode}
%    \begin{macrocode}
%</test1>
%    \end{macrocode}
%
% \section{Installation}
%
% \subsection{Download}
%
% \paragraph{Package.} This package is available on
% CTAN\footnote{\url{http://ctan.org/pkg/kvdefinekeys}}:
% \begin{description}
% \item[\CTAN{macros/latex/contrib/oberdiek/kvdefinekeys.dtx}] The source file.
% \item[\CTAN{macros/latex/contrib/oberdiek/kvdefinekeys.pdf}] Documentation.
% \end{description}
%
%
% \paragraph{Bundle.} All the packages of the bundle `oberdiek'
% are also available in a TDS compliant ZIP archive. There
% the packages are already unpacked and the documentation files
% are generated. The files and directories obey the TDS standard.
% \begin{description}
% \item[\CTAN{install/macros/latex/contrib/oberdiek.tds.zip}]
% \end{description}
% \emph{TDS} refers to the standard ``A Directory Structure
% for \TeX\ Files'' (\CTAN{tds/tds.pdf}). Directories
% with \xfile{texmf} in their name are usually organized this way.
%
% \subsection{Bundle installation}
%
% \paragraph{Unpacking.} Unpack the \xfile{oberdiek.tds.zip} in the
% TDS tree (also known as \xfile{texmf} tree) of your choice.
% Example (linux):
% \begin{quote}
%   |unzip oberdiek.tds.zip -d ~/texmf|
% \end{quote}
%
% \paragraph{Script installation.}
% Check the directory \xfile{TDS:scripts/oberdiek/} for
% scripts that need further installation steps.
% Package \xpackage{attachfile2} comes with the Perl script
% \xfile{pdfatfi.pl} that should be installed in such a way
% that it can be called as \texttt{pdfatfi}.
% Example (linux):
% \begin{quote}
%   |chmod +x scripts/oberdiek/pdfatfi.pl|\\
%   |cp scripts/oberdiek/pdfatfi.pl /usr/local/bin/|
% \end{quote}
%
% \subsection{Package installation}
%
% \paragraph{Unpacking.} The \xfile{.dtx} file is a self-extracting
% \docstrip\ archive. The files are extracted by running the
% \xfile{.dtx} through \plainTeX:
% \begin{quote}
%   \verb|tex kvdefinekeys.dtx|
% \end{quote}
%
% \paragraph{TDS.} Now the different files must be moved into
% the different directories in your installation TDS tree
% (also known as \xfile{texmf} tree):
% \begin{quote}
% \def\t{^^A
% \begin{tabular}{@{}>{\ttfamily}l@{ $\rightarrow$ }>{\ttfamily}l@{}}
%   kvdefinekeys.sty & tex/generic/oberdiek/kvdefinekeys.sty\\
%   kvdefinekeys.pdf & doc/latex/oberdiek/kvdefinekeys.pdf\\
%   test/kvdefinekeys-test1.tex & doc/latex/oberdiek/test/kvdefinekeys-test1.tex\\
%   kvdefinekeys.dtx & source/latex/oberdiek/kvdefinekeys.dtx\\
% \end{tabular}^^A
% }^^A
% \sbox0{\t}^^A
% \ifdim\wd0>\linewidth
%   \begingroup
%     \advance\linewidth by\leftmargin
%     \advance\linewidth by\rightmargin
%   \edef\x{\endgroup
%     \def\noexpand\lw{\the\linewidth}^^A
%   }\x
%   \def\lwbox{^^A
%     \leavevmode
%     \hbox to \linewidth{^^A
%       \kern-\leftmargin\relax
%       \hss
%       \usebox0
%       \hss
%       \kern-\rightmargin\relax
%     }^^A
%   }^^A
%   \ifdim\wd0>\lw
%     \sbox0{\small\t}^^A
%     \ifdim\wd0>\linewidth
%       \ifdim\wd0>\lw
%         \sbox0{\footnotesize\t}^^A
%         \ifdim\wd0>\linewidth
%           \ifdim\wd0>\lw
%             \sbox0{\scriptsize\t}^^A
%             \ifdim\wd0>\linewidth
%               \ifdim\wd0>\lw
%                 \sbox0{\tiny\t}^^A
%                 \ifdim\wd0>\linewidth
%                   \lwbox
%                 \else
%                   \usebox0
%                 \fi
%               \else
%                 \lwbox
%               \fi
%             \else
%               \usebox0
%             \fi
%           \else
%             \lwbox
%           \fi
%         \else
%           \usebox0
%         \fi
%       \else
%         \lwbox
%       \fi
%     \else
%       \usebox0
%     \fi
%   \else
%     \lwbox
%   \fi
% \else
%   \usebox0
% \fi
% \end{quote}
% If you have a \xfile{docstrip.cfg} that configures and enables \docstrip's
% TDS installing feature, then some files can already be in the right
% place, see the documentation of \docstrip.
%
% \subsection{Refresh file name databases}
%
% If your \TeX~distribution
% (\teTeX, \mikTeX, \dots) relies on file name databases, you must refresh
% these. For example, \teTeX\ users run \verb|texhash| or
% \verb|mktexlsr|.
%
% \subsection{Some details for the interested}
%
% \paragraph{Attached source.}
%
% The PDF documentation on CTAN also includes the
% \xfile{.dtx} source file. It can be extracted by
% AcrobatReader 6 or higher. Another option is \textsf{pdftk},
% e.g. unpack the file into the current directory:
% \begin{quote}
%   \verb|pdftk kvdefinekeys.pdf unpack_files output .|
% \end{quote}
%
% \paragraph{Unpacking with \LaTeX.}
% The \xfile{.dtx} chooses its action depending on the format:
% \begin{description}
% \item[\plainTeX:] Run \docstrip\ and extract the files.
% \item[\LaTeX:] Generate the documentation.
% \end{description}
% If you insist on using \LaTeX\ for \docstrip\ (really,
% \docstrip\ does not need \LaTeX), then inform the autodetect routine
% about your intention:
% \begin{quote}
%   \verb|latex \let\install=y% \iffalse meta-comment
%
% File: kvdefinekeys.dtx
% Version: 2016/05/16 v1.4
% Info: Define keys
%
% Copyright (C) 2010, 2011 by
%    Heiko Oberdiek <heiko.oberdiek at googlemail.com>
%    2016
%    https://github.com/ho-tex/oberdiek/issues
%
% This work may be distributed and/or modified under the
% conditions of the LaTeX Project Public License, either
% version 1.3c of this license or (at your option) any later
% version. This version of this license is in
%    http://www.latex-project.org/lppl/lppl-1-3c.txt
% and the latest version of this license is in
%    http://www.latex-project.org/lppl.txt
% and version 1.3 or later is part of all distributions of
% LaTeX version 2005/12/01 or later.
%
% This work has the LPPL maintenance status "maintained".
%
% This Current Maintainer of this work is Heiko Oberdiek.
%
% The Base Interpreter refers to any `TeX-Format',
% because some files are installed in TDS:tex/generic//.
%
% This work consists of the main source file kvdefinekeys.dtx
% and the derived files
%    kvdefinekeys.sty, kvdefinekeys.pdf, kvdefinekeys.ins, kvdefinekeys.drv,
%    kvdefinekeys-test1.tex.
%
% Distribution:
%    CTAN:macros/latex/contrib/oberdiek/kvdefinekeys.dtx
%    CTAN:macros/latex/contrib/oberdiek/kvdefinekeys.pdf
%
% Unpacking:
%    (a) If kvdefinekeys.ins is present:
%           tex kvdefinekeys.ins
%    (b) Without kvdefinekeys.ins:
%           tex kvdefinekeys.dtx
%    (c) If you insist on using LaTeX
%           latex \let\install=y% \iffalse meta-comment
%
% File: kvdefinekeys.dtx
% Version: 2016/05/16 v1.4
% Info: Define keys
%
% Copyright (C) 2010, 2011 by
%    Heiko Oberdiek <heiko.oberdiek at googlemail.com>
%    2016
%    https://github.com/ho-tex/oberdiek/issues
%
% This work may be distributed and/or modified under the
% conditions of the LaTeX Project Public License, either
% version 1.3c of this license or (at your option) any later
% version. This version of this license is in
%    http://www.latex-project.org/lppl/lppl-1-3c.txt
% and the latest version of this license is in
%    http://www.latex-project.org/lppl.txt
% and version 1.3 or later is part of all distributions of
% LaTeX version 2005/12/01 or later.
%
% This work has the LPPL maintenance status "maintained".
%
% This Current Maintainer of this work is Heiko Oberdiek.
%
% The Base Interpreter refers to any `TeX-Format',
% because some files are installed in TDS:tex/generic//.
%
% This work consists of the main source file kvdefinekeys.dtx
% and the derived files
%    kvdefinekeys.sty, kvdefinekeys.pdf, kvdefinekeys.ins, kvdefinekeys.drv,
%    kvdefinekeys-test1.tex.
%
% Distribution:
%    CTAN:macros/latex/contrib/oberdiek/kvdefinekeys.dtx
%    CTAN:macros/latex/contrib/oberdiek/kvdefinekeys.pdf
%
% Unpacking:
%    (a) If kvdefinekeys.ins is present:
%           tex kvdefinekeys.ins
%    (b) Without kvdefinekeys.ins:
%           tex kvdefinekeys.dtx
%    (c) If you insist on using LaTeX
%           latex \let\install=y\input{kvdefinekeys.dtx}
%        (quote the arguments according to the demands of your shell)
%
% Documentation:
%    (a) If kvdefinekeys.drv is present:
%           latex kvdefinekeys.drv
%    (b) Without kvdefinekeys.drv:
%           latex kvdefinekeys.dtx; ...
%    The class ltxdoc loads the configuration file ltxdoc.cfg
%    if available. Here you can specify further options, e.g.
%    use A4 as paper format:
%       \PassOptionsToClass{a4paper}{article}
%
%    Programm calls to get the documentation (example):
%       pdflatex kvdefinekeys.dtx
%       makeindex -s gind.ist kvdefinekeys.idx
%       pdflatex kvdefinekeys.dtx
%       makeindex -s gind.ist kvdefinekeys.idx
%       pdflatex kvdefinekeys.dtx
%
% Installation:
%    TDS:tex/generic/oberdiek/kvdefinekeys.sty
%    TDS:doc/latex/oberdiek/kvdefinekeys.pdf
%    TDS:doc/latex/oberdiek/test/kvdefinekeys-test1.tex
%    TDS:source/latex/oberdiek/kvdefinekeys.dtx
%
%<*ignore>
\begingroup
  \catcode123=1 %
  \catcode125=2 %
  \def\x{LaTeX2e}%
\expandafter\endgroup
\ifcase 0\ifx\install y1\fi\expandafter
         \ifx\csname processbatchFile\endcsname\relax\else1\fi
         \ifx\fmtname\x\else 1\fi\relax
\else\csname fi\endcsname
%</ignore>
%<*install>
\input docstrip.tex
\Msg{************************************************************************}
\Msg{* Installation}
\Msg{* Package: kvdefinekeys 2016/05/16 v1.4 Define keys (HO)}
\Msg{************************************************************************}

\keepsilent
\askforoverwritefalse

\let\MetaPrefix\relax
\preamble

This is a generated file.

Project: kvdefinekeys
Version: 2016/05/16 v1.4

Copyright (C) 2010, 2011 by
   Heiko Oberdiek <heiko.oberdiek at googlemail.com>

This work may be distributed and/or modified under the
conditions of the LaTeX Project Public License, either
version 1.3c of this license or (at your option) any later
version. This version of this license is in
   http://www.latex-project.org/lppl/lppl-1-3c.txt
and the latest version of this license is in
   http://www.latex-project.org/lppl.txt
and version 1.3 or later is part of all distributions of
LaTeX version 2005/12/01 or later.

This work has the LPPL maintenance status "maintained".

This Current Maintainer of this work is Heiko Oberdiek.

The Base Interpreter refers to any `TeX-Format',
because some files are installed in TDS:tex/generic//.

This work consists of the main source file kvdefinekeys.dtx
and the derived files
   kvdefinekeys.sty, kvdefinekeys.pdf, kvdefinekeys.ins, kvdefinekeys.drv,
   kvdefinekeys-test1.tex.

\endpreamble
\let\MetaPrefix\DoubleperCent

\generate{%
  \file{kvdefinekeys.ins}{\from{kvdefinekeys.dtx}{install}}%
  \file{kvdefinekeys.drv}{\from{kvdefinekeys.dtx}{driver}}%
  \usedir{tex/generic/oberdiek}%
  \file{kvdefinekeys.sty}{\from{kvdefinekeys.dtx}{package}}%
  \usedir{doc/latex/oberdiek/test}%
  \file{kvdefinekeys-test1.tex}{\from{kvdefinekeys.dtx}{test1}}%
  \nopreamble
  \nopostamble
  \usedir{source/latex/oberdiek/catalogue}%
  \file{kvdefinekeys.xml}{\from{kvdefinekeys.dtx}{catalogue}}%
}

\catcode32=13\relax% active space
\let =\space%
\Msg{************************************************************************}
\Msg{*}
\Msg{* To finish the installation you have to move the following}
\Msg{* file into a directory searched by TeX:}
\Msg{*}
\Msg{*     kvdefinekeys.sty}
\Msg{*}
\Msg{* To produce the documentation run the file `kvdefinekeys.drv'}
\Msg{* through LaTeX.}
\Msg{*}
\Msg{* Happy TeXing!}
\Msg{*}
\Msg{************************************************************************}

\endbatchfile
%</install>
%<*ignore>
\fi
%</ignore>
%<*driver>
\NeedsTeXFormat{LaTeX2e}
\ProvidesFile{kvdefinekeys.drv}%
  [2016/05/16 v1.4 Define keys (HO)]%
\documentclass{ltxdoc}
\usepackage{holtxdoc}[2011/11/22]
\begin{document}
  \DocInput{kvdefinekeys.dtx}%
\end{document}
%</driver>
% \fi
%
%
% \CharacterTable
%  {Upper-case    \A\B\C\D\E\F\G\H\I\J\K\L\M\N\O\P\Q\R\S\T\U\V\W\X\Y\Z
%   Lower-case    \a\b\c\d\e\f\g\h\i\j\k\l\m\n\o\p\q\r\s\t\u\v\w\x\y\z
%   Digits        \0\1\2\3\4\5\6\7\8\9
%   Exclamation   \!     Double quote  \"     Hash (number) \#
%   Dollar        \$     Percent       \%     Ampersand     \&
%   Acute accent  \'     Left paren    \(     Right paren   \)
%   Asterisk      \*     Plus          \+     Comma         \,
%   Minus         \-     Point         \.     Solidus       \/
%   Colon         \:     Semicolon     \;     Less than     \<
%   Equals        \=     Greater than  \>     Question mark \?
%   Commercial at \@     Left bracket  \[     Backslash     \\
%   Right bracket \]     Circumflex    \^     Underscore    \_
%   Grave accent  \`     Left brace    \{     Vertical bar  \|
%   Right brace   \}     Tilde         \~}
%
% \GetFileInfo{kvdefinekeys.drv}
%
% \title{The \xpackage{kvdefinekeys} package}
% \date{2016/05/16 v1.4}
% \author{Heiko Oberdiek\thanks
% {Please report any issues at https://github.com/ho-tex/oberdiek/issues}\\
% \xemail{heiko.oberdiek at googlemail.com}}
%
% \maketitle
%
% \begin{abstract}
% Package \xpackage{kvdefinekeys} provides \cs{kv@define@key} to define
% keys the same way as \xpackage{keyval}'s \cs{define@key}. However, it
% works also using \iniTeX.
% \end{abstract}
%
% \tableofcontents
%
% \def\M#1{\texttt{\{}\meta{#1}\texttt{\}}}
%
% \section{Documentation}
%
% \subsection{Motivation}
%
% \cs{kvsetkeys} serves as replacement for \xpackage{keyval}'s
% \cs{setkeys}. This package adds macros to define keys, closing
% the gap \cs{kvsetkeys} leaves.
%
% \begin{declcs}{kv@define@key}\,\M{family}\,\M{key}\,[\meta{default}]\,^^A
%   \M{definition}
% \end{declcs}
% Macro \cs{kv@define@key} reimplements \xpackage{keyval}'s
% \cs{define@key}. Differences to the original:
% \begin{itemize}
% \item The defined keys also allow \cs{par} inside values.
% \item Shorthands of package \xpackage{babel} are supported in
%   family and key names.
% \item Macro \cs{kv@define@key} is made robust if
%   \hologo{eTeX}'s \cs{protected} or \hologo{LaTeX}'s
%   \cs{DeclareRobustCommand} are found.
% \end{itemize}
%
% \StopEventually{
% }
%
% \section{Implementation}
%
% \subsection{Identification}
%
%    \begin{macrocode}
%<*package>
%    \end{macrocode}
%    Reload check, especially if the package is not used with \LaTeX.
%    \begin{macrocode}
\begingroup\catcode61\catcode48\catcode32=10\relax%
  \catcode13=5 % ^^M
  \endlinechar=13 %
  \catcode35=6 % #
  \catcode39=12 % '
  \catcode44=12 % ,
  \catcode45=12 % -
  \catcode46=12 % .
  \catcode58=12 % :
  \catcode64=11 % @
  \catcode123=1 % {
  \catcode125=2 % }
  \expandafter\let\expandafter\x\csname ver@kvdefinekeys.sty\endcsname
  \ifx\x\relax % plain-TeX, first loading
  \else
    \def\empty{}%
    \ifx\x\empty % LaTeX, first loading,
      % variable is initialized, but \ProvidesPackage not yet seen
    \else
      \expandafter\ifx\csname PackageInfo\endcsname\relax
        \def\x#1#2{%
          \immediate\write-1{Package #1 Info: #2.}%
        }%
      \else
        \def\x#1#2{\PackageInfo{#1}{#2, stopped}}%
      \fi
      \x{kvdefinekeys}{The package is already loaded}%
      \aftergroup\endinput
    \fi
  \fi
\endgroup%
%    \end{macrocode}
%    Package identification:
%    \begin{macrocode}
\begingroup\catcode61\catcode48\catcode32=10\relax%
  \catcode13=5 % ^^M
  \endlinechar=13 %
  \catcode35=6 % #
  \catcode39=12 % '
  \catcode40=12 % (
  \catcode41=12 % )
  \catcode44=12 % ,
  \catcode45=12 % -
  \catcode46=12 % .
  \catcode47=12 % /
  \catcode58=12 % :
  \catcode64=11 % @
  \catcode91=12 % [
  \catcode93=12 % ]
  \catcode123=1 % {
  \catcode125=2 % }
  \expandafter\ifx\csname ProvidesPackage\endcsname\relax
    \def\x#1#2#3[#4]{\endgroup
      \immediate\write-1{Package: #3 #4}%
      \xdef#1{#4}%
    }%
  \else
    \def\x#1#2[#3]{\endgroup
      #2[{#3}]%
      \ifx#1\@undefined
        \xdef#1{#3}%
      \fi
      \ifx#1\relax
        \xdef#1{#3}%
      \fi
    }%
  \fi
\expandafter\x\csname ver@kvdefinekeys.sty\endcsname
\ProvidesPackage{kvdefinekeys}%
  [2016/05/16 v1.4 Define keys (HO)]%
%    \end{macrocode}
%
%    \begin{macrocode}
\begingroup\catcode61\catcode48\catcode32=10\relax%
  \catcode13=5 % ^^M
  \endlinechar=13 %
  \catcode123=1 % {
  \catcode125=2 % }
  \catcode64=11 % @
  \def\x{\endgroup
    \expandafter\edef\csname KVD@AtEnd\endcsname{%
      \endlinechar=\the\endlinechar\relax
      \catcode13=\the\catcode13\relax
      \catcode32=\the\catcode32\relax
      \catcode35=\the\catcode35\relax
      \catcode61=\the\catcode61\relax
      \catcode64=\the\catcode64\relax
      \catcode123=\the\catcode123\relax
      \catcode125=\the\catcode125\relax
    }%
  }%
\x\catcode61\catcode48\catcode32=10\relax%
\catcode13=5 % ^^M
\endlinechar=13 %
\catcode35=6 % #
\catcode64=11 % @
\catcode123=1 % {
\catcode125=2 % }
\def\TMP@EnsureCode#1#2{%
  \edef\KVD@AtEnd{%
    \KVD@AtEnd
    \catcode#1=\the\catcode#1\relax
  }%
  \catcode#1=#2\relax
}
\TMP@EnsureCode{42}{12}% *
\TMP@EnsureCode{46}{12}% .
\TMP@EnsureCode{47}{12}% /
\TMP@EnsureCode{91}{12}% [
\TMP@EnsureCode{93}{12}% ]
\edef\KVD@AtEnd{\KVD@AtEnd\noexpand\endinput}
%    \end{macrocode}
%
% \subsection{Package loading}
%
%    \begin{macrocode}
\begingroup\expandafter\expandafter\expandafter\endgroup
\expandafter\ifx\csname RequirePackage\endcsname\relax
  \def\TMP@RequirePackage#1[#2]{%
    \begingroup\expandafter\expandafter\expandafter\endgroup
    \expandafter\ifx\csname ver@#1.sty\endcsname\relax
      \input #1.sty\relax
    \fi
  }%
  \TMP@RequirePackage{ltxcmds}[2010/03/01]%
\else
  \RequirePackage{ltxcmds}[2010/03/01]%
\fi
%    \end{macrocode}
%
%
% \subsection{Provide key defining macro}
%
%    \begin{macro}{\kv@define@key}
%    \begin{macrocode}
\ltx@IfUndefined{protected}{%
  \ltx@IfUndefined{DeclareRobustCommand}{%
    \def\kv@define@key#1#2%
  }{%
    \DeclareRobustCommand*{\kv@define@key}[2]%
  }%
}{%
  \protected\def\kv@define@key#1#2%
}%
{%
  \begingroup
    \csname @safe@activestrue\endcsname
    \let\ifincsname\iftrue
    \edef\KVD@temp{\endgroup
      \noexpand\KVD@DefineKey{#1}{#2}%
    }%
  \KVD@temp
}
%    \end{macrocode}
%    \end{macro}
%    \begin{macro}{\KVD@DefineKey}
%    \begin{macrocode}
\def\KVD@DefineKey#1#2{%
  \ltx@ifnextchar[{%
    \KVD@DefineKeyWithDefault{#1}{#2}%
  }{%
    \long\expandafter\def\csname KV@#1@#2\endcsname##1%
  }%
}
%    \end{macrocode}
%    \end{macro}
%    \begin{macro}{\KVD@DefineKeyWithDefault}
%    \begin{macrocode}
\long\def\KVD@DefineKeyWithDefault#1#2[#3]{%
  \expandafter\def\csname KV@#1@#2@default\expandafter\endcsname
  \expandafter{%
    \csname KV@#1@#2\endcsname{#3}%
  }%
  \long\expandafter\def\csname KV@#1@#2\endcsname##1%
}
%    \end{macrocode}
%    \end{macro}
%
%    \begin{macrocode}
\KVD@AtEnd%
%</package>
%    \end{macrocode}
%
% \section{Test}
%
% \subsection{Catcode checks for loading}
%
%    \begin{macrocode}
%<*test1>
%    \end{macrocode}
%    \begin{macrocode}
\catcode`\{=1 %
\catcode`\}=2 %
\catcode`\#=6 %
\catcode`\@=11 %
\expandafter\ifx\csname count@\endcsname\relax
  \countdef\count@=255 %
\fi
\expandafter\ifx\csname @gobble\endcsname\relax
  \long\def\@gobble#1{}%
\fi
\expandafter\ifx\csname @firstofone\endcsname\relax
  \long\def\@firstofone#1{#1}%
\fi
\expandafter\ifx\csname loop\endcsname\relax
  \expandafter\@firstofone
\else
  \expandafter\@gobble
\fi
{%
  \def\loop#1\repeat{%
    \def\body{#1}%
    \iterate
  }%
  \def\iterate{%
    \body
      \let\next\iterate
    \else
      \let\next\relax
    \fi
    \next
  }%
  \let\repeat=\fi
}%
\def\RestoreCatcodes{}
\count@=0 %
\loop
  \edef\RestoreCatcodes{%
    \RestoreCatcodes
    \catcode\the\count@=\the\catcode\count@\relax
  }%
\ifnum\count@<255 %
  \advance\count@ 1 %
\repeat

\def\RangeCatcodeInvalid#1#2{%
  \count@=#1\relax
  \loop
    \catcode\count@=15 %
  \ifnum\count@<#2\relax
    \advance\count@ 1 %
  \repeat
}
\def\RangeCatcodeCheck#1#2#3{%
  \count@=#1\relax
  \loop
    \ifnum#3=\catcode\count@
    \else
      \errmessage{%
        Character \the\count@\space
        with wrong catcode \the\catcode\count@\space
        instead of \number#3%
      }%
    \fi
  \ifnum\count@<#2\relax
    \advance\count@ 1 %
  \repeat
}
\def\space{ }
\expandafter\ifx\csname LoadCommand\endcsname\relax
  \def\LoadCommand{\input kvdefinekeys.sty\relax}%
\fi
\def\Test{%
  \RangeCatcodeInvalid{0}{47}%
  \RangeCatcodeInvalid{58}{64}%
  \RangeCatcodeInvalid{91}{96}%
  \RangeCatcodeInvalid{123}{255}%
  \catcode`\@=12 %
  \catcode`\\=0 %
  \catcode`\%=14 %
  \LoadCommand
  \RangeCatcodeCheck{0}{36}{15}%
  \RangeCatcodeCheck{37}{37}{14}%
  \RangeCatcodeCheck{38}{47}{15}%
  \RangeCatcodeCheck{48}{57}{12}%
  \RangeCatcodeCheck{58}{63}{15}%
  \RangeCatcodeCheck{64}{64}{12}%
  \RangeCatcodeCheck{65}{90}{11}%
  \RangeCatcodeCheck{91}{91}{15}%
  \RangeCatcodeCheck{92}{92}{0}%
  \RangeCatcodeCheck{93}{96}{15}%
  \RangeCatcodeCheck{97}{122}{11}%
  \RangeCatcodeCheck{123}{255}{15}%
  \RestoreCatcodes
}
\Test
\csname @@end\endcsname
\end
%    \end{macrocode}
%    \begin{macrocode}
%</test1>
%    \end{macrocode}
%
% \section{Installation}
%
% \subsection{Download}
%
% \paragraph{Package.} This package is available on
% CTAN\footnote{\url{http://ctan.org/pkg/kvdefinekeys}}:
% \begin{description}
% \item[\CTAN{macros/latex/contrib/oberdiek/kvdefinekeys.dtx}] The source file.
% \item[\CTAN{macros/latex/contrib/oberdiek/kvdefinekeys.pdf}] Documentation.
% \end{description}
%
%
% \paragraph{Bundle.} All the packages of the bundle `oberdiek'
% are also available in a TDS compliant ZIP archive. There
% the packages are already unpacked and the documentation files
% are generated. The files and directories obey the TDS standard.
% \begin{description}
% \item[\CTAN{install/macros/latex/contrib/oberdiek.tds.zip}]
% \end{description}
% \emph{TDS} refers to the standard ``A Directory Structure
% for \TeX\ Files'' (\CTAN{tds/tds.pdf}). Directories
% with \xfile{texmf} in their name are usually organized this way.
%
% \subsection{Bundle installation}
%
% \paragraph{Unpacking.} Unpack the \xfile{oberdiek.tds.zip} in the
% TDS tree (also known as \xfile{texmf} tree) of your choice.
% Example (linux):
% \begin{quote}
%   |unzip oberdiek.tds.zip -d ~/texmf|
% \end{quote}
%
% \paragraph{Script installation.}
% Check the directory \xfile{TDS:scripts/oberdiek/} for
% scripts that need further installation steps.
% Package \xpackage{attachfile2} comes with the Perl script
% \xfile{pdfatfi.pl} that should be installed in such a way
% that it can be called as \texttt{pdfatfi}.
% Example (linux):
% \begin{quote}
%   |chmod +x scripts/oberdiek/pdfatfi.pl|\\
%   |cp scripts/oberdiek/pdfatfi.pl /usr/local/bin/|
% \end{quote}
%
% \subsection{Package installation}
%
% \paragraph{Unpacking.} The \xfile{.dtx} file is a self-extracting
% \docstrip\ archive. The files are extracted by running the
% \xfile{.dtx} through \plainTeX:
% \begin{quote}
%   \verb|tex kvdefinekeys.dtx|
% \end{quote}
%
% \paragraph{TDS.} Now the different files must be moved into
% the different directories in your installation TDS tree
% (also known as \xfile{texmf} tree):
% \begin{quote}
% \def\t{^^A
% \begin{tabular}{@{}>{\ttfamily}l@{ $\rightarrow$ }>{\ttfamily}l@{}}
%   kvdefinekeys.sty & tex/generic/oberdiek/kvdefinekeys.sty\\
%   kvdefinekeys.pdf & doc/latex/oberdiek/kvdefinekeys.pdf\\
%   test/kvdefinekeys-test1.tex & doc/latex/oberdiek/test/kvdefinekeys-test1.tex\\
%   kvdefinekeys.dtx & source/latex/oberdiek/kvdefinekeys.dtx\\
% \end{tabular}^^A
% }^^A
% \sbox0{\t}^^A
% \ifdim\wd0>\linewidth
%   \begingroup
%     \advance\linewidth by\leftmargin
%     \advance\linewidth by\rightmargin
%   \edef\x{\endgroup
%     \def\noexpand\lw{\the\linewidth}^^A
%   }\x
%   \def\lwbox{^^A
%     \leavevmode
%     \hbox to \linewidth{^^A
%       \kern-\leftmargin\relax
%       \hss
%       \usebox0
%       \hss
%       \kern-\rightmargin\relax
%     }^^A
%   }^^A
%   \ifdim\wd0>\lw
%     \sbox0{\small\t}^^A
%     \ifdim\wd0>\linewidth
%       \ifdim\wd0>\lw
%         \sbox0{\footnotesize\t}^^A
%         \ifdim\wd0>\linewidth
%           \ifdim\wd0>\lw
%             \sbox0{\scriptsize\t}^^A
%             \ifdim\wd0>\linewidth
%               \ifdim\wd0>\lw
%                 \sbox0{\tiny\t}^^A
%                 \ifdim\wd0>\linewidth
%                   \lwbox
%                 \else
%                   \usebox0
%                 \fi
%               \else
%                 \lwbox
%               \fi
%             \else
%               \usebox0
%             \fi
%           \else
%             \lwbox
%           \fi
%         \else
%           \usebox0
%         \fi
%       \else
%         \lwbox
%       \fi
%     \else
%       \usebox0
%     \fi
%   \else
%     \lwbox
%   \fi
% \else
%   \usebox0
% \fi
% \end{quote}
% If you have a \xfile{docstrip.cfg} that configures and enables \docstrip's
% TDS installing feature, then some files can already be in the right
% place, see the documentation of \docstrip.
%
% \subsection{Refresh file name databases}
%
% If your \TeX~distribution
% (\teTeX, \mikTeX, \dots) relies on file name databases, you must refresh
% these. For example, \teTeX\ users run \verb|texhash| or
% \verb|mktexlsr|.
%
% \subsection{Some details for the interested}
%
% \paragraph{Attached source.}
%
% The PDF documentation on CTAN also includes the
% \xfile{.dtx} source file. It can be extracted by
% AcrobatReader 6 or higher. Another option is \textsf{pdftk},
% e.g. unpack the file into the current directory:
% \begin{quote}
%   \verb|pdftk kvdefinekeys.pdf unpack_files output .|
% \end{quote}
%
% \paragraph{Unpacking with \LaTeX.}
% The \xfile{.dtx} chooses its action depending on the format:
% \begin{description}
% \item[\plainTeX:] Run \docstrip\ and extract the files.
% \item[\LaTeX:] Generate the documentation.
% \end{description}
% If you insist on using \LaTeX\ for \docstrip\ (really,
% \docstrip\ does not need \LaTeX), then inform the autodetect routine
% about your intention:
% \begin{quote}
%   \verb|latex \let\install=y\input{kvdefinekeys.dtx}|
% \end{quote}
% Do not forget to quote the argument according to the demands
% of your shell.
%
% \paragraph{Generating the documentation.}
% You can use both the \xfile{.dtx} or the \xfile{.drv} to generate
% the documentation. The process can be configured by the
% configuration file \xfile{ltxdoc.cfg}. For instance, put this
% line into this file, if you want to have A4 as paper format:
% \begin{quote}
%   \verb|\PassOptionsToClass{a4paper}{article}|
% \end{quote}
% An example follows how to generate the
% documentation with pdf\LaTeX:
% \begin{quote}
%\begin{verbatim}
%pdflatex kvdefinekeys.dtx
%makeindex -s gind.ist kvdefinekeys.idx
%pdflatex kvdefinekeys.dtx
%makeindex -s gind.ist kvdefinekeys.idx
%pdflatex kvdefinekeys.dtx
%\end{verbatim}
% \end{quote}
%
% \section{Catalogue}
%
% The following XML file can be used as source for the
% \href{http://mirror.ctan.org/help/Catalogue/catalogue.html}{\TeX\ Catalogue}.
% The elements \texttt{caption} and \texttt{description} are imported
% from the original XML file from the Catalogue.
% The name of the XML file in the Catalogue is \xfile{kvdefinekeys.xml}.
%    \begin{macrocode}
%<*catalogue>
<?xml version='1.0' encoding='us-ascii'?>
<!DOCTYPE entry SYSTEM 'catalogue.dtd'>
<entry datestamp='$Date$' modifier='$Author$' id='kvdefinekeys'>
  <name>kvdefinekeys</name>
  <caption>Define keys for use in the kvsetkeys package.</caption>
  <authorref id='auth:oberdiek'/>
  <copyright owner='Heiko Oberdiek' year='2010,2011'/>
  <license type='lppl1.3'/>
  <version number='1.4'/>
  <description>
    The package provides a macro <tt>\kv@define@key</tt> (analogous to
    <xref refid='keyval'>keyval&#x2019;s</xref> <tt>\define@key</tt>, to
    define keys for use by <xref refid='kvsetkeys'>kvsetkeys</xref>.
    <p/>
    The package is part of the <xref refid='oberdiek'>oberdiek</xref>
    bundle.
  </description>
  <documentation details='Package documentation'
      href='ctan:/macros/latex/contrib/oberdiek/kvdefinekeys.pdf'/>
  <ctan file='true' path='/macros/latex/contrib/oberdiek/kvdefinekeys.dtx'/>
  <miktex location='oberdiek'/>
  <texlive location='oberdiek'/>
  <install path='/macros/latex/contrib/oberdiek/oberdiek.tds.zip'/>
</entry>
%</catalogue>
%    \end{macrocode}
%
% \begin{thebibliography}{9}
% \bibitem{keyval}
%   David Carlisle:
%   \textit{The \xpackage{keyval} package};
%   1999/03/16 v1.13;
%   \CTAN{macros/latex/required/graphics/keyval.dtx}.
%
% \end{thebibliography}
%
% \begin{History}
%   \begin{Version}{2010/03/01 v1.0}
%   \item
%     First version.
%   \end{Version}
%   \begin{Version}{2010/08/19 v1.1}
%   \item
%     Documentation fix, no code change.
%   \end{Version}
%   \begin{Version}{2011/01/30 v1.2}
%   \item
%     Already loaded package files are not input in \hologo{plainTeX}.
%   \end{Version}
%   \begin{Version}{2011/04/07 v1.3}
%   \item
%     Support for package \xpackage{babel}'s shorthands added.
%   \item
%     \cs{kv@define@key} is made robust if available.
%   \end{Version}
%   \begin{Version}{2016/05/16 v1.4}
%   \item
%     Documentation updates.
%   \end{Version}
% \end{History}
%
% \PrintIndex
%
% \Finale
\endinput

%        (quote the arguments according to the demands of your shell)
%
% Documentation:
%    (a) If kvdefinekeys.drv is present:
%           latex kvdefinekeys.drv
%    (b) Without kvdefinekeys.drv:
%           latex kvdefinekeys.dtx; ...
%    The class ltxdoc loads the configuration file ltxdoc.cfg
%    if available. Here you can specify further options, e.g.
%    use A4 as paper format:
%       \PassOptionsToClass{a4paper}{article}
%
%    Programm calls to get the documentation (example):
%       pdflatex kvdefinekeys.dtx
%       makeindex -s gind.ist kvdefinekeys.idx
%       pdflatex kvdefinekeys.dtx
%       makeindex -s gind.ist kvdefinekeys.idx
%       pdflatex kvdefinekeys.dtx
%
% Installation:
%    TDS:tex/generic/oberdiek/kvdefinekeys.sty
%    TDS:doc/latex/oberdiek/kvdefinekeys.pdf
%    TDS:doc/latex/oberdiek/test/kvdefinekeys-test1.tex
%    TDS:source/latex/oberdiek/kvdefinekeys.dtx
%
%<*ignore>
\begingroup
  \catcode123=1 %
  \catcode125=2 %
  \def\x{LaTeX2e}%
\expandafter\endgroup
\ifcase 0\ifx\install y1\fi\expandafter
         \ifx\csname processbatchFile\endcsname\relax\else1\fi
         \ifx\fmtname\x\else 1\fi\relax
\else\csname fi\endcsname
%</ignore>
%<*install>
\input docstrip.tex
\Msg{************************************************************************}
\Msg{* Installation}
\Msg{* Package: kvdefinekeys 2016/05/16 v1.4 Define keys (HO)}
\Msg{************************************************************************}

\keepsilent
\askforoverwritefalse

\let\MetaPrefix\relax
\preamble

This is a generated file.

Project: kvdefinekeys
Version: 2016/05/16 v1.4

Copyright (C) 2010, 2011 by
   Heiko Oberdiek <heiko.oberdiek at googlemail.com>

This work may be distributed and/or modified under the
conditions of the LaTeX Project Public License, either
version 1.3c of this license or (at your option) any later
version. This version of this license is in
   http://www.latex-project.org/lppl/lppl-1-3c.txt
and the latest version of this license is in
   http://www.latex-project.org/lppl.txt
and version 1.3 or later is part of all distributions of
LaTeX version 2005/12/01 or later.

This work has the LPPL maintenance status "maintained".

This Current Maintainer of this work is Heiko Oberdiek.

The Base Interpreter refers to any `TeX-Format',
because some files are installed in TDS:tex/generic//.

This work consists of the main source file kvdefinekeys.dtx
and the derived files
   kvdefinekeys.sty, kvdefinekeys.pdf, kvdefinekeys.ins, kvdefinekeys.drv,
   kvdefinekeys-test1.tex.

\endpreamble
\let\MetaPrefix\DoubleperCent

\generate{%
  \file{kvdefinekeys.ins}{\from{kvdefinekeys.dtx}{install}}%
  \file{kvdefinekeys.drv}{\from{kvdefinekeys.dtx}{driver}}%
  \usedir{tex/generic/oberdiek}%
  \file{kvdefinekeys.sty}{\from{kvdefinekeys.dtx}{package}}%
  \usedir{doc/latex/oberdiek/test}%
  \file{kvdefinekeys-test1.tex}{\from{kvdefinekeys.dtx}{test1}}%
  \nopreamble
  \nopostamble
  \usedir{source/latex/oberdiek/catalogue}%
  \file{kvdefinekeys.xml}{\from{kvdefinekeys.dtx}{catalogue}}%
}

\catcode32=13\relax% active space
\let =\space%
\Msg{************************************************************************}
\Msg{*}
\Msg{* To finish the installation you have to move the following}
\Msg{* file into a directory searched by TeX:}
\Msg{*}
\Msg{*     kvdefinekeys.sty}
\Msg{*}
\Msg{* To produce the documentation run the file `kvdefinekeys.drv'}
\Msg{* through LaTeX.}
\Msg{*}
\Msg{* Happy TeXing!}
\Msg{*}
\Msg{************************************************************************}

\endbatchfile
%</install>
%<*ignore>
\fi
%</ignore>
%<*driver>
\NeedsTeXFormat{LaTeX2e}
\ProvidesFile{kvdefinekeys.drv}%
  [2016/05/16 v1.4 Define keys (HO)]%
\documentclass{ltxdoc}
\usepackage{holtxdoc}[2011/11/22]
\begin{document}
  \DocInput{kvdefinekeys.dtx}%
\end{document}
%</driver>
% \fi
%
%
% \CharacterTable
%  {Upper-case    \A\B\C\D\E\F\G\H\I\J\K\L\M\N\O\P\Q\R\S\T\U\V\W\X\Y\Z
%   Lower-case    \a\b\c\d\e\f\g\h\i\j\k\l\m\n\o\p\q\r\s\t\u\v\w\x\y\z
%   Digits        \0\1\2\3\4\5\6\7\8\9
%   Exclamation   \!     Double quote  \"     Hash (number) \#
%   Dollar        \$     Percent       \%     Ampersand     \&
%   Acute accent  \'     Left paren    \(     Right paren   \)
%   Asterisk      \*     Plus          \+     Comma         \,
%   Minus         \-     Point         \.     Solidus       \/
%   Colon         \:     Semicolon     \;     Less than     \<
%   Equals        \=     Greater than  \>     Question mark \?
%   Commercial at \@     Left bracket  \[     Backslash     \\
%   Right bracket \]     Circumflex    \^     Underscore    \_
%   Grave accent  \`     Left brace    \{     Vertical bar  \|
%   Right brace   \}     Tilde         \~}
%
% \GetFileInfo{kvdefinekeys.drv}
%
% \title{The \xpackage{kvdefinekeys} package}
% \date{2016/05/16 v1.4}
% \author{Heiko Oberdiek\thanks
% {Please report any issues at https://github.com/ho-tex/oberdiek/issues}\\
% \xemail{heiko.oberdiek at googlemail.com}}
%
% \maketitle
%
% \begin{abstract}
% Package \xpackage{kvdefinekeys} provides \cs{kv@define@key} to define
% keys the same way as \xpackage{keyval}'s \cs{define@key}. However, it
% works also using \iniTeX.
% \end{abstract}
%
% \tableofcontents
%
% \def\M#1{\texttt{\{}\meta{#1}\texttt{\}}}
%
% \section{Documentation}
%
% \subsection{Motivation}
%
% \cs{kvsetkeys} serves as replacement for \xpackage{keyval}'s
% \cs{setkeys}. This package adds macros to define keys, closing
% the gap \cs{kvsetkeys} leaves.
%
% \begin{declcs}{kv@define@key}\,\M{family}\,\M{key}\,[\meta{default}]\,^^A
%   \M{definition}
% \end{declcs}
% Macro \cs{kv@define@key} reimplements \xpackage{keyval}'s
% \cs{define@key}. Differences to the original:
% \begin{itemize}
% \item The defined keys also allow \cs{par} inside values.
% \item Shorthands of package \xpackage{babel} are supported in
%   family and key names.
% \item Macro \cs{kv@define@key} is made robust if
%   \hologo{eTeX}'s \cs{protected} or \hologo{LaTeX}'s
%   \cs{DeclareRobustCommand} are found.
% \end{itemize}
%
% \StopEventually{
% }
%
% \section{Implementation}
%
% \subsection{Identification}
%
%    \begin{macrocode}
%<*package>
%    \end{macrocode}
%    Reload check, especially if the package is not used with \LaTeX.
%    \begin{macrocode}
\begingroup\catcode61\catcode48\catcode32=10\relax%
  \catcode13=5 % ^^M
  \endlinechar=13 %
  \catcode35=6 % #
  \catcode39=12 % '
  \catcode44=12 % ,
  \catcode45=12 % -
  \catcode46=12 % .
  \catcode58=12 % :
  \catcode64=11 % @
  \catcode123=1 % {
  \catcode125=2 % }
  \expandafter\let\expandafter\x\csname ver@kvdefinekeys.sty\endcsname
  \ifx\x\relax % plain-TeX, first loading
  \else
    \def\empty{}%
    \ifx\x\empty % LaTeX, first loading,
      % variable is initialized, but \ProvidesPackage not yet seen
    \else
      \expandafter\ifx\csname PackageInfo\endcsname\relax
        \def\x#1#2{%
          \immediate\write-1{Package #1 Info: #2.}%
        }%
      \else
        \def\x#1#2{\PackageInfo{#1}{#2, stopped}}%
      \fi
      \x{kvdefinekeys}{The package is already loaded}%
      \aftergroup\endinput
    \fi
  \fi
\endgroup%
%    \end{macrocode}
%    Package identification:
%    \begin{macrocode}
\begingroup\catcode61\catcode48\catcode32=10\relax%
  \catcode13=5 % ^^M
  \endlinechar=13 %
  \catcode35=6 % #
  \catcode39=12 % '
  \catcode40=12 % (
  \catcode41=12 % )
  \catcode44=12 % ,
  \catcode45=12 % -
  \catcode46=12 % .
  \catcode47=12 % /
  \catcode58=12 % :
  \catcode64=11 % @
  \catcode91=12 % [
  \catcode93=12 % ]
  \catcode123=1 % {
  \catcode125=2 % }
  \expandafter\ifx\csname ProvidesPackage\endcsname\relax
    \def\x#1#2#3[#4]{\endgroup
      \immediate\write-1{Package: #3 #4}%
      \xdef#1{#4}%
    }%
  \else
    \def\x#1#2[#3]{\endgroup
      #2[{#3}]%
      \ifx#1\@undefined
        \xdef#1{#3}%
      \fi
      \ifx#1\relax
        \xdef#1{#3}%
      \fi
    }%
  \fi
\expandafter\x\csname ver@kvdefinekeys.sty\endcsname
\ProvidesPackage{kvdefinekeys}%
  [2016/05/16 v1.4 Define keys (HO)]%
%    \end{macrocode}
%
%    \begin{macrocode}
\begingroup\catcode61\catcode48\catcode32=10\relax%
  \catcode13=5 % ^^M
  \endlinechar=13 %
  \catcode123=1 % {
  \catcode125=2 % }
  \catcode64=11 % @
  \def\x{\endgroup
    \expandafter\edef\csname KVD@AtEnd\endcsname{%
      \endlinechar=\the\endlinechar\relax
      \catcode13=\the\catcode13\relax
      \catcode32=\the\catcode32\relax
      \catcode35=\the\catcode35\relax
      \catcode61=\the\catcode61\relax
      \catcode64=\the\catcode64\relax
      \catcode123=\the\catcode123\relax
      \catcode125=\the\catcode125\relax
    }%
  }%
\x\catcode61\catcode48\catcode32=10\relax%
\catcode13=5 % ^^M
\endlinechar=13 %
\catcode35=6 % #
\catcode64=11 % @
\catcode123=1 % {
\catcode125=2 % }
\def\TMP@EnsureCode#1#2{%
  \edef\KVD@AtEnd{%
    \KVD@AtEnd
    \catcode#1=\the\catcode#1\relax
  }%
  \catcode#1=#2\relax
}
\TMP@EnsureCode{42}{12}% *
\TMP@EnsureCode{46}{12}% .
\TMP@EnsureCode{47}{12}% /
\TMP@EnsureCode{91}{12}% [
\TMP@EnsureCode{93}{12}% ]
\edef\KVD@AtEnd{\KVD@AtEnd\noexpand\endinput}
%    \end{macrocode}
%
% \subsection{Package loading}
%
%    \begin{macrocode}
\begingroup\expandafter\expandafter\expandafter\endgroup
\expandafter\ifx\csname RequirePackage\endcsname\relax
  \def\TMP@RequirePackage#1[#2]{%
    \begingroup\expandafter\expandafter\expandafter\endgroup
    \expandafter\ifx\csname ver@#1.sty\endcsname\relax
      \input #1.sty\relax
    \fi
  }%
  \TMP@RequirePackage{ltxcmds}[2010/03/01]%
\else
  \RequirePackage{ltxcmds}[2010/03/01]%
\fi
%    \end{macrocode}
%
%
% \subsection{Provide key defining macro}
%
%    \begin{macro}{\kv@define@key}
%    \begin{macrocode}
\ltx@IfUndefined{protected}{%
  \ltx@IfUndefined{DeclareRobustCommand}{%
    \def\kv@define@key#1#2%
  }{%
    \DeclareRobustCommand*{\kv@define@key}[2]%
  }%
}{%
  \protected\def\kv@define@key#1#2%
}%
{%
  \begingroup
    \csname @safe@activestrue\endcsname
    \let\ifincsname\iftrue
    \edef\KVD@temp{\endgroup
      \noexpand\KVD@DefineKey{#1}{#2}%
    }%
  \KVD@temp
}
%    \end{macrocode}
%    \end{macro}
%    \begin{macro}{\KVD@DefineKey}
%    \begin{macrocode}
\def\KVD@DefineKey#1#2{%
  \ltx@ifnextchar[{%
    \KVD@DefineKeyWithDefault{#1}{#2}%
  }{%
    \long\expandafter\def\csname KV@#1@#2\endcsname##1%
  }%
}
%    \end{macrocode}
%    \end{macro}
%    \begin{macro}{\KVD@DefineKeyWithDefault}
%    \begin{macrocode}
\long\def\KVD@DefineKeyWithDefault#1#2[#3]{%
  \expandafter\def\csname KV@#1@#2@default\expandafter\endcsname
  \expandafter{%
    \csname KV@#1@#2\endcsname{#3}%
  }%
  \long\expandafter\def\csname KV@#1@#2\endcsname##1%
}
%    \end{macrocode}
%    \end{macro}
%
%    \begin{macrocode}
\KVD@AtEnd%
%</package>
%    \end{macrocode}
%
% \section{Test}
%
% \subsection{Catcode checks for loading}
%
%    \begin{macrocode}
%<*test1>
%    \end{macrocode}
%    \begin{macrocode}
\catcode`\{=1 %
\catcode`\}=2 %
\catcode`\#=6 %
\catcode`\@=11 %
\expandafter\ifx\csname count@\endcsname\relax
  \countdef\count@=255 %
\fi
\expandafter\ifx\csname @gobble\endcsname\relax
  \long\def\@gobble#1{}%
\fi
\expandafter\ifx\csname @firstofone\endcsname\relax
  \long\def\@firstofone#1{#1}%
\fi
\expandafter\ifx\csname loop\endcsname\relax
  \expandafter\@firstofone
\else
  \expandafter\@gobble
\fi
{%
  \def\loop#1\repeat{%
    \def\body{#1}%
    \iterate
  }%
  \def\iterate{%
    \body
      \let\next\iterate
    \else
      \let\next\relax
    \fi
    \next
  }%
  \let\repeat=\fi
}%
\def\RestoreCatcodes{}
\count@=0 %
\loop
  \edef\RestoreCatcodes{%
    \RestoreCatcodes
    \catcode\the\count@=\the\catcode\count@\relax
  }%
\ifnum\count@<255 %
  \advance\count@ 1 %
\repeat

\def\RangeCatcodeInvalid#1#2{%
  \count@=#1\relax
  \loop
    \catcode\count@=15 %
  \ifnum\count@<#2\relax
    \advance\count@ 1 %
  \repeat
}
\def\RangeCatcodeCheck#1#2#3{%
  \count@=#1\relax
  \loop
    \ifnum#3=\catcode\count@
    \else
      \errmessage{%
        Character \the\count@\space
        with wrong catcode \the\catcode\count@\space
        instead of \number#3%
      }%
    \fi
  \ifnum\count@<#2\relax
    \advance\count@ 1 %
  \repeat
}
\def\space{ }
\expandafter\ifx\csname LoadCommand\endcsname\relax
  \def\LoadCommand{\input kvdefinekeys.sty\relax}%
\fi
\def\Test{%
  \RangeCatcodeInvalid{0}{47}%
  \RangeCatcodeInvalid{58}{64}%
  \RangeCatcodeInvalid{91}{96}%
  \RangeCatcodeInvalid{123}{255}%
  \catcode`\@=12 %
  \catcode`\\=0 %
  \catcode`\%=14 %
  \LoadCommand
  \RangeCatcodeCheck{0}{36}{15}%
  \RangeCatcodeCheck{37}{37}{14}%
  \RangeCatcodeCheck{38}{47}{15}%
  \RangeCatcodeCheck{48}{57}{12}%
  \RangeCatcodeCheck{58}{63}{15}%
  \RangeCatcodeCheck{64}{64}{12}%
  \RangeCatcodeCheck{65}{90}{11}%
  \RangeCatcodeCheck{91}{91}{15}%
  \RangeCatcodeCheck{92}{92}{0}%
  \RangeCatcodeCheck{93}{96}{15}%
  \RangeCatcodeCheck{97}{122}{11}%
  \RangeCatcodeCheck{123}{255}{15}%
  \RestoreCatcodes
}
\Test
\csname @@end\endcsname
\end
%    \end{macrocode}
%    \begin{macrocode}
%</test1>
%    \end{macrocode}
%
% \section{Installation}
%
% \subsection{Download}
%
% \paragraph{Package.} This package is available on
% CTAN\footnote{\url{http://ctan.org/pkg/kvdefinekeys}}:
% \begin{description}
% \item[\CTAN{macros/latex/contrib/oberdiek/kvdefinekeys.dtx}] The source file.
% \item[\CTAN{macros/latex/contrib/oberdiek/kvdefinekeys.pdf}] Documentation.
% \end{description}
%
%
% \paragraph{Bundle.} All the packages of the bundle `oberdiek'
% are also available in a TDS compliant ZIP archive. There
% the packages are already unpacked and the documentation files
% are generated. The files and directories obey the TDS standard.
% \begin{description}
% \item[\CTAN{install/macros/latex/contrib/oberdiek.tds.zip}]
% \end{description}
% \emph{TDS} refers to the standard ``A Directory Structure
% for \TeX\ Files'' (\CTAN{tds/tds.pdf}). Directories
% with \xfile{texmf} in their name are usually organized this way.
%
% \subsection{Bundle installation}
%
% \paragraph{Unpacking.} Unpack the \xfile{oberdiek.tds.zip} in the
% TDS tree (also known as \xfile{texmf} tree) of your choice.
% Example (linux):
% \begin{quote}
%   |unzip oberdiek.tds.zip -d ~/texmf|
% \end{quote}
%
% \paragraph{Script installation.}
% Check the directory \xfile{TDS:scripts/oberdiek/} for
% scripts that need further installation steps.
% Package \xpackage{attachfile2} comes with the Perl script
% \xfile{pdfatfi.pl} that should be installed in such a way
% that it can be called as \texttt{pdfatfi}.
% Example (linux):
% \begin{quote}
%   |chmod +x scripts/oberdiek/pdfatfi.pl|\\
%   |cp scripts/oberdiek/pdfatfi.pl /usr/local/bin/|
% \end{quote}
%
% \subsection{Package installation}
%
% \paragraph{Unpacking.} The \xfile{.dtx} file is a self-extracting
% \docstrip\ archive. The files are extracted by running the
% \xfile{.dtx} through \plainTeX:
% \begin{quote}
%   \verb|tex kvdefinekeys.dtx|
% \end{quote}
%
% \paragraph{TDS.} Now the different files must be moved into
% the different directories in your installation TDS tree
% (also known as \xfile{texmf} tree):
% \begin{quote}
% \def\t{^^A
% \begin{tabular}{@{}>{\ttfamily}l@{ $\rightarrow$ }>{\ttfamily}l@{}}
%   kvdefinekeys.sty & tex/generic/oberdiek/kvdefinekeys.sty\\
%   kvdefinekeys.pdf & doc/latex/oberdiek/kvdefinekeys.pdf\\
%   test/kvdefinekeys-test1.tex & doc/latex/oberdiek/test/kvdefinekeys-test1.tex\\
%   kvdefinekeys.dtx & source/latex/oberdiek/kvdefinekeys.dtx\\
% \end{tabular}^^A
% }^^A
% \sbox0{\t}^^A
% \ifdim\wd0>\linewidth
%   \begingroup
%     \advance\linewidth by\leftmargin
%     \advance\linewidth by\rightmargin
%   \edef\x{\endgroup
%     \def\noexpand\lw{\the\linewidth}^^A
%   }\x
%   \def\lwbox{^^A
%     \leavevmode
%     \hbox to \linewidth{^^A
%       \kern-\leftmargin\relax
%       \hss
%       \usebox0
%       \hss
%       \kern-\rightmargin\relax
%     }^^A
%   }^^A
%   \ifdim\wd0>\lw
%     \sbox0{\small\t}^^A
%     \ifdim\wd0>\linewidth
%       \ifdim\wd0>\lw
%         \sbox0{\footnotesize\t}^^A
%         \ifdim\wd0>\linewidth
%           \ifdim\wd0>\lw
%             \sbox0{\scriptsize\t}^^A
%             \ifdim\wd0>\linewidth
%               \ifdim\wd0>\lw
%                 \sbox0{\tiny\t}^^A
%                 \ifdim\wd0>\linewidth
%                   \lwbox
%                 \else
%                   \usebox0
%                 \fi
%               \else
%                 \lwbox
%               \fi
%             \else
%               \usebox0
%             \fi
%           \else
%             \lwbox
%           \fi
%         \else
%           \usebox0
%         \fi
%       \else
%         \lwbox
%       \fi
%     \else
%       \usebox0
%     \fi
%   \else
%     \lwbox
%   \fi
% \else
%   \usebox0
% \fi
% \end{quote}
% If you have a \xfile{docstrip.cfg} that configures and enables \docstrip's
% TDS installing feature, then some files can already be in the right
% place, see the documentation of \docstrip.
%
% \subsection{Refresh file name databases}
%
% If your \TeX~distribution
% (\teTeX, \mikTeX, \dots) relies on file name databases, you must refresh
% these. For example, \teTeX\ users run \verb|texhash| or
% \verb|mktexlsr|.
%
% \subsection{Some details for the interested}
%
% \paragraph{Attached source.}
%
% The PDF documentation on CTAN also includes the
% \xfile{.dtx} source file. It can be extracted by
% AcrobatReader 6 or higher. Another option is \textsf{pdftk},
% e.g. unpack the file into the current directory:
% \begin{quote}
%   \verb|pdftk kvdefinekeys.pdf unpack_files output .|
% \end{quote}
%
% \paragraph{Unpacking with \LaTeX.}
% The \xfile{.dtx} chooses its action depending on the format:
% \begin{description}
% \item[\plainTeX:] Run \docstrip\ and extract the files.
% \item[\LaTeX:] Generate the documentation.
% \end{description}
% If you insist on using \LaTeX\ for \docstrip\ (really,
% \docstrip\ does not need \LaTeX), then inform the autodetect routine
% about your intention:
% \begin{quote}
%   \verb|latex \let\install=y% \iffalse meta-comment
%
% File: kvdefinekeys.dtx
% Version: 2016/05/16 v1.4
% Info: Define keys
%
% Copyright (C) 2010, 2011 by
%    Heiko Oberdiek <heiko.oberdiek at googlemail.com>
%    2016
%    https://github.com/ho-tex/oberdiek/issues
%
% This work may be distributed and/or modified under the
% conditions of the LaTeX Project Public License, either
% version 1.3c of this license or (at your option) any later
% version. This version of this license is in
%    http://www.latex-project.org/lppl/lppl-1-3c.txt
% and the latest version of this license is in
%    http://www.latex-project.org/lppl.txt
% and version 1.3 or later is part of all distributions of
% LaTeX version 2005/12/01 or later.
%
% This work has the LPPL maintenance status "maintained".
%
% This Current Maintainer of this work is Heiko Oberdiek.
%
% The Base Interpreter refers to any `TeX-Format',
% because some files are installed in TDS:tex/generic//.
%
% This work consists of the main source file kvdefinekeys.dtx
% and the derived files
%    kvdefinekeys.sty, kvdefinekeys.pdf, kvdefinekeys.ins, kvdefinekeys.drv,
%    kvdefinekeys-test1.tex.
%
% Distribution:
%    CTAN:macros/latex/contrib/oberdiek/kvdefinekeys.dtx
%    CTAN:macros/latex/contrib/oberdiek/kvdefinekeys.pdf
%
% Unpacking:
%    (a) If kvdefinekeys.ins is present:
%           tex kvdefinekeys.ins
%    (b) Without kvdefinekeys.ins:
%           tex kvdefinekeys.dtx
%    (c) If you insist on using LaTeX
%           latex \let\install=y\input{kvdefinekeys.dtx}
%        (quote the arguments according to the demands of your shell)
%
% Documentation:
%    (a) If kvdefinekeys.drv is present:
%           latex kvdefinekeys.drv
%    (b) Without kvdefinekeys.drv:
%           latex kvdefinekeys.dtx; ...
%    The class ltxdoc loads the configuration file ltxdoc.cfg
%    if available. Here you can specify further options, e.g.
%    use A4 as paper format:
%       \PassOptionsToClass{a4paper}{article}
%
%    Programm calls to get the documentation (example):
%       pdflatex kvdefinekeys.dtx
%       makeindex -s gind.ist kvdefinekeys.idx
%       pdflatex kvdefinekeys.dtx
%       makeindex -s gind.ist kvdefinekeys.idx
%       pdflatex kvdefinekeys.dtx
%
% Installation:
%    TDS:tex/generic/oberdiek/kvdefinekeys.sty
%    TDS:doc/latex/oberdiek/kvdefinekeys.pdf
%    TDS:doc/latex/oberdiek/test/kvdefinekeys-test1.tex
%    TDS:source/latex/oberdiek/kvdefinekeys.dtx
%
%<*ignore>
\begingroup
  \catcode123=1 %
  \catcode125=2 %
  \def\x{LaTeX2e}%
\expandafter\endgroup
\ifcase 0\ifx\install y1\fi\expandafter
         \ifx\csname processbatchFile\endcsname\relax\else1\fi
         \ifx\fmtname\x\else 1\fi\relax
\else\csname fi\endcsname
%</ignore>
%<*install>
\input docstrip.tex
\Msg{************************************************************************}
\Msg{* Installation}
\Msg{* Package: kvdefinekeys 2016/05/16 v1.4 Define keys (HO)}
\Msg{************************************************************************}

\keepsilent
\askforoverwritefalse

\let\MetaPrefix\relax
\preamble

This is a generated file.

Project: kvdefinekeys
Version: 2016/05/16 v1.4

Copyright (C) 2010, 2011 by
   Heiko Oberdiek <heiko.oberdiek at googlemail.com>

This work may be distributed and/or modified under the
conditions of the LaTeX Project Public License, either
version 1.3c of this license or (at your option) any later
version. This version of this license is in
   http://www.latex-project.org/lppl/lppl-1-3c.txt
and the latest version of this license is in
   http://www.latex-project.org/lppl.txt
and version 1.3 or later is part of all distributions of
LaTeX version 2005/12/01 or later.

This work has the LPPL maintenance status "maintained".

This Current Maintainer of this work is Heiko Oberdiek.

The Base Interpreter refers to any `TeX-Format',
because some files are installed in TDS:tex/generic//.

This work consists of the main source file kvdefinekeys.dtx
and the derived files
   kvdefinekeys.sty, kvdefinekeys.pdf, kvdefinekeys.ins, kvdefinekeys.drv,
   kvdefinekeys-test1.tex.

\endpreamble
\let\MetaPrefix\DoubleperCent

\generate{%
  \file{kvdefinekeys.ins}{\from{kvdefinekeys.dtx}{install}}%
  \file{kvdefinekeys.drv}{\from{kvdefinekeys.dtx}{driver}}%
  \usedir{tex/generic/oberdiek}%
  \file{kvdefinekeys.sty}{\from{kvdefinekeys.dtx}{package}}%
  \usedir{doc/latex/oberdiek/test}%
  \file{kvdefinekeys-test1.tex}{\from{kvdefinekeys.dtx}{test1}}%
  \nopreamble
  \nopostamble
  \usedir{source/latex/oberdiek/catalogue}%
  \file{kvdefinekeys.xml}{\from{kvdefinekeys.dtx}{catalogue}}%
}

\catcode32=13\relax% active space
\let =\space%
\Msg{************************************************************************}
\Msg{*}
\Msg{* To finish the installation you have to move the following}
\Msg{* file into a directory searched by TeX:}
\Msg{*}
\Msg{*     kvdefinekeys.sty}
\Msg{*}
\Msg{* To produce the documentation run the file `kvdefinekeys.drv'}
\Msg{* through LaTeX.}
\Msg{*}
\Msg{* Happy TeXing!}
\Msg{*}
\Msg{************************************************************************}

\endbatchfile
%</install>
%<*ignore>
\fi
%</ignore>
%<*driver>
\NeedsTeXFormat{LaTeX2e}
\ProvidesFile{kvdefinekeys.drv}%
  [2016/05/16 v1.4 Define keys (HO)]%
\documentclass{ltxdoc}
\usepackage{holtxdoc}[2011/11/22]
\begin{document}
  \DocInput{kvdefinekeys.dtx}%
\end{document}
%</driver>
% \fi
%
%
% \CharacterTable
%  {Upper-case    \A\B\C\D\E\F\G\H\I\J\K\L\M\N\O\P\Q\R\S\T\U\V\W\X\Y\Z
%   Lower-case    \a\b\c\d\e\f\g\h\i\j\k\l\m\n\o\p\q\r\s\t\u\v\w\x\y\z
%   Digits        \0\1\2\3\4\5\6\7\8\9
%   Exclamation   \!     Double quote  \"     Hash (number) \#
%   Dollar        \$     Percent       \%     Ampersand     \&
%   Acute accent  \'     Left paren    \(     Right paren   \)
%   Asterisk      \*     Plus          \+     Comma         \,
%   Minus         \-     Point         \.     Solidus       \/
%   Colon         \:     Semicolon     \;     Less than     \<
%   Equals        \=     Greater than  \>     Question mark \?
%   Commercial at \@     Left bracket  \[     Backslash     \\
%   Right bracket \]     Circumflex    \^     Underscore    \_
%   Grave accent  \`     Left brace    \{     Vertical bar  \|
%   Right brace   \}     Tilde         \~}
%
% \GetFileInfo{kvdefinekeys.drv}
%
% \title{The \xpackage{kvdefinekeys} package}
% \date{2016/05/16 v1.4}
% \author{Heiko Oberdiek\thanks
% {Please report any issues at https://github.com/ho-tex/oberdiek/issues}\\
% \xemail{heiko.oberdiek at googlemail.com}}
%
% \maketitle
%
% \begin{abstract}
% Package \xpackage{kvdefinekeys} provides \cs{kv@define@key} to define
% keys the same way as \xpackage{keyval}'s \cs{define@key}. However, it
% works also using \iniTeX.
% \end{abstract}
%
% \tableofcontents
%
% \def\M#1{\texttt{\{}\meta{#1}\texttt{\}}}
%
% \section{Documentation}
%
% \subsection{Motivation}
%
% \cs{kvsetkeys} serves as replacement for \xpackage{keyval}'s
% \cs{setkeys}. This package adds macros to define keys, closing
% the gap \cs{kvsetkeys} leaves.
%
% \begin{declcs}{kv@define@key}\,\M{family}\,\M{key}\,[\meta{default}]\,^^A
%   \M{definition}
% \end{declcs}
% Macro \cs{kv@define@key} reimplements \xpackage{keyval}'s
% \cs{define@key}. Differences to the original:
% \begin{itemize}
% \item The defined keys also allow \cs{par} inside values.
% \item Shorthands of package \xpackage{babel} are supported in
%   family and key names.
% \item Macro \cs{kv@define@key} is made robust if
%   \hologo{eTeX}'s \cs{protected} or \hologo{LaTeX}'s
%   \cs{DeclareRobustCommand} are found.
% \end{itemize}
%
% \StopEventually{
% }
%
% \section{Implementation}
%
% \subsection{Identification}
%
%    \begin{macrocode}
%<*package>
%    \end{macrocode}
%    Reload check, especially if the package is not used with \LaTeX.
%    \begin{macrocode}
\begingroup\catcode61\catcode48\catcode32=10\relax%
  \catcode13=5 % ^^M
  \endlinechar=13 %
  \catcode35=6 % #
  \catcode39=12 % '
  \catcode44=12 % ,
  \catcode45=12 % -
  \catcode46=12 % .
  \catcode58=12 % :
  \catcode64=11 % @
  \catcode123=1 % {
  \catcode125=2 % }
  \expandafter\let\expandafter\x\csname ver@kvdefinekeys.sty\endcsname
  \ifx\x\relax % plain-TeX, first loading
  \else
    \def\empty{}%
    \ifx\x\empty % LaTeX, first loading,
      % variable is initialized, but \ProvidesPackage not yet seen
    \else
      \expandafter\ifx\csname PackageInfo\endcsname\relax
        \def\x#1#2{%
          \immediate\write-1{Package #1 Info: #2.}%
        }%
      \else
        \def\x#1#2{\PackageInfo{#1}{#2, stopped}}%
      \fi
      \x{kvdefinekeys}{The package is already loaded}%
      \aftergroup\endinput
    \fi
  \fi
\endgroup%
%    \end{macrocode}
%    Package identification:
%    \begin{macrocode}
\begingroup\catcode61\catcode48\catcode32=10\relax%
  \catcode13=5 % ^^M
  \endlinechar=13 %
  \catcode35=6 % #
  \catcode39=12 % '
  \catcode40=12 % (
  \catcode41=12 % )
  \catcode44=12 % ,
  \catcode45=12 % -
  \catcode46=12 % .
  \catcode47=12 % /
  \catcode58=12 % :
  \catcode64=11 % @
  \catcode91=12 % [
  \catcode93=12 % ]
  \catcode123=1 % {
  \catcode125=2 % }
  \expandafter\ifx\csname ProvidesPackage\endcsname\relax
    \def\x#1#2#3[#4]{\endgroup
      \immediate\write-1{Package: #3 #4}%
      \xdef#1{#4}%
    }%
  \else
    \def\x#1#2[#3]{\endgroup
      #2[{#3}]%
      \ifx#1\@undefined
        \xdef#1{#3}%
      \fi
      \ifx#1\relax
        \xdef#1{#3}%
      \fi
    }%
  \fi
\expandafter\x\csname ver@kvdefinekeys.sty\endcsname
\ProvidesPackage{kvdefinekeys}%
  [2016/05/16 v1.4 Define keys (HO)]%
%    \end{macrocode}
%
%    \begin{macrocode}
\begingroup\catcode61\catcode48\catcode32=10\relax%
  \catcode13=5 % ^^M
  \endlinechar=13 %
  \catcode123=1 % {
  \catcode125=2 % }
  \catcode64=11 % @
  \def\x{\endgroup
    \expandafter\edef\csname KVD@AtEnd\endcsname{%
      \endlinechar=\the\endlinechar\relax
      \catcode13=\the\catcode13\relax
      \catcode32=\the\catcode32\relax
      \catcode35=\the\catcode35\relax
      \catcode61=\the\catcode61\relax
      \catcode64=\the\catcode64\relax
      \catcode123=\the\catcode123\relax
      \catcode125=\the\catcode125\relax
    }%
  }%
\x\catcode61\catcode48\catcode32=10\relax%
\catcode13=5 % ^^M
\endlinechar=13 %
\catcode35=6 % #
\catcode64=11 % @
\catcode123=1 % {
\catcode125=2 % }
\def\TMP@EnsureCode#1#2{%
  \edef\KVD@AtEnd{%
    \KVD@AtEnd
    \catcode#1=\the\catcode#1\relax
  }%
  \catcode#1=#2\relax
}
\TMP@EnsureCode{42}{12}% *
\TMP@EnsureCode{46}{12}% .
\TMP@EnsureCode{47}{12}% /
\TMP@EnsureCode{91}{12}% [
\TMP@EnsureCode{93}{12}% ]
\edef\KVD@AtEnd{\KVD@AtEnd\noexpand\endinput}
%    \end{macrocode}
%
% \subsection{Package loading}
%
%    \begin{macrocode}
\begingroup\expandafter\expandafter\expandafter\endgroup
\expandafter\ifx\csname RequirePackage\endcsname\relax
  \def\TMP@RequirePackage#1[#2]{%
    \begingroup\expandafter\expandafter\expandafter\endgroup
    \expandafter\ifx\csname ver@#1.sty\endcsname\relax
      \input #1.sty\relax
    \fi
  }%
  \TMP@RequirePackage{ltxcmds}[2010/03/01]%
\else
  \RequirePackage{ltxcmds}[2010/03/01]%
\fi
%    \end{macrocode}
%
%
% \subsection{Provide key defining macro}
%
%    \begin{macro}{\kv@define@key}
%    \begin{macrocode}
\ltx@IfUndefined{protected}{%
  \ltx@IfUndefined{DeclareRobustCommand}{%
    \def\kv@define@key#1#2%
  }{%
    \DeclareRobustCommand*{\kv@define@key}[2]%
  }%
}{%
  \protected\def\kv@define@key#1#2%
}%
{%
  \begingroup
    \csname @safe@activestrue\endcsname
    \let\ifincsname\iftrue
    \edef\KVD@temp{\endgroup
      \noexpand\KVD@DefineKey{#1}{#2}%
    }%
  \KVD@temp
}
%    \end{macrocode}
%    \end{macro}
%    \begin{macro}{\KVD@DefineKey}
%    \begin{macrocode}
\def\KVD@DefineKey#1#2{%
  \ltx@ifnextchar[{%
    \KVD@DefineKeyWithDefault{#1}{#2}%
  }{%
    \long\expandafter\def\csname KV@#1@#2\endcsname##1%
  }%
}
%    \end{macrocode}
%    \end{macro}
%    \begin{macro}{\KVD@DefineKeyWithDefault}
%    \begin{macrocode}
\long\def\KVD@DefineKeyWithDefault#1#2[#3]{%
  \expandafter\def\csname KV@#1@#2@default\expandafter\endcsname
  \expandafter{%
    \csname KV@#1@#2\endcsname{#3}%
  }%
  \long\expandafter\def\csname KV@#1@#2\endcsname##1%
}
%    \end{macrocode}
%    \end{macro}
%
%    \begin{macrocode}
\KVD@AtEnd%
%</package>
%    \end{macrocode}
%
% \section{Test}
%
% \subsection{Catcode checks for loading}
%
%    \begin{macrocode}
%<*test1>
%    \end{macrocode}
%    \begin{macrocode}
\catcode`\{=1 %
\catcode`\}=2 %
\catcode`\#=6 %
\catcode`\@=11 %
\expandafter\ifx\csname count@\endcsname\relax
  \countdef\count@=255 %
\fi
\expandafter\ifx\csname @gobble\endcsname\relax
  \long\def\@gobble#1{}%
\fi
\expandafter\ifx\csname @firstofone\endcsname\relax
  \long\def\@firstofone#1{#1}%
\fi
\expandafter\ifx\csname loop\endcsname\relax
  \expandafter\@firstofone
\else
  \expandafter\@gobble
\fi
{%
  \def\loop#1\repeat{%
    \def\body{#1}%
    \iterate
  }%
  \def\iterate{%
    \body
      \let\next\iterate
    \else
      \let\next\relax
    \fi
    \next
  }%
  \let\repeat=\fi
}%
\def\RestoreCatcodes{}
\count@=0 %
\loop
  \edef\RestoreCatcodes{%
    \RestoreCatcodes
    \catcode\the\count@=\the\catcode\count@\relax
  }%
\ifnum\count@<255 %
  \advance\count@ 1 %
\repeat

\def\RangeCatcodeInvalid#1#2{%
  \count@=#1\relax
  \loop
    \catcode\count@=15 %
  \ifnum\count@<#2\relax
    \advance\count@ 1 %
  \repeat
}
\def\RangeCatcodeCheck#1#2#3{%
  \count@=#1\relax
  \loop
    \ifnum#3=\catcode\count@
    \else
      \errmessage{%
        Character \the\count@\space
        with wrong catcode \the\catcode\count@\space
        instead of \number#3%
      }%
    \fi
  \ifnum\count@<#2\relax
    \advance\count@ 1 %
  \repeat
}
\def\space{ }
\expandafter\ifx\csname LoadCommand\endcsname\relax
  \def\LoadCommand{\input kvdefinekeys.sty\relax}%
\fi
\def\Test{%
  \RangeCatcodeInvalid{0}{47}%
  \RangeCatcodeInvalid{58}{64}%
  \RangeCatcodeInvalid{91}{96}%
  \RangeCatcodeInvalid{123}{255}%
  \catcode`\@=12 %
  \catcode`\\=0 %
  \catcode`\%=14 %
  \LoadCommand
  \RangeCatcodeCheck{0}{36}{15}%
  \RangeCatcodeCheck{37}{37}{14}%
  \RangeCatcodeCheck{38}{47}{15}%
  \RangeCatcodeCheck{48}{57}{12}%
  \RangeCatcodeCheck{58}{63}{15}%
  \RangeCatcodeCheck{64}{64}{12}%
  \RangeCatcodeCheck{65}{90}{11}%
  \RangeCatcodeCheck{91}{91}{15}%
  \RangeCatcodeCheck{92}{92}{0}%
  \RangeCatcodeCheck{93}{96}{15}%
  \RangeCatcodeCheck{97}{122}{11}%
  \RangeCatcodeCheck{123}{255}{15}%
  \RestoreCatcodes
}
\Test
\csname @@end\endcsname
\end
%    \end{macrocode}
%    \begin{macrocode}
%</test1>
%    \end{macrocode}
%
% \section{Installation}
%
% \subsection{Download}
%
% \paragraph{Package.} This package is available on
% CTAN\footnote{\url{http://ctan.org/pkg/kvdefinekeys}}:
% \begin{description}
% \item[\CTAN{macros/latex/contrib/oberdiek/kvdefinekeys.dtx}] The source file.
% \item[\CTAN{macros/latex/contrib/oberdiek/kvdefinekeys.pdf}] Documentation.
% \end{description}
%
%
% \paragraph{Bundle.} All the packages of the bundle `oberdiek'
% are also available in a TDS compliant ZIP archive. There
% the packages are already unpacked and the documentation files
% are generated. The files and directories obey the TDS standard.
% \begin{description}
% \item[\CTAN{install/macros/latex/contrib/oberdiek.tds.zip}]
% \end{description}
% \emph{TDS} refers to the standard ``A Directory Structure
% for \TeX\ Files'' (\CTAN{tds/tds.pdf}). Directories
% with \xfile{texmf} in their name are usually organized this way.
%
% \subsection{Bundle installation}
%
% \paragraph{Unpacking.} Unpack the \xfile{oberdiek.tds.zip} in the
% TDS tree (also known as \xfile{texmf} tree) of your choice.
% Example (linux):
% \begin{quote}
%   |unzip oberdiek.tds.zip -d ~/texmf|
% \end{quote}
%
% \paragraph{Script installation.}
% Check the directory \xfile{TDS:scripts/oberdiek/} for
% scripts that need further installation steps.
% Package \xpackage{attachfile2} comes with the Perl script
% \xfile{pdfatfi.pl} that should be installed in such a way
% that it can be called as \texttt{pdfatfi}.
% Example (linux):
% \begin{quote}
%   |chmod +x scripts/oberdiek/pdfatfi.pl|\\
%   |cp scripts/oberdiek/pdfatfi.pl /usr/local/bin/|
% \end{quote}
%
% \subsection{Package installation}
%
% \paragraph{Unpacking.} The \xfile{.dtx} file is a self-extracting
% \docstrip\ archive. The files are extracted by running the
% \xfile{.dtx} through \plainTeX:
% \begin{quote}
%   \verb|tex kvdefinekeys.dtx|
% \end{quote}
%
% \paragraph{TDS.} Now the different files must be moved into
% the different directories in your installation TDS tree
% (also known as \xfile{texmf} tree):
% \begin{quote}
% \def\t{^^A
% \begin{tabular}{@{}>{\ttfamily}l@{ $\rightarrow$ }>{\ttfamily}l@{}}
%   kvdefinekeys.sty & tex/generic/oberdiek/kvdefinekeys.sty\\
%   kvdefinekeys.pdf & doc/latex/oberdiek/kvdefinekeys.pdf\\
%   test/kvdefinekeys-test1.tex & doc/latex/oberdiek/test/kvdefinekeys-test1.tex\\
%   kvdefinekeys.dtx & source/latex/oberdiek/kvdefinekeys.dtx\\
% \end{tabular}^^A
% }^^A
% \sbox0{\t}^^A
% \ifdim\wd0>\linewidth
%   \begingroup
%     \advance\linewidth by\leftmargin
%     \advance\linewidth by\rightmargin
%   \edef\x{\endgroup
%     \def\noexpand\lw{\the\linewidth}^^A
%   }\x
%   \def\lwbox{^^A
%     \leavevmode
%     \hbox to \linewidth{^^A
%       \kern-\leftmargin\relax
%       \hss
%       \usebox0
%       \hss
%       \kern-\rightmargin\relax
%     }^^A
%   }^^A
%   \ifdim\wd0>\lw
%     \sbox0{\small\t}^^A
%     \ifdim\wd0>\linewidth
%       \ifdim\wd0>\lw
%         \sbox0{\footnotesize\t}^^A
%         \ifdim\wd0>\linewidth
%           \ifdim\wd0>\lw
%             \sbox0{\scriptsize\t}^^A
%             \ifdim\wd0>\linewidth
%               \ifdim\wd0>\lw
%                 \sbox0{\tiny\t}^^A
%                 \ifdim\wd0>\linewidth
%                   \lwbox
%                 \else
%                   \usebox0
%                 \fi
%               \else
%                 \lwbox
%               \fi
%             \else
%               \usebox0
%             \fi
%           \else
%             \lwbox
%           \fi
%         \else
%           \usebox0
%         \fi
%       \else
%         \lwbox
%       \fi
%     \else
%       \usebox0
%     \fi
%   \else
%     \lwbox
%   \fi
% \else
%   \usebox0
% \fi
% \end{quote}
% If you have a \xfile{docstrip.cfg} that configures and enables \docstrip's
% TDS installing feature, then some files can already be in the right
% place, see the documentation of \docstrip.
%
% \subsection{Refresh file name databases}
%
% If your \TeX~distribution
% (\teTeX, \mikTeX, \dots) relies on file name databases, you must refresh
% these. For example, \teTeX\ users run \verb|texhash| or
% \verb|mktexlsr|.
%
% \subsection{Some details for the interested}
%
% \paragraph{Attached source.}
%
% The PDF documentation on CTAN also includes the
% \xfile{.dtx} source file. It can be extracted by
% AcrobatReader 6 or higher. Another option is \textsf{pdftk},
% e.g. unpack the file into the current directory:
% \begin{quote}
%   \verb|pdftk kvdefinekeys.pdf unpack_files output .|
% \end{quote}
%
% \paragraph{Unpacking with \LaTeX.}
% The \xfile{.dtx} chooses its action depending on the format:
% \begin{description}
% \item[\plainTeX:] Run \docstrip\ and extract the files.
% \item[\LaTeX:] Generate the documentation.
% \end{description}
% If you insist on using \LaTeX\ for \docstrip\ (really,
% \docstrip\ does not need \LaTeX), then inform the autodetect routine
% about your intention:
% \begin{quote}
%   \verb|latex \let\install=y\input{kvdefinekeys.dtx}|
% \end{quote}
% Do not forget to quote the argument according to the demands
% of your shell.
%
% \paragraph{Generating the documentation.}
% You can use both the \xfile{.dtx} or the \xfile{.drv} to generate
% the documentation. The process can be configured by the
% configuration file \xfile{ltxdoc.cfg}. For instance, put this
% line into this file, if you want to have A4 as paper format:
% \begin{quote}
%   \verb|\PassOptionsToClass{a4paper}{article}|
% \end{quote}
% An example follows how to generate the
% documentation with pdf\LaTeX:
% \begin{quote}
%\begin{verbatim}
%pdflatex kvdefinekeys.dtx
%makeindex -s gind.ist kvdefinekeys.idx
%pdflatex kvdefinekeys.dtx
%makeindex -s gind.ist kvdefinekeys.idx
%pdflatex kvdefinekeys.dtx
%\end{verbatim}
% \end{quote}
%
% \section{Catalogue}
%
% The following XML file can be used as source for the
% \href{http://mirror.ctan.org/help/Catalogue/catalogue.html}{\TeX\ Catalogue}.
% The elements \texttt{caption} and \texttt{description} are imported
% from the original XML file from the Catalogue.
% The name of the XML file in the Catalogue is \xfile{kvdefinekeys.xml}.
%    \begin{macrocode}
%<*catalogue>
<?xml version='1.0' encoding='us-ascii'?>
<!DOCTYPE entry SYSTEM 'catalogue.dtd'>
<entry datestamp='$Date$' modifier='$Author$' id='kvdefinekeys'>
  <name>kvdefinekeys</name>
  <caption>Define keys for use in the kvsetkeys package.</caption>
  <authorref id='auth:oberdiek'/>
  <copyright owner='Heiko Oberdiek' year='2010,2011'/>
  <license type='lppl1.3'/>
  <version number='1.4'/>
  <description>
    The package provides a macro <tt>\kv@define@key</tt> (analogous to
    <xref refid='keyval'>keyval&#x2019;s</xref> <tt>\define@key</tt>, to
    define keys for use by <xref refid='kvsetkeys'>kvsetkeys</xref>.
    <p/>
    The package is part of the <xref refid='oberdiek'>oberdiek</xref>
    bundle.
  </description>
  <documentation details='Package documentation'
      href='ctan:/macros/latex/contrib/oberdiek/kvdefinekeys.pdf'/>
  <ctan file='true' path='/macros/latex/contrib/oberdiek/kvdefinekeys.dtx'/>
  <miktex location='oberdiek'/>
  <texlive location='oberdiek'/>
  <install path='/macros/latex/contrib/oberdiek/oberdiek.tds.zip'/>
</entry>
%</catalogue>
%    \end{macrocode}
%
% \begin{thebibliography}{9}
% \bibitem{keyval}
%   David Carlisle:
%   \textit{The \xpackage{keyval} package};
%   1999/03/16 v1.13;
%   \CTAN{macros/latex/required/graphics/keyval.dtx}.
%
% \end{thebibliography}
%
% \begin{History}
%   \begin{Version}{2010/03/01 v1.0}
%   \item
%     First version.
%   \end{Version}
%   \begin{Version}{2010/08/19 v1.1}
%   \item
%     Documentation fix, no code change.
%   \end{Version}
%   \begin{Version}{2011/01/30 v1.2}
%   \item
%     Already loaded package files are not input in \hologo{plainTeX}.
%   \end{Version}
%   \begin{Version}{2011/04/07 v1.3}
%   \item
%     Support for package \xpackage{babel}'s shorthands added.
%   \item
%     \cs{kv@define@key} is made robust if available.
%   \end{Version}
%   \begin{Version}{2016/05/16 v1.4}
%   \item
%     Documentation updates.
%   \end{Version}
% \end{History}
%
% \PrintIndex
%
% \Finale
\endinput
|
% \end{quote}
% Do not forget to quote the argument according to the demands
% of your shell.
%
% \paragraph{Generating the documentation.}
% You can use both the \xfile{.dtx} or the \xfile{.drv} to generate
% the documentation. The process can be configured by the
% configuration file \xfile{ltxdoc.cfg}. For instance, put this
% line into this file, if you want to have A4 as paper format:
% \begin{quote}
%   \verb|\PassOptionsToClass{a4paper}{article}|
% \end{quote}
% An example follows how to generate the
% documentation with pdf\LaTeX:
% \begin{quote}
%\begin{verbatim}
%pdflatex kvdefinekeys.dtx
%makeindex -s gind.ist kvdefinekeys.idx
%pdflatex kvdefinekeys.dtx
%makeindex -s gind.ist kvdefinekeys.idx
%pdflatex kvdefinekeys.dtx
%\end{verbatim}
% \end{quote}
%
% \section{Catalogue}
%
% The following XML file can be used as source for the
% \href{http://mirror.ctan.org/help/Catalogue/catalogue.html}{\TeX\ Catalogue}.
% The elements \texttt{caption} and \texttt{description} are imported
% from the original XML file from the Catalogue.
% The name of the XML file in the Catalogue is \xfile{kvdefinekeys.xml}.
%    \begin{macrocode}
%<*catalogue>
<?xml version='1.0' encoding='us-ascii'?>
<!DOCTYPE entry SYSTEM 'catalogue.dtd'>
<entry datestamp='$Date$' modifier='$Author$' id='kvdefinekeys'>
  <name>kvdefinekeys</name>
  <caption>Define keys for use in the kvsetkeys package.</caption>
  <authorref id='auth:oberdiek'/>
  <copyright owner='Heiko Oberdiek' year='2010,2011'/>
  <license type='lppl1.3'/>
  <version number='1.4'/>
  <description>
    The package provides a macro <tt>\kv@define@key</tt> (analogous to
    <xref refid='keyval'>keyval&#x2019;s</xref> <tt>\define@key</tt>, to
    define keys for use by <xref refid='kvsetkeys'>kvsetkeys</xref>.
    <p/>
    The package is part of the <xref refid='oberdiek'>oberdiek</xref>
    bundle.
  </description>
  <documentation details='Package documentation'
      href='ctan:/macros/latex/contrib/oberdiek/kvdefinekeys.pdf'/>
  <ctan file='true' path='/macros/latex/contrib/oberdiek/kvdefinekeys.dtx'/>
  <miktex location='oberdiek'/>
  <texlive location='oberdiek'/>
  <install path='/macros/latex/contrib/oberdiek/oberdiek.tds.zip'/>
</entry>
%</catalogue>
%    \end{macrocode}
%
% \begin{thebibliography}{9}
% \bibitem{keyval}
%   David Carlisle:
%   \textit{The \xpackage{keyval} package};
%   1999/03/16 v1.13;
%   \CTAN{macros/latex/required/graphics/keyval.dtx}.
%
% \end{thebibliography}
%
% \begin{History}
%   \begin{Version}{2010/03/01 v1.0}
%   \item
%     First version.
%   \end{Version}
%   \begin{Version}{2010/08/19 v1.1}
%   \item
%     Documentation fix, no code change.
%   \end{Version}
%   \begin{Version}{2011/01/30 v1.2}
%   \item
%     Already loaded package files are not input in \hologo{plainTeX}.
%   \end{Version}
%   \begin{Version}{2011/04/07 v1.3}
%   \item
%     Support for package \xpackage{babel}'s shorthands added.
%   \item
%     \cs{kv@define@key} is made robust if available.
%   \end{Version}
%   \begin{Version}{2016/05/16 v1.4}
%   \item
%     Documentation updates.
%   \end{Version}
% \end{History}
%
% \PrintIndex
%
% \Finale
\endinput
|
% \end{quote}
% Do not forget to quote the argument according to the demands
% of your shell.
%
% \paragraph{Generating the documentation.}
% You can use both the \xfile{.dtx} or the \xfile{.drv} to generate
% the documentation. The process can be configured by the
% configuration file \xfile{ltxdoc.cfg}. For instance, put this
% line into this file, if you want to have A4 as paper format:
% \begin{quote}
%   \verb|\PassOptionsToClass{a4paper}{article}|
% \end{quote}
% An example follows how to generate the
% documentation with pdf\LaTeX:
% \begin{quote}
%\begin{verbatim}
%pdflatex kvdefinekeys.dtx
%makeindex -s gind.ist kvdefinekeys.idx
%pdflatex kvdefinekeys.dtx
%makeindex -s gind.ist kvdefinekeys.idx
%pdflatex kvdefinekeys.dtx
%\end{verbatim}
% \end{quote}
%
% \section{Catalogue}
%
% The following XML file can be used as source for the
% \href{http://mirror.ctan.org/help/Catalogue/catalogue.html}{\TeX\ Catalogue}.
% The elements \texttt{caption} and \texttt{description} are imported
% from the original XML file from the Catalogue.
% The name of the XML file in the Catalogue is \xfile{kvdefinekeys.xml}.
%    \begin{macrocode}
%<*catalogue>
<?xml version='1.0' encoding='us-ascii'?>
<!DOCTYPE entry SYSTEM 'catalogue.dtd'>
<entry datestamp='$Date$' modifier='$Author$' id='kvdefinekeys'>
  <name>kvdefinekeys</name>
  <caption>Define keys for use in the kvsetkeys package.</caption>
  <authorref id='auth:oberdiek'/>
  <copyright owner='Heiko Oberdiek' year='2010,2011'/>
  <license type='lppl1.3'/>
  <version number='1.4'/>
  <description>
    The package provides a macro <tt>\kv@define@key</tt> (analogous to
    <xref refid='keyval'>keyval&#x2019;s</xref> <tt>\define@key</tt>, to
    define keys for use by <xref refid='kvsetkeys'>kvsetkeys</xref>.
    <p/>
    The package is part of the <xref refid='oberdiek'>oberdiek</xref>
    bundle.
  </description>
  <documentation details='Package documentation'
      href='ctan:/macros/latex/contrib/oberdiek/kvdefinekeys.pdf'/>
  <ctan file='true' path='/macros/latex/contrib/oberdiek/kvdefinekeys.dtx'/>
  <miktex location='oberdiek'/>
  <texlive location='oberdiek'/>
  <install path='/macros/latex/contrib/oberdiek/oberdiek.tds.zip'/>
</entry>
%</catalogue>
%    \end{macrocode}
%
% \begin{thebibliography}{9}
% \bibitem{keyval}
%   David Carlisle:
%   \textit{The \xpackage{keyval} package};
%   1999/03/16 v1.13;
%   \CTAN{macros/latex/required/graphics/keyval.dtx}.
%
% \end{thebibliography}
%
% \begin{History}
%   \begin{Version}{2010/03/01 v1.0}
%   \item
%     First version.
%   \end{Version}
%   \begin{Version}{2010/08/19 v1.1}
%   \item
%     Documentation fix, no code change.
%   \end{Version}
%   \begin{Version}{2011/01/30 v1.2}
%   \item
%     Already loaded package files are not input in \hologo{plainTeX}.
%   \end{Version}
%   \begin{Version}{2011/04/07 v1.3}
%   \item
%     Support for package \xpackage{babel}'s shorthands added.
%   \item
%     \cs{kv@define@key} is made robust if available.
%   \end{Version}
%   \begin{Version}{2016/05/16 v1.4}
%   \item
%     Documentation updates.
%   \end{Version}
% \end{History}
%
% \PrintIndex
%
% \Finale
\endinput

%        (quote the arguments according to the demands of your shell)
%
% Documentation:
%    (a) If kvdefinekeys.drv is present:
%           latex kvdefinekeys.drv
%    (b) Without kvdefinekeys.drv:
%           latex kvdefinekeys.dtx; ...
%    The class ltxdoc loads the configuration file ltxdoc.cfg
%    if available. Here you can specify further options, e.g.
%    use A4 as paper format:
%       \PassOptionsToClass{a4paper}{article}
%
%    Programm calls to get the documentation (example):
%       pdflatex kvdefinekeys.dtx
%       makeindex -s gind.ist kvdefinekeys.idx
%       pdflatex kvdefinekeys.dtx
%       makeindex -s gind.ist kvdefinekeys.idx
%       pdflatex kvdefinekeys.dtx
%
% Installation:
%    TDS:tex/generic/oberdiek/kvdefinekeys.sty
%    TDS:doc/latex/oberdiek/kvdefinekeys.pdf
%    TDS:doc/latex/oberdiek/test/kvdefinekeys-test1.tex
%    TDS:source/latex/oberdiek/kvdefinekeys.dtx
%
%<*ignore>
\begingroup
  \catcode123=1 %
  \catcode125=2 %
  \def\x{LaTeX2e}%
\expandafter\endgroup
\ifcase 0\ifx\install y1\fi\expandafter
         \ifx\csname processbatchFile\endcsname\relax\else1\fi
         \ifx\fmtname\x\else 1\fi\relax
\else\csname fi\endcsname
%</ignore>
%<*install>
\input docstrip.tex
\Msg{************************************************************************}
\Msg{* Installation}
\Msg{* Package: kvdefinekeys 2016/05/16 v1.4 Define keys (HO)}
\Msg{************************************************************************}

\keepsilent
\askforoverwritefalse

\let\MetaPrefix\relax
\preamble

This is a generated file.

Project: kvdefinekeys
Version: 2016/05/16 v1.4

Copyright (C) 2010, 2011 by
   Heiko Oberdiek <heiko.oberdiek at googlemail.com>

This work may be distributed and/or modified under the
conditions of the LaTeX Project Public License, either
version 1.3c of this license or (at your option) any later
version. This version of this license is in
   http://www.latex-project.org/lppl/lppl-1-3c.txt
and the latest version of this license is in
   http://www.latex-project.org/lppl.txt
and version 1.3 or later is part of all distributions of
LaTeX version 2005/12/01 or later.

This work has the LPPL maintenance status "maintained".

This Current Maintainer of this work is Heiko Oberdiek.

The Base Interpreter refers to any `TeX-Format',
because some files are installed in TDS:tex/generic//.

This work consists of the main source file kvdefinekeys.dtx
and the derived files
   kvdefinekeys.sty, kvdefinekeys.pdf, kvdefinekeys.ins, kvdefinekeys.drv,
   kvdefinekeys-test1.tex.

\endpreamble
\let\MetaPrefix\DoubleperCent

\generate{%
  \file{kvdefinekeys.ins}{\from{kvdefinekeys.dtx}{install}}%
  \file{kvdefinekeys.drv}{\from{kvdefinekeys.dtx}{driver}}%
  \usedir{tex/generic/oberdiek}%
  \file{kvdefinekeys.sty}{\from{kvdefinekeys.dtx}{package}}%
  \usedir{doc/latex/oberdiek/test}%
  \file{kvdefinekeys-test1.tex}{\from{kvdefinekeys.dtx}{test1}}%
  \nopreamble
  \nopostamble
  \usedir{source/latex/oberdiek/catalogue}%
  \file{kvdefinekeys.xml}{\from{kvdefinekeys.dtx}{catalogue}}%
}

\catcode32=13\relax% active space
\let =\space%
\Msg{************************************************************************}
\Msg{*}
\Msg{* To finish the installation you have to move the following}
\Msg{* file into a directory searched by TeX:}
\Msg{*}
\Msg{*     kvdefinekeys.sty}
\Msg{*}
\Msg{* To produce the documentation run the file `kvdefinekeys.drv'}
\Msg{* through LaTeX.}
\Msg{*}
\Msg{* Happy TeXing!}
\Msg{*}
\Msg{************************************************************************}

\endbatchfile
%</install>
%<*ignore>
\fi
%</ignore>
%<*driver>
\NeedsTeXFormat{LaTeX2e}
\ProvidesFile{kvdefinekeys.drv}%
  [2016/05/16 v1.4 Define keys (HO)]%
\documentclass{ltxdoc}
\usepackage{holtxdoc}[2011/11/22]
\begin{document}
  \DocInput{kvdefinekeys.dtx}%
\end{document}
%</driver>
% \fi
%
%
% \CharacterTable
%  {Upper-case    \A\B\C\D\E\F\G\H\I\J\K\L\M\N\O\P\Q\R\S\T\U\V\W\X\Y\Z
%   Lower-case    \a\b\c\d\e\f\g\h\i\j\k\l\m\n\o\p\q\r\s\t\u\v\w\x\y\z
%   Digits        \0\1\2\3\4\5\6\7\8\9
%   Exclamation   \!     Double quote  \"     Hash (number) \#
%   Dollar        \$     Percent       \%     Ampersand     \&
%   Acute accent  \'     Left paren    \(     Right paren   \)
%   Asterisk      \*     Plus          \+     Comma         \,
%   Minus         \-     Point         \.     Solidus       \/
%   Colon         \:     Semicolon     \;     Less than     \<
%   Equals        \=     Greater than  \>     Question mark \?
%   Commercial at \@     Left bracket  \[     Backslash     \\
%   Right bracket \]     Circumflex    \^     Underscore    \_
%   Grave accent  \`     Left brace    \{     Vertical bar  \|
%   Right brace   \}     Tilde         \~}
%
% \GetFileInfo{kvdefinekeys.drv}
%
% \title{The \xpackage{kvdefinekeys} package}
% \date{2016/05/16 v1.4}
% \author{Heiko Oberdiek\thanks
% {Please report any issues at https://github.com/ho-tex/oberdiek/issues}\\
% \xemail{heiko.oberdiek at googlemail.com}}
%
% \maketitle
%
% \begin{abstract}
% Package \xpackage{kvdefinekeys} provides \cs{kv@define@key} to define
% keys the same way as \xpackage{keyval}'s \cs{define@key}. However, it
% works also using \iniTeX.
% \end{abstract}
%
% \tableofcontents
%
% \def\M#1{\texttt{\{}\meta{#1}\texttt{\}}}
%
% \section{Documentation}
%
% \subsection{Motivation}
%
% \cs{kvsetkeys} serves as replacement for \xpackage{keyval}'s
% \cs{setkeys}. This package adds macros to define keys, closing
% the gap \cs{kvsetkeys} leaves.
%
% \begin{declcs}{kv@define@key}\,\M{family}\,\M{key}\,[\meta{default}]\,^^A
%   \M{definition}
% \end{declcs}
% Macro \cs{kv@define@key} reimplements \xpackage{keyval}'s
% \cs{define@key}. Differences to the original:
% \begin{itemize}
% \item The defined keys also allow \cs{par} inside values.
% \item Shorthands of package \xpackage{babel} are supported in
%   family and key names.
% \item Macro \cs{kv@define@key} is made robust if
%   \hologo{eTeX}'s \cs{protected} or \hologo{LaTeX}'s
%   \cs{DeclareRobustCommand} are found.
% \end{itemize}
%
% \StopEventually{
% }
%
% \section{Implementation}
%
% \subsection{Identification}
%
%    \begin{macrocode}
%<*package>
%    \end{macrocode}
%    Reload check, especially if the package is not used with \LaTeX.
%    \begin{macrocode}
\begingroup\catcode61\catcode48\catcode32=10\relax%
  \catcode13=5 % ^^M
  \endlinechar=13 %
  \catcode35=6 % #
  \catcode39=12 % '
  \catcode44=12 % ,
  \catcode45=12 % -
  \catcode46=12 % .
  \catcode58=12 % :
  \catcode64=11 % @
  \catcode123=1 % {
  \catcode125=2 % }
  \expandafter\let\expandafter\x\csname ver@kvdefinekeys.sty\endcsname
  \ifx\x\relax % plain-TeX, first loading
  \else
    \def\empty{}%
    \ifx\x\empty % LaTeX, first loading,
      % variable is initialized, but \ProvidesPackage not yet seen
    \else
      \expandafter\ifx\csname PackageInfo\endcsname\relax
        \def\x#1#2{%
          \immediate\write-1{Package #1 Info: #2.}%
        }%
      \else
        \def\x#1#2{\PackageInfo{#1}{#2, stopped}}%
      \fi
      \x{kvdefinekeys}{The package is already loaded}%
      \aftergroup\endinput
    \fi
  \fi
\endgroup%
%    \end{macrocode}
%    Package identification:
%    \begin{macrocode}
\begingroup\catcode61\catcode48\catcode32=10\relax%
  \catcode13=5 % ^^M
  \endlinechar=13 %
  \catcode35=6 % #
  \catcode39=12 % '
  \catcode40=12 % (
  \catcode41=12 % )
  \catcode44=12 % ,
  \catcode45=12 % -
  \catcode46=12 % .
  \catcode47=12 % /
  \catcode58=12 % :
  \catcode64=11 % @
  \catcode91=12 % [
  \catcode93=12 % ]
  \catcode123=1 % {
  \catcode125=2 % }
  \expandafter\ifx\csname ProvidesPackage\endcsname\relax
    \def\x#1#2#3[#4]{\endgroup
      \immediate\write-1{Package: #3 #4}%
      \xdef#1{#4}%
    }%
  \else
    \def\x#1#2[#3]{\endgroup
      #2[{#3}]%
      \ifx#1\@undefined
        \xdef#1{#3}%
      \fi
      \ifx#1\relax
        \xdef#1{#3}%
      \fi
    }%
  \fi
\expandafter\x\csname ver@kvdefinekeys.sty\endcsname
\ProvidesPackage{kvdefinekeys}%
  [2016/05/16 v1.4 Define keys (HO)]%
%    \end{macrocode}
%
%    \begin{macrocode}
\begingroup\catcode61\catcode48\catcode32=10\relax%
  \catcode13=5 % ^^M
  \endlinechar=13 %
  \catcode123=1 % {
  \catcode125=2 % }
  \catcode64=11 % @
  \def\x{\endgroup
    \expandafter\edef\csname KVD@AtEnd\endcsname{%
      \endlinechar=\the\endlinechar\relax
      \catcode13=\the\catcode13\relax
      \catcode32=\the\catcode32\relax
      \catcode35=\the\catcode35\relax
      \catcode61=\the\catcode61\relax
      \catcode64=\the\catcode64\relax
      \catcode123=\the\catcode123\relax
      \catcode125=\the\catcode125\relax
    }%
  }%
\x\catcode61\catcode48\catcode32=10\relax%
\catcode13=5 % ^^M
\endlinechar=13 %
\catcode35=6 % #
\catcode64=11 % @
\catcode123=1 % {
\catcode125=2 % }
\def\TMP@EnsureCode#1#2{%
  \edef\KVD@AtEnd{%
    \KVD@AtEnd
    \catcode#1=\the\catcode#1\relax
  }%
  \catcode#1=#2\relax
}
\TMP@EnsureCode{42}{12}% *
\TMP@EnsureCode{46}{12}% .
\TMP@EnsureCode{47}{12}% /
\TMP@EnsureCode{91}{12}% [
\TMP@EnsureCode{93}{12}% ]
\edef\KVD@AtEnd{\KVD@AtEnd\noexpand\endinput}
%    \end{macrocode}
%
% \subsection{Package loading}
%
%    \begin{macrocode}
\begingroup\expandafter\expandafter\expandafter\endgroup
\expandafter\ifx\csname RequirePackage\endcsname\relax
  \def\TMP@RequirePackage#1[#2]{%
    \begingroup\expandafter\expandafter\expandafter\endgroup
    \expandafter\ifx\csname ver@#1.sty\endcsname\relax
      \input #1.sty\relax
    \fi
  }%
  \TMP@RequirePackage{ltxcmds}[2010/03/01]%
\else
  \RequirePackage{ltxcmds}[2010/03/01]%
\fi
%    \end{macrocode}
%
%
% \subsection{Provide key defining macro}
%
%    \begin{macro}{\kv@define@key}
%    \begin{macrocode}
\ltx@IfUndefined{protected}{%
  \ltx@IfUndefined{DeclareRobustCommand}{%
    \def\kv@define@key#1#2%
  }{%
    \DeclareRobustCommand*{\kv@define@key}[2]%
  }%
}{%
  \protected\def\kv@define@key#1#2%
}%
{%
  \begingroup
    \csname @safe@activestrue\endcsname
    \let\ifincsname\iftrue
    \edef\KVD@temp{\endgroup
      \noexpand\KVD@DefineKey{#1}{#2}%
    }%
  \KVD@temp
}
%    \end{macrocode}
%    \end{macro}
%    \begin{macro}{\KVD@DefineKey}
%    \begin{macrocode}
\def\KVD@DefineKey#1#2{%
  \ltx@ifnextchar[{%
    \KVD@DefineKeyWithDefault{#1}{#2}%
  }{%
    \long\expandafter\def\csname KV@#1@#2\endcsname##1%
  }%
}
%    \end{macrocode}
%    \end{macro}
%    \begin{macro}{\KVD@DefineKeyWithDefault}
%    \begin{macrocode}
\long\def\KVD@DefineKeyWithDefault#1#2[#3]{%
  \expandafter\def\csname KV@#1@#2@default\expandafter\endcsname
  \expandafter{%
    \csname KV@#1@#2\endcsname{#3}%
  }%
  \long\expandafter\def\csname KV@#1@#2\endcsname##1%
}
%    \end{macrocode}
%    \end{macro}
%
%    \begin{macrocode}
\KVD@AtEnd%
%</package>
%    \end{macrocode}
%
% \section{Test}
%
% \subsection{Catcode checks for loading}
%
%    \begin{macrocode}
%<*test1>
%    \end{macrocode}
%    \begin{macrocode}
\catcode`\{=1 %
\catcode`\}=2 %
\catcode`\#=6 %
\catcode`\@=11 %
\expandafter\ifx\csname count@\endcsname\relax
  \countdef\count@=255 %
\fi
\expandafter\ifx\csname @gobble\endcsname\relax
  \long\def\@gobble#1{}%
\fi
\expandafter\ifx\csname @firstofone\endcsname\relax
  \long\def\@firstofone#1{#1}%
\fi
\expandafter\ifx\csname loop\endcsname\relax
  \expandafter\@firstofone
\else
  \expandafter\@gobble
\fi
{%
  \def\loop#1\repeat{%
    \def\body{#1}%
    \iterate
  }%
  \def\iterate{%
    \body
      \let\next\iterate
    \else
      \let\next\relax
    \fi
    \next
  }%
  \let\repeat=\fi
}%
\def\RestoreCatcodes{}
\count@=0 %
\loop
  \edef\RestoreCatcodes{%
    \RestoreCatcodes
    \catcode\the\count@=\the\catcode\count@\relax
  }%
\ifnum\count@<255 %
  \advance\count@ 1 %
\repeat

\def\RangeCatcodeInvalid#1#2{%
  \count@=#1\relax
  \loop
    \catcode\count@=15 %
  \ifnum\count@<#2\relax
    \advance\count@ 1 %
  \repeat
}
\def\RangeCatcodeCheck#1#2#3{%
  \count@=#1\relax
  \loop
    \ifnum#3=\catcode\count@
    \else
      \errmessage{%
        Character \the\count@\space
        with wrong catcode \the\catcode\count@\space
        instead of \number#3%
      }%
    \fi
  \ifnum\count@<#2\relax
    \advance\count@ 1 %
  \repeat
}
\def\space{ }
\expandafter\ifx\csname LoadCommand\endcsname\relax
  \def\LoadCommand{\input kvdefinekeys.sty\relax}%
\fi
\def\Test{%
  \RangeCatcodeInvalid{0}{47}%
  \RangeCatcodeInvalid{58}{64}%
  \RangeCatcodeInvalid{91}{96}%
  \RangeCatcodeInvalid{123}{255}%
  \catcode`\@=12 %
  \catcode`\\=0 %
  \catcode`\%=14 %
  \LoadCommand
  \RangeCatcodeCheck{0}{36}{15}%
  \RangeCatcodeCheck{37}{37}{14}%
  \RangeCatcodeCheck{38}{47}{15}%
  \RangeCatcodeCheck{48}{57}{12}%
  \RangeCatcodeCheck{58}{63}{15}%
  \RangeCatcodeCheck{64}{64}{12}%
  \RangeCatcodeCheck{65}{90}{11}%
  \RangeCatcodeCheck{91}{91}{15}%
  \RangeCatcodeCheck{92}{92}{0}%
  \RangeCatcodeCheck{93}{96}{15}%
  \RangeCatcodeCheck{97}{122}{11}%
  \RangeCatcodeCheck{123}{255}{15}%
  \RestoreCatcodes
}
\Test
\csname @@end\endcsname
\end
%    \end{macrocode}
%    \begin{macrocode}
%</test1>
%    \end{macrocode}
%
% \section{Installation}
%
% \subsection{Download}
%
% \paragraph{Package.} This package is available on
% CTAN\footnote{\url{http://ctan.org/pkg/kvdefinekeys}}:
% \begin{description}
% \item[\CTAN{macros/latex/contrib/oberdiek/kvdefinekeys.dtx}] The source file.
% \item[\CTAN{macros/latex/contrib/oberdiek/kvdefinekeys.pdf}] Documentation.
% \end{description}
%
%
% \paragraph{Bundle.} All the packages of the bundle `oberdiek'
% are also available in a TDS compliant ZIP archive. There
% the packages are already unpacked and the documentation files
% are generated. The files and directories obey the TDS standard.
% \begin{description}
% \item[\CTAN{install/macros/latex/contrib/oberdiek.tds.zip}]
% \end{description}
% \emph{TDS} refers to the standard ``A Directory Structure
% for \TeX\ Files'' (\CTAN{tds/tds.pdf}). Directories
% with \xfile{texmf} in their name are usually organized this way.
%
% \subsection{Bundle installation}
%
% \paragraph{Unpacking.} Unpack the \xfile{oberdiek.tds.zip} in the
% TDS tree (also known as \xfile{texmf} tree) of your choice.
% Example (linux):
% \begin{quote}
%   |unzip oberdiek.tds.zip -d ~/texmf|
% \end{quote}
%
% \paragraph{Script installation.}
% Check the directory \xfile{TDS:scripts/oberdiek/} for
% scripts that need further installation steps.
% Package \xpackage{attachfile2} comes with the Perl script
% \xfile{pdfatfi.pl} that should be installed in such a way
% that it can be called as \texttt{pdfatfi}.
% Example (linux):
% \begin{quote}
%   |chmod +x scripts/oberdiek/pdfatfi.pl|\\
%   |cp scripts/oberdiek/pdfatfi.pl /usr/local/bin/|
% \end{quote}
%
% \subsection{Package installation}
%
% \paragraph{Unpacking.} The \xfile{.dtx} file is a self-extracting
% \docstrip\ archive. The files are extracted by running the
% \xfile{.dtx} through \plainTeX:
% \begin{quote}
%   \verb|tex kvdefinekeys.dtx|
% \end{quote}
%
% \paragraph{TDS.} Now the different files must be moved into
% the different directories in your installation TDS tree
% (also known as \xfile{texmf} tree):
% \begin{quote}
% \def\t{^^A
% \begin{tabular}{@{}>{\ttfamily}l@{ $\rightarrow$ }>{\ttfamily}l@{}}
%   kvdefinekeys.sty & tex/generic/oberdiek/kvdefinekeys.sty\\
%   kvdefinekeys.pdf & doc/latex/oberdiek/kvdefinekeys.pdf\\
%   test/kvdefinekeys-test1.tex & doc/latex/oberdiek/test/kvdefinekeys-test1.tex\\
%   kvdefinekeys.dtx & source/latex/oberdiek/kvdefinekeys.dtx\\
% \end{tabular}^^A
% }^^A
% \sbox0{\t}^^A
% \ifdim\wd0>\linewidth
%   \begingroup
%     \advance\linewidth by\leftmargin
%     \advance\linewidth by\rightmargin
%   \edef\x{\endgroup
%     \def\noexpand\lw{\the\linewidth}^^A
%   }\x
%   \def\lwbox{^^A
%     \leavevmode
%     \hbox to \linewidth{^^A
%       \kern-\leftmargin\relax
%       \hss
%       \usebox0
%       \hss
%       \kern-\rightmargin\relax
%     }^^A
%   }^^A
%   \ifdim\wd0>\lw
%     \sbox0{\small\t}^^A
%     \ifdim\wd0>\linewidth
%       \ifdim\wd0>\lw
%         \sbox0{\footnotesize\t}^^A
%         \ifdim\wd0>\linewidth
%           \ifdim\wd0>\lw
%             \sbox0{\scriptsize\t}^^A
%             \ifdim\wd0>\linewidth
%               \ifdim\wd0>\lw
%                 \sbox0{\tiny\t}^^A
%                 \ifdim\wd0>\linewidth
%                   \lwbox
%                 \else
%                   \usebox0
%                 \fi
%               \else
%                 \lwbox
%               \fi
%             \else
%               \usebox0
%             \fi
%           \else
%             \lwbox
%           \fi
%         \else
%           \usebox0
%         \fi
%       \else
%         \lwbox
%       \fi
%     \else
%       \usebox0
%     \fi
%   \else
%     \lwbox
%   \fi
% \else
%   \usebox0
% \fi
% \end{quote}
% If you have a \xfile{docstrip.cfg} that configures and enables \docstrip's
% TDS installing feature, then some files can already be in the right
% place, see the documentation of \docstrip.
%
% \subsection{Refresh file name databases}
%
% If your \TeX~distribution
% (\teTeX, \mikTeX, \dots) relies on file name databases, you must refresh
% these. For example, \teTeX\ users run \verb|texhash| or
% \verb|mktexlsr|.
%
% \subsection{Some details for the interested}
%
% \paragraph{Attached source.}
%
% The PDF documentation on CTAN also includes the
% \xfile{.dtx} source file. It can be extracted by
% AcrobatReader 6 or higher. Another option is \textsf{pdftk},
% e.g. unpack the file into the current directory:
% \begin{quote}
%   \verb|pdftk kvdefinekeys.pdf unpack_files output .|
% \end{quote}
%
% \paragraph{Unpacking with \LaTeX.}
% The \xfile{.dtx} chooses its action depending on the format:
% \begin{description}
% \item[\plainTeX:] Run \docstrip\ and extract the files.
% \item[\LaTeX:] Generate the documentation.
% \end{description}
% If you insist on using \LaTeX\ for \docstrip\ (really,
% \docstrip\ does not need \LaTeX), then inform the autodetect routine
% about your intention:
% \begin{quote}
%   \verb|latex \let\install=y% \iffalse meta-comment
%
% File: kvdefinekeys.dtx
% Version: 2016/05/16 v1.4
% Info: Define keys
%
% Copyright (C) 2010, 2011 by
%    Heiko Oberdiek <heiko.oberdiek at googlemail.com>
%    2016
%    https://github.com/ho-tex/oberdiek/issues
%
% This work may be distributed and/or modified under the
% conditions of the LaTeX Project Public License, either
% version 1.3c of this license or (at your option) any later
% version. This version of this license is in
%    http://www.latex-project.org/lppl/lppl-1-3c.txt
% and the latest version of this license is in
%    http://www.latex-project.org/lppl.txt
% and version 1.3 or later is part of all distributions of
% LaTeX version 2005/12/01 or later.
%
% This work has the LPPL maintenance status "maintained".
%
% This Current Maintainer of this work is Heiko Oberdiek.
%
% The Base Interpreter refers to any `TeX-Format',
% because some files are installed in TDS:tex/generic//.
%
% This work consists of the main source file kvdefinekeys.dtx
% and the derived files
%    kvdefinekeys.sty, kvdefinekeys.pdf, kvdefinekeys.ins, kvdefinekeys.drv,
%    kvdefinekeys-test1.tex.
%
% Distribution:
%    CTAN:macros/latex/contrib/oberdiek/kvdefinekeys.dtx
%    CTAN:macros/latex/contrib/oberdiek/kvdefinekeys.pdf
%
% Unpacking:
%    (a) If kvdefinekeys.ins is present:
%           tex kvdefinekeys.ins
%    (b) Without kvdefinekeys.ins:
%           tex kvdefinekeys.dtx
%    (c) If you insist on using LaTeX
%           latex \let\install=y% \iffalse meta-comment
%
% File: kvdefinekeys.dtx
% Version: 2016/05/16 v1.4
% Info: Define keys
%
% Copyright (C) 2010, 2011 by
%    Heiko Oberdiek <heiko.oberdiek at googlemail.com>
%    2016
%    https://github.com/ho-tex/oberdiek/issues
%
% This work may be distributed and/or modified under the
% conditions of the LaTeX Project Public License, either
% version 1.3c of this license or (at your option) any later
% version. This version of this license is in
%    http://www.latex-project.org/lppl/lppl-1-3c.txt
% and the latest version of this license is in
%    http://www.latex-project.org/lppl.txt
% and version 1.3 or later is part of all distributions of
% LaTeX version 2005/12/01 or later.
%
% This work has the LPPL maintenance status "maintained".
%
% This Current Maintainer of this work is Heiko Oberdiek.
%
% The Base Interpreter refers to any `TeX-Format',
% because some files are installed in TDS:tex/generic//.
%
% This work consists of the main source file kvdefinekeys.dtx
% and the derived files
%    kvdefinekeys.sty, kvdefinekeys.pdf, kvdefinekeys.ins, kvdefinekeys.drv,
%    kvdefinekeys-test1.tex.
%
% Distribution:
%    CTAN:macros/latex/contrib/oberdiek/kvdefinekeys.dtx
%    CTAN:macros/latex/contrib/oberdiek/kvdefinekeys.pdf
%
% Unpacking:
%    (a) If kvdefinekeys.ins is present:
%           tex kvdefinekeys.ins
%    (b) Without kvdefinekeys.ins:
%           tex kvdefinekeys.dtx
%    (c) If you insist on using LaTeX
%           latex \let\install=y% \iffalse meta-comment
%
% File: kvdefinekeys.dtx
% Version: 2016/05/16 v1.4
% Info: Define keys
%
% Copyright (C) 2010, 2011 by
%    Heiko Oberdiek <heiko.oberdiek at googlemail.com>
%    2016
%    https://github.com/ho-tex/oberdiek/issues
%
% This work may be distributed and/or modified under the
% conditions of the LaTeX Project Public License, either
% version 1.3c of this license or (at your option) any later
% version. This version of this license is in
%    http://www.latex-project.org/lppl/lppl-1-3c.txt
% and the latest version of this license is in
%    http://www.latex-project.org/lppl.txt
% and version 1.3 or later is part of all distributions of
% LaTeX version 2005/12/01 or later.
%
% This work has the LPPL maintenance status "maintained".
%
% This Current Maintainer of this work is Heiko Oberdiek.
%
% The Base Interpreter refers to any `TeX-Format',
% because some files are installed in TDS:tex/generic//.
%
% This work consists of the main source file kvdefinekeys.dtx
% and the derived files
%    kvdefinekeys.sty, kvdefinekeys.pdf, kvdefinekeys.ins, kvdefinekeys.drv,
%    kvdefinekeys-test1.tex.
%
% Distribution:
%    CTAN:macros/latex/contrib/oberdiek/kvdefinekeys.dtx
%    CTAN:macros/latex/contrib/oberdiek/kvdefinekeys.pdf
%
% Unpacking:
%    (a) If kvdefinekeys.ins is present:
%           tex kvdefinekeys.ins
%    (b) Without kvdefinekeys.ins:
%           tex kvdefinekeys.dtx
%    (c) If you insist on using LaTeX
%           latex \let\install=y\input{kvdefinekeys.dtx}
%        (quote the arguments according to the demands of your shell)
%
% Documentation:
%    (a) If kvdefinekeys.drv is present:
%           latex kvdefinekeys.drv
%    (b) Without kvdefinekeys.drv:
%           latex kvdefinekeys.dtx; ...
%    The class ltxdoc loads the configuration file ltxdoc.cfg
%    if available. Here you can specify further options, e.g.
%    use A4 as paper format:
%       \PassOptionsToClass{a4paper}{article}
%
%    Programm calls to get the documentation (example):
%       pdflatex kvdefinekeys.dtx
%       makeindex -s gind.ist kvdefinekeys.idx
%       pdflatex kvdefinekeys.dtx
%       makeindex -s gind.ist kvdefinekeys.idx
%       pdflatex kvdefinekeys.dtx
%
% Installation:
%    TDS:tex/generic/oberdiek/kvdefinekeys.sty
%    TDS:doc/latex/oberdiek/kvdefinekeys.pdf
%    TDS:doc/latex/oberdiek/test/kvdefinekeys-test1.tex
%    TDS:source/latex/oberdiek/kvdefinekeys.dtx
%
%<*ignore>
\begingroup
  \catcode123=1 %
  \catcode125=2 %
  \def\x{LaTeX2e}%
\expandafter\endgroup
\ifcase 0\ifx\install y1\fi\expandafter
         \ifx\csname processbatchFile\endcsname\relax\else1\fi
         \ifx\fmtname\x\else 1\fi\relax
\else\csname fi\endcsname
%</ignore>
%<*install>
\input docstrip.tex
\Msg{************************************************************************}
\Msg{* Installation}
\Msg{* Package: kvdefinekeys 2016/05/16 v1.4 Define keys (HO)}
\Msg{************************************************************************}

\keepsilent
\askforoverwritefalse

\let\MetaPrefix\relax
\preamble

This is a generated file.

Project: kvdefinekeys
Version: 2016/05/16 v1.4

Copyright (C) 2010, 2011 by
   Heiko Oberdiek <heiko.oberdiek at googlemail.com>

This work may be distributed and/or modified under the
conditions of the LaTeX Project Public License, either
version 1.3c of this license or (at your option) any later
version. This version of this license is in
   http://www.latex-project.org/lppl/lppl-1-3c.txt
and the latest version of this license is in
   http://www.latex-project.org/lppl.txt
and version 1.3 or later is part of all distributions of
LaTeX version 2005/12/01 or later.

This work has the LPPL maintenance status "maintained".

This Current Maintainer of this work is Heiko Oberdiek.

The Base Interpreter refers to any `TeX-Format',
because some files are installed in TDS:tex/generic//.

This work consists of the main source file kvdefinekeys.dtx
and the derived files
   kvdefinekeys.sty, kvdefinekeys.pdf, kvdefinekeys.ins, kvdefinekeys.drv,
   kvdefinekeys-test1.tex.

\endpreamble
\let\MetaPrefix\DoubleperCent

\generate{%
  \file{kvdefinekeys.ins}{\from{kvdefinekeys.dtx}{install}}%
  \file{kvdefinekeys.drv}{\from{kvdefinekeys.dtx}{driver}}%
  \usedir{tex/generic/oberdiek}%
  \file{kvdefinekeys.sty}{\from{kvdefinekeys.dtx}{package}}%
  \usedir{doc/latex/oberdiek/test}%
  \file{kvdefinekeys-test1.tex}{\from{kvdefinekeys.dtx}{test1}}%
  \nopreamble
  \nopostamble
  \usedir{source/latex/oberdiek/catalogue}%
  \file{kvdefinekeys.xml}{\from{kvdefinekeys.dtx}{catalogue}}%
}

\catcode32=13\relax% active space
\let =\space%
\Msg{************************************************************************}
\Msg{*}
\Msg{* To finish the installation you have to move the following}
\Msg{* file into a directory searched by TeX:}
\Msg{*}
\Msg{*     kvdefinekeys.sty}
\Msg{*}
\Msg{* To produce the documentation run the file `kvdefinekeys.drv'}
\Msg{* through LaTeX.}
\Msg{*}
\Msg{* Happy TeXing!}
\Msg{*}
\Msg{************************************************************************}

\endbatchfile
%</install>
%<*ignore>
\fi
%</ignore>
%<*driver>
\NeedsTeXFormat{LaTeX2e}
\ProvidesFile{kvdefinekeys.drv}%
  [2016/05/16 v1.4 Define keys (HO)]%
\documentclass{ltxdoc}
\usepackage{holtxdoc}[2011/11/22]
\begin{document}
  \DocInput{kvdefinekeys.dtx}%
\end{document}
%</driver>
% \fi
%
%
% \CharacterTable
%  {Upper-case    \A\B\C\D\E\F\G\H\I\J\K\L\M\N\O\P\Q\R\S\T\U\V\W\X\Y\Z
%   Lower-case    \a\b\c\d\e\f\g\h\i\j\k\l\m\n\o\p\q\r\s\t\u\v\w\x\y\z
%   Digits        \0\1\2\3\4\5\6\7\8\9
%   Exclamation   \!     Double quote  \"     Hash (number) \#
%   Dollar        \$     Percent       \%     Ampersand     \&
%   Acute accent  \'     Left paren    \(     Right paren   \)
%   Asterisk      \*     Plus          \+     Comma         \,
%   Minus         \-     Point         \.     Solidus       \/
%   Colon         \:     Semicolon     \;     Less than     \<
%   Equals        \=     Greater than  \>     Question mark \?
%   Commercial at \@     Left bracket  \[     Backslash     \\
%   Right bracket \]     Circumflex    \^     Underscore    \_
%   Grave accent  \`     Left brace    \{     Vertical bar  \|
%   Right brace   \}     Tilde         \~}
%
% \GetFileInfo{kvdefinekeys.drv}
%
% \title{The \xpackage{kvdefinekeys} package}
% \date{2016/05/16 v1.4}
% \author{Heiko Oberdiek\thanks
% {Please report any issues at https://github.com/ho-tex/oberdiek/issues}\\
% \xemail{heiko.oberdiek at googlemail.com}}
%
% \maketitle
%
% \begin{abstract}
% Package \xpackage{kvdefinekeys} provides \cs{kv@define@key} to define
% keys the same way as \xpackage{keyval}'s \cs{define@key}. However, it
% works also using \iniTeX.
% \end{abstract}
%
% \tableofcontents
%
% \def\M#1{\texttt{\{}\meta{#1}\texttt{\}}}
%
% \section{Documentation}
%
% \subsection{Motivation}
%
% \cs{kvsetkeys} serves as replacement for \xpackage{keyval}'s
% \cs{setkeys}. This package adds macros to define keys, closing
% the gap \cs{kvsetkeys} leaves.
%
% \begin{declcs}{kv@define@key}\,\M{family}\,\M{key}\,[\meta{default}]\,^^A
%   \M{definition}
% \end{declcs}
% Macro \cs{kv@define@key} reimplements \xpackage{keyval}'s
% \cs{define@key}. Differences to the original:
% \begin{itemize}
% \item The defined keys also allow \cs{par} inside values.
% \item Shorthands of package \xpackage{babel} are supported in
%   family and key names.
% \item Macro \cs{kv@define@key} is made robust if
%   \hologo{eTeX}'s \cs{protected} or \hologo{LaTeX}'s
%   \cs{DeclareRobustCommand} are found.
% \end{itemize}
%
% \StopEventually{
% }
%
% \section{Implementation}
%
% \subsection{Identification}
%
%    \begin{macrocode}
%<*package>
%    \end{macrocode}
%    Reload check, especially if the package is not used with \LaTeX.
%    \begin{macrocode}
\begingroup\catcode61\catcode48\catcode32=10\relax%
  \catcode13=5 % ^^M
  \endlinechar=13 %
  \catcode35=6 % #
  \catcode39=12 % '
  \catcode44=12 % ,
  \catcode45=12 % -
  \catcode46=12 % .
  \catcode58=12 % :
  \catcode64=11 % @
  \catcode123=1 % {
  \catcode125=2 % }
  \expandafter\let\expandafter\x\csname ver@kvdefinekeys.sty\endcsname
  \ifx\x\relax % plain-TeX, first loading
  \else
    \def\empty{}%
    \ifx\x\empty % LaTeX, first loading,
      % variable is initialized, but \ProvidesPackage not yet seen
    \else
      \expandafter\ifx\csname PackageInfo\endcsname\relax
        \def\x#1#2{%
          \immediate\write-1{Package #1 Info: #2.}%
        }%
      \else
        \def\x#1#2{\PackageInfo{#1}{#2, stopped}}%
      \fi
      \x{kvdefinekeys}{The package is already loaded}%
      \aftergroup\endinput
    \fi
  \fi
\endgroup%
%    \end{macrocode}
%    Package identification:
%    \begin{macrocode}
\begingroup\catcode61\catcode48\catcode32=10\relax%
  \catcode13=5 % ^^M
  \endlinechar=13 %
  \catcode35=6 % #
  \catcode39=12 % '
  \catcode40=12 % (
  \catcode41=12 % )
  \catcode44=12 % ,
  \catcode45=12 % -
  \catcode46=12 % .
  \catcode47=12 % /
  \catcode58=12 % :
  \catcode64=11 % @
  \catcode91=12 % [
  \catcode93=12 % ]
  \catcode123=1 % {
  \catcode125=2 % }
  \expandafter\ifx\csname ProvidesPackage\endcsname\relax
    \def\x#1#2#3[#4]{\endgroup
      \immediate\write-1{Package: #3 #4}%
      \xdef#1{#4}%
    }%
  \else
    \def\x#1#2[#3]{\endgroup
      #2[{#3}]%
      \ifx#1\@undefined
        \xdef#1{#3}%
      \fi
      \ifx#1\relax
        \xdef#1{#3}%
      \fi
    }%
  \fi
\expandafter\x\csname ver@kvdefinekeys.sty\endcsname
\ProvidesPackage{kvdefinekeys}%
  [2016/05/16 v1.4 Define keys (HO)]%
%    \end{macrocode}
%
%    \begin{macrocode}
\begingroup\catcode61\catcode48\catcode32=10\relax%
  \catcode13=5 % ^^M
  \endlinechar=13 %
  \catcode123=1 % {
  \catcode125=2 % }
  \catcode64=11 % @
  \def\x{\endgroup
    \expandafter\edef\csname KVD@AtEnd\endcsname{%
      \endlinechar=\the\endlinechar\relax
      \catcode13=\the\catcode13\relax
      \catcode32=\the\catcode32\relax
      \catcode35=\the\catcode35\relax
      \catcode61=\the\catcode61\relax
      \catcode64=\the\catcode64\relax
      \catcode123=\the\catcode123\relax
      \catcode125=\the\catcode125\relax
    }%
  }%
\x\catcode61\catcode48\catcode32=10\relax%
\catcode13=5 % ^^M
\endlinechar=13 %
\catcode35=6 % #
\catcode64=11 % @
\catcode123=1 % {
\catcode125=2 % }
\def\TMP@EnsureCode#1#2{%
  \edef\KVD@AtEnd{%
    \KVD@AtEnd
    \catcode#1=\the\catcode#1\relax
  }%
  \catcode#1=#2\relax
}
\TMP@EnsureCode{42}{12}% *
\TMP@EnsureCode{46}{12}% .
\TMP@EnsureCode{47}{12}% /
\TMP@EnsureCode{91}{12}% [
\TMP@EnsureCode{93}{12}% ]
\edef\KVD@AtEnd{\KVD@AtEnd\noexpand\endinput}
%    \end{macrocode}
%
% \subsection{Package loading}
%
%    \begin{macrocode}
\begingroup\expandafter\expandafter\expandafter\endgroup
\expandafter\ifx\csname RequirePackage\endcsname\relax
  \def\TMP@RequirePackage#1[#2]{%
    \begingroup\expandafter\expandafter\expandafter\endgroup
    \expandafter\ifx\csname ver@#1.sty\endcsname\relax
      \input #1.sty\relax
    \fi
  }%
  \TMP@RequirePackage{ltxcmds}[2010/03/01]%
\else
  \RequirePackage{ltxcmds}[2010/03/01]%
\fi
%    \end{macrocode}
%
%
% \subsection{Provide key defining macro}
%
%    \begin{macro}{\kv@define@key}
%    \begin{macrocode}
\ltx@IfUndefined{protected}{%
  \ltx@IfUndefined{DeclareRobustCommand}{%
    \def\kv@define@key#1#2%
  }{%
    \DeclareRobustCommand*{\kv@define@key}[2]%
  }%
}{%
  \protected\def\kv@define@key#1#2%
}%
{%
  \begingroup
    \csname @safe@activestrue\endcsname
    \let\ifincsname\iftrue
    \edef\KVD@temp{\endgroup
      \noexpand\KVD@DefineKey{#1}{#2}%
    }%
  \KVD@temp
}
%    \end{macrocode}
%    \end{macro}
%    \begin{macro}{\KVD@DefineKey}
%    \begin{macrocode}
\def\KVD@DefineKey#1#2{%
  \ltx@ifnextchar[{%
    \KVD@DefineKeyWithDefault{#1}{#2}%
  }{%
    \long\expandafter\def\csname KV@#1@#2\endcsname##1%
  }%
}
%    \end{macrocode}
%    \end{macro}
%    \begin{macro}{\KVD@DefineKeyWithDefault}
%    \begin{macrocode}
\long\def\KVD@DefineKeyWithDefault#1#2[#3]{%
  \expandafter\def\csname KV@#1@#2@default\expandafter\endcsname
  \expandafter{%
    \csname KV@#1@#2\endcsname{#3}%
  }%
  \long\expandafter\def\csname KV@#1@#2\endcsname##1%
}
%    \end{macrocode}
%    \end{macro}
%
%    \begin{macrocode}
\KVD@AtEnd%
%</package>
%    \end{macrocode}
%
% \section{Test}
%
% \subsection{Catcode checks for loading}
%
%    \begin{macrocode}
%<*test1>
%    \end{macrocode}
%    \begin{macrocode}
\catcode`\{=1 %
\catcode`\}=2 %
\catcode`\#=6 %
\catcode`\@=11 %
\expandafter\ifx\csname count@\endcsname\relax
  \countdef\count@=255 %
\fi
\expandafter\ifx\csname @gobble\endcsname\relax
  \long\def\@gobble#1{}%
\fi
\expandafter\ifx\csname @firstofone\endcsname\relax
  \long\def\@firstofone#1{#1}%
\fi
\expandafter\ifx\csname loop\endcsname\relax
  \expandafter\@firstofone
\else
  \expandafter\@gobble
\fi
{%
  \def\loop#1\repeat{%
    \def\body{#1}%
    \iterate
  }%
  \def\iterate{%
    \body
      \let\next\iterate
    \else
      \let\next\relax
    \fi
    \next
  }%
  \let\repeat=\fi
}%
\def\RestoreCatcodes{}
\count@=0 %
\loop
  \edef\RestoreCatcodes{%
    \RestoreCatcodes
    \catcode\the\count@=\the\catcode\count@\relax
  }%
\ifnum\count@<255 %
  \advance\count@ 1 %
\repeat

\def\RangeCatcodeInvalid#1#2{%
  \count@=#1\relax
  \loop
    \catcode\count@=15 %
  \ifnum\count@<#2\relax
    \advance\count@ 1 %
  \repeat
}
\def\RangeCatcodeCheck#1#2#3{%
  \count@=#1\relax
  \loop
    \ifnum#3=\catcode\count@
    \else
      \errmessage{%
        Character \the\count@\space
        with wrong catcode \the\catcode\count@\space
        instead of \number#3%
      }%
    \fi
  \ifnum\count@<#2\relax
    \advance\count@ 1 %
  \repeat
}
\def\space{ }
\expandafter\ifx\csname LoadCommand\endcsname\relax
  \def\LoadCommand{\input kvdefinekeys.sty\relax}%
\fi
\def\Test{%
  \RangeCatcodeInvalid{0}{47}%
  \RangeCatcodeInvalid{58}{64}%
  \RangeCatcodeInvalid{91}{96}%
  \RangeCatcodeInvalid{123}{255}%
  \catcode`\@=12 %
  \catcode`\\=0 %
  \catcode`\%=14 %
  \LoadCommand
  \RangeCatcodeCheck{0}{36}{15}%
  \RangeCatcodeCheck{37}{37}{14}%
  \RangeCatcodeCheck{38}{47}{15}%
  \RangeCatcodeCheck{48}{57}{12}%
  \RangeCatcodeCheck{58}{63}{15}%
  \RangeCatcodeCheck{64}{64}{12}%
  \RangeCatcodeCheck{65}{90}{11}%
  \RangeCatcodeCheck{91}{91}{15}%
  \RangeCatcodeCheck{92}{92}{0}%
  \RangeCatcodeCheck{93}{96}{15}%
  \RangeCatcodeCheck{97}{122}{11}%
  \RangeCatcodeCheck{123}{255}{15}%
  \RestoreCatcodes
}
\Test
\csname @@end\endcsname
\end
%    \end{macrocode}
%    \begin{macrocode}
%</test1>
%    \end{macrocode}
%
% \section{Installation}
%
% \subsection{Download}
%
% \paragraph{Package.} This package is available on
% CTAN\footnote{\url{http://ctan.org/pkg/kvdefinekeys}}:
% \begin{description}
% \item[\CTAN{macros/latex/contrib/oberdiek/kvdefinekeys.dtx}] The source file.
% \item[\CTAN{macros/latex/contrib/oberdiek/kvdefinekeys.pdf}] Documentation.
% \end{description}
%
%
% \paragraph{Bundle.} All the packages of the bundle `oberdiek'
% are also available in a TDS compliant ZIP archive. There
% the packages are already unpacked and the documentation files
% are generated. The files and directories obey the TDS standard.
% \begin{description}
% \item[\CTAN{install/macros/latex/contrib/oberdiek.tds.zip}]
% \end{description}
% \emph{TDS} refers to the standard ``A Directory Structure
% for \TeX\ Files'' (\CTAN{tds/tds.pdf}). Directories
% with \xfile{texmf} in their name are usually organized this way.
%
% \subsection{Bundle installation}
%
% \paragraph{Unpacking.} Unpack the \xfile{oberdiek.tds.zip} in the
% TDS tree (also known as \xfile{texmf} tree) of your choice.
% Example (linux):
% \begin{quote}
%   |unzip oberdiek.tds.zip -d ~/texmf|
% \end{quote}
%
% \paragraph{Script installation.}
% Check the directory \xfile{TDS:scripts/oberdiek/} for
% scripts that need further installation steps.
% Package \xpackage{attachfile2} comes with the Perl script
% \xfile{pdfatfi.pl} that should be installed in such a way
% that it can be called as \texttt{pdfatfi}.
% Example (linux):
% \begin{quote}
%   |chmod +x scripts/oberdiek/pdfatfi.pl|\\
%   |cp scripts/oberdiek/pdfatfi.pl /usr/local/bin/|
% \end{quote}
%
% \subsection{Package installation}
%
% \paragraph{Unpacking.} The \xfile{.dtx} file is a self-extracting
% \docstrip\ archive. The files are extracted by running the
% \xfile{.dtx} through \plainTeX:
% \begin{quote}
%   \verb|tex kvdefinekeys.dtx|
% \end{quote}
%
% \paragraph{TDS.} Now the different files must be moved into
% the different directories in your installation TDS tree
% (also known as \xfile{texmf} tree):
% \begin{quote}
% \def\t{^^A
% \begin{tabular}{@{}>{\ttfamily}l@{ $\rightarrow$ }>{\ttfamily}l@{}}
%   kvdefinekeys.sty & tex/generic/oberdiek/kvdefinekeys.sty\\
%   kvdefinekeys.pdf & doc/latex/oberdiek/kvdefinekeys.pdf\\
%   test/kvdefinekeys-test1.tex & doc/latex/oberdiek/test/kvdefinekeys-test1.tex\\
%   kvdefinekeys.dtx & source/latex/oberdiek/kvdefinekeys.dtx\\
% \end{tabular}^^A
% }^^A
% \sbox0{\t}^^A
% \ifdim\wd0>\linewidth
%   \begingroup
%     \advance\linewidth by\leftmargin
%     \advance\linewidth by\rightmargin
%   \edef\x{\endgroup
%     \def\noexpand\lw{\the\linewidth}^^A
%   }\x
%   \def\lwbox{^^A
%     \leavevmode
%     \hbox to \linewidth{^^A
%       \kern-\leftmargin\relax
%       \hss
%       \usebox0
%       \hss
%       \kern-\rightmargin\relax
%     }^^A
%   }^^A
%   \ifdim\wd0>\lw
%     \sbox0{\small\t}^^A
%     \ifdim\wd0>\linewidth
%       \ifdim\wd0>\lw
%         \sbox0{\footnotesize\t}^^A
%         \ifdim\wd0>\linewidth
%           \ifdim\wd0>\lw
%             \sbox0{\scriptsize\t}^^A
%             \ifdim\wd0>\linewidth
%               \ifdim\wd0>\lw
%                 \sbox0{\tiny\t}^^A
%                 \ifdim\wd0>\linewidth
%                   \lwbox
%                 \else
%                   \usebox0
%                 \fi
%               \else
%                 \lwbox
%               \fi
%             \else
%               \usebox0
%             \fi
%           \else
%             \lwbox
%           \fi
%         \else
%           \usebox0
%         \fi
%       \else
%         \lwbox
%       \fi
%     \else
%       \usebox0
%     \fi
%   \else
%     \lwbox
%   \fi
% \else
%   \usebox0
% \fi
% \end{quote}
% If you have a \xfile{docstrip.cfg} that configures and enables \docstrip's
% TDS installing feature, then some files can already be in the right
% place, see the documentation of \docstrip.
%
% \subsection{Refresh file name databases}
%
% If your \TeX~distribution
% (\teTeX, \mikTeX, \dots) relies on file name databases, you must refresh
% these. For example, \teTeX\ users run \verb|texhash| or
% \verb|mktexlsr|.
%
% \subsection{Some details for the interested}
%
% \paragraph{Attached source.}
%
% The PDF documentation on CTAN also includes the
% \xfile{.dtx} source file. It can be extracted by
% AcrobatReader 6 or higher. Another option is \textsf{pdftk},
% e.g. unpack the file into the current directory:
% \begin{quote}
%   \verb|pdftk kvdefinekeys.pdf unpack_files output .|
% \end{quote}
%
% \paragraph{Unpacking with \LaTeX.}
% The \xfile{.dtx} chooses its action depending on the format:
% \begin{description}
% \item[\plainTeX:] Run \docstrip\ and extract the files.
% \item[\LaTeX:] Generate the documentation.
% \end{description}
% If you insist on using \LaTeX\ for \docstrip\ (really,
% \docstrip\ does not need \LaTeX), then inform the autodetect routine
% about your intention:
% \begin{quote}
%   \verb|latex \let\install=y\input{kvdefinekeys.dtx}|
% \end{quote}
% Do not forget to quote the argument according to the demands
% of your shell.
%
% \paragraph{Generating the documentation.}
% You can use both the \xfile{.dtx} or the \xfile{.drv} to generate
% the documentation. The process can be configured by the
% configuration file \xfile{ltxdoc.cfg}. For instance, put this
% line into this file, if you want to have A4 as paper format:
% \begin{quote}
%   \verb|\PassOptionsToClass{a4paper}{article}|
% \end{quote}
% An example follows how to generate the
% documentation with pdf\LaTeX:
% \begin{quote}
%\begin{verbatim}
%pdflatex kvdefinekeys.dtx
%makeindex -s gind.ist kvdefinekeys.idx
%pdflatex kvdefinekeys.dtx
%makeindex -s gind.ist kvdefinekeys.idx
%pdflatex kvdefinekeys.dtx
%\end{verbatim}
% \end{quote}
%
% \section{Catalogue}
%
% The following XML file can be used as source for the
% \href{http://mirror.ctan.org/help/Catalogue/catalogue.html}{\TeX\ Catalogue}.
% The elements \texttt{caption} and \texttt{description} are imported
% from the original XML file from the Catalogue.
% The name of the XML file in the Catalogue is \xfile{kvdefinekeys.xml}.
%    \begin{macrocode}
%<*catalogue>
<?xml version='1.0' encoding='us-ascii'?>
<!DOCTYPE entry SYSTEM 'catalogue.dtd'>
<entry datestamp='$Date$' modifier='$Author$' id='kvdefinekeys'>
  <name>kvdefinekeys</name>
  <caption>Define keys for use in the kvsetkeys package.</caption>
  <authorref id='auth:oberdiek'/>
  <copyright owner='Heiko Oberdiek' year='2010,2011'/>
  <license type='lppl1.3'/>
  <version number='1.4'/>
  <description>
    The package provides a macro <tt>\kv@define@key</tt> (analogous to
    <xref refid='keyval'>keyval&#x2019;s</xref> <tt>\define@key</tt>, to
    define keys for use by <xref refid='kvsetkeys'>kvsetkeys</xref>.
    <p/>
    The package is part of the <xref refid='oberdiek'>oberdiek</xref>
    bundle.
  </description>
  <documentation details='Package documentation'
      href='ctan:/macros/latex/contrib/oberdiek/kvdefinekeys.pdf'/>
  <ctan file='true' path='/macros/latex/contrib/oberdiek/kvdefinekeys.dtx'/>
  <miktex location='oberdiek'/>
  <texlive location='oberdiek'/>
  <install path='/macros/latex/contrib/oberdiek/oberdiek.tds.zip'/>
</entry>
%</catalogue>
%    \end{macrocode}
%
% \begin{thebibliography}{9}
% \bibitem{keyval}
%   David Carlisle:
%   \textit{The \xpackage{keyval} package};
%   1999/03/16 v1.13;
%   \CTAN{macros/latex/required/graphics/keyval.dtx}.
%
% \end{thebibliography}
%
% \begin{History}
%   \begin{Version}{2010/03/01 v1.0}
%   \item
%     First version.
%   \end{Version}
%   \begin{Version}{2010/08/19 v1.1}
%   \item
%     Documentation fix, no code change.
%   \end{Version}
%   \begin{Version}{2011/01/30 v1.2}
%   \item
%     Already loaded package files are not input in \hologo{plainTeX}.
%   \end{Version}
%   \begin{Version}{2011/04/07 v1.3}
%   \item
%     Support for package \xpackage{babel}'s shorthands added.
%   \item
%     \cs{kv@define@key} is made robust if available.
%   \end{Version}
%   \begin{Version}{2016/05/16 v1.4}
%   \item
%     Documentation updates.
%   \end{Version}
% \end{History}
%
% \PrintIndex
%
% \Finale
\endinput

%        (quote the arguments according to the demands of your shell)
%
% Documentation:
%    (a) If kvdefinekeys.drv is present:
%           latex kvdefinekeys.drv
%    (b) Without kvdefinekeys.drv:
%           latex kvdefinekeys.dtx; ...
%    The class ltxdoc loads the configuration file ltxdoc.cfg
%    if available. Here you can specify further options, e.g.
%    use A4 as paper format:
%       \PassOptionsToClass{a4paper}{article}
%
%    Programm calls to get the documentation (example):
%       pdflatex kvdefinekeys.dtx
%       makeindex -s gind.ist kvdefinekeys.idx
%       pdflatex kvdefinekeys.dtx
%       makeindex -s gind.ist kvdefinekeys.idx
%       pdflatex kvdefinekeys.dtx
%
% Installation:
%    TDS:tex/generic/oberdiek/kvdefinekeys.sty
%    TDS:doc/latex/oberdiek/kvdefinekeys.pdf
%    TDS:doc/latex/oberdiek/test/kvdefinekeys-test1.tex
%    TDS:source/latex/oberdiek/kvdefinekeys.dtx
%
%<*ignore>
\begingroup
  \catcode123=1 %
  \catcode125=2 %
  \def\x{LaTeX2e}%
\expandafter\endgroup
\ifcase 0\ifx\install y1\fi\expandafter
         \ifx\csname processbatchFile\endcsname\relax\else1\fi
         \ifx\fmtname\x\else 1\fi\relax
\else\csname fi\endcsname
%</ignore>
%<*install>
\input docstrip.tex
\Msg{************************************************************************}
\Msg{* Installation}
\Msg{* Package: kvdefinekeys 2016/05/16 v1.4 Define keys (HO)}
\Msg{************************************************************************}

\keepsilent
\askforoverwritefalse

\let\MetaPrefix\relax
\preamble

This is a generated file.

Project: kvdefinekeys
Version: 2016/05/16 v1.4

Copyright (C) 2010, 2011 by
   Heiko Oberdiek <heiko.oberdiek at googlemail.com>

This work may be distributed and/or modified under the
conditions of the LaTeX Project Public License, either
version 1.3c of this license or (at your option) any later
version. This version of this license is in
   http://www.latex-project.org/lppl/lppl-1-3c.txt
and the latest version of this license is in
   http://www.latex-project.org/lppl.txt
and version 1.3 or later is part of all distributions of
LaTeX version 2005/12/01 or later.

This work has the LPPL maintenance status "maintained".

This Current Maintainer of this work is Heiko Oberdiek.

The Base Interpreter refers to any `TeX-Format',
because some files are installed in TDS:tex/generic//.

This work consists of the main source file kvdefinekeys.dtx
and the derived files
   kvdefinekeys.sty, kvdefinekeys.pdf, kvdefinekeys.ins, kvdefinekeys.drv,
   kvdefinekeys-test1.tex.

\endpreamble
\let\MetaPrefix\DoubleperCent

\generate{%
  \file{kvdefinekeys.ins}{\from{kvdefinekeys.dtx}{install}}%
  \file{kvdefinekeys.drv}{\from{kvdefinekeys.dtx}{driver}}%
  \usedir{tex/generic/oberdiek}%
  \file{kvdefinekeys.sty}{\from{kvdefinekeys.dtx}{package}}%
  \usedir{doc/latex/oberdiek/test}%
  \file{kvdefinekeys-test1.tex}{\from{kvdefinekeys.dtx}{test1}}%
  \nopreamble
  \nopostamble
  \usedir{source/latex/oberdiek/catalogue}%
  \file{kvdefinekeys.xml}{\from{kvdefinekeys.dtx}{catalogue}}%
}

\catcode32=13\relax% active space
\let =\space%
\Msg{************************************************************************}
\Msg{*}
\Msg{* To finish the installation you have to move the following}
\Msg{* file into a directory searched by TeX:}
\Msg{*}
\Msg{*     kvdefinekeys.sty}
\Msg{*}
\Msg{* To produce the documentation run the file `kvdefinekeys.drv'}
\Msg{* through LaTeX.}
\Msg{*}
\Msg{* Happy TeXing!}
\Msg{*}
\Msg{************************************************************************}

\endbatchfile
%</install>
%<*ignore>
\fi
%</ignore>
%<*driver>
\NeedsTeXFormat{LaTeX2e}
\ProvidesFile{kvdefinekeys.drv}%
  [2016/05/16 v1.4 Define keys (HO)]%
\documentclass{ltxdoc}
\usepackage{holtxdoc}[2011/11/22]
\begin{document}
  \DocInput{kvdefinekeys.dtx}%
\end{document}
%</driver>
% \fi
%
%
% \CharacterTable
%  {Upper-case    \A\B\C\D\E\F\G\H\I\J\K\L\M\N\O\P\Q\R\S\T\U\V\W\X\Y\Z
%   Lower-case    \a\b\c\d\e\f\g\h\i\j\k\l\m\n\o\p\q\r\s\t\u\v\w\x\y\z
%   Digits        \0\1\2\3\4\5\6\7\8\9
%   Exclamation   \!     Double quote  \"     Hash (number) \#
%   Dollar        \$     Percent       \%     Ampersand     \&
%   Acute accent  \'     Left paren    \(     Right paren   \)
%   Asterisk      \*     Plus          \+     Comma         \,
%   Minus         \-     Point         \.     Solidus       \/
%   Colon         \:     Semicolon     \;     Less than     \<
%   Equals        \=     Greater than  \>     Question mark \?
%   Commercial at \@     Left bracket  \[     Backslash     \\
%   Right bracket \]     Circumflex    \^     Underscore    \_
%   Grave accent  \`     Left brace    \{     Vertical bar  \|
%   Right brace   \}     Tilde         \~}
%
% \GetFileInfo{kvdefinekeys.drv}
%
% \title{The \xpackage{kvdefinekeys} package}
% \date{2016/05/16 v1.4}
% \author{Heiko Oberdiek\thanks
% {Please report any issues at https://github.com/ho-tex/oberdiek/issues}\\
% \xemail{heiko.oberdiek at googlemail.com}}
%
% \maketitle
%
% \begin{abstract}
% Package \xpackage{kvdefinekeys} provides \cs{kv@define@key} to define
% keys the same way as \xpackage{keyval}'s \cs{define@key}. However, it
% works also using \iniTeX.
% \end{abstract}
%
% \tableofcontents
%
% \def\M#1{\texttt{\{}\meta{#1}\texttt{\}}}
%
% \section{Documentation}
%
% \subsection{Motivation}
%
% \cs{kvsetkeys} serves as replacement for \xpackage{keyval}'s
% \cs{setkeys}. This package adds macros to define keys, closing
% the gap \cs{kvsetkeys} leaves.
%
% \begin{declcs}{kv@define@key}\,\M{family}\,\M{key}\,[\meta{default}]\,^^A
%   \M{definition}
% \end{declcs}
% Macro \cs{kv@define@key} reimplements \xpackage{keyval}'s
% \cs{define@key}. Differences to the original:
% \begin{itemize}
% \item The defined keys also allow \cs{par} inside values.
% \item Shorthands of package \xpackage{babel} are supported in
%   family and key names.
% \item Macro \cs{kv@define@key} is made robust if
%   \hologo{eTeX}'s \cs{protected} or \hologo{LaTeX}'s
%   \cs{DeclareRobustCommand} are found.
% \end{itemize}
%
% \StopEventually{
% }
%
% \section{Implementation}
%
% \subsection{Identification}
%
%    \begin{macrocode}
%<*package>
%    \end{macrocode}
%    Reload check, especially if the package is not used with \LaTeX.
%    \begin{macrocode}
\begingroup\catcode61\catcode48\catcode32=10\relax%
  \catcode13=5 % ^^M
  \endlinechar=13 %
  \catcode35=6 % #
  \catcode39=12 % '
  \catcode44=12 % ,
  \catcode45=12 % -
  \catcode46=12 % .
  \catcode58=12 % :
  \catcode64=11 % @
  \catcode123=1 % {
  \catcode125=2 % }
  \expandafter\let\expandafter\x\csname ver@kvdefinekeys.sty\endcsname
  \ifx\x\relax % plain-TeX, first loading
  \else
    \def\empty{}%
    \ifx\x\empty % LaTeX, first loading,
      % variable is initialized, but \ProvidesPackage not yet seen
    \else
      \expandafter\ifx\csname PackageInfo\endcsname\relax
        \def\x#1#2{%
          \immediate\write-1{Package #1 Info: #2.}%
        }%
      \else
        \def\x#1#2{\PackageInfo{#1}{#2, stopped}}%
      \fi
      \x{kvdefinekeys}{The package is already loaded}%
      \aftergroup\endinput
    \fi
  \fi
\endgroup%
%    \end{macrocode}
%    Package identification:
%    \begin{macrocode}
\begingroup\catcode61\catcode48\catcode32=10\relax%
  \catcode13=5 % ^^M
  \endlinechar=13 %
  \catcode35=6 % #
  \catcode39=12 % '
  \catcode40=12 % (
  \catcode41=12 % )
  \catcode44=12 % ,
  \catcode45=12 % -
  \catcode46=12 % .
  \catcode47=12 % /
  \catcode58=12 % :
  \catcode64=11 % @
  \catcode91=12 % [
  \catcode93=12 % ]
  \catcode123=1 % {
  \catcode125=2 % }
  \expandafter\ifx\csname ProvidesPackage\endcsname\relax
    \def\x#1#2#3[#4]{\endgroup
      \immediate\write-1{Package: #3 #4}%
      \xdef#1{#4}%
    }%
  \else
    \def\x#1#2[#3]{\endgroup
      #2[{#3}]%
      \ifx#1\@undefined
        \xdef#1{#3}%
      \fi
      \ifx#1\relax
        \xdef#1{#3}%
      \fi
    }%
  \fi
\expandafter\x\csname ver@kvdefinekeys.sty\endcsname
\ProvidesPackage{kvdefinekeys}%
  [2016/05/16 v1.4 Define keys (HO)]%
%    \end{macrocode}
%
%    \begin{macrocode}
\begingroup\catcode61\catcode48\catcode32=10\relax%
  \catcode13=5 % ^^M
  \endlinechar=13 %
  \catcode123=1 % {
  \catcode125=2 % }
  \catcode64=11 % @
  \def\x{\endgroup
    \expandafter\edef\csname KVD@AtEnd\endcsname{%
      \endlinechar=\the\endlinechar\relax
      \catcode13=\the\catcode13\relax
      \catcode32=\the\catcode32\relax
      \catcode35=\the\catcode35\relax
      \catcode61=\the\catcode61\relax
      \catcode64=\the\catcode64\relax
      \catcode123=\the\catcode123\relax
      \catcode125=\the\catcode125\relax
    }%
  }%
\x\catcode61\catcode48\catcode32=10\relax%
\catcode13=5 % ^^M
\endlinechar=13 %
\catcode35=6 % #
\catcode64=11 % @
\catcode123=1 % {
\catcode125=2 % }
\def\TMP@EnsureCode#1#2{%
  \edef\KVD@AtEnd{%
    \KVD@AtEnd
    \catcode#1=\the\catcode#1\relax
  }%
  \catcode#1=#2\relax
}
\TMP@EnsureCode{42}{12}% *
\TMP@EnsureCode{46}{12}% .
\TMP@EnsureCode{47}{12}% /
\TMP@EnsureCode{91}{12}% [
\TMP@EnsureCode{93}{12}% ]
\edef\KVD@AtEnd{\KVD@AtEnd\noexpand\endinput}
%    \end{macrocode}
%
% \subsection{Package loading}
%
%    \begin{macrocode}
\begingroup\expandafter\expandafter\expandafter\endgroup
\expandafter\ifx\csname RequirePackage\endcsname\relax
  \def\TMP@RequirePackage#1[#2]{%
    \begingroup\expandafter\expandafter\expandafter\endgroup
    \expandafter\ifx\csname ver@#1.sty\endcsname\relax
      \input #1.sty\relax
    \fi
  }%
  \TMP@RequirePackage{ltxcmds}[2010/03/01]%
\else
  \RequirePackage{ltxcmds}[2010/03/01]%
\fi
%    \end{macrocode}
%
%
% \subsection{Provide key defining macro}
%
%    \begin{macro}{\kv@define@key}
%    \begin{macrocode}
\ltx@IfUndefined{protected}{%
  \ltx@IfUndefined{DeclareRobustCommand}{%
    \def\kv@define@key#1#2%
  }{%
    \DeclareRobustCommand*{\kv@define@key}[2]%
  }%
}{%
  \protected\def\kv@define@key#1#2%
}%
{%
  \begingroup
    \csname @safe@activestrue\endcsname
    \let\ifincsname\iftrue
    \edef\KVD@temp{\endgroup
      \noexpand\KVD@DefineKey{#1}{#2}%
    }%
  \KVD@temp
}
%    \end{macrocode}
%    \end{macro}
%    \begin{macro}{\KVD@DefineKey}
%    \begin{macrocode}
\def\KVD@DefineKey#1#2{%
  \ltx@ifnextchar[{%
    \KVD@DefineKeyWithDefault{#1}{#2}%
  }{%
    \long\expandafter\def\csname KV@#1@#2\endcsname##1%
  }%
}
%    \end{macrocode}
%    \end{macro}
%    \begin{macro}{\KVD@DefineKeyWithDefault}
%    \begin{macrocode}
\long\def\KVD@DefineKeyWithDefault#1#2[#3]{%
  \expandafter\def\csname KV@#1@#2@default\expandafter\endcsname
  \expandafter{%
    \csname KV@#1@#2\endcsname{#3}%
  }%
  \long\expandafter\def\csname KV@#1@#2\endcsname##1%
}
%    \end{macrocode}
%    \end{macro}
%
%    \begin{macrocode}
\KVD@AtEnd%
%</package>
%    \end{macrocode}
%
% \section{Test}
%
% \subsection{Catcode checks for loading}
%
%    \begin{macrocode}
%<*test1>
%    \end{macrocode}
%    \begin{macrocode}
\catcode`\{=1 %
\catcode`\}=2 %
\catcode`\#=6 %
\catcode`\@=11 %
\expandafter\ifx\csname count@\endcsname\relax
  \countdef\count@=255 %
\fi
\expandafter\ifx\csname @gobble\endcsname\relax
  \long\def\@gobble#1{}%
\fi
\expandafter\ifx\csname @firstofone\endcsname\relax
  \long\def\@firstofone#1{#1}%
\fi
\expandafter\ifx\csname loop\endcsname\relax
  \expandafter\@firstofone
\else
  \expandafter\@gobble
\fi
{%
  \def\loop#1\repeat{%
    \def\body{#1}%
    \iterate
  }%
  \def\iterate{%
    \body
      \let\next\iterate
    \else
      \let\next\relax
    \fi
    \next
  }%
  \let\repeat=\fi
}%
\def\RestoreCatcodes{}
\count@=0 %
\loop
  \edef\RestoreCatcodes{%
    \RestoreCatcodes
    \catcode\the\count@=\the\catcode\count@\relax
  }%
\ifnum\count@<255 %
  \advance\count@ 1 %
\repeat

\def\RangeCatcodeInvalid#1#2{%
  \count@=#1\relax
  \loop
    \catcode\count@=15 %
  \ifnum\count@<#2\relax
    \advance\count@ 1 %
  \repeat
}
\def\RangeCatcodeCheck#1#2#3{%
  \count@=#1\relax
  \loop
    \ifnum#3=\catcode\count@
    \else
      \errmessage{%
        Character \the\count@\space
        with wrong catcode \the\catcode\count@\space
        instead of \number#3%
      }%
    \fi
  \ifnum\count@<#2\relax
    \advance\count@ 1 %
  \repeat
}
\def\space{ }
\expandafter\ifx\csname LoadCommand\endcsname\relax
  \def\LoadCommand{\input kvdefinekeys.sty\relax}%
\fi
\def\Test{%
  \RangeCatcodeInvalid{0}{47}%
  \RangeCatcodeInvalid{58}{64}%
  \RangeCatcodeInvalid{91}{96}%
  \RangeCatcodeInvalid{123}{255}%
  \catcode`\@=12 %
  \catcode`\\=0 %
  \catcode`\%=14 %
  \LoadCommand
  \RangeCatcodeCheck{0}{36}{15}%
  \RangeCatcodeCheck{37}{37}{14}%
  \RangeCatcodeCheck{38}{47}{15}%
  \RangeCatcodeCheck{48}{57}{12}%
  \RangeCatcodeCheck{58}{63}{15}%
  \RangeCatcodeCheck{64}{64}{12}%
  \RangeCatcodeCheck{65}{90}{11}%
  \RangeCatcodeCheck{91}{91}{15}%
  \RangeCatcodeCheck{92}{92}{0}%
  \RangeCatcodeCheck{93}{96}{15}%
  \RangeCatcodeCheck{97}{122}{11}%
  \RangeCatcodeCheck{123}{255}{15}%
  \RestoreCatcodes
}
\Test
\csname @@end\endcsname
\end
%    \end{macrocode}
%    \begin{macrocode}
%</test1>
%    \end{macrocode}
%
% \section{Installation}
%
% \subsection{Download}
%
% \paragraph{Package.} This package is available on
% CTAN\footnote{\url{http://ctan.org/pkg/kvdefinekeys}}:
% \begin{description}
% \item[\CTAN{macros/latex/contrib/oberdiek/kvdefinekeys.dtx}] The source file.
% \item[\CTAN{macros/latex/contrib/oberdiek/kvdefinekeys.pdf}] Documentation.
% \end{description}
%
%
% \paragraph{Bundle.} All the packages of the bundle `oberdiek'
% are also available in a TDS compliant ZIP archive. There
% the packages are already unpacked and the documentation files
% are generated. The files and directories obey the TDS standard.
% \begin{description}
% \item[\CTAN{install/macros/latex/contrib/oberdiek.tds.zip}]
% \end{description}
% \emph{TDS} refers to the standard ``A Directory Structure
% for \TeX\ Files'' (\CTAN{tds/tds.pdf}). Directories
% with \xfile{texmf} in their name are usually organized this way.
%
% \subsection{Bundle installation}
%
% \paragraph{Unpacking.} Unpack the \xfile{oberdiek.tds.zip} in the
% TDS tree (also known as \xfile{texmf} tree) of your choice.
% Example (linux):
% \begin{quote}
%   |unzip oberdiek.tds.zip -d ~/texmf|
% \end{quote}
%
% \paragraph{Script installation.}
% Check the directory \xfile{TDS:scripts/oberdiek/} for
% scripts that need further installation steps.
% Package \xpackage{attachfile2} comes with the Perl script
% \xfile{pdfatfi.pl} that should be installed in such a way
% that it can be called as \texttt{pdfatfi}.
% Example (linux):
% \begin{quote}
%   |chmod +x scripts/oberdiek/pdfatfi.pl|\\
%   |cp scripts/oberdiek/pdfatfi.pl /usr/local/bin/|
% \end{quote}
%
% \subsection{Package installation}
%
% \paragraph{Unpacking.} The \xfile{.dtx} file is a self-extracting
% \docstrip\ archive. The files are extracted by running the
% \xfile{.dtx} through \plainTeX:
% \begin{quote}
%   \verb|tex kvdefinekeys.dtx|
% \end{quote}
%
% \paragraph{TDS.} Now the different files must be moved into
% the different directories in your installation TDS tree
% (also known as \xfile{texmf} tree):
% \begin{quote}
% \def\t{^^A
% \begin{tabular}{@{}>{\ttfamily}l@{ $\rightarrow$ }>{\ttfamily}l@{}}
%   kvdefinekeys.sty & tex/generic/oberdiek/kvdefinekeys.sty\\
%   kvdefinekeys.pdf & doc/latex/oberdiek/kvdefinekeys.pdf\\
%   test/kvdefinekeys-test1.tex & doc/latex/oberdiek/test/kvdefinekeys-test1.tex\\
%   kvdefinekeys.dtx & source/latex/oberdiek/kvdefinekeys.dtx\\
% \end{tabular}^^A
% }^^A
% \sbox0{\t}^^A
% \ifdim\wd0>\linewidth
%   \begingroup
%     \advance\linewidth by\leftmargin
%     \advance\linewidth by\rightmargin
%   \edef\x{\endgroup
%     \def\noexpand\lw{\the\linewidth}^^A
%   }\x
%   \def\lwbox{^^A
%     \leavevmode
%     \hbox to \linewidth{^^A
%       \kern-\leftmargin\relax
%       \hss
%       \usebox0
%       \hss
%       \kern-\rightmargin\relax
%     }^^A
%   }^^A
%   \ifdim\wd0>\lw
%     \sbox0{\small\t}^^A
%     \ifdim\wd0>\linewidth
%       \ifdim\wd0>\lw
%         \sbox0{\footnotesize\t}^^A
%         \ifdim\wd0>\linewidth
%           \ifdim\wd0>\lw
%             \sbox0{\scriptsize\t}^^A
%             \ifdim\wd0>\linewidth
%               \ifdim\wd0>\lw
%                 \sbox0{\tiny\t}^^A
%                 \ifdim\wd0>\linewidth
%                   \lwbox
%                 \else
%                   \usebox0
%                 \fi
%               \else
%                 \lwbox
%               \fi
%             \else
%               \usebox0
%             \fi
%           \else
%             \lwbox
%           \fi
%         \else
%           \usebox0
%         \fi
%       \else
%         \lwbox
%       \fi
%     \else
%       \usebox0
%     \fi
%   \else
%     \lwbox
%   \fi
% \else
%   \usebox0
% \fi
% \end{quote}
% If you have a \xfile{docstrip.cfg} that configures and enables \docstrip's
% TDS installing feature, then some files can already be in the right
% place, see the documentation of \docstrip.
%
% \subsection{Refresh file name databases}
%
% If your \TeX~distribution
% (\teTeX, \mikTeX, \dots) relies on file name databases, you must refresh
% these. For example, \teTeX\ users run \verb|texhash| or
% \verb|mktexlsr|.
%
% \subsection{Some details for the interested}
%
% \paragraph{Attached source.}
%
% The PDF documentation on CTAN also includes the
% \xfile{.dtx} source file. It can be extracted by
% AcrobatReader 6 or higher. Another option is \textsf{pdftk},
% e.g. unpack the file into the current directory:
% \begin{quote}
%   \verb|pdftk kvdefinekeys.pdf unpack_files output .|
% \end{quote}
%
% \paragraph{Unpacking with \LaTeX.}
% The \xfile{.dtx} chooses its action depending on the format:
% \begin{description}
% \item[\plainTeX:] Run \docstrip\ and extract the files.
% \item[\LaTeX:] Generate the documentation.
% \end{description}
% If you insist on using \LaTeX\ for \docstrip\ (really,
% \docstrip\ does not need \LaTeX), then inform the autodetect routine
% about your intention:
% \begin{quote}
%   \verb|latex \let\install=y% \iffalse meta-comment
%
% File: kvdefinekeys.dtx
% Version: 2016/05/16 v1.4
% Info: Define keys
%
% Copyright (C) 2010, 2011 by
%    Heiko Oberdiek <heiko.oberdiek at googlemail.com>
%    2016
%    https://github.com/ho-tex/oberdiek/issues
%
% This work may be distributed and/or modified under the
% conditions of the LaTeX Project Public License, either
% version 1.3c of this license or (at your option) any later
% version. This version of this license is in
%    http://www.latex-project.org/lppl/lppl-1-3c.txt
% and the latest version of this license is in
%    http://www.latex-project.org/lppl.txt
% and version 1.3 or later is part of all distributions of
% LaTeX version 2005/12/01 or later.
%
% This work has the LPPL maintenance status "maintained".
%
% This Current Maintainer of this work is Heiko Oberdiek.
%
% The Base Interpreter refers to any `TeX-Format',
% because some files are installed in TDS:tex/generic//.
%
% This work consists of the main source file kvdefinekeys.dtx
% and the derived files
%    kvdefinekeys.sty, kvdefinekeys.pdf, kvdefinekeys.ins, kvdefinekeys.drv,
%    kvdefinekeys-test1.tex.
%
% Distribution:
%    CTAN:macros/latex/contrib/oberdiek/kvdefinekeys.dtx
%    CTAN:macros/latex/contrib/oberdiek/kvdefinekeys.pdf
%
% Unpacking:
%    (a) If kvdefinekeys.ins is present:
%           tex kvdefinekeys.ins
%    (b) Without kvdefinekeys.ins:
%           tex kvdefinekeys.dtx
%    (c) If you insist on using LaTeX
%           latex \let\install=y\input{kvdefinekeys.dtx}
%        (quote the arguments according to the demands of your shell)
%
% Documentation:
%    (a) If kvdefinekeys.drv is present:
%           latex kvdefinekeys.drv
%    (b) Without kvdefinekeys.drv:
%           latex kvdefinekeys.dtx; ...
%    The class ltxdoc loads the configuration file ltxdoc.cfg
%    if available. Here you can specify further options, e.g.
%    use A4 as paper format:
%       \PassOptionsToClass{a4paper}{article}
%
%    Programm calls to get the documentation (example):
%       pdflatex kvdefinekeys.dtx
%       makeindex -s gind.ist kvdefinekeys.idx
%       pdflatex kvdefinekeys.dtx
%       makeindex -s gind.ist kvdefinekeys.idx
%       pdflatex kvdefinekeys.dtx
%
% Installation:
%    TDS:tex/generic/oberdiek/kvdefinekeys.sty
%    TDS:doc/latex/oberdiek/kvdefinekeys.pdf
%    TDS:doc/latex/oberdiek/test/kvdefinekeys-test1.tex
%    TDS:source/latex/oberdiek/kvdefinekeys.dtx
%
%<*ignore>
\begingroup
  \catcode123=1 %
  \catcode125=2 %
  \def\x{LaTeX2e}%
\expandafter\endgroup
\ifcase 0\ifx\install y1\fi\expandafter
         \ifx\csname processbatchFile\endcsname\relax\else1\fi
         \ifx\fmtname\x\else 1\fi\relax
\else\csname fi\endcsname
%</ignore>
%<*install>
\input docstrip.tex
\Msg{************************************************************************}
\Msg{* Installation}
\Msg{* Package: kvdefinekeys 2016/05/16 v1.4 Define keys (HO)}
\Msg{************************************************************************}

\keepsilent
\askforoverwritefalse

\let\MetaPrefix\relax
\preamble

This is a generated file.

Project: kvdefinekeys
Version: 2016/05/16 v1.4

Copyright (C) 2010, 2011 by
   Heiko Oberdiek <heiko.oberdiek at googlemail.com>

This work may be distributed and/or modified under the
conditions of the LaTeX Project Public License, either
version 1.3c of this license or (at your option) any later
version. This version of this license is in
   http://www.latex-project.org/lppl/lppl-1-3c.txt
and the latest version of this license is in
   http://www.latex-project.org/lppl.txt
and version 1.3 or later is part of all distributions of
LaTeX version 2005/12/01 or later.

This work has the LPPL maintenance status "maintained".

This Current Maintainer of this work is Heiko Oberdiek.

The Base Interpreter refers to any `TeX-Format',
because some files are installed in TDS:tex/generic//.

This work consists of the main source file kvdefinekeys.dtx
and the derived files
   kvdefinekeys.sty, kvdefinekeys.pdf, kvdefinekeys.ins, kvdefinekeys.drv,
   kvdefinekeys-test1.tex.

\endpreamble
\let\MetaPrefix\DoubleperCent

\generate{%
  \file{kvdefinekeys.ins}{\from{kvdefinekeys.dtx}{install}}%
  \file{kvdefinekeys.drv}{\from{kvdefinekeys.dtx}{driver}}%
  \usedir{tex/generic/oberdiek}%
  \file{kvdefinekeys.sty}{\from{kvdefinekeys.dtx}{package}}%
  \usedir{doc/latex/oberdiek/test}%
  \file{kvdefinekeys-test1.tex}{\from{kvdefinekeys.dtx}{test1}}%
  \nopreamble
  \nopostamble
  \usedir{source/latex/oberdiek/catalogue}%
  \file{kvdefinekeys.xml}{\from{kvdefinekeys.dtx}{catalogue}}%
}

\catcode32=13\relax% active space
\let =\space%
\Msg{************************************************************************}
\Msg{*}
\Msg{* To finish the installation you have to move the following}
\Msg{* file into a directory searched by TeX:}
\Msg{*}
\Msg{*     kvdefinekeys.sty}
\Msg{*}
\Msg{* To produce the documentation run the file `kvdefinekeys.drv'}
\Msg{* through LaTeX.}
\Msg{*}
\Msg{* Happy TeXing!}
\Msg{*}
\Msg{************************************************************************}

\endbatchfile
%</install>
%<*ignore>
\fi
%</ignore>
%<*driver>
\NeedsTeXFormat{LaTeX2e}
\ProvidesFile{kvdefinekeys.drv}%
  [2016/05/16 v1.4 Define keys (HO)]%
\documentclass{ltxdoc}
\usepackage{holtxdoc}[2011/11/22]
\begin{document}
  \DocInput{kvdefinekeys.dtx}%
\end{document}
%</driver>
% \fi
%
%
% \CharacterTable
%  {Upper-case    \A\B\C\D\E\F\G\H\I\J\K\L\M\N\O\P\Q\R\S\T\U\V\W\X\Y\Z
%   Lower-case    \a\b\c\d\e\f\g\h\i\j\k\l\m\n\o\p\q\r\s\t\u\v\w\x\y\z
%   Digits        \0\1\2\3\4\5\6\7\8\9
%   Exclamation   \!     Double quote  \"     Hash (number) \#
%   Dollar        \$     Percent       \%     Ampersand     \&
%   Acute accent  \'     Left paren    \(     Right paren   \)
%   Asterisk      \*     Plus          \+     Comma         \,
%   Minus         \-     Point         \.     Solidus       \/
%   Colon         \:     Semicolon     \;     Less than     \<
%   Equals        \=     Greater than  \>     Question mark \?
%   Commercial at \@     Left bracket  \[     Backslash     \\
%   Right bracket \]     Circumflex    \^     Underscore    \_
%   Grave accent  \`     Left brace    \{     Vertical bar  \|
%   Right brace   \}     Tilde         \~}
%
% \GetFileInfo{kvdefinekeys.drv}
%
% \title{The \xpackage{kvdefinekeys} package}
% \date{2016/05/16 v1.4}
% \author{Heiko Oberdiek\thanks
% {Please report any issues at https://github.com/ho-tex/oberdiek/issues}\\
% \xemail{heiko.oberdiek at googlemail.com}}
%
% \maketitle
%
% \begin{abstract}
% Package \xpackage{kvdefinekeys} provides \cs{kv@define@key} to define
% keys the same way as \xpackage{keyval}'s \cs{define@key}. However, it
% works also using \iniTeX.
% \end{abstract}
%
% \tableofcontents
%
% \def\M#1{\texttt{\{}\meta{#1}\texttt{\}}}
%
% \section{Documentation}
%
% \subsection{Motivation}
%
% \cs{kvsetkeys} serves as replacement for \xpackage{keyval}'s
% \cs{setkeys}. This package adds macros to define keys, closing
% the gap \cs{kvsetkeys} leaves.
%
% \begin{declcs}{kv@define@key}\,\M{family}\,\M{key}\,[\meta{default}]\,^^A
%   \M{definition}
% \end{declcs}
% Macro \cs{kv@define@key} reimplements \xpackage{keyval}'s
% \cs{define@key}. Differences to the original:
% \begin{itemize}
% \item The defined keys also allow \cs{par} inside values.
% \item Shorthands of package \xpackage{babel} are supported in
%   family and key names.
% \item Macro \cs{kv@define@key} is made robust if
%   \hologo{eTeX}'s \cs{protected} or \hologo{LaTeX}'s
%   \cs{DeclareRobustCommand} are found.
% \end{itemize}
%
% \StopEventually{
% }
%
% \section{Implementation}
%
% \subsection{Identification}
%
%    \begin{macrocode}
%<*package>
%    \end{macrocode}
%    Reload check, especially if the package is not used with \LaTeX.
%    \begin{macrocode}
\begingroup\catcode61\catcode48\catcode32=10\relax%
  \catcode13=5 % ^^M
  \endlinechar=13 %
  \catcode35=6 % #
  \catcode39=12 % '
  \catcode44=12 % ,
  \catcode45=12 % -
  \catcode46=12 % .
  \catcode58=12 % :
  \catcode64=11 % @
  \catcode123=1 % {
  \catcode125=2 % }
  \expandafter\let\expandafter\x\csname ver@kvdefinekeys.sty\endcsname
  \ifx\x\relax % plain-TeX, first loading
  \else
    \def\empty{}%
    \ifx\x\empty % LaTeX, first loading,
      % variable is initialized, but \ProvidesPackage not yet seen
    \else
      \expandafter\ifx\csname PackageInfo\endcsname\relax
        \def\x#1#2{%
          \immediate\write-1{Package #1 Info: #2.}%
        }%
      \else
        \def\x#1#2{\PackageInfo{#1}{#2, stopped}}%
      \fi
      \x{kvdefinekeys}{The package is already loaded}%
      \aftergroup\endinput
    \fi
  \fi
\endgroup%
%    \end{macrocode}
%    Package identification:
%    \begin{macrocode}
\begingroup\catcode61\catcode48\catcode32=10\relax%
  \catcode13=5 % ^^M
  \endlinechar=13 %
  \catcode35=6 % #
  \catcode39=12 % '
  \catcode40=12 % (
  \catcode41=12 % )
  \catcode44=12 % ,
  \catcode45=12 % -
  \catcode46=12 % .
  \catcode47=12 % /
  \catcode58=12 % :
  \catcode64=11 % @
  \catcode91=12 % [
  \catcode93=12 % ]
  \catcode123=1 % {
  \catcode125=2 % }
  \expandafter\ifx\csname ProvidesPackage\endcsname\relax
    \def\x#1#2#3[#4]{\endgroup
      \immediate\write-1{Package: #3 #4}%
      \xdef#1{#4}%
    }%
  \else
    \def\x#1#2[#3]{\endgroup
      #2[{#3}]%
      \ifx#1\@undefined
        \xdef#1{#3}%
      \fi
      \ifx#1\relax
        \xdef#1{#3}%
      \fi
    }%
  \fi
\expandafter\x\csname ver@kvdefinekeys.sty\endcsname
\ProvidesPackage{kvdefinekeys}%
  [2016/05/16 v1.4 Define keys (HO)]%
%    \end{macrocode}
%
%    \begin{macrocode}
\begingroup\catcode61\catcode48\catcode32=10\relax%
  \catcode13=5 % ^^M
  \endlinechar=13 %
  \catcode123=1 % {
  \catcode125=2 % }
  \catcode64=11 % @
  \def\x{\endgroup
    \expandafter\edef\csname KVD@AtEnd\endcsname{%
      \endlinechar=\the\endlinechar\relax
      \catcode13=\the\catcode13\relax
      \catcode32=\the\catcode32\relax
      \catcode35=\the\catcode35\relax
      \catcode61=\the\catcode61\relax
      \catcode64=\the\catcode64\relax
      \catcode123=\the\catcode123\relax
      \catcode125=\the\catcode125\relax
    }%
  }%
\x\catcode61\catcode48\catcode32=10\relax%
\catcode13=5 % ^^M
\endlinechar=13 %
\catcode35=6 % #
\catcode64=11 % @
\catcode123=1 % {
\catcode125=2 % }
\def\TMP@EnsureCode#1#2{%
  \edef\KVD@AtEnd{%
    \KVD@AtEnd
    \catcode#1=\the\catcode#1\relax
  }%
  \catcode#1=#2\relax
}
\TMP@EnsureCode{42}{12}% *
\TMP@EnsureCode{46}{12}% .
\TMP@EnsureCode{47}{12}% /
\TMP@EnsureCode{91}{12}% [
\TMP@EnsureCode{93}{12}% ]
\edef\KVD@AtEnd{\KVD@AtEnd\noexpand\endinput}
%    \end{macrocode}
%
% \subsection{Package loading}
%
%    \begin{macrocode}
\begingroup\expandafter\expandafter\expandafter\endgroup
\expandafter\ifx\csname RequirePackage\endcsname\relax
  \def\TMP@RequirePackage#1[#2]{%
    \begingroup\expandafter\expandafter\expandafter\endgroup
    \expandafter\ifx\csname ver@#1.sty\endcsname\relax
      \input #1.sty\relax
    \fi
  }%
  \TMP@RequirePackage{ltxcmds}[2010/03/01]%
\else
  \RequirePackage{ltxcmds}[2010/03/01]%
\fi
%    \end{macrocode}
%
%
% \subsection{Provide key defining macro}
%
%    \begin{macro}{\kv@define@key}
%    \begin{macrocode}
\ltx@IfUndefined{protected}{%
  \ltx@IfUndefined{DeclareRobustCommand}{%
    \def\kv@define@key#1#2%
  }{%
    \DeclareRobustCommand*{\kv@define@key}[2]%
  }%
}{%
  \protected\def\kv@define@key#1#2%
}%
{%
  \begingroup
    \csname @safe@activestrue\endcsname
    \let\ifincsname\iftrue
    \edef\KVD@temp{\endgroup
      \noexpand\KVD@DefineKey{#1}{#2}%
    }%
  \KVD@temp
}
%    \end{macrocode}
%    \end{macro}
%    \begin{macro}{\KVD@DefineKey}
%    \begin{macrocode}
\def\KVD@DefineKey#1#2{%
  \ltx@ifnextchar[{%
    \KVD@DefineKeyWithDefault{#1}{#2}%
  }{%
    \long\expandafter\def\csname KV@#1@#2\endcsname##1%
  }%
}
%    \end{macrocode}
%    \end{macro}
%    \begin{macro}{\KVD@DefineKeyWithDefault}
%    \begin{macrocode}
\long\def\KVD@DefineKeyWithDefault#1#2[#3]{%
  \expandafter\def\csname KV@#1@#2@default\expandafter\endcsname
  \expandafter{%
    \csname KV@#1@#2\endcsname{#3}%
  }%
  \long\expandafter\def\csname KV@#1@#2\endcsname##1%
}
%    \end{macrocode}
%    \end{macro}
%
%    \begin{macrocode}
\KVD@AtEnd%
%</package>
%    \end{macrocode}
%
% \section{Test}
%
% \subsection{Catcode checks for loading}
%
%    \begin{macrocode}
%<*test1>
%    \end{macrocode}
%    \begin{macrocode}
\catcode`\{=1 %
\catcode`\}=2 %
\catcode`\#=6 %
\catcode`\@=11 %
\expandafter\ifx\csname count@\endcsname\relax
  \countdef\count@=255 %
\fi
\expandafter\ifx\csname @gobble\endcsname\relax
  \long\def\@gobble#1{}%
\fi
\expandafter\ifx\csname @firstofone\endcsname\relax
  \long\def\@firstofone#1{#1}%
\fi
\expandafter\ifx\csname loop\endcsname\relax
  \expandafter\@firstofone
\else
  \expandafter\@gobble
\fi
{%
  \def\loop#1\repeat{%
    \def\body{#1}%
    \iterate
  }%
  \def\iterate{%
    \body
      \let\next\iterate
    \else
      \let\next\relax
    \fi
    \next
  }%
  \let\repeat=\fi
}%
\def\RestoreCatcodes{}
\count@=0 %
\loop
  \edef\RestoreCatcodes{%
    \RestoreCatcodes
    \catcode\the\count@=\the\catcode\count@\relax
  }%
\ifnum\count@<255 %
  \advance\count@ 1 %
\repeat

\def\RangeCatcodeInvalid#1#2{%
  \count@=#1\relax
  \loop
    \catcode\count@=15 %
  \ifnum\count@<#2\relax
    \advance\count@ 1 %
  \repeat
}
\def\RangeCatcodeCheck#1#2#3{%
  \count@=#1\relax
  \loop
    \ifnum#3=\catcode\count@
    \else
      \errmessage{%
        Character \the\count@\space
        with wrong catcode \the\catcode\count@\space
        instead of \number#3%
      }%
    \fi
  \ifnum\count@<#2\relax
    \advance\count@ 1 %
  \repeat
}
\def\space{ }
\expandafter\ifx\csname LoadCommand\endcsname\relax
  \def\LoadCommand{\input kvdefinekeys.sty\relax}%
\fi
\def\Test{%
  \RangeCatcodeInvalid{0}{47}%
  \RangeCatcodeInvalid{58}{64}%
  \RangeCatcodeInvalid{91}{96}%
  \RangeCatcodeInvalid{123}{255}%
  \catcode`\@=12 %
  \catcode`\\=0 %
  \catcode`\%=14 %
  \LoadCommand
  \RangeCatcodeCheck{0}{36}{15}%
  \RangeCatcodeCheck{37}{37}{14}%
  \RangeCatcodeCheck{38}{47}{15}%
  \RangeCatcodeCheck{48}{57}{12}%
  \RangeCatcodeCheck{58}{63}{15}%
  \RangeCatcodeCheck{64}{64}{12}%
  \RangeCatcodeCheck{65}{90}{11}%
  \RangeCatcodeCheck{91}{91}{15}%
  \RangeCatcodeCheck{92}{92}{0}%
  \RangeCatcodeCheck{93}{96}{15}%
  \RangeCatcodeCheck{97}{122}{11}%
  \RangeCatcodeCheck{123}{255}{15}%
  \RestoreCatcodes
}
\Test
\csname @@end\endcsname
\end
%    \end{macrocode}
%    \begin{macrocode}
%</test1>
%    \end{macrocode}
%
% \section{Installation}
%
% \subsection{Download}
%
% \paragraph{Package.} This package is available on
% CTAN\footnote{\url{http://ctan.org/pkg/kvdefinekeys}}:
% \begin{description}
% \item[\CTAN{macros/latex/contrib/oberdiek/kvdefinekeys.dtx}] The source file.
% \item[\CTAN{macros/latex/contrib/oberdiek/kvdefinekeys.pdf}] Documentation.
% \end{description}
%
%
% \paragraph{Bundle.} All the packages of the bundle `oberdiek'
% are also available in a TDS compliant ZIP archive. There
% the packages are already unpacked and the documentation files
% are generated. The files and directories obey the TDS standard.
% \begin{description}
% \item[\CTAN{install/macros/latex/contrib/oberdiek.tds.zip}]
% \end{description}
% \emph{TDS} refers to the standard ``A Directory Structure
% for \TeX\ Files'' (\CTAN{tds/tds.pdf}). Directories
% with \xfile{texmf} in their name are usually organized this way.
%
% \subsection{Bundle installation}
%
% \paragraph{Unpacking.} Unpack the \xfile{oberdiek.tds.zip} in the
% TDS tree (also known as \xfile{texmf} tree) of your choice.
% Example (linux):
% \begin{quote}
%   |unzip oberdiek.tds.zip -d ~/texmf|
% \end{quote}
%
% \paragraph{Script installation.}
% Check the directory \xfile{TDS:scripts/oberdiek/} for
% scripts that need further installation steps.
% Package \xpackage{attachfile2} comes with the Perl script
% \xfile{pdfatfi.pl} that should be installed in such a way
% that it can be called as \texttt{pdfatfi}.
% Example (linux):
% \begin{quote}
%   |chmod +x scripts/oberdiek/pdfatfi.pl|\\
%   |cp scripts/oberdiek/pdfatfi.pl /usr/local/bin/|
% \end{quote}
%
% \subsection{Package installation}
%
% \paragraph{Unpacking.} The \xfile{.dtx} file is a self-extracting
% \docstrip\ archive. The files are extracted by running the
% \xfile{.dtx} through \plainTeX:
% \begin{quote}
%   \verb|tex kvdefinekeys.dtx|
% \end{quote}
%
% \paragraph{TDS.} Now the different files must be moved into
% the different directories in your installation TDS tree
% (also known as \xfile{texmf} tree):
% \begin{quote}
% \def\t{^^A
% \begin{tabular}{@{}>{\ttfamily}l@{ $\rightarrow$ }>{\ttfamily}l@{}}
%   kvdefinekeys.sty & tex/generic/oberdiek/kvdefinekeys.sty\\
%   kvdefinekeys.pdf & doc/latex/oberdiek/kvdefinekeys.pdf\\
%   test/kvdefinekeys-test1.tex & doc/latex/oberdiek/test/kvdefinekeys-test1.tex\\
%   kvdefinekeys.dtx & source/latex/oberdiek/kvdefinekeys.dtx\\
% \end{tabular}^^A
% }^^A
% \sbox0{\t}^^A
% \ifdim\wd0>\linewidth
%   \begingroup
%     \advance\linewidth by\leftmargin
%     \advance\linewidth by\rightmargin
%   \edef\x{\endgroup
%     \def\noexpand\lw{\the\linewidth}^^A
%   }\x
%   \def\lwbox{^^A
%     \leavevmode
%     \hbox to \linewidth{^^A
%       \kern-\leftmargin\relax
%       \hss
%       \usebox0
%       \hss
%       \kern-\rightmargin\relax
%     }^^A
%   }^^A
%   \ifdim\wd0>\lw
%     \sbox0{\small\t}^^A
%     \ifdim\wd0>\linewidth
%       \ifdim\wd0>\lw
%         \sbox0{\footnotesize\t}^^A
%         \ifdim\wd0>\linewidth
%           \ifdim\wd0>\lw
%             \sbox0{\scriptsize\t}^^A
%             \ifdim\wd0>\linewidth
%               \ifdim\wd0>\lw
%                 \sbox0{\tiny\t}^^A
%                 \ifdim\wd0>\linewidth
%                   \lwbox
%                 \else
%                   \usebox0
%                 \fi
%               \else
%                 \lwbox
%               \fi
%             \else
%               \usebox0
%             \fi
%           \else
%             \lwbox
%           \fi
%         \else
%           \usebox0
%         \fi
%       \else
%         \lwbox
%       \fi
%     \else
%       \usebox0
%     \fi
%   \else
%     \lwbox
%   \fi
% \else
%   \usebox0
% \fi
% \end{quote}
% If you have a \xfile{docstrip.cfg} that configures and enables \docstrip's
% TDS installing feature, then some files can already be in the right
% place, see the documentation of \docstrip.
%
% \subsection{Refresh file name databases}
%
% If your \TeX~distribution
% (\teTeX, \mikTeX, \dots) relies on file name databases, you must refresh
% these. For example, \teTeX\ users run \verb|texhash| or
% \verb|mktexlsr|.
%
% \subsection{Some details for the interested}
%
% \paragraph{Attached source.}
%
% The PDF documentation on CTAN also includes the
% \xfile{.dtx} source file. It can be extracted by
% AcrobatReader 6 or higher. Another option is \textsf{pdftk},
% e.g. unpack the file into the current directory:
% \begin{quote}
%   \verb|pdftk kvdefinekeys.pdf unpack_files output .|
% \end{quote}
%
% \paragraph{Unpacking with \LaTeX.}
% The \xfile{.dtx} chooses its action depending on the format:
% \begin{description}
% \item[\plainTeX:] Run \docstrip\ and extract the files.
% \item[\LaTeX:] Generate the documentation.
% \end{description}
% If you insist on using \LaTeX\ for \docstrip\ (really,
% \docstrip\ does not need \LaTeX), then inform the autodetect routine
% about your intention:
% \begin{quote}
%   \verb|latex \let\install=y\input{kvdefinekeys.dtx}|
% \end{quote}
% Do not forget to quote the argument according to the demands
% of your shell.
%
% \paragraph{Generating the documentation.}
% You can use both the \xfile{.dtx} or the \xfile{.drv} to generate
% the documentation. The process can be configured by the
% configuration file \xfile{ltxdoc.cfg}. For instance, put this
% line into this file, if you want to have A4 as paper format:
% \begin{quote}
%   \verb|\PassOptionsToClass{a4paper}{article}|
% \end{quote}
% An example follows how to generate the
% documentation with pdf\LaTeX:
% \begin{quote}
%\begin{verbatim}
%pdflatex kvdefinekeys.dtx
%makeindex -s gind.ist kvdefinekeys.idx
%pdflatex kvdefinekeys.dtx
%makeindex -s gind.ist kvdefinekeys.idx
%pdflatex kvdefinekeys.dtx
%\end{verbatim}
% \end{quote}
%
% \section{Catalogue}
%
% The following XML file can be used as source for the
% \href{http://mirror.ctan.org/help/Catalogue/catalogue.html}{\TeX\ Catalogue}.
% The elements \texttt{caption} and \texttt{description} are imported
% from the original XML file from the Catalogue.
% The name of the XML file in the Catalogue is \xfile{kvdefinekeys.xml}.
%    \begin{macrocode}
%<*catalogue>
<?xml version='1.0' encoding='us-ascii'?>
<!DOCTYPE entry SYSTEM 'catalogue.dtd'>
<entry datestamp='$Date$' modifier='$Author$' id='kvdefinekeys'>
  <name>kvdefinekeys</name>
  <caption>Define keys for use in the kvsetkeys package.</caption>
  <authorref id='auth:oberdiek'/>
  <copyright owner='Heiko Oberdiek' year='2010,2011'/>
  <license type='lppl1.3'/>
  <version number='1.4'/>
  <description>
    The package provides a macro <tt>\kv@define@key</tt> (analogous to
    <xref refid='keyval'>keyval&#x2019;s</xref> <tt>\define@key</tt>, to
    define keys for use by <xref refid='kvsetkeys'>kvsetkeys</xref>.
    <p/>
    The package is part of the <xref refid='oberdiek'>oberdiek</xref>
    bundle.
  </description>
  <documentation details='Package documentation'
      href='ctan:/macros/latex/contrib/oberdiek/kvdefinekeys.pdf'/>
  <ctan file='true' path='/macros/latex/contrib/oberdiek/kvdefinekeys.dtx'/>
  <miktex location='oberdiek'/>
  <texlive location='oberdiek'/>
  <install path='/macros/latex/contrib/oberdiek/oberdiek.tds.zip'/>
</entry>
%</catalogue>
%    \end{macrocode}
%
% \begin{thebibliography}{9}
% \bibitem{keyval}
%   David Carlisle:
%   \textit{The \xpackage{keyval} package};
%   1999/03/16 v1.13;
%   \CTAN{macros/latex/required/graphics/keyval.dtx}.
%
% \end{thebibliography}
%
% \begin{History}
%   \begin{Version}{2010/03/01 v1.0}
%   \item
%     First version.
%   \end{Version}
%   \begin{Version}{2010/08/19 v1.1}
%   \item
%     Documentation fix, no code change.
%   \end{Version}
%   \begin{Version}{2011/01/30 v1.2}
%   \item
%     Already loaded package files are not input in \hologo{plainTeX}.
%   \end{Version}
%   \begin{Version}{2011/04/07 v1.3}
%   \item
%     Support for package \xpackage{babel}'s shorthands added.
%   \item
%     \cs{kv@define@key} is made robust if available.
%   \end{Version}
%   \begin{Version}{2016/05/16 v1.4}
%   \item
%     Documentation updates.
%   \end{Version}
% \end{History}
%
% \PrintIndex
%
% \Finale
\endinput
|
% \end{quote}
% Do not forget to quote the argument according to the demands
% of your shell.
%
% \paragraph{Generating the documentation.}
% You can use both the \xfile{.dtx} or the \xfile{.drv} to generate
% the documentation. The process can be configured by the
% configuration file \xfile{ltxdoc.cfg}. For instance, put this
% line into this file, if you want to have A4 as paper format:
% \begin{quote}
%   \verb|\PassOptionsToClass{a4paper}{article}|
% \end{quote}
% An example follows how to generate the
% documentation with pdf\LaTeX:
% \begin{quote}
%\begin{verbatim}
%pdflatex kvdefinekeys.dtx
%makeindex -s gind.ist kvdefinekeys.idx
%pdflatex kvdefinekeys.dtx
%makeindex -s gind.ist kvdefinekeys.idx
%pdflatex kvdefinekeys.dtx
%\end{verbatim}
% \end{quote}
%
% \section{Catalogue}
%
% The following XML file can be used as source for the
% \href{http://mirror.ctan.org/help/Catalogue/catalogue.html}{\TeX\ Catalogue}.
% The elements \texttt{caption} and \texttt{description} are imported
% from the original XML file from the Catalogue.
% The name of the XML file in the Catalogue is \xfile{kvdefinekeys.xml}.
%    \begin{macrocode}
%<*catalogue>
<?xml version='1.0' encoding='us-ascii'?>
<!DOCTYPE entry SYSTEM 'catalogue.dtd'>
<entry datestamp='$Date$' modifier='$Author$' id='kvdefinekeys'>
  <name>kvdefinekeys</name>
  <caption>Define keys for use in the kvsetkeys package.</caption>
  <authorref id='auth:oberdiek'/>
  <copyright owner='Heiko Oberdiek' year='2010,2011'/>
  <license type='lppl1.3'/>
  <version number='1.4'/>
  <description>
    The package provides a macro <tt>\kv@define@key</tt> (analogous to
    <xref refid='keyval'>keyval&#x2019;s</xref> <tt>\define@key</tt>, to
    define keys for use by <xref refid='kvsetkeys'>kvsetkeys</xref>.
    <p/>
    The package is part of the <xref refid='oberdiek'>oberdiek</xref>
    bundle.
  </description>
  <documentation details='Package documentation'
      href='ctan:/macros/latex/contrib/oberdiek/kvdefinekeys.pdf'/>
  <ctan file='true' path='/macros/latex/contrib/oberdiek/kvdefinekeys.dtx'/>
  <miktex location='oberdiek'/>
  <texlive location='oberdiek'/>
  <install path='/macros/latex/contrib/oberdiek/oberdiek.tds.zip'/>
</entry>
%</catalogue>
%    \end{macrocode}
%
% \begin{thebibliography}{9}
% \bibitem{keyval}
%   David Carlisle:
%   \textit{The \xpackage{keyval} package};
%   1999/03/16 v1.13;
%   \CTAN{macros/latex/required/graphics/keyval.dtx}.
%
% \end{thebibliography}
%
% \begin{History}
%   \begin{Version}{2010/03/01 v1.0}
%   \item
%     First version.
%   \end{Version}
%   \begin{Version}{2010/08/19 v1.1}
%   \item
%     Documentation fix, no code change.
%   \end{Version}
%   \begin{Version}{2011/01/30 v1.2}
%   \item
%     Already loaded package files are not input in \hologo{plainTeX}.
%   \end{Version}
%   \begin{Version}{2011/04/07 v1.3}
%   \item
%     Support for package \xpackage{babel}'s shorthands added.
%   \item
%     \cs{kv@define@key} is made robust if available.
%   \end{Version}
%   \begin{Version}{2016/05/16 v1.4}
%   \item
%     Documentation updates.
%   \end{Version}
% \end{History}
%
% \PrintIndex
%
% \Finale
\endinput

%        (quote the arguments according to the demands of your shell)
%
% Documentation:
%    (a) If kvdefinekeys.drv is present:
%           latex kvdefinekeys.drv
%    (b) Without kvdefinekeys.drv:
%           latex kvdefinekeys.dtx; ...
%    The class ltxdoc loads the configuration file ltxdoc.cfg
%    if available. Here you can specify further options, e.g.
%    use A4 as paper format:
%       \PassOptionsToClass{a4paper}{article}
%
%    Programm calls to get the documentation (example):
%       pdflatex kvdefinekeys.dtx
%       makeindex -s gind.ist kvdefinekeys.idx
%       pdflatex kvdefinekeys.dtx
%       makeindex -s gind.ist kvdefinekeys.idx
%       pdflatex kvdefinekeys.dtx
%
% Installation:
%    TDS:tex/generic/oberdiek/kvdefinekeys.sty
%    TDS:doc/latex/oberdiek/kvdefinekeys.pdf
%    TDS:doc/latex/oberdiek/test/kvdefinekeys-test1.tex
%    TDS:source/latex/oberdiek/kvdefinekeys.dtx
%
%<*ignore>
\begingroup
  \catcode123=1 %
  \catcode125=2 %
  \def\x{LaTeX2e}%
\expandafter\endgroup
\ifcase 0\ifx\install y1\fi\expandafter
         \ifx\csname processbatchFile\endcsname\relax\else1\fi
         \ifx\fmtname\x\else 1\fi\relax
\else\csname fi\endcsname
%</ignore>
%<*install>
\input docstrip.tex
\Msg{************************************************************************}
\Msg{* Installation}
\Msg{* Package: kvdefinekeys 2016/05/16 v1.4 Define keys (HO)}
\Msg{************************************************************************}

\keepsilent
\askforoverwritefalse

\let\MetaPrefix\relax
\preamble

This is a generated file.

Project: kvdefinekeys
Version: 2016/05/16 v1.4

Copyright (C) 2010, 2011 by
   Heiko Oberdiek <heiko.oberdiek at googlemail.com>

This work may be distributed and/or modified under the
conditions of the LaTeX Project Public License, either
version 1.3c of this license or (at your option) any later
version. This version of this license is in
   http://www.latex-project.org/lppl/lppl-1-3c.txt
and the latest version of this license is in
   http://www.latex-project.org/lppl.txt
and version 1.3 or later is part of all distributions of
LaTeX version 2005/12/01 or later.

This work has the LPPL maintenance status "maintained".

This Current Maintainer of this work is Heiko Oberdiek.

The Base Interpreter refers to any `TeX-Format',
because some files are installed in TDS:tex/generic//.

This work consists of the main source file kvdefinekeys.dtx
and the derived files
   kvdefinekeys.sty, kvdefinekeys.pdf, kvdefinekeys.ins, kvdefinekeys.drv,
   kvdefinekeys-test1.tex.

\endpreamble
\let\MetaPrefix\DoubleperCent

\generate{%
  \file{kvdefinekeys.ins}{\from{kvdefinekeys.dtx}{install}}%
  \file{kvdefinekeys.drv}{\from{kvdefinekeys.dtx}{driver}}%
  \usedir{tex/generic/oberdiek}%
  \file{kvdefinekeys.sty}{\from{kvdefinekeys.dtx}{package}}%
  \usedir{doc/latex/oberdiek/test}%
  \file{kvdefinekeys-test1.tex}{\from{kvdefinekeys.dtx}{test1}}%
  \nopreamble
  \nopostamble
  \usedir{source/latex/oberdiek/catalogue}%
  \file{kvdefinekeys.xml}{\from{kvdefinekeys.dtx}{catalogue}}%
}

\catcode32=13\relax% active space
\let =\space%
\Msg{************************************************************************}
\Msg{*}
\Msg{* To finish the installation you have to move the following}
\Msg{* file into a directory searched by TeX:}
\Msg{*}
\Msg{*     kvdefinekeys.sty}
\Msg{*}
\Msg{* To produce the documentation run the file `kvdefinekeys.drv'}
\Msg{* through LaTeX.}
\Msg{*}
\Msg{* Happy TeXing!}
\Msg{*}
\Msg{************************************************************************}

\endbatchfile
%</install>
%<*ignore>
\fi
%</ignore>
%<*driver>
\NeedsTeXFormat{LaTeX2e}
\ProvidesFile{kvdefinekeys.drv}%
  [2016/05/16 v1.4 Define keys (HO)]%
\documentclass{ltxdoc}
\usepackage{holtxdoc}[2011/11/22]
\begin{document}
  \DocInput{kvdefinekeys.dtx}%
\end{document}
%</driver>
% \fi
%
%
% \CharacterTable
%  {Upper-case    \A\B\C\D\E\F\G\H\I\J\K\L\M\N\O\P\Q\R\S\T\U\V\W\X\Y\Z
%   Lower-case    \a\b\c\d\e\f\g\h\i\j\k\l\m\n\o\p\q\r\s\t\u\v\w\x\y\z
%   Digits        \0\1\2\3\4\5\6\7\8\9
%   Exclamation   \!     Double quote  \"     Hash (number) \#
%   Dollar        \$     Percent       \%     Ampersand     \&
%   Acute accent  \'     Left paren    \(     Right paren   \)
%   Asterisk      \*     Plus          \+     Comma         \,
%   Minus         \-     Point         \.     Solidus       \/
%   Colon         \:     Semicolon     \;     Less than     \<
%   Equals        \=     Greater than  \>     Question mark \?
%   Commercial at \@     Left bracket  \[     Backslash     \\
%   Right bracket \]     Circumflex    \^     Underscore    \_
%   Grave accent  \`     Left brace    \{     Vertical bar  \|
%   Right brace   \}     Tilde         \~}
%
% \GetFileInfo{kvdefinekeys.drv}
%
% \title{The \xpackage{kvdefinekeys} package}
% \date{2016/05/16 v1.4}
% \author{Heiko Oberdiek\thanks
% {Please report any issues at https://github.com/ho-tex/oberdiek/issues}\\
% \xemail{heiko.oberdiek at googlemail.com}}
%
% \maketitle
%
% \begin{abstract}
% Package \xpackage{kvdefinekeys} provides \cs{kv@define@key} to define
% keys the same way as \xpackage{keyval}'s \cs{define@key}. However, it
% works also using \iniTeX.
% \end{abstract}
%
% \tableofcontents
%
% \def\M#1{\texttt{\{}\meta{#1}\texttt{\}}}
%
% \section{Documentation}
%
% \subsection{Motivation}
%
% \cs{kvsetkeys} serves as replacement for \xpackage{keyval}'s
% \cs{setkeys}. This package adds macros to define keys, closing
% the gap \cs{kvsetkeys} leaves.
%
% \begin{declcs}{kv@define@key}\,\M{family}\,\M{key}\,[\meta{default}]\,^^A
%   \M{definition}
% \end{declcs}
% Macro \cs{kv@define@key} reimplements \xpackage{keyval}'s
% \cs{define@key}. Differences to the original:
% \begin{itemize}
% \item The defined keys also allow \cs{par} inside values.
% \item Shorthands of package \xpackage{babel} are supported in
%   family and key names.
% \item Macro \cs{kv@define@key} is made robust if
%   \hologo{eTeX}'s \cs{protected} or \hologo{LaTeX}'s
%   \cs{DeclareRobustCommand} are found.
% \end{itemize}
%
% \StopEventually{
% }
%
% \section{Implementation}
%
% \subsection{Identification}
%
%    \begin{macrocode}
%<*package>
%    \end{macrocode}
%    Reload check, especially if the package is not used with \LaTeX.
%    \begin{macrocode}
\begingroup\catcode61\catcode48\catcode32=10\relax%
  \catcode13=5 % ^^M
  \endlinechar=13 %
  \catcode35=6 % #
  \catcode39=12 % '
  \catcode44=12 % ,
  \catcode45=12 % -
  \catcode46=12 % .
  \catcode58=12 % :
  \catcode64=11 % @
  \catcode123=1 % {
  \catcode125=2 % }
  \expandafter\let\expandafter\x\csname ver@kvdefinekeys.sty\endcsname
  \ifx\x\relax % plain-TeX, first loading
  \else
    \def\empty{}%
    \ifx\x\empty % LaTeX, first loading,
      % variable is initialized, but \ProvidesPackage not yet seen
    \else
      \expandafter\ifx\csname PackageInfo\endcsname\relax
        \def\x#1#2{%
          \immediate\write-1{Package #1 Info: #2.}%
        }%
      \else
        \def\x#1#2{\PackageInfo{#1}{#2, stopped}}%
      \fi
      \x{kvdefinekeys}{The package is already loaded}%
      \aftergroup\endinput
    \fi
  \fi
\endgroup%
%    \end{macrocode}
%    Package identification:
%    \begin{macrocode}
\begingroup\catcode61\catcode48\catcode32=10\relax%
  \catcode13=5 % ^^M
  \endlinechar=13 %
  \catcode35=6 % #
  \catcode39=12 % '
  \catcode40=12 % (
  \catcode41=12 % )
  \catcode44=12 % ,
  \catcode45=12 % -
  \catcode46=12 % .
  \catcode47=12 % /
  \catcode58=12 % :
  \catcode64=11 % @
  \catcode91=12 % [
  \catcode93=12 % ]
  \catcode123=1 % {
  \catcode125=2 % }
  \expandafter\ifx\csname ProvidesPackage\endcsname\relax
    \def\x#1#2#3[#4]{\endgroup
      \immediate\write-1{Package: #3 #4}%
      \xdef#1{#4}%
    }%
  \else
    \def\x#1#2[#3]{\endgroup
      #2[{#3}]%
      \ifx#1\@undefined
        \xdef#1{#3}%
      \fi
      \ifx#1\relax
        \xdef#1{#3}%
      \fi
    }%
  \fi
\expandafter\x\csname ver@kvdefinekeys.sty\endcsname
\ProvidesPackage{kvdefinekeys}%
  [2016/05/16 v1.4 Define keys (HO)]%
%    \end{macrocode}
%
%    \begin{macrocode}
\begingroup\catcode61\catcode48\catcode32=10\relax%
  \catcode13=5 % ^^M
  \endlinechar=13 %
  \catcode123=1 % {
  \catcode125=2 % }
  \catcode64=11 % @
  \def\x{\endgroup
    \expandafter\edef\csname KVD@AtEnd\endcsname{%
      \endlinechar=\the\endlinechar\relax
      \catcode13=\the\catcode13\relax
      \catcode32=\the\catcode32\relax
      \catcode35=\the\catcode35\relax
      \catcode61=\the\catcode61\relax
      \catcode64=\the\catcode64\relax
      \catcode123=\the\catcode123\relax
      \catcode125=\the\catcode125\relax
    }%
  }%
\x\catcode61\catcode48\catcode32=10\relax%
\catcode13=5 % ^^M
\endlinechar=13 %
\catcode35=6 % #
\catcode64=11 % @
\catcode123=1 % {
\catcode125=2 % }
\def\TMP@EnsureCode#1#2{%
  \edef\KVD@AtEnd{%
    \KVD@AtEnd
    \catcode#1=\the\catcode#1\relax
  }%
  \catcode#1=#2\relax
}
\TMP@EnsureCode{42}{12}% *
\TMP@EnsureCode{46}{12}% .
\TMP@EnsureCode{47}{12}% /
\TMP@EnsureCode{91}{12}% [
\TMP@EnsureCode{93}{12}% ]
\edef\KVD@AtEnd{\KVD@AtEnd\noexpand\endinput}
%    \end{macrocode}
%
% \subsection{Package loading}
%
%    \begin{macrocode}
\begingroup\expandafter\expandafter\expandafter\endgroup
\expandafter\ifx\csname RequirePackage\endcsname\relax
  \def\TMP@RequirePackage#1[#2]{%
    \begingroup\expandafter\expandafter\expandafter\endgroup
    \expandafter\ifx\csname ver@#1.sty\endcsname\relax
      \input #1.sty\relax
    \fi
  }%
  \TMP@RequirePackage{ltxcmds}[2010/03/01]%
\else
  \RequirePackage{ltxcmds}[2010/03/01]%
\fi
%    \end{macrocode}
%
%
% \subsection{Provide key defining macro}
%
%    \begin{macro}{\kv@define@key}
%    \begin{macrocode}
\ltx@IfUndefined{protected}{%
  \ltx@IfUndefined{DeclareRobustCommand}{%
    \def\kv@define@key#1#2%
  }{%
    \DeclareRobustCommand*{\kv@define@key}[2]%
  }%
}{%
  \protected\def\kv@define@key#1#2%
}%
{%
  \begingroup
    \csname @safe@activestrue\endcsname
    \let\ifincsname\iftrue
    \edef\KVD@temp{\endgroup
      \noexpand\KVD@DefineKey{#1}{#2}%
    }%
  \KVD@temp
}
%    \end{macrocode}
%    \end{macro}
%    \begin{macro}{\KVD@DefineKey}
%    \begin{macrocode}
\def\KVD@DefineKey#1#2{%
  \ltx@ifnextchar[{%
    \KVD@DefineKeyWithDefault{#1}{#2}%
  }{%
    \long\expandafter\def\csname KV@#1@#2\endcsname##1%
  }%
}
%    \end{macrocode}
%    \end{macro}
%    \begin{macro}{\KVD@DefineKeyWithDefault}
%    \begin{macrocode}
\long\def\KVD@DefineKeyWithDefault#1#2[#3]{%
  \expandafter\def\csname KV@#1@#2@default\expandafter\endcsname
  \expandafter{%
    \csname KV@#1@#2\endcsname{#3}%
  }%
  \long\expandafter\def\csname KV@#1@#2\endcsname##1%
}
%    \end{macrocode}
%    \end{macro}
%
%    \begin{macrocode}
\KVD@AtEnd%
%</package>
%    \end{macrocode}
%
% \section{Test}
%
% \subsection{Catcode checks for loading}
%
%    \begin{macrocode}
%<*test1>
%    \end{macrocode}
%    \begin{macrocode}
\catcode`\{=1 %
\catcode`\}=2 %
\catcode`\#=6 %
\catcode`\@=11 %
\expandafter\ifx\csname count@\endcsname\relax
  \countdef\count@=255 %
\fi
\expandafter\ifx\csname @gobble\endcsname\relax
  \long\def\@gobble#1{}%
\fi
\expandafter\ifx\csname @firstofone\endcsname\relax
  \long\def\@firstofone#1{#1}%
\fi
\expandafter\ifx\csname loop\endcsname\relax
  \expandafter\@firstofone
\else
  \expandafter\@gobble
\fi
{%
  \def\loop#1\repeat{%
    \def\body{#1}%
    \iterate
  }%
  \def\iterate{%
    \body
      \let\next\iterate
    \else
      \let\next\relax
    \fi
    \next
  }%
  \let\repeat=\fi
}%
\def\RestoreCatcodes{}
\count@=0 %
\loop
  \edef\RestoreCatcodes{%
    \RestoreCatcodes
    \catcode\the\count@=\the\catcode\count@\relax
  }%
\ifnum\count@<255 %
  \advance\count@ 1 %
\repeat

\def\RangeCatcodeInvalid#1#2{%
  \count@=#1\relax
  \loop
    \catcode\count@=15 %
  \ifnum\count@<#2\relax
    \advance\count@ 1 %
  \repeat
}
\def\RangeCatcodeCheck#1#2#3{%
  \count@=#1\relax
  \loop
    \ifnum#3=\catcode\count@
    \else
      \errmessage{%
        Character \the\count@\space
        with wrong catcode \the\catcode\count@\space
        instead of \number#3%
      }%
    \fi
  \ifnum\count@<#2\relax
    \advance\count@ 1 %
  \repeat
}
\def\space{ }
\expandafter\ifx\csname LoadCommand\endcsname\relax
  \def\LoadCommand{\input kvdefinekeys.sty\relax}%
\fi
\def\Test{%
  \RangeCatcodeInvalid{0}{47}%
  \RangeCatcodeInvalid{58}{64}%
  \RangeCatcodeInvalid{91}{96}%
  \RangeCatcodeInvalid{123}{255}%
  \catcode`\@=12 %
  \catcode`\\=0 %
  \catcode`\%=14 %
  \LoadCommand
  \RangeCatcodeCheck{0}{36}{15}%
  \RangeCatcodeCheck{37}{37}{14}%
  \RangeCatcodeCheck{38}{47}{15}%
  \RangeCatcodeCheck{48}{57}{12}%
  \RangeCatcodeCheck{58}{63}{15}%
  \RangeCatcodeCheck{64}{64}{12}%
  \RangeCatcodeCheck{65}{90}{11}%
  \RangeCatcodeCheck{91}{91}{15}%
  \RangeCatcodeCheck{92}{92}{0}%
  \RangeCatcodeCheck{93}{96}{15}%
  \RangeCatcodeCheck{97}{122}{11}%
  \RangeCatcodeCheck{123}{255}{15}%
  \RestoreCatcodes
}
\Test
\csname @@end\endcsname
\end
%    \end{macrocode}
%    \begin{macrocode}
%</test1>
%    \end{macrocode}
%
% \section{Installation}
%
% \subsection{Download}
%
% \paragraph{Package.} This package is available on
% CTAN\footnote{\url{http://ctan.org/pkg/kvdefinekeys}}:
% \begin{description}
% \item[\CTAN{macros/latex/contrib/oberdiek/kvdefinekeys.dtx}] The source file.
% \item[\CTAN{macros/latex/contrib/oberdiek/kvdefinekeys.pdf}] Documentation.
% \end{description}
%
%
% \paragraph{Bundle.} All the packages of the bundle `oberdiek'
% are also available in a TDS compliant ZIP archive. There
% the packages are already unpacked and the documentation files
% are generated. The files and directories obey the TDS standard.
% \begin{description}
% \item[\CTAN{install/macros/latex/contrib/oberdiek.tds.zip}]
% \end{description}
% \emph{TDS} refers to the standard ``A Directory Structure
% for \TeX\ Files'' (\CTAN{tds/tds.pdf}). Directories
% with \xfile{texmf} in their name are usually organized this way.
%
% \subsection{Bundle installation}
%
% \paragraph{Unpacking.} Unpack the \xfile{oberdiek.tds.zip} in the
% TDS tree (also known as \xfile{texmf} tree) of your choice.
% Example (linux):
% \begin{quote}
%   |unzip oberdiek.tds.zip -d ~/texmf|
% \end{quote}
%
% \paragraph{Script installation.}
% Check the directory \xfile{TDS:scripts/oberdiek/} for
% scripts that need further installation steps.
% Package \xpackage{attachfile2} comes with the Perl script
% \xfile{pdfatfi.pl} that should be installed in such a way
% that it can be called as \texttt{pdfatfi}.
% Example (linux):
% \begin{quote}
%   |chmod +x scripts/oberdiek/pdfatfi.pl|\\
%   |cp scripts/oberdiek/pdfatfi.pl /usr/local/bin/|
% \end{quote}
%
% \subsection{Package installation}
%
% \paragraph{Unpacking.} The \xfile{.dtx} file is a self-extracting
% \docstrip\ archive. The files are extracted by running the
% \xfile{.dtx} through \plainTeX:
% \begin{quote}
%   \verb|tex kvdefinekeys.dtx|
% \end{quote}
%
% \paragraph{TDS.} Now the different files must be moved into
% the different directories in your installation TDS tree
% (also known as \xfile{texmf} tree):
% \begin{quote}
% \def\t{^^A
% \begin{tabular}{@{}>{\ttfamily}l@{ $\rightarrow$ }>{\ttfamily}l@{}}
%   kvdefinekeys.sty & tex/generic/oberdiek/kvdefinekeys.sty\\
%   kvdefinekeys.pdf & doc/latex/oberdiek/kvdefinekeys.pdf\\
%   test/kvdefinekeys-test1.tex & doc/latex/oberdiek/test/kvdefinekeys-test1.tex\\
%   kvdefinekeys.dtx & source/latex/oberdiek/kvdefinekeys.dtx\\
% \end{tabular}^^A
% }^^A
% \sbox0{\t}^^A
% \ifdim\wd0>\linewidth
%   \begingroup
%     \advance\linewidth by\leftmargin
%     \advance\linewidth by\rightmargin
%   \edef\x{\endgroup
%     \def\noexpand\lw{\the\linewidth}^^A
%   }\x
%   \def\lwbox{^^A
%     \leavevmode
%     \hbox to \linewidth{^^A
%       \kern-\leftmargin\relax
%       \hss
%       \usebox0
%       \hss
%       \kern-\rightmargin\relax
%     }^^A
%   }^^A
%   \ifdim\wd0>\lw
%     \sbox0{\small\t}^^A
%     \ifdim\wd0>\linewidth
%       \ifdim\wd0>\lw
%         \sbox0{\footnotesize\t}^^A
%         \ifdim\wd0>\linewidth
%           \ifdim\wd0>\lw
%             \sbox0{\scriptsize\t}^^A
%             \ifdim\wd0>\linewidth
%               \ifdim\wd0>\lw
%                 \sbox0{\tiny\t}^^A
%                 \ifdim\wd0>\linewidth
%                   \lwbox
%                 \else
%                   \usebox0
%                 \fi
%               \else
%                 \lwbox
%               \fi
%             \else
%               \usebox0
%             \fi
%           \else
%             \lwbox
%           \fi
%         \else
%           \usebox0
%         \fi
%       \else
%         \lwbox
%       \fi
%     \else
%       \usebox0
%     \fi
%   \else
%     \lwbox
%   \fi
% \else
%   \usebox0
% \fi
% \end{quote}
% If you have a \xfile{docstrip.cfg} that configures and enables \docstrip's
% TDS installing feature, then some files can already be in the right
% place, see the documentation of \docstrip.
%
% \subsection{Refresh file name databases}
%
% If your \TeX~distribution
% (\teTeX, \mikTeX, \dots) relies on file name databases, you must refresh
% these. For example, \teTeX\ users run \verb|texhash| or
% \verb|mktexlsr|.
%
% \subsection{Some details for the interested}
%
% \paragraph{Attached source.}
%
% The PDF documentation on CTAN also includes the
% \xfile{.dtx} source file. It can be extracted by
% AcrobatReader 6 or higher. Another option is \textsf{pdftk},
% e.g. unpack the file into the current directory:
% \begin{quote}
%   \verb|pdftk kvdefinekeys.pdf unpack_files output .|
% \end{quote}
%
% \paragraph{Unpacking with \LaTeX.}
% The \xfile{.dtx} chooses its action depending on the format:
% \begin{description}
% \item[\plainTeX:] Run \docstrip\ and extract the files.
% \item[\LaTeX:] Generate the documentation.
% \end{description}
% If you insist on using \LaTeX\ for \docstrip\ (really,
% \docstrip\ does not need \LaTeX), then inform the autodetect routine
% about your intention:
% \begin{quote}
%   \verb|latex \let\install=y% \iffalse meta-comment
%
% File: kvdefinekeys.dtx
% Version: 2016/05/16 v1.4
% Info: Define keys
%
% Copyright (C) 2010, 2011 by
%    Heiko Oberdiek <heiko.oberdiek at googlemail.com>
%    2016
%    https://github.com/ho-tex/oberdiek/issues
%
% This work may be distributed and/or modified under the
% conditions of the LaTeX Project Public License, either
% version 1.3c of this license or (at your option) any later
% version. This version of this license is in
%    http://www.latex-project.org/lppl/lppl-1-3c.txt
% and the latest version of this license is in
%    http://www.latex-project.org/lppl.txt
% and version 1.3 or later is part of all distributions of
% LaTeX version 2005/12/01 or later.
%
% This work has the LPPL maintenance status "maintained".
%
% This Current Maintainer of this work is Heiko Oberdiek.
%
% The Base Interpreter refers to any `TeX-Format',
% because some files are installed in TDS:tex/generic//.
%
% This work consists of the main source file kvdefinekeys.dtx
% and the derived files
%    kvdefinekeys.sty, kvdefinekeys.pdf, kvdefinekeys.ins, kvdefinekeys.drv,
%    kvdefinekeys-test1.tex.
%
% Distribution:
%    CTAN:macros/latex/contrib/oberdiek/kvdefinekeys.dtx
%    CTAN:macros/latex/contrib/oberdiek/kvdefinekeys.pdf
%
% Unpacking:
%    (a) If kvdefinekeys.ins is present:
%           tex kvdefinekeys.ins
%    (b) Without kvdefinekeys.ins:
%           tex kvdefinekeys.dtx
%    (c) If you insist on using LaTeX
%           latex \let\install=y% \iffalse meta-comment
%
% File: kvdefinekeys.dtx
% Version: 2016/05/16 v1.4
% Info: Define keys
%
% Copyright (C) 2010, 2011 by
%    Heiko Oberdiek <heiko.oberdiek at googlemail.com>
%    2016
%    https://github.com/ho-tex/oberdiek/issues
%
% This work may be distributed and/or modified under the
% conditions of the LaTeX Project Public License, either
% version 1.3c of this license or (at your option) any later
% version. This version of this license is in
%    http://www.latex-project.org/lppl/lppl-1-3c.txt
% and the latest version of this license is in
%    http://www.latex-project.org/lppl.txt
% and version 1.3 or later is part of all distributions of
% LaTeX version 2005/12/01 or later.
%
% This work has the LPPL maintenance status "maintained".
%
% This Current Maintainer of this work is Heiko Oberdiek.
%
% The Base Interpreter refers to any `TeX-Format',
% because some files are installed in TDS:tex/generic//.
%
% This work consists of the main source file kvdefinekeys.dtx
% and the derived files
%    kvdefinekeys.sty, kvdefinekeys.pdf, kvdefinekeys.ins, kvdefinekeys.drv,
%    kvdefinekeys-test1.tex.
%
% Distribution:
%    CTAN:macros/latex/contrib/oberdiek/kvdefinekeys.dtx
%    CTAN:macros/latex/contrib/oberdiek/kvdefinekeys.pdf
%
% Unpacking:
%    (a) If kvdefinekeys.ins is present:
%           tex kvdefinekeys.ins
%    (b) Without kvdefinekeys.ins:
%           tex kvdefinekeys.dtx
%    (c) If you insist on using LaTeX
%           latex \let\install=y\input{kvdefinekeys.dtx}
%        (quote the arguments according to the demands of your shell)
%
% Documentation:
%    (a) If kvdefinekeys.drv is present:
%           latex kvdefinekeys.drv
%    (b) Without kvdefinekeys.drv:
%           latex kvdefinekeys.dtx; ...
%    The class ltxdoc loads the configuration file ltxdoc.cfg
%    if available. Here you can specify further options, e.g.
%    use A4 as paper format:
%       \PassOptionsToClass{a4paper}{article}
%
%    Programm calls to get the documentation (example):
%       pdflatex kvdefinekeys.dtx
%       makeindex -s gind.ist kvdefinekeys.idx
%       pdflatex kvdefinekeys.dtx
%       makeindex -s gind.ist kvdefinekeys.idx
%       pdflatex kvdefinekeys.dtx
%
% Installation:
%    TDS:tex/generic/oberdiek/kvdefinekeys.sty
%    TDS:doc/latex/oberdiek/kvdefinekeys.pdf
%    TDS:doc/latex/oberdiek/test/kvdefinekeys-test1.tex
%    TDS:source/latex/oberdiek/kvdefinekeys.dtx
%
%<*ignore>
\begingroup
  \catcode123=1 %
  \catcode125=2 %
  \def\x{LaTeX2e}%
\expandafter\endgroup
\ifcase 0\ifx\install y1\fi\expandafter
         \ifx\csname processbatchFile\endcsname\relax\else1\fi
         \ifx\fmtname\x\else 1\fi\relax
\else\csname fi\endcsname
%</ignore>
%<*install>
\input docstrip.tex
\Msg{************************************************************************}
\Msg{* Installation}
\Msg{* Package: kvdefinekeys 2016/05/16 v1.4 Define keys (HO)}
\Msg{************************************************************************}

\keepsilent
\askforoverwritefalse

\let\MetaPrefix\relax
\preamble

This is a generated file.

Project: kvdefinekeys
Version: 2016/05/16 v1.4

Copyright (C) 2010, 2011 by
   Heiko Oberdiek <heiko.oberdiek at googlemail.com>

This work may be distributed and/or modified under the
conditions of the LaTeX Project Public License, either
version 1.3c of this license or (at your option) any later
version. This version of this license is in
   http://www.latex-project.org/lppl/lppl-1-3c.txt
and the latest version of this license is in
   http://www.latex-project.org/lppl.txt
and version 1.3 or later is part of all distributions of
LaTeX version 2005/12/01 or later.

This work has the LPPL maintenance status "maintained".

This Current Maintainer of this work is Heiko Oberdiek.

The Base Interpreter refers to any `TeX-Format',
because some files are installed in TDS:tex/generic//.

This work consists of the main source file kvdefinekeys.dtx
and the derived files
   kvdefinekeys.sty, kvdefinekeys.pdf, kvdefinekeys.ins, kvdefinekeys.drv,
   kvdefinekeys-test1.tex.

\endpreamble
\let\MetaPrefix\DoubleperCent

\generate{%
  \file{kvdefinekeys.ins}{\from{kvdefinekeys.dtx}{install}}%
  \file{kvdefinekeys.drv}{\from{kvdefinekeys.dtx}{driver}}%
  \usedir{tex/generic/oberdiek}%
  \file{kvdefinekeys.sty}{\from{kvdefinekeys.dtx}{package}}%
  \usedir{doc/latex/oberdiek/test}%
  \file{kvdefinekeys-test1.tex}{\from{kvdefinekeys.dtx}{test1}}%
  \nopreamble
  \nopostamble
  \usedir{source/latex/oberdiek/catalogue}%
  \file{kvdefinekeys.xml}{\from{kvdefinekeys.dtx}{catalogue}}%
}

\catcode32=13\relax% active space
\let =\space%
\Msg{************************************************************************}
\Msg{*}
\Msg{* To finish the installation you have to move the following}
\Msg{* file into a directory searched by TeX:}
\Msg{*}
\Msg{*     kvdefinekeys.sty}
\Msg{*}
\Msg{* To produce the documentation run the file `kvdefinekeys.drv'}
\Msg{* through LaTeX.}
\Msg{*}
\Msg{* Happy TeXing!}
\Msg{*}
\Msg{************************************************************************}

\endbatchfile
%</install>
%<*ignore>
\fi
%</ignore>
%<*driver>
\NeedsTeXFormat{LaTeX2e}
\ProvidesFile{kvdefinekeys.drv}%
  [2016/05/16 v1.4 Define keys (HO)]%
\documentclass{ltxdoc}
\usepackage{holtxdoc}[2011/11/22]
\begin{document}
  \DocInput{kvdefinekeys.dtx}%
\end{document}
%</driver>
% \fi
%
%
% \CharacterTable
%  {Upper-case    \A\B\C\D\E\F\G\H\I\J\K\L\M\N\O\P\Q\R\S\T\U\V\W\X\Y\Z
%   Lower-case    \a\b\c\d\e\f\g\h\i\j\k\l\m\n\o\p\q\r\s\t\u\v\w\x\y\z
%   Digits        \0\1\2\3\4\5\6\7\8\9
%   Exclamation   \!     Double quote  \"     Hash (number) \#
%   Dollar        \$     Percent       \%     Ampersand     \&
%   Acute accent  \'     Left paren    \(     Right paren   \)
%   Asterisk      \*     Plus          \+     Comma         \,
%   Minus         \-     Point         \.     Solidus       \/
%   Colon         \:     Semicolon     \;     Less than     \<
%   Equals        \=     Greater than  \>     Question mark \?
%   Commercial at \@     Left bracket  \[     Backslash     \\
%   Right bracket \]     Circumflex    \^     Underscore    \_
%   Grave accent  \`     Left brace    \{     Vertical bar  \|
%   Right brace   \}     Tilde         \~}
%
% \GetFileInfo{kvdefinekeys.drv}
%
% \title{The \xpackage{kvdefinekeys} package}
% \date{2016/05/16 v1.4}
% \author{Heiko Oberdiek\thanks
% {Please report any issues at https://github.com/ho-tex/oberdiek/issues}\\
% \xemail{heiko.oberdiek at googlemail.com}}
%
% \maketitle
%
% \begin{abstract}
% Package \xpackage{kvdefinekeys} provides \cs{kv@define@key} to define
% keys the same way as \xpackage{keyval}'s \cs{define@key}. However, it
% works also using \iniTeX.
% \end{abstract}
%
% \tableofcontents
%
% \def\M#1{\texttt{\{}\meta{#1}\texttt{\}}}
%
% \section{Documentation}
%
% \subsection{Motivation}
%
% \cs{kvsetkeys} serves as replacement for \xpackage{keyval}'s
% \cs{setkeys}. This package adds macros to define keys, closing
% the gap \cs{kvsetkeys} leaves.
%
% \begin{declcs}{kv@define@key}\,\M{family}\,\M{key}\,[\meta{default}]\,^^A
%   \M{definition}
% \end{declcs}
% Macro \cs{kv@define@key} reimplements \xpackage{keyval}'s
% \cs{define@key}. Differences to the original:
% \begin{itemize}
% \item The defined keys also allow \cs{par} inside values.
% \item Shorthands of package \xpackage{babel} are supported in
%   family and key names.
% \item Macro \cs{kv@define@key} is made robust if
%   \hologo{eTeX}'s \cs{protected} or \hologo{LaTeX}'s
%   \cs{DeclareRobustCommand} are found.
% \end{itemize}
%
% \StopEventually{
% }
%
% \section{Implementation}
%
% \subsection{Identification}
%
%    \begin{macrocode}
%<*package>
%    \end{macrocode}
%    Reload check, especially if the package is not used with \LaTeX.
%    \begin{macrocode}
\begingroup\catcode61\catcode48\catcode32=10\relax%
  \catcode13=5 % ^^M
  \endlinechar=13 %
  \catcode35=6 % #
  \catcode39=12 % '
  \catcode44=12 % ,
  \catcode45=12 % -
  \catcode46=12 % .
  \catcode58=12 % :
  \catcode64=11 % @
  \catcode123=1 % {
  \catcode125=2 % }
  \expandafter\let\expandafter\x\csname ver@kvdefinekeys.sty\endcsname
  \ifx\x\relax % plain-TeX, first loading
  \else
    \def\empty{}%
    \ifx\x\empty % LaTeX, first loading,
      % variable is initialized, but \ProvidesPackage not yet seen
    \else
      \expandafter\ifx\csname PackageInfo\endcsname\relax
        \def\x#1#2{%
          \immediate\write-1{Package #1 Info: #2.}%
        }%
      \else
        \def\x#1#2{\PackageInfo{#1}{#2, stopped}}%
      \fi
      \x{kvdefinekeys}{The package is already loaded}%
      \aftergroup\endinput
    \fi
  \fi
\endgroup%
%    \end{macrocode}
%    Package identification:
%    \begin{macrocode}
\begingroup\catcode61\catcode48\catcode32=10\relax%
  \catcode13=5 % ^^M
  \endlinechar=13 %
  \catcode35=6 % #
  \catcode39=12 % '
  \catcode40=12 % (
  \catcode41=12 % )
  \catcode44=12 % ,
  \catcode45=12 % -
  \catcode46=12 % .
  \catcode47=12 % /
  \catcode58=12 % :
  \catcode64=11 % @
  \catcode91=12 % [
  \catcode93=12 % ]
  \catcode123=1 % {
  \catcode125=2 % }
  \expandafter\ifx\csname ProvidesPackage\endcsname\relax
    \def\x#1#2#3[#4]{\endgroup
      \immediate\write-1{Package: #3 #4}%
      \xdef#1{#4}%
    }%
  \else
    \def\x#1#2[#3]{\endgroup
      #2[{#3}]%
      \ifx#1\@undefined
        \xdef#1{#3}%
      \fi
      \ifx#1\relax
        \xdef#1{#3}%
      \fi
    }%
  \fi
\expandafter\x\csname ver@kvdefinekeys.sty\endcsname
\ProvidesPackage{kvdefinekeys}%
  [2016/05/16 v1.4 Define keys (HO)]%
%    \end{macrocode}
%
%    \begin{macrocode}
\begingroup\catcode61\catcode48\catcode32=10\relax%
  \catcode13=5 % ^^M
  \endlinechar=13 %
  \catcode123=1 % {
  \catcode125=2 % }
  \catcode64=11 % @
  \def\x{\endgroup
    \expandafter\edef\csname KVD@AtEnd\endcsname{%
      \endlinechar=\the\endlinechar\relax
      \catcode13=\the\catcode13\relax
      \catcode32=\the\catcode32\relax
      \catcode35=\the\catcode35\relax
      \catcode61=\the\catcode61\relax
      \catcode64=\the\catcode64\relax
      \catcode123=\the\catcode123\relax
      \catcode125=\the\catcode125\relax
    }%
  }%
\x\catcode61\catcode48\catcode32=10\relax%
\catcode13=5 % ^^M
\endlinechar=13 %
\catcode35=6 % #
\catcode64=11 % @
\catcode123=1 % {
\catcode125=2 % }
\def\TMP@EnsureCode#1#2{%
  \edef\KVD@AtEnd{%
    \KVD@AtEnd
    \catcode#1=\the\catcode#1\relax
  }%
  \catcode#1=#2\relax
}
\TMP@EnsureCode{42}{12}% *
\TMP@EnsureCode{46}{12}% .
\TMP@EnsureCode{47}{12}% /
\TMP@EnsureCode{91}{12}% [
\TMP@EnsureCode{93}{12}% ]
\edef\KVD@AtEnd{\KVD@AtEnd\noexpand\endinput}
%    \end{macrocode}
%
% \subsection{Package loading}
%
%    \begin{macrocode}
\begingroup\expandafter\expandafter\expandafter\endgroup
\expandafter\ifx\csname RequirePackage\endcsname\relax
  \def\TMP@RequirePackage#1[#2]{%
    \begingroup\expandafter\expandafter\expandafter\endgroup
    \expandafter\ifx\csname ver@#1.sty\endcsname\relax
      \input #1.sty\relax
    \fi
  }%
  \TMP@RequirePackage{ltxcmds}[2010/03/01]%
\else
  \RequirePackage{ltxcmds}[2010/03/01]%
\fi
%    \end{macrocode}
%
%
% \subsection{Provide key defining macro}
%
%    \begin{macro}{\kv@define@key}
%    \begin{macrocode}
\ltx@IfUndefined{protected}{%
  \ltx@IfUndefined{DeclareRobustCommand}{%
    \def\kv@define@key#1#2%
  }{%
    \DeclareRobustCommand*{\kv@define@key}[2]%
  }%
}{%
  \protected\def\kv@define@key#1#2%
}%
{%
  \begingroup
    \csname @safe@activestrue\endcsname
    \let\ifincsname\iftrue
    \edef\KVD@temp{\endgroup
      \noexpand\KVD@DefineKey{#1}{#2}%
    }%
  \KVD@temp
}
%    \end{macrocode}
%    \end{macro}
%    \begin{macro}{\KVD@DefineKey}
%    \begin{macrocode}
\def\KVD@DefineKey#1#2{%
  \ltx@ifnextchar[{%
    \KVD@DefineKeyWithDefault{#1}{#2}%
  }{%
    \long\expandafter\def\csname KV@#1@#2\endcsname##1%
  }%
}
%    \end{macrocode}
%    \end{macro}
%    \begin{macro}{\KVD@DefineKeyWithDefault}
%    \begin{macrocode}
\long\def\KVD@DefineKeyWithDefault#1#2[#3]{%
  \expandafter\def\csname KV@#1@#2@default\expandafter\endcsname
  \expandafter{%
    \csname KV@#1@#2\endcsname{#3}%
  }%
  \long\expandafter\def\csname KV@#1@#2\endcsname##1%
}
%    \end{macrocode}
%    \end{macro}
%
%    \begin{macrocode}
\KVD@AtEnd%
%</package>
%    \end{macrocode}
%
% \section{Test}
%
% \subsection{Catcode checks for loading}
%
%    \begin{macrocode}
%<*test1>
%    \end{macrocode}
%    \begin{macrocode}
\catcode`\{=1 %
\catcode`\}=2 %
\catcode`\#=6 %
\catcode`\@=11 %
\expandafter\ifx\csname count@\endcsname\relax
  \countdef\count@=255 %
\fi
\expandafter\ifx\csname @gobble\endcsname\relax
  \long\def\@gobble#1{}%
\fi
\expandafter\ifx\csname @firstofone\endcsname\relax
  \long\def\@firstofone#1{#1}%
\fi
\expandafter\ifx\csname loop\endcsname\relax
  \expandafter\@firstofone
\else
  \expandafter\@gobble
\fi
{%
  \def\loop#1\repeat{%
    \def\body{#1}%
    \iterate
  }%
  \def\iterate{%
    \body
      \let\next\iterate
    \else
      \let\next\relax
    \fi
    \next
  }%
  \let\repeat=\fi
}%
\def\RestoreCatcodes{}
\count@=0 %
\loop
  \edef\RestoreCatcodes{%
    \RestoreCatcodes
    \catcode\the\count@=\the\catcode\count@\relax
  }%
\ifnum\count@<255 %
  \advance\count@ 1 %
\repeat

\def\RangeCatcodeInvalid#1#2{%
  \count@=#1\relax
  \loop
    \catcode\count@=15 %
  \ifnum\count@<#2\relax
    \advance\count@ 1 %
  \repeat
}
\def\RangeCatcodeCheck#1#2#3{%
  \count@=#1\relax
  \loop
    \ifnum#3=\catcode\count@
    \else
      \errmessage{%
        Character \the\count@\space
        with wrong catcode \the\catcode\count@\space
        instead of \number#3%
      }%
    \fi
  \ifnum\count@<#2\relax
    \advance\count@ 1 %
  \repeat
}
\def\space{ }
\expandafter\ifx\csname LoadCommand\endcsname\relax
  \def\LoadCommand{\input kvdefinekeys.sty\relax}%
\fi
\def\Test{%
  \RangeCatcodeInvalid{0}{47}%
  \RangeCatcodeInvalid{58}{64}%
  \RangeCatcodeInvalid{91}{96}%
  \RangeCatcodeInvalid{123}{255}%
  \catcode`\@=12 %
  \catcode`\\=0 %
  \catcode`\%=14 %
  \LoadCommand
  \RangeCatcodeCheck{0}{36}{15}%
  \RangeCatcodeCheck{37}{37}{14}%
  \RangeCatcodeCheck{38}{47}{15}%
  \RangeCatcodeCheck{48}{57}{12}%
  \RangeCatcodeCheck{58}{63}{15}%
  \RangeCatcodeCheck{64}{64}{12}%
  \RangeCatcodeCheck{65}{90}{11}%
  \RangeCatcodeCheck{91}{91}{15}%
  \RangeCatcodeCheck{92}{92}{0}%
  \RangeCatcodeCheck{93}{96}{15}%
  \RangeCatcodeCheck{97}{122}{11}%
  \RangeCatcodeCheck{123}{255}{15}%
  \RestoreCatcodes
}
\Test
\csname @@end\endcsname
\end
%    \end{macrocode}
%    \begin{macrocode}
%</test1>
%    \end{macrocode}
%
% \section{Installation}
%
% \subsection{Download}
%
% \paragraph{Package.} This package is available on
% CTAN\footnote{\url{http://ctan.org/pkg/kvdefinekeys}}:
% \begin{description}
% \item[\CTAN{macros/latex/contrib/oberdiek/kvdefinekeys.dtx}] The source file.
% \item[\CTAN{macros/latex/contrib/oberdiek/kvdefinekeys.pdf}] Documentation.
% \end{description}
%
%
% \paragraph{Bundle.} All the packages of the bundle `oberdiek'
% are also available in a TDS compliant ZIP archive. There
% the packages are already unpacked and the documentation files
% are generated. The files and directories obey the TDS standard.
% \begin{description}
% \item[\CTAN{install/macros/latex/contrib/oberdiek.tds.zip}]
% \end{description}
% \emph{TDS} refers to the standard ``A Directory Structure
% for \TeX\ Files'' (\CTAN{tds/tds.pdf}). Directories
% with \xfile{texmf} in their name are usually organized this way.
%
% \subsection{Bundle installation}
%
% \paragraph{Unpacking.} Unpack the \xfile{oberdiek.tds.zip} in the
% TDS tree (also known as \xfile{texmf} tree) of your choice.
% Example (linux):
% \begin{quote}
%   |unzip oberdiek.tds.zip -d ~/texmf|
% \end{quote}
%
% \paragraph{Script installation.}
% Check the directory \xfile{TDS:scripts/oberdiek/} for
% scripts that need further installation steps.
% Package \xpackage{attachfile2} comes with the Perl script
% \xfile{pdfatfi.pl} that should be installed in such a way
% that it can be called as \texttt{pdfatfi}.
% Example (linux):
% \begin{quote}
%   |chmod +x scripts/oberdiek/pdfatfi.pl|\\
%   |cp scripts/oberdiek/pdfatfi.pl /usr/local/bin/|
% \end{quote}
%
% \subsection{Package installation}
%
% \paragraph{Unpacking.} The \xfile{.dtx} file is a self-extracting
% \docstrip\ archive. The files are extracted by running the
% \xfile{.dtx} through \plainTeX:
% \begin{quote}
%   \verb|tex kvdefinekeys.dtx|
% \end{quote}
%
% \paragraph{TDS.} Now the different files must be moved into
% the different directories in your installation TDS tree
% (also known as \xfile{texmf} tree):
% \begin{quote}
% \def\t{^^A
% \begin{tabular}{@{}>{\ttfamily}l@{ $\rightarrow$ }>{\ttfamily}l@{}}
%   kvdefinekeys.sty & tex/generic/oberdiek/kvdefinekeys.sty\\
%   kvdefinekeys.pdf & doc/latex/oberdiek/kvdefinekeys.pdf\\
%   test/kvdefinekeys-test1.tex & doc/latex/oberdiek/test/kvdefinekeys-test1.tex\\
%   kvdefinekeys.dtx & source/latex/oberdiek/kvdefinekeys.dtx\\
% \end{tabular}^^A
% }^^A
% \sbox0{\t}^^A
% \ifdim\wd0>\linewidth
%   \begingroup
%     \advance\linewidth by\leftmargin
%     \advance\linewidth by\rightmargin
%   \edef\x{\endgroup
%     \def\noexpand\lw{\the\linewidth}^^A
%   }\x
%   \def\lwbox{^^A
%     \leavevmode
%     \hbox to \linewidth{^^A
%       \kern-\leftmargin\relax
%       \hss
%       \usebox0
%       \hss
%       \kern-\rightmargin\relax
%     }^^A
%   }^^A
%   \ifdim\wd0>\lw
%     \sbox0{\small\t}^^A
%     \ifdim\wd0>\linewidth
%       \ifdim\wd0>\lw
%         \sbox0{\footnotesize\t}^^A
%         \ifdim\wd0>\linewidth
%           \ifdim\wd0>\lw
%             \sbox0{\scriptsize\t}^^A
%             \ifdim\wd0>\linewidth
%               \ifdim\wd0>\lw
%                 \sbox0{\tiny\t}^^A
%                 \ifdim\wd0>\linewidth
%                   \lwbox
%                 \else
%                   \usebox0
%                 \fi
%               \else
%                 \lwbox
%               \fi
%             \else
%               \usebox0
%             \fi
%           \else
%             \lwbox
%           \fi
%         \else
%           \usebox0
%         \fi
%       \else
%         \lwbox
%       \fi
%     \else
%       \usebox0
%     \fi
%   \else
%     \lwbox
%   \fi
% \else
%   \usebox0
% \fi
% \end{quote}
% If you have a \xfile{docstrip.cfg} that configures and enables \docstrip's
% TDS installing feature, then some files can already be in the right
% place, see the documentation of \docstrip.
%
% \subsection{Refresh file name databases}
%
% If your \TeX~distribution
% (\teTeX, \mikTeX, \dots) relies on file name databases, you must refresh
% these. For example, \teTeX\ users run \verb|texhash| or
% \verb|mktexlsr|.
%
% \subsection{Some details for the interested}
%
% \paragraph{Attached source.}
%
% The PDF documentation on CTAN also includes the
% \xfile{.dtx} source file. It can be extracted by
% AcrobatReader 6 or higher. Another option is \textsf{pdftk},
% e.g. unpack the file into the current directory:
% \begin{quote}
%   \verb|pdftk kvdefinekeys.pdf unpack_files output .|
% \end{quote}
%
% \paragraph{Unpacking with \LaTeX.}
% The \xfile{.dtx} chooses its action depending on the format:
% \begin{description}
% \item[\plainTeX:] Run \docstrip\ and extract the files.
% \item[\LaTeX:] Generate the documentation.
% \end{description}
% If you insist on using \LaTeX\ for \docstrip\ (really,
% \docstrip\ does not need \LaTeX), then inform the autodetect routine
% about your intention:
% \begin{quote}
%   \verb|latex \let\install=y\input{kvdefinekeys.dtx}|
% \end{quote}
% Do not forget to quote the argument according to the demands
% of your shell.
%
% \paragraph{Generating the documentation.}
% You can use both the \xfile{.dtx} or the \xfile{.drv} to generate
% the documentation. The process can be configured by the
% configuration file \xfile{ltxdoc.cfg}. For instance, put this
% line into this file, if you want to have A4 as paper format:
% \begin{quote}
%   \verb|\PassOptionsToClass{a4paper}{article}|
% \end{quote}
% An example follows how to generate the
% documentation with pdf\LaTeX:
% \begin{quote}
%\begin{verbatim}
%pdflatex kvdefinekeys.dtx
%makeindex -s gind.ist kvdefinekeys.idx
%pdflatex kvdefinekeys.dtx
%makeindex -s gind.ist kvdefinekeys.idx
%pdflatex kvdefinekeys.dtx
%\end{verbatim}
% \end{quote}
%
% \section{Catalogue}
%
% The following XML file can be used as source for the
% \href{http://mirror.ctan.org/help/Catalogue/catalogue.html}{\TeX\ Catalogue}.
% The elements \texttt{caption} and \texttt{description} are imported
% from the original XML file from the Catalogue.
% The name of the XML file in the Catalogue is \xfile{kvdefinekeys.xml}.
%    \begin{macrocode}
%<*catalogue>
<?xml version='1.0' encoding='us-ascii'?>
<!DOCTYPE entry SYSTEM 'catalogue.dtd'>
<entry datestamp='$Date$' modifier='$Author$' id='kvdefinekeys'>
  <name>kvdefinekeys</name>
  <caption>Define keys for use in the kvsetkeys package.</caption>
  <authorref id='auth:oberdiek'/>
  <copyright owner='Heiko Oberdiek' year='2010,2011'/>
  <license type='lppl1.3'/>
  <version number='1.4'/>
  <description>
    The package provides a macro <tt>\kv@define@key</tt> (analogous to
    <xref refid='keyval'>keyval&#x2019;s</xref> <tt>\define@key</tt>, to
    define keys for use by <xref refid='kvsetkeys'>kvsetkeys</xref>.
    <p/>
    The package is part of the <xref refid='oberdiek'>oberdiek</xref>
    bundle.
  </description>
  <documentation details='Package documentation'
      href='ctan:/macros/latex/contrib/oberdiek/kvdefinekeys.pdf'/>
  <ctan file='true' path='/macros/latex/contrib/oberdiek/kvdefinekeys.dtx'/>
  <miktex location='oberdiek'/>
  <texlive location='oberdiek'/>
  <install path='/macros/latex/contrib/oberdiek/oberdiek.tds.zip'/>
</entry>
%</catalogue>
%    \end{macrocode}
%
% \begin{thebibliography}{9}
% \bibitem{keyval}
%   David Carlisle:
%   \textit{The \xpackage{keyval} package};
%   1999/03/16 v1.13;
%   \CTAN{macros/latex/required/graphics/keyval.dtx}.
%
% \end{thebibliography}
%
% \begin{History}
%   \begin{Version}{2010/03/01 v1.0}
%   \item
%     First version.
%   \end{Version}
%   \begin{Version}{2010/08/19 v1.1}
%   \item
%     Documentation fix, no code change.
%   \end{Version}
%   \begin{Version}{2011/01/30 v1.2}
%   \item
%     Already loaded package files are not input in \hologo{plainTeX}.
%   \end{Version}
%   \begin{Version}{2011/04/07 v1.3}
%   \item
%     Support for package \xpackage{babel}'s shorthands added.
%   \item
%     \cs{kv@define@key} is made robust if available.
%   \end{Version}
%   \begin{Version}{2016/05/16 v1.4}
%   \item
%     Documentation updates.
%   \end{Version}
% \end{History}
%
% \PrintIndex
%
% \Finale
\endinput

%        (quote the arguments according to the demands of your shell)
%
% Documentation:
%    (a) If kvdefinekeys.drv is present:
%           latex kvdefinekeys.drv
%    (b) Without kvdefinekeys.drv:
%           latex kvdefinekeys.dtx; ...
%    The class ltxdoc loads the configuration file ltxdoc.cfg
%    if available. Here you can specify further options, e.g.
%    use A4 as paper format:
%       \PassOptionsToClass{a4paper}{article}
%
%    Programm calls to get the documentation (example):
%       pdflatex kvdefinekeys.dtx
%       makeindex -s gind.ist kvdefinekeys.idx
%       pdflatex kvdefinekeys.dtx
%       makeindex -s gind.ist kvdefinekeys.idx
%       pdflatex kvdefinekeys.dtx
%
% Installation:
%    TDS:tex/generic/oberdiek/kvdefinekeys.sty
%    TDS:doc/latex/oberdiek/kvdefinekeys.pdf
%    TDS:doc/latex/oberdiek/test/kvdefinekeys-test1.tex
%    TDS:source/latex/oberdiek/kvdefinekeys.dtx
%
%<*ignore>
\begingroup
  \catcode123=1 %
  \catcode125=2 %
  \def\x{LaTeX2e}%
\expandafter\endgroup
\ifcase 0\ifx\install y1\fi\expandafter
         \ifx\csname processbatchFile\endcsname\relax\else1\fi
         \ifx\fmtname\x\else 1\fi\relax
\else\csname fi\endcsname
%</ignore>
%<*install>
\input docstrip.tex
\Msg{************************************************************************}
\Msg{* Installation}
\Msg{* Package: kvdefinekeys 2016/05/16 v1.4 Define keys (HO)}
\Msg{************************************************************************}

\keepsilent
\askforoverwritefalse

\let\MetaPrefix\relax
\preamble

This is a generated file.

Project: kvdefinekeys
Version: 2016/05/16 v1.4

Copyright (C) 2010, 2011 by
   Heiko Oberdiek <heiko.oberdiek at googlemail.com>

This work may be distributed and/or modified under the
conditions of the LaTeX Project Public License, either
version 1.3c of this license or (at your option) any later
version. This version of this license is in
   http://www.latex-project.org/lppl/lppl-1-3c.txt
and the latest version of this license is in
   http://www.latex-project.org/lppl.txt
and version 1.3 or later is part of all distributions of
LaTeX version 2005/12/01 or later.

This work has the LPPL maintenance status "maintained".

This Current Maintainer of this work is Heiko Oberdiek.

The Base Interpreter refers to any `TeX-Format',
because some files are installed in TDS:tex/generic//.

This work consists of the main source file kvdefinekeys.dtx
and the derived files
   kvdefinekeys.sty, kvdefinekeys.pdf, kvdefinekeys.ins, kvdefinekeys.drv,
   kvdefinekeys-test1.tex.

\endpreamble
\let\MetaPrefix\DoubleperCent

\generate{%
  \file{kvdefinekeys.ins}{\from{kvdefinekeys.dtx}{install}}%
  \file{kvdefinekeys.drv}{\from{kvdefinekeys.dtx}{driver}}%
  \usedir{tex/generic/oberdiek}%
  \file{kvdefinekeys.sty}{\from{kvdefinekeys.dtx}{package}}%
  \usedir{doc/latex/oberdiek/test}%
  \file{kvdefinekeys-test1.tex}{\from{kvdefinekeys.dtx}{test1}}%
  \nopreamble
  \nopostamble
  \usedir{source/latex/oberdiek/catalogue}%
  \file{kvdefinekeys.xml}{\from{kvdefinekeys.dtx}{catalogue}}%
}

\catcode32=13\relax% active space
\let =\space%
\Msg{************************************************************************}
\Msg{*}
\Msg{* To finish the installation you have to move the following}
\Msg{* file into a directory searched by TeX:}
\Msg{*}
\Msg{*     kvdefinekeys.sty}
\Msg{*}
\Msg{* To produce the documentation run the file `kvdefinekeys.drv'}
\Msg{* through LaTeX.}
\Msg{*}
\Msg{* Happy TeXing!}
\Msg{*}
\Msg{************************************************************************}

\endbatchfile
%</install>
%<*ignore>
\fi
%</ignore>
%<*driver>
\NeedsTeXFormat{LaTeX2e}
\ProvidesFile{kvdefinekeys.drv}%
  [2016/05/16 v1.4 Define keys (HO)]%
\documentclass{ltxdoc}
\usepackage{holtxdoc}[2011/11/22]
\begin{document}
  \DocInput{kvdefinekeys.dtx}%
\end{document}
%</driver>
% \fi
%
%
% \CharacterTable
%  {Upper-case    \A\B\C\D\E\F\G\H\I\J\K\L\M\N\O\P\Q\R\S\T\U\V\W\X\Y\Z
%   Lower-case    \a\b\c\d\e\f\g\h\i\j\k\l\m\n\o\p\q\r\s\t\u\v\w\x\y\z
%   Digits        \0\1\2\3\4\5\6\7\8\9
%   Exclamation   \!     Double quote  \"     Hash (number) \#
%   Dollar        \$     Percent       \%     Ampersand     \&
%   Acute accent  \'     Left paren    \(     Right paren   \)
%   Asterisk      \*     Plus          \+     Comma         \,
%   Minus         \-     Point         \.     Solidus       \/
%   Colon         \:     Semicolon     \;     Less than     \<
%   Equals        \=     Greater than  \>     Question mark \?
%   Commercial at \@     Left bracket  \[     Backslash     \\
%   Right bracket \]     Circumflex    \^     Underscore    \_
%   Grave accent  \`     Left brace    \{     Vertical bar  \|
%   Right brace   \}     Tilde         \~}
%
% \GetFileInfo{kvdefinekeys.drv}
%
% \title{The \xpackage{kvdefinekeys} package}
% \date{2016/05/16 v1.4}
% \author{Heiko Oberdiek\thanks
% {Please report any issues at https://github.com/ho-tex/oberdiek/issues}\\
% \xemail{heiko.oberdiek at googlemail.com}}
%
% \maketitle
%
% \begin{abstract}
% Package \xpackage{kvdefinekeys} provides \cs{kv@define@key} to define
% keys the same way as \xpackage{keyval}'s \cs{define@key}. However, it
% works also using \iniTeX.
% \end{abstract}
%
% \tableofcontents
%
% \def\M#1{\texttt{\{}\meta{#1}\texttt{\}}}
%
% \section{Documentation}
%
% \subsection{Motivation}
%
% \cs{kvsetkeys} serves as replacement for \xpackage{keyval}'s
% \cs{setkeys}. This package adds macros to define keys, closing
% the gap \cs{kvsetkeys} leaves.
%
% \begin{declcs}{kv@define@key}\,\M{family}\,\M{key}\,[\meta{default}]\,^^A
%   \M{definition}
% \end{declcs}
% Macro \cs{kv@define@key} reimplements \xpackage{keyval}'s
% \cs{define@key}. Differences to the original:
% \begin{itemize}
% \item The defined keys also allow \cs{par} inside values.
% \item Shorthands of package \xpackage{babel} are supported in
%   family and key names.
% \item Macro \cs{kv@define@key} is made robust if
%   \hologo{eTeX}'s \cs{protected} or \hologo{LaTeX}'s
%   \cs{DeclareRobustCommand} are found.
% \end{itemize}
%
% \StopEventually{
% }
%
% \section{Implementation}
%
% \subsection{Identification}
%
%    \begin{macrocode}
%<*package>
%    \end{macrocode}
%    Reload check, especially if the package is not used with \LaTeX.
%    \begin{macrocode}
\begingroup\catcode61\catcode48\catcode32=10\relax%
  \catcode13=5 % ^^M
  \endlinechar=13 %
  \catcode35=6 % #
  \catcode39=12 % '
  \catcode44=12 % ,
  \catcode45=12 % -
  \catcode46=12 % .
  \catcode58=12 % :
  \catcode64=11 % @
  \catcode123=1 % {
  \catcode125=2 % }
  \expandafter\let\expandafter\x\csname ver@kvdefinekeys.sty\endcsname
  \ifx\x\relax % plain-TeX, first loading
  \else
    \def\empty{}%
    \ifx\x\empty % LaTeX, first loading,
      % variable is initialized, but \ProvidesPackage not yet seen
    \else
      \expandafter\ifx\csname PackageInfo\endcsname\relax
        \def\x#1#2{%
          \immediate\write-1{Package #1 Info: #2.}%
        }%
      \else
        \def\x#1#2{\PackageInfo{#1}{#2, stopped}}%
      \fi
      \x{kvdefinekeys}{The package is already loaded}%
      \aftergroup\endinput
    \fi
  \fi
\endgroup%
%    \end{macrocode}
%    Package identification:
%    \begin{macrocode}
\begingroup\catcode61\catcode48\catcode32=10\relax%
  \catcode13=5 % ^^M
  \endlinechar=13 %
  \catcode35=6 % #
  \catcode39=12 % '
  \catcode40=12 % (
  \catcode41=12 % )
  \catcode44=12 % ,
  \catcode45=12 % -
  \catcode46=12 % .
  \catcode47=12 % /
  \catcode58=12 % :
  \catcode64=11 % @
  \catcode91=12 % [
  \catcode93=12 % ]
  \catcode123=1 % {
  \catcode125=2 % }
  \expandafter\ifx\csname ProvidesPackage\endcsname\relax
    \def\x#1#2#3[#4]{\endgroup
      \immediate\write-1{Package: #3 #4}%
      \xdef#1{#4}%
    }%
  \else
    \def\x#1#2[#3]{\endgroup
      #2[{#3}]%
      \ifx#1\@undefined
        \xdef#1{#3}%
      \fi
      \ifx#1\relax
        \xdef#1{#3}%
      \fi
    }%
  \fi
\expandafter\x\csname ver@kvdefinekeys.sty\endcsname
\ProvidesPackage{kvdefinekeys}%
  [2016/05/16 v1.4 Define keys (HO)]%
%    \end{macrocode}
%
%    \begin{macrocode}
\begingroup\catcode61\catcode48\catcode32=10\relax%
  \catcode13=5 % ^^M
  \endlinechar=13 %
  \catcode123=1 % {
  \catcode125=2 % }
  \catcode64=11 % @
  \def\x{\endgroup
    \expandafter\edef\csname KVD@AtEnd\endcsname{%
      \endlinechar=\the\endlinechar\relax
      \catcode13=\the\catcode13\relax
      \catcode32=\the\catcode32\relax
      \catcode35=\the\catcode35\relax
      \catcode61=\the\catcode61\relax
      \catcode64=\the\catcode64\relax
      \catcode123=\the\catcode123\relax
      \catcode125=\the\catcode125\relax
    }%
  }%
\x\catcode61\catcode48\catcode32=10\relax%
\catcode13=5 % ^^M
\endlinechar=13 %
\catcode35=6 % #
\catcode64=11 % @
\catcode123=1 % {
\catcode125=2 % }
\def\TMP@EnsureCode#1#2{%
  \edef\KVD@AtEnd{%
    \KVD@AtEnd
    \catcode#1=\the\catcode#1\relax
  }%
  \catcode#1=#2\relax
}
\TMP@EnsureCode{42}{12}% *
\TMP@EnsureCode{46}{12}% .
\TMP@EnsureCode{47}{12}% /
\TMP@EnsureCode{91}{12}% [
\TMP@EnsureCode{93}{12}% ]
\edef\KVD@AtEnd{\KVD@AtEnd\noexpand\endinput}
%    \end{macrocode}
%
% \subsection{Package loading}
%
%    \begin{macrocode}
\begingroup\expandafter\expandafter\expandafter\endgroup
\expandafter\ifx\csname RequirePackage\endcsname\relax
  \def\TMP@RequirePackage#1[#2]{%
    \begingroup\expandafter\expandafter\expandafter\endgroup
    \expandafter\ifx\csname ver@#1.sty\endcsname\relax
      \input #1.sty\relax
    \fi
  }%
  \TMP@RequirePackage{ltxcmds}[2010/03/01]%
\else
  \RequirePackage{ltxcmds}[2010/03/01]%
\fi
%    \end{macrocode}
%
%
% \subsection{Provide key defining macro}
%
%    \begin{macro}{\kv@define@key}
%    \begin{macrocode}
\ltx@IfUndefined{protected}{%
  \ltx@IfUndefined{DeclareRobustCommand}{%
    \def\kv@define@key#1#2%
  }{%
    \DeclareRobustCommand*{\kv@define@key}[2]%
  }%
}{%
  \protected\def\kv@define@key#1#2%
}%
{%
  \begingroup
    \csname @safe@activestrue\endcsname
    \let\ifincsname\iftrue
    \edef\KVD@temp{\endgroup
      \noexpand\KVD@DefineKey{#1}{#2}%
    }%
  \KVD@temp
}
%    \end{macrocode}
%    \end{macro}
%    \begin{macro}{\KVD@DefineKey}
%    \begin{macrocode}
\def\KVD@DefineKey#1#2{%
  \ltx@ifnextchar[{%
    \KVD@DefineKeyWithDefault{#1}{#2}%
  }{%
    \long\expandafter\def\csname KV@#1@#2\endcsname##1%
  }%
}
%    \end{macrocode}
%    \end{macro}
%    \begin{macro}{\KVD@DefineKeyWithDefault}
%    \begin{macrocode}
\long\def\KVD@DefineKeyWithDefault#1#2[#3]{%
  \expandafter\def\csname KV@#1@#2@default\expandafter\endcsname
  \expandafter{%
    \csname KV@#1@#2\endcsname{#3}%
  }%
  \long\expandafter\def\csname KV@#1@#2\endcsname##1%
}
%    \end{macrocode}
%    \end{macro}
%
%    \begin{macrocode}
\KVD@AtEnd%
%</package>
%    \end{macrocode}
%
% \section{Test}
%
% \subsection{Catcode checks for loading}
%
%    \begin{macrocode}
%<*test1>
%    \end{macrocode}
%    \begin{macrocode}
\catcode`\{=1 %
\catcode`\}=2 %
\catcode`\#=6 %
\catcode`\@=11 %
\expandafter\ifx\csname count@\endcsname\relax
  \countdef\count@=255 %
\fi
\expandafter\ifx\csname @gobble\endcsname\relax
  \long\def\@gobble#1{}%
\fi
\expandafter\ifx\csname @firstofone\endcsname\relax
  \long\def\@firstofone#1{#1}%
\fi
\expandafter\ifx\csname loop\endcsname\relax
  \expandafter\@firstofone
\else
  \expandafter\@gobble
\fi
{%
  \def\loop#1\repeat{%
    \def\body{#1}%
    \iterate
  }%
  \def\iterate{%
    \body
      \let\next\iterate
    \else
      \let\next\relax
    \fi
    \next
  }%
  \let\repeat=\fi
}%
\def\RestoreCatcodes{}
\count@=0 %
\loop
  \edef\RestoreCatcodes{%
    \RestoreCatcodes
    \catcode\the\count@=\the\catcode\count@\relax
  }%
\ifnum\count@<255 %
  \advance\count@ 1 %
\repeat

\def\RangeCatcodeInvalid#1#2{%
  \count@=#1\relax
  \loop
    \catcode\count@=15 %
  \ifnum\count@<#2\relax
    \advance\count@ 1 %
  \repeat
}
\def\RangeCatcodeCheck#1#2#3{%
  \count@=#1\relax
  \loop
    \ifnum#3=\catcode\count@
    \else
      \errmessage{%
        Character \the\count@\space
        with wrong catcode \the\catcode\count@\space
        instead of \number#3%
      }%
    \fi
  \ifnum\count@<#2\relax
    \advance\count@ 1 %
  \repeat
}
\def\space{ }
\expandafter\ifx\csname LoadCommand\endcsname\relax
  \def\LoadCommand{\input kvdefinekeys.sty\relax}%
\fi
\def\Test{%
  \RangeCatcodeInvalid{0}{47}%
  \RangeCatcodeInvalid{58}{64}%
  \RangeCatcodeInvalid{91}{96}%
  \RangeCatcodeInvalid{123}{255}%
  \catcode`\@=12 %
  \catcode`\\=0 %
  \catcode`\%=14 %
  \LoadCommand
  \RangeCatcodeCheck{0}{36}{15}%
  \RangeCatcodeCheck{37}{37}{14}%
  \RangeCatcodeCheck{38}{47}{15}%
  \RangeCatcodeCheck{48}{57}{12}%
  \RangeCatcodeCheck{58}{63}{15}%
  \RangeCatcodeCheck{64}{64}{12}%
  \RangeCatcodeCheck{65}{90}{11}%
  \RangeCatcodeCheck{91}{91}{15}%
  \RangeCatcodeCheck{92}{92}{0}%
  \RangeCatcodeCheck{93}{96}{15}%
  \RangeCatcodeCheck{97}{122}{11}%
  \RangeCatcodeCheck{123}{255}{15}%
  \RestoreCatcodes
}
\Test
\csname @@end\endcsname
\end
%    \end{macrocode}
%    \begin{macrocode}
%</test1>
%    \end{macrocode}
%
% \section{Installation}
%
% \subsection{Download}
%
% \paragraph{Package.} This package is available on
% CTAN\footnote{\url{http://ctan.org/pkg/kvdefinekeys}}:
% \begin{description}
% \item[\CTAN{macros/latex/contrib/oberdiek/kvdefinekeys.dtx}] The source file.
% \item[\CTAN{macros/latex/contrib/oberdiek/kvdefinekeys.pdf}] Documentation.
% \end{description}
%
%
% \paragraph{Bundle.} All the packages of the bundle `oberdiek'
% are also available in a TDS compliant ZIP archive. There
% the packages are already unpacked and the documentation files
% are generated. The files and directories obey the TDS standard.
% \begin{description}
% \item[\CTAN{install/macros/latex/contrib/oberdiek.tds.zip}]
% \end{description}
% \emph{TDS} refers to the standard ``A Directory Structure
% for \TeX\ Files'' (\CTAN{tds/tds.pdf}). Directories
% with \xfile{texmf} in their name are usually organized this way.
%
% \subsection{Bundle installation}
%
% \paragraph{Unpacking.} Unpack the \xfile{oberdiek.tds.zip} in the
% TDS tree (also known as \xfile{texmf} tree) of your choice.
% Example (linux):
% \begin{quote}
%   |unzip oberdiek.tds.zip -d ~/texmf|
% \end{quote}
%
% \paragraph{Script installation.}
% Check the directory \xfile{TDS:scripts/oberdiek/} for
% scripts that need further installation steps.
% Package \xpackage{attachfile2} comes with the Perl script
% \xfile{pdfatfi.pl} that should be installed in such a way
% that it can be called as \texttt{pdfatfi}.
% Example (linux):
% \begin{quote}
%   |chmod +x scripts/oberdiek/pdfatfi.pl|\\
%   |cp scripts/oberdiek/pdfatfi.pl /usr/local/bin/|
% \end{quote}
%
% \subsection{Package installation}
%
% \paragraph{Unpacking.} The \xfile{.dtx} file is a self-extracting
% \docstrip\ archive. The files are extracted by running the
% \xfile{.dtx} through \plainTeX:
% \begin{quote}
%   \verb|tex kvdefinekeys.dtx|
% \end{quote}
%
% \paragraph{TDS.} Now the different files must be moved into
% the different directories in your installation TDS tree
% (also known as \xfile{texmf} tree):
% \begin{quote}
% \def\t{^^A
% \begin{tabular}{@{}>{\ttfamily}l@{ $\rightarrow$ }>{\ttfamily}l@{}}
%   kvdefinekeys.sty & tex/generic/oberdiek/kvdefinekeys.sty\\
%   kvdefinekeys.pdf & doc/latex/oberdiek/kvdefinekeys.pdf\\
%   test/kvdefinekeys-test1.tex & doc/latex/oberdiek/test/kvdefinekeys-test1.tex\\
%   kvdefinekeys.dtx & source/latex/oberdiek/kvdefinekeys.dtx\\
% \end{tabular}^^A
% }^^A
% \sbox0{\t}^^A
% \ifdim\wd0>\linewidth
%   \begingroup
%     \advance\linewidth by\leftmargin
%     \advance\linewidth by\rightmargin
%   \edef\x{\endgroup
%     \def\noexpand\lw{\the\linewidth}^^A
%   }\x
%   \def\lwbox{^^A
%     \leavevmode
%     \hbox to \linewidth{^^A
%       \kern-\leftmargin\relax
%       \hss
%       \usebox0
%       \hss
%       \kern-\rightmargin\relax
%     }^^A
%   }^^A
%   \ifdim\wd0>\lw
%     \sbox0{\small\t}^^A
%     \ifdim\wd0>\linewidth
%       \ifdim\wd0>\lw
%         \sbox0{\footnotesize\t}^^A
%         \ifdim\wd0>\linewidth
%           \ifdim\wd0>\lw
%             \sbox0{\scriptsize\t}^^A
%             \ifdim\wd0>\linewidth
%               \ifdim\wd0>\lw
%                 \sbox0{\tiny\t}^^A
%                 \ifdim\wd0>\linewidth
%                   \lwbox
%                 \else
%                   \usebox0
%                 \fi
%               \else
%                 \lwbox
%               \fi
%             \else
%               \usebox0
%             \fi
%           \else
%             \lwbox
%           \fi
%         \else
%           \usebox0
%         \fi
%       \else
%         \lwbox
%       \fi
%     \else
%       \usebox0
%     \fi
%   \else
%     \lwbox
%   \fi
% \else
%   \usebox0
% \fi
% \end{quote}
% If you have a \xfile{docstrip.cfg} that configures and enables \docstrip's
% TDS installing feature, then some files can already be in the right
% place, see the documentation of \docstrip.
%
% \subsection{Refresh file name databases}
%
% If your \TeX~distribution
% (\teTeX, \mikTeX, \dots) relies on file name databases, you must refresh
% these. For example, \teTeX\ users run \verb|texhash| or
% \verb|mktexlsr|.
%
% \subsection{Some details for the interested}
%
% \paragraph{Attached source.}
%
% The PDF documentation on CTAN also includes the
% \xfile{.dtx} source file. It can be extracted by
% AcrobatReader 6 or higher. Another option is \textsf{pdftk},
% e.g. unpack the file into the current directory:
% \begin{quote}
%   \verb|pdftk kvdefinekeys.pdf unpack_files output .|
% \end{quote}
%
% \paragraph{Unpacking with \LaTeX.}
% The \xfile{.dtx} chooses its action depending on the format:
% \begin{description}
% \item[\plainTeX:] Run \docstrip\ and extract the files.
% \item[\LaTeX:] Generate the documentation.
% \end{description}
% If you insist on using \LaTeX\ for \docstrip\ (really,
% \docstrip\ does not need \LaTeX), then inform the autodetect routine
% about your intention:
% \begin{quote}
%   \verb|latex \let\install=y% \iffalse meta-comment
%
% File: kvdefinekeys.dtx
% Version: 2016/05/16 v1.4
% Info: Define keys
%
% Copyright (C) 2010, 2011 by
%    Heiko Oberdiek <heiko.oberdiek at googlemail.com>
%    2016
%    https://github.com/ho-tex/oberdiek/issues
%
% This work may be distributed and/or modified under the
% conditions of the LaTeX Project Public License, either
% version 1.3c of this license or (at your option) any later
% version. This version of this license is in
%    http://www.latex-project.org/lppl/lppl-1-3c.txt
% and the latest version of this license is in
%    http://www.latex-project.org/lppl.txt
% and version 1.3 or later is part of all distributions of
% LaTeX version 2005/12/01 or later.
%
% This work has the LPPL maintenance status "maintained".
%
% This Current Maintainer of this work is Heiko Oberdiek.
%
% The Base Interpreter refers to any `TeX-Format',
% because some files are installed in TDS:tex/generic//.
%
% This work consists of the main source file kvdefinekeys.dtx
% and the derived files
%    kvdefinekeys.sty, kvdefinekeys.pdf, kvdefinekeys.ins, kvdefinekeys.drv,
%    kvdefinekeys-test1.tex.
%
% Distribution:
%    CTAN:macros/latex/contrib/oberdiek/kvdefinekeys.dtx
%    CTAN:macros/latex/contrib/oberdiek/kvdefinekeys.pdf
%
% Unpacking:
%    (a) If kvdefinekeys.ins is present:
%           tex kvdefinekeys.ins
%    (b) Without kvdefinekeys.ins:
%           tex kvdefinekeys.dtx
%    (c) If you insist on using LaTeX
%           latex \let\install=y\input{kvdefinekeys.dtx}
%        (quote the arguments according to the demands of your shell)
%
% Documentation:
%    (a) If kvdefinekeys.drv is present:
%           latex kvdefinekeys.drv
%    (b) Without kvdefinekeys.drv:
%           latex kvdefinekeys.dtx; ...
%    The class ltxdoc loads the configuration file ltxdoc.cfg
%    if available. Here you can specify further options, e.g.
%    use A4 as paper format:
%       \PassOptionsToClass{a4paper}{article}
%
%    Programm calls to get the documentation (example):
%       pdflatex kvdefinekeys.dtx
%       makeindex -s gind.ist kvdefinekeys.idx
%       pdflatex kvdefinekeys.dtx
%       makeindex -s gind.ist kvdefinekeys.idx
%       pdflatex kvdefinekeys.dtx
%
% Installation:
%    TDS:tex/generic/oberdiek/kvdefinekeys.sty
%    TDS:doc/latex/oberdiek/kvdefinekeys.pdf
%    TDS:doc/latex/oberdiek/test/kvdefinekeys-test1.tex
%    TDS:source/latex/oberdiek/kvdefinekeys.dtx
%
%<*ignore>
\begingroup
  \catcode123=1 %
  \catcode125=2 %
  \def\x{LaTeX2e}%
\expandafter\endgroup
\ifcase 0\ifx\install y1\fi\expandafter
         \ifx\csname processbatchFile\endcsname\relax\else1\fi
         \ifx\fmtname\x\else 1\fi\relax
\else\csname fi\endcsname
%</ignore>
%<*install>
\input docstrip.tex
\Msg{************************************************************************}
\Msg{* Installation}
\Msg{* Package: kvdefinekeys 2016/05/16 v1.4 Define keys (HO)}
\Msg{************************************************************************}

\keepsilent
\askforoverwritefalse

\let\MetaPrefix\relax
\preamble

This is a generated file.

Project: kvdefinekeys
Version: 2016/05/16 v1.4

Copyright (C) 2010, 2011 by
   Heiko Oberdiek <heiko.oberdiek at googlemail.com>

This work may be distributed and/or modified under the
conditions of the LaTeX Project Public License, either
version 1.3c of this license or (at your option) any later
version. This version of this license is in
   http://www.latex-project.org/lppl/lppl-1-3c.txt
and the latest version of this license is in
   http://www.latex-project.org/lppl.txt
and version 1.3 or later is part of all distributions of
LaTeX version 2005/12/01 or later.

This work has the LPPL maintenance status "maintained".

This Current Maintainer of this work is Heiko Oberdiek.

The Base Interpreter refers to any `TeX-Format',
because some files are installed in TDS:tex/generic//.

This work consists of the main source file kvdefinekeys.dtx
and the derived files
   kvdefinekeys.sty, kvdefinekeys.pdf, kvdefinekeys.ins, kvdefinekeys.drv,
   kvdefinekeys-test1.tex.

\endpreamble
\let\MetaPrefix\DoubleperCent

\generate{%
  \file{kvdefinekeys.ins}{\from{kvdefinekeys.dtx}{install}}%
  \file{kvdefinekeys.drv}{\from{kvdefinekeys.dtx}{driver}}%
  \usedir{tex/generic/oberdiek}%
  \file{kvdefinekeys.sty}{\from{kvdefinekeys.dtx}{package}}%
  \usedir{doc/latex/oberdiek/test}%
  \file{kvdefinekeys-test1.tex}{\from{kvdefinekeys.dtx}{test1}}%
  \nopreamble
  \nopostamble
  \usedir{source/latex/oberdiek/catalogue}%
  \file{kvdefinekeys.xml}{\from{kvdefinekeys.dtx}{catalogue}}%
}

\catcode32=13\relax% active space
\let =\space%
\Msg{************************************************************************}
\Msg{*}
\Msg{* To finish the installation you have to move the following}
\Msg{* file into a directory searched by TeX:}
\Msg{*}
\Msg{*     kvdefinekeys.sty}
\Msg{*}
\Msg{* To produce the documentation run the file `kvdefinekeys.drv'}
\Msg{* through LaTeX.}
\Msg{*}
\Msg{* Happy TeXing!}
\Msg{*}
\Msg{************************************************************************}

\endbatchfile
%</install>
%<*ignore>
\fi
%</ignore>
%<*driver>
\NeedsTeXFormat{LaTeX2e}
\ProvidesFile{kvdefinekeys.drv}%
  [2016/05/16 v1.4 Define keys (HO)]%
\documentclass{ltxdoc}
\usepackage{holtxdoc}[2011/11/22]
\begin{document}
  \DocInput{kvdefinekeys.dtx}%
\end{document}
%</driver>
% \fi
%
%
% \CharacterTable
%  {Upper-case    \A\B\C\D\E\F\G\H\I\J\K\L\M\N\O\P\Q\R\S\T\U\V\W\X\Y\Z
%   Lower-case    \a\b\c\d\e\f\g\h\i\j\k\l\m\n\o\p\q\r\s\t\u\v\w\x\y\z
%   Digits        \0\1\2\3\4\5\6\7\8\9
%   Exclamation   \!     Double quote  \"     Hash (number) \#
%   Dollar        \$     Percent       \%     Ampersand     \&
%   Acute accent  \'     Left paren    \(     Right paren   \)
%   Asterisk      \*     Plus          \+     Comma         \,
%   Minus         \-     Point         \.     Solidus       \/
%   Colon         \:     Semicolon     \;     Less than     \<
%   Equals        \=     Greater than  \>     Question mark \?
%   Commercial at \@     Left bracket  \[     Backslash     \\
%   Right bracket \]     Circumflex    \^     Underscore    \_
%   Grave accent  \`     Left brace    \{     Vertical bar  \|
%   Right brace   \}     Tilde         \~}
%
% \GetFileInfo{kvdefinekeys.drv}
%
% \title{The \xpackage{kvdefinekeys} package}
% \date{2016/05/16 v1.4}
% \author{Heiko Oberdiek\thanks
% {Please report any issues at https://github.com/ho-tex/oberdiek/issues}\\
% \xemail{heiko.oberdiek at googlemail.com}}
%
% \maketitle
%
% \begin{abstract}
% Package \xpackage{kvdefinekeys} provides \cs{kv@define@key} to define
% keys the same way as \xpackage{keyval}'s \cs{define@key}. However, it
% works also using \iniTeX.
% \end{abstract}
%
% \tableofcontents
%
% \def\M#1{\texttt{\{}\meta{#1}\texttt{\}}}
%
% \section{Documentation}
%
% \subsection{Motivation}
%
% \cs{kvsetkeys} serves as replacement for \xpackage{keyval}'s
% \cs{setkeys}. This package adds macros to define keys, closing
% the gap \cs{kvsetkeys} leaves.
%
% \begin{declcs}{kv@define@key}\,\M{family}\,\M{key}\,[\meta{default}]\,^^A
%   \M{definition}
% \end{declcs}
% Macro \cs{kv@define@key} reimplements \xpackage{keyval}'s
% \cs{define@key}. Differences to the original:
% \begin{itemize}
% \item The defined keys also allow \cs{par} inside values.
% \item Shorthands of package \xpackage{babel} are supported in
%   family and key names.
% \item Macro \cs{kv@define@key} is made robust if
%   \hologo{eTeX}'s \cs{protected} or \hologo{LaTeX}'s
%   \cs{DeclareRobustCommand} are found.
% \end{itemize}
%
% \StopEventually{
% }
%
% \section{Implementation}
%
% \subsection{Identification}
%
%    \begin{macrocode}
%<*package>
%    \end{macrocode}
%    Reload check, especially if the package is not used with \LaTeX.
%    \begin{macrocode}
\begingroup\catcode61\catcode48\catcode32=10\relax%
  \catcode13=5 % ^^M
  \endlinechar=13 %
  \catcode35=6 % #
  \catcode39=12 % '
  \catcode44=12 % ,
  \catcode45=12 % -
  \catcode46=12 % .
  \catcode58=12 % :
  \catcode64=11 % @
  \catcode123=1 % {
  \catcode125=2 % }
  \expandafter\let\expandafter\x\csname ver@kvdefinekeys.sty\endcsname
  \ifx\x\relax % plain-TeX, first loading
  \else
    \def\empty{}%
    \ifx\x\empty % LaTeX, first loading,
      % variable is initialized, but \ProvidesPackage not yet seen
    \else
      \expandafter\ifx\csname PackageInfo\endcsname\relax
        \def\x#1#2{%
          \immediate\write-1{Package #1 Info: #2.}%
        }%
      \else
        \def\x#1#2{\PackageInfo{#1}{#2, stopped}}%
      \fi
      \x{kvdefinekeys}{The package is already loaded}%
      \aftergroup\endinput
    \fi
  \fi
\endgroup%
%    \end{macrocode}
%    Package identification:
%    \begin{macrocode}
\begingroup\catcode61\catcode48\catcode32=10\relax%
  \catcode13=5 % ^^M
  \endlinechar=13 %
  \catcode35=6 % #
  \catcode39=12 % '
  \catcode40=12 % (
  \catcode41=12 % )
  \catcode44=12 % ,
  \catcode45=12 % -
  \catcode46=12 % .
  \catcode47=12 % /
  \catcode58=12 % :
  \catcode64=11 % @
  \catcode91=12 % [
  \catcode93=12 % ]
  \catcode123=1 % {
  \catcode125=2 % }
  \expandafter\ifx\csname ProvidesPackage\endcsname\relax
    \def\x#1#2#3[#4]{\endgroup
      \immediate\write-1{Package: #3 #4}%
      \xdef#1{#4}%
    }%
  \else
    \def\x#1#2[#3]{\endgroup
      #2[{#3}]%
      \ifx#1\@undefined
        \xdef#1{#3}%
      \fi
      \ifx#1\relax
        \xdef#1{#3}%
      \fi
    }%
  \fi
\expandafter\x\csname ver@kvdefinekeys.sty\endcsname
\ProvidesPackage{kvdefinekeys}%
  [2016/05/16 v1.4 Define keys (HO)]%
%    \end{macrocode}
%
%    \begin{macrocode}
\begingroup\catcode61\catcode48\catcode32=10\relax%
  \catcode13=5 % ^^M
  \endlinechar=13 %
  \catcode123=1 % {
  \catcode125=2 % }
  \catcode64=11 % @
  \def\x{\endgroup
    \expandafter\edef\csname KVD@AtEnd\endcsname{%
      \endlinechar=\the\endlinechar\relax
      \catcode13=\the\catcode13\relax
      \catcode32=\the\catcode32\relax
      \catcode35=\the\catcode35\relax
      \catcode61=\the\catcode61\relax
      \catcode64=\the\catcode64\relax
      \catcode123=\the\catcode123\relax
      \catcode125=\the\catcode125\relax
    }%
  }%
\x\catcode61\catcode48\catcode32=10\relax%
\catcode13=5 % ^^M
\endlinechar=13 %
\catcode35=6 % #
\catcode64=11 % @
\catcode123=1 % {
\catcode125=2 % }
\def\TMP@EnsureCode#1#2{%
  \edef\KVD@AtEnd{%
    \KVD@AtEnd
    \catcode#1=\the\catcode#1\relax
  }%
  \catcode#1=#2\relax
}
\TMP@EnsureCode{42}{12}% *
\TMP@EnsureCode{46}{12}% .
\TMP@EnsureCode{47}{12}% /
\TMP@EnsureCode{91}{12}% [
\TMP@EnsureCode{93}{12}% ]
\edef\KVD@AtEnd{\KVD@AtEnd\noexpand\endinput}
%    \end{macrocode}
%
% \subsection{Package loading}
%
%    \begin{macrocode}
\begingroup\expandafter\expandafter\expandafter\endgroup
\expandafter\ifx\csname RequirePackage\endcsname\relax
  \def\TMP@RequirePackage#1[#2]{%
    \begingroup\expandafter\expandafter\expandafter\endgroup
    \expandafter\ifx\csname ver@#1.sty\endcsname\relax
      \input #1.sty\relax
    \fi
  }%
  \TMP@RequirePackage{ltxcmds}[2010/03/01]%
\else
  \RequirePackage{ltxcmds}[2010/03/01]%
\fi
%    \end{macrocode}
%
%
% \subsection{Provide key defining macro}
%
%    \begin{macro}{\kv@define@key}
%    \begin{macrocode}
\ltx@IfUndefined{protected}{%
  \ltx@IfUndefined{DeclareRobustCommand}{%
    \def\kv@define@key#1#2%
  }{%
    \DeclareRobustCommand*{\kv@define@key}[2]%
  }%
}{%
  \protected\def\kv@define@key#1#2%
}%
{%
  \begingroup
    \csname @safe@activestrue\endcsname
    \let\ifincsname\iftrue
    \edef\KVD@temp{\endgroup
      \noexpand\KVD@DefineKey{#1}{#2}%
    }%
  \KVD@temp
}
%    \end{macrocode}
%    \end{macro}
%    \begin{macro}{\KVD@DefineKey}
%    \begin{macrocode}
\def\KVD@DefineKey#1#2{%
  \ltx@ifnextchar[{%
    \KVD@DefineKeyWithDefault{#1}{#2}%
  }{%
    \long\expandafter\def\csname KV@#1@#2\endcsname##1%
  }%
}
%    \end{macrocode}
%    \end{macro}
%    \begin{macro}{\KVD@DefineKeyWithDefault}
%    \begin{macrocode}
\long\def\KVD@DefineKeyWithDefault#1#2[#3]{%
  \expandafter\def\csname KV@#1@#2@default\expandafter\endcsname
  \expandafter{%
    \csname KV@#1@#2\endcsname{#3}%
  }%
  \long\expandafter\def\csname KV@#1@#2\endcsname##1%
}
%    \end{macrocode}
%    \end{macro}
%
%    \begin{macrocode}
\KVD@AtEnd%
%</package>
%    \end{macrocode}
%
% \section{Test}
%
% \subsection{Catcode checks for loading}
%
%    \begin{macrocode}
%<*test1>
%    \end{macrocode}
%    \begin{macrocode}
\catcode`\{=1 %
\catcode`\}=2 %
\catcode`\#=6 %
\catcode`\@=11 %
\expandafter\ifx\csname count@\endcsname\relax
  \countdef\count@=255 %
\fi
\expandafter\ifx\csname @gobble\endcsname\relax
  \long\def\@gobble#1{}%
\fi
\expandafter\ifx\csname @firstofone\endcsname\relax
  \long\def\@firstofone#1{#1}%
\fi
\expandafter\ifx\csname loop\endcsname\relax
  \expandafter\@firstofone
\else
  \expandafter\@gobble
\fi
{%
  \def\loop#1\repeat{%
    \def\body{#1}%
    \iterate
  }%
  \def\iterate{%
    \body
      \let\next\iterate
    \else
      \let\next\relax
    \fi
    \next
  }%
  \let\repeat=\fi
}%
\def\RestoreCatcodes{}
\count@=0 %
\loop
  \edef\RestoreCatcodes{%
    \RestoreCatcodes
    \catcode\the\count@=\the\catcode\count@\relax
  }%
\ifnum\count@<255 %
  \advance\count@ 1 %
\repeat

\def\RangeCatcodeInvalid#1#2{%
  \count@=#1\relax
  \loop
    \catcode\count@=15 %
  \ifnum\count@<#2\relax
    \advance\count@ 1 %
  \repeat
}
\def\RangeCatcodeCheck#1#2#3{%
  \count@=#1\relax
  \loop
    \ifnum#3=\catcode\count@
    \else
      \errmessage{%
        Character \the\count@\space
        with wrong catcode \the\catcode\count@\space
        instead of \number#3%
      }%
    \fi
  \ifnum\count@<#2\relax
    \advance\count@ 1 %
  \repeat
}
\def\space{ }
\expandafter\ifx\csname LoadCommand\endcsname\relax
  \def\LoadCommand{\input kvdefinekeys.sty\relax}%
\fi
\def\Test{%
  \RangeCatcodeInvalid{0}{47}%
  \RangeCatcodeInvalid{58}{64}%
  \RangeCatcodeInvalid{91}{96}%
  \RangeCatcodeInvalid{123}{255}%
  \catcode`\@=12 %
  \catcode`\\=0 %
  \catcode`\%=14 %
  \LoadCommand
  \RangeCatcodeCheck{0}{36}{15}%
  \RangeCatcodeCheck{37}{37}{14}%
  \RangeCatcodeCheck{38}{47}{15}%
  \RangeCatcodeCheck{48}{57}{12}%
  \RangeCatcodeCheck{58}{63}{15}%
  \RangeCatcodeCheck{64}{64}{12}%
  \RangeCatcodeCheck{65}{90}{11}%
  \RangeCatcodeCheck{91}{91}{15}%
  \RangeCatcodeCheck{92}{92}{0}%
  \RangeCatcodeCheck{93}{96}{15}%
  \RangeCatcodeCheck{97}{122}{11}%
  \RangeCatcodeCheck{123}{255}{15}%
  \RestoreCatcodes
}
\Test
\csname @@end\endcsname
\end
%    \end{macrocode}
%    \begin{macrocode}
%</test1>
%    \end{macrocode}
%
% \section{Installation}
%
% \subsection{Download}
%
% \paragraph{Package.} This package is available on
% CTAN\footnote{\url{http://ctan.org/pkg/kvdefinekeys}}:
% \begin{description}
% \item[\CTAN{macros/latex/contrib/oberdiek/kvdefinekeys.dtx}] The source file.
% \item[\CTAN{macros/latex/contrib/oberdiek/kvdefinekeys.pdf}] Documentation.
% \end{description}
%
%
% \paragraph{Bundle.} All the packages of the bundle `oberdiek'
% are also available in a TDS compliant ZIP archive. There
% the packages are already unpacked and the documentation files
% are generated. The files and directories obey the TDS standard.
% \begin{description}
% \item[\CTAN{install/macros/latex/contrib/oberdiek.tds.zip}]
% \end{description}
% \emph{TDS} refers to the standard ``A Directory Structure
% for \TeX\ Files'' (\CTAN{tds/tds.pdf}). Directories
% with \xfile{texmf} in their name are usually organized this way.
%
% \subsection{Bundle installation}
%
% \paragraph{Unpacking.} Unpack the \xfile{oberdiek.tds.zip} in the
% TDS tree (also known as \xfile{texmf} tree) of your choice.
% Example (linux):
% \begin{quote}
%   |unzip oberdiek.tds.zip -d ~/texmf|
% \end{quote}
%
% \paragraph{Script installation.}
% Check the directory \xfile{TDS:scripts/oberdiek/} for
% scripts that need further installation steps.
% Package \xpackage{attachfile2} comes with the Perl script
% \xfile{pdfatfi.pl} that should be installed in such a way
% that it can be called as \texttt{pdfatfi}.
% Example (linux):
% \begin{quote}
%   |chmod +x scripts/oberdiek/pdfatfi.pl|\\
%   |cp scripts/oberdiek/pdfatfi.pl /usr/local/bin/|
% \end{quote}
%
% \subsection{Package installation}
%
% \paragraph{Unpacking.} The \xfile{.dtx} file is a self-extracting
% \docstrip\ archive. The files are extracted by running the
% \xfile{.dtx} through \plainTeX:
% \begin{quote}
%   \verb|tex kvdefinekeys.dtx|
% \end{quote}
%
% \paragraph{TDS.} Now the different files must be moved into
% the different directories in your installation TDS tree
% (also known as \xfile{texmf} tree):
% \begin{quote}
% \def\t{^^A
% \begin{tabular}{@{}>{\ttfamily}l@{ $\rightarrow$ }>{\ttfamily}l@{}}
%   kvdefinekeys.sty & tex/generic/oberdiek/kvdefinekeys.sty\\
%   kvdefinekeys.pdf & doc/latex/oberdiek/kvdefinekeys.pdf\\
%   test/kvdefinekeys-test1.tex & doc/latex/oberdiek/test/kvdefinekeys-test1.tex\\
%   kvdefinekeys.dtx & source/latex/oberdiek/kvdefinekeys.dtx\\
% \end{tabular}^^A
% }^^A
% \sbox0{\t}^^A
% \ifdim\wd0>\linewidth
%   \begingroup
%     \advance\linewidth by\leftmargin
%     \advance\linewidth by\rightmargin
%   \edef\x{\endgroup
%     \def\noexpand\lw{\the\linewidth}^^A
%   }\x
%   \def\lwbox{^^A
%     \leavevmode
%     \hbox to \linewidth{^^A
%       \kern-\leftmargin\relax
%       \hss
%       \usebox0
%       \hss
%       \kern-\rightmargin\relax
%     }^^A
%   }^^A
%   \ifdim\wd0>\lw
%     \sbox0{\small\t}^^A
%     \ifdim\wd0>\linewidth
%       \ifdim\wd0>\lw
%         \sbox0{\footnotesize\t}^^A
%         \ifdim\wd0>\linewidth
%           \ifdim\wd0>\lw
%             \sbox0{\scriptsize\t}^^A
%             \ifdim\wd0>\linewidth
%               \ifdim\wd0>\lw
%                 \sbox0{\tiny\t}^^A
%                 \ifdim\wd0>\linewidth
%                   \lwbox
%                 \else
%                   \usebox0
%                 \fi
%               \else
%                 \lwbox
%               \fi
%             \else
%               \usebox0
%             \fi
%           \else
%             \lwbox
%           \fi
%         \else
%           \usebox0
%         \fi
%       \else
%         \lwbox
%       \fi
%     \else
%       \usebox0
%     \fi
%   \else
%     \lwbox
%   \fi
% \else
%   \usebox0
% \fi
% \end{quote}
% If you have a \xfile{docstrip.cfg} that configures and enables \docstrip's
% TDS installing feature, then some files can already be in the right
% place, see the documentation of \docstrip.
%
% \subsection{Refresh file name databases}
%
% If your \TeX~distribution
% (\teTeX, \mikTeX, \dots) relies on file name databases, you must refresh
% these. For example, \teTeX\ users run \verb|texhash| or
% \verb|mktexlsr|.
%
% \subsection{Some details for the interested}
%
% \paragraph{Attached source.}
%
% The PDF documentation on CTAN also includes the
% \xfile{.dtx} source file. It can be extracted by
% AcrobatReader 6 or higher. Another option is \textsf{pdftk},
% e.g. unpack the file into the current directory:
% \begin{quote}
%   \verb|pdftk kvdefinekeys.pdf unpack_files output .|
% \end{quote}
%
% \paragraph{Unpacking with \LaTeX.}
% The \xfile{.dtx} chooses its action depending on the format:
% \begin{description}
% \item[\plainTeX:] Run \docstrip\ and extract the files.
% \item[\LaTeX:] Generate the documentation.
% \end{description}
% If you insist on using \LaTeX\ for \docstrip\ (really,
% \docstrip\ does not need \LaTeX), then inform the autodetect routine
% about your intention:
% \begin{quote}
%   \verb|latex \let\install=y\input{kvdefinekeys.dtx}|
% \end{quote}
% Do not forget to quote the argument according to the demands
% of your shell.
%
% \paragraph{Generating the documentation.}
% You can use both the \xfile{.dtx} or the \xfile{.drv} to generate
% the documentation. The process can be configured by the
% configuration file \xfile{ltxdoc.cfg}. For instance, put this
% line into this file, if you want to have A4 as paper format:
% \begin{quote}
%   \verb|\PassOptionsToClass{a4paper}{article}|
% \end{quote}
% An example follows how to generate the
% documentation with pdf\LaTeX:
% \begin{quote}
%\begin{verbatim}
%pdflatex kvdefinekeys.dtx
%makeindex -s gind.ist kvdefinekeys.idx
%pdflatex kvdefinekeys.dtx
%makeindex -s gind.ist kvdefinekeys.idx
%pdflatex kvdefinekeys.dtx
%\end{verbatim}
% \end{quote}
%
% \section{Catalogue}
%
% The following XML file can be used as source for the
% \href{http://mirror.ctan.org/help/Catalogue/catalogue.html}{\TeX\ Catalogue}.
% The elements \texttt{caption} and \texttt{description} are imported
% from the original XML file from the Catalogue.
% The name of the XML file in the Catalogue is \xfile{kvdefinekeys.xml}.
%    \begin{macrocode}
%<*catalogue>
<?xml version='1.0' encoding='us-ascii'?>
<!DOCTYPE entry SYSTEM 'catalogue.dtd'>
<entry datestamp='$Date$' modifier='$Author$' id='kvdefinekeys'>
  <name>kvdefinekeys</name>
  <caption>Define keys for use in the kvsetkeys package.</caption>
  <authorref id='auth:oberdiek'/>
  <copyright owner='Heiko Oberdiek' year='2010,2011'/>
  <license type='lppl1.3'/>
  <version number='1.4'/>
  <description>
    The package provides a macro <tt>\kv@define@key</tt> (analogous to
    <xref refid='keyval'>keyval&#x2019;s</xref> <tt>\define@key</tt>, to
    define keys for use by <xref refid='kvsetkeys'>kvsetkeys</xref>.
    <p/>
    The package is part of the <xref refid='oberdiek'>oberdiek</xref>
    bundle.
  </description>
  <documentation details='Package documentation'
      href='ctan:/macros/latex/contrib/oberdiek/kvdefinekeys.pdf'/>
  <ctan file='true' path='/macros/latex/contrib/oberdiek/kvdefinekeys.dtx'/>
  <miktex location='oberdiek'/>
  <texlive location='oberdiek'/>
  <install path='/macros/latex/contrib/oberdiek/oberdiek.tds.zip'/>
</entry>
%</catalogue>
%    \end{macrocode}
%
% \begin{thebibliography}{9}
% \bibitem{keyval}
%   David Carlisle:
%   \textit{The \xpackage{keyval} package};
%   1999/03/16 v1.13;
%   \CTAN{macros/latex/required/graphics/keyval.dtx}.
%
% \end{thebibliography}
%
% \begin{History}
%   \begin{Version}{2010/03/01 v1.0}
%   \item
%     First version.
%   \end{Version}
%   \begin{Version}{2010/08/19 v1.1}
%   \item
%     Documentation fix, no code change.
%   \end{Version}
%   \begin{Version}{2011/01/30 v1.2}
%   \item
%     Already loaded package files are not input in \hologo{plainTeX}.
%   \end{Version}
%   \begin{Version}{2011/04/07 v1.3}
%   \item
%     Support for package \xpackage{babel}'s shorthands added.
%   \item
%     \cs{kv@define@key} is made robust if available.
%   \end{Version}
%   \begin{Version}{2016/05/16 v1.4}
%   \item
%     Documentation updates.
%   \end{Version}
% \end{History}
%
% \PrintIndex
%
% \Finale
\endinput
|
% \end{quote}
% Do not forget to quote the argument according to the demands
% of your shell.
%
% \paragraph{Generating the documentation.}
% You can use both the \xfile{.dtx} or the \xfile{.drv} to generate
% the documentation. The process can be configured by the
% configuration file \xfile{ltxdoc.cfg}. For instance, put this
% line into this file, if you want to have A4 as paper format:
% \begin{quote}
%   \verb|\PassOptionsToClass{a4paper}{article}|
% \end{quote}
% An example follows how to generate the
% documentation with pdf\LaTeX:
% \begin{quote}
%\begin{verbatim}
%pdflatex kvdefinekeys.dtx
%makeindex -s gind.ist kvdefinekeys.idx
%pdflatex kvdefinekeys.dtx
%makeindex -s gind.ist kvdefinekeys.idx
%pdflatex kvdefinekeys.dtx
%\end{verbatim}
% \end{quote}
%
% \section{Catalogue}
%
% The following XML file can be used as source for the
% \href{http://mirror.ctan.org/help/Catalogue/catalogue.html}{\TeX\ Catalogue}.
% The elements \texttt{caption} and \texttt{description} are imported
% from the original XML file from the Catalogue.
% The name of the XML file in the Catalogue is \xfile{kvdefinekeys.xml}.
%    \begin{macrocode}
%<*catalogue>
<?xml version='1.0' encoding='us-ascii'?>
<!DOCTYPE entry SYSTEM 'catalogue.dtd'>
<entry datestamp='$Date$' modifier='$Author$' id='kvdefinekeys'>
  <name>kvdefinekeys</name>
  <caption>Define keys for use in the kvsetkeys package.</caption>
  <authorref id='auth:oberdiek'/>
  <copyright owner='Heiko Oberdiek' year='2010,2011'/>
  <license type='lppl1.3'/>
  <version number='1.4'/>
  <description>
    The package provides a macro <tt>\kv@define@key</tt> (analogous to
    <xref refid='keyval'>keyval&#x2019;s</xref> <tt>\define@key</tt>, to
    define keys for use by <xref refid='kvsetkeys'>kvsetkeys</xref>.
    <p/>
    The package is part of the <xref refid='oberdiek'>oberdiek</xref>
    bundle.
  </description>
  <documentation details='Package documentation'
      href='ctan:/macros/latex/contrib/oberdiek/kvdefinekeys.pdf'/>
  <ctan file='true' path='/macros/latex/contrib/oberdiek/kvdefinekeys.dtx'/>
  <miktex location='oberdiek'/>
  <texlive location='oberdiek'/>
  <install path='/macros/latex/contrib/oberdiek/oberdiek.tds.zip'/>
</entry>
%</catalogue>
%    \end{macrocode}
%
% \begin{thebibliography}{9}
% \bibitem{keyval}
%   David Carlisle:
%   \textit{The \xpackage{keyval} package};
%   1999/03/16 v1.13;
%   \CTAN{macros/latex/required/graphics/keyval.dtx}.
%
% \end{thebibliography}
%
% \begin{History}
%   \begin{Version}{2010/03/01 v1.0}
%   \item
%     First version.
%   \end{Version}
%   \begin{Version}{2010/08/19 v1.1}
%   \item
%     Documentation fix, no code change.
%   \end{Version}
%   \begin{Version}{2011/01/30 v1.2}
%   \item
%     Already loaded package files are not input in \hologo{plainTeX}.
%   \end{Version}
%   \begin{Version}{2011/04/07 v1.3}
%   \item
%     Support for package \xpackage{babel}'s shorthands added.
%   \item
%     \cs{kv@define@key} is made robust if available.
%   \end{Version}
%   \begin{Version}{2016/05/16 v1.4}
%   \item
%     Documentation updates.
%   \end{Version}
% \end{History}
%
% \PrintIndex
%
% \Finale
\endinput
|
% \end{quote}
% Do not forget to quote the argument according to the demands
% of your shell.
%
% \paragraph{Generating the documentation.}
% You can use both the \xfile{.dtx} or the \xfile{.drv} to generate
% the documentation. The process can be configured by the
% configuration file \xfile{ltxdoc.cfg}. For instance, put this
% line into this file, if you want to have A4 as paper format:
% \begin{quote}
%   \verb|\PassOptionsToClass{a4paper}{article}|
% \end{quote}
% An example follows how to generate the
% documentation with pdf\LaTeX:
% \begin{quote}
%\begin{verbatim}
%pdflatex kvdefinekeys.dtx
%makeindex -s gind.ist kvdefinekeys.idx
%pdflatex kvdefinekeys.dtx
%makeindex -s gind.ist kvdefinekeys.idx
%pdflatex kvdefinekeys.dtx
%\end{verbatim}
% \end{quote}
%
% \section{Catalogue}
%
% The following XML file can be used as source for the
% \href{http://mirror.ctan.org/help/Catalogue/catalogue.html}{\TeX\ Catalogue}.
% The elements \texttt{caption} and \texttt{description} are imported
% from the original XML file from the Catalogue.
% The name of the XML file in the Catalogue is \xfile{kvdefinekeys.xml}.
%    \begin{macrocode}
%<*catalogue>
<?xml version='1.0' encoding='us-ascii'?>
<!DOCTYPE entry SYSTEM 'catalogue.dtd'>
<entry datestamp='$Date$' modifier='$Author$' id='kvdefinekeys'>
  <name>kvdefinekeys</name>
  <caption>Define keys for use in the kvsetkeys package.</caption>
  <authorref id='auth:oberdiek'/>
  <copyright owner='Heiko Oberdiek' year='2010,2011'/>
  <license type='lppl1.3'/>
  <version number='1.4'/>
  <description>
    The package provides a macro <tt>\kv@define@key</tt> (analogous to
    <xref refid='keyval'>keyval&#x2019;s</xref> <tt>\define@key</tt>, to
    define keys for use by <xref refid='kvsetkeys'>kvsetkeys</xref>.
    <p/>
    The package is part of the <xref refid='oberdiek'>oberdiek</xref>
    bundle.
  </description>
  <documentation details='Package documentation'
      href='ctan:/macros/latex/contrib/oberdiek/kvdefinekeys.pdf'/>
  <ctan file='true' path='/macros/latex/contrib/oberdiek/kvdefinekeys.dtx'/>
  <miktex location='oberdiek'/>
  <texlive location='oberdiek'/>
  <install path='/macros/latex/contrib/oberdiek/oberdiek.tds.zip'/>
</entry>
%</catalogue>
%    \end{macrocode}
%
% \begin{thebibliography}{9}
% \bibitem{keyval}
%   David Carlisle:
%   \textit{The \xpackage{keyval} package};
%   1999/03/16 v1.13;
%   \CTAN{macros/latex/required/graphics/keyval.dtx}.
%
% \end{thebibliography}
%
% \begin{History}
%   \begin{Version}{2010/03/01 v1.0}
%   \item
%     First version.
%   \end{Version}
%   \begin{Version}{2010/08/19 v1.1}
%   \item
%     Documentation fix, no code change.
%   \end{Version}
%   \begin{Version}{2011/01/30 v1.2}
%   \item
%     Already loaded package files are not input in \hologo{plainTeX}.
%   \end{Version}
%   \begin{Version}{2011/04/07 v1.3}
%   \item
%     Support for package \xpackage{babel}'s shorthands added.
%   \item
%     \cs{kv@define@key} is made robust if available.
%   \end{Version}
%   \begin{Version}{2016/05/16 v1.4}
%   \item
%     Documentation updates.
%   \end{Version}
% \end{History}
%
% \PrintIndex
%
% \Finale
\endinput
|
% \end{quote}
% Do not forget to quote the argument according to the demands
% of your shell.
%
% \paragraph{Generating the documentation.}
% You can use both the \xfile{.dtx} or the \xfile{.drv} to generate
% the documentation. The process can be configured by the
% configuration file \xfile{ltxdoc.cfg}. For instance, put this
% line into this file, if you want to have A4 as paper format:
% \begin{quote}
%   \verb|\PassOptionsToClass{a4paper}{article}|
% \end{quote}
% An example follows how to generate the
% documentation with pdf\LaTeX:
% \begin{quote}
%\begin{verbatim}
%pdflatex kvdefinekeys.dtx
%makeindex -s gind.ist kvdefinekeys.idx
%pdflatex kvdefinekeys.dtx
%makeindex -s gind.ist kvdefinekeys.idx
%pdflatex kvdefinekeys.dtx
%\end{verbatim}
% \end{quote}
%
% \section{Catalogue}
%
% The following XML file can be used as source for the
% \href{http://mirror.ctan.org/help/Catalogue/catalogue.html}{\TeX\ Catalogue}.
% The elements \texttt{caption} and \texttt{description} are imported
% from the original XML file from the Catalogue.
% The name of the XML file in the Catalogue is \xfile{kvdefinekeys.xml}.
%    \begin{macrocode}
%<*catalogue>
<?xml version='1.0' encoding='us-ascii'?>
<!DOCTYPE entry SYSTEM 'catalogue.dtd'>
<entry datestamp='$Date$' modifier='$Author$' id='kvdefinekeys'>
  <name>kvdefinekeys</name>
  <caption>Define keys for use in the kvsetkeys package.</caption>
  <authorref id='auth:oberdiek'/>
  <copyright owner='Heiko Oberdiek' year='2010,2011'/>
  <license type='lppl1.3'/>
  <version number='1.4'/>
  <description>
    The package provides a macro <tt>\kv@define@key</tt> (analogous to
    <xref refid='keyval'>keyval&#x2019;s</xref> <tt>\define@key</tt>, to
    define keys for use by <xref refid='kvsetkeys'>kvsetkeys</xref>.
    <p/>
    The package is part of the <xref refid='oberdiek'>oberdiek</xref>
    bundle.
  </description>
  <documentation details='Package documentation'
      href='ctan:/macros/latex/contrib/oberdiek/kvdefinekeys.pdf'/>
  <ctan file='true' path='/macros/latex/contrib/oberdiek/kvdefinekeys.dtx'/>
  <miktex location='oberdiek'/>
  <texlive location='oberdiek'/>
  <install path='/macros/latex/contrib/oberdiek/oberdiek.tds.zip'/>
</entry>
%</catalogue>
%    \end{macrocode}
%
% \begin{thebibliography}{9}
% \bibitem{keyval}
%   David Carlisle:
%   \textit{The \xpackage{keyval} package};
%   1999/03/16 v1.13;
%   \CTAN{macros/latex/required/graphics/keyval.dtx}.
%
% \end{thebibliography}
%
% \begin{History}
%   \begin{Version}{2010/03/01 v1.0}
%   \item
%     First version.
%   \end{Version}
%   \begin{Version}{2010/08/19 v1.1}
%   \item
%     Documentation fix, no code change.
%   \end{Version}
%   \begin{Version}{2011/01/30 v1.2}
%   \item
%     Already loaded package files are not input in \hologo{plainTeX}.
%   \end{Version}
%   \begin{Version}{2011/04/07 v1.3}
%   \item
%     Support for package \xpackage{babel}'s shorthands added.
%   \item
%     \cs{kv@define@key} is made robust if available.
%   \end{Version}
%   \begin{Version}{2016/05/16 v1.4}
%   \item
%     Documentation updates.
%   \end{Version}
% \end{History}
%
% \PrintIndex
%
% \Finale
\endinput
