% file: psnfss2e.tex as of 2004-09-15
%
% Copyright 2000--2004 Walter Schmidt
%
%  This file may be distributed and/or modified under the
%  conditions of the LaTeX Project Public License, either version 1.2
%  of this license or (at your option) any later version.
%  The latest version of this license is in
%    http://www.latex-project.org/lppl.txt
%  and version 1.2 or later is part of all distributions of LaTeX
%  version 1999/12/01 or later.

\newif\ifutopia

% \utopiatrue
% If you have got the Utopia fonts, uncomment the above line,
% or put \utopiatrue into your ltxguide.cfg.

\NeedsTeXFormat{LaTeX2e}[1995/12/01]

\documentclass[11pt]{ltxguide}[1995/11/28]
\DeleteShortVerb{\|}

% You may  provide a configuration file ltxguide.cfg
% to set up your preferred paper size and page layout.
% The .cfg file should, however, not change the fonts used!

\usepackage{mathptmx,courier}
\usepackage[scaled=0.92]{helvet}
\normalfont
\usepackage{pifont,tabularx,varioref,url}
\usepackage[T1]{fontenc}
\usepackage{textcomp}
\usepackage{ragged2e}

\usepackage[colorlinks=false, pdfborder={0 0 0}]{hyperref}

% some dirty hacks to make the ltxguide class look better:
\makeatletter
\renewcommand\section{\@startsection{section}{1}{\z@}%
{-3.5ex \@plus -.75ex}%
{1ex}%
{\normalfont\Large\bfseries}}
\renewcommand\subsection{\@startsection{subsection}{2}{\z@}%
{-2.5ex plus -.5ex}%
{.75ex}%
{\normalfont\large\bfseries}}
\renewcommand\subsubsection{\@startsection{subsubsection}{3}{\z@}%
{-2.5ex plus -.5ex}%
{.5ex}%
{\normalsize\bfseries}}
\setlength{\parskip}{1ex plus 2pt minus 1pt \relax}
% borrowed from tdsguide.cls:
\labelsep=1.1em         % increase distance between item & text
\topsep=0pt             % no extra skip above list in paragraph
\partopsep=0pt          % no extra skip above list starting par.
\itemsep=0pt            % no extra space between items
\parsep=.9\parskip      % between paragraphs in items
\def\@listI{%
    \leftmargin\leftmargini
    }
\let\@listi\@listI
\@listi
\def\@listii{%
    \leftmargin\leftmarginii
    \labelwidth\leftmarginii  \advance \labelwidth-\labelsep
    }
\def\@listiii{%
    \leftmargin\leftmarginiii
    \labelwidth\leftmarginiii  \advance \labelwidth-\labelsep
    }
\makeatother

% the (La)TeX logos for use with Times:
\def\ptmTeX{T\kern-.1667em\lower.5ex\hbox{E}\kern-.075emX\@}
\makeatletter
\DeclareRobustCommand{\ptmLaTeX}{L\kern-.25em
        {\setbox0\hbox{T}%
         \vbox to\ht0{\hbox{%
                            \csname S@\f@size\endcsname
                            \fontsize\sf@size\z@
                            \math@fontsfalse\selectfont
                            A}%
                      \vss}%
        }%
        \kern-.10em
        \TeX}
\makeatother
\let\TeX=\ptmTeX
\let\LaTeX=\ptmLaTeX

% a bit of logical markup:
\newcommand{\ps}{PostScript}
\newcommand{\Lpack}[1]{\textsf{#1}}

% the macros for the typeface samples:
\newlength{\rightwidth}
  \rightwidth=11cm
\newlength{\leftwidth}
  \leftwidth=\textwidth
  \addtolength{\leftwidth}{-\rightwidth}
  \addtolength{\leftwidth}{-1em} % ?
\newcommand{\sample}[5]{%
  \parbox[t]{\leftwidth}{%
    {\raggedright\footnotesize
    #1\\
    #3/#4\par}}
  \hfill
  \parbox[t]{\rightwidth}{
  {\RaggedRight \fontfamily{#2}\fontsize{#3}{#4}\selectfont #5
  The sun was just rising as Dr.\ Robert entered his wife's room.
  An orange glow, and against it the jagged silhouette of the mountains.
  Then suddenly a dazzling sickle of incandescence between two peaks.
  The sickle became a half circle and the first long shadows,
  the first shafts of golden light crossed the garden outside the window.
  And when one looked up again at the mountains there was the whole
  unbearable glory of the risen sun.
  \par}}
  \par
  \medskip}


\title{Using common \ps\ fonts with \LaTeX}

\author{Walter Schmidt}

\date{PSNFSS version 9.2 \\ 2004-09-15}


\hypersetup{pdfauthor={Walter Schmidt},
            pdftitle={Using common PostScript fonts with LaTeX},
            pdfsubject={PSNFSS v9.2},
            pdfkeywords={LaTeX PSNFSS PostScript fonts}}


\begin{document}
\MakeShortVerb{\+}

\maketitle

\tableofcontents
%\pagebreak


\section{What is PSNFSS\,?}

The PSNFSS collection includes a set of files
that provide a complete working setup of the \LaTeX{} font selection scheme
(NFSS2) for use with common \ps\ fonts.
It covers the so-called `Base~35' fonts
(which are built into any Level~2 \ps\ printing device
and the Ghostscript interpreter)
and %the free Charter, Utopia and Pazo fonts.
a number of free fonts.



\section{Package overview}

The easiest way to make use of the above-mentioned typefaces is to completely
replace one or more of the font families used by \LaTeX{} as
`roman', `sans serif' and `typewriter' family and for math.
This is accomplished by the packages listed in table \ref{tab:packages}.
Its first row lists the default (Computer Modern) font families.
An empty column indicates that a package does not change the particular
font family.  Some of these packages need more detailed explanation and
are described in the below sections \ref{sec:helvet}, \ref{sec:mathpazo}
and \ref{sec:mathptmx}.

The PSNFSS distribution includes also a package
\Lpack{pifont}, which serves for accessing symbol fonts (aka `Pi~fonts'),
such as Symbol and Zapf Dingbats, see section~\ref{sec:pifont}.


\begin{table}[h!]
\caption{Packages for using common \ps\ fonts}
\label{tab:packages}
\medskip
{\footnotesize
\begin{center}
\renewcommand{\arraystretch}{1.2}
\begin{tabular}{|l|p{1.8cm}p{2.2cm}p{2.4cm}p{2.2cm}|}
\hline
\textbf{package} & \textbf{roman}  & \textbf{sans serif} & \textbf{typewriter} & \textbf{formulas} \\\hline\hline
(none)           & CM Roman        & CM Sans Serif       & CM Typewriter       & $\approx$ CM Roman\\\hline
\Lpack{mathpazo} & Palatino
                 &
                 &
                 & $\approx$ Palatino\\\hline
\Lpack{mathptmx} & Times
                 &
                 &
                 & $\approx$ Times\\\hline
\Lpack{helvet}   &
                 & Helvetica
                 &
                 & \\\hline
\Lpack{avant}    &
                 & Avant~Garde
                 &
                 & \\\hline
\Lpack{courier}  &
                 &
                 & Courier
                 & \\\hline
\Lpack{chancery} & Zapf Chancery
                 &
                 &
                 & \\\hline
\Lpack{bookman}  & Bookman
                 & Avant~Garde
                 & Courier
                 & \\\hline
\Lpack{newcent}  & New Century Schoolbook
                 & Avant~Garde
                 & Courier
                 & \\\hline
\Lpack{charter}  & Charter
                 &
                 &
                 & \\\hline
\end{tabular}
\end{center}
}
\end{table}



\section{Special considerations}

\subsection{Output font encoding}
None of the packages listed in table~\ref{tab:packages} changes the
output font encoding from its default setting OT1.
It is, however, highly recommended to use the fonts with the extended
T1 and TS1 (text symbols) encodings by means of the commands:
\begin{quote}
  +\usepackage[T1]{fontenc}+\\
  +\usepackage{textcomp}+
\end{quote}
When using \ps\ fonts that  come from `outside the \TeX{} world',
there is no reason at all to stay with the obsolete OT1~encoding,
which would not provide access to all available glyphs.
However, since these fonts were not particularly designed
for use with \TeX{}, they do \emph{not} include all of the
text companion (TS1) symbols.


\subsection{Euro support}
%% \NEWfeature{2003-11-17 v9.1}
From PSNFSS version~9.1 on, all supported text font families,
with the exception of \texttt{put} (Utopia), provide
a built-in Euro symbol +\texteuro+.
Using this command requires the \Lpack{textcomp} package; see above.


\subsection{Inter-line spacing}
With certain font families, the leading of the standard \LaTeX{}
document classes may be too small.
This results from the larger x-height of these typefaces, as compared with
Computer Modern.
Since it is a question of document design and line width,
the packages of the PSNFSS bundle do \emph{not} take care of this.
Issuing the command
\begin{quote}
  +\linespread+ \arg{factor}
\end{quote}
in the preamble will globally enlarge the leading by the given factor.
Typical values for \m{factor} are in the range of $1.04\dots1.1$.


\subsection{Using sans serif fonts}
The packages \Lpack{helvet} and \Lpack{avant} do not change the
default text font family from `roman'.
If required, the additional command
\begin{quote}
  +\renewcommand{\familydefault}{\sfdefault}+
\end{quote}
makes \LaTeX{} use the sans serif font family (Helvetica or Avant~Garde)
as the default one in text mode.  Notice, however, that this does not change the fonts
used in the formulas!



\section{The package \Lpack{helvet}}
\label{sec:helvet}
Helvetica is actually somewhat larger than other typefaces
of the same nominal size.
As a result, mixing, \eg Times and Helvetica within running text
may look bad.
\begin{decl}
  \texttt{[scaled=}\m{scale}\texttt{]}\\
  \texttt{[scaled]}
\end{decl}
This can be fixed by loading the package with the option
\texttt{[scaled=}\m{scale}\texttt{]},
for instance:
+\usepackage[scaled=.92]{helvet}+.
As a result, the font family \texttt{phv} (Helvetica) will be
scaled down to 92\% of its `natural' size, which is suitable
for use with Adobe Times.
Specifying +[scaled]+ alone is equivalent to +[scaled=0.95]+% and makes
%the height of the Helvetica capital letters comply with many other typeface
%families
.



\section{The package \Lpack{mathpazo}}
\label{sec:mathpazo}
Loading
%% \NEWfeature{2001-06-04 v8.2}
this package changes the default roman font family
to Adobe Palatino, and the virtual `mathpazo' fonts will be used
for math.
These virtual fonts are made up basically from Palatino Italic, with the
missing math symbols coming from the CM and Pazo math fonts.

\subsection{Package options}

\begin{decl}
  +[sc]+\\
  +[osf]+
\end{decl}
By default, the package \Lpack{mathpazo} uses
the typeface family +ppl+ as the roman text font family.
The option +[sc]+ selects Palatino with real smallcaps (family +pplx+) insread.
Correspondingly, the option +[osf]+ selects Palatino with smallcaps and default
oldstyle figures (family +pplj+).
Of course, oldstyle figures will be used only in text mode, as opposed to formulas.
\NEWfeature{2004-09-15 v9.2}
Using either option is strongly recommended:
Beside the real smallcaps, the font families +pplx+ and +pplj+ show  further
improvements over +ppl+: Increased word space, enhanced kerning tables,
additional `dotlessj' glyph.

\begin{decl}
  +[slantedGreek]+
\end{decl}
When the package is loaded with the +[slantedGreek]+ option,
uppercase Greek letters in math will be italic by default.

\begin{decl}
  +[noBBpl]+
\end{decl}
This option disables the use of the Pazo fonts as a partial
+\mathbb+ alphabet -- see below.  The option should be specified,
if you want to use a different `blackboard bold' font.


\subsection{New commands}

\begin{decl}
  +\upGamma+, +\upDelta+ \dots +\upOmega+
\end{decl}
\NEWfeature{2004-09-15 v9.2}
Regardless of the \texttt{slantedGreek} option,
these commands always yield upright uppercase Greek letters.
Upright \emph{lowercase} Greek is, however, not available.

\begin{decl}
  +\mathbold+\\
  +\mathbb+
\end{decl}
+\mathbold+ is a math alphabet for typesetting variables (incl.\ Greek)
in a {\bfseries\itshape bold italic} style.  Do not mix this up with +\mathbf+,
which selects a {\bfseries\upshape bold upright} text font
for use in math!

+\mathbb+ is a `blackboard bold' alphabet, whose characters
are taken from the Pazo fonts.  %% \NEWfeature{2002-04-10 v9.0}
All upper case letters and the numeral `1' are available.
If you want to use a different, external, doublestroke  alphabet,
load the package \Lpack{mathpazo} with the package option +[noBBppl]+, see above.

\iffalse
\begin{decl}
  +\ppleuro+
\end{decl}
The command +\ppleuro+ typesets the Euro
symbol in a way that suits Palatino using the appropriate glyphs
from the Pazo Math font family.  It is compatible with both the
\Lpack{eurofont} and \Lpack{europs} packages, and one can continue to use either
one of these packages, using the +\euro+ command from the
\Lpack{eurofont} package or the +\EUR+ command from the \Lpack{europs} package.
\fi


\subsection{Font size of the `large' math symbols}
With \Lpack{mathpazo}, the `large' math symbols are automatically
scaled to fit the base font size.
In contrast to standard \LaTeX{} you need not
load the package \Lpack{exscale} for this purpose!


\subsection{Known problems}

In contrast to the standard CM fonts, the virtual \Lpack{mathpazo} fonts
do not provide any Greek letters in the math alphabet +\mathrm+.
Applying this math alphabet command to Greek letters
will result in garbage output.



\section{The package \Lpack{mathptmx}}
\label{sec:mathptmx}
Loading this package changes the default roman font family
to Times, and the virtual `mathptmx' fonts will be used
for math.
These virtual fonts are made up basically from Times Italic, with the
missing math symbols coming from CM, RSFS (for +\mathcal+) and
Adobe~Symbol.

\subsection{Package options}
\begin{decl}
  +[slantedGreek]+
\end{decl}
%% \NEWfeature{2001-06-04 v8.2}
When the package is loaded with this option,
uppercase Greek letters in math will be italic by default.

\subsection{New commands}
\begin{decl}
  +\upGamma+, +\upDelta+ \dots +\upOmega+
\end{decl}
\NEWfeature{2004-09-15 v9.2}
Regardless of the \texttt{slantedGreek} option,
these commands always yield upright uppercase Greek letters.
Upright \emph{lowercase} Greek is, however, not available.


\subsection{Font size of the `large' math symbols}
With \Lpack{mathptmx}, the `large' math symbols are automatically scaled
to fit the base font size.  In contrast to standard \LaTeX{} you need not
load the package \Lpack{exscale} for this purpose!


\subsection{Known bugs and deficiencies}
\begin{itemize}
  \item There are no bold math fonts, and +\boldmath+ has no effect.
  Use of the package \Lpack{bm} in conjunction with \Lpack{mathptmx}
  is not recommended.
  \item The symbols
   +\jmath+, +\coprod +and +\amalg+
   are not available.
\end{itemize}



\section{The package \Lpack{pifont}}
\label{sec:pifont}

Using symbol fonts is supported by means of the
\Lpack{pifont} package, providing commands for using the Zapf Dingbats font,
as well as an interface to other families.\footnote{%
This section was adopted, with minor changes,
from \cite{companion}, 1st ed.}

\subsection{Commands for using Zapf Dingbats}

\begin{decl}
  +\ding+ \arg{number}
\end{decl}

A given character can be chosen via the +\ding+ command.
Its parameter is an integer that specifies the character to be
typeset.  For example, +\ding{38}+ gives \ding{38};
see table~\vref{tab:dingbats}.

\begin{decl}
  +\begin{dinglist}+ \arg{number}\\
  +\begin{dingautolist}+ \arg{number}
\end{decl}

The +dinglist+ environment is a special itemized list.
The argument specifies the number of the character to be used
as the beginning of each item.  For example,
\begin{verbatim}
\begin{dinglist}{43}
  \item The first item in the list
  \item The second item in the list
  \item The third item in the list
\end{dinglist}
\end{verbatim}
prints
\begin{dinglist}{43}
  \item The first item in the list
  \item The second item in the list
  \item The third item in the list
\end{dinglist}

There also exists an environment +dingautolist+, which
allows you to build an enumerated list with a set of Zapf Dingbats
characters.  In this case, the argument specifies the number
of the first character in the list.  Subsequent items will be numbered
with the character following the previous one.  E.g.,
\begin{verbatim}
\begin{dingautolist}{192}
  \item The first item
  \item The second item
  \item The third item
\end{dingautolist}
\end{verbatim}
prints
\begin{dingautolist}{192}
  \item The first item
  \item The second item
  \item The third item
\end{dingautolist}

\begin{decl}
  +\dingfill+ \arg{number}\\
  +\dingline+ \arg{number}
\end{decl}

+\dingfill+ acts like the other filling commands in \TeX, but
fills the space with a chosen symbol \dingfill{224} like that.

+\dingline+ generates a freestanding line filled with the given symbol,
with a little space on the left and right:
\dingline{34}


\begin{table}[bt!]
  \caption{The characters in the \ps{} font Zapf Dingbats}
  \label{tab:dingbats}
  \medskip

{\footnotesize
\begin{tabular}{|rr|rr|rr|rr|rr|rr|rr|rr|}
\hline
32 &  \ding{32} & 33 &  \ding{33} & 34 &  \ding{34} & 35 &  \ding{35} & 36 &  \ding{36} & 37 &  \ding{37} & 38 &  \ding{38} & 39 &  \ding{39}  \\ \hline
40 &  \ding{40} & 41 &  \ding{41} & 42 &  \ding{42} & 43 &  \ding{43} & 44 &  \ding{44} & 45 &  \ding{45} & 46 &  \ding{46} & 47 &  \ding{47}  \\ \hline
48 &  \ding{48} & 49 &  \ding{49} & 50 &  \ding{50} & 51 &  \ding{51} & 52 &  \ding{52} & 53 &  \ding{53} & 54 &  \ding{54} & 55 &  \ding{55}  \\ \hline
56 &  \ding{56} & 57 &  \ding{57} & 58 &  \ding{58} & 59 &  \ding{59} & 60 &  \ding{60} & 61 &  \ding{61} & 62 &  \ding{62} & 63 &  \ding{63}  \\ \hline
64 &  \ding{64} & 65 &  \ding{65} & 66 &  \ding{66} & 67 &  \ding{67} & 68 &  \ding{68} & 69 &  \ding{69} & 70 &  \ding{70} & 71 &  \ding{71}  \\ \hline
72 &  \ding{72} & 73 &  \ding{73} & 74 &  \ding{74} & 75 &  \ding{75} & 76 &  \ding{76} & 77 &  \ding{77} & 78 &  \ding{78} & 79 &  \ding{79}  \\ \hline
80 &  \ding{80} & 81 &  \ding{81} & 82 &  \ding{82} & 83 &  \ding{83} & 84 &  \ding{84} & 85 &  \ding{85} & 86 &  \ding{86} & 87 &  \ding{87}  \\ \hline
88 &  \ding{88} & 89 &  \ding{89} & 90 &  \ding{90} & 91 &  \ding{91} & 92 &  \ding{92} & 93 &  \ding{93} & 94 &  \ding{94} & 95 &  \ding{95}  \\ \hline
96 &  \ding{96} & 97 &  \ding{97} & 98 &  \ding{98} & 99 &  \ding{99} & 100 &  \ding{100} & 101 &  \ding{101} & 102 &  \ding{102} & 103 &  \ding{103}  \\ \hline
104 &  \ding{104} & 105 &  \ding{105} & 106 &  \ding{106} & 107 &  \ding{107} & 108 &  \ding{108} & 109 &  \ding{109} & 110 &  \ding{110} & 111 &  \ding{111}  \\ \hline
112 &  \ding{112} & 113 &  \ding{113} & 114 &  \ding{114} & 115 &  \ding{115} & 116 &  \ding{116} & 117 &  \ding{117} & 118 &  \ding{118} & 119 &  \ding{119}  \\ \hline
120 &  \ding{120} & 121 &  \ding{121} & 122 &  \ding{122} & 123 &  \ding{123} & 124 &  \ding{124} & 125 &  \ding{125} & 126 &  \ding{126} &     &              \\ \hline
    &             & 161 &  \ding{161} & 162 &  \ding{162} & 163 &  \ding{163} & 164 &  \ding{164} & 165 &  \ding{165} & 166 &  \ding{166} & 167 &  \ding{167}  \\ \hline
168 &  \ding{168} & 169 &  \ding{169} & 170 &  \ding{170} & 171 &  \ding{171} & 172 &  \ding{172} & 173 &  \ding{173} & 174 &  \ding{174} & 175 &  \ding{175}  \\ \hline
176 &  \ding{176} & 177 &  \ding{177} & 178 &  \ding{178} & 179 &  \ding{179} & 180 &  \ding{180} & 181 &  \ding{181} & 182 &  \ding{182} & 183 &  \ding{183}  \\ \hline
184 &  \ding{184} & 185 &  \ding{185} & 186 &  \ding{186} & 187 &  \ding{187} & 188 &  \ding{188} & 189 &  \ding{189} & 190 &  \ding{190} & 191 &  \ding{191}  \\ \hline
192 &  \ding{192} & 193 &  \ding{193} & 194 &  \ding{194} & 195 &  \ding{195} & 196 &  \ding{196} & 197 &  \ding{197} & 198 &  \ding{198} & 199 &  \ding{199}  \\ \hline
200 &  \ding{200} & 201 &  \ding{201} & 202 &  \ding{202} & 203 &  \ding{203} & 204 &  \ding{204} & 205 &  \ding{205} & 206 &  \ding{206} & 207 &  \ding{207}  \\ \hline
208 &  \ding{208} & 209 &  \ding{209} & 210 &  \ding{210} & 211 &  \ding{211} & 212 &  \ding{212} & 213 &  \ding{213} & 214 &  \ding{214} & 215 &  \ding{215}  \\ \hline
216 &  \ding{216} & 217 &  \ding{217} & 218 &  \ding{218} & 219 &  \ding{219} & 220 &  \ding{220} & 221 &  \ding{221} & 222 &  \ding{222} & 223 &  \ding{223}  \\ \hline
224 &  \ding{224} & 225 &  \ding{225} & 226 &  \ding{226} & 227 &  \ding{227} & 228 &  \ding{228} & 229 &  \ding{229} & 230 &  \ding{230} & 231 &  \ding{231}  \\ \hline
232 &  \ding{232} & 233 &  \ding{233} & 234 &  \ding{234} & 235 &  \ding{235} & 236 &  \ding{236} & 237 &  \ding{237} & 238 &  \ding{238} & 239 &  \ding{239}  \\ \hline
    &             & 241 &  \ding{241} & 242 &  \ding{242} & 243 &  \ding{243} & 244 &  \ding{244} & 245 &  \ding{245} & 246 &  \ding{246} & 247 &  \ding{247}  \\ \hline
248 &  \ding{248} & 249 &  \ding{249} & 250 &  \ding{250} & 251 &  \ding{251} & 252 &  \ding{252} & 253 &  \ding{253} & 254 &  \ding{254} &     &              \\ \hline
\end{tabular}
\par}

\end{table}

\subsection{Generic commands}
The \Lpack{pifont} package has a general mechanism for coping with
Pi fonts. It provides the following generic commands with, in each case,
the first argument \m{family} specifying the name of the
Pi font family in question (such as \texttt{psy} for the Symbol font,
and \texttt{pzd} for the Zapf Dingbats font, see table~\vref{tab:families}).
If indicated, a second argument \m{number}
specifies the decimal position of a symbol in that font.

\begin{decl}
  +\Pifont+ \arg{family}
\end{decl}

This switches to the font family \m{family}
and the encoding U.

\begin{decl}
  +\Pisymbol+ \arg{family} \arg{number}
\end{decl}

This command typesets the specified symbol
(compare this with the +\ding+ command).

\begin{decl}
  +\begin{Pilist}+ \arg{family} \arg{number}\\
  +\begin{Piautolist}+ \arg{family} \arg{number}
\end{decl}

In the +Pilist+ environment the specified symbol is used in front
of each item in an itemized list (compare with the +dinglist+
environment).

+Piautolist+ is an environment where a series of symbols starting
with the one at the decimal position \m{number} in font family
\m{family} is used to number the items in an enumerated list
(compare with the +dingautolist+ environment).

\begin{decl}
  +\Pifill+ \arg{family} \arg{number}\\
  +\Piline+ \arg{family} \arg{number}
\end{decl}

+\Pifill+ acts like the other filling commands in \TeX, but
fills the space with a chosen symbol (compare with +\dingfill+).

+\Piline+ typesets a line consisting of several copies of
the specified symbol, with some space at the left and right
(compare with +\dingline+).



\section{NFSS classification}
Table~\vref{tab:families} lists all text and symbol font shapes
supported by the basic PSNFSS distribution,
and the related \ps\ fonts.
\NEWdescription{2004-09-15 v9.2}
With the exception of Charter and Utopia, these fonts are commercial products.
Therefore, most \TeX\ systems include free substitutes instead.

Available encodings are OT1, T1 and TS1, except for
Symbol and Zapf~Dingbats, which are implemented with encoding U.
See \cite{fntguide} for how to access a given font shape directly.

Only the font families +pplx+ and +pplj+ provide true small capitals
(and only in the regular series).
With the other families the shape `sc' refers to so-called `faked' small capitals,
whose typographical quality is -- at least -- questionable.

The math font families loaded by the
\Lpack{mathptm}, \Lpack{mathptmx}, \Lpack{mathpazo} and \Lpack{mathpple} packages
are not listed here.
See the documented source file \texttt{psfonts.dtx}
for information on this topic.

\begin{table}[p]
  \caption{Font shapes supported by the basic PSNFSS distribution}
  \label{tab:families}
  \medskip

  {\small
  \begin{tabularx}{\linewidth}{|l|l|l|>{\raggedright\arraybackslash}X|}
  \hline
  \textbf{family} & \textbf{series} & \textbf{shape(s)} & \textbf{\ps{} font names}\\
  \hline\hline
  \multicolumn{4}{|c|}{\textit{Avant Garde}}\\ \hline
   pag & m & n, sl, sc & AvantGarde-Book, AvantGarde-BookOblique\\ \hline
   pag & b & n, sl, sc & AvantGarde-Demi, AvantGarde-DemiOblique\\ \hline \hline
  \multicolumn{4}{|c|}{\textit{Bookman}}\\ \hline
   pbk & l & n, sl, it, sc & Bookman-Light, Bookman-LightItalic\\ \hline
   pbk & db & n, sl, it, sc & Bookman-Demi, Bookman-DemiItalic\\ \hline \hline
  \multicolumn{4}{|c|}{\textit{Charter}}\\ \hline
   bch & m & n, sl, it, sc & CharterBT-Roman, CharterBT-Italic\\ \hline
   bch & b & n, sl, it, sc & CharterBT-Bold, CharterBT-BoldItalic\\ \hline \hline
  \multicolumn{4}{|c|}{\textit{Courier}}\\ \hline
   pcr & m & n, sl, sc & Courier, CourierOblique\\ \hline
   pcr & b & n, sl, sc & Courier-Bold, Courier-BoldOblique\\ \hline \hline
  \multicolumn{4}{|c|}{\textit{Helvetica}}\\ \hline
   phv & m & n, sl, sc & Helvetica, Helvetica-Oblique\\ \hline
   phv & b & n, sl, sc & Helvetica-Bold, Helvetica-BoldOblique\\ \hline
   phv & mc & n, sl, sc & Helvetica-Narrow, Helvetica-Narrow-Oblique\\ \hline
   phv & bc & n, sl, sc & Helvetica-Narrow-Bold, Helvetica-Narrow-BoldOblique\\\hline \hline
  \multicolumn{4}{|c|}{\textit{New Century Schoolbook}}\\ \hline
   pnc & m & n, sl, it, sc & NewCenturySchlbk-Roman, NewCenturySchlbk-Italic\\ \hline
   pnc & b & n, sl, it, sc & NewCenturySchlbk-Bold, NewCenturySchlbk-BoldItalic\\ \hline \hline
  \multicolumn{4}{|c|}{\textit{Palatino}}\\ \hline
   ppl & m & n, sl, it, sc & Palatino-Roman, Palatino-Italic\\ \hline
   ppl & b & n, sl, it, sc & Palatino-Bold, Palatino-BoldItalic\\ \hline
   pplx & m & n, it, sc & Palatino-Roman, Palatino-Italic, Palatino-SC\\ \hline
   pplx & b & n, it     & Palatino-Bold, Palatino-BoldItalic\\ \hline
   pplj & m & n, it, sc & Palatino-Roman, Palatino-SC, Palatino-Italic, Palatino-ItalicOsF\\ \hline
   pplj & b & n, it     & Palatino-Bold, Palatino-BoldOsF, Palatino-BoldItalic, Palatino-BoldItalicOsF\\ \hline \hline
  \multicolumn{4}{|c|}{\textit{Times}}\\ \hline
   ptm & m & n, sl, it, sc  & Times-Roman, Times-Italic\\ \hline
   ptm & b & n, sl, it, sc  & Times-Bold, Times-BoldItalic\\ \hline \hline
  \multicolumn{4}{|c|}{\textit{Zapf Chancery}}\\ \hline
   pzc & mb & it & ZapfChancery-MediumItalic\\ \hline \hline
  \multicolumn{4}{|c|}{\textit{Utopia}}\\ \hline
   put & m & n, sl, it, sc & Utopia-Regular, Utopia-Italic\\ \hline
   put & b & n, sl, it, sc & Utopia-Bold, Utopia-BoldItalic\\ \hline \hline
  \multicolumn{4}{|c|}{\textit{Symbol}}\\ \hline
   psy & m & n & Symbol\\ \hline \hline
  \multicolumn{4}{|c|}{\textit{Zapf Dingbats}}\\ \hline
   pzd & m & n & ZapfDingbats\\ \hline
  \end{tabularx}
  \par}
\end{table}



\section{Obsolete packages}
The macro packages listed in table \vref{tab:obsolete} should be considered as obsolete.
They are provided for compatibility with existing documents only.

\begin{table}[hbt]
\caption{Obsolete packages in the PSNFSS collection}
\label{tab:obsolete}
\medskip
{\footnotesize
\begin{center}
\renewcommand{\arraystretch}{1.2}
\begin{tabular}{|l|p{1.8cm}p{2.2cm}p{2.4cm}p{2.2cm}|}
\hline
\textbf{package} & \textbf{roman}  & \textbf{sans serif} & \textbf{typewriter} & \textbf{math} \\\hline\hline
\Lpack{times}    & Times
                 & Helvetica
                 & Courier
                 & \\\hline
\Lpack{palatino} & Palatino
                 & Helvetica
                 & Courier
                 & \\\hline
\Lpack{mathptm}  & Times
                 &
                 &
                 & $\approx$ Times\\\hline
\Lpack{mathpple} & Palatino
                 &
                 &
                 & $\approx$ Palatino\\\hline
\Lpack{utopia}   & Utopia
                 &
                 &
                 & \\\hline
\end{tabular}
\end{center}
}
\end{table}


\subsection{The packages \Lpack{times} and \Lpack{palatino}}
These packages do not load suitable math fonts,
and they do not scale the Helvetica fonts appropriately
to match Times and Palatino -- see section \ref{sec:helvet}.
Use \Lpack{mathptmx} or \Lpack{mathpazo} in conjunction with
\Lpack{helvet} and \Lpack{courier} instead!

In case you need to load Times or Palatino \emph{without} the
related math fonts of the PSNFSS bundle, you can still use the
basic NFSS commands.  For instance,
\begin{quote}
+\renewcommand{\rmdefault}{ptm}+
\end{quote}
changes only the default roman font family to
\texttt{ptm}, i.e.\ Times.

\subsection{The package \Lpack{mathptm}}
The package \Lpack{mathptm} is a predecessor to \Lpack{mathptmx}.
In contrast to the latter and to \LaTeX's standard behavior,
lowercase Greek in math is typeset upright.
Zapf Chancery is used as the calligraphic math alphabet,
which causes some problems with proper spacing.
\Lpack{mathptm} needs the font \texttt{cmex9}, which may not
be available in Type~1 format.

\subsection{The package \Lpack{mathpple}}
\Lpack{mathpple} is a predecessor to \Lpack{mathpazo},
using also a set of virtual math fonts to go with Palatino.
The Greek alphabet is, however, taken from the
Euler fonts (which get slanted), rather than from the Pazo fonts.
The package \Lpack{mathpple} does not support the
Palatino SC/OsF fonts, and there is no `blackboard bold' math alphabet.
Further flaws are:
\begin{itemize}
  \item The spacing within numbers and function names in formulas
   is somewhat too loose.
  \item The +\coprod+ symbol is missing.
  \item There are no bold variants of +\partial+ and +\infty+.
  \item +\jmath+ is taken from the CM math italic font, which does
  not blend well with Palatino.
  \item DVI viewers may exhibit problems as to rendering of the artificially slanted
  Greek letters.
\end{itemize}
The newer \Lpack{mathpazo} package can  be considered as
superior; yet you may still use \Lpack{mathpple}, if you prefer
the shape of its Greek letters.

\subsection{The package \Lpack{utopia}}
Use %%\NEWdescription{2003-11-17 v9.1}
of the \Lpack{utopia}
package is no longer recommended, because
the newer package \Lpack{fourier} provides a basically improved
interface to the Utopia typeface and loads suitable math fonts, too.
Notice that this package does \emph{not} belong to the PSNFSS collection!

Furthermore, the \LaTeX3 team does no longer regard the Utopia
fonts as a \emph{required} component of \LaTeX{},
because their license does not comply with the strict
guidelines of the FSF.


\section{Typeface samples}

The following samples show the regular font of each typeface family
supported by PSNFSS.
The particular font size and baselineskip is indicated below the font name.
Notice that Helvetica is scaled to 92\,\% of the nominal size.

\medskip

\sample{Times}{ptm}{10}{12pt}{}

\sample{Palatino}{pplx}{10}{12.4pt}{}

\sample{Bookman}{pbk}{9.6}{11.5pt}{}

\sample{Charter}{bch}{10}{12.4pt}{}

\sample{New Century Schoolbook}{pnc}{9.6}{12pt}{}

\ifutopia
  \sample{Utopia}{put}{9.6}{12pt}{}
\fi

\sample{Helvetica}{phv}{10}{12pt}{}

\parbox[t]{\leftwidth}{%
  {\raggedright\footnotesize
  \mbox{}Avant\-Garde\\
  9.6pt
  \par}}
\hfill
\parbox[t]{\rightwidth}{
{\fontfamily{pag}\fontsize{9.6}{11.5pt}\selectfont\raggedright
 Don't use Avant Garde for typesetting larger portions of text\,!
\par}}\medskip

\parbox[t]{\leftwidth}{
  {\raggedright\footnotesize
  Courier\\
  10/12pt
  \par}}
\hfill
\parbox[t]{\rightwidth}{
{\fontfamily{pcr}\fontsize{10}{12pt}\selectfont\raggedright
 A monospaced typeface, suitable for typesetting filenames, URLs etc.
\par}}\medskip

\parbox[t]{\leftwidth}{
  {\raggedright\footnotesize
  Zapf Chancery\\
  14.4pt
  \par}}
\hfill
\parbox[t]{\rightwidth}{
{\fontfamily{pzc}\Large\raggedright
To Hermann Zapf -- whose strokes are the best.
\par}}\medskip



\section*{Credits}
The PSNFSS system was originally developed by Sebastian Rahtz.

The virtual \Lpack{mathptm} and \Lpack{mathptmx} fonts and the related packages
were created by Alan Jeffrey, Sebastian Rathz and Ulrik Vieth.

The \Lpack{mathpple} package and its virtual fonts are based
on earlier work by Aloysius Helminck. Special thanks to Daniel Schlieper
without whose initiative the package would not have been developed.

The Pazo math fonts and the related virtual fonts were created
by Diego Puga.



\begin{thebibliography}{1}
\raggedright

\bibitem{companion}
Frank Mittelbach et al.:
\textit{The LaTeX Companion}.\\
2nd edition. Addison Wesley, 2004.

%\bibitem{gcompanion}
%Michel Goossens, Sebastian Rahtz, and
%Frank Mittelbach:\\
%\textit{The LaTeX Graphics Companion}.\\
%Addison Wesley Longman, 1997.

\bibitem{fntguide}
\LaTeX3 Project Team (Ed.):
\textit{LaTeX2e font selection.}\\
CTAN: \path{macros/latex/doc/fntguide.pdf}\\
(Part of the \LaTeX{} online documentation)
\end{thebibliography}


\end{document}
