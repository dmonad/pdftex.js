\documentclass[11pt]{book}
\makeatletter
\def\listoffigures{\@restonecolfalse\if@twocolumn\@restonecoltrue\onecolumn\fi
\chapter*{List of Figures}
{\let\\ \ \markboth{Title}{List of Figures}}
\addcontentsline{toc}{chapter}{\protect
 \numberline{}LIST OF FIGURES}\@starttoc{lof}\if@restonecol \twocolumn\fi}
\makeatother
\usepackage{epsfig,longtable}
\usepackage{makeidx}
\usepackage{varioref}
\usepackage{xr-hyper}
\usepackage{amsmath}
\usepackage[verbose,hypertexnames=false,bookmarksopenlevel=1,bookmarksopen,bookmarksnumbered,colorlinks,plainpages=false,linktocpage]{hyperref}
\def\boldindex#1{\textbf{\hyperpage{#1}}}
\makeindex
\makeatletter
\externaldocument{test2}
\makeatother
%\setcounter{page}{34}
\title{Testing Hyperref}
\author{Sebastian Rahtz}
\date{May 1999}
\setcounter{secnumdepth}{2}
\setcounter{tocdepth}{1}
\begin{document}
\pagenumbering{roman}
%
\makeatletter
\long\def\hyper@section@backref#1#2#3{%
 \typeout{BACK REF #1 / #2 / #3}%
\hyperlink{#3}{#2}}
%
\makeatother
\pdfbookmark{Title}{tit}
\setlongtables
\maketitle\index{title}
\tableofcontents
\listoftables
\listoffigures
\pagenumbering{arabic}
\newcommand{\eqnref}[1]{{\autoref{#1}}}
\newtheorem{Argument}{Here we go}[section]
\def\thesubsection{\Roman{section} -- \arabic{subsection}}
\setcounter{tocdepth}{1}

\chapter[First part]{First part, leading to \protect\autoref{horrid}
  next}
\section[Test section]{Our \protect\LaTeX\ test section (leading to \autoref{One})
 for 100\% of \AE horrid $X[Y]Z$ 
things, like $42$\label{horrid}}
and so see \autoref{horrid} on page \pageref{horrid}.

\section{Section One --- cats}\label{One}
see section \vref{Three} about
cats\index{animals!cats} and cite 
\cite{Barcelo:1992:caa}

\section{one.1 -- can we see}
some text with a footnote\footnote{WISH UPON A STAR}
and another one with an extended 
footnote\footnote{\protect\label{foot}This is the way the world ends not with a
  bang but a whimper. This is the way the world ends not with a
  bang but a whimper. This is the way the world ends not with a
  bang but a whimper.}
and a reference to a long table\index{tables!long!longtables},
\autoref{mylongtab}.

\subsection{one.2}
dogs

\href{file:test2#page.2}{See page  2 in file test2}, 
on page 3 of this file.

\href{file:test2.pdf#page.2}{See page 2 in file test2.pdf} 
on page 3 of this file.

And can we  see \autoref{ss:first}

in the file test2.pdf? alternatively,
\href{file:test2.pdf#subsection.1.2}{the link like this}

All Or this? \url{test2.pdf#section.1}

\newpage
\section{Section Two --- \TeX\ is a dog}
\subsection{two.1}
\subsection{two.2}
\newpage
cite \cite{Barcelo:1992:caa} again.

\chapter{Second part}
\section{Section Three --- Camels}\label{Three}
see \autoref{One}
\subsection{three.1}
 some text with a footnote\footnote{OVER THE RAINBOW}
\index{rainbows}
\subsection{three.2}

\newpage
\section[Section Four --- Butterflies]{Section Four --- Butterflies and so on}
\subsection{four.1}
\subsection{four.2}
camels
Refer to \hyperref{}{test}{test1}{with these words}
\newpage
\section{Introduction}\label{sec1}
\subsection{subsec}
\newpage
\subsection{subsec}\label{subsec1.2}
Define a marker \hyperdef{test}{test1}{here} while this one is a
PostScript picture acting as marker:
\index{PS pictures}
\hyperlink{testpiccy}{\epsfig{figure=picture,height=1in}}

This is a picture: \epsfig{figure=picture,height=1in}
\newpage
\section{two}\label{sec2}
\subsection{Subsection 2}
\subsection{Subsection 3}
\newpage
\section{three}
This is a reference to section 1 (\autoref{sec1}), subsection 1.2
(\autoref{subsec1.2}) and section 2 (\autoref{sec2}). References to
\cite{Barcelo:1992:caa,Dallas:aia}.


\begin{figure}

xxxx

\hypertarget{testpiccy}{Test picture}

xxxxx

xxxxx
\caption{{A cat}}
\label{fig1}
\end{figure}

\begin{figure}

xxxx

xxxxx
\caption{Another cat with a link inside it,
so see  \cite{Dallas:aia} xxxx \label{fig2}}
\end{figure}

\texttt{<<where is \eqnref{eq1}>>}

\section{Some URLs}

\begin{minipage}{1.5in}
\url{http://www.aw.com/cp/tlgc.html#Describe}

\url{http://nsi.net.kiae.su/latex/latex2e.html}

\url{http://www.lehigh.edu/~dlj0/LyriX.html}

\url{http://www.cs.wisc.edu/~ghost/index.html}

\url{http://www.win.tue.nl/win/math/dw/personalpages/dickie/idvi/}

\url{http://www.tug.org/interest.html#projects}

\url{ftp://ftp.cbr.dit.csiro.au/staff/gjw/www/tex.html}


This is a URL: \url{http://srahtz/attend.html#sebastian}

\hyperref{file:test2.pdf}{equation}{1}{hello}
\end{minipage}

\newpage

\section{Back to math}
\begin{equation}
  zzzz + b
  \label{eq1}
\end{equation}
and what next?

\begin{equation}
  d - e
  \label{eq2}
\end{equation}

\begin{eqnarray}
  y &=&z\\
  g &=&h\\
  \label{eq3}
\end{eqnarray}
We need some lists:
\begin{enumerate}
\item oranges\index{oranges|boldindex}
\item lemons\index{lemons|boldindex}
\item beer\index{beer|boldindex}
  \begin{enumerate}
  \item Samuel Smiths
  \item Labatts
  \end{enumerate}
\end{enumerate}

Lets look at labels in lists:
\begin{enumerate}
\item oranges\label{oranges}
\item lemons\label{lemons}
\item beer\label{beer}
  \begin{enumerate}
  \item Samuel Smiths\label{smiths}
  \item Labatts\label{labatts}
  \end{enumerate}
\end{enumerate}
\clearpage

from which see \autoref{oranges}, \ref{lemons}, \ref{smiths} and
\autoref{labatts}


see
sec1: \autoref{sec1}
sec2: \autoref{sec2}
eq1: \autoref{eq1}
fig1: \autoref{fig1}
and cite \cite{Barcelo:1992:caa} again.
\onecolumn
\begin{longtable}{lll}
\caption{A test long table (see \protect\cite{Dallas:aia} 
and section \protect\ref{sec1}}\label{mylongtab}\\
a & b & c \\a
a & b & c \\a
a & b & c \\a
a & b & c \\a
a & b & c \\a
a & b & c \\a
a & b & c \\a
a & b & c \\a
a & b & c \\a
a & b & c \\a
a & b & c \\a
a & b & c \\a
a & b & c \\a
a & b & c \\a
a & b & c \\a
a & b & c \\a
a & b & c \\a
a & b & c \\a
a & b & c \\a
a & b & c \\a
a & b & c \\a
a & b & c \\a
a & b & c \\a
a & b & c \\a
a & b & c \\a
a & b & c \\a
a & b & c \\a
a & b & c \\a
a & b & c \\a
a & b & c \\a
a & b & c \\a
a & b & c \\a
a & b & c \\a
a & b & c \\a
a & b & c \\a
a & b & c \\a
a & b & c \\a
a & b & c \\a
a & b & c \\a
a & b & c \\a
a & b & c \\a
a & b & c \\a
a & b & c \\a
a & b & c \\a
a & b & c \\a
a & b & c \\a
a & b & c \\a
a & b & c \\a
a & b & c \\a
a & b & c \\a
a & b & c \\a
a & b & c \\a
a & b & c \\a
a & b & c \\a
\end{longtable}

% !!! Does not work with hypertexnames=false !!!
Does \hyperref{}{equation}{2.2}{this} point to the second equation?

Does anything point to the eqnarray (\autoref{eq3})?
\index{cats}

\begin{thebibliography}{99}
\addcontentsline{toc}{chapter}{Bibliography}
\bibitem{Barcelo:1992:caa}
{Barcel\'o, J.} 1992.
\newblock Programming an intelligent database in archaeology. In \emph{Computer
  Applications and Quantitative Methods in Archaeology 1991}, {Lock, G. \&
  J.~Moffett} (eds),   21--28, Oxford: British Archaeological Reports.

\bibitem[Dallas 1992]{Dallas:aia}
{Dallas, C.~J.} 1992.
\newblock Syntax and semantics of figurative art: a formal approach. In
  \emph{Archaeology and the Information Age}, {Reilly, P. \& S.~Rahtz} (eds),
  chapter~16, London: Routledge.

\bibitem[Stankovic 1988]{stankovic:88}
J.~Stankovic, ``Misconceptions about real-time computing: a serious problem for
  next-generation systems,'' {\em Computer}, vol.~21, no.~10, pp.~10--19, Oct.
  1988.

\end{thebibliography}

\clearpage
An index entry for gnus\index{gnus}
\clearpage
An index entry for gnus\index{gnus}
\clearpage
An index entry for gnus\index{gnus}
\clearpage
An index entry for gnus\index{gnus}
\clearpage
An index entry for gnus\index{gnus}
\clearpage
An index entry for gnus\index{gnus}
\chapter*{An appendix --- the Index}
\printindex
\end{document}
