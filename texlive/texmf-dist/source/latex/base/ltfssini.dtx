% \iffalse meta-comment
%
% Copyright 1993-2016
% The LaTeX3 Project and any individual authors listed elsewhere
% in this file.
%
% This file is part of the LaTeX base system.
% -------------------------------------------
%
% It may be distributed and/or modified under the
% conditions of the LaTeX Project Public License, either version 1.3c
% of this license or (at your option) any later version.
% The latest version of this license is in
%    http://www.latex-project.org/lppl.txt
% and version 1.3c or later is part of all distributions of LaTeX
% version 2005/12/01 or later.
%
% This file has the LPPL maintenance status "maintained".
%
% The list of all files belonging to the LaTeX base distribution is
% given in the file `manifest.txt'. See also `legal.txt' for additional
% information.
%
% The list of derived (unpacked) files belonging to the distribution
% and covered by LPPL is defined by the unpacking scripts (with
% extension .ins) which are part of the distribution.
%
% \fi
% \iffalse
%%% From File: ltfssini.dtx
%% Copyright (C) 1989-2001 Frank Mittelbach and Rainer Sch\"opf,
%% all rights reserved.
%
%<*driver>
% \fi
%
%
\ProvidesFile{ltfssini.dtx}
             [2015/01/09 v3.1a LaTeX Kernel (NFSS Initialisation)]
% \iffalse
\documentclass{ltxdoc}
\begin{document}
\DocInput{ltfssini.dtx}
\end{document}
%</driver>
% \fi
%
% \iffalse
%<+checkmem>\def\CHECKMEM{\tracingstats=2
%<+checkmem>  \newlinechar=`\^^J
%<+checkmem>  \message{^^JMemory usage: \filename}\shipout\hbox{}}
%<+checkmem>\CHECKMEM
% \fi
%
%
%
% \GetFileInfo{ltfssini.dtx}
% \title{A new font selection scheme for \TeX{} macro packages\\
%        (Initialisation)\thanks
%       {This file has version number
%       \fileversion\ dated \filedate}}
%
% \author{Frank Mittelbach \and Rainer Sch\"opf}
%
% \MaintainedByLaTeXTeam{latex}
% \maketitle
%
% This file contains the top level \LaTeX\ interface to the font
% selection scheme commands. See other parts of the \LaTeX\
% distribution, or \emph{The \LaTeX\ Companion} for higher level
% documentation of these commands.
%
% \StopEventually{}
%
%
% \changes{v3.0i}{1998/08/17}{(RmS) Minor documentation fixes.}
% \changes{v3.0b}{1995/06/28}
%      {(DPC) Fix documentation typos}
% \changes{v3.0a}{1995/05/24}
%      {(DPC) Make file from previous file, lfonts.dtx 1995/05/23 v2.2e}
%
%
%
% \section{NFSS Initialisation}
%
% \iffalse
%<+checkmem>\CHECKMEM
% \fi
%
%
% \changes{v2.2a}{1994/10/14}{New coding for cfg files}
% \changes{v2.1a}{1993/12/03}{update for LaTeX2e}
% \changes{v1.2c}{1992/01/06}{added slitex code}
% \changes{v1.2d}{1992/03/21}{Renamed \cs{text} to \cs{nfss@text}
%                            to make it internal.}
% \changes{v1.2a}{1991/11/27}{All \cs{family}, \cs{shape} etc. renamed
%                        to \cs{fontfamily} etc.}
% \changes{v1.1i}{1990/04/02}{\cs{input} of files now handled
%                          by docstrip.}
% \changes{v1.1g}{1990/02/08}{Protected the commands
%         \cs{family}, \cs{series},
%         \cs{shape}, \cs{size}, \cs{selectfont}, and \cs{mathversion}.}
% \changes{v1.1c}{1989/12/03}{Some internal macros renamed to make them
%           inaccessible.}
% \changes{v1.1b}{1989/12/02}{\cs{rmmath} renamed to \cs{mathrm}}
%
% \changes{v1.0i}{1989/11/07}{All family, series, and shape names
%           abbreviated.}
% \changes{v1.0g}{1989/05/22}{Lines shortened to 72 characters}
% \changes{v1.0f}{1989/04/29}{Corrections to \LaTeX\ tabular env.
%                         added.}
% \changes{v1.0e}{1989/04/27}{Definitions of \LaTeX\ symbols corrected.}
% \changes{v1.0d}{1989/04/26}{\cs{xpt} added.}
% \changes{v1.0c}{1989/04/21}{Changed to conform to fam.tex.}
% \changes{v1.0b}{1989/04/15}{\cs{mathfontset} renamed to
%                                              \cs{mathversion.}}
% \changes{v1.0a}{1989/04/10}{Starting with version numbers!
%           \cs{newif} for \cs{@tempswa} added since this switch is
%           unknown at the time when this file is read in.
%           (latex.tex is loaded later.)
%           \cs{math@famname} changed to \cs{math@version.}}
%
%
% \changes{v2.1j}{1994/05/13}{Removed file identification typeout}
%
% Finally, there are six commands that are to be used in \LaTeX{}
% and that we will therefore protect against expansion at the
% wrong point:
% |\fontfamily|, |\fontseries|, |\fontshape|, |\fontsize|,
% |\selectfont|, and |\mathversion|.
%
% \changes{v2.1i}{1994/05/12}{Moved \cs{fontfamily} to fam.dtx}
% \changes{v2.1i}{1994/05/12}{Moved \cs{fontencoding} to fam.dtx}
% \changes{v2.1i}{1994/05/12}{Moved \cs{fontseries} to fam.dtx}
% \changes{v2.1i}{1994/05/12}{Moved \cs{fontshape} to fam.dtx}
% \changes{v2.1i}{1994/05/12}{Moved \cs{fontsize} to fam.dtx}
% \changes{v2.1i}{1994/05/12}{Moved \cs{mathversion} to fam.dtx}
% \changes{v2.1i}{1994/05/12}{Moved \cs{selectfont} to tracefnt.dtx}
%
%
%    \begin{macrocode}
%<*2ekernel>
%    \end{macrocode}
%
% \subsection{Providing math \emph{versions}}
%
%  \LaTeX{} provides two \emph{versions}. We call them
%  \textsf{normal} and \textsf{bold}, respectively.
%    \begin{macrocode}
\DeclareMathVersion{normal}
\DeclareMathVersion{bold}
%    \end{macrocode}
%
%
%    Now we define the standard font change commands.
%    We don't allow the use of |\rmfamily| etc.\ in math mode.
%
%    First the changes to another \emph{family}:
%    \begin{macrocode}
\DeclareRobustCommand\rmfamily
        {\not@math@alphabet\rmfamily\mathrm
         \fontfamily\rmdefault\selectfont}
\DeclareRobustCommand\sffamily
        {\not@math@alphabet\sffamily\mathsf
         \fontfamily\sfdefault\selectfont}
\DeclareRobustCommand\ttfamily
        {\not@math@alphabet\ttfamily\mathtt
         \fontfamily\ttdefault\selectfont}
%    \end{macrocode}
%    Then the commands changing the \emph{series}:
%    \begin{macrocode}
\DeclareRobustCommand\bfseries
        {\not@math@alphabet\bfseries\mathbf
         \fontseries\bfdefault\selectfont}
\DeclareRobustCommand\mdseries
        {\not@math@alphabet\mdseries\relax
         \fontseries\mddefault\selectfont}
\DeclareRobustCommand\upshape
        {\not@math@alphabet\upshape\relax
         \fontshape\updefault\selectfont}
%    \end{macrocode}
%    Then the commands changing the \emph{shape}:
%    \begin{macrocode}
\DeclareRobustCommand\slshape
        {\not@math@alphabet\slshape\relax
         \fontshape\sldefault\selectfont}
\DeclareRobustCommand\scshape
        {\not@math@alphabet\scshape\relax
         \fontshape\scdefault\selectfont}
\DeclareRobustCommand\itshape
        {\not@math@alphabet\itshape\mathit
         \fontshape\itdefault\selectfont}
%    \end{macrocode}
%
%
%
% \begin{macro}{\em}
% \changes{v1.2b}{1990/01/28}{Call to \cs{@nomath} added.}
% \changes{v3.1a}{2015/01/09}{Allow \cs{emph} to produce small caps (latexrelease)}
% \begin{macro}{\eminnershape}
% \changes{v3.1a}{2015/01/09}{macro added (latexrelease)}
% We also have to define the {\em emphasize\/} font change command
% (i.e.\ |\em|). This command will look is the current font is
% sloped (i.e.\ has a positive |\fontdimen1|) and will then
% select either |\upshape| or |\itshape|.
%    \begin{macrocode}
%</2ekernel>
%<latexrelease>\IncludeInRelease{2015/01/01}{\eminnershape}{\eminnershape}%
%<*2ekernel|latexrelease>
\DeclareRobustCommand\em
        {\@nomath\em \ifdim \fontdimen\@ne\font >\z@
                       \eminnershape \else \itshape \fi}%
%    \end{macrocode}
%
%    \begin{macrocode}
\def\eminnershape{\upshape}%
%</2ekernel|latexrelease>
%<latexrelease>\EndIncludeInRelease
%<latexrelease>\IncludeInRelease{0000/00/00}{\eminnershape}{\eminnershape}%
%<latexrelease>\DeclareRobustCommand\em
%<latexrelease>        {\@nomath\em \ifdim \fontdimen\@ne\font >\z@
%<latexrelease>                       \upshape \else \itshape \fi}%
%<latexrelease>\let\eminnershape\@undefined
%<latexrelease>\EndIncludeInRelease
%<*2ekernel>
%    \end{macrocode}
% \end{macro}
% \end{macro}
%
%
%  \begin{macro}{\not@math@alphabet}
%    This function generates an error message when it is called in
%    math mode. The same function should be defined in
%    \texttt{newlfont.sty}.
% \changes{v1.4d}{1992/09/21}{Macro defined.}
% \changes{v2.1e}{1994/01/17}{Message changed}
% \changes{v2.1f}{1994/01/18}{Message corrected}
% \changes{v2.1g}{1994/04/22}{Message changed again}
% \changes{v2.2d}{1995/04/02}{add \cs{noexpand} to second part of message}
%    \begin{macrocode}
\def\not@math@alphabet#1#2{%
   \relax
   \ifmmode
     \@latex@error{Command \noexpand#1invalid in math mode}%
        {%
         Please
         \ifx#2\relax
            define a new math alphabet^^J%
            if you want to use a special font in math mode%
          \else
%    \end{macrocode}
%    We have to a |\noexpand| below to prevent expansion of |#2|. In
%    case of |#1| we can omit this (due to the current definition of
%    robust commands since they do come out right there :-).
%    \begin{macrocode}
            use the math alphabet \noexpand#2instead of
            the #1command%
         \fi
         .
        }%
   \fi}
%    \end{macrocode}
%  \end{macro}
%
%
%
% Finally we provide two abbreviations to switch to the \LaTeX{}
% \emph{versions}.
%    \begin{macrocode}
\def\boldmath{\@nomath\boldmath
              \mathversion{bold}}
\def\unboldmath{\@nomath\unboldmath
              \mathversion{normal}}
%    \end{macrocode}
% Here we switch to the default math version by defining the internal
% macro |\math@version|. We dare not to call |\mathversion|
% at this place because this would call |\glb@settings|.
%    \begin{macrocode}
\def\math@version{normal}
%    \end{macrocode}
%
% \subsection{Miscellaneous}
%
% \begin{macro}{\newfont}
% \changes{v1.2g}{1991/03/30}{Definition added.}
% \changes{v2.2e}{1995/05/23}{Font assignment made local again.}
% \begin{macro}{\symbol}
% \changes{v1.2g}{1991/03/30}{Definition added.}
%    We start by defining a few macros that are part of
%    standard \LaTeX's user interface. The use of these functions is
%    not encouraged, but they will allow to process older documents
%    without changes to the source.
%    \begin{macrocode}
\def\newfont#1#2{\@ifdefinable#1{\font#1=#2\relax}}
\def\symbol#1{\char #1\relax}
%    \end{macrocode}
% \end{macro}
% \end{macro}
%
% \begin{macro}{\@setfontsize}
% \begin{macro}{\@setsize}
%    This abbreviation is used by \LaTeX's user level size changing
%    commands, such as |\large|.
%    \begin{macrocode}
\def\@setfontsize#1#2#3{\@nomath#1%
%    \end{macrocode}
%    For the benefit of people relying on keeping the name of the
%    current font command saved in |\@currsize| we define it. To ensure
%    that |\@setfontsize| keeps being robust we omit this assignment
%    during times where |\protect| differs from |\@typeset@protect|.
% \changes{v1.4b}{1992/08/20}{Added \cs{@currsize}.}
% \changes{v2.2b}{1994/11/06}{Use \cs{@typeset@protect}}
%    \begin{macrocode}
    \ifx\protect\@typeset@protect
      \let\@currsize#1%
    \fi
    \fontsize{#2}{#3}\selectfont}
%    \end{macrocode}
%    For compatibility  we also define |\@setsize| the 209 command
%    \begin{macrocode}
%<*compat>
\def\@setsize#1#2#3#4{\@setfontsize#1{#4}{#2}}
%</compat>
%    \end{macrocode}
% \end{macro}
% \end{macro}
%
%
%  \begin{macro}{\oldstylenums}
%    This macro implements old style numerals but only works if we
%    assume that the standard math fonts are used. Thus it needs
%    changing in case other math encodings are used.
%    \begin{macrocode}
\def\oldstylenums#1{%
   \begingroup
%    \end{macrocode}
%    Provide spacing using the interword space of the current font.
%    \begin{macrocode}
    \spaceskip\fontdimen\tw@\font
%    \end{macrocode}
%    Then switch to the math italic font. We don't change the current
%    value of |\f@series| which means that you can use bold numerals
%    if |\bfseries| is in force. As family we use |\rmdefault| which
%    means that this only works if there exist an |OML| encoded
%    version of that font or rather a corresponding |.fd| file (which
%    is the case for standard \LaTeX{} fonts even though they only
%    contain substitutions).
% \changes{v3.0j}{1999/02/12}{Use \cs{rmdefault} instead of \texttt{cmm}
%                 (pr/2954)}
%    \begin{macrocode}
    \usefont{OML}{\rmdefault}{\f@series}{it}%
    \mathgroup\symletters #1%
   \endgroup
}
%    \end{macrocode}
%  \end{macro}
%
% \begin{macro}{\hexnumber@}
%    To set up \LaTeX's special math character
%    definitions we first provide a macro to generate hexadecimal
%    numbers.  It is a rather simple |\ifcase|.
% \changes{v?}{1992/11/13}{Made expandable.}
% \changes{v?}{1992/12/03}{Make it accept counters.}
%    \begin{macrocode}
\def\hexnumber@#1{\ifcase\number#1
 0\or 1\or 2\or 3\or 4\or 5\or 6\or 7\or 8\or
 9\or A\or B\or C\or D\or E\or F\fi}
%    \end{macrocode}
%  \end{macro}
%
%
%
% \begin{macro}{\nfss@text}
% \changes{v1.1e}{1990/01/25}{Macro added.}
%    In it simplest form |\nfss@text| is an |\mbox|.  This will
%    produce unbreakable text outside math and inside math you will
%    get text with the same fonts as outside. The only drawback is
%    that such item won't change sizes in subscripts. But this
%    behavior can be easily changed. With the \texttt{amstex} style
%    option one will get a sub style called \texttt{amstext} which will
%    redefine the |\nfss@text| macro to produce correct text in all
%    sizes.
%
%    We have to use |\def| instead of the shorter |\let| since
%    |\mbox| is undefined when we reach this point.
% \changes{v1.1k}{1990/06/23}{Changed to \cs{mbox}.}
% \changes{v2.1n}{1994/05/17}{Added braces to allow use in subscripts}
%    \begin{macrocode}
\def\nfss@text#1{{\mbox{#1}}}
%    \end{macrocode}
% \end{macro}
%
% \begin{macro}{\copyright}
%    The definition of |\copyright| was changed so
%    that it works in other type styles,
%    and to make it robust. We leave the family untouched so that
%    the copyright notice will come out differently if a different
%    font family is in use.
%    This command is commented out, since it is now defined in
%    ltoutenc.dtx.
% \changes{v1.1m}{1991/03/28}{Extra braces added.}
% \changes{v2.1n}{1994/05/17}{Really add extra braces}
% \changes{v2.2c}{1994/12/02}{\cs{copyright} is now in ltoutenc.
%    ASAJ}
%    \begin{macrocode}
%\DeclareRobustCommand\copyright
%    {{\ooalign{\hfil
%     \raise.07ex\hbox{\mdseries\upshape c}\hfil\crcr
%     \mathhexbox20D}}}
%    \end{macrocode}
% \end{macro}
%
% \changes{v2.1a}{1993/11/24}{Removed \cs{xpt} stuff}
%
%
% \begin{macro}{\normalfont}
% \changes{v2.1a}{1993/11/11}{Macro added}
% \begin{macro}{\reset@font}
% \begin{macro}{\p@reset@font}
% \changes{v1.1n}{1991/08/26}{Macro introduced}
%    The macro |\reset@font| is used in \LaTeX{} to switch to a standard
%    font, in order to initialize the current font in situations where
%    typesetting is done in a new visual context (e.g.\ in a
%    footnote). We define it here to allow the test for the new
%    \LaTeX{} version above but nevertheless are able to run all kind
%    of mixtures.
% \changes{v1.1o}{1991/11/21}{Changed to protected version of macro.}
% \changes{v1.1o}{1991/11/21}{Added extra braces for robustness.}
%
%    The user interface name for |\reset@font| is |\normalfont|:
% \changes{v2.1k}{1994/05/14}{Remove surplus braces}
% \changes{v3.0f}{1995/10/16}{Added \cs{relax} after \cs{usefont},
%              as the latter eats up spaces.}
%    \begin{macrocode}
\DeclareRobustCommand\normalfont
                 {\usefont\encodingdefault
                          \familydefault
                          \seriesdefault
                          \shapedefault
                  \relax}
\let\reset@font\normalfont
%    \end{macrocode}
% \end{macro}
% \end{macro}
% \end{macro}
%
%
%
% We left out the special \LaTeX{} fonts which are not automatically
% included in the base version of the font selection since these fonts
% contain only a few characters which are also included in the AMS
% fonts so anybody who is using these fonts doesn't need them.
% But for compatibility reasons we will define these symbols.
%
% \changes{v2.1g}{1994/02/22}{Correct error message}
%    \begin{macrocode}
\def\not@base#1{\@latex@error
  {Command \noexpand#1not provided in base LaTeX2e}%
  {Load the latexsym or the amsfonts package to
   define this symbol}}
\def\mho{\not@base\mho}
\def\Join{\not@base\Join}
\def\Box{\not@base\Box}
\def\Diamond{\not@base\Diamond}
\def\leadsto{\not@base\leadsto}
\def\sqsubset{\not@base\sqsubset}
\def\sqsupset{\not@base\sqsupset}
\def\lhd{\not@base\lhd}
\def\unlhd{\not@base\unlhd}
\def\rhd{\not@base\rhd}
\def\unrhd{\not@base\unrhd}
%    \end{macrocode}
%
%
%
%    We now initialize all variables set by |\DeclareErrorFont|. These
%    values are not really important since they will be overwritten
%    later on by the definition in |fontdef.ltx|.
%
%    However, if \texttt{fontdef.cfg} is corrupted then at least a
%    hopefully suitable error font is present.
%
% \changes{v2.1k}{1994/05/14}{Init error font just before checking for
%                             fontdef.cfg}
%    \begin{macrocode}
\DeclareErrorFont{OT1}{cmr}{m}{n}{10}  %% don't modify this setting
                                       %% overwrite it in fontdef.cfg
                                       %% if necessary
%    \end{macrocode}
%
%
%
%
% We now load the customizable parts of NFSS.
% \changes{v3.0d}{1995/07/19}
%      {(DPC) TeX2 support}
% \changes{v3.0e}{1995/09/15}
%      {(DPC) Modify TeX2 message}
% \changes{v3.0g}{1995/11/01}
%      {(DPC) Switch meaning of \cs{@addtofilelist} for cfg files}
% \changes{v3.0h}{1996/12/06}
%      {(DPC) Remove *** from messages internal/2338}
%    \begin{macrocode}
\ifnum\inputlineno=\m@ne
%    \end{macrocode}
% Still using \TeX2. need a configuration file to avoid setting the 8bit
% characters.
%    \begin{macrocode}
\InputIfFileExists{fonttext.cfg}
           {\typeout{====================================^^J%
                     ^^J%
                      Local config file fonttext.cfg used^^J%
                     ^^J%
                     ====================================}%
             \def\@addtofilelist##1{\xdef\@filelist{\@filelist,##1}}%
            }
           {\typeout{!!!!!!!!!!!!!!!!!!!!!!!!!!!!!!!!!!!!!^^J%
                     !^^J%
                     ! You MUST use a fonttext.cfg file!^^J%
                     ! As you are still using TeX2!!!!!^^J%
                     !^^J%
                     ! See the documentation file tex2.txt^^J%
                     !^^J%
                     !!!!!!!!!!!!!!!!!!!!!!!!!!!!!!!!!!!!!}%
                    \batchmode \@@end}
\else
%    \end{macrocode}
% With \TeX3 can use the standard |ltx| file if no configuration file
% exists.
%    \begin{macrocode}
\InputIfFileExists{fonttext.cfg}
           {\typeout{====================================^^J%
                     ^^J%
                      Local config file fonttext.cfg used^^J%
                     ^^J%
                     ====================================}%
             \def\@addtofilelist##1{\xdef\@filelist{\@filelist,##1}}%
            }
           {%%
%% This is file `fonttext.cfg',
%% generated with the docstrip utility.
%%
%% The original source files were:
%%
%% fontdef.dtx  (with options: `cfgtext')
%% 
%% This is a generated file.
%% 
%% Copyright 1993-2016
%% The LaTeX3 Project and any individual authors listed elsewhere
%% in this file.
%% 
%% This file was generated from file(s) of the LaTeX base system.
%% --------------------------------------------------------------
%% 
%% It may be distributed and/or modified under the
%% conditions of the LaTeX Project Public License, either version 1.3c
%% of this license or (at your option) any later version.
%% The latest version of this license is in
%%    http://www.latex-project.org/lppl.txt
%% and version 1.3c or later is part of all distributions of LaTeX
%% version 2005/12/01 or later.
%% 
%% This file may only be distributed together with a copy of the LaTeX
%% base system. You may however distribute the LaTeX base system without
%% such generated files.
%% 
%% The list of all files belonging to the LaTeX base distribution is
%% given in the file `manifest.txt'. See also `legal.txt' for additional
%% information.
%% 
%% Details of how to use a configuration file to modify this part of
%% the system are in the document `cfgguide.tex'.
%% 
%% 
%%% From File: fontdef.dtx
\ProvidesFile{fonttext.cfg}
           [2014/09/29 v2.3a LaTeX Kernel
(Uncustomised text
           font setup)]
%%
%%
%%
%% Load the standard setup:
%%
%%
%% This is file `fonttext.cfg',
%% generated with the docstrip utility.
%%
%% The original source files were:
%%
%% fontdef.dtx  (with options: `cfgtext')
%% 
%% This is a generated file.
%% 
%% Copyright 1993-2016
%% The LaTeX3 Project and any individual authors listed elsewhere
%% in this file.
%% 
%% This file was generated from file(s) of the LaTeX base system.
%% --------------------------------------------------------------
%% 
%% It may be distributed and/or modified under the
%% conditions of the LaTeX Project Public License, either version 1.3c
%% of this license or (at your option) any later version.
%% The latest version of this license is in
%%    http://www.latex-project.org/lppl.txt
%% and version 1.3c or later is part of all distributions of LaTeX
%% version 2005/12/01 or later.
%% 
%% This file may only be distributed together with a copy of the LaTeX
%% base system. You may however distribute the LaTeX base system without
%% such generated files.
%% 
%% The list of all files belonging to the LaTeX base distribution is
%% given in the file `manifest.txt'. See also `legal.txt' for additional
%% information.
%% 
%% Details of how to use a configuration file to modify this part of
%% the system are in the document `cfgguide.tex'.
%% 
%% 
%%% From File: fontdef.dtx
\ProvidesFile{fonttext.cfg}
           [2014/09/29 v2.3a LaTeX Kernel
(Uncustomised text
           font setup)]
%%
%%
%%
%% Load the standard setup:
%%
%%
%% This is file `fonttext.cfg',
%% generated with the docstrip utility.
%%
%% The original source files were:
%%
%% fontdef.dtx  (with options: `cfgtext')
%% 
%% This is a generated file.
%% 
%% Copyright 1993-2016
%% The LaTeX3 Project and any individual authors listed elsewhere
%% in this file.
%% 
%% This file was generated from file(s) of the LaTeX base system.
%% --------------------------------------------------------------
%% 
%% It may be distributed and/or modified under the
%% conditions of the LaTeX Project Public License, either version 1.3c
%% of this license or (at your option) any later version.
%% The latest version of this license is in
%%    http://www.latex-project.org/lppl.txt
%% and version 1.3c or later is part of all distributions of LaTeX
%% version 2005/12/01 or later.
%% 
%% This file may only be distributed together with a copy of the LaTeX
%% base system. You may however distribute the LaTeX base system without
%% such generated files.
%% 
%% The list of all files belonging to the LaTeX base distribution is
%% given in the file `manifest.txt'. See also `legal.txt' for additional
%% information.
%% 
%% Details of how to use a configuration file to modify this part of
%% the system are in the document `cfgguide.tex'.
%% 
%% 
%%% From File: fontdef.dtx
\ProvidesFile{fonttext.cfg}
           [2014/09/29 v2.3a LaTeX Kernel
(Uncustomised text
           font setup)]
%%
%%
%%
%% Load the standard setup:
%%
\input{fonttext.ltx}
%%
%% Small changes could go here; see documentation in cfgguide.tex for
%% allowed modifications.
%%
%% In particular it is not allowed to misuse this configuration file
%% to modify internal LaTeX commands!
%%
%% If you use this file as the basis for configuration please change
%% the \ProvidesFile lines to clearly identify your modification, e.g.,
%%
%%  \ProvidesFile{fonttext.cfg}[2001/06/01
%%                              Customised local font setup]
%%
%%
\endinput
%%
%% End of file `fonttext.cfg'.

%%
%% Small changes could go here; see documentation in cfgguide.tex for
%% allowed modifications.
%%
%% In particular it is not allowed to misuse this configuration file
%% to modify internal LaTeX commands!
%%
%% If you use this file as the basis for configuration please change
%% the \ProvidesFile lines to clearly identify your modification, e.g.,
%%
%%  \ProvidesFile{fonttext.cfg}[2001/06/01
%%                              Customised local font setup]
%%
%%
\endinput
%%
%% End of file `fonttext.cfg'.

%%
%% Small changes could go here; see documentation in cfgguide.tex for
%% allowed modifications.
%%
%% In particular it is not allowed to misuse this configuration file
%% to modify internal LaTeX commands!
%%
%% If you use this file as the basis for configuration please change
%% the \ProvidesFile lines to clearly identify your modification, e.g.,
%%
%%  \ProvidesFile{fonttext.cfg}[2001/06/01
%%                              Customised local font setup]
%%
%%
\endinput
%%
%% End of file `fonttext.cfg'.
}
\fi
\let\@addtofilelist\@gobble
%    \end{macrocode}
%
% Ditto for math although I don't think that we will get a lot of
% customisation :-)
%    \begin{macrocode}
\InputIfFileExists{fontmath.cfg}
           {\typeout{====================================^^J%
                     ^^J%
                      Local config file fontmath.cfg used^^J%
                     ^^J%
                     ====================================}%
             \def\@addtofilelist##1{\xdef\@filelist{\@filelist,##1}}%
            }
           {%%
%% This is file `fontmath.cfg',
%% generated with the docstrip utility.
%%
%% The original source files were:
%%
%% fontdef.dtx  (with options: `cfgmath')
%% 
%% This is a generated file.
%% 
%% Copyright 1993-2016
%% The LaTeX3 Project and any individual authors listed elsewhere
%% in this file.
%% 
%% This file was generated from file(s) of the LaTeX base system.
%% --------------------------------------------------------------
%% 
%% It may be distributed and/or modified under the
%% conditions of the LaTeX Project Public License, either version 1.3c
%% of this license or (at your option) any later version.
%% The latest version of this license is in
%%    http://www.latex-project.org/lppl.txt
%% and version 1.3c or later is part of all distributions of LaTeX
%% version 2005/12/01 or later.
%% 
%% This file may only be distributed together with a copy of the LaTeX
%% base system. You may however distribute the LaTeX base system without
%% such generated files.
%% 
%% The list of all files belonging to the LaTeX base distribution is
%% given in the file `manifest.txt'. See also `legal.txt' for additional
%% information.
%% 
%% Details of how to use a configuration file to modify this part of
%% the system are in the document `cfgguide.tex'.
%% 
%% 
%%% From File: fontdef.dtx
\ProvidesFile{fontmath.cfg}
           [2014/09/29 v2.3a LaTeX Kernel
(Uncustomised math
           font setup)]
%%
%%
%%
%% Load the standard setup:
%%
%%
%% This is file `fontmath.cfg',
%% generated with the docstrip utility.
%%
%% The original source files were:
%%
%% fontdef.dtx  (with options: `cfgmath')
%% 
%% This is a generated file.
%% 
%% Copyright 1993-2016
%% The LaTeX3 Project and any individual authors listed elsewhere
%% in this file.
%% 
%% This file was generated from file(s) of the LaTeX base system.
%% --------------------------------------------------------------
%% 
%% It may be distributed and/or modified under the
%% conditions of the LaTeX Project Public License, either version 1.3c
%% of this license or (at your option) any later version.
%% The latest version of this license is in
%%    http://www.latex-project.org/lppl.txt
%% and version 1.3c or later is part of all distributions of LaTeX
%% version 2005/12/01 or later.
%% 
%% This file may only be distributed together with a copy of the LaTeX
%% base system. You may however distribute the LaTeX base system without
%% such generated files.
%% 
%% The list of all files belonging to the LaTeX base distribution is
%% given in the file `manifest.txt'. See also `legal.txt' for additional
%% information.
%% 
%% Details of how to use a configuration file to modify this part of
%% the system are in the document `cfgguide.tex'.
%% 
%% 
%%% From File: fontdef.dtx
\ProvidesFile{fontmath.cfg}
           [2014/09/29 v2.3a LaTeX Kernel
(Uncustomised math
           font setup)]
%%
%%
%%
%% Load the standard setup:
%%
%%
%% This is file `fontmath.cfg',
%% generated with the docstrip utility.
%%
%% The original source files were:
%%
%% fontdef.dtx  (with options: `cfgmath')
%% 
%% This is a generated file.
%% 
%% Copyright 1993-2016
%% The LaTeX3 Project and any individual authors listed elsewhere
%% in this file.
%% 
%% This file was generated from file(s) of the LaTeX base system.
%% --------------------------------------------------------------
%% 
%% It may be distributed and/or modified under the
%% conditions of the LaTeX Project Public License, either version 1.3c
%% of this license or (at your option) any later version.
%% The latest version of this license is in
%%    http://www.latex-project.org/lppl.txt
%% and version 1.3c or later is part of all distributions of LaTeX
%% version 2005/12/01 or later.
%% 
%% This file may only be distributed together with a copy of the LaTeX
%% base system. You may however distribute the LaTeX base system without
%% such generated files.
%% 
%% The list of all files belonging to the LaTeX base distribution is
%% given in the file `manifest.txt'. See also `legal.txt' for additional
%% information.
%% 
%% Details of how to use a configuration file to modify this part of
%% the system are in the document `cfgguide.tex'.
%% 
%% 
%%% From File: fontdef.dtx
\ProvidesFile{fontmath.cfg}
           [2014/09/29 v2.3a LaTeX Kernel
(Uncustomised math
           font setup)]
%%
%%
%%
%% Load the standard setup:
%%
\input{fontmath.ltx}
%%
%% Small changes could go here; see documentation in cfgguide.tex for
%% allowed modifications.
%%
%% In particular it is not allowed to misuse this configuration file
%% to modify internal LaTeX commands!
%%
%% If you use this file as the basis for configuration please change
%% the \ProvidesFile lines to clearly identify your modification, e.g.,
%%
%%  \ProvidesFile{fonttext.cfg}[2001/06/01
%%                              Customised local font setup]
%%
%%
\endinput
%%
%% End of file `fontmath.cfg'.

%%
%% Small changes could go here; see documentation in cfgguide.tex for
%% allowed modifications.
%%
%% In particular it is not allowed to misuse this configuration file
%% to modify internal LaTeX commands!
%%
%% If you use this file as the basis for configuration please change
%% the \ProvidesFile lines to clearly identify your modification, e.g.,
%%
%%  \ProvidesFile{fonttext.cfg}[2001/06/01
%%                              Customised local font setup]
%%
%%
\endinput
%%
%% End of file `fontmath.cfg'.

%%
%% Small changes could go here; see documentation in cfgguide.tex for
%% allowed modifications.
%%
%% In particular it is not allowed to misuse this configuration file
%% to modify internal LaTeX commands!
%%
%% If you use this file as the basis for configuration please change
%% the \ProvidesFile lines to clearly identify your modification, e.g.,
%%
%%  \ProvidesFile{fonttext.cfg}[2001/06/01
%%                              Customised local font setup]
%%
%%
\endinput
%%
%% End of file `fontmath.cfg'.
}
\let\@addtofilelist\@gobble
%    \end{macrocode}
%
% Then we preload several fonts. This file might be customized
% \emph{without} changing the behavior of the format (i.e.\ necessary
% font definitions will be loaded at runtime if they are not
% preloaded).  This is done in the file \texttt{preload.ltx}.
%    \begin{macrocode}
\InputIfFileExists{preload.cfg}
           {\typeout{====================================^^J%
                     ^^J%
                      Local config file preload.cfg used^^J%
                     ^^J%
                     =====================================}%
             \def\@addtofilelist##1{\xdef\@filelist{\@filelist,##1}}%
            }
           {% \iffalse meta-comment
%
% Copyright 1993-2016
% The LaTeX3 Project and any individual authors listed elsewhere
% in this file.
%
% This file is part of the LaTeX base system.
% -------------------------------------------
%
% It may be distributed and/or modified under the
% conditions of the LaTeX Project Public License, either version 1.3c
% of this license or (at your option) any later version.
% The latest version of this license is in
%    http://www.latex-project.org/lppl.txt
% and version 1.3c or later is part of all distributions of LaTeX
% version 2005/12/01 or later.
%
% This file has the LPPL maintenance status "maintained".
%
% The list of all files belonging to the LaTeX base distribution is
% given in the file `manifest.txt'. See also `legal.txt' for additional
% information.
%
% The list of derived (unpacked) files belonging to the distribution
% and covered by LPPL is defined by the unpacking scripts (with
% extension .ins) which are part of the distribution.
%
% \fi
%
% \iffalse
%%% From File: preload.dtx
%<*dtx>
           \ProvidesFile{preload.dtx}
%</dtx>
%<*preload>
%<*!tex>
%<+cm>  \ProvidesFile{cmpreloa.%
%<+dc>  \ProvidesFile{dcpreloa.%
%<+xpt>                         xpt}
%<+xipt>                        xip}
%<+xiipt>                       xii}
%<+min> \ProvidesFile{preload.min}
%<+ori> \ProvidesFile{preload.ori}
%</!tex>
%<+tex> \ProvidesFile{preload.ltx}
% \fi
%       \ProvidesFile{preload.dtx}
         [2014/09/29 v2.1g LaTeX Kernel (Font Preloading)]
%
%
%
%\iffalse       This is a META comment
%
% File `preload.dtx'.
% Copyright (C) 1989-1994 Frank Mittelbach and Rainer Sch\"opf,
% all rights reserved.
%
% \fi
%
% \GetFileInfo{preload.dtx}
% \title{The \texttt{preload.dtx} file\thanks {This file has version
%    number \fileversion, dated \filedate}\\ for use with \LaTeXe}
% \date{\filedate}
% \author{Frank Mittelbach \and Rainer Sch\"opf}
%
% \changes{v2.0b}{1993/03/08}{Added 12pt preloads}
% \changes{v2.1e}{1994/11/07}{(DPC) Updated to use \cs{ProvidesFile}}
% \changes{v2.1g}{1998/08/17}{(RmS) Minor documentation fixes.}
%
% \def\dst{\expandafter{\normalfont\scshape docstrip}}
%
% \setcounter{StandardModuleDepth}{1}
%
% \MaintainedByLaTeXTeam{latex}
% \maketitle
%
% \section{Overview}
%
%   This file contains an number of possible settings for preloading
%   fonts during installation of NFSS2 (which is used by \LaTeXe).  It
%   will be used to generate the following files:
%   \begin{center}
%   \begin{tabular}{ll}
%   preload.min   &  minimal subset of fonts necessary to run NFSS2 \\
%   preload.ori   &  preload of CM fonts similar to the old
%                        \texttt{lfonts.tex}                       \\
%   preload.ltx    &  The standard selection of preloads \\
%   cmpreloa.xpt   &  preload of CM fonts for 10pt document size\\
%   cmpreloa.xip   &  preload of CM fonts for 11pt document size\\
%   cmpreloa.xii   &  preload of CM fonts for 12pt document size\\
%   dcpreloa.xpt   &  preload of DC fonts for 10pt size \\
%   dcpreloa.xip   &  preload of DC fonts for 11pt size \\
%   dcpreloa.xii   &  preload of DC fonts for 12pt size \\
%   \end{tabular}
%   \end{center}
%
%    These files are for installations that make use of Computer
%    Modern fonts either old encoding (OT1) or Cork encoding (T1). The
%    Computer Modern fonts with Cork encoding are known as DC-fonts.
%
%    Most important is \texttt{preload.ltx} which is used during
%    format generation. You are \emph{not} allowed to change this file.
%
% \section{Customization}
%
%    You can customize the preloaded fonts in your \LaTeXe{} system by
%    installing a file with the name \texttt{preload.cfg}. If this
%    file exists it will be used in place of the system file
%    \texttt{preload.ltx}.  You can, for example, copy one of the
%    files mentioned above (that can be generated from this source) to
%    \texttt{preload.cfg}.
%
%    Or you can define completely other preloads. In that case start
%    from \texttt{preload.min} since that contains the fonts that have
%    to be preloaded by *all* \LaTeXe{} systems.
%
%    Avoid using \texttt{preload.ori}, it will load so many fonts that
%    on most installations it is nearly impossible to load other font
%    families afterwards. This file is only generated to show what
%    fonts have been preloaded by \LaTeX~2.09.
%
%    If you normally use other fonts than Computer Modern
%    \texttt{preload.min} might be best.
%
%    \begin{quote} \textbf{Warning:} If you preload fonts with
%    encodings other than the normally supported encodings you have to
%    declare that encoding in a \texttt{fontdef.cfg} configuration
%    file (see the documentation in the file \texttt{fontdef.dtx}).
%    Adding an extra encoding to the format might produce non-portable
%    documents, thus this should be avoided if possible.
%    \end{quote}
%
%
% \StopEventually{}
%
% \section{Module switches for the \dst{} program}
%
%  The \dst{} will generate the above file from this source using the
%  following module directives:
% \begin{center}
% \begin{tabular}{ll}
%   driver & produce a documentation driver file \\
%   preload& produce a preload\ldots file \\[2pt]
%   cm     & for OT1 encoded Computer Modern \\
%   dc     & for T1 encoded Computer Modern \\[2pt]
%   min    & produce minimal subset \\
%   xpt    & produce 10pt preloads \\
%   xipt   & produce 11pt preloads \\
%   xiipt  & produce 12pt preloads \\
%   ori    & produce preloads similar to old \texttt{lfonts.tex}\\
%   tex    & produce preload.ltx\\
% \end{tabular}
% \end{center}
% A typical \dst{} command file would then have entries like:
% \begin{verbatim}
%\generateFile{preload.min}{t}{\from{preload.dtx}{preload,min}}
%\end{verbatim}
% for generating preload files.
%
% \section{A driver for this document}
%
%    The next bit of code contains the documentation driver file for
%    \TeX{}, i.e., the file that will produce the documentation you
%    are currently reading. It will be extracted from this file by the
%    \dst{} program.
%    \begin{macrocode}
%<*driver>
\documentclass{ltxdoc}
%\OnlyDescription  % comment out for implementation details
\begin{document}
   \DocInput{preload.dtx}
\end{document}
%</driver>
%    \end{macrocode}
%
%
% \section{The code}
%
%    We begin by loading the math extension font (cmex10)
%    and the \LaTeX{} line and circle fonts.
%    It is necessary to do this explicitly since these are
%    used by \texttt{lplain.tex} and \texttt{latex.tex}.
%    Since the internal font name contains |/| characters
%    and digits we construct the name via |\csname|.
%    These are the only fonts (!) that must be loaded in this file.
%
%    All |\DeclarePreloadSizes| can be removed or others can be added,
%    they only influence the processing speed.
% \changes{v2.0c}{1993/08/13}{Added \cs{relax} at end of font names.}
%    \begin{macrocode}
\expandafter\font\csname OMX/cmex/m/n/10\endcsname=cmex10\relax
\font\tenln  =line10   \font\tenlnw  =linew10\relax
\font\tencirc=lcircle10 \font\tencircw=lcirclew10\relax
%    \end{macrocode}
%    The above fonts should not be touched but anything below this
%    point here in the preload suggestions can be modified without any
%    problems.
%    \begin{macrocode}
%<-tex>%*******************************************
%<-tex>% Start any modification below this point **
%<-tex>%*******************************************
%<-tex>
%%
%% Computer Modern Roman:
%%-----------------------
%<*ori>
\DeclarePreloadSizes{OT1}{cmr}{m}{n}
        {5,6,7,8,9,10,10.95,12,14.4,17.28,20.74,24.88}
\DeclarePreloadSizes{OT1}{cmr}{bx}{n}{9,10,10.95,12,14.4,17.28}
\DeclarePreloadSizes{OT1}{cmr}{m}{sl}{10,10.95,12}
\DeclarePreloadSizes{OT1}{cmr}{m}{it}{7,8,9,10,10.95,12}
%</ori>
%<+xpt&cm> \DeclarePreloadSizes{OT1}{cmr}{m}{n}{5,7,10}
%<+xpt&dc> \DeclarePreloadSizes{T1}{cmr}{m}{n}{5,7,10}
%<+xipt&cm> \DeclarePreloadSizes{OT1}{cmr}{m}{n}{6,8,10.95}
%<+xipt&dc> \DeclarePreloadSizes{T1}{cmr}{m}{n}{6,8,10.95}
%<+xiipt&cm> \DeclarePreloadSizes{OT1}{cmr}{m}{n}{6,8,12}
%<+xiipt&dc> \DeclarePreloadSizes{T1}{cmr}{m}{n}{6,8,12}
%%
%% Computer Modern Sans:
%%----------------------
%<+ori> \DeclarePreloadSizes{OT1}{cmss}{m}{n}{10,10.95,12}
%%
%% Computer Modern Typewriter:
%%----------------------------
%<+ori> \DeclarePreloadSizes{OT1}{cmtt}{m}{n}{9,10,10.95,12}
%%
%% Computer Modern Math:
%%----------------------
%<*ori>
\DeclarePreloadSizes{OML}{cmm}{m}{it}
         {5,6,7,8,9,10,10.95,12,14.4,17.28,20.74}
\DeclarePreloadSizes{OMS}{cmsy}{m}{n}
         {5,6,7,8,9,10,10.95,12,14.4,17.28,20.74}
%</ori>
%    \end{macrocode}
%
%    The math fonts are the same for both DC and CM fonts. So far
%    there isn't an agreed on standard.
% \changes{v2.4e}{1995/12/04}
%      {Ulrik Vieth. added 12pt OMS and OML preloads  /1989}
%    \begin{macrocode}
%<*xpt>
\DeclarePreloadSizes{OML}{cmm}{m}{it}{5,7,10}
\DeclarePreloadSizes{OMS}{cmsy}{m}{n}{5,7,10}
%</xpt>
%<*xipt>
\DeclarePreloadSizes{OML}{cmm}{m}{it}{6,8,10.95}
\DeclarePreloadSizes{OMS}{cmsy}{m}{n}{6,8,10.95}
%</xipt>
%<*xiipt>
\DeclarePreloadSizes{OML}{cmm}{m}{it}{6,8,12}
\DeclarePreloadSizes{OMS}{cmsy}{m}{n}{6,8,12}
%</xiipt>
%%
%% LaTeX symbol fonts:
%%--------------------
%<*ori>
\DeclarePreloadSizes{U}{lasy}{m}{n}
         {5,6,7,8,9,10,10.95,12,14.4,17.28,20.74}
%</ori>
%</preload>
%    \end{macrocode}
%
%
%
% \Finale
%
\endinput
}
\let\@addtofilelist\@gobble
%    \end{macrocode}
%
%
% \begin{macro}{\@acci}
% \begin{macro}{\@accii}
% \begin{macro}{\@acciii}
% \changes{v2.1m}{1994/05/16}{Define saved versions of accents}
%    We also save the values of some accents in |\@acci|, |\@accii|
%    and |\@acciii| so they can be restored by a |minipage| inside a
%    |tabbing| environment.
%    \begin{macrocode}
\let\@acci\' \let\@accii\` \let\@acciii\=
%    \end{macrocode}
% \end{macro}
% \end{macro}
% \end{macro}
%
%
% \begin{macro}{\cal}
% \changes{v3.0a}{1995/05/24}
%      {(DPC) Remove definition}
% \begin{macro}{\mit}
% \changes{v3.0a}{1995/05/24}
%      {(DPC) Remove definition}
%    Here were the two old \meta{alphabet identifiers}.
% \end{macro}
% \end{macro}
%
%
% \iffalse
%<+checkmem>\CHECKMEM
% \fi
%
%    \begin{macrocode}
%</2ekernel>
%    \end{macrocode}
%
% \Finale
%
