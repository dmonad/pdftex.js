% \iffalse meta-comment
%
% Copyright 1993-2016
% The LaTeX3 Project and any individual authors listed elsewhere
% in this file.
%
% This file is part of the LaTeX base system.
% -------------------------------------------
%
% It may be distributed and/or modified under the
% conditions of the LaTeX Project Public License, either version 1.3c
% of this license or (at your option) any later version.
% The latest version of this license is in
%    http://www.latex-project.org/lppl.txt
% and version 1.3c or later is part of all distributions of LaTeX
% version 2005/12/01 or later.
%
% This file has the LPPL maintenance status "maintained".
%
% The list of all files belonging to the LaTeX base distribution is
% given in the file `manifest.txt'. See also `legal.txt' for additional
% information.
%
% The list of derived (unpacked) files belonging to the distribution
% and covered by LPPL is defined by the unpacking scripts (with
% extension .ins) which are part of the distribution.
%
% \fi
% \iffalse
%%% From File: ltoutput.dtx
%<flafter>\ProvidesPackage{flafter}
%<fltrace>\ProvidesPackage{fltrace}
%<flafter,fltrace>       [2015/09/05 v1.3a
%<flafter>                  Standard LaTeX floats after reference (FMi)]
%<fltrace>                  Tracing LaTeX floats algorithm (FMi)]
%
%<*driver>
% \fi
\ProvidesFile{ltoutput.dtx}
             [2015/09/05 v1.3a LaTeX Kernel (Output Routine)]
% \iffalse
\documentclass{ltxdoc}
\GetFileInfo{ltoutput.dtx}
\title{\filename}
\date{\filedate}
\author{Leslie Lamport, Frank Mittelbach, Chris Rowley}
\begin{document}
 \MaintainedByLaTeXTeam{latex}
 \maketitle
 \DocInput{\filename}
\end{document}
%</driver>
% \fi
%
%
%
% \iffalse
% LATEX VERSION 2e
% Copyright (C) 1992 by Leslie Lamport
% Copyright (C) 1994-2000 by Leslie Lamport, LaTeX3 project
%
% LaTeX 2e kernel file for the output routine.
%
% Part of this file is the latest (not greatest, it still
% deserves its name) version of kludge.sty.
%
% It also contains a few enhancements and many changes (corrections
% and tidyings) to the float mechanism and other parts of the output
% routine.
%
% The tracing in the file at present is mainly to help in testing the
% code but it may well be developed into a full float tracing
% package one day.
%
% ============================
% \fi
%
% \changes{v1.1p}{1995/08/25}{Support autoloading feature (FMi).}
% \task{CAR}{Update documentation.}
% \changes{v1.0k}{1994/02/08}{Documentation and tasks tidied.}
% \changes{v1.0l}{1994/03/15}{Driver added and further tidying.}
% \changes{v1.0l}{1994/03/15}{Some boxmaxdepth settings removed.}
% \changes{v1.0l}{1994/03/15}{Added some warnings when page gets full of
% top floats.}
% \changes{v1.0l}{1994/03/15}{Removed duplicated code and corrected
% docstrip options.}
% \changes{v1.0m}{1994/04/24}{Removed some long lines and other
% aesthetic changes.}
% \changes{v1.0m}{1994/04/24}{Corrected unverbed commands in
% documentation.}
% \changes{v1.0m}{1994/04/24}{Changed \cs{@normalsize} to
% \cs{normalsize}.}
% \changes{v1.0m}{1994/04/24}{Warning messages changed/corrected.}
% \changes{v1.0n}{1994/04/30}{Added \cs{col@number}.}
% \changes{v1.0n}{1994/04/30}{Fixed bug from \cs{dblfigrule} with
% \cs{@topnewpage}.}
% \changes{v1.0n}{1994/04/30}{Documentation tidied.}
% \changes{v1.0n}{1994/04/30}{\cs{@activechar@warning} changed to an
% info message.}
% \changes{v1.0n}{1994/04/30}{Full of floats action improved.}
% \changes{v1.0n}{1994/04/30}{Empty column action added.}
% \changes{v1.0o}{1994/05/02}{Code of \cs{@resethfps} shortened.}
% \changes{v1.0p}{1994/05/12}{\cs{normalcolor}added in various
% places (DPC).}
% \changes{v1.0q}{1994/05/16}{Changed setting of accents (FMi).}
% \changes{v1.0s}{1994/05/20}{Added setting of \cs{protect}
% during \cs{shipout}.}
% \changes{v1.0t}{1994/05/22}{Changed warnings and infos to
% new commands.}
% \changes{v1.0u}{1994/05/23}{Added \cs{MessageBreak}.}
% \changes{v1.0u}{1994/05/23}{Changed resetting of \cs{protect} after
% shipout.}
% \changes{v1.0v}{1994/05/25}{Extra documentation.}
% \changes{v1.0w}{1994/06/01}{Tidied up typesetting.}
%
% \changes{v1.1f}{1994/11/14}{Removed old definition of \cs{@testfp}.}
% \changes{v1.1h}{1994/11/17}
%         {\cs{@tempa} to \cs{reserved@a}.}
% \changes{v1.1m}{1995/05/07}{Use \cs{hb@xt@}.}
% \changes{v1.2n}{2015/02/21}
%         {Removed autoload code}
%
% \StopEventually{}
%
% \def \ie {i.e.~}
% \def \eg {e.g.~}
%
% \section{Output Routine}
%
% \subsection{Floats}
%
% The `2ekernel' code ensures that a |\usepackage{autoout1}| is
% essentially ignored if a `full' format is being used that has
% the autoload file mode already in the format.
%    \begin{macrocode}
%<defx>\begingroup
%<defx>\makeatletter
%<defx>\nfss@catcodes
%<2ekernel>\expandafter\let\csname ver@autoout1.sty\endcsname\fmtversion
%    \end{macrocode}
%
%
% \begin{oldcomments}
%    \begin{macrocode}
%<*2ekernel>
\message{output,}
%    \end{macrocode}
%     ****************************************
%     *               OUTPUT                 *
%     ****************************************
%
%
%  PAGE LAYOUT PARAMETERS
%
%   \topmargin      : Extra space added to top of page.
%   @twoside        : boolean.  T if two-sided printing
%   \oddsidemargin  : IF @twoside = T
%                       THEN extra space added to left of odd-numbered
%                            pages.
%                       ELSE extra space added to left of all pages.
%   \evensidemargin : IF @twoside = T
%                       THEN extra space added to left of even-numbered
%                            pages.
%   \headheight     : height of head
%   \headsep        : separation between head and text
%   \footskip       : distance separation between baseline of last
%                     line of text and baseline of foot.
%                     Note difference between \footSKIP and \headSEP.
%   \textheight     : height of text on page, excluding head and foot
%   \textwidth      : width of printing on page
%   \columnsep      : IF @twocolumn = T
%                       THEN width of space between columns
%   \columnseprule  : IF @twocolumn = T
%                       THEN width of rule between columns (0 if none).
%   \columnwidth    : IF @twocolumn = T
%                       THEN (\textwidth - \columnsep)/2
%                       ELSE \textwidth
%                     It is set by the \twocolumn and
%                     \onecolumn commands.
%   \@textbottom    : Command executed at bottom of vbox holding text of
%                     page (including figures).  The \raggedbottom
%                     command almost \let's this to \vfil (actually sets
%                     it to \vskip \z@ plus.0001fil).
%                     Should have depth 0pt.
%
%   \@texttop       : Command executed at top of vbox holding text of
%                     page (including figures).  Used by letter style;
%                     can also be used to produce centered pages.
%                     Let to \relax by \raggedbottom and \flushbottom.
%
%   Page layout must initialize \@colht and \@colroom to \textheight.
%
%  PAGE STYLE PARAMETERS:
%
%   \floatsep       : Space left between floats.
%   \textfloatsep   : Space between last top float or first bottom float
%                     and the text.
%   \topfigrule     : Command to place rule (or whatever) between floats
%                     at top of page and text.  Executed in inner
%                     vertical mode right before the \textfloatsep skip
%                     separating the floats from the text.  Must occupy
%                     zero vertical space.  (See \footnoterule.)
%   \botfigrule     : Same as \topfigrule, but put after the
%                     \textfloatsep skip separating text from the
%                     floats at bottom of page.
%   \intextsep      : Space left on top and bottom of an in-text float.
%   \dblfloatsep    : Space between double-column floats.
%   \dbltextfloatsep : Space between top double-column floats
%                      and text.
%   \dblfigrule     : Similar to \topfigrule, but for double-column
%                     floats.
%   \@fptop         : Glue to go at top of float column -- must be 0pt +
%                     stretch
%   \@fpsep         : Glue to go between floats in a float column.
%   \@fpbot         : Glue to go at bottom of float column
%                       -- must be 0pt +
%                     stretch
%   \@dblfptop, \@dblfpsep, \@dblfpbot
%                   : Analogous for double-column float page in
%                     two-column format.
%
%  FOOTNOTES: As in PLAIN, footnotes use \insert\footins.
%
%  PAGE LAYOUT SWITCHES AND MACROS
%
%   @twocolumn      : Boolean.  T if two columns per page globally.
%
%  PAGE STYLE MACROS AND SWITCHES
%
%   \@oddhead        : IF @twoside = T
%                           THEN macro to generate head of odd-numbered
%                                pages.
%                           ELSE macro to generate head of all pages.
%   \@evenhead       : IF @twoside = T
%                           THEN macro to generate head of even-numbered
%                                pages.
%   \@oddfoot        : IF @twoside = T
%                           THEN macro to generate foot of odd-numbered
%                                pages.
%                           ELSE macro to generate foot of all pages.
%   \@evenfoot       : IF @twoside = T
%                           THEN macro to generate foot of even-numbered
%                                pages.
%   @specialpage    : boolean.  T if current page is to have a special
%                               format.
%  \@specialstyle  : If its value is  foo then
%                     IF @specialpage = T
%                       THEN the command \ps@foo is executed to
%                            temporarily reset the page style parameters
%                            before composing the current page.
%                            This command should execute only \def's and
%                            \edef's, making only local definitions.
%
%  FLOAT PLACEMENT PARAMETERS
%
% The following parameters are set by the macro \@floatplacement.
% When \@floatplacement is called,
% \@colht is the height of the page or column being built.  I.e.:
%         * For single-column page it equals \textheight.
%         * For double-column page it equals \textheight - height
%           of double-column floats on page.
% Note that some are set globally and some locally:
%    \@topnum  :=G Maximum number of floats allowed on the top of a
%                  column.
%    \@toproom :=G Maximum amount of top of column devoted to floats--
%                  excluding \textfloatsep separation below the floats
%                  and \floatsep separation between them.  For
%                  two-column output, should be computed as a function
%                  of \@colht.
%    \@botnum, \@botroom
%                : Analogous to above.
%    \@colnum  :=G Maximum number of floats allowed in a column,
%                  including in-text floats.
%    \@textmin :=L Minimum amount of text (excluding footnotes) that
%                  must appear on a text page.
%                     %% 27 Sep 85 : made local to
%                     %% \@addtocurcol and \@addtonextcol
%                  It is now also used locally in processing double
%                  floats.
%    \@fpmin   :=L Minimum height of floats in a float column.
%
% The macro \@dblfloatplacement sets the following parameters.
%    \@dbltopnum  :=G Maximum number of double-column floats allowed at
%                     the top of a two-column page.
%    \@dbltoproom :=G Maximum height of double-column floats allowed at
%                     top of two-column page.
%    \@fpmin      :=L Minimum height of floats in a float column.
% It should also perform the following local assignments where necessary
% -- i.e., where the new value differs from the old one:
%      \@fptop       :=L \@dblfptop
%      \@fpsep       :=L \@dblfpsep
%      \@fpbot       :=L \@dblfpbot
%
%  OUTPUT ROUTINE VARIABLES
%
%  \@colht : The total height of the current column.  In single column
%            style, it equals \textheight.  In two-column style, it is
%            \textheight minus the height of the double-column floats
%            on the current page.  MUST BE INITIALIZED TO \textheight.
%
%  \@colroom : The height available in the current column for text and
%              footnotes.  It equals \@colht minus the height of all
%              floats committed to the top and bottom of the current
%              column.
%
%  \@textfloatsheight : The total height of in-text floats on the
%                       current page.
%
%  \footins : Footnote insertion number.
%
%  \@maxdepth : Saved value of TeX's \maxdepth.  Must be set
%               when any routine sets \maxdepth.
%
%            CALLING THE OUTPUT ROUTINE
%            --------------------------
%
% The output routine is called either by TeX's normal page-breaking
% mechanism, or by a macro putting a penalty < or = -10000 in the output
% list.  In the latter case, the penalty indicates why the output
% routine was called, using the following code.
%
%   penalty   reason
%   -------   ------
%   -10000    \pagebreak
%             \newpage
%   -10001    \clearpage (\penalty -10000 \vbox{} \penalty -10001)
%   -10002    float insertion, called from horizontal mode
%   -10003    float insertion, called from vertical mode.
%   -10004    float insertion.
%
% Note: A float or marginpar puts the following sequence in the output
%       list:  (i) a penalty of -10004,
%             (ii) a null \vbox
%            (iii) a penalty of -10002 or -10003.
%       This solves two special problems:
%         1. If the float comes right after a \newpage or \clearpage,
%            then the first penalty is ignored, but the second one
%            invokes the output routine.
%         2. If there is a split footnote on the page, the second 'page'
%            puts out the rest of the footnote.
%
%             THE OUTPUT ROUTINE
%             ------------------
%
% FUNCTIONS USED IN THE OUTPUT ROUTINE:
%
% \@outputpage : Produces an output page with the contents of box
%              \@outputbox as the text part.
%              Also sets \@colht :=G \textheight.
%              The page style is determined as follows.
%                IF  @thispagestyle = true
%                  THEN  use \thispagestyle style
%                  ELSE  use ordinary page style.
%
% \@tryfcolumn\FLIST : Tries to form a float column composed of floats
%         from \FLIST (if nonempty) with the following parameters:
%                \@colht : height of box
%                \@fpmin : minimum height of floats in the box
%                \@fpsep : interfloat space
%                \@fptop : glue at top of box
%                \@fpbot : glue at bottom of box.
%              If it succeeds, then it does the following:
%                * \@outputbox :=L the composed float box.
%                * @fcolmade   :=G true
%                * \FLIST      :=G \FLIST - floats put in box
%                * \@freelist  :=G \@freelist + floats put in box
%              If it fails, then:
%                * @fcolmade :=G false
%           NOTE: BIT MUST BE A SINGLE TOKEN!
%
% \@makefcolumn \FLIST : Same as \@tryfcolumn except that it
%             fails to make a float column only if \FLIST is empty.
%             Otherwise, it makes a float column containing at least
%             the first box in \FLIST, disregarding \@fpmin.
%
% \@startcolumn :
%       Calls \@tryfcolumn\@deferlist.  If \@tryfcolumn returns with
%       (globally set) @fcolmade = false, then:
%                * Globally sets \@toplist and \@botlist to floats
%                  from \@deferlist to go at top and bottom of column,
%                  deleting them from \@deferlist.  It does
%                  this using \@colht as the total height, the page
%                  style parameters \@floatsep and \@textfloatsep, and
%                  the float placement parameters \@topnum, \@toproom,
%                  \@botnum, \@botroom, \@colnum and \textfraction.
%                * Globally sets \@colroom to \@colht minus the height
%                  of the added floats.
%
% \@startdblcolumn :
%      Calls \@tryfcolumn\@dbldeferlist{8}.  If \@tryfcolumn returns
%      with (globally set) @fcolmade = false, then:
%               * Globally sets \@dbltoplist to floats from
%                 \@dbldeferlist to go at top and bottom of column,
%                 deleting them from \@dbldeferlist.
%                 It does this using \textheight as the
%                 total height, and the parameters \@dblfloatsep, etc.
%               * Globally sets \@colht to \textheight minus the height
%                 of the added floats.
%
% \@combinefloats : Combines the text from box
%              \@outputbox with the floats from \@toplist and \@botlist,
%              putting the new box in \@outputbox.  It uses \floatsep
%              and \textfloatsep for the appropriate separations.
%              It puts the elements of \TOPLIST and \BOTLIST onto
%              \@freelist, and makes those lists null.
%
% \@makecol : Makes the contents of \box255 plus the accumulated
%              footnotes, plus the floats in \@toplist and \@botlist,
%              into a single column of height \@colht (unless the page
%              height has been locally changed), which it puts
%              into box \@outputbox.  It puts boxes in \@midlist back
%              onto \@freelist and restores \maxdepth.
%
% \@opcol : Outputs a column whose text is in box \@outputbox
%              If @twocolumn = false, then it calls \@outputpage,
%              sets \@colht :=G \textheight, and calls \@floatplacement.
%
%              If @twocolumn = true, then:
%                  If @firstcolumn = true, then it puts box \@outputbox
%                  into \@leftcolumn and sets @firstcolumn :=G false.
%
%                  If @firstcolumn = false, then it puts out the current
%                  two-column page, any possible two-column float pages,
%                  and determines \@dbltoplist for the next page.
%
%
%
%            USER COMMANDS THAT CALL OR AFFECT THE OUTPUT ROUTINE
%            ----------------------------------------------------
%
% \newpage == BEGIN \par\vfil\penalty -10000 END
%
% \clearpage == BEGIN \newpage
%                     \write -1{}    % Part of hack to make sure no
%                     \vbox{}        % \write's get lost.
%                     \penalty -10001
%               END
%
% \cleardoublepage == BEGIN \clearpage
%                           if @twoside = true and c@page is even
%                             then \hbox{} \newpage fi
%                     END
%
%
% \twocolumn[BOX] : starts a new page, changing to twocolumn setting
%     and puts BOX in a parbox of width \textwidth across the top.
%     Useful for full-width titles for double-column pages.
%     SURPRISE: The stretch from \@dbltextfloatsep will be inserted
%               between the BOX and the top of the two columns.
%
%
%            FLOAT-HANDLING MECHANISMS
%            -------------------------
%
% The float environment obtains an insertion number B from the
% \@freelist (see below for a description of list manipulation), puts
% the float into box B and sets \count B to a FLOAT SPECIFIER.  For
% a normal (not double-column) float, it then causes a page break
% in one of the following two ways:
%   - In outer hmode: \vadjust{\penalty -10002}
%   - In vmode :      \penalty -10003.
% For a double-column float, it puts B onto the \@dbldeferlist.
% The float specifier has two components:
%    * A PLACEMENT SPECIFICATION, describing where the float may
%      be placed.
%    * A TYPE, which is a power of two--e.g., figures might be
%      type 1 floats, tables type 2 floats, programs type 4 floats, etc.
% The float specifier is encoded as follows, where bit 0 is the least
% significant bit.
%
%  Bit    Meaning
%  ---    -------
%   0     1 iff the float may go where it appears in the text.
%   1     1 iff the float may go on the top of a page.
%   2     1 iff the float may go on the bottom of a page.
%   3     1 iff the float may go on a float page.
%   4     1 unless the PLACEMENT includes a !
%   5     1 iff a type 1 float
%   6     1 iff a type 2 float
%   etc.
%
%  A negative float specifier is used to indicate a marginal note.
%
%     MACROS AND DATA STRUCTURES FOR PROCESSING FLOATS
%     ------------------------------------------------
%
%  A FLOAT LIST consisting of the floats in boxes \boxa ... \boxN has
%  the form:
%         \@elt \boxa ... \@elt \boxN
%  where  \boxI is defined by
%         \newinsert\boxI
%  Normally, \@elt is \let to \relax.  A test can be performed on the
%  entire float list by locally \def'ing \@elt appropriately and
%  executing the list.
%  This is a lot more efficient than looping through the list.
%
%  The following macros are used for manipulating float lists.
%
%  \@next \CS \LIST {NONEMPTY}{EMPTY} ==  %% NOTE: ASSUME \@elt = \relax
%    BEGIN  assume that \LIST == \@elt \B1 ... \@elt \Bn
%           if n = 0
%             then  EMPTY
%             else  \CS    :=L \B1
%                   \LIST  :=G \@elt \B2 ... \@elt \Bn
%                   NONEMPTY
%           fi
%    END
%
%
%  \@bitor\NUM\LIST : Globally sets switch @test to the disjunction for
%         all I of bit  log2 \NUM of the float specifiers of all the
%         floats in \LIST.
%         I.e., @test is set to true iff there is at least one
%         float in \LIST having bit  log2 \NUM  of its float specifier
%         equal to 1.
%
%  Note: log2 [(\count I)/32] is the bit number corresponding to the
%  type of float I.  To see if there is any float in \LIST having
%  the same type as float I, you run \@bitor with
%    \NUM = [(\count I)/32] * 32.
%
% \@bitor\NUM\LIST ==
%   BEGIN
%      @test :=G false
%      { \@elt \CTR ==  if \NUM <> 0 then
%                          if \count\CTR / \NUM is odd
%                             then  @test := true       fi fi
%        \LIST
%      }
%   END
%
%
% \@cons\LIST\NUM : Globally sets \LIST := \LIST * \@elt \NUM
%
% \@cons\LIST\NUM ==
%   BEGIN {  \@elt == \relax
%            \LIST :=G \LIST \@elt \NUM
%         }
%
%  BOX LISTS FOR FLOAT-PLACEMENT ALGORITHMS
%
%    \@freelist     : List of empty boxes for placing new floats.
%    \@toplist      : List of floats to go at top of current column.
%    \@midlist      : List of floats in middle of current column.
%    \@botlist      : List of floats to go at bottom of current column.
%    \@deferlist    : List of floats to go after current column.
%    \@dbltoplist   : List of double-col. floats to go at top of current
%                     page.
%    \@dbldeferlist : List of double-column floats to go on subsequent
%                     pages.
%
%  FLOAT-PLACEMENT ALGORITHMS
%
%
%  \@addtobot : Tries to put insert \@currbox on \@botlist.
%               Called only when:
%                  * \ht BOX < \@colroom
%                  * type of \@currbox not on \@deferlist
%                  * \@colnum > 0
%                  * @insert = false
%               If it succeeds, then:
%                  * sets @insert true
%                  * decrements \@botroom by \ht BOX
%                  * decrements \@botnum and \@colnum by 1
%                  * decrements \@colroom by \ht BOX + either \floatsep
%                    or \textfloatsep, as appropriate.
%                  * sets \maxdepth to 0pt
%
%  \@addtotoporbot : Tries to put insert \@currbox on \@toplist or
%                    \@botlist.
%                    Called only under same conditions as \@addtobot.
%                    If it succeeds, then:
%                       * sets @insert true
%                       * decrements \@toproom or \@botroom by \ht BOX
%                       * decrements \@colnum and either \@topnum or
%                         \@botnum by 1
%                       * decrements \@colroom by \ht BOX + \floatsep
%                         or \textfloatsep, as appropriate.
%
% \@addtocurcol : Tries to add \@currbox to current column, setting
%                 @insert true if it succeeds, false otherwise.
%                 It will add \@currbox to top only if bit 0 of
%                 \count \@currbox is 0, and to the bottom only if
%                 bit 0 = 0 or an earlier float of the same type is
%                 put on the bottom.
%                 If the float is put in the text, then
%                 \penalty\interlinepenalty is put
%                 right after the float, before the following \vskip,
%                 and \outputpenalty :=L 0.
%
% \@addtonextcol : Tries to add \@currbox to the next column, setting
%                  @insert true if it succeeds, false otherwise.
%
% \@addtodblcol : Tries to add \@currbox to the next double-column page,
%                 adding it to \@dbltoplist if it succeeds and
%                 \@dbldeferlist if it fails.
%
%
%  \@addmarginpar ==
%   BEGIN
%     if \@currlist nonempty
%       then remove \@marbox from \@currlist
%            add \@marbox and \@currbox to \@freelist
%                 %% NOTE: \@currbox = left box
%       else LaTeX error: ?  %% shouldn't happen
%     fi
%     \@tempcnta := 1     %% 1 = right, -1 = left
%     if @twocolumn = true
%       then if @firstcolumn = true
%              then \@tempcnta := -1
%            fi
%       else if @mparswitch = true
%              then if count0 odd
%                     else \@tempcnta := -1
%                   fi
%            fi
%            if @reversemargin = true
%               then \@tempcnta := -\@tempcnta
%            fi
%     fi
%     if \@tempcnta < 0 then \box\@marbox :=G \box\@currbox
%     fi
%     \@tempdima   :=L maximum(\@mparbottom - \@pageht
%                                           + ht of \@marbox, 0)
%     if \@tempdima > 0 then LaTeX warning: 'marginpar moved' fi
%     \@mparbottom :=G \@pageht + \@tempdima + depth of \@marbox
%                          + \marginparpush
%     \@tempdima   :=L \@tempdima - ht of \@marbox
%     \box\@marbox :=G \box\@currbox
%                                \vbox { \vskip \@tempdima
%                                        \box\@marbox
%                                       }
%     height of \@marbox :=G depth of \@marbox :=G 0
%     \kern -\@pagedp
%     \nointerlineskip
%     \hbox{ if @tempcnta > 0 then \hskip \columnwidth
%                                 \hskip \marginparsep
%                            else \hskip -\marginparsep
%                                 \hskip -\marginparwidth
%            fi
%            \box\@marbox \hss
%          }
%     \nobreak
%     \nointerlineskip
%     \hbox{\vrule height 0 width 0 depth \@pagedp}
%   END
%
%   Floats and marginpars add a lot of dead cycles.
%    \begin{macrocode}
\maxdeadcycles = 100
%    \end{macrocode}
%
%    \begin{macrocode}
\let\@elt\relax
%    \end{macrocode}
%
%    \begin{macrocode}
\def\@next#1#2#3#4{\ifx#2\@empty #4\else
   \expandafter\@xnext #2\@@#1#2#3\fi}
%    \end{macrocode}
%
%    \begin{macrocode}
\def\@xnext \@elt #1#2\@@#3#4{\def#3{#1}\gdef#4{#2}}
%    \end{macrocode}
%
% \changes{v1.1v}{1996/07/26}{put \cs{global} into definition}
%    \begin{macrocode}
\def\@testfalse{\global\let\if@test\iffalse}
\def\@testtrue {\global\let\if@test\iftrue}
\@testfalse
%    \end{macrocode}
%
% \changes{v1.1v}{1996/07/26}{remove \cs{global} before \cs{@test...}}
%    \begin{macrocode}
\def\@bitor#1#2{\@testfalse {\let\@elt\@xbitor
   \@tempcnta #1\relax #2}}
%    \end{macrocode}
%    RmS 91/11/22: Added test for |\count#1 = 0|.
%                  Suggested by Chris Rowley.
%
%
% \changes{v1.1v}{1996/07/26}{remove \cs{global} before \cs{@test...}}
%    \begin{macrocode}
\def\@xbitor #1{\@tempcntb \count#1
   \ifnum \@tempcnta =\z@
   \else
     \divide\@tempcntb\@tempcnta
     \ifodd\@tempcntb \@testtrue\fi
   \fi}
%    \end{macrocode}
%
% DEFINITION OF FLOAT BOXES:
% \changes{v1.3a}{2015/09/205}
%         {extended \cs{@freelist}}
%    \begin{macrocode}
%</2ekernel>
%<latexrelease>\IncludeInRelease{2015/10/01}%
%<latexrelease>                 {\bx@ZZ}{Extended float list}%
%<*2ekernel|latexrelease>
\let\@elt\newinsert
%<*2ekernel>
\def\@freelist{%
  \@elt\bx@A\@elt\bx@B\@elt\bx@C\@elt\bx@D\@elt\bx@E
  \@elt\bx@F\@elt\bx@G\@elt\bx@H\@elt\bx@I\@elt\bx@J
  \@elt\bx@K\@elt\bx@L\@elt\bx@M\@elt\bx@N
  \@elt\bx@O\@elt\bx@P\@elt\bx@Q\@elt\bx@R}
\@freelist
%</2ekernel>
\ifx\numexpr\@undefined\else
\def\reserved@a{%
  \@elt\bx@S\@elt\bx@T\@elt\bx@U\@elt\bx@V
  \@elt\bx@W\@elt\bx@X\@elt\bx@Y\@elt\bx@Z
  \@elt\bx@AA\@elt\bx@BB\@elt\bx@CC\@elt\bx@DD\@elt\bx@EE
  \@elt\bx@FF\@elt\bx@GG\@elt\bx@HH\@elt\bx@II\@elt\bx@JJ
  \@elt\bx@KK\@elt\bx@LL\@elt\bx@MM\@elt\bx@NN
  \@elt\bx@OO\@elt\bx@PP\@elt\bx@QQ\@elt\bx@RR
  \@elt\bx@SS\@elt\bx@TT\@elt\bx@UU\@elt\bx@VV
  \@elt\bx@WW\@elt\bx@XX\@elt\bx@YY\@elt\bx@ZZ}
\reserved@a
\def\@elt{\noexpand\@elt\noexpand}
\edef\@freelist{\@freelist\reserved@a}
\fi
\let\reserved@a\relax
\let\@elt\relax
%</2ekernel|latexrelease>
%<latexrelease>\EndIncludeInRelease
%<latexrelease>\IncludeInRelease{0000/00/00}%
%<latexrelease>                 {\bx@ZZ}{Extended float list}%
%<latexrelease>\def\@freelist{%
%<latexrelease>  \@elt\bx@A\@elt\bx@B\@elt\bx@C\@elt\bx@D\@elt\bx@E
%<latexrelease>  \@elt\bx@F\@elt\bx@G\@elt\bx@H\@elt\bx@I\@elt\bx@J
%<latexrelease>  \@elt\bx@K\@elt\bx@L\@elt\bx@M\@elt\bx@N
%<latexrelease>  \@elt\bx@O\@elt\bx@P\@elt\bx@Q\@elt\bx@R}
%<latexrelease>  \insc@unt=234
%<latexrelease>\EndIncludeInRelease
%<*2ekernel>
%    \end{macrocode}
%
%    \begin{macrocode}
\gdef\@toplist{}
\gdef\@botlist{}
\gdef\@midlist{}
\gdef\@currlist{}
\gdef\@deferlist{}
\gdef\@dbltoplist{}
%    \begin{macrocode}
% \changes{v1.2m}{2015/03/12}
%         {initialise \cs{@dbldeferlist} again}
%    The new algorithm stores page wide floats together with column floats
%    in a single |\@deferlist| list. We keep |\@dbldeferlist| 
%    initialised  as empty so that packages that are testing for
%    deferred floats can use the same code for old or new float
%    handling. 
%    \end{macrocode}
\gdef\@dbldeferlist{}
%    \end{macrocode}
%
% PAGE LAYOUT PARAMETERS
%    \begin{macrocode}
\newdimen\topmargin
\newdimen\oddsidemargin
\newdimen\evensidemargin
\let\@themargin=\oddsidemargin
\newdimen\headheight
\newdimen\headsep
\newdimen\footskip
\newdimen\textheight
\newdimen\textwidth
\newdimen\columnwidth
\newdimen\columnsep
\newdimen\columnseprule
\newdimen\marginparwidth
\newdimen\marginparsep
\newdimen\marginparpush
%    \end{macrocode}
% \end{oldcomments}
%
% \begin{macro}{\AtBeginDvi}
% \changes{v1.1c}{1994/11/05}
%         {Added macro}
% \begin{macro}{\@begindvibox}
% \changes{v1.1c}{1994/11/05}
%         {Added macro}
% \changes{v1.1f}{1994/11/14}{Use normal box register: why a box?}
% \changes{v1.1l}{1995/04/24}{Add \cs{vbox} latex/1392}
%
%    We use a box register in which to put
%    stuff that must appear before anything else in the
%    |.dvi| file.
%
%    The stuff in the box should not add any typeset material to the
%    page when it is unboxed.
%    \begin{macrocode}
\newbox\@begindvibox
\def \AtBeginDvi #1{%
  \global \setbox \@begindvibox
    \vbox{\unvbox \@begindvibox #1}%
}
%    \end{macrocode}
% \end{macro}
% \end{macro}
%
%  \begin{macro}{\@maxdepth}
%    This is not the right place to set this; it needs to be set in a
%    class/style file when |\maxdepth| is set.
%
%    Also, many settings to |\maxdepth| should be to |\@maxdepth|,
%    probably?
% \task{All}{Sort out maxdepth: both should be set in the class files.}
%    \begin{macrocode}
\newdimen\@maxdepth
\@maxdepth = \maxdepth
%    \end{macrocode}
%  \end{macro}
% \begin{macro}{\paperheight}
% \changes{v0.1a}{1993/11/23}{Register added}
% \begin{macro}{\paperwidth}
% \changes{v0.1a}{1993/11/23}{Register added}
%    New |\paper|\ldots\ registers.
%    \begin{macrocode}
\newdimen\paperheight
\newdimen\paperwidth
%    \end{macrocode}
% \end{macro}
% \end{macro}
%
% \begin{macro}{\if@insert}
% \begin{macro}{\if@fcolmade}
% \begin{macro}{\if@specialpage}
% \begin{macro}{\if@firstcolumn}
% \begin{macro}{\if@twocolumn}
% \begin{macro}{\if@twoside}
% \begin{macro}{\if@reversemarginpar}
% \begin{macro}{\if@mparswitch}
% \begin{macro}{\col@number}
% \changes{v1.0n}{1994/04/30}{Added \cs{col@number}}
%    Local switches first:
%    \begin{macrocode}
\newif \if@insert
%    \end{macrocode}
%    These should definitely be global:
%    \begin{macrocode}
\newif \if@fcolmade
\newif \if@specialpage \@specialpagefalse
%    \end{macrocode}
%    These should be global but are not always set globally in other
%    files.
%    \begin{macrocode}
\newif \if@firstcolumn \@firstcolumntrue
\newif \if@twocolumn   \@twocolumnfalse
%    \end{macrocode}
%    Not sure about these: two questions.
%    Should things which must apply to a whole document be local or
%    global (they probably should be `preamble only' commands)?
%    Are these three such things?
%    \begin{macrocode}
\newif \if@twoside     \@twosidefalse
\newif \if@reversemargin \@reversemarginfalse
\newif \if@mparswitch  \@mparswitchfalse
%    \end{macrocode}
%    This counter has been imported from `multicol'.
%    \begin{macrocode}
\newcount \col@number
\col@number \@ne
%    \end{macrocode}
% \end{macro}
% \end{macro}
% \end{macro}
% \end{macro}
% \end{macro}
% \end{macro}
% \end{macro}
% \end{macro}
% \end{macro}
%
% \begin{oldcomments}
% INTERNAL REGISTERS
%
%    \begin{macrocode}
\newcount\@topnum
\newdimen\@toproom
\newcount\@dbltopnum
\newdimen\@dbltoproom
\newcount\@botnum
\newdimen\@botroom
\newcount\@colnum
\newdimen\@textmin
\newdimen\@fpmin
\newdimen\@colht
\newdimen\@colroom
\newdimen\@pageht
\newdimen\@pagedp
\newdimen\@mparbottom \@mparbottom\z@
\newcount\@currtype
\newbox\@outputbox
\newbox\@leftcolumn
\newbox\@holdpg
%    \end{macrocode}
%
%    \begin{macrocode}
\def\@thehead{\@oddhead} % initialization
\def\@thefoot{\@oddfoot}
%    \end{macrocode}
% \end{oldcomments}
%
%  \begin{macro}{\clearpage}
% \changes{v1.1r}{1995/10/11}{Added a check so that it does not lose
% the argument of \cs{twocolumn[...]}}
%
% The tests at the beginning are an experimental attempt to avoid a
% completely empty page after a |\twocolumn[...]|.  This prevents the
% text from the argument vanishing into a float box, never to be seen
% again.  We hope that it does not produce wrong formatting in other
% cases.
% \changes{v1.1v}{1996/07/26}{add number of missing percents}
%    \begin{macrocode}
\def\clearpage{%
  \ifvmode
    \ifnum \@dbltopnum =\m@ne
      \ifdim \pagetotal <\topskip
        \hbox{}%
      \fi
    \fi
  \fi
  \newpage
  \write\m@ne{}%
  \vbox{}%
  \penalty -\@Mi
}
%    \end{macrocode}
% \end{macro}
%
%  \begin{macro}{\cleardoublepage}
%
%    \begin{macrocode}
\def\cleardoublepage{\clearpage\if@twoside \ifodd\c@page\else
    \hbox{}\newpage\if@twocolumn\hbox{}\newpage\fi\fi\fi}
%</2ekernel>
%    \end{macrocode}
% \end{macro}
%
%  \begin{macro}{\onecolumn}
% \changes{v1.0n}{1994/04/30}{Added setting of \cs{col@number}}
%    \begin{macrocode}
%<*2ekernel|fltrace>
\def\onecolumn{%
  \clearpage
  \global\columnwidth\textwidth
  \global\hsize\columnwidth
  \global\linewidth\columnwidth
  \global\@twocolumnfalse
  \col@number \@ne
  \@floatplacement}
%    \end{macrocode}
% \end{macro}
%
%  \begin{macro}{\newpage}
% \changes{v1.1x}{1996/09/29}{Checks for noskipsec and inlabel added}
% \changes{v1.1x}{1996/09/29}{Checks for noskipsec and inlabel added}
% \changes{v1.1y}{1996/09/30}{Checks for noskipsec and inlabel removed
%                 pending further tests}
% \changes{v1.1z}{1996/10/24}{Better checks for noskipsec and
%                 inlabel added, plus nobreak}
% \changes{v1.2a}{1996/10/25}{Reset all flags explicitly}
%    The two checks at the beginning ensure that an item label or
%    run-in section title immediately before a |\newpage| get printed
%    on the correct page, the one before the page break.
%
%    All three tests are largely to make error processing more robust;
%    that is why they all reset the flags explicitly, even when it
%    would appear that this would be done by a |\leavevmode|.
%    \begin{macrocode}
\def \newpage {%
  \if@noskipsec
    \ifx \@nodocument\relax
      \leavevmode
      \global \@noskipsecfalse
    \fi
  \fi
  \if@inlabel
    \leavevmode
    \global \@inlabelfalse
  \fi
  \if@nobreak \@nobreakfalse \everypar{}\fi
  \par
  \vfil
  \penalty -\@M}
%    \end{macrocode}
%  \end{macro}
%
%  \begin{macro}{\@emptycol}
% \changes{v1.0n}{1994/04/30}{Empty column action added:
%                 \cs{@emptycol}}
%    It may be better to use an invisible rule rather than an empty
%    box here.
%    \begin{macrocode}
\def \@emptycol {\vbox{}\penalty -\@M}
%    \end{macrocode}
%  \end{macro}
%
%  \begin{macro}{\twocolumn}
%  \begin{macro}{\@topnewpage}
% \changes{v1.0f}{1993/12/05}{Commands changed}
% \changes{v1.0g}{1993/12/06}{\cs{@floatplacement} placement bug
% fixed}
% \changes{v1.0j}{1993/12/17}{check for vsize too small added}
% \changes{v1.2c}{1997/11/09}{Documentation of vsize check enhanced}
%    There are several bug fixes to the two-column stuff here.
%
% \changes{v1.0n}{1994/04/30}{Added setting of \cs{col@number}}
%    \begin{macrocode}
\def \twocolumn {%
  \clearpage
  \global\columnwidth\textwidth
  \global\advance\columnwidth-\columnsep
  \global\divide\columnwidth\tw@
  \global\hsize\columnwidth
  \global\linewidth\columnwidth
  \global\@twocolumntrue
  \global\@firstcolumntrue
  \col@number \tw@
%    \end{macrocode}
%    There is no reason to put a |\@dblfloatplacement| here since
%    |\@topnewpage| ignores these settings.
%    The |\@floatplacement| is needed in case this comes after some
%    changes.
% \changes{v1.0h}{1993/12/12}{braces removed}
%    \begin{macrocode}
  \@ifnextchar [\@topnewpage\@floatplacement
}
%    \end{macrocode}
%
%    Note that here, getting a box from the freelist can assume
%    success since this comes just after a |\clearpage|.
% \changes{v1.0n}{1994/04/30}{Added setting of \cs{col@number}}
% \changes{v1.1a}{1994/10/31}
%     {(DPC/CAR) Use \cs{color@begingroup} for colour}
% \changes{v1.1a}{1994/10/31}
%     {(DPC/CAR) Use \cs{normalcolor}}
% \changes{v1.1a}{1994/10/31}
%     {(DPC/CAR) Extra box added to remove colour resetting from vmode}
% \changes{v1.1b}{1994/11/05}
%         {Use new \cs{color@hbox} concept.}
% \changes{v1.1i}{1994/11/21}{Changed to \cs{color@vbox}}
% \changes{v1.1z}{1996/10/24}{Added \cs{@nodocument} to trap
%                  \cs{twocolumn} in the preamble}
%    \begin{macrocode}
\long\def \@topnewpage [#1]{%
  \@nodocument
  \@next\@currbox\@freelist{}{}%
  \global \setbox\@currbox
    \color@vbox
      \normalcolor
      \vbox {%
        \hsize\textwidth
        \@parboxrestore
        \col@number \@ne
        #1%
        \vskip -\dbltextfloatsep
             }%
    \color@endbox
%    \end{macrocode}
%    Added size test and warning message; perhaps we should use
%    an error message.
% \changes{v1.0l}{1994/03/15}{Corrected and amended warning message}
% \changes{v1.0m}{1994/04/24}{Warning message removed as it will be
% generated later}
%    \begin{macrocode}
  \ifdim \ht\@currbox>\textheight
    \ht\@currbox \textheight
  \fi
%    \end{macrocode}
%    This next line is not essential but it is more robust to make this
%    value non-zero, in case of weird errors.
%
%    This next bit is what is needed from |\@addtodblcol|, plus some
%    extra checks for error trapping.
%    \begin{macrocode}
  \global \count\@currbox \tw@
  \@tempdima -\ht\@currbox
  \advance \@tempdima -\dbltextfloatsep
  \global \advance \@colht \@tempdima
  \ifx \@dbltoplist \@empty
  \else
    \@latexerr{Float(s) lost}\@ehb
    \let \@dbltoplist \@empty
  \fi
  \@cons \@dbltoplist \@currbox
%    \end{macrocode}
%    This setting of |\@dbltopnum| is used only to change the
%    typesetting in\\ |\@combinedblfloats|.
%    \begin{macrocode}
  \global \@dbltopnum \m@ne
%<*trace>
    \fl@trace{dbltopnum set to -1 (= \the \@dbltopnum) (topnewpage)}%
%</trace>
%    \end{macrocode}
%    At points such as this we need to check that there is still a
%    minimal amount of room left on the page; this uses an arbitrary
%    small value at present; but note that this value is larger than
%    that used when checking that page is too full of normal floats.
%
%    If there is little room left we just force a page-break, OK?
%    This involves producing two empty columns.  The second empty
%    column may be produced by |\output|, in which case an extra,
%    misleading, warning will be generated, OK?  (This happens only
%    when there is too little room left on the page for any float.)
%    Otherwise (\ie if the size is such that it is allowed as a normal
%    float) the extra |\@emptycol| will be invoked in the second
%    column by the conditional code guarded by the |\if@firstcolumn|
%    test.
%
%    I now think that the cut-off point here should be |3\baselineskip|,
%    but we make it a bit less so that 3 lines of text will be
%    allowed, OK?
%
%    Since this happens only when there is nothing on the page but the
%    `top-box', the empty box should not cause any problem other than
%    some overfull box messages, which is not entirely misleading.
%
%    Here we need two page-ends since both columns need to be empty.
% \task{???}{Make it less arbitrary?}
% \changes{v1.0j}{1993/12/17}{Page room test added}
% \changes{v1.0l}{1994/03/15}{Warning added: it should be improved}
% \changes{v1.0m}{1994/04/24}{Message changed to give more info}
% \changes{v1.0n}{1994/04/30}{Message changed for Frank}
% \changes{v1.0n}{1994/04/30}{Empty column action added:
% \cs{@emptycol}}
% \changes{v1.0n}{1994/04/30}{Cut-off point changed to
% 3\cs{baselineskip}}
% \changes{v1.1t}{1996/05/24}{Cut-off point changed to
% 2.5\cs{baselineskip}}
%    \begin{macrocode}
  \ifdim \@colht<2.5\baselineskip
    \@latex@warning@no@line {Optional argument of \noexpand\twocolumn
                too tall on page \thepage}%
    \@emptycol
    \if@firstcolumn
    \else
      \@emptycol
    \fi
  \else
    \global \vsize \@colht
    \global \@colroom \@colht
    \@floatplacement
  \fi
}
%    \end{macrocode}
%  \end{macro}
%  \end{macro}
%
%
%  \begin{macro}{\output}
% \changes{v1.0f}{1993/12/05}{Command changed}
%  \begin{macro}{\@specialoutput}
% \changes{v0.1c}{1993/11/23}{Command changed}
% \changes{v1.0f}{1993/12/06}{Unboxing of 255 added to rescue writes}
%    This needs some small adjustments.  We cannot
%    guarantee that the float mechanism will interact correctly with
%    this stuff, but that mechanism does not always work properly
%    with footnotes already.
%
%    RmS 91/09/29:
%
%    added reset of |\par| to the output routine.
%    This avoids problems when the output routine is
%    called within a list where |\par| may be a no-op.
%
%    \begin{macrocode}
\output {%
  \let \par \@@par
  \ifnum \outputpenalty<-\@M
    \@specialoutput
  \else
    \@makecol
    \@opcol
%    \end{macrocode}
%    Moved to |\@opcol|: |\@floatplacement|.
%    \begin{macrocode}
    \@startcolumn
%    \end{macrocode}
%    This loop could be replaced by an |\expandafter| tail
%    recursion in |\@startcolumn|.
%    \begin{macrocode}
    \@whilesw \if@fcolmade \fi
      {%
%<*trace>
       \fl@trace{PAGE: float \if@twocolumn column \else page \fi
                    completed}%
%</trace>
       \@opcol\@startcolumn}%
  \fi
  \ifnum \outputpenalty>-\@Miv
%    \end{macrocode}
%    At points such as this we need to check that there is still a
%    minimal amount of room left on the page; this uses an arbitrary
%    small value at present.  If there is little room left we just
%    force a page-break, OK?
%
%    This bit is essential only if a float has just been processed so
%    maybe it should be moved; but this is the natural place at which
%    to set the vsize and a test would need to be done anyway.  A
%    check has been added to ensure that there really has been a
%    change in the value of |\@colroom|.
%
%    Since this happens only when there is nothing on the page but
%    floats, the empty box should not cause any problem other than
%    some overfull box messages, which is not entirely misleading.
%
%    The twocolumn case does not need any extra code here since this
%    is the |\output| itself; in the second column there will still
%    not be enough room left so |\@emptycol| will be executed again
%    when the OR is called by the-page builder when it gets to the
%    penalty inserted by the first execution.  (The page-builder is
%    never invoked whilst the OR is being executed since it builds a
%    inner vlist; thus any conditional code for the two-column case
%    within |\output| may not get executed with the correct value of
%    |\if@firstcolumn|.
%
% \task{???}{Make it less arbitrary?}
% \changes{v1.0j}{1993/12/17}{Page room test added}
% \changes{v1.0m}{1994/04/24}{Message changed to give more info and
% `top' removed}
% \changes{v1.0n}{1994/04/30}{Extra empty column added for
% twocolumn case (wrong, see below)}
% \changes{v1.0n}{1994/04/30}{Extra empty column added for
% twocolumn case}
% \changes{v1.0n}{1994/04/30}{Empty column action added:
% \cs{@emptycol}}
% \changes{v1.0n}{1994/04/30}{Cut-off point changed to
% 2\cs{baselineskip}}
% \changes{v1.1t}{1996/05/24}{Cut-off point changed to
% 1.5\cs{baselineskip}}
% \changes{v1.1t}{1996/05/24}{Check that \cs{@colroom} is less
% than \cs{vsize}, indicating that a float has been added}
% \changes{v1.1u}{1996/05/25}{Correct the above check}
% \changes{v1.2c}{1997/11/09}{Remove incorrect code: only one
% \cs{@emptycol} is needed here}
%    \begin{macrocode}
    \ifdim \@colroom<1.5\baselineskip
      \ifdim \@colroom<\textheight
        \@latex@warning@no@line {Text page \thepage\space
                               contains only floats}%
        \@emptycol
%         \if@twocolumn
%           \if@firstcolumn
%           \else
%             \@emptycol
%           \fi
%         \fi
      \else
        \global \vsize \@colroom
      \fi
    \else
      \global \vsize \@colroom
    \fi
  \else
    \global \vsize \maxdimen
  \fi
}
%    \end{macrocode}
%
% \begin{oldcomments}
% CHANGES TO \@specialoutput:
% * \penalty\z@ changed to \penalty\interlinepenalty so \samepage
%   works properly with figure and table environments.
%   (Changed 23 Oct 86)
%
% * Definition of \@specialoutput changed 26 Feb 88 so \@pageht and
%   \@pagedp aren't changed for a marginal note.
%   (Change suggested by Chris Rowley.)
% \end{oldcomments}
%
%    \begin{macrocode}
\gdef\@specialoutput{%
   \ifnum \outputpenalty>-\@Mii
     \@doclearpage
   \else
     \ifnum \outputpenalty<-\@Miii
       \ifnum \outputpenalty<-\@MM \deadcycles \z@ \fi
       \global \setbox\@holdpg \vbox {\unvbox\@cclv}%
     \else
%    \end{macrocode}
%    Note that |\boxmaxdepth| should not be set here since we wish to
%    record the natural depth of the holdpg box.
%
%    This is changed so as to not lose anything, such as writes
%    and marks, which may get into box 255 and should be returned to
%    the list.  This should only happen when the first penalty in the
%    mechanism is discarded and therefore |\@holdpg| should always be
%    void in this case.  This can happen because a penalty is
%    discarded whenever there is no box on the list.
%
%    It was just: |\setbox\@tempboxa \box \@cclv|.
%
%    The last box which is removed is the box put there by the
%    double-penalty mechanism.  The |\unskip| then removes the
%    |\topskip| which is put there since the box is the first on the
%    page.
% \task{CAR/FMi}{Is it more efficient to check whether the holdpg box
% is void and then act accordingly?}
%    \begin{macrocode}
       \global \setbox\@holdpg \vbox{%
                      \unvbox\@holdpg
                      \unvbox\@cclv
%    \end{macrocode}
%    We must now remove the box added by the float mechanism and the
%    |\topskip| glue therefore added above it by \TeX.
%    \begin{macrocode}
                      \setbox\@tempboxa \lastbox
                      \unskip
                                     }%
%    \end{macrocode}
%    These two are needed as separate dimensions only by
%    |\@addmarginpar|; for other purposes we put the whole size into
%    |\@pageht| (see below).
%    \begin{macrocode}
       \@pagedp \dp\@holdpg
       \@pageht \ht\@holdpg
       \unvbox \@holdpg
       \@next\@currbox\@currlist{%
         \ifnum \count\@currbox>\z@
%    \end{macrocode}
%    Putting the whole size into |\@pageht| (see above).
%    \begin{macrocode}
           \advance \@pageht \@pagedp
           \ifvoid\footins \else
             \advance \@pageht \ht\footins
             \advance \@pageht \skip\footins
             \advance \@pageht \dp\footins
           \fi
           \ifvbox \@kludgeins
%    \end{macrocode}
%    We want to make the adjustment due to this insert only if the
%    non-star form is used.  The *-form will probably not work with
%    floats, but maybe it still could make some adjustment here even
%    so?
%    \begin{macrocode}
             \ifdim \wd\@kludgeins=\z@
               \advance \@pageht \ht\@kludgeins
%<*trace>
               \fl@trace {Extra size added: \the \ht\@kludgeins}%
%</trace>
             \fi
           \fi
%    \end{macrocode}
%    This version puts the inserts back just before the additional
%    material; it could be moved earlier, before unboxing the
%    page-so-far.  Neither is guaranteed not to put things on the wrong
%    page.  This version is similar to the original version.
%    \begin{macrocode}
           \@reinserts
           \@addtocurcol
         \else
           \@reinserts
           \@addmarginpar
         \fi
         }\@latexbug
%    \end{macrocode}
%    A 2e change: use |\addpenalty| instead of |\penalty| here.  Some
%    penalty is needed to create a potential break-point immediately
%    after the reinserts (or the marginal).  Otherwise there can be no
%    possibility to break here and this can cause the reinserts or the
%    marginal to appear on the next page (which is often incorrect).
%    However, if the nobreak flag is true, a |\nobreak| must be
%    correct.
% \changes{v1.1i}{1994/11/21}{Added \cs{if@nobreak} test}
% \changes{v1.1z}{1996/10/24}{Added \cs{nobreak} as appropriate}
%    \begin{macrocode}
       \ifnum \outputpenalty<\z@
         \if@nobreak
           \nobreak
         \else
           \addpenalty \interlinepenalty
         \fi
       \fi
     \fi
   \fi
}
%</2ekernel|fltrace>
%    \end{macrocode}
%  \end{macro}
%  \end{macro}
%
% \begin{macro}{\@testwrongwidth}
% \begin{macro}{\f@depth}
% \changes{v1.2n}{2015/02/21}{macro added(latexrelease)}
%
% Test if the float box has the wrong width when trying to place it
% into some area. (Actually
% the test is for a conventional depth setting rather than for the
% width of the float. For that reason the box depth was explicitly
%   tailored when the float was created).
%    \begin{macrocode}
%<latexrelease>\IncludeInRelease{2015/01/01}%
%<latexrelease>                 {\@testwrongwidth}{float order in 2-column}%
%<*2ekernel|latexrelease|fltrace>
%    \end{macrocode}
%
%    \begin{macrocode}
\def\@testwrongwidth #1{%
  \ifdim\dp#1=\f@depth
%<*trace>
    \fl@trace{\string#1 
              \ifdim\f@depth=\z@ single \else double \fi 
              column float -- ok}%
%</trace>
  \else
    \global\@testtrue
%<*trace>
    \fl@trace{\string#1
              \ifdim\f@depth=\z@ double \else single \fi
              column float -- wrong}%
%</trace>
  \fi}%
%    \end{macrocode}
%
% Normally looking for single column floats, which have zero depth.
%    \begin{macrocode}
\let\f@depth\z@
%    \end{macrocode}
%
%    \begin{macrocode}
%</2ekernel|latexrelease|fltrace>
%<latexrelease>\EndIncludeInRelease
%<latexrelease>\IncludeInRelease{0000/00/00}%
%<latexrelease>                 {\@testwrongwidth}{float order in 2-column}%
%<latexrelease>\let\@testwrongwidth\@undefined
%<latexrelease>\let\f@depth\@undefined
%<latexrelease>\EndIncludeInRelease
%    \end{macrocode}
% \end{macro}
% \end{macro}
%
%
%  \begin{macro}{\@doclearpage}
%  \task{CAR/FMi}{Investigate better float clearance}
%  \task{CAR/FMi}{Put non-boxes back in the `right place', or nowhere?}
%    This is a very much an emergency action, just dumping everything:
%    footnotes first then floats.  A more sophisticated version is
%    needed; but even more urgent is a bug-free version (see, for
%    example, pr/3528).
%
%    Also, it puts any left-over non-boxes (writes, specials, etc.) back
%    after any float pages created: this is a very bad bug since,
%    for example, a kludge insert will be in quite the wrong place
%    and, worse, be irremovable and uncancelable.
%
% \changes{v1.2n}{2015/01/14}{float order in 2-column (latexrelease)}
% All the remaining changes are replacing the double column defer list
% or inserting the extra test |\@testwrongwidth|\marg{box} at suitable
% places. That is at places where a box is taken off the deferlist.
%    \begin{macrocode}
%<latexrelease>\IncludeInRelease{2015/01/01}{\@doclearpage}%
%<latexrelease>                             {float order in 2-column}%
%<*2ekernel|latexrelease>
\def \@doclearpage {%
     \ifvoid\footins
%    \end{macrocode}
% \changes{v1.2n}{2015/01/14}{Empty kludgeins box if necessary, pr/3528}
%    \begin{macrocode}
       \ifvbox\@kludgeins
         {\setbox \@tempboxa \box \@kludgeins}%
%<*trace>
         \fl@trace {kludgeins box made void}%
%</trace>
       \fi
       \setbox\@tempboxa\vsplit\@cclv to\z@ \unvbox\@tempboxa
       \setbox\@tempboxa\box\@cclv
       \xdef\@deferlist{\@toplist\@botlist\@deferlist}%
       \global \let \@toplist \@empty
       \global \let \@botlist \@empty
       \global \@colroom \@colht
       \ifx \@currlist\@empty
       \else
          \@latexerr{Float(s) lost}\@ehb
          \global \let \@currlist \@empty
       \fi
       \@makefcolumn\@deferlist
       \@whilesw\if@fcolmade \fi{\@opcol\@makefcolumn\@deferlist}%
       \if@twocolumn
         \if@firstcolumn
%    \end{macrocode}
%
%    \begin{macrocode}
           \xdef\@deferlist{\@dbltoplist\@deferlist}%
%    \end{macrocode}
%
%    \begin{macrocode}
           \global \let \@dbltoplist \@empty
           \global \@colht \textheight
           \begingroup
              \@dblfloatplacement
%    \end{macrocode}
%
%    \begin{macrocode}
              \@makefcolumn\@deferlist
              \@whilesw\if@fcolmade \fi{\@outputpage
                                        \@makefcolumn\@deferlist}%
%    \end{macrocode}
%
%    \begin{macrocode}
           \endgroup
         \else
           \vbox{}\clearpage
         \fi
       \fi
%    \end{macrocode}
%    the next line is needed to avoid losing floats in certain
%    circumstances a single call to the original |\doclearpage|
%    will now no longer output all floats.
% \changes{v1.2b}{2000/09/24}{FMi: ensure \cs{doclearpage}
%     is called again until all floats are output.}
%    \begin{macrocode}
       \ifx\@deferlist\@empty \else\clearpage \fi
     \else
       \setbox\@cclv\vbox{\box\@cclv\vfil}%
       \@makecol\@opcol
       \clearpage
     \fi
}%
%</2ekernel|latexrelease>
%<latexrelease>\EndIncludeInRelease
%<latexrelease>\IncludeInRelease{0000/00/00}{\@doclearpage}%
%<latexrelease>                             {float order in 2-column}%
%<latexrelease>\def \@doclearpage {%
%<latexrelease>     \ifvoid\footins
%    \end{macrocode}
% \changes{v1.2l}{2004/02/07}{Empty kludgeins box if necessary, pr/3528}
%    We empty any left over kludge insert box here; this is a temporary fix.
%    It should perhaps be applied to one page of cleared floats, but
%    who cares?  The whole of this stuff needs completely redoing for
%    many such reasons.
%    \begin{macrocode}
%<latexrelease>       \ifvbox\@kludgeins
%<latexrelease>         {\setbox \@tempboxa \box \@kludgeins}%
%<*trace>
%<latexrelease>         \fl@trace {kludgeins box made void}%
%</trace>
%<latexrelease>       \fi
%<latexrelease>       \setbox\@tempboxa\vsplit\@cclv to\z@ \unvbox\@tempboxa
%<latexrelease>       \setbox\@tempboxa\box\@cclv
%<latexrelease>       \xdef\@deferlist{\@toplist\@botlist\@deferlist}%
%    \end{macrocode}
% \changes{v1.0h}{1993/12/12}{defs changed to lets}
%    \begin{macrocode}
%<latexrelease>       \global \let \@toplist \@empty
%<latexrelease>       \global \let \@botlist \@empty
%<latexrelease>       \global \@colroom \@colht
%<latexrelease>       \ifx \@currlist\@empty
%<latexrelease>       \else
%<latexrelease>          \@latexerr{Float(s) lost}\@ehb
%    \end{macrocode}
% \changes{v1.0h}{1993/12/12}{defs changed to lets}
%    \begin{macrocode}
%<latexrelease>          \global \let \@currlist \@empty
%<latexrelease>       \fi
%<latexrelease>       \@makefcolumn\@deferlist
%<latexrelease>       \@whilesw\if@fcolmade \fi
%<latexrelease>                     {\@opcol\@makefcolumn\@deferlist}%
%<latexrelease>       \if@twocolumn
%<latexrelease>         \if@firstcolumn
%<latexrelease>           \xdef\@dbldeferlist{\@dbltoplist\@dbldeferlist}%
%    \end{macrocode}
% \changes{v1.0h}{1993/12/12}{defs changed to lets}
%    \begin{macrocode}
%<latexrelease>           \global \let \@dbltoplist \@empty
%<latexrelease>           \global \@colht \textheight
%<latexrelease>           \begingroup
%<latexrelease>              \@dblfloatplacement
%<latexrelease>              \@makefcolumn\@dbldeferlist
%<latexrelease>              \@whilesw\if@fcolmade \fi
%<latexrelease>                    {\@outputpage\@makefcolumn\@dbldeferlist}%
%<latexrelease>           \endgroup
%<latexrelease>         \else
%<latexrelease>           \vbox{}\clearpage
%<latexrelease>         \fi
%<latexrelease>       \fi
%<latexrelease>     \else
%<latexrelease>       \setbox\@cclv\vbox{\box\@cclv\vfil}%
%<latexrelease>       \@makecol\@opcol
%<latexrelease>       \clearpage
%<latexrelease>     \fi
%<latexrelease>  }%
%<latexrelease>\EndIncludeInRelease
%    \end{macrocode}
% \end{macro}
%
%  \begin{macro}{\@opcol}
% \changes{v1.0f}{1993/12/05}{Command changed}
% \changes{v1.0f}{1993/12/05}{Hook added}
% \changes{v1.0j}{1993/12/17}{Hook removed}
%    Several changes in detail here.
%
%    \begin{macrocode}
%<*2ekernel|fltrace>
\def \@opcol {%
  \if@twocolumn
    \@outputdblcol
  \else
    \@outputpage
%<*trace>
    \fl@trace{PAGE: one column (float? see above) page completed}%
%</trace>
%    \end{macrocode}
%    Not needed since it comes after |\@outputpage|:
%    \begin{macrocode}
%   \global\@colht\textheight
  \fi
%    \end{macrocode}
%    These do not need to be done every time |\@opcol| is used: they
%    should be grouped together since they all need to be done at the
%    end of the non-special output routine, or at the end of a clearpage
%    one.
%    \begin{macrocode}
  \global \@mparbottom \z@ \global \@textfloatsheight \z@
  \@floatplacement
}
%</2ekernel|fltrace>
%    \end{macrocode}
%  \end{macro}
%
%
%  \begin{macro}{\@makecol}
% \changes{v0.1c}{1993/11/23}{Command changed}
% \changes{v1.0b}{1993/11/29}{\cs{@makespecialcolbox} added}
%    We must rewrite this macro to alllow for variations in page-makeup
%    required by changes in page-length.
%
%    This uses a different macro if a special-length column is being
%    produced.
%
%    \begin{macrocode}
%<*2ekernel>
\gdef \@makecol {%
   \ifvoid\footins
     \setbox\@outputbox \box\@cclv
   \else
     \setbox\@outputbox \vbox {%
%    \end{macrocode}
%    This |\boxmaxdepth| setting is to ensure that  deep footnotes
%    do not overwrite the footer (on account of the negative skip
%    added later): it should use |\@maxdepth| otherwise the change is
%    pointless when there are footnotes.
% \task{CAR}{Investigate providing an option to put the footnotes
%    below the bottom floats.}
%
%    But see also its use when combining floats.
% \changes{v1.0l}{1994/03/15}{\cs{maxdepth} changed to \cs{@maxdepth}}
%    \begin{macrocode}
       \boxmaxdepth \@maxdepth
%    \end{macrocode}
%
% \changes{v1.2e}{1999/01/06}{Added negative vskip, as when processing
%      outputbox below: suggested by Fred Bartlett pr/2892}
% \changes{1.2i}{2000/07/12}{Removed negative vskip, as it gives
%    unacceptable results when the depth is large: pr/3189}
% \changes{v1.2m}{2004/10/20}{Removed dead code}
% \task{CAR}{As can be seen from the above two changes, getting a good
%    layout here is difficult: fix!}
%    \begin{macrocode}
%       \@tempdima\dp\@cclv
       \unvbox \@cclv
%       \vskip-\@tempdima
       \vskip \skip\footins
%    \end{macrocode}
% \changes{v1.1a}{1994/10/31}
%     {(DPC/CAR)  Colour resetting moved to here}
%    \begin{macrocode}
       \color@begingroup
         \normalcolor
         \footnoterule
         \unvbox \footins
       \color@endgroup
       }%
   \fi
%    \end{macrocode}
%    The h floats have now been finally committed to this page so we
%    can reset their list.  The top and bottom floats are then added
%    to the page.
% \changes{v1.2k}{2003/12/16}{Ensure that \cs{@elt} has a defined
%                             state (pr/3586)}
%    \begin{macrocode}
   \let\@elt\relax
   \xdef\@freelist{\@freelist\@midlist}%
%    \end{macrocode}
% \changes{v1.0h}{1993/12/12}{defs changed to lets}
%    \begin{macrocode}
   \global \let \@midlist \@empty
   \@combinefloats
%    \end{macrocode}
%    The variations start here in case |\enlargethispage| has
%    been used.
%    \begin{macrocode}
   \ifvbox\@kludgeins
     \@makespecialcolbox
   \else
%    \end{macrocode}
%    This extra reboxing is only needed to add the
%    |\@texttop| and |\@textbotttom| but this could be done earlier,
%    when the floats are added.
%
%    The |\boxmaxdepth| resetting here will have no effect unless
%    |\@textbottom| ends with a box or rule.  So is this (or possibly
%    |\@maxdepth|) the correct value?
%
%    The |\vskip -\dimen@|
%    ensures that the visible depth of the box does not
%    affect the placement of anything on the page.
%    Thus very deep pages will overprint the footer; but these should
%    have been prevented by suitable settings of the maxdepths at
%    appropriate times.
%
%    If |\@textbottom| ends with a box or rule of non-zero depth
%    then this skip adjustment should be done again after it.
%
%    I think that the final boxing of the main text page could have a
%    common ending which may make it simpler to see what is going on.
%
%    This needs further investigation, especially in the `special
%    case'.
%    \task{CAR}{Further investigation of makecol}
%
%    Also, the |\boxmaxdepth| setting here affects what happens within
%    |\@texttop| and |\@textbottom|, should it?  Is it needed at all?
% \changes{v1.0l}{1994/03/15}{Removed boxmaxdepth setting.}
%
% RmS 91/10/22: Replaced |\dimen128| by |\dimen@|.
%
%    \begin{macrocode}
     \setbox\@outputbox \vbox to\@colht {%
%       \boxmaxdepth \maxdepth                   %??
       \@texttop
       \dimen@ \dp\@outputbox
       \unvbox \@outputbox
       \vskip -\dimen@
       \@textbottom
       }%
   \fi
   \global \maxdepth \@maxdepth
}
%    \end{macrocode}
%  \end{macro}
%
%  \begin{macro}{\@reinserts}
% \changes{v1.0e}{1993/12/01}{Command added}
%    This is the code which reinserts the inserts.  It puts them all
%    in one place; this can make some of them come out on the wrong
%    page.
%    It has been put into a separate macro to expedite experimentation.
%    \begin{macrocode}
\gdef \@reinserts{%
  \ifvoid\footins\else\insert\footins{\unvbox\footins}\fi
  \ifvbox\@kludgeins\insert\@kludgeins
                      {\unvbox\@kludgeins}\fi
}
%</2ekernel>
%    \end{macrocode}
%  \end{macro}
%
%
%
%  \begin{macro}{\@makespecialcolbox}
% \changes{v1.0b}{1993/11/29}{Command added}
%    This implements certain variations in page-makeup.
%    \begin{macrocode}
%<*2ekernel|fltrace>
\gdef \@makespecialcolbox {%
%<*trace>
   \fl@trace{Kludgeins ht \the\ht\@kludgeins\space
                           dp \the\dp\@kludgeins\space
                           wd \the\wd\@kludgeins}%
%</trace>
%    \end{macrocode}
%    First we find the natural height of the column.
%
%    See above for discussion of what is happening here.
%
%    This needs further investigation, especially in this `special
%    case'.
% \changes{v1.0k}{1994/02/08}{boxmaxdepth setting removed}
%    \begin{macrocode}
   \setbox\@outputbox \vbox {%
     \@texttop
     \dimen@ \dp\@outputbox
     \unvbox\@outputbox
     \vskip-\dimen@
     }%
   \@tempdima \@colht
   \ifdim \wd\@kludgeins>\z@
%    \end{macrocode}
%    Note that in this case (the *-version), the height of the
%    |\@kludgeins| box is not used since its value is somewhat
%    arbitrary: it need only be big enough to ensure that the
%    page-break is not taken prematurely.
%
%    Here we calculate how much vertical space needs to be added in
%    order to enable the column to fit into a box of size |\@colht|
%    using the best information we have about the amount of shrink
%    available (another thing which is known internally about a box,
%    but cannot be accessed at the \TeX{} level!).
%
%    This needs \TeX3 otherwise |\pageshrink| is zero anyway; it may
%    not be exactly the figure we wish as it is the total available
%    from the all the material collected before the page-break
%    decision is made.  It will, we think, always be an overestimate
%    of the actual shrink in the box; therefore this should always
%    force the shortest possible column with the possibility of an
%    overfull box.
%
%    This should work for bothe flush- and ragged-bottom setting since
%    it makes the contents no smaller than the size (|\@colht|) of the
%    box into which they are put.
%
%    Their should perhaps be an upper limit, of 0pt?, on the extra
%    space added to force shrinking.
%    \task{CAR}{Further investigation of kludge-* space}
%
%    See above for a discussion of the |\boxmaxdepth| setting here.
%
% \changes{v1.0k}{1994/02/08}{boxmaxdepth setting added}
% \changes{v1.0l}{1994/03/15}{Removed boxmaxdepth setting.}
%    \begin{macrocode}
     \advance \@tempdima -\ht\@outputbox
     \advance \@tempdima \pageshrink
%<*trace>
     \fl@trace {Natural ht of col: \the \ht\@outputbox}%
     \fl@trace {\string \@colht: \the \@colht}%
     \fl@trace {Pageshrink added: \the \pageshrink}%
     \fl@trace {Hence, space added: \the \@tempdima}%
%</trace>
     \setbox\@outputbox \vbox to \@colht {%
%       \boxmaxdepth \maxdepth
       \unvbox\@outputbox
       \vskip \@tempdima
       \@textbottom
       }%
%    \end{macrocode}
%    For the unstarred version, the final size of the page is
%    precisely specified.  Therefore, at least for the flush-bottom
%    case, we need to ensure that, visually, it has this size exactly.
%
%    Thus we calculate this size and set the material in a box of this
%    size, which is then put into a box of size |\@colht| with |\vss|
%    at the bottom.
%    \begin{macrocode}
   \else
     \advance \@tempdima -\ht\@kludgeins
%<*trace>
     \fl@trace {Natural ht of col: \the \ht\@outputbox}%
     \fl@trace {\string \@colht: \the \@colht}%
     \fl@trace {Extra size added: -\the \ht \@kludgeins}%
     \fl@trace {Hence, height of inner box: \the \@tempdima}%
     \fl@trace {Max? pageshrink available: \the \pageshrink}%
%</trace>
%    \end{macrocode}
%    This type of final packaging could be done always; this may
%    simplify all of this page-makeup.
%
%    It is not necessary to set |\boxmaxdepth| here since the
%    |\@outputbox| ends with glue.
% \changes{v1.0k}{1994/02/03}{correct mistakes in the documentation}
%    \begin{macrocode}
     \setbox \@outputbox \vbox to \@colht {%
       \vbox to \@tempdima {%
         \unvbox\@outputbox
         \@textbottom}%
       \vss}%
   \fi
%    \end{macrocode}
%    Finally we need to explicitly make the insert box void.
%    \begin{macrocode}
   {\setbox \@tempboxa \box \@kludgeins}%
%<*trace>
     \fl@trace {kludgeins box made void}%
%</trace>
}
%</2ekernel|fltrace>
%    \end{macrocode}
%  \end{macro}
%
%  \begin{macro}{\@texttop}
%  \begin{macro}{\@textbottom}
%    These do nothing as a default.
%    \begin{macrocode}
%<*2ekernel>
\let \@texttop \relax
\let \@textbottom \relax
%    \end{macrocode}
%  \end{macro}
%  \end{macro}
%
%  \begin{macro}{\@resetactivechars}
%  \begin{macro}{\@activechar@info}
%
% \changes{v1.0n}{1994/04/30}{\cs{@activechar@warning} changed to
% \cs{@activechar@info}}
% RmS 93/09/06:
% added hook to protect against certain active characters in the
% output routine. Default checks are for active space and end-of-line.
% \changes{v1.0u}{1994/05/23}{Added \cs{MessageBreak}}
%    \begin{macrocode}
\def\@activechar@info #1{%
      \@latex@info@no@line {Active #1 character found while
                            output routine is active
                            \MessageBreak
                            This may be a bug in a package file
                            you are using}%
}
%    \end{macrocode}
%
%    Do not put any spaces in this next bit!
% \changes{1.2f}{1999/10/09}{Reset definition of active prime
%    character (used in math mode)}
%    \begin{macrocode}
\begingroup
\obeylines\obeyspaces%
\catcode`\'\active%
\gdef\@resetactivechars{%
\def^^M{\@activechar@info{EOL}\space}%
\def {\@activechar@info{space}\space}%
\let'\active@math@prime}%
\endgroup
%    \end{macrocode}
%  \end{macro}
%  \end{macro}
%
%  \begin{macro}{\@outputpage}
% \changes{v1.0f}{1993/12/05}{Command changed}
%  \begin{macro}{\@shipoutsetup}
% \changes{v1.0f}{1993/12/05}{Command added}
% \changes{v1.1s}{1995/11/10}{Command removed}
%  \begin{macro}{\@writesetup}
% \changes{v1.1s}{1995/11/10}{Command removed}
% \changes{v1.1o}{1995/07/21}{Command added}
% \changes{v1.0f}{1993/12/05}{\cs{@startpagehook} added}
% \changes{v1.0j}{1993/12/17}{---and then removed}
% \changes{v1.0p}{1994/05/12}{\cs{normalcolor}added}
% \changes{v1.0s}{1994/05/20}{Added setting of \cs{protect}
%                    during \cs{shipout}.}
% \changes{v1.0u}{1994/05/23}{Changed resetting of \cs{protect} after
% shipout to use \cs{aftergroup}}
% \changes{v1.1b}{1994/11/05}
%         {Use new \cs{color@hbox} concept.}
% \changes{v1.1b}{1994/11/05}
%         {Change protect settings for new-style, protect-free
%         aux-files.}
% \changes{v1.1c}{1994/11/05}
%         {Add new \cs{AtBeginDvi} concept}
% \changes{v1.1e}{1994/11/10}
%         {Modify \cs{protect} setting}
% \changes{v1.1f}{1994/11/14}
%         {Modify new \cs{AtBeginDvi} concept}
% \changes{v1.1o}{1995/07/21}
%         {New, experimental, versions: need in-lining}
% \changes{v1.1s}{1995/11/10}{In-lined}
% \changes{v1.1w}{1996/09/21}{Added \cs{@parboxrestore} and made
% consequent deletions: wait for the howls of protest}
%
%    The |\color@hbox| hooks here are used to avoid putting just a
%    colour special into an otherwise empty box (in a header or
%    footer).  These boxes are often set to be completely empty and so
%    adding a special produces a very underfull box message.
%
%    There has been extensive tidying up of the old code here;
%    including the removal of a level of grouping.
%
%    The setting of |\protect| immediately before the |\shipout|
%    is needed so that protected commands within |\write|s are
%    handled correctly.
%
%    Within shipout's vbox it is reset to its default value, |\relax|.
%
%    Resetting it to its default value after the shipout has been
%    completed (and the contents of the writes have been expanded)
%    must be done by use of |\aftergroup|.
%    This is because it must have the value |\relax|
%    before macros coming from other uses of |\aftergroup| within
%    this box are expanded.
%
%    Putting this into the |\aftergroup| token list does not affect
%    the definition used in expanding the |\write|s because the
%    aftergroup token list is only constructed when popping the
%    save-stack, it is not expanded until after the shipout is
%    completed.
%
%    Question: should things from an |\aftergroup| within the shipped
%    out box be executed in the environment set up for the writes, or
%    after it finishes?
%
%    A lot of this code has been in-lined tp prevent mis-use of
%    internal commands as hooks.
%    \begin{macrocode}
\def\@outputpage{%
\begingroup           % the \endgroup is put in by \aftergroup
%    \end{macrocode}
%    Now all the set-up stuff has been in-lined for Frank.
%
%    First the stuff for the writes.
%
%    From here \ldots\ was in the command |\@writesetup|.
%    \begin{macrocode}
  \let \protect \noexpand
%    \end{macrocode}
%
%    RmS 93/08/19: Redefined accents to allow changes in font encoding;
%    but exactly why was this needed?
%
%    The |\catcode`\ = 10| was removed as it was considered useless
%    (presumably because nothing gets tokenised during shipout).
%
%    This was put in as some error produced active spaces in a mark, I
%    think.
%
%    Why was the hyphen reset?
%
%    \begin{macrocode}
  \@resetactivechars
%    \end{macrocode}
%    If a page break happens between the start of a list and its first
%    item the |@newlist| will be true and this will mess up any list
%    that is used in the header or footer of the page. So we have to
%    reset that flag.
% \changes{v1.2h}{2000/07/19}
%  {Reset and restore \cs{@if@newlist} for internal/3231}
% \changes{v1.2j}{2001/01/07}
%  {And do it in the right macro! (pr/3286)}
%    \begin{macrocode}
  \global\let\@@if@newlist\if@newlist
  \global\@newlistfalse
%    \end{macrocode}
%
% \changes{v1.0q}{1994/05/16}{Changed setting of accents (FMi):
%     with the new encoding setup they can use \cs{let}.
%     It could also use the new internal commands?}
% \changes{v1.1l}{1995/04/24}{Reset \cmd\\ latex/1451 (DPC)}
%    This next hook replaces the following:
%    \begin{verbatim}
%      \let\-\@dischyph
%      \let\'\@acci\let\`\@accii\let\=\@acciii
%      \let\\\@normalcr
%      \let\par\@@par %% 15 Sep 87 (this was once inside the box)
%    \end{verbatim}
%    and it does more than they did; in particular it sets:
%    \begin{verbatim}
%      \parindent\z@
%      \parskip\z@skip
%      \everypar{}%
%      \leftskip\z@skip
%      \rightskip\z@skip
%      \parfillskip\@flushglue
%      \lineskip\normallineskip
%      \baselineskip\normalbaselineskip
%      \sloppy
%    \end{verbatim}
%
%    \begin{macrocode}
  \@parboxrestore
%    \end{macrocode}
%    \ldots\ to here was in the command |\@writesetup|.
%    \begin{macrocode}
  \shipout \vbox{%
    \set@typeset@protect
    \aftergroup \endgroup
    \aftergroup \set@typeset@protect
                                % correct? or just restore by ending
                                % the group?
%    \end{macrocode}
%    This first bit has been moved inside the shipped out box.
%
%    Now the setup inside the shipped out box; this should contain all
%    the stuff that could only affect typesetting; other stuff may need
%    to be reset for the writes also.
%
%    From here \ldots\ was in the command |\@shipoutsetup|.
%    \begin{macrocode}
  \if@specialpage
    \global\@specialpagefalse\@nameuse{ps@\@specialstyle}%
  \fi
  \if@twoside
    \ifodd\count\z@ \let\@thehead\@oddhead \let\@thefoot\@oddfoot
         \let\@themargin\oddsidemargin
    \else \let\@thehead\@evenhead
       \let\@thefoot\@evenfoot \let\@themargin\evensidemargin
    \fi
  \fi
%    \end{macrocode}
%
%    The rest was always inside the box.
%
%    RmS 91/08/15: aded this line:
%    \begin{macrocode}
  \reset@font
%    \end{macrocode}
%    RmS 93/08/06 Added |\lineskiplimit=0pt| to guard against it being
%              nonzero: e.g. by |\offinterlineskip| being in effect.
%
%    There are probably lots of other things that may need resetting.
%
%    \begin{macrocode}
  \normalsize
%    \end{macrocode}
% Reset the space factors.
% \changes{v1.2b}{1997/04/14}
%     {Call \cs{normalsfcodes} (from patch file) latex/2404}
%    \begin{macrocode}
  \normalsfcodes
%    \end{macrocode}
%
% Reset these here (previously reset separately for head and foot)
% \changes{v1.2b}{1997/04/14}
%     {Move \cs{label} and \cs{index} (from patch file)}
%    \begin{macrocode}
  \let\label\@gobble
  \let\index\@gobble
  \let\glossary\@gobble
%    \end{macrocode}
%
%    \begin{macrocode}
  \baselineskip\z@skip \lineskip\z@skip \lineskiplimit\z@
%    \end{macrocode}
%    \ldots\ to here was in the command |\@shipoutsetup|.
%    \begin{macrocode}
    \@begindvi
    \vskip \topmargin
    \moveright\@themargin \vbox {%
      \setbox\@tempboxa \vbox to\headheight{%
        \vfil
        \color@hbox
          \normalcolor
          \hb@xt@\textwidth{\@thehead}%
        \color@endbox
        }%                        %% 22 Feb 87
      \dp\@tempboxa \z@
      \box\@tempboxa
      \vskip \headsep
      \box\@outputbox
      \baselineskip \footskip
      \color@hbox
        \normalcolor
        \hb@xt@\textwidth{\@thefoot}%
      \color@endbox
      }%
    }%
%    \end{macrocode}
%   |\endgroup| now inserted by |\aftergroup|
%
% Restore |\if@newlist|
%    \begin{macrocode}
  \global\let\if@newlist\@@if@newlist
%    \end{macrocode}
%
%    \begin{macrocode}
  \global \@colht \textheight
  \stepcounter{page}%
%    \end{macrocode}
%    It is now clear that this does something useful, thanks to Piet
%    van Oostrum.  It is needed because a float page is made without
%    using TeX's page-builder; thus the output routine is never called
%    so the marks are not updated.
%    \begin{macrocode}
  \let\firstmark\botmark
}
%    \end{macrocode}
%  \end{macro}
%  \end{macro}
%  \end{macro}
%
% \begin{macro}{\@begindvi}
% \changes{v1.1c}{1994/11/05}
%         {Added macro}
% \changes{v1.1f}{1994/11/14}{Use normal box register: why a box?}
%
%    This unboxes stuff that must appear before anything else in the
%    |.dvi| file, then returns that box register to the free list and
%    cancels itself.
%
%    The stuff in the box should not add any typeset material to the
%    page.
%    \begin{macrocode}
\def \@begindvi{%
  \unvbox \@begindvibox
  \global\let \@begindvi \@empty
}
%    \end{macrocode}
% \end{macro}
%
% \begin{macro}{\@combinefloats}
% \begin{macro}{\@cflt}
% \changes{v1.0h}{1993/12/12}{name changed}
% \begin{macro}{\@cflb}
%    The |\boxmaxdepth| setting here was not made local to
%    a box so was dangerous.  It is needed only within the box made
%    by |\@cflt| (and not normally even there), so it has been
%    moved there; this also agrees with the original pseudocode.
%
% \changes{v1.0h}{1993/12/12}{boxmaxdepth setting moved}
%    \begin{macrocode}
\def \@combinefloats {%
%    \boxmaxdepth \maxdepth
    \ifx \@toplist\@empty \else \@cflt \fi
    \ifx \@botlist\@empty \else \@cflb \fi
}
%    \end{macrocode}
%
% \changes{v1.0h}{1993/12/12}{defs changed to lets}
%    \begin{macrocode}
\def \@cflt{%
    \let \@elt \@comflelt
    \setbox\@tempboxa \vbox{}%
    \@toplist
    \setbox\@outputbox \vbox{%
                             \boxmaxdepth \maxdepth
                             \unvbox\@tempboxa
                             \vskip -\floatsep
                             \topfigrule
                             \vskip \textfloatsep
                             \unvbox\@outputbox
                             }%
    \let\@elt\relax
    \xdef\@freelist{\@freelist\@toplist}%
    \global\let\@toplist\@empty
}
%    \end{macrocode}
%
%    \begin{macrocode}
\def \@cflb {%
    \let\@elt\@comflelt
    \setbox\@tempboxa \vbox{}%
    \@botlist
    \setbox\@outputbox \vbox{%
                             \unvbox\@outputbox
                             \vskip \textfloatsep
                             \botfigrule
                             \unvbox\@tempboxa
                             \vskip -\floatsep
                             }%
    \let\@elt\relax
    \xdef\@freelist{\@freelist\@botlist}%
    \global \let \@botlist\@empty
}
%    \end{macrocode}
%  \end{macro}
%  \end{macro}
%  \end{macro}
%
% \begin{macro}{\@comflelt}
% \begin{macro}{\@comdblflelt}
% \begin{macro}{\@combinedblfloats}
%
%    \begin{macrocode}
\def\@comflelt#1{\setbox\@tempboxa
      \vbox{\unvbox\@tempboxa\box #1\vskip\floatsep}}
%    \end{macrocode}
%
%    \begin{macrocode}
\def\@comdblflelt#1{\setbox\@tempboxa
      \vbox{\unvbox\@tempboxa\box #1\vskip\dblfloatsep}}
%    \end{macrocode}
%
%    \begin{macrocode}
\def \@combinedblfloats{%
  \ifx \@dbltoplist \@empty
  \else
    \setbox\@tempboxa \vbox{}%
    \let \@elt \@comdblflelt
    \@dbltoplist
    \let \@elt \relax
    \xdef \@freelist {\@freelist\@dbltoplist}%
    \global\let \@dbltoplist \@empty
    \setbox\@outputbox \vbox to\textheight
%    \end{macrocode}
%
%    The setting of |\boxmaxdepth| here has no effect since the
%    |\@outputbox| should already have depth zero.  Even so, it would
%    have no effect on the layout of the page.
% \changes{v1.0l}{1994/03/15}{Removed boxmaxdepth setting.}
%    \begin{macrocode}
      {%\boxmaxdepth\maxdepth   %% probably not needed, CAR
       \unvbox\@tempboxa\vskip-\dblfloatsep
%    \end{macrocode}
%    Here we need different typesetting if the top float comes from
%    |\@topnewpage|.
% \changes{v1.0n}{1994/04/30}{Removed rule in topnewpage case}
%    \begin{macrocode}
       \ifnum \@dbltopnum>\m@ne
         \dblfigrule
       \fi
       \vskip \dbltextfloatsep
       \box\@outputbox
       }%
  \fi
}
%</2ekernel>
%    \end{macrocode}
%  \end{macro}
%  \end{macro}
%  \end{macro}
%
%
%  \begin{macro}{\@startcolumn}
% \changes{v1.0f}{1993/12/05}{Command changed}
%  \begin{macro}{\@startdblcolumn}
% \changes{v1.0f}{1993/12/05}{Command changed}
%
%    We could combine (most of) these two into |\@startcol <list>|.
%    Note that |\@xstartcol| was only used once (\ie in
%    |\@startcolumn|); it has therefore been removed.  This is not quite
%    as efficient but it now has the same structure as
%    |\@startdblcolumn|.
%
%    The empty-list test has been moved to |\@tryfcolumn|.
%
%    \begin{macrocode}
%<*2ekernel|fltrace>
\def \@startcolumn {%
  \global \@colroom \@colht
  \@tryfcolumn \@deferlist
  \if@fcolmade
%<*trace>
    \fl@trace{PAGE: float \if@twocolumn column \else page \fi
                completed}%
%</trace>
  \else
%    \end{macrocode}
% \changes{v1.0h}{1993/12/12}{defs changed to lets}
%    \begin{macrocode}
    \begingroup
      \let \reserved@b \@deferlist
      \global \let \@deferlist \@empty
      \let \@elt \@scolelt
      \reserved@b
    \endgroup
  \fi
}
%    \end{macrocode}
%
%    This one does not need to set |\@colht|.
%
% \changes{v1.2n}{2015/01/14}{float order in 2-column (latexrelease)}
%    \begin{macrocode}
%</2ekernel|fltrace>
%<latexrelease|fltrace>\IncludeInRelease{2015/01/01}%
%<latexrelease|fltrace>  {\@startdblcolumn}{float order in 2-column}%
%<*2ekernel|latexrelease|fltrace>
\def \@startdblcolumn {%
  \@tryfcolumn \@deferlist
  \if@fcolmade
%<fltrace>    \fl@trace{PAGE: double float page completed}%
  \else
    \begingroup
      \let \reserved@b \@deferlist
      \global \let \@deferlist \@empty
      \let \@elt \@sdblcolelt
      \reserved@b
    \endgroup
  \fi
}%
%</2ekernel|latexrelease|fltrace>
%<latexrelease|fltrace>\EndIncludeInRelease
%<latexrelease|fltrace>\IncludeInRelease{0000/00/00}%
%<latexrelease|fltrace>  {\@startdblcolumn}{float order in 2-column}%
%<latexrelease|fltrace>\def \@startdblcolumn {%
%    \end{macrocode}
% Not needed since this always comes after |\@outputpage|:
%    \begin{macrocode}
%<latexrelease|fltrace>% \global \@colht \textheight
%<latexrelease|fltrace>  \@tryfcolumn \@dbldeferlist
%<latexrelease|fltrace>  \if@fcolmade
%<*trace>
%<latexrelease|fltrace>    \fl@trace{PAGE: double float page completed}%
%</trace>
%<latexrelease|fltrace>  \else
%    \end{macrocode}
% \changes{v1.0h}{1993/12/12}{defs changed to lets}
%    \begin{macrocode}
%<latexrelease|fltrace>    \begingroup
%<latexrelease|fltrace>      \let \reserved@b \@dbldeferlist
%<latexrelease|fltrace>      \global \let \@dbldeferlist \@empty
%<latexrelease|fltrace>      \let \@elt \@sdblcolelt
%<latexrelease|fltrace>      \reserved@b
%<latexrelease|fltrace>    \endgroup
%<latexrelease|fltrace>  \fi
%<latexrelease|fltrace>}%
%<latexrelease|fltrace>\EndIncludeInRelease
%<*2ekernel|fltrace>
%    \end{macrocode}
%  \end{macro}
%  \end{macro}
%
%  \begin{macro}{\@tryfcolumn}
% \changes{v1.0f}{1993/12/05}{Command changed}
%    Now tests if its list is empty before any further exertion.
%
%    \begin{macrocode}
\def \@tryfcolumn #1{%
  \global \@fcolmadefalse
  \ifx #1\@empty
  \else
%<*trace>
     \fl@trace{PAGE: try float \if@twocolumn column/page\else page\fi
                  ---\string #1}%
     \fl@trace{----- \string #1: #1}%
%</trace>
%    \end{macrocode}
% \changes{v1.0h}{1993/12/12}{defs changed to lets}
%    \begin{macrocode}
    \xdef\@trylist{#1}%
    \global \let \@failedlist \@empty
    \begingroup
      \let \@elt \@xtryfc \@trylist
    \endgroup
    \if@fcolmade
      \@vtryfc #1%
    \fi
  \fi
}
%</2ekernel|fltrace>
%    \end{macrocode}
%
%  \end{macro}
%
%    \begin{macrocode}
%<*2ekernel>
%    \end{macrocode}
%
% \begin{macro}{\@scolelt}
%    \begin{macrocode}
\def\@scolelt#1{\def\@currbox{#1}\@addtonextcol}
%    \end{macrocode}
% \end{macro}
%
% \begin{macro}{\@sdblcolelt}
%    \begin{macrocode}
\def\@sdblcolelt#1{\def\@currbox{#1}\@addtodblcol}
%    \end{macrocode}
% \end{macro}
%
% \begin{macro}{\@vtryfc}
% \changes{v1.2d}{1997/11/19}{Reindent code, to be understandable(DPC).}
%    \begin{macrocode}
\def\@vtryfc #1{%
  \global\setbox\@outputbox\vbox{}%
  \let\@elt\@wtryfc
  \@flsucceed
  \global\setbox\@outputbox \vbox to\@colht{%
    \vskip \@fptop
    \vskip -\@fpsep
    \unvbox \@outputbox
    \vskip \@fpbot}%
  \let\@elt\relax
  \xdef #1{\@failedlist\@flfail}%
  \xdef\@freelist{\@freelist\@flsucceed}}
%    \end{macrocode}
% \end{macro}
%
% \begin{macro}{\@wtryfc}
%    \begin{macrocode}
\def\@wtryfc #1{%
  \global\setbox\@outputbox\vbox{%
    \unvbox\@outputbox
    \vskip\@fpsep
    \box #1}}
%    \end{macrocode}
% \end{macro}
%
% \begin{macro}{\@xtryfc}
% \changes{v1.1v}{1996/07/26}{remove \cs{global} before \cs{@test...}}
% \changes{v1.2n}{2015/01/14}{float order in 2-column (latexrelease)}
%    \begin{macrocode}
%</2ekernel>
%<latexrelease>\IncludeInRelease{2015/01/01}{\@xtryfc}%
%<latexrelease>                             {float order in 2-column}%
%<*2ekernel|latexrelease>
\def\@xtryfc #1{%
  \@next\reserved@a\@trylist{}{}%
  \@currtype \count #1%
  \divide\@currtype\@xxxii
  \multiply\@currtype\@xxxii
  \@bitor \@currtype \@failedlist
  \@testfp #1%
%    \end{macrocode}
%
%    \begin{macrocode}
  \@testwrongwidth #1%
%    \end{macrocode}
%
%    \begin{macrocode}
  \ifdim \ht #1>\@colht
     \@testtrue
  \fi
  \if@test
    \@cons\@failedlist #1%
  \else
    \@ytryfc #1%
  \fi}%
%</2ekernel|latexrelease>
%<latexrelease>\EndIncludeInRelease
%<latexrelease>\IncludeInRelease{0000/00/00}{\@xtryfc}%
%<latexrelease>                             {float order in 2-column}%
%<latexrelease>\def\@xtryfc #1{%
%<latexrelease>  \@next\reserved@a\@trylist{}{}%
%<latexrelease>  \@currtype \count #1%
%<latexrelease>  \divide\@currtype\@xxxii
%<latexrelease>  \multiply\@currtype\@xxxii
%<latexrelease>  \@bitor \@currtype \@failedlist
%<latexrelease>  \@testfp #1%
%<latexrelease>  \ifdim \ht #1>\@colht
%<latexrelease>    \@testtrue
%<latexrelease>  \fi
%<latexrelease>  \if@test
%<latexrelease>    \@cons\@failedlist #1%
%<latexrelease>  \else
%<latexrelease>    \@ytryfc #1%
%<latexrelease>  \fi}%
%<latexrelease>\EndIncludeInRelease
%<*2ekernel>
%    \end{macrocode}
% \end{macro}
%
% \begin{macro}{\@ytryfc}
%    \begin{macrocode}
\def\@ytryfc #1{%
  \begingroup
    \gdef\@flsucceed{\@elt #1}%
    \global\let\@flfail\@empty
    \@tempdima\ht #1%
    \let\@elt\@ztryfc
    \@trylist
    \ifdim \@tempdima >\@fpmin
      \global\@fcolmadetrue
    \else
      \@cons\@failedlist #1%
    \fi
  \endgroup
  \if@fcolmade
    \let\@elt\@gobble
  \fi}
%    \end{macrocode}
% \end{macro}
%
% \begin{macro}{\@ztryfc}
% \changes{v1.1v}{1996/07/26}{remove \cs{global} before \cs{@test...}}
% \changes{v1.2n}{2015/01/14}{float order in 2-column (latexrelease)}
%    \begin{macrocode}
%</2ekernel>
%<latexrelease>\IncludeInRelease{2015/01/01}{@ztryfc}%
%<latexrelease>                             {float order in 2-column}%
%<*2ekernel|latexrelease>
\def\@ztryfc #1{%
  \@tempcnta\count #1%
  \divide\@tempcnta\@xxxii
  \multiply\@tempcnta\@xxxii
  \@bitor \@tempcnta {\@failedlist \@flfail}%
  \@testfp #1%
%    \end{macrocode}
%
% not in fixfloats?
%    \begin{macrocode}
  \@testwrongwidth #1%
%    \end{macrocode}
%
%    \begin{macrocode}
  \@tempdimb\@tempdima
  \advance\@tempdimb\ht #1%
  \advance\@tempdimb\@fpsep
  \ifdim \@tempdimb >\@colht
    \@testtrue
  \fi
  \if@test
    \@cons\@flfail #1%
  \else
    \@cons\@flsucceed #1%
    \@tempdima\@tempdimb
  \fi}%
%</2ekernel|latexrelease>
%<latexrelease>\EndIncludeInRelease
%<latexrelease>\IncludeInRelease{0000/00/00}{@ztryfc}%
%<latexrelease>                             {float order in 2-column}%
%<latexrelease>\def\@ztryfc #1{%
%<latexrelease>  \@tempcnta \count#1%
%<latexrelease>  \divide\@tempcnta\@xxxii
%<latexrelease>  \multiply\@tempcnta\@xxxii
%<latexrelease>  \@bitor \@tempcnta {\@failedlist \@flfail}%
%<latexrelease>  \@testfp #1%
%<latexrelease>  \@tempdimb\@tempdima
%<latexrelease>  \advance\@tempdimb \ht#1%
%<latexrelease>  \advance\@tempdimb\@fpsep
%<latexrelease>  \ifdim \@tempdimb >\@colht
%<latexrelease>    \@testtrue
%<latexrelease>  \fi
%<latexrelease>  \if@test
%<latexrelease>    \@cons\@flfail #1%
%<latexrelease>  \else
%<latexrelease>    \@cons\@flsucceed #1%
%<latexrelease>    \@tempdima\@tempdimb
%<latexrelease>  \fi}%
%<latexrelease>\EndIncludeInRelease
%    \end{macrocode}
% \end{macro}
%
%
% The major changes for float suppression and the changes to the float
% mechanism to make it conform to the documentation are in these next
% macros.
%
%  \begin{macro}{\@addtobot}
% \changes{v1.0f}{1993/12/05}{Command changed}
%    Lots of changes.
%
%    \begin{macrocode}
%<*2ekernel|fltrace>
\def \@addtobot {%
%<*trace>
   \fl@trace{***Start addtobot}%
%</trace>
   \@getfpsbit 4\relax
%<*trace>
   \fl@trace{fpstype \ifodd \@tempcnta OK \else not \fi bot:
                                                     \the \@fpstype}%
%</trace>
   \ifodd \@tempcnta
     \@flsetnum \@botnum
     \ifnum \@botnum>\z@
       \@tempswafalse
       \@flcheckspace \@botroom \@botlist
       \if@tempswa
%    \end{macrocode}
%    This next line means that this page is produced with box 255
%    having depth zero, rather than the normal maxdepth: is this
%    needed, useful?
% \task{CAR/FMi}{Investigate resetting of maxdepth: I do not think it is
% necessary here; Frank does.}
%    \begin{macrocode}
         \global \maxdepth \z@
         \@flupdates \@botnum \@botroom \@botlist
%<*trace>
         \fl@trace{colroom (after-bot) = \the \@colroom}%
         \fl@trace{colnum (after-bot) = \the \@colnum}%
         \fl@trace{botnum (after-bot) = \the \@botnum}%
         \fl@trace{***Success: bot}%
%</trace>
         \@inserttrue
       \fi
%<*trace>
     \else
       \fl@trace{Fail: botnum = \the \@botnum:
                                  fpstype \the \@fpstype=ORD?}%
       \ifnum \@fpstype<\sixt@@n
         \fl@trace{ERROR: !b float not successful (addtobot)}%
       \fi
%</trace>
     \fi
   \fi
}
%    \end{macrocode}
%  \end{macro}
%
%  \begin{macro}{\@addtotoporbot}
% \changes{v1.0f}{1993/12/05}{Command changed}
%    Lots of changes.
%
%    \begin{macrocode}
\def \@addtotoporbot {%
%<*trace>
   \fl@trace{***Start addtotoporbot}%
%</trace>
   \@getfpsbit \tw@
%<*trace>
   \fl@trace{fpstype \ifodd \@tempcnta OK \else not \fi top:
                                                     \the \@fpstype}%
%</trace>
   \ifodd \@tempcnta
     \@flsetnum \@topnum
     \ifnum \@topnum>\z@
       \@tempswafalse
       \@flcheckspace \@toproom \@toplist
       \if@tempswa
         \@bitor\@currtype{\@midlist\@botlist}%
%<*trace>
           \fl@trace{(mid+bot)list: \@midlist, \@botlist:
                              (addtotoporbot-before)}%
%</trace>
         \if@test
%<*trace>
         \fl@trace{type already on list: mid or bot---sent to addtobot}%
%</trace>
         \else
          \@flupdates \@topnum \@toproom \@toplist
%<*trace>
          \fl@trace{colroom (after-top) = \the \@colroom}%
          \fl@trace{colnum (after-top) = \the \@colnum}%
          \fl@trace{topnum (after-top) = \the \@topnum}%
          \fl@trace{***Success: top}%
%</trace>
          \@inserttrue
         \fi
       \fi
%<*trace>
     \else
       \fl@trace{Fail: topnum = \the \@topnum: fpstype
                                            \the \@fpstype=ORD?}%
       \ifnum \@fpstype<\sixt@@n
         \fl@trace{ERROR: !t float not successful (addtotoporbot)}%
       \fi
%</trace>
     \fi
   \fi
   \if@insert
   \else
%<*trace>
     \fl@trace{sent to addtobot (addtotoporbot)}%
%</trace>
     \@addtobot
   \fi
}
%</2ekernel|fltrace>
%    \end{macrocode}
%  \end{macro}
%
%  \begin{macro}{\@addtocurcol}
% \changes{v1.0f}{1993/12/05}{Command changed}
% \task{CAR}{Add rules around h floats for FMi}
% \task{CAR}{Investigate pagebreak option possibilities}
%    Lots of changes.
%
% \changes{v1.2n}{2015/01/14}{float order in 2-column (latexrelease)}
%    \begin{macrocode}
%<latexrelease|fltrace|flafter>\IncludeInRelease{2015/01/01}%
%<latexrelease|fltrace|flafter>  {\@addtocurcol}{float order in 2-column}%
%<*2ekernel|latexrelease|fltrace|flafter>
\def \@addtocurcol {%
%<*trace>
  \fl@trace{***Start addtocurcol}%
%</trace>
   \@insertfalse
   \@setfloattypecounts
   \ifnum \@fpstype=8
%<*trace>
     \fl@trace{fpstype !p only (addtocurcol): \the \@fpstype = 8?}%
%</trace>
   \else
     \ifnum \@fpstype=24
%<*trace>
       \fl@trace{fpstype p only (addtocurcol): \the \@fpstype = 24?}%
%</trace>
     \else
       \@flsettextmin
%    \end{macrocode}
% This is a new adjustment which is quite a major change in
% functionality; but it implements the documentation.
% Note that |\@reqcolroom| will include the whole of the
% page-so-far, and hence includes |\@textfloatsheight| of floats,
% so before comparing it with |\@textmin|, we add this to
% |\@textmin| also.
%    \begin{macrocode}
%<*trace>
       \fl@trace{textfloatsheight (before) = \the \@textfloatsheight}%
%</trace>
       \advance \@textmin \@textfloatsheight
       \@reqcolroom \@pageht
%    \end{macrocode}
% This line must be removed since |\@specialoutput| changed.
%    \begin{macrocode}
%       \advance \@reqcolroom \@pagedp
%<*trace>
       \fl@trace{textmin + textfloatsheight: \the \@textmin}%
       \fl@trace{page-so-far: \the \@reqcolroom}%
%</trace>
       \ifdim \@textmin>\@reqcolroom
         \@reqcolroom \@textmin
%<*trace>
         \fl@trace{ORD? textmin being used}%
%</trace>
       \fi
       \advance \@reqcolroom \ht\@currbox
%<*trace>
       \fl@trace{float size = \the \ht \@currbox (addtocurcol)}%
       \fl@trace{colroom = \the \@colroom (addtocurcol)}%
       \fl@trace{reqcolroom = \the \@reqcolroom (addtocurcol)}%
%</trace>
       \ifdim \@colroom>\@reqcolroom
         \@flsetnum \@colnum
         \ifnum \@colnum>\z@
           \@bitor\@currtype\@deferlist
%    \end{macrocode}
%    We need to defer the float also if its width
%    doesn't fit.
% \changes{v1.2n}{2000/09/24}{FMi: test for wide float was
%    in wrong place}
%    \begin{macrocode}
          \@testwrongwidth\@currbox
%    \end{macrocode}
%
%    \begin{macrocode}
%<*trace>
           \fl@trace{deferlist: \@deferlist: (addtocurcol-before)}%
%</trace>
           \if@test
%<*trace>
             \fl@trace{type already on list: defer (addtocurcol)}%
%</trace>
           \else
             \@bitor\@currtype\@botlist
%<*trace>
           \fl@trace{botlist: \@botlist: (addtocurcol-before)}%
%</trace>
             \if@test
%<*trace>
               \fl@trace{type already on list: bot---sent to addtobot}%
%</trace>
               \@addtobot
             \else
%<*trace>
               \fl@trace{fpstype \ifodd \@tempcnta OK \else not \fi
                      here: \the \@fpstype}%
%</trace>
               \ifodd \count\@currbox
                 \advance \@reqcolroom \intextsep
                 \ifdim \@colroom>\@reqcolroom
                   \global \advance \@colnum \m@ne
                   \global \advance \@textfloatsheight \ht\@currbox
%    \end{macrocode}
% This may sometimes give an overestimate.
%    \begin{macrocode}
                   \global \advance \@textfloatsheight 2\intextsep
                   \@cons \@midlist \@currbox
%<*trace>
                 \fl@trace{***Success: here}%
                 \fl@trace{textfloatsheight (after-here) =
                      \the \@textfloatsheight}%
                 \fl@trace{colnum (after-here) = \the \@colnum}%
%</trace>
%    \end{macrocode}
%
% CHANGE TO |\@addtocurcol|:
%
% |\penalty\z@| changed to |\penalty\interlinepenalty| so |\samepage|
% works properly with figure and table environments.
% (Changed 23 Oct 86)
%
% There is also an |\addpenalty\interlinepenalty| above.
%
% Since in 2e |\samepage| is no longer supported, these could be
% removed.
%
% Although it is best to use |\addvspace| in case two h floats come
% together, this makes other spacing more difficult to adjust; whereas
% if a user specifies two h floats together then they can more easily
% get the spacing correct by ad hoc commands.
%
% It is necessary to adjust for the addition of |\parskip| here in
% case the float is added between paragraphs (\ie when in vertical
% mode).
%
% If the nobreak switch is true we need to reset it and clear
% |\everypar| since the float may not reset the flag and cannot reset
% the |\everypar| globally.
% \changes{v1.0l}{1994/03/15}{Changed \cs{addvspace} to \cs{vskip}}
% \changes{v1.1i}{1994/11/21}
%   {Added \cs{if@nobreak} test before float box}
% \changes{v1.1z}{1996/10/24}{Added \cs{nobreak}, etc as appropriate}
%
% Typesetting starts here (we are in vertical mode).
%    \begin{macrocode}
                   \if@nobreak
                     \nobreak
                     \@nobreakfalse
                     \everypar{}%
                   \else
                     \addpenalty \interlinepenalty
                   \fi
                   \vskip \intextsep
                   \box\@currbox
                   \penalty\interlinepenalty
                   \vskip\intextsep
                   \ifnum\outputpenalty <-\@Mii \vskip -\parskip\fi
%    \end{macrocode}
% Typesetting ends here.
%    \begin{macrocode}
                   \outputpenalty \z@
                   \@inserttrue
%<*trace>
                 \else
                   \fl@trace{Fail---no room at 2nd test of colroom
                                 (addtocorcol \string\intextsep)}%
%</trace>
                 \fi
               \fi
               \if@insert
               \else
%    \end{macrocode}
%   Next set of docstrip guards are a bit weird, essentially
%   |\@addtotoporbot| ends up inside the kernel and the
%   \texttt{fltrace} package and |\@addtobot| shows up in the
%   \texttt{flafter} package. Guess that could have been done a bit
%   more obvious :-)
%    \begin{macrocode}
%<*2ekernel|fltrace|latexrelease>
%<*trace>
                 \fl@trace{not here: sent to addtotoporbot}%
%</trace>
                 \@addtotoporbot
%</2ekernel|fltrace|latexrelease>
%<*!2ekernel&!fltrace&!latexrelease>
%<*trace>
                 \fl@trace{not here: sent to addtobot}%
%</trace>
                 \@addtobot
%</!2ekernel&!fltrace&!latexrelease>
               \fi
             \fi
           \fi
%<*trace>
         \else
           \fl@trace{Fail: colnum = \the \@colnum:
                        fpstype \the \@fpstype=ORD?}%
           \ifnum \@fpstype<\sixt@@n
             \fl@trace{ERROR: BANG float not successful (addtocurcol)}%
           \fi
%</trace>
         \fi
%<*trace>
       \else
         \fl@trace{Fail---no room: fl box ht: \the \ht \@currbox
                                                     (addtocurcol)}%
%</trace>
       \fi
     \fi
   \fi
   \if@insert
   \else
     \@resethfps
%<*trace>
     \fl@trace{put on deferlist (addtocurcol)}%
%</trace>
     \@cons\@deferlist\@currbox
%<*trace>
     \fl@trace{deferlist: \@deferlist: (addtocurcol-after)}%
%</trace>
   \fi
}%
%</2ekernel|latexrelease|fltrace|flafter>
%<latexrelease|fltrace|flafter>\EndIncludeInRelease
%<latexrelease|fltrace|flafter>\IncludeInRelease{0000/00/00}%
%<latexrelease|fltrace|flafter>  {\@addtocurcol}{float order in 2-column}%
%<latexrelease|fltrace|flafter>\def \@addtocurcol {%
%<*trace>
%<latexrelease|fltrace|flafter>  \fl@trace{***Start addtocurcol}%
%</trace>
%<latexrelease|fltrace|flafter>   \@insertfalse
%<latexrelease|fltrace|flafter>   \@setfloattypecounts
%<latexrelease|fltrace|flafter>   \ifnum \@fpstype=8
%<*trace>
%<latexrelease|fltrace|flafter>     \fl@trace{fpstype !p only (addtocurcol):
%<latexrelease|fltrace|flafter>                            \the \@fpstype = 8?}%
%</trace>
%<latexrelease|fltrace|flafter>   \else
%<latexrelease|fltrace|flafter>     \ifnum \@fpstype=24
%<*trace>
%<latexrelease|fltrace|flafter>      \fl@trace{fpstype p only (addtocurcol):
%<latexrelease|fltrace|flafter>                           \the \@fpstype = 24?}%
%</trace>
%<latexrelease|fltrace|flafter>     \else
%<latexrelease|fltrace|flafter>       \@flsettextmin
%    \end{macrocode}
% This is a new adjustment which is quite a major change in
% functionality; but it implements the documentation.
% Note that |\@reqcolroom| will include the whole of the
% page-so-far, and hence includes |\@textfloatsheight| of floats,
% so before comparing it with |\@textmin|, we add this to
% |\@textmin| also.
%    \begin{macrocode}
%<*trace>
%<latexrelease|fltrace|flafter>       \fl@trace{textfloatsheight (before) = 
%<latexrelease|fltrace|flafter>                        \the \@textfloatsheight}%
%</trace>
%<latexrelease|fltrace|flafter>       \advance \@textmin \@textfloatsheight
%<latexrelease|fltrace|flafter>       \@reqcolroom \@pageht
%    \end{macrocode}
% This line must be removed since |\@specialoutput| changed.
%    \begin{macrocode}
%       \advance \@reqcolroom \@pagedp
%<*trace>
%<latexrelease|fltrace|flafter>       \fl@trace{textmin + textfloatsheight:
%<latexrelease|fltrace|flafter>                                 \the \@textmin}%
%<latexrelease|fltrace|flafter>       \fl@trace{page-so-far: \the \@reqcolroom}%
%<latexrelease|fltrace|flafter>
%</trace>
%<latexrelease|fltrace|flafter>       \ifdim \@textmin>\@reqcolroom
%<latexrelease|fltrace|flafter>         \@reqcolroom \@textmin
%<*trace>
%<latexrelease|fltrace|flafter>         \fl@trace{ORD? textmin being used}%
%</trace>
%<latexrelease|fltrace|flafter>       \fi
%<latexrelease|fltrace|flafter>       \advance \@reqcolroom \ht\@currbox
%<*trace>
%<latexrelease|fltrace|flafter>       \fl@trace{float size = 
%<latexrelease|fltrace|flafter>               \the \ht \@currbox (addtocurcol)}%
%<latexrelease|fltrace|flafter>       \fl@trace{colroom =
%<latexrelease|fltrace|flafter>                   \the \@colroom (addtocurcol)}%
%<latexrelease|fltrace|flafter>       \fl@trace{reqcolroom =
%<latexrelease|fltrace|flafter>                \the \@reqcolroom (addtocurcol)}%
%</trace>
%<latexrelease|fltrace|flafter>       \ifdim \@colroom>\@reqcolroom
%<latexrelease|fltrace|flafter>         \@flsetnum \@colnum
%<latexrelease|fltrace|flafter>         \ifnum \@colnum>\z@
%<latexrelease|fltrace|flafter>           \@bitor\@currtype\@deferlist
%<*trace>
%<latexrelease|fltrace|flafter>           \fl@trace{deferlist:
%<latexrelease|fltrace|flafter>              \@deferlist: (addtocurcol-before)}%
%</trace>
%<latexrelease|fltrace|flafter>           \if@test
%<*trace>
%<latexrelease|fltrace|flafter>             \fl@trace{type already on list:
%<latexrelease|fltrace|flafter>                            defer (addtocurcol)}%
%</trace>
%<latexrelease|fltrace|flafter>           \else
%<latexrelease|fltrace|flafter>             \@bitor\@currtype\@botlist
%<*trace>
%<latexrelease|fltrace|flafter>           \fl@trace{botlist: \@botlist:
%<latexrelease|fltrace|flafter>                           (addtocurcol-before)}%
%</trace>
%<latexrelease|fltrace|flafter>             \if@test
%<*trace>
%<latexrelease|fltrace|flafter>               \fl@trace{type already on list:
%<latexrelease|fltrace|flafter>                         bot---sent to addtobot}%
%</trace>
%<latexrelease|fltrace|flafter>               \@addtobot
%<latexrelease|fltrace|flafter>             \else
%<*trace>
%<latexrelease|fltrace|flafter>               \fl@trace{fpstype
%<latexrelease|fltrace|flafter>               \ifodd \@tempcnta OK \else not \fi
%<latexrelease|fltrace|flafter>               here: \the \@fpstype}%
%</trace>
%<latexrelease|fltrace|flafter>               \ifodd \count\@currbox
%<latexrelease|fltrace|flafter>                 \advance \@reqcolroom \intextsep
%<latexrelease|fltrace|flafter>                 \ifdim \@colroom>\@reqcolroom
%<latexrelease|fltrace|flafter>                  \global \advance \@colnum \m@ne
%<latexrelease|fltrace|flafter>                  \global \advance
%<latexrelease|fltrace|flafter>                   \@textfloatsheight\ht\@currbox
%    \end{macrocode}
% This may sometimes give an overestimate.
%    \begin{macrocode}
%<latexrelease|fltrace|flafter>                  \global \advance
%<latexrelease|fltrace|flafter>                   \@textfloatsheight 2\intextsep
%<latexrelease|fltrace|flafter>                   \@cons \@midlist \@currbox
%<*trace>
%<latexrelease|fltrace|flafter>                 \fl@trace{***Success: here}%
%<latexrelease|fltrace|flafter>                 \fl@trace{textfloatsheight
%<latexrelease|fltrace|flafter>                      (after-here) =
%<latexrelease|fltrace|flafter>                      \the \@textfloatsheight}%
%<latexrelease|fltrace|flafter>                 \fl@trace{colnum (after-here) =
%<latexrelease|fltrace|flafter>                          \the \@colnum}%
%</trace>
%    \end{macrocode}
%
% CHANGE TO |\@addtocurcol|:
%
% |\penalty\z@| changed to |\penalty\interlinepenalty| so |\samepage|
% works properly with figure and table environments.
% (Changed 23 Oct 86)
%
% There is also an |\addpenalty\interlinepenalty| above.
%
% Since in 2e |\samepage| is no longer supported, these could be
% removed.
%
% Although it is best to use |\addvspace| in case two h floats come
% together, this makes other spacing more difficult to adjust; whereas
% if a user specifies two h floats together then they can more easily
% get the spacing correct by ad hoc commands.
%
% It is necessary to adjust for the addition of |\parskip| here in
% case the float is added between paragraphs (\ie when in vertical
% mode).
%
% If the nobreak switch is true we need to reset it and clear
% |\everypar| since the float may not reset the flag and cannot reset
% the |\everypar| globally.
% \changes{v1.0l}{1994/03/15}{Changed \cs{addvspace} to \cs{vskip}}
% \changes{v1.1i}{1994/11/21}
%   {Added \cs{if@nobreak} test before float box}
% \changes{v1.1z}{1996/10/24}{Added \cs{nobreak}, etc as appropriate}
%
% Typesetting starts here (we are in vertical mode).
%    \begin{macrocode}
%<latexrelease|fltrace|flafter>                   \if@nobreak
%<latexrelease|fltrace|flafter>                     \nobreak
%<latexrelease|fltrace|flafter>                     \@nobreakfalse
%<latexrelease|fltrace|flafter>                     \everypar{}%
%<latexrelease|fltrace|flafter>                   \else
%<latexrelease|fltrace|flafter>                     \addpenalty\interlinepenalty
%<latexrelease|fltrace|flafter>                   \fi
%<latexrelease|fltrace|flafter>                   \vskip \intextsep
%<latexrelease|fltrace|flafter>                   \box\@currbox
%<latexrelease|fltrace|flafter>                   \penalty\interlinepenalty
%<latexrelease|fltrace|flafter>                   \vskip\intextsep
%<latexrelease|fltrace|flafter>                   \ifnum\outputpenalty 
%<latexrelease|fltrace|flafter>                               <-\@Mii \vskip
%<latexrelease|fltrace|flafter>                        -\parskip\fi
%    \end{macrocode}
% Typesetting ends here.
%    \begin{macrocode}
%<latexrelease|fltrace|flafter>                   \outputpenalty \z@
%<latexrelease|fltrace|flafter>                   \@inserttrue
%<*trace>
%<latexrelease|fltrace|flafter>                 \else
%<latexrelease|fltrace|flafter>  \fl@trace{Fail---no room at 2nd test of colroom
%<latexrelease|fltrace|flafter>                (addtocorcol \string\intextsep)}%
%</trace>
%<latexrelease|fltrace|flafter>                 \fi
%<latexrelease|fltrace|flafter>               \fi
%<latexrelease|fltrace|flafter>               \if@insert
%<latexrelease|fltrace|flafter>               \else
%    \end{macrocode}
%   Next set of docstrip guards are a bit weird, essentially
%   |\@addtotoporbot| ends up inside the kernel and the
%   \texttt{fltrace} package and |\@addtotoporbot| shows up in the
%   \texttt{flafter} package. Guess that could have been done a bit
%   more obvious :-)
%    \begin{macrocode}
%<*2ekernel|fltrace>
%<*trace>
%<latexrelease|fltrace|flafter>      \fl@trace{not here: sent to addtotoporbot}%
%</trace>
%<latexrelease|fltrace|flafter>                 \@addtotoporbot
%</2ekernel|fltrace>
%<*!2ekernel&!autoload&!fltrace>
%<*trace>
%<latexrelease|fltrace|flafter>       \fl@trace{not here: sent to addtobot}%
%</trace>
%<latexrelease|fltrace|flafter>                 \@addtobot
%</!2ekernel&!autoload&!fltrace>
%<latexrelease|fltrace|flafter>               \fi
%<latexrelease|fltrace|flafter>             \fi
%<latexrelease|fltrace|flafter>           \fi
%<*trace>
%<latexrelease|fltrace|flafter>         \else
%<latexrelease|fltrace|flafter>         \fl@trace{Fail: colnum = \the \@colnum:
%<latexrelease|fltrace|flafter>                    fpstype \the \@fpstype=ORD?}%
%<latexrelease|fltrace|flafter>         \ifnum \@fpstype<\sixt@@n
%<latexrelease|fltrace|flafter>  \fl@trace{ERROR: BANG float not successful 
%<latexrelease|fltrace|flafter>                                  (addtocurcol)}%
%<latexrelease|fltrace|flafter>           \fi
%</trace>
%<latexrelease|fltrace|flafter>         \fi
%<*trace>
%<latexrelease|fltrace|flafter>       \else
%<latexrelease|fltrace|flafter>         \fl@trace{Fail---no room: fl box ht: 
%<latexrelease|fltrace|flafter>               \the \ht \@currbox (addtocurcol)}%
%</trace>
%<latexrelease|fltrace|flafter>       \fi
%<latexrelease|fltrace|flafter>     \fi
%<latexrelease|fltrace|flafter>   \fi
%<latexrelease|fltrace|flafter>   \if@insert
%<latexrelease|fltrace|flafter>   \else
%<latexrelease|fltrace|flafter>     \@resethfps
%<*trace>
%<latexrelease|fltrace|flafter>     \fl@trace{put on deferlist (addtocurcol)}%
%</trace>
%<latexrelease|fltrace|flafter>     \@cons\@deferlist\@currbox
%<*trace>
%<latexrelease|fltrace|flafter>     \fl@trace{deferlist: \@deferlist:
%<latexrelease|fltrace|flafter>                          (addtocurcol-after)}%
%</trace>
%<latexrelease|fltrace|flafter>   \fi
%<latexrelease|fltrace|flafter>  }%
%<latexrelease|fltrace|flafter>\EndIncludeInRelease
%    \end{macrocode}
%  \end{macro}
%
%  \begin{macro}{\@addtonextcol}
% \changes{v1.0f}{1993/12/05}{Command changed}
%    Lots of changes.
%
% \changes{v1.2n}{2015/01/14}{float order in 2-column (latexrelease)}
%    \begin{macrocode}
%<latexrelease|fltrace>\IncludeInRelease{2015/01/01}
%<latexrelease|fltrace>  {\@addtonextcol}{float order in 2-column}%
%<*2ekernel|fltrace>
\def\@addtonextcol{%
  \begingroup
%<*trace>
   \fl@trace{***Start addtonextcol}%
%</trace>
   \@insertfalse
   \@setfloattypecounts
   \ifnum \@fpstype=8
%<*trace>
     \fl@trace{fpstype not curcol: \the \@fpstype = 8?}%
%</trace>
   \else
     \ifnum \@fpstype=24
%<*trace>
       \fl@trace{fpstype not curcol: \the \@fpstype = 24?}%
%</trace>
     \else
       \@flsettextmin
%<*trace>
       \fl@trace{text-so-far: 0pt (top of col)}%
%</trace>
       \@reqcolroom \ht\@currbox
%<*trace>
       \fl@trace{float size: \the \@reqcolroom (addtonextcol)}%
%</trace>
       \advance \@reqcolroom \@textmin
%<*trace>
       \fl@trace{colroom = \the \@colroom (addtonextcol)}%
       \fl@trace{reqcolroom = \the \@reqcolroom (addtonextcol)}%
%</trace>
       \ifdim \@colroom>\@reqcolroom
         \@flsetnum \@colnum
         \ifnum\@colnum>\z@
            \@bitor\@currtype\@deferlist
%<*trace>
            \fl@trace{deferlist: \@deferlist: (addtonextcol-before)}%
%</trace>
%    \end{macrocode}
%
%    \begin{macrocode}
            \@testwrongwidth\@currbox
%    \end{macrocode}
%
%    \begin{macrocode}
            \if@test
%<*trace>
              \fl@trace{type already on list: defer (addtonextcol)}%
%</trace>
            \else
%<*trace>
              \fl@trace{sent to addtotoporbot (addtonextcol)}%
%</trace>
              \@addtotoporbot
            \fi
         \fi
%<*trace>
       \else
         \fl@trace{Fail---no room: fl box ht: \the \ht \@currbox
                                                  (addtonextcol)}%
%</trace>
       \fi
     \fi
   \fi
   \if@insert
   \else
%<*trace>
     \fl@trace{put back on deferlist (addtonextcol)}%
%</trace>
     \@cons\@deferlist\@currbox
%<*trace>
     \fl@trace{deferlist: \@deferlist: (addtonextcol-after)}%
%</trace>
   \fi
%<*trace>
  \fl@trace{End of addtonextcol -- locally counts:}%
  \fl@trace{col: \the\@colnum. top: \the \@topnum. bot: \the \@botnum.}%
%</trace>
  \endgroup
%<*trace>
 \fl@trace{End of addtonextcol -- globally counts:}%
 \fl@trace{col: \the\@colnum. top: \the \@topnum. bot: \the \@botnum.}%
%</trace>
}%
%</2ekernel|fltrace>
%<latexrelease|fltrace>\EndIncludeInRelease
%<latexrelease|fltrace>\IncludeInRelease{0000/00/00}%
%<latexrelease|fltrace>  {\@addtonextcol}{float order in 2-column}%
%<latexrelease|fltrace>\def\@addtonextcol{%
%<latexrelease|fltrace>  \begingroup
%<*trace>
%<latexrelease|fltrace>   \fl@trace{***Start addtonextcol}%
%</trace>
%<latexrelease|fltrace>   \@insertfalse
%<latexrelease|fltrace>   \@setfloattypecounts
%<latexrelease|fltrace>   \ifnum \@fpstype=8
%<*trace>
%<latexrelease|fltrace>     \fl@trace{fpstype not curcol:
%<latexrelease|fltrace>                    \the \@fpstype = 8?}%
%</trace>
%<latexrelease|fltrace>   \else
%<latexrelease|fltrace>     \ifnum \@fpstype=24
%<*trace>
%<latexrelease|fltrace>       \fl@trace{fpstype not curcol:
%<latexrelease|fltrace>                       \the \@fpstype = 24?}%
%</trace>
%<latexrelease|fltrace>     \else
%<latexrelease|fltrace>       \@flsettextmin
%<*trace>
%<latexrelease|fltrace>       \fl@trace{text-so-far: 0pt (top of col)}%
%</trace>
%<latexrelease|fltrace>       \@reqcolroom \ht\@currbox
%<*trace>
%<latexrelease|fltrace>       \fl@trace{float size:
%<latexrelease|fltrace>               \the \@reqcolroom (addtonextcol)}%
%<latexrelease|fltrace>
%</trace>
%<latexrelease|fltrace>       \advance \@reqcolroom \@textmin
%<*trace>
%<latexrelease|fltrace>       \fl@trace{colroom =
%<latexrelease|fltrace>                  \the \@colroom (addtonextcol)}%
%<latexrelease|fltrace>       \fl@trace{reqcolroom =
%<latexrelease|fltrace>               \the \@reqcolroom (addtonextcol)}%
%</trace>
%<latexrelease|fltrace>       \ifdim \@colroom>\@reqcolroom
%<latexrelease|fltrace>         \@flsetnum \@colnum
%<latexrelease|fltrace>         \ifnum\@colnum>\z@
%<latexrelease|fltrace>            \@bitor\@currtype\@deferlist
%<*trace>
%<latexrelease|fltrace>            \fl@trace{deferlist: \@deferlist:
%<latexrelease|fltrace>                          (addtonextcol-before)}%
%</trace>
%<latexrelease|fltrace>            \if@test
%<*trace>
%<latexrelease|fltrace>              \fl@trace{type already on list:
%<latexrelease|fltrace>                           defer (addtonextcol)}%
%</trace>
%<latexrelease|fltrace>            \else
%<*trace>
%<latexrelease|fltrace>              \fl@trace{sent to addtotoporbot
%<latexrelease|fltrace>                                 (addtonextcol)}%
%</trace>
%<latexrelease|fltrace>              \@addtotoporbot
%<latexrelease|fltrace>            \fi
%<latexrelease|fltrace>         \fi
%<*trace>
%<latexrelease|fltrace>       \else
%<latexrelease|fltrace>         \fl@trace{Fail---no room: fl box ht:
%<latexrelease|fltrace>              \the \ht \@currbox (addtonextcol)}%
%</trace>
%<latexrelease|fltrace>       \fi
%<latexrelease|fltrace>     \fi
%<latexrelease|fltrace>   \fi
%<latexrelease|fltrace>   \if@insert
%<latexrelease|fltrace>   \else
%<*trace>
%<latexrelease|fltrace>     \fl@trace{put back on deferlist
%<latexrelease|fltrace>                                (addtonextcol)}%
%</trace>
%<latexrelease|fltrace>     \@cons\@deferlist\@currbox
%<*trace>
%<latexrelease|fltrace>     \fl@trace{deferlist: \@deferlist:
%<latexrelease|fltrace>                           (addtonextcol-after)}%
%</trace>
%<latexrelease|fltrace>   \fi
%<*trace>
%<latexrelease|fltrace>   \fl@trace{End of addtonextcol --
%<latexrelease|fltrace>                             locally counts:}%
%<latexrelease|fltrace>   \fl@trace{col: \the \@colnum. 
%<latexrelease|fltrace>      top: \the \@topnum. bot: \the \@botnum.}%
%</trace>
%<latexrelease|fltrace>  \endgroup
%<*trace>
%<latexrelease|fltrace>  \fl@trace{End of addtonextcol --
%<latexrelease|fltrace>                           globally counts:}%
%<latexrelease|fltrace>  \fl@trace{col: \the \@colnum.
%<latexrelease|fltrace>        top: \the \@topnum. bot: \the \@botnum.}%
%</trace>
%<latexrelease|fltrace>}%
%<latexrelease|fltrace>\EndIncludeInRelease
%    \end{macrocode}
%
%  \end{macro}
%
%  \begin{macro}{\@addtodblcol}
% \changes{v1.0f}{1993/12/05}{Command changed}
%    Lots of changes.
%
% \changes{v1.2n}{2015/01/14}{float order in 2-column (latexrelease)}
%    \begin{macrocode}
%<latexrelease|fltrace>\IncludeInRelease{2015/01/01}%
%<latexrelease|fltrace>  {\@addtodblcol}{float order in 2-column}%
%<*2ekernel|latexrelease|fltrace>
\def\@addtodblcol{%
  \begingroup
%<*trace>
  \fl@trace{***Start addtodblcol}%
%</trace>
   \@insertfalse
   \@setfloattypecounts
   \@getfpsbit \tw@
%<*trace>
   \fl@trace{fpstype \ifodd \@tempcnta OK \else not \fi dbltop:
                                                     \the \@fpstype}%
%</trace>
   \ifodd\@tempcnta
     \@flsetnum \@dbltopnum
     \ifnum \@dbltopnum>\z@
       \@tempswafalse
       \ifdim \@dbltoproom>\ht\@currbox
         \@tempswatrue
%<*trace>
         \fl@trace{Space OK: \@dbltoproom =
                \the \@dbltoproom > \the \ht \@currbox
                                         (dbltoproom)}%
%</trace>
       \else
%<*trace>
         \fl@trace{fpstype: \the \@fpstype (addtodblcol)}%
%</trace>
         \ifnum \@fpstype<\sixt@@n
%<*trace>
           \fl@trace{BANG float ignoring \@dbltoproom}%
           \fl@trace{\@spaces \@dbltoproom = \the \@dbltoproom.
                           Ht float: \the \ht \@currbox-BANG}%
%</trace>
%    \end{macrocode}
% Need to check that there is room on the page, using the local value
% of |\@textmin| to make the necessary adjustment to |\@dbltoproom|.
%    \begin{macrocode}
           \advance \@dbltoproom \@textmin
%<*trace>
           \fl@trace{Local value of texmin: \the\@textmin}%
           \fl@trace{\@spaces space on page = \the \@dbltoproom.
                           Ht float: \the \ht \@currbox-BANG}%
%</trace>
           \ifdim \@dbltoproom>\ht\@currbox
             \@tempswatrue
%<*trace>
             \fl@trace{Space OK BANG: space on page =
                         \the \@dbltoproom > \the \ht \@currbox}%
           \else
             \fl@trace{fpstype: \the \@fpstype}%
             \fl@trace{Fail---no room dbltoproom-BANG?:}%
             \fl@trace{\@spaces space on page = \the \@dbltoproom.
                           Ht float: \the \ht \@currbox}%
%</trace>
           \fi
           \advance \@dbltoproom -\@textmin
%<*trace>
         \else
           \fl@trace{fpstype: \the \@fpstype}%
           \fl@trace{Fail---no room dbltoproom-ORD?:}%
           \fl@trace{\@spaces \@dbltoproom = \the \@dbltoproom.
                           Ht float: \the \ht \@currbox}%
%</trace>
         \fi
       \fi
       \if@tempswa
           \@bitor \@currtype \@deferlist
%<*trace>
           \fl@trace{(dbl)deferlist: \@deferlist: (before)}%
%</trace>
%    \end{macrocode}
%
% not in fixfloats?
%    \begin{macrocode}
          \@testwrongwidth\@currbox
%    \end{macrocode}
%
%    \begin{macrocode}
           \if@test
%<*trace>
              \fl@trace{type already on list: (dbl)defer}%
%</trace>
           \else
              \@tempdima -\ht\@currbox
              \advance\@tempdima
                -\ifx \@dbltoplist\@empty \dbltextfloatsep \else
                                          \dblfloatsep \fi
              \global \advance \@dbltoproom \@tempdima
              \global \advance \@colht \@tempdima
              \global \advance \@dbltopnum \m@ne
              \@cons \@dbltoplist \@currbox
%<*trace>
              \fl@trace{dbltopnum (after) = \the \@dbltopnum}%
              \fl@trace{***Success: dbltop}%
%</trace>
              \@inserttrue
           \fi
       \fi
%<*trace>
     \else
       \fl@trace{Fail: dbltopnum = \the \@dbltopnum: fpstype
                                                  \the \@fpstype=ORD?}%
       \ifnum \@fpstype<\sixt@@n
         \fl@trace{ERROR: !t float not successful (addtodblcol)}%
       \fi
%</trace>
     \fi
   \fi
   \if@insert
   \else
%<*trace>
     \fl@trace{put on deferlist}%
%</trace>
     \@cons\@deferlist\@currbox
%<*trace>
     \fl@trace{(dbl)deferlist: \@deferlist: (after)}%
%</trace>
   \fi
%<*trace>
   \fl@trace{End of addtodblcol -- locally count:}%
   \fl@trace{ dbltop: \the \@dbltopnum.}%
%</trace>
  \endgroup
%<*trace>
  \fl@trace{End of addtodblcol -- globally count:}%
  \fl@trace{dbltop: \the \@dbltopnum.}%
%</trace>
}%
%</2ekernel|latexrelease|fltrace>
%<latexrelease|fltrace>\EndIncludeInRelease
%<latexrelease|fltrace>\IncludeInRelease{0000/00/00}%
%<latexrelease|fltrace>  {\@addtodblcol}{float order in 2-column}%
%<latexrelease|fltrace>\def\@addtodblcol{%
%<latexrelease|fltrace>  \begingroup
%<*trace>
%<latexrelease|fltrace>  \fl@trace{***Start addtodblcol}%
%</trace>
%<latexrelease|fltrace>   \@insertfalse
%<latexrelease|fltrace>   \@setfloattypecounts
%<latexrelease|fltrace>   \@getfpsbit \tw@
%<*trace>
%<latexrelease|fltrace>   \fl@trace{fpstype \ifodd \@tempcnta OK
%<latexrelease|fltrace>                \else not \fi dbltop: \the \@fpstype}%
%</trace>
%<latexrelease|fltrace>   \ifodd\@tempcnta
%<latexrelease|fltrace>     \@flsetnum \@dbltopnum
%<latexrelease|fltrace>     \ifnum \@dbltopnum>\z@
%<latexrelease|fltrace>       \@tempswafalse
%<latexrelease|fltrace>       \ifdim \@dbltoproom>\ht\@currbox
%<latexrelease|fltrace>         \@tempswatrue
%<*trace>
%<latexrelease|fltrace>         \fl@trace{Space OK: \@dbltoproom =
%<latexrelease|fltrace>                \the \@dbltoproom > \the \ht \@currbox
%<latexrelease|fltrace>                                         (dbltoproom)}%
%</trace>
%<latexrelease|fltrace>       \else
%<*trace>
%<latexrelease|fltrace>      \fl@trace{fpstype: \the \@fpstype (addtodblcol)}%
%</trace>
%<latexrelease|fltrace>         \ifnum \@fpstype<\sixt@@n
%<*trace>
%<latexrelease|fltrace>           \fl@trace{BANG float ignoring \@dbltoproom}%
%<latexrelease|fltrace>           \fl@trace{\@spaces \@dbltoproom =
%<latexrelease|fltrace>                    \the \@dbltoproom.
%<latexrelease|fltrace>                    Ht float: \the \ht \@currbox-BANG}%
%</trace>
%    \end{macrocode}
% Need to check that there is room on the page, using the local value
% of |\@textmin| to make the necessary adjustment to |\@dbltoproom|.
%    \begin{macrocode}
%<latexrelease|fltrace>           \advance \@dbltoproom \@textmin
%<*trace>
%<latexrelease|fltrace>       \fl@trace{Local value of texmin: \the\@textmin}%
%<latexrelease|fltrace>       \fl@trace{\@spaces space on page =
%<latexrelease|fltrace>                   \the \@dbltoproom.
%<latexrelease|fltrace>                    Ht float: \the \ht \@currbox-BANG}%
%</trace>
%<latexrelease|fltrace>           \ifdim \@dbltoproom>\ht\@currbox
%<latexrelease|fltrace>             \@tempswatrue
%<*trace>
%<latexrelease|fltrace>         \fl@trace{Space OK BANG: space on page =
%<latexrelease|fltrace>                 \the\@dbltoproom > \the\ht\@currbox}%
%<latexrelease|fltrace>           \else
%<latexrelease|fltrace>         \fl@trace{fpstype: \the \@fpstype}%
%<latexrelease|fltrace>         \fl@trace{Fail---no room dbltoproom-BANG?:}%
%<latexrelease|fltrace>         \fl@trace{\@spaces space on page =
%<latexrelease|fltrace>                      \the \@dbltoproom.
%<latexrelease|fltrace>                       Ht float: \the \ht \@currbox}%
%</trace>
%<latexrelease|fltrace>           \fi
%<latexrelease|fltrace>           \advance \@dbltoproom -\@textmin
%<*trace>
%<latexrelease|fltrace>         \else
%<latexrelease|fltrace>           \fl@trace{fpstype: \the \@fpstype}%
%<latexrelease|fltrace>           \fl@trace{Fail---no room dbltoproom-ORD?:}%
%<latexrelease|fltrace>           \fl@trace{\@spaces \@dbltoproom =
%<latexrelease|fltrace>              \the \@dbltoproom.
%<latexrelease|fltrace>              Ht float: \the \ht \@currbox}%
%</trace>
%<latexrelease|fltrace>         \fi
%<latexrelease|fltrace>       \fi
%<latexrelease|fltrace>       \if@tempswa
%<latexrelease|fltrace>           \@bitor \@currtype \@dbldeferlist
%<*trace>
%<latexrelease|fltrace>           \fl@trace{dbldeferlist:
%<latexrelease|fltrace>                      \@dbldeferlist: (before)}%
%</trace>
%<latexrelease|fltrace>           \if@test
%<*trace>
%<latexrelease|fltrace>             \fl@trace{type already on list: dbldefer}%
%</trace>
%<latexrelease|fltrace>           \else
%<latexrelease|fltrace>              \@tempdima -\ht\@currbox
%<latexrelease|fltrace>              \advance\@tempdima
%<latexrelease|fltrace>                -\ifx \@dbltoplist\@empty
%<latexrelease|fltrace>                      \dbltextfloatsep
%<latexrelease|fltrace>                 \else \dblfloatsep \fi
%<latexrelease|fltrace>              \global \advance \@dbltoproom \@tempdima
%<latexrelease|fltrace>              \global \advance \@colht \@tempdima
%<latexrelease|fltrace>              \global \advance \@dbltopnum \m@ne
%<latexrelease|fltrace>              \@cons \@dbltoplist \@currbox
%<*trace>
%<latexrelease|fltrace>              \fl@trace{dbltopnum (after) =
%<latexrelease|fltrace>                                    \the \@dbltopnum}%
%<latexrelease|fltrace>              \fl@trace{***Success: dbltop}%
%</trace>
%<latexrelease|fltrace>              \@inserttrue
%<latexrelease|fltrace>           \fi
%<latexrelease|fltrace>       \fi
%<*trace>
%<latexrelease|fltrace>     \else
%<latexrelease|fltrace>       \fl@trace{Fail: dbltopnum = \the \@dbltopnum:
%<latexrelease|fltrace>                          fpstype \the \@fpstype=ORD?}%
%<latexrelease|fltrace>       \ifnum \@fpstype<\sixt@@n
%<latexrelease|fltrace>         \fl@trace{ERROR: !t float not successful
%<latexrelease|fltrace>                                        (addtodblcol)}%
%<latexrelease|fltrace>       \fi
%</trace>
%<latexrelease|fltrace>     \fi
%<latexrelease|fltrace>   \fi
%<latexrelease|fltrace>   \if@insert
%<latexrelease|fltrace>   \else
%<*trace>
%<latexrelease|fltrace>     \fl@trace{put on dbldeferlist}%
%</trace>
%<latexrelease|fltrace>     \@cons\@dbldeferlist\@currbox
%<*trace>
%<latexrelease|fltrace>     \fl@trace{dbldeferlist: \@dbldeferlist: (after)}%
%</trace>
%<latexrelease|fltrace>   \fi
%<*trace>
%<latexrelease|fltrace>   \fl@trace{End of addtodblcol -- locally count:}%
%<latexrelease|fltrace>   \fl@trace{ dbltop: \the \@dbltopnum.}%
%</trace>
%<latexrelease|fltrace>  \endgroup
%<*trace>
%<latexrelease|fltrace>  \fl@trace{End of addtodblcol -- globally count:}%
%<latexrelease|fltrace>  \fl@trace{dbltop: \the \@dbltopnum.}%
%</trace>
%<latexrelease|fltrace>}%
%<latexrelease|fltrace>\EndIncludeInRelease
%    \end{macrocode}
%  \end{macro}
%
%
%
% \begin{macro}{\@addmarginpar}
%    \begin{macrocode}
%<*2ekernel>
\def\@addmarginpar{\@next\@marbox\@currlist{\@cons\@freelist\@marbox
    \@cons\@freelist\@currbox}\@latexbug\@tempcnta\@ne
    \if@twocolumn
        \if@firstcolumn \@tempcnta\m@ne \fi
    \else
      \if@mparswitch
         \ifodd\c@page \else\@tempcnta\m@ne \fi
      \fi
      \if@reversemargin \@tempcnta -\@tempcnta \fi
    \fi
    \ifnum\@tempcnta <\z@  \global\setbox\@marbox\box\@currbox \fi
    \@tempdima\@mparbottom
    \advance\@tempdima -\@pageht
    \advance\@tempdima\ht\@marbox
    \ifdim\@tempdima >\z@
      \@latex@warning@no@line {Marginpar on page \thepage\space moved}%
    \else
      \@tempdima\z@
    \fi
    \global\@mparbottom\@pageht
    \global\advance\@mparbottom\@tempdima
    \global\advance\@mparbottom\dp\@marbox
    \global\advance\@mparbottom\marginparpush
    \advance\@tempdima -\ht\@marbox
%    \end{macrocode}
% Putting box movement inside the `marbox':
%    \begin{macrocode}
    \global\setbox \@marbox
                   \vbox {\vskip \@tempdima
                          \box \@marbox}%
    \global \ht\@marbox \z@
    \global \dp\@marbox \z@
%    \end{macrocode}
% Sticking (rather than gluing:-) the `marbox' to the line above,
% changed vskip to kern:
%    \begin{macrocode}
    \kern -\@pagedp
    \nointerlineskip
    \hb@xt@\columnwidth
      {\ifnum \@tempcnta >\z@
          \hskip\columnwidth \hskip\marginparsep
       \else
          \hskip -\marginparsep \hskip -\marginparwidth
       \fi
       \box\@marbox \hss}%
%    \end{macrocode}
%    For this reason the following code can vanish:
%\begin{verbatim}
%    \nobreak             %% No longer needed.  CAR92/12
%    \vskip -\@tempdima   %% No longer needed.  CAR92/12
%\end{verbatim}
%    \begin{macrocode}
    \nointerlineskip
    \hbox{\vrule \@height\z@ \@width\z@ \@depth\@pagedp}}
%    \end{macrocode}
% \end{macro}
%
% \subsubsection{Kludgeins}
%
% This part of the file is part of the implementation of the following
% two new commands for \LaTeX2e{}.
%
%
% \begin{verbatim}
% \enlargethispage{<dim>}
% \end{verbatim}
%
% Adds |<dim>| to the height of the current column only. On the printed
% page the bottom of this column is extended downwards by exactly
% |<dim>| without having any effect on the placement of the footer; this
% may result in an overprinting.
%
% \begin{verbatim}
% \enlargethispage*{<dim>}
% \end{verbatim}
%
% Similar to |\enlargethispage| but it tries to squeeze the column to
% be printed in as small a space as possible, ie it uses any
% shrinkability in the column. If the column was not explicitly broken
% (\eg with |\pagebreak|) this may result in an overfull box message but
% execpt for this it will come out as expected (if you know what to
% expect).
%
% The star form of this command is dedicated to Leslie Lamport, the
% other we need for ourselves (FMi, CAR).
%
% These commands may well have unwanted effects if used soon
% before a |\clearpage|: please give keep them clear of such places.
%
%  \begin{macro}{\@kludgeins}
% \changes{v0.1c}{1993/11/23}{Insert added}
%    The insert which makes \TeX{} do a lot of the necessary work.
%    All we need to put into it is the amount by which the pagegoal
%    should be changed.
%    \begin{macrocode}
\newinsert \@kludgeins
\global\dimen\@kludgeins \maxdimen
\global\count\@kludgeins 1000
%    \end{macrocode}
%  \end{macro}
%
%
%  \begin{macro}{\enlargethispage}
%  \begin{macro}{\enlargethispage*}
% \changes{v0.1c}{1993/11/23}{Commands added}
%    The user command.
%    \begin{macrocode}
\gdef \enlargethispage {%
   \@ifstar
     {%
%<*trace>
      \fl@trace{Enlarging page height * }%
%</trace>
      \@enlargepage{\hbox{\kern\p@}}}%
     {%
%<*trace>
      \fl@trace{Enlarging page height exactly---}%
%</trace>
      \@enlargepage\@empty}%
}
%    \end{macrocode}
%  \end{macro}
%  \end{macro}
%
%
%  \begin{macro}{\@enlargepage}
% \changes{v0.1c}{1993/11/23}{Command added}
%    This actually inserts the insert, after checking for extreme
%    values of the change.
%    \begin{macrocode}
\gdef\@enlargepage#1#2{%
%<*trace>
   \fl@trace{\@spaces\@spaces by #2}%
%</trace>
   \@tempskipa#2\relax
   \ifdim \@tempskipa>.5\maxdimen
     \@latexerr{Suggested\space extra\space height\space
                (\the\@tempskipa)\space dangerously\space
                large}\@eha
   \else
     \ifdim \vsize<.5\maxdimen
%<*trace>
       \fl@trace {Kludgeins added--pagegoal before: \the\pagegoal}%
%</trace>
       \@bsphack
         \insert\@kludgeins{#1\vskip-\@tempskipa}%
       \@esphack
%    \end{macrocode}
%    This next bit is for tracing only:
%    \begin{macrocode}
%<*trace>
       \ifvmode \par
         \fl@trace {Kludgeins added--pagegoal after: \the \pagegoal}%
       \fi
%</trace>
     \else
       \@latexerr{Page\space height\space already\space
                  too\space large}\@eha
     \fi
   \fi
}
%</2ekernel>
%    \end{macrocode}
%  \end{macro}
%
% \subsubsection{Float control}
%
% This part implements controllable floats and other changes
% to the float mechanism.
%
% It provides, at the document level, the following command for
% inclusion in \LaTeX2e{}.
%
% \begin{verbatim}
%     \suppressfloats
% \end{verbatim}
%
% This suppresses all further floats on the current page.
%
% With an optional argument it suppresses only floats only in certain
% positions on the current page.
% \begin{quote}
%  |[t]|\quad suppresses only floats at the top of the page
%  |[b]|\quad suppresses only floats at the bottom of the page
% \end{quote}
%
% It also enables the use of an extra specifier, {\tt !}, in the
% location optional argument of a float.  If this is present then,
% just for this particular float, whenever it is processed by the float
% mechanism the followinhg are ignored:
%
% \begin{itemize}
% \item  all restrictions on the number of floats which can appear;
% \item  all explicit restrictions on the amount of space which should
%   (not) be occupied by floats and/or text.
% \end{itemize}
%
% The mechanism will still attempt to ensure that pages are not
% overfull.
%
% These specifiers override, for the single float, the suppression
% commands described above.
%
%
% In its current form, it also supplies a reasonably exhaustive, and
% somewhat baroque, means of tracing some aspects of the float
% mechanism.
%
% More tracing.
%  \begin{macro}{\fl@trace}
%  \begin{macro}{\tracefloatsoff}
%  \begin{macro}{\tracefloats}
%  \begin{macro}{\fl@traceval}
%  \begin{macro}{\tracefloatvals}
%  \begin{macro}{\fl@tracemessage}
% \changes{v1.0c}{1993/11/30}{Commands added}
% \changes{v1.0h}{1993/12/12}{Commands changed}
% \changes{v1.0j}{1993/12/17}{tracefloatvals made a document command}
%    Set-up tracing for floats independent of other tracing as it
%    produces mega-output.  Default is no tracing.
% \changes{v1.1j}{1995/04/24}
%   {Do not add to kernel unless `trace' specified}
% \changes{v1.2n}{2014/04/24}
%   {Renamed internal trace commands; provide as package}
%
%    \begin{macrocode}
%<*fltrace>
\def \fl@tracemessage #1{{\let\@elt\@empty\typeout{LaTeX2e: #1}}}
\def \tracefloats{\let \fl@trace \fl@tracemessage}
\def \tracefloatsoff {\let \fl@trace \@gobble}
\tracefloatsoff
\def \fl@traceval #1{\fl@trace{\string #1 = \the #1}}
\IncludeInRelease{2015/01/01}{\tracefloatvals}%
                          {trace float vals}%
\def \tracefloatvals{%
%    \end{macrocode}
%    As |\@dblfloatplacement| sets |\f@depth| it needs to be run
%   inside a group, otherwise the float placement will test for the
%   wrong value.\footnote{This is a somewhat questionable design.}
%    \begin{macrocode}
 \begingroup
  \@dblfloatplacement
  \@floatplacement
  \fl@trace{***Float placement parameters:}%
  \fl@traceval\@colnum
  \fl@traceval\@colroom
  \fl@traceval\@topnum
  \fl@traceval\@toproom
  \fl@traceval\@botnum
  \fl@traceval\@botroom
  \fl@traceval\@fpmin
  \fl@trace{\string\textfraction = \textfraction}%
  \fl@traceval\@dbltopnum
  \fl@traceval\@dbltoproom
  \fl@trace{\string\textfraction = \textfraction}%
  \fl@trace{toplist: \@toplist}%
  \fl@trace{botlist: \@botlist}%
  \fl@trace{midlist: \@midlist}%
  \fl@trace{deferlist: \@deferlist}%
  \fl@trace{dbltoplist: \@dbltoplist}%
%FMi  \fl@trace{dbldeferlist: \@dbldeferlist}%
 \endgroup
}
\EndIncludeInRelease
\IncludeInRelease{0000/00/00}{\tracefloatvals}%
                          {trace float vals}%
\def \tracefloatvals{%
 \begingroup
  \@dblfloatplacement
  \@floatplacement
  \fl@trace{***Float placement parameters:}%
  \fl@traceval\@colnum
  \fl@traceval\@colroom
  \fl@traceval\@topnum
  \fl@traceval\@toproom
  \fl@traceval\@botnum
  \fl@traceval\@botroom
  \fl@traceval\@fpmin
  \fl@trace{\string\textfraction = \textfraction}%
  \fl@traceval\@dbltopnum
  \fl@traceval\@dbltoproom
  \fl@trace{\string\textfraction = \textfraction}%
  \fl@trace{toplist: \@toplist}%
  \fl@trace{botlist: \@botlist}%
  \fl@trace{midlist: \@midlist}%
  \fl@trace{deferlist: \@deferlist}%
  \fl@trace{dbltoplist: \@dbltoplist}%
% next line only in old releases
  \fl@trace{dbldeferlist: \@dbldeferlist}%
 \endgroup
}
\EndIncludeInRelease
%    \end{macrocode}
%    We need to make sure that \texttt{fltrace} comes before
%    \texttt{flafter} to make the tracing work.
%    \begin{macrocode}
\@ifpackageloaded{flafter}
 {\PackageWarningNoLine
      {fltrace}{Load 'fltrace' before 'flafter'\MessageBreak
                Attempting to recover by reloading 'flafter'}%
%    \end{macrocode}
%    Hide the fact that \texttt{flafter} was already loaded and then
%    request it anew.
% \changes{v1.2m}{2015/01/20}{Reset \cs{IncludeInRelease} flags}
%    \begin{macrocode}
    \expandafter\let\csname ver@flafter.sty\endcsname\relax
    \def\reserved@a#1{%
      \expandafter\let\csname\string#1+flafter+IIR\endcsname\relax}%
    \reserved@a\@addtocurcol
    \reserved@a\@addtonextcol
    \RequirePackage{flafter}}{}
%</fltrace>
%    \end{macrocode}
%    As the code for \texttt{flafter} will contain tracing calls so
%    that it works in conjunction with \texttt{fltrace} we need to
%    provide a dummy definition for |\fl@trace| in that package.
%    \begin{macrocode}
%<*flafter>
\providecommand\fl@trace[1]{}
%</flafter>
%    \end{macrocode}
%  \end{macro}
%  \end{macro}
%  \end{macro}
%  \end{macro}
%  \end{macro}
%  \end{macro}
%
%  \begin{macro}{\suppressfloats}
%  \begin{macro}{\@flstop}
% \changes{v1.0f}{1993/12/05}{Commands added}
% Float suppression commands: these set the relevant counter
% globally to zero.  Thus they are overridden for a particular float
% by an ! specifier.
%
%    \begin{macrocode}
%<*2ekernel>
\def \suppressfloats {%
   \@ifnextchar [%
     \@flstop
    {\global \@colnum \z@}%
}
%    \end{macrocode}
% Maybe this should be a loop over |#1|?
%    \begin{macrocode}
\def \@flstop [#1]{%
   \if t#1%
     \global \@topnum \z@
   \fi
   \if b#1%
     \global \@botnum \z@
   \fi
}
%    \end{macrocode}
%  \end{macro}
%  \end{macro}
%
%
% Manipulation of float placement and type; both their strings and the
% corresponding count registers.
%
%  \begin{macro}{\@fpstype}
%  \begin{macro}{\@reqcolroom}
%  \begin{macro}{\@textfloatsheight}
% \changes{v1.0f}{1993/12/05}{Commands added}
% First a new count register to go with |\@currtype|.
%
% Then a new skip register, for information needed to remove the
% |\@maxsep| conservatism: it is possible that this could use a
% temporary register.
%
% Finally a dimension register to hold the total height of in-text
% floats on the current page.  This is needed to implement a
% major change in the functionality of |\@addtocurcol| which is,
% nevertheless, a bug fix.
% It is not local and therefore cannot be a temporary register.
%
%    \begin{macrocode}
\newcount \@fpstype
\newdimen \@reqcolroom
\newdimen \@textfloatsheight
%</2ekernel>
%    \end{macrocode}
%  \end{macro}
%  \end{macro}
%  \end{macro}
%
%  \begin{macro}{\@fpsadddefault}
% \changes{v1.0f}{1993/12/05}{Command added}
% Adds the default placement to what is already there.
%
% Should not need to change this, but could do it as follows:
% \begin{verbatim}
%\def \@fpsadddefault {%
%   \@temptokena \expandafter\expandafter\expandafter
%                {\csname fps@\@captype \endcsname}%
%   \edef \reserved@a {\the\@temptokena}%
%   \@onelevel@sanitize \reserved@a
%   \edef \@fps {\@fps\reserved@a}%
%}
% \end{verbatim}
%
%    \begin{macrocode}
%<*2ekernel|fltrace>
\def \@fpsadddefault {%
%<*trace>
   \fl@trace{fps changed from: \@fps}%
%</trace>
   \edef \@fps {\@fps\csname fps@\@captype \endcsname}%
   \@latex@warning {%
     No positions in optional float specifier.\MessageBreak
     Default added (so using `\@fps')}%
}
%    \end{macrocode}
%  \end{macro}
%
%  \begin{macro}{\@setfloattypecounts}
% \changes{v1.0f}{1993/12/05}{Command added}
% Sets counters |\@fpstype| and |\@currtype|.
%
% BANG $==$ bit4 of $|\count\@currbox| = 0$.
%
%    \begin{macrocode}
\def \@setfloattypecounts {%
  \@currtype \count\@currbox
  \@fpstype \count\@currbox
  \divide\@currtype\@xxxii \multiply\@currtype\@xxxii
  \advance \@fpstype -\@currtype
%<*trace>
  \fl@trace{(mod 32) fpstype: \the \@fpstype}%
  \fl@trace{(mult of 32) currtype: \the \@currtype}%
% Tracing only: but some should be changed into real errors/warnings?
  \ifnum \@fpstype<\sixt@@n
    \ifnum \@fpstype=\z@
      \fl@trace{ERROR: no PLACEMENT, fpstype = \the \@fpstype = 0?}%
    \fi
    \ifnum \@fpstype=\@ne
      \fl@trace{WARNING: only h, fpstype = \the \@fpstype = 1?}%
    \fi
    \fl@trace{BANG float}%
  \else
    \ifnum \@fpstype=\sixt@@n
      \fl@trace{ERROR: no PLACEMENT, fpstype = \the \@fpstype = 16?}%
    \fi
    \ifnum \@fpstype=17
      \fl@trace{WARNING: only h, fpstype = \the \@fpstype = 17?}%
    \fi
    \fl@trace{ORD float}%
  \fi
%</trace>
}
%</2ekernel|fltrace>
%    \end{macrocode}
%  \end{macro}
%
% Macros for getting, testing and setting bits of the fps.
%
%
%  \begin{macro}{\@getfpsbit}
% \changes{v1.0f}{1993/12/05}{Command added}
% Sets |\@tempcnta| to required bit of |\count\@currbox|.
%
%    \begin{macrocode}
%<*2ekernel>
\def \@getfpsbit {%
   \@boxfpsbit \@currbox
}
%    \end{macrocode}
%  \end{macro}
%
%
%  \begin{macro}{\@boxfpsbit}
% \changes{v1.0f}{1993/12/05}{Command added}
%    Used above.
%    \begin{macrocode}
\def \@boxfpsbit #1#2{%
   \@tempcnta \count#1%
   \divide \@tempcnta #2\relax
}
%    \end{macrocode}
%  \end{macro}
%
%  \begin{macro}{\@testfp}
% \changes{v1.0f}{1993/12/05}{Command added}
% New definition of the float page test.
%
% \changes{v1.1v}{1996/07/26}{remove \cs{global} before \cs{@test...}}
%    \begin{macrocode}
\def \@testfp #1{%
   \@boxfpsbit #18\relax % Really `#1 8' for human readers!
   \ifodd \@tempcnta
   \else
     \@testtrue
   \fi
}
%    \end{macrocode}
%  \end{macro}
%
%
%  \begin{macro}{\@setfpsbit}
% \changes{v1.0f}{1993/12/05}{Command added}
% Sets required bit of |\@tempcnta| (to 1).
%
%    \begin{macrocode}
\def \@setfpsbit #1{%
   \@tempcntb \@tempcnta
   \divide \@tempcntb #1\relax
   \ifodd \@tempcntb
   \else
     \advance \@tempcnta #1\relax
   \fi
}
%</2ekernel>
%    \end{macrocode}
%  \end{macro}
%
%
%  \begin{macro}{\@resethfps}
% \changes{v1.0f}{1993/12/05}{Command added}
% \changes{v1.0h}{1993/12/12}{Warnings added: minimal}
% \changes{v1.0m}{1994/04/24}{Warning changed}
% \changes{v1.0m}{1994/04/24}{Number 2 changed to \cs{tw@}}
% \changes{v1.0o}{1994/05/02}{Code shortened}
% Globally adds t as a possible location for an h or !h only placement:
% this must be done using the count.
%
% Although it will leave |\@fpstype| set to 17 even if it was
% originally 1, this does not matter since it is the last thing in
% |\@addtocurcol|.
%    \begin{macrocode}
%<*2ekernel|fltrace>
\def \@resethfps {%
   \let\reserved@a\@empty
   \ifnum \@fpstype=\@ne
      \def \reserved@a {!}%
      \@fpstype 17
   \fi
   \ifnum \@fpstype=17
     \global \advance \count\@currbox \tw@
     \@latex@warning@no@line {%
       `\reserved@a h' float specifier changed to `\reserved@a ht'}%
%<*trace>
     \fl@trace{%
        `t' added to `\reserved@a h'- new Count: \the \count\@currbox}%
%</trace>
   \fi
}
%    \end{macrocode}
%  \end{macro}
%
%
% Special stuff for BANG floats.
%
%  \begin{macro}{\@flsetnum}
% \changes{v1.0f}{1993/12/05}{Command added}
%
% Ignores any zero float counter value in case BANG.
%
% It uses a local assignment to the normally global counter: a bit
% naughty, perhaps?
%
% These assignments are safe so long as the counter involved is only
% consulted once (\ie only for the `bang float') with the changed value.
% This is the case within |\@addtocurcol| because it is used only
% once within a call of the output routine (which forms a group).
%
% For |\@addtonextcol| this is achieved by putting a group around its
% code; this is needed because it is called (by |\@startcolumn|) for
% each float which was on the deferlist.  Almost identical
% considerations pertain to |\@addtodblcol|.  There may be more
% efficient ways to handle this, but the group seems to be the simplest.
%
% \changes{v1.0n}{1994/04/30}{Rogue space removed}
%    \begin{macrocode}
\def \@flsetnum #1{%
%<*trace>
   \fl@trace{fpstype: \the \@fpstype (flsetnum \string#1)}%
%</trace>
   \ifnum \@fpstype<\sixt@@n
     \ifnum #1=\z@
%<*trace>
       \fl@trace{BANG float resetting \string#1 to 1}%
%</trace>
       #1\@ne
     \fi
   \fi
%<*trace>
   \fl@trace{#1 (before) = \the #1}%
%</trace>
}
%    \end{macrocode}
%  \end{macro}
%
%
%  \begin{macro}{\@flsettextmin}
% \changes{v1.0f}{1993/12/05}{Command added}
% This ignores |\textfraction| space restriction in case BANG.
%
%    \begin{macrocode}
\def \@flsettextmin {%
%<*trace>
   \fl@trace{fpstype: \the \@fpstype (flsettextmin)}%
%</trace>
   \ifnum \@fpstype<\sixt@@n
%<*trace>
     \fl@trace{BANG ignoring textmin}%
%</trace>
     \@textmin \z@
   \else
     \@textmin \textfraction\@colht
%<*trace>
     \fl@trace{ORD textmin = \the \@textmin}%
%</trace>
   \fi
}
%    \end{macrocode}
%  \end{macro}
%
%
%  \begin{macro}{\@flcheckspace}
% \changes{v1.0f}{1993/12/05}{Command added}
% This ignores space restriction in case BANG; this is still slightly
% conservative since it does not allow for the fact that, if there is
% no text in the column then |\textfloatsep| is not needed.
% Sets |@tempswa| true if there is room for |\@currbox|.
%
%    \begin{macrocode}
\def \@flcheckspace #1#2{%
   \advance \@reqcolroom
     \ifx #2\@empty \textfloatsep \else \floatsep \fi
%<*trace>
   \fl@trace{colroom = \the \@colroom 
                                   (flcheckspace \string#1 \string#2)}%
   \fl@trace{reqcolroom = \the \@reqcolroom
                                   (flcheckspace \string#1 \string#2)}%
%</trace>
   \ifdim \@colroom>\@reqcolroom
     \ifdim #1>\ht\@currbox
       \@tempswatrue
%<*trace>
       \fl@trace{Space OK: #1 = \the #1 > \the \ht \@currbox
                                   (flcheckspace \string#1 \string#2)}%
%</trace>
     \else
%<*trace>
       \fl@trace{fpstype: \the \@fpstype
                                   (flcheckspace \string#1 \string#2)}%
%</trace>
       \ifnum \@fpstype<\sixt@@n
%<*trace>
         \fl@trace{BANG float ignoring #1
                                   (flcheckspace \string#1 \string#2):}%
         \fl@trace{\@spaces #1 = \the #1.  Ht float: \the \ht \@currbox
                                                          BANG}%
%</trace>
         \@tempswatrue
%<*trace>
       \else
         \fl@trace{Fail---no room (flcheckspace \string#1 \string#2)
                       (fpstype \the \@fpstype=ORD?):}%
         \fl@trace{\@spaces #1 = \the #1.  Ht float: \the \ht \@currbox
                                                          ORD?}%
%</trace>
       \fi
     \fi
%<*trace>
   \else
     \fl@trace{Fail---no room at 2nd test of colroom
                   (flcheckspace \string#1 \string#2)}%
%</trace>
   \fi
}
%</2ekernel|fltrace>
%    \end{macrocode}
%  \end{macro}
%
%
%  \begin{macro}{\@flupdates}
% \changes{v1.0f}{1993/12/05}{Command added}
%    This updates everything when a float is placed.
%
%    \begin{macrocode}
%<*2ekernel>
\def \@flupdates #1#2#3{%
   \global \advance #1\m@ne
   \global \advance \@colnum \m@ne
   \@tempdima -\ht\@currbox
   \advance \@tempdima
     -\ifx #3\@empty \textfloatsep \else \floatsep \fi
   \global \advance #2\@tempdima
   \global \advance \@colroom \@tempdima
   \@cons #3\@currbox
}
%</2ekernel>
%    \end{macrocode}
%  \end{macro}
%
%
% Interesting facts about float mechanisms past and present, together
% with a summary of various features, some unresolved:
%
% \begin{enumerate}
%   \item  The value |\textfraction| does not affect the processing
%     of doublecol floats: this seems sensible, but should be
%     documented.
%   \item |\twocolumn| floatplacement was wrong: dbl not needed, ord
%     needed.
%   \item |\@floatplacement| was not called after |\@startdblcol|
%     or |\@topnewpage|.  This has been changed; it is clearly a bug
%     fix.
%   \item The use |\@topnewpage| when |\dblfigrule| is non-trivial
%   produced a rule in the wrong place.  This has been fixed by not
%   using |\dblfigrule| when processing the `float' from
%   |\@topnewpage|.
%   \item  If the specifier was just h and the float could not be put
%     here, it went on the deferlist and stayed there until a clearpage.
%     It now gets changed to a `th': this is only an error-recovery
%     action, putting just h or !h should be deprecated.
%   \item |\@dblmaxsep| was `the maximum of |\dblfloatsep| and
%     |\dbltexfloatsep|'. But it was never used!  Now gone completely,
%     like |\@maxsep|.
%   \item After an h float is put on a page, it was counted as text when
%     applying the |\textfraction| test; this is possibly too big a
%     change although it is a bug fix?
%   \item  Two consecutive h floats are separated by twice |\intextsep|:
%     this could be changed to one by use of |\addvspace|, OK?
%     Note that it would also mean that less space is put in if an h
%     float  immediately follows other spaces.  This is also possibly
%     too big a change, at least for compatibility mode?
%     Or it may be simply wrong!  It has not been changed.
%   \item Now |\@addtocurcol| checks first for just p fps.  I think
%     that this is an increase in efficiency, but maybe the coding
%     should be made even more efficient.
%   \item |\@tryfcolumn| now tests if the list is empty first, otherwise
%     lots of wasted time!  Thus this test has been removed from
%     |\@startcolumn|.
%     As Frank pointed out, this makes |\@startcolumn| less
%     efficient. But it is now the same as |\@startdblcolumn|: I can
%     see no reason why they should be different, but which is best?
%   \item Why is |\@colroom| set in |\@doclearpage|?
%   \item  Footnotes. Check what |\clearpage| does when footnotes are
%     left over.  Footnotes are not put on float pages and, also,
%     |\@addtonextcol| ignores the existence of held-over footnotes
%     in deciding what floats can go on the page.  Not changed.
%   \item  |\clearpage| can still lose non-boxes, at least when floats
%     are involved.  It also moves some to the `wrong page', but this
%     may be a coding problem.
%   \item  The ! option makes it necessary to check in |\output| that
%     there is enough room left on the page after adding a float.  (This
%     would have been necessary anyway if anyone set |\@textmin| too
%     close to zero!  A similar danger existed also if the text in a
%     |\twocolumn[text]| entity gets too large.)
%     The current implementation of this also makes the normal case a
%     little less efficient, OK?
%     Not enough room means, at present, less than  |\baselineskip|,
%     with a warning: is this OK?  Should it be made generic (another
%     parameter)?
%   \item  There are four possibilities for supporting this:
%
%     |\twocolumn[\maketitle more text]|
%
%     One is to change
%     |\maketitle| slightly to allow this.  Another is to change
%     |\@topnewpage| so that more than one |\twocolumn[]| command is
%     allowed; in this case |\maketitle\twocolumn[more text]| will work.
%     The former is more robust from the user's viewpoint, but makes the
%     code for |\maketitle| rather ad hoc (maybe it is already?).
%     Another is to misuse the global twocolumn flag locally within
%     |\@topnewpage|.
%     Yet another is to move the column count register from the multicol
%     package into the kernel.  This has been done.
%   \item  Where should the reinserts be put to maximise the
%     probability that footmotes come out on the correct page?
%     Or should we go for as much compatibility as possible (but see
%     next item)?
%   \item  Should we continue to support (as much as possible)
%     |\samepage|?  Some of its intended functionality is now advertised
%     as being provided by |\enlargethispage|.  Use of either is likely
%     to result in wrongly placed footnotes, marginals, etc.
%     Which should have priority: obeying the pagination instructions,
%     or correct placement of notes/marginalia?
%   \item  Is the adjustment of space to cause shrinking in the
%      kludge-* case correct?  Should it be limited to 0pt?
%   \item  Is the setting of |\boxmaxdepth| in makecol and friends
%     needed?  It only has any effect if |\@textbottom| ends with a box
%     or rule, in which case the vskip to allow for its depth should
%     also be added.  If it is kept, it should probably be the last
%     thing in the box.  It has now been removed.
%
%     It would perhaps be better to document that |\@textbottom|
%     and |\@texttop| must have natural height 0pt.
%   \item  I cannot see why the vskip adjustment for the depth
%     is needed if boxmaxdepth is used to ensure that there is never
%     a too deep box.
%   \item  The value of |\boxmaxdepth| should be explicitly set
%     whenever necessary: it is too risky to assume that it has any
%     particular value.  Care is needed in deciding what to set it to.
%
%     It is interesting to note that the value of |\boxmaxdepth| is
%     unique in being read before the local settings for the box group
%     are reset; all other parameter settings which affect the box
%     construction use their values outside the box group.
%   \item  Should |\@maxdepth| store the setting of |\maxdepth| from
%     lplain?  Or should we provide a proper interface to class files
%     for setting these?
% \end{enumerate}
%
% An analysis of various other macros.
%
%    |\@opcol| should do |\@floatplacement|, but where?  Right at the
%    end, since it always occurs at the start of a column.
% \begin{verbatim}
% \def\@opcol{%
%   % Why is this done first?
%   \global \@mparbottom \z@
%   \if@twocolumn
%     \@outputdblcol
%   \else
%     \@outputpage
%     % This is not needed since it is done at the end of
%     %   |\@outputpage|:
%     \global \@colht \textheight
%   \fi}
% \end{verbatim}
%
% Only tracing has been added to these.
%
%    \begin{macrocode}
%<*2ekernel|fltrace>
\def\@makefcolumn #1{%
  \begingroup
    \@fpmin \z@
    \let \@testfp \@gobble
    \@tryfcolumn #1%
  \endgroup
%<*trace>
  \if@fcolmade
    \fl@trace{PAGE: in \string\clearpage
                              \if@twocolumn ---twocolumn\fi---}%
    \fl@trace{----- float column/page completed from \string#1}%
  \fi
%</trace>
}
%    \end{macrocode}
% This will line up the last baselines in the two
% columns provided they are constructed in the normal way: \ie ending
% in a skip of minus the original depth, with |\@textbottom| adding
% nothing.
%
% Thus again it is essential for |\@textbottom| to have depth 0pt.
% \changes{1.2g}{2000/07/12}{Ensure that rule is in \cs{normalcolor}}
% \changes{v1.2m}{2014/12/30}{Command updated (latexrelease)}
%    \begin{macrocode}
%</2ekernel|fltrace>
%<latexrelease|fltrace>\IncludeInRelease{2015/01/01}%
%<latexrelease|fltrace>  {\@outputdblcol}{2 column marks}%
%<*2ekernel|fltrace|latexrelease>
%    \end{macrocode}
%
%
% This is just a change to the single command |\@outputdblcol|
% so that it saves mark information for the first column and restores
% it in the second column.
%    \begin{macrocode}
\def\@outputdblcol{%
  \if@firstcolumn
    \global\@firstcolumnfalse
%    \end{macrocode}
% Save the left column
%    \begin{macrocode}
    \global\setbox\@leftcolumn\copy\@outputbox
%<fltrace>    \fl@trace{PAGE: first column boxed}%
%    \end{macrocode}
%
% Remember the marks from the first column
%    \begin{macrocode}
    \splitmaxdepth\maxdimen
    \vbadness\maxdimen
%    \end{macrocode}
%  In case of |\enlargethispage| we will have infinite negative glue
%  at the bottom of the page (coming from |\vss|) and that will earn
%  us an error message if we |\vsplit| to get at the marks.  So we
%  need to remove thek last glue (if any) at the end of |\@outputbox|
%  as we are only interested in marks that change doesn't matter.
% \changes{v1.1o}{2014/04/18}{Handle infinite glue from
%  \cs{enlargethispage} (pr/4023)}
%    \begin{macrocode}
     \setbox\@outputbox\vbox{\unvbox\@outputbox\unskip}%
     \setbox\@outputbox\vsplit\@outputbox to\maxdimen
%    \end{macrocode}
%
% One minor difference from the current |fixmarks| package, pass the
% marks through a token register to stop any |#| tokens causing an
% error in a |\def|.
%    \begin{macrocode}
    \toks@\expandafter{\topmark}%
    \xdef\@firstcoltopmark{\the\toks@}%
    \toks@\expandafter{\splitfirstmark}%
    \xdef\@firstcolfirstmark{\the\toks@}%
%    \end{macrocode}
%
% This test does not work if truly empty marks have been inserted, but
% \LaTeX\ marks should always have (at least) two brace groups.
% (Except before the first mark is used, when the marks are empty,
% but that is OK here.)
%    \begin{macrocode}
    \ifx\@firstcolfirstmark\@empty
      \global\let\@setmarks\relax
    \else
      \gdef\@setmarks{%
        \let\firstmark\@firstcolfirstmark
        \let\topmark\@firstcoltopmark}%
    \fi
%    \end{macrocode}
%
% End of change
%    \begin{macrocode}
  \else
    \global\@firstcolumntrue
    \setbox\@outputbox\vbox{%
     \hb@xt@\textwidth{%
        \hb@xt@\columnwidth{\box\@leftcolumn \hss}%
        \hfil
%    \end{macrocode}
% \changes{v1.1m}{2006/09/13}{Ensure that rule is in \cs{normalcolor}}
% The color of the \cs{vrule} should be \cs{normalcolor} as to not
% inherit the color from the column.
%    \begin{macrocode}
        {\normalcolor\vrule \@width\columnseprule}%
        \hfil
       \hb@xt@\columnwidth{\box\@outputbox \hss}}}%
%<fltrace>    \fl@trace{PAGE: second column also boxed}%
  \@combinedblfloats
%    \end{macrocode}
% Override current first and top with those of first column if necessary
%    \begin{macrocode}
    \@setmarks
%    \end{macrocode}
% End of change
%    \begin{macrocode}
    \@outputpage
%<fltrace>    \fl@trace{PAGE: two column page completed}%
    \begingroup
      \@dblfloatplacement
      \@startdblcolumn
      \@whilesw\if@fcolmade \fi{\@outputpage
%<fltrace>      \fl@trace{PAGE: double float page completed}%
     \@startdblcolumn}%
    \endgroup
  \fi}%
%    \end{macrocode}
%
%    \begin{macrocode}
%<latexrelease|fltrace>\EndIncludeInRelease
%<latexrelease|fltrace>\IncludeInRelease{0000/00/00}%
%<latexrelease|fltrace>  {\@outputdblcol}{2 column marks}%
%<latexrelease|fltrace>\def\@outputdblcol{%
%<latexrelease|fltrace>  \if@firstcolumn
%<latexrelease|fltrace>    \global \@firstcolumnfalse
%<latexrelease|fltrace>    \global \setbox\@leftcolumn \box\@outputbox
%<*trace>
%<latexrelease|fltrace>    \fl@trace{PAGE: first column boxed}%
%</trace>
%<latexrelease|fltrace>  \else
%<latexrelease|fltrace>    \global \@firstcolumntrue
%<latexrelease|fltrace>    \setbox\@outputbox \vbox {%
%<latexrelease|fltrace>                         \hb@xt@\textwidth {%
%<latexrelease|fltrace>                           \hb@xt@\columnwidth {%
%<latexrelease|fltrace>                             \box\@leftcolumn \hss}%
%<latexrelease|fltrace>                           \hfil
%<latexrelease|fltrace>                           {\normalcolor\vrule
%<latexrelease|fltrace>                               \@width\columnseprule}%
%<latexrelease|fltrace>                           \hfil
%<latexrelease|fltrace>                           \hb@xt@\columnwidth {%
%<latexrelease|fltrace>                             \box\@outputbox \hss}%
%<latexrelease|fltrace>                                             }%
%<latexrelease|fltrace>                              }%
%<*trace>
%<latexrelease|fltrace>    \fl@trace{PAGE: second column also boxed}%
%</trace>
%<latexrelease|fltrace>    \@combinedblfloats
%<latexrelease|fltrace>    \@outputpage
%<*trace>
%<latexrelease|fltrace>    \fl@trace{PAGE: two column page completed}%
%</trace>
%<latexrelease|fltrace>    \begingroup
%<latexrelease|fltrace>      \@dblfloatplacement
%<latexrelease|fltrace>      \@startdblcolumn
%    \end{macrocode}
%    This loop could be replaced by an |\expandafter| tail
%    recursion in\\ |\@startdblcolumn|.
%    \begin{macrocode}
%<latexrelease|fltrace>      \@whilesw\if@fcolmade \fi
%<latexrelease|fltrace>        {\@outputpage
%<*trace>
%<latexrelease|fltrace>      \fl@trace{PAGE: double float page completed}%
%</trace>
%<latexrelease|fltrace>         \@startdblcolumn}%
%<latexrelease|fltrace>    \endgroup
%<latexrelease|fltrace>  \fi
%<latexrelease|fltrace>}%
%<latexrelease|fltrace>\EndIncludeInRelease
%</2ekernel|fltrace|latexrelease>
%    \end{macrocode}
%
% \subsubsection{Float placement parameters}
% \changes{v1.0i}{1993/12/14}{Section added to declare all parameters}
%
% The main purpose of this section is to ensure that all the
% float-placement parameters which need to be set in a class file or
% package have been declared.  It also describes their use and sets
% values for them which are reasonable for typical documents using
% US letter or A4 sized paper.
%
% \paragraph{Limits for the placement of floating objects}
%
% \begin{macro}{\c@topnumber}
%    This counter holds the maximum number of
%    floats that can appear at the top of a text page or column.
%    \begin{macrocode}
%<*2ekernel>
\newcount\c@topnumber
\setcounter{topnumber}{2}
%    \end{macrocode}
% \end{macro}
%
% \begin{macro}{\topfraction}
%    This macro holds the maximum proportion (as a decimal number) of
%    a text page or column that can be occupied by floats at the top.
%    \begin{macrocode}
\newcommand\topfraction{.7}
%    \end{macrocode}
% \end{macro}
%
% \begin{macro}{\c@bottomnumber}
%    This counter holds the maximum number of
%    floats that can appear at the bottom of a text page or column.
%    \begin{macrocode}
\newcount\c@bottomnumber
\setcounter{bottomnumber}{1}
%    \end{macrocode}
% \end{macro}
%
% \begin{macro}{\bottomfraction}
%    This macro holds the maximum proportion (as a decimal number) of
%    a text page or column that can be occupied by floats at the bottom.
%    \begin{macrocode}
\newcommand\bottomfraction{.3}
%    \end{macrocode}
% \end{macro}
%
% \begin{macro}{\c@totalnumber}
%    This counter holds the maximum number of floats that can appear on
%    any text page or column.
%    \begin{macrocode}
\newcount\c@totalnumber
\setcounter{totalnumber}{3}
%    \end{macrocode}
% \end{macro}
%
% \begin{macro}{\textfraction}
%    This macro holds the minimum proportion (as a decimal number) of
%    a text page or column that must be occupied by text.
%    \begin{macrocode}
\newcommand\textfraction{.2}
%    \end{macrocode}
% \end{macro}
%
% \begin{macro}{\floatpagefraction}
%    This macro holds the minimum proportion (as a decimal number) of
%    a page or column that must be occupied by floating objects before a
%    `float page' is produced.
%    \begin{macrocode}
\newcommand\floatpagefraction{.5}
%    \end{macrocode}
% \end{macro}
%
% \begin{macro}{\c@dbltopnumber}
%    This counter holds the maximum number of double-column floats that
%    can appear on the top of a two-column text page.
%    \begin{macrocode}
\newcount\c@dbltopnumber
\setcounter{dbltopnumber}{2}
%    \end{macrocode}
% \end{macro}
%
% \begin{macro}{\dbltopfraction}
%    This macro holds the maximum proportion (as a decimal number) of
%    a two-column text page that can be occupied by double-column floats
%    at the top.
%    \begin{macrocode}
\newcommand\dbltopfraction{.7}
%    \end{macrocode}
% \end{macro}
%
% \begin{macro}{\dblfloatpagefraction}
%    This macro holds the minimum proportion (as a decimal number) of
%    a page that must be occupied by double-column floating objects
%    before a `double-column float page' is produced.
%    \begin{macrocode}
\newcommand\dblfloatpagefraction{.5}
%    \end{macrocode}
% \end{macro}
%
% \paragraph{Floats on a text page}
%
% \begin{macro}{\floatsep}
% \begin{macro}{\textfloatsep}
% \begin{macro}{\intextsep}
%    When a floating object is placed on a page with text, these
%    parameters control the separation between the float and the other
%    objects on the page. These parameters are used for both
%    one-column mode and single-column floats in two-column mode.
%    They are all rubber lengths.
%
%    |\floatsep| is the space between adjacent floats that are placed
%    at the top or bottom of the text page or column.
%
%    |\textfloatsep| is the space between the main text and floats
%    at the top or bottom of the page or column.
%
%    |\intextsep| is the space between in-text floats and the text.
%    \begin{macrocode}
\newskip\floatsep
\newskip\textfloatsep
\newskip\intextsep
\setlength\floatsep    {12\p@ \@plus 2\p@ \@minus 2\p@}
\setlength\textfloatsep{20\p@ \@plus 2\p@ \@minus 4\p@}
\setlength\intextsep   {12\p@ \@plus 2\p@ \@minus 2\p@}
%    \end{macrocode}
% \end{macro}
% \end{macro}
% \end{macro}
%
% \begin{macro}{\dblfloatsep}
% \begin{macro}{\dbltextfloatsep}
%    When double-column floats (floating objects that span the whole
%    |\textwidth|) are placed at the top of a text page in two-column
%    mode, the separation between the float and the text is controlled
%    by |\dblfloatsep| and |\dbltextfloatsep|.  They are rubber lengths.
%
%    |\dblfloatsep| is the space between adjacent double-column floats
%    placed at the top of the text page.
%
%    |\dbltextfloatsep| is the space between the main text and
%    double-column floats at the top of the page.
%    \begin{macrocode}
\newskip\dblfloatsep
\newskip\dbltextfloatsep
\setlength\dblfloatsep    {12\p@ \@plus 2\p@ \@minus 2\p@}
\setlength\dbltextfloatsep{20\p@ \@plus 2\p@ \@minus 4\p@}
%    \end{macrocode}
% \end{macro}
% \end{macro}
%
% \paragraph{Floats on their own page or column}
%
% \begin{macro}{\@fptop}
% \begin{macro}{\@fpsep}
% \begin{macro}{\@fpbot}
%    When floating objects are placed on a separate page or column,
%    called a `float page', the layout of the page is controlled by
%    these parameters, which are rubber lengths.
%
%    At the top of the page |\@fptop| is inserted;
%    typically this supplies some stretchable whitespace.
%    At the bottom of the page |\@fpbot| ais inserted.
%    Between adjacent floats |\@fpsep| is inserted.
%
%    These parameters are used for all floating objects on a
%    `float page' in one-column mode, and for single-column
%    floats in two-column mode.
%
%    Note that at least one of the two parameters |\@fptop| and
%    |\@fpbot| should contain a |plus ...fil| so as to fill the
%    remaining empty space.
%    \begin{macrocode}
\newskip\@fptop
\newskip\@fpsep
\newskip\@fpbot
\setlength\@fptop{0\p@ \@plus 1fil}
\setlength\@fpsep{8\p@ \@plus 2fil}
\setlength\@fpbot{0\p@ \@plus 1fil}
%    \end{macrocode}
% \end{macro}
% \end{macro}
% \end{macro}
%
% \begin{macro}{\@dblfptop}
% \begin{macro}{\@dblfpsep}
% \begin{macro}{\@dblfpbot}
%    Double-column `float pages' in two-column mode use similar
%    parameters.
%    \begin{macrocode}
\newskip\@dblfptop
\newskip\@dblfpsep
\newskip\@dblfpbot
\setlength\@dblfptop{0\p@ \@plus 1fil}
\setlength\@dblfpsep{8\p@ \@plus 2fil}
\setlength\@dblfpbot{0\p@ \@plus 1fil}
%    \end{macrocode}
% \end{macro}
% \end{macro}
% \end{macro}
%
% \begin{macro}{\topfigrule}
% \begin{macro}{\botfigrule}
% \begin{macro}{\dblfigrule}
%    The macros can be used to put in rules between floats and text;
%    whatever they insert should be vertical mode material which takes
%    up zero space.
% \task{CAR}{Add more rules (for Frank in addtocurcol)}
%    \begin{macrocode}
\let\topfigrule=\relax
\let\botfigrule=\relax
\let\dblfigrule=\relax
%</2ekernel>
%    \end{macrocode}
% \end{macro}
% \end{macro}
% \end{macro}
%
%
% \Finale
\endinput
